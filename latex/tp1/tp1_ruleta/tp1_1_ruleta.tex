\documentclass{article}

\usepackage{arxiv}

\usepackage[utf8]{inputenc}             % allow utf-8 input
\usepackage[spanish]{babel}             % idioma español
\usepackage[T1]{fontenc}                % use 8-bit T1 fonts
\usepackage[pdfencoding=auto]{hyperref} % hyperlinks, fixed PDF encoding on LuaLaTeX
\usepackage{url}                        % simple URL typesetting
\usepackage{booktabs}                   % professional-quality tables
\usepackage{amsfonts}                   % blackboard math symbols
\usepackage{nicefrac}                   % compact symbols for 1/2, etc.
\usepackage{microtype}                  % microtypography
\usepackage{graphicx}
\graphicspath{ {./images/} }
\usepackage{kpfonts}                    % use the same fonts for text and maths

\usepackage{float}                      % force figure placement with the `H' float specifier
\usepackage{pgfplots}                   % TikZ graphics
\pgfplotsset{compat=1.15}

\usepackage{shellesc}                   % needed for TikZ externalization to work on LuaLaTeX
\usetikzlibrary{external}               % TikZ figure externalization
\tikzexternalize[prefix=tikzcache/]

\usepackage{mytikz}

\title{TP 1.1 - Simulación de una Ruleta}

\author{
 Marcelo G. Catellano \\
  UTN -- FRRo \\
  Legajo 39028 \\
  \texttt{marce.geek22@gmail.com} \\
}

\begin{document}
\maketitle
\begin{abstract}
Simulación de un modelo simple de una ruleta empleando el lenguaje de programación Python 3.x.
\end{abstract}

% keywords can be removed
\keywords{Simulación \and Trabajo práctico \and Ruleta \and Números aleatorios \and Estadística descriptiva \and Python 3.x}

\section[Introducción]{Introducción \cite{wiki-ruleta}}
La ruleta es un juego de azar típico de los casinos, cuyo nombre viene del término francés roulette, que significa ``ruedita'' o ``rueda pequeña''. Su uso como elemento de juego de azar, aún en configuraciones distintas de la actual, no está documentado hasta bien entrada la Edad Media. Es de suponer que su referencia más antigua es la llamada Rueda de la Fortuna, de la que hay noticias a lo largo de toda la historia, prácticamente en todos los campos del saber humano.

La ``magia'' del movimiento de las ruedas tuvo que impactar a todas las generaciones. La aparente quietud del centro, el aumento de velocidad conforme nos alejamos de él, la posibilidad de que se detenga en un punto al azar; todo esto tuvo que influir en el desarrollo de distintos juegos que tienen la rueda como base.

Las ruedas, y por extensión las ruletas, siempre han tenido conexión con el mundo mágico y esotérico. Así, una de ellas forma parte del tarot, más precisamente de los que se conocen como arcanos mayores.

Según los indicios, la creación de una ruleta y sus normas de juego, muy similares a las que conocemos hoy en día, se debe a Blaise Pascal, matemático francés, quien ideó una ruleta con treinta y seis números (sin el cero), en la que se halla un extremado equilibrio en la posición en que está colocado cada número. La elección de 36 números da un alcance aún más vinculado a la magia (la suma de los primeros 36 números da el número mágico por excelencia: seiscientos sesenta y seis).

Esta ruleta podía usarse como entretenimiento en círculos de amistades. Sin embargo, a nivel de empresa que pone los medios y el personal para el entretenimiento de sus clientes, no era rentable, ya que estadísticamente todo lo que se apostaba se repartía en premios (probabilidad de 1/36 de acertar el número y ganar 36 veces lo apostado).

En 1842, los hermanos Blanc modificaron la ruleta añadiéndole un nuevo número, el 0, y la introdujeron inicialmente en el Casino de Montecarlo. Ésta es la ruleta que se conoce hoy en día, con una probabilidad de acertar de 1/37 y ganar 36 veces lo apostado, consiguiendo un margen para la casa del $2.7\%$ (1/37).

Más adelante, en algunas ruletas (sobre todo las que se usan en países anglosajones) se añadió un nuevo número (el doble cero), con lo cual el beneficio para el casino resultó ser doble (2/38 o $5.26\%$).

\section{Descripción del trabajo a realizar}
En este trabajo se realizará la simulación del comportamiento de una ruleta tipo Montecarlo (números del 0 al 36, 37 números en total) mediante la generación de números aleatorios en el lenguaje de programación Python 3.x. Se asumirá una distribución de probabilidad uniforme discreta.

Para el trabajo se generarán 10 poblaciones aleatorias de 5000 tiradas de la ruleta cada una. Luego se calcularán las frecuencias relativas, promedios, desvíos y varianzas para todas las poblaciones con respecto al número de tiradas (1 a 5000). A continuación, se graficarán los resultados, primero para una de las poblaciones, la primera generada, y luego se realizarán gráficas agregando los resultados para todas las poblaciones simultáneamente.

Finalmente, se realizarán las conclusiones a partir de la información extraída de las gráficas.

\section{Estadísticas}
\subsection{Distribución uniforme discreta \cite{wiki-uniforme-discreta}}
La distribución uniforme discreta es una distribución de probabilidad simétrica en la que un número finito de valores tienen la misma probabilidad de ser observados; cada uno de los $n$ valores tiene la misma probabilidad $1/n$.

La distribución uniforme discreta en sí resulta inherentemente no paramétrica. Es conveniente, sin embargo, representar sus valores de manera general como todos los enteros dentro de un intervalo $[a,b]$, de manera que $a$ y $b$ se convierten en los parámetros principales de la distribución.

Función de probabilidad:
\begin{equation}
p_X(x_i) = P(X = x_i) = \frac{1}{n}
\end{equation}

Esperanza matemática:
\begin{equation}
\operatorname{E}[X] = \frac{a+b}{2}
\end{equation}

Varianza:
\begin{equation}
\operatorname{Var}(X) = \sigma^{2} = \frac{(b-a+1)^{2}-1}{12}
\end{equation}

Desvío estándar:
\begin{equation}
\sigma = \sqrt{\operatorname{Var}(X)} = \sqrt{\frac{(b-a+1)^{2}-1}{12}}
\end{equation}

\subsection{Fórmulas muestrales}
Frecuencia relativa \cite{wiki-frecuencia}
\begin{equation}
f_{i} = f_{r}(x_{i}) = \frac {n_{i}}{N}
\end{equation}

Esperanza matemática \cite{wiki-esperanza}:
\begin{equation}
\operatorname{E}[X] = \sum_{i=1}^{n}x_{i}\,f_{i}=x_{1}f_{1}+x_{2}f_{2}+\cdots +x_{n}f_{n} = \frac{\sum_{i=1}^{n}x_{i}\,n_{i}}{N}
\end{equation}

Varianza \cite{wiki-varianza}:
\begin{equation}
\operatorname{Var}(X) = \sigma^{2} = \operatorname{E}\left[(X - \mu)^{2}\right] = \operatorname{E}\left[(X - \operatorname{E}[X])^{2}\right]
\end{equation}

Desvío estándar:
\begin{equation}
\sigma = \sqrt{\operatorname{Var}(X)} = \sqrt{\operatorname{E}\left[(X - \operatorname{E}[X])^{2}\right]}
\end{equation}

\section{Gráficas}
\begin{figure}[H]
  \begin{mytikzresize}{0.8\textwidth}
    \centering
    \input{generated/frecuencia.tex}
    \caption{frecuencia relativa con respecto al número de tiradas}
  \end{mytikzresize}
\end{figure}

\begin{figure}[H]
  \begin{mytikzresize}{0.8\textwidth}
    \centering
    \input{generated/promedio.tex}
    \caption{valor promedio con respecto al número de tiradas}
  \end{mytikzresize}
\end{figure}

\begin{figure}[H]
  \begin{mytikzresize}{0.8\textwidth}
    \centering
    \input{generated/desvio.tex}
    \caption{valor del desvío con respecto al número de tiradas}
  \end{mytikzresize}
\end{figure}

\begin{figure}[H]
  \begin{mytikzresize}{0.8\textwidth}
    \centering
    \input{generated/varianza.tex}
    \caption{valor de la varianza con respecto al número de tiradas}
  \end{mytikzresize}
\end{figure}

\begin{figure}[H]
  \begin{mytikzresize}{0.8\textwidth}
    \centering
    \input{generated/frecuencia-multi.tex}
    \caption{frecuencia relativa para 10 corridas del experimento}
    \label{fig:frec-multi}
  \end{mytikzresize}
\end{figure}

\begin{figure}[H]
  \begin{mytikzresize}{0.8\textwidth}
    \centering
    \input{generated/promedio-multi.tex}
    \caption{valor promedio para 10 corridas del experimento}
    \label{fig:prom-multi}
  \end{mytikzresize}
\end{figure}

\begin{figure}[H]
  \begin{mytikzresize}{0.8\textwidth}
    \centering
    \input{generated/desvio-multi.tex}
    \caption{valor del desvío para 10 corridas del experimento}
    \label{fig:desvio-multi}
  \end{mytikzresize}
\end{figure}

\begin{figure}[H]
  \begin{mytikzresize}{0.8\textwidth}
    \centering
    \input{generated/varianza-multi.tex}
    \caption{valor de la varianza para 10 corridas del experimento}
    \label{fig:var-multi}
  \end{mytikzresize}
\end{figure}

\section{Conclusiones}

En las gráficas se puede ver claramente cómo los valores obtenidos se aproximan a los valores teóricos esperados para una distribución uniforme discreta conforme se va incrementando el número de tiradas de la ruleta. Además, en las figuras \ref{fig:frec-multi}, \ref{fig:prom-multi}, \ref{fig:desvio-multi} y \ref{fig:var-multi} puede verse cómo los valores obtenidos caen todos alrededor de los valores esperados.

\bibliographystyle{unsrt}
\bibliography{references}


\end{document}
