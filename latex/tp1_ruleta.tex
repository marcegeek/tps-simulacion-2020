\documentclass{article}

\usepackage{arxiv}

\usepackage[utf8]{inputenc} % allow utf-8 input
\usepackage[spanish]{babel} % idioma español
\usepackage[T1]{fontenc}    % use 8-bit T1 fonts
\usepackage{hyperref}       % hyperlinks
\usepackage{url}            % simple URL typesetting
\usepackage{booktabs}       % professional-quality tables
\usepackage{amsfonts}       % blackboard math symbols
\usepackage{nicefrac}       % compact symbols for 1/2, etc.
\usepackage{microtype}      % microtypography
\usepackage{graphicx}
\graphicspath{ {./images/} }
\usepackage{kpfonts}        % use the same fonts for text and maths

\usepackage{pgfplots}       % TikZ graphics
\pgfplotsset{compat=1.15}

\usepackage{mytikz}

\title{TP 1.1 - Simulación de una Ruleta}

\author{
 Marcelo G. Catellano \\
  UTN -- FRRo \\
  \texttt{marce.geek22@gmail.com} \\
}

\begin{document}
\maketitle
\begin{abstract}
Simulación de un modelo simple de una ruleta empleando el lenguaje de programación Python 3.x.
\end{abstract}

% keywords can be removed
%\keywords{First keyword \and Second keyword \and More}

\section[Introducción]{Introducción\footnote{Wikipedia - \url{https://es.wikipedia.org/wiki/Ruleta}}}
La ruleta es un juego de azar típico de los casinos, cuyo nombre viene del término francés roulette, que significa ``ruedita'' o ``rueda pequeña''. Su uso como elemento de juego de azar, aún en configuraciones distintas de la actual, no está documentado hasta bien entrada la Edad Media. Es de suponer que su referencia más antigua es la llamada Rueda de la Fortuna, de la que hay noticias a lo largo de toda la historia, prácticamente en todos los campos del saber humano.

La ``magia'' del movimiento de las ruedas tuvo que impactar a todas las generaciones. La aparente quietud del centro, el aumento de velocidad conforme nos alejamos de él, la posibilidad de que se detenga en un punto al azar; todo esto tuvo que influir en el desarrollo de distintos juegos que tienen la rueda como base.

Las ruedas, y por extensión las ruletas, siempre han tenido conexión con el mundo mágico y esotérico. Así, una de ellas forma parte del tarot, más precisamente de los que se conocen como arcanos mayores.

Según los indicios, la creación de una ruleta y sus normas de juego, muy similares a las que conocemos hoy en día, se debe a Blaise Pascal, matemático francés, quien ideó una ruleta con treinta y seis números (sin el cero), en la que se halla un extremado equilibrio en la posición en que está colocado cada número. La elección de 36 números da un alcance aún más vinculado a la magia (la suma de los primeros 36 números da el número mágico por excelencia: seiscientos sesenta y seis).

Esta ruleta podía usarse como entretenimiento en círculos de amistades. Sin embargo, a nivel de empresa que pone los medios y el personal para el entretenimiento de sus clientes, no era rentable, ya que estadísticamente todo lo que se apostaba se repartía en premios (probabilidad de 1/36 de acertar el número y ganar 36 veces lo apostado).

En 1842, los hermanos Blanc modificaron la ruleta añadiéndole un nuevo número, el 0, y la introdujeron inicialmente en el Casino de Montecarlo. Ésta es la ruleta que se conoce hoy en día, con una probabilidad de acertar de 1/37 y ganar 36 veces lo apostado, consiguiendo un margen para la casa del $2.7\%$ (1/37).

Más adelante, en algunas ruletas (sobre todo las que se usan en países anglosajones) se añadió un nuevo número (el doble cero), con lo cual el beneficio para el casino resultó ser doble (2/38 o $5.26\%$).

\section{Gráficas}
\begin{figure}[!htbp]
  \begin{mytikzresize}{0.6\textwidth}
    \centering
    % This file was created by tikzplotlib v0.9.1.
\begin{tikzpicture}

\definecolor{color0}{rgb}{0.12156862745098,0.466666666666667,0.705882352941177}
\definecolor{color1}{rgb}{1,0.498039215686275,0.0549019607843137}

\begin{axis}[
legend cell align={left},
legend style={fill opacity=0.5, draw opacity=1, text opacity=1, draw=white!80!black},
scaled ticks=false,
tick align=outside,
tick pos=left,
width=\figW,
x grid style={white!69.0196078431373!black},
xlabel={\(\displaystyle n\) (número de tiradas)},
xmajorgrids,
xmin=-48.95, xmax=1049.95,
xtick style={color=black},
xticklabel style={/pgf/number format/.cd,fixed,precision=2},
y grid style={white!69.0196078431373!black},
ylabel={\(\displaystyle f_{r}\) (frecuencia relativa)},
ymajorgrids,
ymin=-0.00327868852459016, ymax=0.0688524590163934,
ytick style={color=black},
yticklabel style={/pgf/number format/.cd,fixed,precision=2}
]
\addplot [semithick, color0]
table {%
1 0
2 0
3 0
4 0
5 0
6 0
7 0
8 0
9 0
10 0
11 0
12 0
13 0
14 0
15 0
16 0
17 0
18 0
19 0
20 0
21 0
22 0
23 0
24 0
25 0
26 0
27 0.037037037037037
28 0.0357142857142857
29 0.0344827586206897
30 0.0333333333333333
31 0.032258064516129
32 0.03125
33 0.0303030303030303
34 0.0294117647058824
35 0.0285714285714286
36 0.0277777777777778
37 0.027027027027027
38 0.0263157894736842
39 0.0256410256410256
40 0.025
41 0.0487804878048781
42 0.0476190476190476
43 0.0465116279069767
44 0.0454545454545455
45 0.0444444444444444
46 0.0434782608695652
47 0.0425531914893617
48 0.0416666666666667
49 0.0408163265306122
50 0.04
51 0.0392156862745098
52 0.0384615384615385
53 0.0377358490566038
54 0.037037037037037
55 0.0545454545454545
56 0.0535714285714286
57 0.0526315789473684
58 0.0517241379310345
59 0.0508474576271186
60 0.05
61 0.0655737704918033
62 0.0645161290322581
63 0.0634920634920635
64 0.0625
65 0.0615384615384615
66 0.0606060606060606
67 0.0597014925373134
68 0.0588235294117647
69 0.0579710144927536
70 0.0571428571428571
71 0.0563380281690141
72 0.0555555555555556
73 0.0547945205479452
74 0.0540540540540541
75 0.0533333333333333
76 0.0526315789473684
77 0.051948051948052
78 0.0512820512820513
79 0.0506329113924051
80 0.05
81 0.0493827160493827
82 0.0487804878048781
83 0.0481927710843374
84 0.0476190476190476
85 0.0470588235294118
86 0.0465116279069767
87 0.0459770114942529
88 0.0454545454545455
89 0.0449438202247191
90 0.0444444444444444
91 0.0549450549450549
92 0.0543478260869565
93 0.0537634408602151
94 0.0531914893617021
95 0.0526315789473684
96 0.0520833333333333
97 0.0515463917525773
98 0.0510204081632653
99 0.0505050505050505
100 0.05
101 0.0495049504950495
102 0.0490196078431373
103 0.0485436893203883
104 0.0480769230769231
105 0.0476190476190476
106 0.0471698113207547
107 0.0467289719626168
108 0.0462962962962963
109 0.0458715596330275
110 0.0454545454545455
111 0.045045045045045
112 0.0446428571428571
113 0.0442477876106195
114 0.043859649122807
115 0.0434782608695652
116 0.0431034482758621
117 0.0427350427350427
118 0.0423728813559322
119 0.0420168067226891
120 0.0416666666666667
121 0.0413223140495868
122 0.040983606557377
123 0.040650406504065
124 0.0403225806451613
125 0.04
126 0.0396825396825397
127 0.0393700787401575
128 0.0390625
129 0.0387596899224806
130 0.0384615384615385
131 0.0381679389312977
132 0.0378787878787879
133 0.037593984962406
134 0.0373134328358209
135 0.037037037037037
136 0.0367647058823529
137 0.0364963503649635
138 0.036231884057971
139 0.0359712230215827
140 0.0357142857142857
141 0.0354609929078014
142 0.0352112676056338
143 0.034965034965035
144 0.0347222222222222
145 0.0344827586206897
146 0.0342465753424658
147 0.0340136054421769
148 0.0337837837837838
149 0.0335570469798658
150 0.0333333333333333
151 0.033112582781457
152 0.0328947368421053
153 0.0326797385620915
154 0.0324675324675325
155 0.032258064516129
156 0.032051282051282
157 0.0318471337579618
158 0.0316455696202532
159 0.0314465408805031
160 0.03125
161 0.031055900621118
162 0.0308641975308642
163 0.0306748466257669
164 0.0304878048780488
165 0.0303030303030303
166 0.0301204819277108
167 0.029940119760479
168 0.0297619047619048
169 0.029585798816568
170 0.0294117647058824
171 0.0292397660818713
172 0.0290697674418605
173 0.0289017341040462
174 0.028735632183908
175 0.0285714285714286
176 0.0284090909090909
177 0.0282485875706215
178 0.0280898876404494
179 0.0279329608938547
180 0.0277777777777778
181 0.0276243093922652
182 0.0274725274725275
183 0.0273224043715847
184 0.0271739130434783
185 0.027027027027027
186 0.0268817204301075
187 0.0267379679144385
188 0.0265957446808511
189 0.0264550264550265
190 0.0263157894736842
191 0.0261780104712042
192 0.0260416666666667
193 0.0259067357512953
194 0.0257731958762887
195 0.0307692307692308
196 0.0306122448979592
197 0.0304568527918782
198 0.0303030303030303
199 0.0301507537688442
200 0.03
201 0.0298507462686567
202 0.0297029702970297
203 0.0295566502463054
204 0.0294117647058824
205 0.0292682926829268
206 0.029126213592233
207 0.0289855072463768
208 0.0288461538461538
209 0.0287081339712919
210 0.0285714285714286
211 0.028436018957346
212 0.0283018867924528
213 0.028169014084507
214 0.0280373831775701
215 0.027906976744186
216 0.0277777777777778
217 0.0276497695852535
218 0.0275229357798165
219 0.0273972602739726
220 0.0272727272727273
221 0.0271493212669683
222 0.027027027027027
223 0.0269058295964126
224 0.0267857142857143
225 0.0266666666666667
226 0.0265486725663717
227 0.026431718061674
228 0.0307017543859649
229 0.0305676855895196
230 0.0304347826086957
231 0.0303030303030303
232 0.0301724137931034
233 0.0300429184549356
234 0.0341880341880342
235 0.0340425531914894
236 0.0338983050847458
237 0.0337552742616034
238 0.0336134453781513
239 0.0334728033472803
240 0.0333333333333333
241 0.033195020746888
242 0.0330578512396694
243 0.0329218106995885
244 0.0327868852459016
245 0.0326530612244898
246 0.032520325203252
247 0.0323886639676113
248 0.032258064516129
249 0.0321285140562249
250 0.032
251 0.0318725099601594
252 0.0317460317460317
253 0.0316205533596838
254 0.031496062992126
255 0.0313725490196078
256 0.03125
257 0.0311284046692607
258 0.0310077519379845
259 0.0308880308880309
260 0.0307692307692308
261 0.0306513409961686
262 0.0305343511450382
263 0.0304182509505703
264 0.0303030303030303
265 0.030188679245283
266 0.0300751879699248
267 0.0299625468164794
268 0.0298507462686567
269 0.033457249070632
270 0.0333333333333333
271 0.033210332103321
272 0.0330882352941176
273 0.032967032967033
274 0.0328467153284672
275 0.0327272727272727
276 0.0326086956521739
277 0.0324909747292419
278 0.0323741007194245
279 0.032258064516129
280 0.0321428571428571
281 0.0320284697508897
282 0.0319148936170213
283 0.0318021201413428
284 0.0316901408450704
285 0.0315789473684211
286 0.0314685314685315
287 0.0313588850174216
288 0.03125
289 0.0311418685121107
290 0.0310344827586207
291 0.0309278350515464
292 0.0308219178082192
293 0.0307167235494881
294 0.0306122448979592
295 0.0305084745762712
296 0.0304054054054054
297 0.0303030303030303
298 0.0302013422818792
299 0.0301003344481605
300 0.03
301 0.0299003322259136
302 0.0298013245033113
303 0.0297029702970297
304 0.0296052631578947
305 0.0327868852459016
306 0.0359477124183007
307 0.0358306188925081
308 0.0357142857142857
309 0.0355987055016181
310 0.0354838709677419
311 0.0353697749196141
312 0.0352564102564103
313 0.0351437699680511
314 0.035031847133758
315 0.0349206349206349
316 0.0348101265822785
317 0.0347003154574132
318 0.0345911949685535
319 0.0344827586206897
320 0.034375
321 0.0342679127725857
322 0.0341614906832298
323 0.0340557275541796
324 0.0339506172839506
325 0.0338461538461538
326 0.0337423312883436
327 0.0336391437308868
328 0.0335365853658537
329 0.033434650455927
330 0.0333333333333333
331 0.0332326283987915
332 0.0331325301204819
333 0.033033033033033
334 0.0329341317365269
335 0.0328358208955224
336 0.0357142857142857
337 0.0356083086053412
338 0.0355029585798817
339 0.0353982300884956
340 0.0352941176470588
341 0.0351906158357771
342 0.0350877192982456
343 0.0349854227405248
344 0.0348837209302326
345 0.0347826086956522
346 0.0346820809248555
347 0.0345821325648415
348 0.0344827586206897
349 0.0343839541547278
350 0.0342857142857143
351 0.0341880341880342
352 0.0340909090909091
353 0.0339943342776204
354 0.0338983050847458
355 0.0338028169014084
356 0.0337078651685393
357 0.0336134453781513
358 0.0335195530726257
359 0.0334261838440111
360 0.0333333333333333
361 0.0332409972299169
362 0.0331491712707182
363 0.0330578512396694
364 0.032967032967033
365 0.0328767123287671
366 0.0327868852459016
367 0.0326975476839237
368 0.0326086956521739
369 0.032520325203252
370 0.0324324324324324
371 0.032345013477089
372 0.032258064516129
373 0.032171581769437
374 0.0320855614973262
375 0.032
376 0.0319148936170213
377 0.0318302387267905
378 0.0317460317460317
379 0.0316622691292876
380 0.0315789473684211
381 0.031496062992126
382 0.031413612565445
383 0.031331592689295
384 0.03125
385 0.0311688311688312
386 0.0310880829015544
387 0.0310077519379845
388 0.0309278350515464
389 0.0308483290488432
390 0.0307692307692308
391 0.030690537084399
392 0.0306122448979592
393 0.0305343511450382
394 0.0304568527918782
395 0.030379746835443
396 0.0303030303030303
397 0.0302267002518892
398 0.0301507537688442
399 0.0300751879699248
400 0.03
401 0.029925187032419
402 0.0298507462686567
403 0.0297766749379653
404 0.0297029702970297
405 0.0296296296296296
406 0.0320197044334975
407 0.0319410319410319
408 0.0318627450980392
409 0.0317848410757946
410 0.0317073170731707
411 0.0316301703163017
412 0.0315533980582524
413 0.0314769975786925
414 0.0314009661835749
415 0.0313253012048193
416 0.03125
417 0.0311750599520384
418 0.0311004784688995
419 0.0310262529832936
420 0.030952380952381
421 0.0308788598574822
422 0.0308056872037915
423 0.0307328605200946
424 0.0306603773584906
425 0.0305882352941176
426 0.0305164319248826
427 0.0304449648711944
428 0.0303738317757009
429 0.0303030303030303
430 0.0302325581395349
431 0.0301624129930394
432 0.0300925925925926
433 0.0300230946882217
434 0.0299539170506912
435 0.0298850574712644
436 0.0298165137614679
437 0.0297482837528604
438 0.0296803652968037
439 0.0296127562642369
440 0.0295454545454545
441 0.0294784580498866
442 0.0294117647058824
443 0.0293453724604966
444 0.0292792792792793
445 0.0292134831460674
446 0.0291479820627803
447 0.029082774049217
448 0.0290178571428571
449 0.0289532293986637
450 0.0288888888888889
451 0.0288248337028825
452 0.0287610619469027
453 0.0286975717439294
454 0.0286343612334802
455 0.0285714285714286
456 0.0285087719298246
457 0.0284463894967177
458 0.0283842794759825
459 0.028322440087146
460 0.0282608695652174
461 0.0281995661605206
462 0.0281385281385281
463 0.0280777537796976
464 0.0280172413793103
465 0.0279569892473118
466 0.0278969957081545
467 0.0278372591006424
468 0.0277777777777778
469 0.0277185501066098
470 0.0276595744680851
471 0.0276008492569002
472 0.0296610169491525
473 0.0295983086680761
474 0.029535864978903
475 0.0294736842105263
476 0.0294117647058824
477 0.0293501048218029
478 0.0292887029288703
479 0.0292275574112735
480 0.0291666666666667
481 0.0311850311850312
482 0.0311203319502075
483 0.031055900621118
484 0.0309917355371901
485 0.0309278350515464
486 0.0308641975308642
487 0.0308008213552361
488 0.0307377049180328
489 0.0306748466257669
490 0.0306122448979592
491 0.0305498981670061
492 0.0304878048780488
493 0.0304259634888438
494 0.0303643724696356
495 0.0323232323232323
496 0.032258064516129
497 0.0321931589537223
498 0.0321285140562249
499 0.032064128256513
500 0.032
501 0.031936127744511
502 0.0318725099601594
503 0.0318091451292246
504 0.0317460317460317
505 0.0316831683168317
506 0.0316205533596838
507 0.0315581854043393
508 0.031496062992126
509 0.031434184675835
510 0.0313725490196078
511 0.0313111545988258
512 0.03125
513 0.0311890838206628
514 0.0311284046692607
515 0.0310679611650485
516 0.0310077519379845
517 0.0309477756286267
518 0.0308880308880309
519 0.0308285163776493
520 0.0307692307692308
521 0.0307101727447217
522 0.0306513409961686
523 0.0305927342256214
524 0.0305343511450382
525 0.0304761904761905
526 0.0304182509505703
527 0.0303605313092979
528 0.0303030303030303
529 0.0302457466918715
530 0.030188679245283
531 0.0301318267419962
532 0.0300751879699248
533 0.0300187617260788
534 0.0299625468164794
535 0.0299065420560748
536 0.0298507462686567
537 0.0297951582867784
538 0.0297397769516729
539 0.0296846011131725
540 0.0296296296296296
541 0.0295748613678373
542 0.029520295202952
543 0.0294659300184162
544 0.0294117647058824
545 0.0293577981651376
546 0.0293040293040293
547 0.0292504570383912
548 0.0291970802919708
549 0.029143897996357
550 0.0290909090909091
551 0.029038112522686
552 0.0289855072463768
553 0.0289330922242315
554 0.0288808664259928
555 0.0288288288288288
556 0.0287769784172662
557 0.0287253141831239
558 0.028673835125448
559 0.0286225402504472
560 0.0285714285714286
561 0.0285204991087344
562 0.0284697508896797
563 0.0284191829484902
564 0.0283687943262411
565 0.0283185840707965
566 0.0282685512367491
567 0.0282186948853616
568 0.028169014084507
569 0.0281195079086116
570 0.0280701754385965
571 0.0280210157618214
572 0.0297202797202797
573 0.0296684118673647
574 0.029616724738676
575 0.0295652173913043
576 0.0295138888888889
577 0.0294627383015598
578 0.0294117647058824
579 0.0310880829015544
580 0.0310344827586207
581 0.0309810671256454
582 0.0309278350515464
583 0.0308747855917667
584 0.0308219178082192
585 0.0307692307692308
586 0.0307167235494881
587 0.030664395229983
588 0.0306122448979592
589 0.0305602716468591
590 0.0305084745762712
591 0.0304568527918782
592 0.0304054054054054
593 0.03035413153457
594 0.0303030303030303
595 0.0302521008403361
596 0.0302013422818792
597 0.0301507537688442
598 0.0301003344481605
599 0.0317195325542571
600 0.0316666666666667
601 0.0316139767054908
602 0.0315614617940199
603 0.0315091210613599
604 0.0314569536423841
605 0.031404958677686
606 0.0313531353135314
607 0.0313014827018122
608 0.03125
609 0.0328407224958949
610 0.0327868852459016
611 0.0327332242225859
612 0.0326797385620915
613 0.032626427406199
614 0.0325732899022801
615 0.032520325203252
616 0.0324675324675325
617 0.0324149108589951
618 0.0323624595469256
619 0.0323101777059774
620 0.032258064516129
621 0.0322061191626409
622 0.0321543408360129
623 0.0321027287319422
624 0.032051282051282
625 0.032
626 0.0319488817891374
627 0.0318979266347687
628 0.0318471337579618
629 0.0317965023847377
630 0.0333333333333333
631 0.0332805071315372
632 0.0332278481012658
633 0.033175355450237
634 0.0331230283911672
635 0.0330708661417323
636 0.0330188679245283
637 0.032967032967033
638 0.0329153605015674
639 0.0328638497652582
640 0.0328125
641 0.0327613104524181
642 0.0327102803738318
643 0.0326594090202177
644 0.0326086956521739
645 0.0325581395348837
646 0.0325077399380805
647 0.0324574961360124
648 0.0324074074074074
649 0.0323574730354391
650 0.0323076923076923
651 0.032258064516129
652 0.0322085889570552
653 0.0321592649310873
654 0.0321100917431193
655 0.0320610687022901
656 0.0320121951219512
657 0.0319634703196347
658 0.0319148936170213
659 0.031866464339909
660 0.0318181818181818
661 0.0317700453857791
662 0.0317220543806647
663 0.0316742081447964
664 0.0316265060240964
665 0.0315789473684211
666 0.0315315315315315
667 0.0314842578710645
668 0.031437125748503
669 0.031390134529148
670 0.0313432835820895
671 0.0327868852459016
672 0.0327380952380952
673 0.0341753343239227
674 0.0341246290801187
675 0.0340740740740741
676 0.0340236686390533
677 0.03397341211226
678 0.0339233038348083
679 0.0338733431516937
680 0.0338235294117647
681 0.0337738619676946
682 0.0337243401759531
683 0.0336749633967789
684 0.033625730994152
685 0.0335766423357664
686 0.0335276967930029
687 0.0334788937409025
688 0.0334302325581395
689 0.0333817126269956
690 0.0333333333333333
691 0.0332850940665702
692 0.0332369942196532
693 0.0331890331890332
694 0.0331412103746398
695 0.0330935251798561
696 0.0330459770114943
697 0.0329985652797704
698 0.0329512893982808
699 0.0329041487839771
700 0.0328571428571429
701 0.0328102710413695
702 0.0327635327635328
703 0.0327169274537696
704 0.0326704545454545
705 0.0340425531914894
706 0.0339943342776204
707 0.0339462517680339
708 0.0338983050847458
709 0.0338504936530324
710 0.0338028169014084
711 0.0337552742616034
712 0.0337078651685393
713 0.0336605890603086
714 0.0336134453781513
715 0.0335664335664336
716 0.0335195530726257
717 0.0334728033472803
718 0.0334261838440111
719 0.0333796940194715
720 0.0333333333333333
721 0.0332871012482663
722 0.0332409972299169
723 0.033195020746888
724 0.0331491712707182
725 0.0331034482758621
726 0.0330578512396694
727 0.0330123796423659
728 0.032967032967033
729 0.0329218106995885
730 0.0328767123287671
731 0.0328317373461012
732 0.0327868852459016
733 0.0327421555252387
734 0.0326975476839237
735 0.0326530612244898
736 0.0326086956521739
737 0.0325644504748982
738 0.032520325203252
739 0.0324763193504736
740 0.0324324324324324
741 0.0323886639676113
742 0.032345013477089
743 0.0323014804845222
744 0.032258064516129
745 0.0322147651006711
746 0.032171581769437
747 0.0321285140562249
748 0.0320855614973262
749 0.0320427236315087
750 0.032
751 0.0319573901464714
752 0.0319148936170213
753 0.0318725099601594
754 0.0318302387267905
755 0.0317880794701987
756 0.0317460317460317
757 0.0317040951122853
758 0.0316622691292876
759 0.0316205533596838
760 0.0315789473684211
761 0.0315374507227332
762 0.031496062992126
763 0.0314547837483617
764 0.031413612565445
765 0.0313725490196078
766 0.031331592689295
767 0.0312907431551499
768 0.03125
769 0.0312093628088427
770 0.0311688311688312
771 0.0311284046692607
772 0.0310880829015544
773 0.0310478654592497
774 0.0310077519379845
775 0.0309677419354839
776 0.0309278350515464
777 0.0308880308880309
778 0.0308483290488432
779 0.030808729139923
780 0.0307692307692308
781 0.030729833546735
782 0.030690537084399
783 0.0306513409961686
784 0.0306122448979592
785 0.0305732484076433
786 0.0305343511450382
787 0.0304955527318933
788 0.0304568527918782
789 0.0304182509505703
790 0.030379746835443
791 0.0303413400758533
792 0.0303030303030303
793 0.0302648171500631
794 0.0302267002518892
795 0.030188679245283
796 0.0301507537688442
797 0.0301129234629862
798 0.0300751879699248
799 0.0300375469336671
800 0.03
801 0.0299625468164794
802 0.0311720698254364
803 0.0311332503113325
804 0.0310945273631841
805 0.031055900621118
806 0.0310173697270471
807 0.0309789343246592
808 0.0309405940594059
809 0.030902348578492
810 0.0308641975308642
811 0.030826140567201
812 0.0307881773399015
813 0.030750307503075
814 0.0307125307125307
815 0.0306748466257669
816 0.0306372549019608
817 0.0305997552019584
818 0.0305623471882641
819 0.0305250305250305
820 0.0304878048780488
821 0.0304506699147381
822 0.0304136253041363
823 0.0303766707168894
824 0.0303398058252427
825 0.0303030303030303
826 0.0302663438256659
827 0.030229746070133
828 0.0301932367149758
829 0.0301568154402895
830 0.0301204819277108
831 0.0300842358604091
832 0.0300480769230769
833 0.0300120048019208
834 0.0299760191846523
835 0.029940119760479
836 0.0299043062200957
837 0.031063321385902
838 0.0310262529832936
839 0.0309892729439809
840 0.030952380952381
841 0.0309155766944114
842 0.0308788598574822
843 0.0308422301304864
844 0.0308056872037915
845 0.0307692307692308
846 0.0307328605200946
847 0.0306965761511216
848 0.0306603773584906
849 0.0306242638398115
850 0.0305882352941176
851 0.0305522914218566
852 0.0305164319248826
853 0.0304806565064478
854 0.0304449648711944
855 0.0304093567251462
856 0.0303738317757009
857 0.0303383897316219
858 0.0303030303030303
859 0.030267753201397
860 0.0302325581395349
861 0.0301974448315912
862 0.0301624129930394
863 0.0301274623406721
864 0.0300925925925926
865 0.0300578034682081
866 0.0300230946882217
867 0.0299884659746251
868 0.0299539170506912
869 0.0299194476409666
870 0.0298850574712644
871 0.0298507462686567
872 0.0298165137614679
873 0.0297823596792669
874 0.0297482837528604
875 0.0297142857142857
876 0.0296803652968037
877 0.0296465222348917
878 0.030751708428246
879 0.0307167235494881
880 0.0306818181818182
881 0.0306469920544835
882 0.0306122448979592
883 0.0305775764439411
884 0.0305429864253394
885 0.0305084745762712
886 0.0304740406320542
887 0.0304396843291995
888 0.0304054054054054
889 0.0303712035995501
890 0.0303370786516854
891 0.0303030303030303
892 0.0302690582959641
893 0.0302351623740202
894 0.0302013422818792
895 0.0301675977653631
896 0.0301339285714286
897 0.0301003344481605
898 0.0300668151447661
899 0.0300333704115684
900 0.03
901 0.0299667036625971
902 0.0299334811529933
903 0.0299003322259136
904 0.0298672566371681
905 0.0298342541436464
906 0.0298013245033113
907 0.0297684674751929
908 0.0297356828193833
909 0.0297029702970297
910 0.0296703296703297
911 0.0296377607025247
912 0.0296052631578947
913 0.0295728368017525
914 0.0306345733041575
915 0.0306010928961749
916 0.0305676855895196
917 0.0305343511450382
918 0.0305010893246187
919 0.0304678998911861
920 0.0304347826086957
921 0.0304017372421281
922 0.0303687635574837
923 0.0303358613217768
924 0.0303030303030303
925 0.0302702702702703
926 0.0302375809935205
927 0.0302049622437972
928 0.0301724137931034
929 0.0301399354144241
930 0.0301075268817204
931 0.0300751879699248
932 0.0300429184549356
933 0.030010718113612
934 0.0299785867237687
935 0.0299465240641711
936 0.0299145299145299
937 0.0298826040554963
938 0.0309168443496802
939 0.0308839190628328
940 0.0308510638297872
941 0.0308182784272051
942 0.0307855626326964
943 0.0307529162248144
944 0.0307203389830508
945 0.0306878306878307
946 0.0306553911205074
947 0.030623020063358
948 0.0305907172995781
949 0.0305584826132771
950 0.0305263157894737
951 0.0304942166140904
952 0.0304621848739496
953 0.0304302203567681
954 0.030398322851153
955 0.0303664921465969
956 0.0303347280334728
957 0.0303030303030303
958 0.0302713987473904
959 0.0302398331595412
960 0.0302083333333333
961 0.0301768990634755
962 0.0301455301455301
963 0.0301142263759086
964 0.0300829875518672
965 0.0300518134715026
966 0.0300207039337474
967 0.0299896587383661
968 0.0299586776859504
969 0.0299277605779154
970 0.0298969072164948
971 0.0298661174047374
972 0.0298353909465021
973 0.0298047276464543
974 0.0297741273100616
975 0.0297435897435897
976 0.0297131147540984
977 0.0296827021494371
978 0.0296523517382413
979 0.0296220633299285
980 0.0295918367346939
981 0.0295616717635066
982 0.0295315682281059
983 0.0295015259409969
984 0.0294715447154472
985 0.0294416243654822
986 0.0294117647058824
987 0.0293819655521783
988 0.0293522267206478
989 0.0293225480283114
990 0.0292929292929293
991 0.029263370332997
992 0.0292338709677419
993 0.0292044310171198
994 0.0291750503018109
995 0.0291457286432161
996 0.0291164658634538
997 0.0290872617853561
998 0.0290581162324649
999 0.029029029029029
1000 0.029
};
\addlegendentry{$f_{r}$ (frecuencia relativa de $18$)}
\addplot [semithick, color1]
table {%
1 0.027027027027027
2 0.027027027027027
3 0.027027027027027
4 0.027027027027027
5 0.027027027027027
6 0.027027027027027
7 0.027027027027027
8 0.027027027027027
9 0.027027027027027
10 0.027027027027027
11 0.027027027027027
12 0.027027027027027
13 0.027027027027027
14 0.027027027027027
15 0.027027027027027
16 0.027027027027027
17 0.027027027027027
18 0.027027027027027
19 0.027027027027027
20 0.027027027027027
21 0.027027027027027
22 0.027027027027027
23 0.027027027027027
24 0.027027027027027
25 0.027027027027027
26 0.027027027027027
27 0.027027027027027
28 0.027027027027027
29 0.027027027027027
30 0.027027027027027
31 0.027027027027027
32 0.027027027027027
33 0.027027027027027
34 0.027027027027027
35 0.027027027027027
36 0.027027027027027
37 0.027027027027027
38 0.027027027027027
39 0.027027027027027
40 0.027027027027027
41 0.027027027027027
42 0.027027027027027
43 0.027027027027027
44 0.027027027027027
45 0.027027027027027
46 0.027027027027027
47 0.027027027027027
48 0.027027027027027
49 0.027027027027027
50 0.027027027027027
51 0.027027027027027
52 0.027027027027027
53 0.027027027027027
54 0.027027027027027
55 0.027027027027027
56 0.027027027027027
57 0.027027027027027
58 0.027027027027027
59 0.027027027027027
60 0.027027027027027
61 0.027027027027027
62 0.027027027027027
63 0.027027027027027
64 0.027027027027027
65 0.027027027027027
66 0.027027027027027
67 0.027027027027027
68 0.027027027027027
69 0.027027027027027
70 0.027027027027027
71 0.027027027027027
72 0.027027027027027
73 0.027027027027027
74 0.027027027027027
75 0.027027027027027
76 0.027027027027027
77 0.027027027027027
78 0.027027027027027
79 0.027027027027027
80 0.027027027027027
81 0.027027027027027
82 0.027027027027027
83 0.027027027027027
84 0.027027027027027
85 0.027027027027027
86 0.027027027027027
87 0.027027027027027
88 0.027027027027027
89 0.027027027027027
90 0.027027027027027
91 0.027027027027027
92 0.027027027027027
93 0.027027027027027
94 0.027027027027027
95 0.027027027027027
96 0.027027027027027
97 0.027027027027027
98 0.027027027027027
99 0.027027027027027
100 0.027027027027027
101 0.027027027027027
102 0.027027027027027
103 0.027027027027027
104 0.027027027027027
105 0.027027027027027
106 0.027027027027027
107 0.027027027027027
108 0.027027027027027
109 0.027027027027027
110 0.027027027027027
111 0.027027027027027
112 0.027027027027027
113 0.027027027027027
114 0.027027027027027
115 0.027027027027027
116 0.027027027027027
117 0.027027027027027
118 0.027027027027027
119 0.027027027027027
120 0.027027027027027
121 0.027027027027027
122 0.027027027027027
123 0.027027027027027
124 0.027027027027027
125 0.027027027027027
126 0.027027027027027
127 0.027027027027027
128 0.027027027027027
129 0.027027027027027
130 0.027027027027027
131 0.027027027027027
132 0.027027027027027
133 0.027027027027027
134 0.027027027027027
135 0.027027027027027
136 0.027027027027027
137 0.027027027027027
138 0.027027027027027
139 0.027027027027027
140 0.027027027027027
141 0.027027027027027
142 0.027027027027027
143 0.027027027027027
144 0.027027027027027
145 0.027027027027027
146 0.027027027027027
147 0.027027027027027
148 0.027027027027027
149 0.027027027027027
150 0.027027027027027
151 0.027027027027027
152 0.027027027027027
153 0.027027027027027
154 0.027027027027027
155 0.027027027027027
156 0.027027027027027
157 0.027027027027027
158 0.027027027027027
159 0.027027027027027
160 0.027027027027027
161 0.027027027027027
162 0.027027027027027
163 0.027027027027027
164 0.027027027027027
165 0.027027027027027
166 0.027027027027027
167 0.027027027027027
168 0.027027027027027
169 0.027027027027027
170 0.027027027027027
171 0.027027027027027
172 0.027027027027027
173 0.027027027027027
174 0.027027027027027
175 0.027027027027027
176 0.027027027027027
177 0.027027027027027
178 0.027027027027027
179 0.027027027027027
180 0.027027027027027
181 0.027027027027027
182 0.027027027027027
183 0.027027027027027
184 0.027027027027027
185 0.027027027027027
186 0.027027027027027
187 0.027027027027027
188 0.027027027027027
189 0.027027027027027
190 0.027027027027027
191 0.027027027027027
192 0.027027027027027
193 0.027027027027027
194 0.027027027027027
195 0.027027027027027
196 0.027027027027027
197 0.027027027027027
198 0.027027027027027
199 0.027027027027027
200 0.027027027027027
201 0.027027027027027
202 0.027027027027027
203 0.027027027027027
204 0.027027027027027
205 0.027027027027027
206 0.027027027027027
207 0.027027027027027
208 0.027027027027027
209 0.027027027027027
210 0.027027027027027
211 0.027027027027027
212 0.027027027027027
213 0.027027027027027
214 0.027027027027027
215 0.027027027027027
216 0.027027027027027
217 0.027027027027027
218 0.027027027027027
219 0.027027027027027
220 0.027027027027027
221 0.027027027027027
222 0.027027027027027
223 0.027027027027027
224 0.027027027027027
225 0.027027027027027
226 0.027027027027027
227 0.027027027027027
228 0.027027027027027
229 0.027027027027027
230 0.027027027027027
231 0.027027027027027
232 0.027027027027027
233 0.027027027027027
234 0.027027027027027
235 0.027027027027027
236 0.027027027027027
237 0.027027027027027
238 0.027027027027027
239 0.027027027027027
240 0.027027027027027
241 0.027027027027027
242 0.027027027027027
243 0.027027027027027
244 0.027027027027027
245 0.027027027027027
246 0.027027027027027
247 0.027027027027027
248 0.027027027027027
249 0.027027027027027
250 0.027027027027027
251 0.027027027027027
252 0.027027027027027
253 0.027027027027027
254 0.027027027027027
255 0.027027027027027
256 0.027027027027027
257 0.027027027027027
258 0.027027027027027
259 0.027027027027027
260 0.027027027027027
261 0.027027027027027
262 0.027027027027027
263 0.027027027027027
264 0.027027027027027
265 0.027027027027027
266 0.027027027027027
267 0.027027027027027
268 0.027027027027027
269 0.027027027027027
270 0.027027027027027
271 0.027027027027027
272 0.027027027027027
273 0.027027027027027
274 0.027027027027027
275 0.027027027027027
276 0.027027027027027
277 0.027027027027027
278 0.027027027027027
279 0.027027027027027
280 0.027027027027027
281 0.027027027027027
282 0.027027027027027
283 0.027027027027027
284 0.027027027027027
285 0.027027027027027
286 0.027027027027027
287 0.027027027027027
288 0.027027027027027
289 0.027027027027027
290 0.027027027027027
291 0.027027027027027
292 0.027027027027027
293 0.027027027027027
294 0.027027027027027
295 0.027027027027027
296 0.027027027027027
297 0.027027027027027
298 0.027027027027027
299 0.027027027027027
300 0.027027027027027
301 0.027027027027027
302 0.027027027027027
303 0.027027027027027
304 0.027027027027027
305 0.027027027027027
306 0.027027027027027
307 0.027027027027027
308 0.027027027027027
309 0.027027027027027
310 0.027027027027027
311 0.027027027027027
312 0.027027027027027
313 0.027027027027027
314 0.027027027027027
315 0.027027027027027
316 0.027027027027027
317 0.027027027027027
318 0.027027027027027
319 0.027027027027027
320 0.027027027027027
321 0.027027027027027
322 0.027027027027027
323 0.027027027027027
324 0.027027027027027
325 0.027027027027027
326 0.027027027027027
327 0.027027027027027
328 0.027027027027027
329 0.027027027027027
330 0.027027027027027
331 0.027027027027027
332 0.027027027027027
333 0.027027027027027
334 0.027027027027027
335 0.027027027027027
336 0.027027027027027
337 0.027027027027027
338 0.027027027027027
339 0.027027027027027
340 0.027027027027027
341 0.027027027027027
342 0.027027027027027
343 0.027027027027027
344 0.027027027027027
345 0.027027027027027
346 0.027027027027027
347 0.027027027027027
348 0.027027027027027
349 0.027027027027027
350 0.027027027027027
351 0.027027027027027
352 0.027027027027027
353 0.027027027027027
354 0.027027027027027
355 0.027027027027027
356 0.027027027027027
357 0.027027027027027
358 0.027027027027027
359 0.027027027027027
360 0.027027027027027
361 0.027027027027027
362 0.027027027027027
363 0.027027027027027
364 0.027027027027027
365 0.027027027027027
366 0.027027027027027
367 0.027027027027027
368 0.027027027027027
369 0.027027027027027
370 0.027027027027027
371 0.027027027027027
372 0.027027027027027
373 0.027027027027027
374 0.027027027027027
375 0.027027027027027
376 0.027027027027027
377 0.027027027027027
378 0.027027027027027
379 0.027027027027027
380 0.027027027027027
381 0.027027027027027
382 0.027027027027027
383 0.027027027027027
384 0.027027027027027
385 0.027027027027027
386 0.027027027027027
387 0.027027027027027
388 0.027027027027027
389 0.027027027027027
390 0.027027027027027
391 0.027027027027027
392 0.027027027027027
393 0.027027027027027
394 0.027027027027027
395 0.027027027027027
396 0.027027027027027
397 0.027027027027027
398 0.027027027027027
399 0.027027027027027
400 0.027027027027027
401 0.027027027027027
402 0.027027027027027
403 0.027027027027027
404 0.027027027027027
405 0.027027027027027
406 0.027027027027027
407 0.027027027027027
408 0.027027027027027
409 0.027027027027027
410 0.027027027027027
411 0.027027027027027
412 0.027027027027027
413 0.027027027027027
414 0.027027027027027
415 0.027027027027027
416 0.027027027027027
417 0.027027027027027
418 0.027027027027027
419 0.027027027027027
420 0.027027027027027
421 0.027027027027027
422 0.027027027027027
423 0.027027027027027
424 0.027027027027027
425 0.027027027027027
426 0.027027027027027
427 0.027027027027027
428 0.027027027027027
429 0.027027027027027
430 0.027027027027027
431 0.027027027027027
432 0.027027027027027
433 0.027027027027027
434 0.027027027027027
435 0.027027027027027
436 0.027027027027027
437 0.027027027027027
438 0.027027027027027
439 0.027027027027027
440 0.027027027027027
441 0.027027027027027
442 0.027027027027027
443 0.027027027027027
444 0.027027027027027
445 0.027027027027027
446 0.027027027027027
447 0.027027027027027
448 0.027027027027027
449 0.027027027027027
450 0.027027027027027
451 0.027027027027027
452 0.027027027027027
453 0.027027027027027
454 0.027027027027027
455 0.027027027027027
456 0.027027027027027
457 0.027027027027027
458 0.027027027027027
459 0.027027027027027
460 0.027027027027027
461 0.027027027027027
462 0.027027027027027
463 0.027027027027027
464 0.027027027027027
465 0.027027027027027
466 0.027027027027027
467 0.027027027027027
468 0.027027027027027
469 0.027027027027027
470 0.027027027027027
471 0.027027027027027
472 0.027027027027027
473 0.027027027027027
474 0.027027027027027
475 0.027027027027027
476 0.027027027027027
477 0.027027027027027
478 0.027027027027027
479 0.027027027027027
480 0.027027027027027
481 0.027027027027027
482 0.027027027027027
483 0.027027027027027
484 0.027027027027027
485 0.027027027027027
486 0.027027027027027
487 0.027027027027027
488 0.027027027027027
489 0.027027027027027
490 0.027027027027027
491 0.027027027027027
492 0.027027027027027
493 0.027027027027027
494 0.027027027027027
495 0.027027027027027
496 0.027027027027027
497 0.027027027027027
498 0.027027027027027
499 0.027027027027027
500 0.027027027027027
501 0.027027027027027
502 0.027027027027027
503 0.027027027027027
504 0.027027027027027
505 0.027027027027027
506 0.027027027027027
507 0.027027027027027
508 0.027027027027027
509 0.027027027027027
510 0.027027027027027
511 0.027027027027027
512 0.027027027027027
513 0.027027027027027
514 0.027027027027027
515 0.027027027027027
516 0.027027027027027
517 0.027027027027027
518 0.027027027027027
519 0.027027027027027
520 0.027027027027027
521 0.027027027027027
522 0.027027027027027
523 0.027027027027027
524 0.027027027027027
525 0.027027027027027
526 0.027027027027027
527 0.027027027027027
528 0.027027027027027
529 0.027027027027027
530 0.027027027027027
531 0.027027027027027
532 0.027027027027027
533 0.027027027027027
534 0.027027027027027
535 0.027027027027027
536 0.027027027027027
537 0.027027027027027
538 0.027027027027027
539 0.027027027027027
540 0.027027027027027
541 0.027027027027027
542 0.027027027027027
543 0.027027027027027
544 0.027027027027027
545 0.027027027027027
546 0.027027027027027
547 0.027027027027027
548 0.027027027027027
549 0.027027027027027
550 0.027027027027027
551 0.027027027027027
552 0.027027027027027
553 0.027027027027027
554 0.027027027027027
555 0.027027027027027
556 0.027027027027027
557 0.027027027027027
558 0.027027027027027
559 0.027027027027027
560 0.027027027027027
561 0.027027027027027
562 0.027027027027027
563 0.027027027027027
564 0.027027027027027
565 0.027027027027027
566 0.027027027027027
567 0.027027027027027
568 0.027027027027027
569 0.027027027027027
570 0.027027027027027
571 0.027027027027027
572 0.027027027027027
573 0.027027027027027
574 0.027027027027027
575 0.027027027027027
576 0.027027027027027
577 0.027027027027027
578 0.027027027027027
579 0.027027027027027
580 0.027027027027027
581 0.027027027027027
582 0.027027027027027
583 0.027027027027027
584 0.027027027027027
585 0.027027027027027
586 0.027027027027027
587 0.027027027027027
588 0.027027027027027
589 0.027027027027027
590 0.027027027027027
591 0.027027027027027
592 0.027027027027027
593 0.027027027027027
594 0.027027027027027
595 0.027027027027027
596 0.027027027027027
597 0.027027027027027
598 0.027027027027027
599 0.027027027027027
600 0.027027027027027
601 0.027027027027027
602 0.027027027027027
603 0.027027027027027
604 0.027027027027027
605 0.027027027027027
606 0.027027027027027
607 0.027027027027027
608 0.027027027027027
609 0.027027027027027
610 0.027027027027027
611 0.027027027027027
612 0.027027027027027
613 0.027027027027027
614 0.027027027027027
615 0.027027027027027
616 0.027027027027027
617 0.027027027027027
618 0.027027027027027
619 0.027027027027027
620 0.027027027027027
621 0.027027027027027
622 0.027027027027027
623 0.027027027027027
624 0.027027027027027
625 0.027027027027027
626 0.027027027027027
627 0.027027027027027
628 0.027027027027027
629 0.027027027027027
630 0.027027027027027
631 0.027027027027027
632 0.027027027027027
633 0.027027027027027
634 0.027027027027027
635 0.027027027027027
636 0.027027027027027
637 0.027027027027027
638 0.027027027027027
639 0.027027027027027
640 0.027027027027027
641 0.027027027027027
642 0.027027027027027
643 0.027027027027027
644 0.027027027027027
645 0.027027027027027
646 0.027027027027027
647 0.027027027027027
648 0.027027027027027
649 0.027027027027027
650 0.027027027027027
651 0.027027027027027
652 0.027027027027027
653 0.027027027027027
654 0.027027027027027
655 0.027027027027027
656 0.027027027027027
657 0.027027027027027
658 0.027027027027027
659 0.027027027027027
660 0.027027027027027
661 0.027027027027027
662 0.027027027027027
663 0.027027027027027
664 0.027027027027027
665 0.027027027027027
666 0.027027027027027
667 0.027027027027027
668 0.027027027027027
669 0.027027027027027
670 0.027027027027027
671 0.027027027027027
672 0.027027027027027
673 0.027027027027027
674 0.027027027027027
675 0.027027027027027
676 0.027027027027027
677 0.027027027027027
678 0.027027027027027
679 0.027027027027027
680 0.027027027027027
681 0.027027027027027
682 0.027027027027027
683 0.027027027027027
684 0.027027027027027
685 0.027027027027027
686 0.027027027027027
687 0.027027027027027
688 0.027027027027027
689 0.027027027027027
690 0.027027027027027
691 0.027027027027027
692 0.027027027027027
693 0.027027027027027
694 0.027027027027027
695 0.027027027027027
696 0.027027027027027
697 0.027027027027027
698 0.027027027027027
699 0.027027027027027
700 0.027027027027027
701 0.027027027027027
702 0.027027027027027
703 0.027027027027027
704 0.027027027027027
705 0.027027027027027
706 0.027027027027027
707 0.027027027027027
708 0.027027027027027
709 0.027027027027027
710 0.027027027027027
711 0.027027027027027
712 0.027027027027027
713 0.027027027027027
714 0.027027027027027
715 0.027027027027027
716 0.027027027027027
717 0.027027027027027
718 0.027027027027027
719 0.027027027027027
720 0.027027027027027
721 0.027027027027027
722 0.027027027027027
723 0.027027027027027
724 0.027027027027027
725 0.027027027027027
726 0.027027027027027
727 0.027027027027027
728 0.027027027027027
729 0.027027027027027
730 0.027027027027027
731 0.027027027027027
732 0.027027027027027
733 0.027027027027027
734 0.027027027027027
735 0.027027027027027
736 0.027027027027027
737 0.027027027027027
738 0.027027027027027
739 0.027027027027027
740 0.027027027027027
741 0.027027027027027
742 0.027027027027027
743 0.027027027027027
744 0.027027027027027
745 0.027027027027027
746 0.027027027027027
747 0.027027027027027
748 0.027027027027027
749 0.027027027027027
750 0.027027027027027
751 0.027027027027027
752 0.027027027027027
753 0.027027027027027
754 0.027027027027027
755 0.027027027027027
756 0.027027027027027
757 0.027027027027027
758 0.027027027027027
759 0.027027027027027
760 0.027027027027027
761 0.027027027027027
762 0.027027027027027
763 0.027027027027027
764 0.027027027027027
765 0.027027027027027
766 0.027027027027027
767 0.027027027027027
768 0.027027027027027
769 0.027027027027027
770 0.027027027027027
771 0.027027027027027
772 0.027027027027027
773 0.027027027027027
774 0.027027027027027
775 0.027027027027027
776 0.027027027027027
777 0.027027027027027
778 0.027027027027027
779 0.027027027027027
780 0.027027027027027
781 0.027027027027027
782 0.027027027027027
783 0.027027027027027
784 0.027027027027027
785 0.027027027027027
786 0.027027027027027
787 0.027027027027027
788 0.027027027027027
789 0.027027027027027
790 0.027027027027027
791 0.027027027027027
792 0.027027027027027
793 0.027027027027027
794 0.027027027027027
795 0.027027027027027
796 0.027027027027027
797 0.027027027027027
798 0.027027027027027
799 0.027027027027027
800 0.027027027027027
801 0.027027027027027
802 0.027027027027027
803 0.027027027027027
804 0.027027027027027
805 0.027027027027027
806 0.027027027027027
807 0.027027027027027
808 0.027027027027027
809 0.027027027027027
810 0.027027027027027
811 0.027027027027027
812 0.027027027027027
813 0.027027027027027
814 0.027027027027027
815 0.027027027027027
816 0.027027027027027
817 0.027027027027027
818 0.027027027027027
819 0.027027027027027
820 0.027027027027027
821 0.027027027027027
822 0.027027027027027
823 0.027027027027027
824 0.027027027027027
825 0.027027027027027
826 0.027027027027027
827 0.027027027027027
828 0.027027027027027
829 0.027027027027027
830 0.027027027027027
831 0.027027027027027
832 0.027027027027027
833 0.027027027027027
834 0.027027027027027
835 0.027027027027027
836 0.027027027027027
837 0.027027027027027
838 0.027027027027027
839 0.027027027027027
840 0.027027027027027
841 0.027027027027027
842 0.027027027027027
843 0.027027027027027
844 0.027027027027027
845 0.027027027027027
846 0.027027027027027
847 0.027027027027027
848 0.027027027027027
849 0.027027027027027
850 0.027027027027027
851 0.027027027027027
852 0.027027027027027
853 0.027027027027027
854 0.027027027027027
855 0.027027027027027
856 0.027027027027027
857 0.027027027027027
858 0.027027027027027
859 0.027027027027027
860 0.027027027027027
861 0.027027027027027
862 0.027027027027027
863 0.027027027027027
864 0.027027027027027
865 0.027027027027027
866 0.027027027027027
867 0.027027027027027
868 0.027027027027027
869 0.027027027027027
870 0.027027027027027
871 0.027027027027027
872 0.027027027027027
873 0.027027027027027
874 0.027027027027027
875 0.027027027027027
876 0.027027027027027
877 0.027027027027027
878 0.027027027027027
879 0.027027027027027
880 0.027027027027027
881 0.027027027027027
882 0.027027027027027
883 0.027027027027027
884 0.027027027027027
885 0.027027027027027
886 0.027027027027027
887 0.027027027027027
888 0.027027027027027
889 0.027027027027027
890 0.027027027027027
891 0.027027027027027
892 0.027027027027027
893 0.027027027027027
894 0.027027027027027
895 0.027027027027027
896 0.027027027027027
897 0.027027027027027
898 0.027027027027027
899 0.027027027027027
900 0.027027027027027
901 0.027027027027027
902 0.027027027027027
903 0.027027027027027
904 0.027027027027027
905 0.027027027027027
906 0.027027027027027
907 0.027027027027027
908 0.027027027027027
909 0.027027027027027
910 0.027027027027027
911 0.027027027027027
912 0.027027027027027
913 0.027027027027027
914 0.027027027027027
915 0.027027027027027
916 0.027027027027027
917 0.027027027027027
918 0.027027027027027
919 0.027027027027027
920 0.027027027027027
921 0.027027027027027
922 0.027027027027027
923 0.027027027027027
924 0.027027027027027
925 0.027027027027027
926 0.027027027027027
927 0.027027027027027
928 0.027027027027027
929 0.027027027027027
930 0.027027027027027
931 0.027027027027027
932 0.027027027027027
933 0.027027027027027
934 0.027027027027027
935 0.027027027027027
936 0.027027027027027
937 0.027027027027027
938 0.027027027027027
939 0.027027027027027
940 0.027027027027027
941 0.027027027027027
942 0.027027027027027
943 0.027027027027027
944 0.027027027027027
945 0.027027027027027
946 0.027027027027027
947 0.027027027027027
948 0.027027027027027
949 0.027027027027027
950 0.027027027027027
951 0.027027027027027
952 0.027027027027027
953 0.027027027027027
954 0.027027027027027
955 0.027027027027027
956 0.027027027027027
957 0.027027027027027
958 0.027027027027027
959 0.027027027027027
960 0.027027027027027
961 0.027027027027027
962 0.027027027027027
963 0.027027027027027
964 0.027027027027027
965 0.027027027027027
966 0.027027027027027
967 0.027027027027027
968 0.027027027027027
969 0.027027027027027
970 0.027027027027027
971 0.027027027027027
972 0.027027027027027
973 0.027027027027027
974 0.027027027027027
975 0.027027027027027
976 0.027027027027027
977 0.027027027027027
978 0.027027027027027
979 0.027027027027027
980 0.027027027027027
981 0.027027027027027
982 0.027027027027027
983 0.027027027027027
984 0.027027027027027
985 0.027027027027027
986 0.027027027027027
987 0.027027027027027
988 0.027027027027027
989 0.027027027027027
990 0.027027027027027
991 0.027027027027027
992 0.027027027027027
993 0.027027027027027
994 0.027027027027027
995 0.027027027027027
996 0.027027027027027
997 0.027027027027027
998 0.027027027027027
999 0.027027027027027
1000 0.027027027027027
};
\addlegendentry{$f_{r_{e}}$ (frecuencia relativa esperada de $18$)}
\end{axis}

\end{tikzpicture}

    \caption{frecuencia relativa con respecto al número de tiradas}
  \end{mytikzresize}
\end{figure}

\begin{figure}[!htbp]
  \begin{mytikzresize}{0.6\textwidth}
    \centering
    % This file was created by tikzplotlib v0.9.1.
\begin{tikzpicture}

\definecolor{color0}{rgb}{0.12156862745098,0.466666666666667,0.705882352941177}
\definecolor{color1}{rgb}{1,0.498039215686275,0.0549019607843137}

\begin{axis}[
legend cell align={left},
legend style={fill opacity=0.5, draw opacity=1, text opacity=1, draw=white!80!black},
scaled ticks=false,
tick align=outside,
tick pos=left,
width=\figW,
x grid style={white!69.0196078431373!black},
xlabel={\(\displaystyle n\) (número de tiradas)},
xmajorgrids,
xmin=-48.95, xmax=1049.95,
xtick style={color=black},
xticklabel style={/pgf/number format/.cd,fixed,precision=2},
y grid style={white!69.0196078431373!black},
ylabel={\(\displaystyle v_{p}\) (valor promedio)},
ymajorgrids,
ymin=12.825, ymax=27.675,
ytick style={color=black},
yticklabel style={/pgf/number format/.cd,fixed,precision=2}
]
\addplot [semithick, color0]
table {%
1 27
2 15
3 15.3333333333333
4 13.5
5 15.2
6 14
7 17.1428571428571
8 18.5
9 19.6666666666667
10 20
11 20.0909090909091
12 19.5833333333333
13 20.2307692307692
14 20.4285714285714
15 19.6
16 18.5625
17 19.3529411764706
18 18.3888888888889
19 18.1578947368421
20 19.05
21 19.6666666666667
22 19.8636363636364
23 20.0434782608696
24 20
25 20.48
26 20.5769230769231
27 20.5925925925926
28 20.75
29 20.5172413793103
30 21
31 21.3548387096774
32 21.15625
33 20.7878787878788
34 20.7941176470588
35 20.2
36 20.4722222222222
37 20.5135135135135
38 20.5
39 20.3076923076923
40 20.15
41 19.9268292682927
42 19.7857142857143
43 19.9767441860465
44 20.2272727272727
45 20.5333333333333
46 20.8478260869565
47 20.8510638297872
48 21.125
49 20.9387755102041
50 21.08
51 20.8627450980392
52 20.6730769230769
53 20.377358490566
54 20.0925925925926
55 19.7272727272727
56 20
57 19.719298245614
58 19.6551724137931
59 19.4915254237288
60 19.55
61 19.5573770491803
62 19.5322580645161
63 19.3968253968254
64 19.609375
65 19.8
66 19.6212121212121
67 19.6268656716418
68 19.3382352941176
69 19.1304347826087
70 18.8857142857143
71 18.6338028169014
72 18.8472222222222
73 18.986301369863
74 18.8918918918919
75 19.0133333333333
76 19.0526315789474
77 19.1168831168831
78 19.2435897435897
79 19.2911392405063
80 19.125
81 19.2469135802469
82 19.1829268292683
83 19.0722891566265
84 19.0119047619048
85 18.9882352941176
86 18.9883720930233
87 18.9540229885057
88 18.75
89 18.6292134831461
90 18.6
91 18.7912087912088
92 18.6630434782609
93 18.5161290322581
94 18.3936170212766
95 18.3789473684211
96 18.3854166666667
97 18.5257731958763
98 18.6122448979592
99 18.6868686868687
100 18.73
101 18.6138613861386
102 18.4411764705882
103 18.2912621359223
104 18.1538461538462
105 18.1619047619048
106 18.122641509434
107 18.2429906542056
108 18.1481481481481
109 18.1743119266055
110 18.3272727272727
111 18.2792792792793
112 18.3214285714286
113 18.3362831858407
114 18.3508771929825
115 18.3826086956522
116 18.4051724137931
117 18.5384615384615
118 18.4152542372881
119 18.4789915966387
120 18.6166666666667
121 18.5289256198347
122 18.6311475409836
123 18.7560975609756
124 18.8709677419355
125 18.864
126 18.8968253968254
127 18.9763779527559
128 18.9140625
129 18.8759689922481
130 18.8461538461538
131 18.793893129771
132 18.8787878787879
133 18.7744360902256
134 18.8507462686567
135 18.8444444444444
136 18.8235294117647
137 18.7883211678832
138 18.7101449275362
139 18.6906474820144
140 18.7714285714286
141 18.7943262411348
142 18.8732394366197
143 18.7622377622378
144 18.8263888888889
145 18.9310344827586
146 18.9246575342466
147 18.9251700680272
148 18.8108108108108
149 18.8724832214765
150 18.7533333333333
151 18.7218543046358
152 18.8026315789474
153 18.8954248366013
154 18.8961038961039
155 18.941935483871
156 18.9038461538462
157 18.9490445859873
158 18.8670886075949
159 18.937106918239
160 19.0125
161 18.9813664596273
162 19.0185185185185
163 18.9631901840491
164 18.890243902439
165 18.8424242424242
166 18.7349397590361
167 18.7784431137725
168 18.8690476190476
169 18.887573964497
170 18.9764705882353
171 18.9356725146199
172 18.9593023255814
173 18.9537572254335
174 18.9367816091954
175 18.9371428571429
176 18.8693181818182
177 18.9152542372881
178 18.8314606741573
179 18.8324022346369
180 18.9055555555556
181 18.9226519337017
182 18.9340659340659
183 18.8524590163934
184 18.8695652173913
185 18.8162162162162
186 18.7849462365591
187 18.7700534759358
188 18.6702127659574
189 18.6984126984127
190 18.7315789473684
191 18.6753926701571
192 18.6510416666667
193 18.6891191709845
194 18.7319587628866
195 18.6717948717949
196 18.6071428571429
197 18.6446700507614
198 18.6767676767677
199 18.7085427135678
200 18.795
201 18.7810945273632
202 18.7475247524752
203 18.6600985221675
204 18.5980392156863
205 18.5804878048781
206 18.621359223301
207 18.536231884058
208 18.5480769230769
209 18.5406698564593
210 18.5619047619048
211 18.5592417061611
212 18.622641509434
213 18.6525821596244
214 18.7336448598131
215 18.7720930232558
216 18.7222222222222
217 18.7188940092166
218 18.697247706422
219 18.7260273972603
220 18.7909090909091
221 18.8371040723982
222 18.8693693693694
223 18.8968609865471
224 18.8883928571429
225 18.9377777777778
226 19
227 18.920704845815
228 18.9473684210526
229 18.9825327510917
230 18.9695652173913
231 18.952380952381
232 18.9698275862069
233 18.9871244635193
234 18.982905982906
235 19.0340425531915
236 19.0847457627119
237 19.1561181434599
238 19.2226890756303
239 19.255230125523
240 19.2916666666667
241 19.2614107883817
242 19.2231404958678
243 19.2592592592593
244 19.2909836065574
245 19.2857142857143
246 19.2357723577236
247 19.2145748987854
248 19.2096774193548
249 19.1807228915663
250 19.136
251 19.1274900398406
252 19.1865079365079
253 19.1739130434783
254 19.0984251968504
255 19.121568627451
256 19.16796875
257 19.1634241245136
258 19.2286821705426
259 19.2779922779923
260 19.2923076923077
261 19.2835249042146
262 19.2442748091603
263 19.2813688212928
264 19.2689393939394
265 19.222641509434
266 19.2067669172932
267 19.2696629213483
268 19.2089552238806
269 19.2453531598513
270 19.2037037037037
271 19.2472324723247
272 19.2720588235294
273 19.2930402930403
274 19.2992700729927
275 19.3490909090909
276 19.2826086956522
277 19.2238267148014
278 19.2697841726619
279 19.2759856630824
280 19.3035714285714
281 19.3451957295374
282 19.3900709219858
283 19.3816254416961
284 19.4049295774648
285 19.3614035087719
286 19.3321678321678
287 19.3414634146341
288 19.3194444444444
289 19.3598615916955
290 19.3586206896552
291 19.3058419243986
292 19.3356164383562
293 19.3071672354949
294 19.3163265306122
295 19.3016949152542
296 19.3006756756757
297 19.3164983164983
298 19.2751677852349
299 19.2508361204013
300 19.2166666666667
301 19.2691029900332
302 19.3211920529801
303 19.2574257425743
304 19.2368421052632
305 19.2032786885246
306 19.1732026143791
307 19.1368078175896
308 19.1071428571429
309 19.126213592233
310 19.1258064516129
311 19.0932475884244
312 19.0641025641026
313 19.0575079872204
314 19.0477707006369
315 19.0984126984127
316 19.0696202531646
317 19.0946372239748
318 19.0503144654088
319 19.0815047021944
320 19.109375
321 19.1308411214953
322 19.0869565217391
323 19.0743034055728
324 19.0895061728395
325 19.0523076923077
326 19.0153374233129
327 18.9724770642202
328 18.984756097561
329 18.951367781155
330 18.9181818181818
331 18.8972809667674
332 18.9487951807229
333 18.987987987988
334 18.9431137724551
335 18.9880597014925
336 19.0029761904762
337 18.9881305637982
338 18.9319526627219
339 18.9675516224189
340 19.0058823529412
341 19.0058651026393
342 19.0029239766082
343 19.0116618075802
344 19.031976744186
345 19.0231884057971
346 19.0173410404624
347 19.0259365994236
348 19.0028735632184
349 18.9713467048711
350 18.9371428571429
351 18.960113960114
352 18.9545454545455
353 18.9036827195467
354 18.9406779661017
355 18.943661971831
356 18.9550561797753
357 18.9019607843137
358 18.8519553072626
359 18.8105849582173
360 18.7833333333333
361 18.7590027700831
362 18.7651933701657
363 18.7548209366391
364 18.7994505494506
365 18.7780821917808
366 18.7814207650273
367 18.7356948228883
368 18.7663043478261
369 18.7940379403794
370 18.7756756756757
371 18.7547169811321
372 18.8010752688172
373 18.8471849865952
374 18.8609625668449
375 18.8266666666667
376 18.8191489361702
377 18.8328912466844
378 18.8280423280423
379 18.7941952506596
380 18.8078947368421
381 18.8372703412074
382 18.8115183246073
383 18.822454308094
384 18.8541666666667
385 18.8779220779221
386 18.8652849740933
387 18.8191214470284
388 18.8479381443299
389 18.879177377892
390 18.9153846153846
391 18.9590792838875
392 18.9744897959184
393 18.9312977099237
394 18.9670050761421
395 19.0025316455696
396 18.9722222222222
397 19.0025188916877
398 18.9547738693467
399 18.9774436090226
400 18.9425
401 18.9201995012469
402 18.9353233830846
403 18.9677419354839
404 18.9975247524752
405 18.9703703703704
406 18.9926108374384
407 18.985257985258
408 19.0196078431373
409 19.0562347188264
410 19.0463414634146
411 19.0316301703163
412 19.0145631067961
413 19.0121065375303
414 19.0096618357488
415 19.0409638554217
416 19.0024038461538
417 19.0359712230216
418 19.0574162679426
419 19.0525059665871
420 19.0285714285714
421 19.0403800475059
422 19.0284360189573
423 19.0118203309693
424 19.0235849056604
425 19.0282352941176
426 18.9882629107981
427 18.9484777517564
428 18.9532710280374
429 18.958041958042
430 18.9581395348837
431 18.9350348027842
432 18.9583333333333
433 18.9838337182448
434 19.0161290322581
435 18.9931034482759
436 18.9839449541284
437 18.9908466819222
438 18.9908675799087
439 18.9658314350797
440 18.9454545454545
441 18.9229024943311
442 18.893665158371
443 18.8555304740406
444 18.8918918918919
445 18.9078651685393
446 18.8811659192825
447 18.8769574944072
448 18.8816964285714
449 18.8507795100223
450 18.8511111111111
451 18.8337028824834
452 18.8274336283186
453 18.8520971302428
454 18.8259911894273
455 18.8395604395604
456 18.8552631578947
457 18.8577680525164
458 18.8471615720524
459 18.8627450980392
460 18.8760869565217
461 18.8633405639913
462 18.8441558441558
463 18.8250539956803
464 18.8103448275862
465 18.8301075268817
466 18.7896995708155
467 18.7880085653105
468 18.8183760683761
469 18.8336886993603
470 18.8510638297872
471 18.8577494692144
472 18.8940677966102
473 18.8541226215645
474 18.8481012658228
475 18.8778947368421
476 18.8529411764706
477 18.832285115304
478 18.8640167364017
479 18.866388308977
480 18.8666666666667
481 18.8274428274428
482 18.8070539419087
483 18.7722567287785
484 18.7334710743802
485 18.7463917525773
486 18.7716049382716
487 18.7946611909651
488 18.7643442622951
489 18.7709611451943
490 18.7755102040816
491 18.8085539714868
492 18.8170731707317
493 18.7789046653144
494 18.7773279352227
495 18.7494949494949
496 18.7762096774194
497 18.7867203219316
498 18.7971887550201
499 18.8216432865731
500 18.808
501 18.8183632734531
502 18.8227091633466
503 18.8011928429423
504 18.8313492063492
505 18.819801980198
506 18.798418972332
507 18.8244575936884
508 18.7972440944882
509 18.8055009823183
510 18.8
511 18.8160469667319
512 18.810546875
513 18.7933723196881
514 18.8054474708171
515 18.7941747572816
516 18.8042635658915
517 18.816247582205
518 18.8166023166023
519 18.8092485549133
520 18.7942307692308
521 18.7811900191939
522 18.7567049808429
523 18.7284894837476
524 18.7099236641221
525 18.7104761904762
526 18.6825095057034
527 18.6660341555977
528 18.6780303030303
529 18.6729678638941
530 18.6547169811321
531 18.6723163841808
532 18.6484962406015
533 18.6322701688555
534 18.6554307116105
535 18.6654205607477
536 18.6716417910448
537 18.659217877095
538 18.6710037174721
539 18.69573283859
540 18.6888888888889
541 18.6783733826248
542 18.6771217712177
543 18.707182320442
544 18.6893382352941
545 18.7155963302752
546 18.7289377289377
547 18.7312614259598
548 18.7116788321168
549 18.6848816029144
550 18.6763636363636
551 18.6769509981851
552 18.6992753623188
553 18.7052441229656
554 18.7346570397112
555 18.7009009009009
556 18.7194244604317
557 18.7253141831239
558 18.7186379928315
559 18.7155635062612
560 18.6964285714286
561 18.7112299465241
562 18.7366548042705
563 18.7211367673179
564 18.7517730496454
565 18.7646017699115
566 18.7879858657244
567 18.7689594356261
568 18.75
569 18.7381370826011
570 18.7105263157895
571 18.6952714535902
572 18.6853146853147
573 18.6876090750436
574 18.7160278745645
575 18.6852173913043
576 18.6614583333333
577 18.6516464471404
578 18.6557093425606
579 18.6407599309154
580 18.6551724137931
581 18.6437177280551
582 18.6477663230241
583 18.6518010291595
584 18.652397260274
585 18.6529914529915
586 18.6296928327645
587 18.6371379897785
588 18.6462585034014
589 18.6621392190153
590 18.664406779661
591 18.6565143824027
592 18.6266891891892
593 18.6273187183811
594 18.6565656565657
595 18.6588235294118
596 18.6677852348993
597 18.6616415410385
598 18.6705685618729
599 18.6761268781302
600 18.6766666666667
601 18.6672212978369
602 18.6926910299003
603 18.6616915422886
604 18.6837748344371
605 18.6611570247934
606 18.6551155115512
607 18.67215815486
608 18.6759868421053
609 18.6469622331691
610 18.6393442622951
611 18.6301145662848
612 18.6584967320261
613 18.6362153344209
614 18.6205211726384
615 18.6308943089431
616 18.6168831168831
617 18.5915721231767
618 18.6084142394822
619 18.6058158319871
620 18.5854838709677
621 18.5571658615137
622 18.548231511254
623 18.5762439807384
624 18.5897435897436
625 18.5904
626 18.5607028753994
627 18.5358851674641
628 18.5063694267516
629 18.5007949125596
630 18.5047619047619
631 18.5118858954041
632 18.5332278481013
633 18.521327014218
634 18.5441640378549
635 18.555905511811
636 18.5440251572327
637 18.5510204081633
638 18.564263322884
639 18.5492957746479
640 18.546875
641 18.5694227769111
642 18.5451713395639
643 18.5629860031104
644 18.5341614906832
645 18.5612403100775
646 18.5650154798762
647 18.58114374034
648 18.5570987654321
649 18.5469953775039
650 18.5723076923077
651 18.594470046083
652 18.5828220858896
653 18.5681470137825
654 18.5948012232416
655 18.5664122137405
656 18.5762195121951
657 18.5905631659056
658 18.5714285714286
659 18.5751138088012
660 18.5984848484848
661 18.5703479576399
662 18.5891238670695
663 18.6003016591252
664 18.5843373493976
665 18.5834586466165
666 18.5765765765766
667 18.5712143928036
668 18.5973053892216
669 18.6083707025411
670 18.5850746268657
671 18.5767511177347
672 18.59375
673 18.6166419019317
674 18.6172106824926
675 18.6237037037037
676 18.6198224852071
677 18.5982274741507
678 18.6017699115044
679 18.5861561119293
680 18.5926470588235
681 18.5932452276065
682 18.5938416422287
683 18.5783308931186
684 18.5760233918129
685 18.6
686 18.5889212827988
687 18.5618631732169
688 18.5436046511628
689 18.5312046444122
690 18.536231884058
691 18.5282199710564
692 18.5346820809249
693 18.5324675324675
694 18.5518731988473
695 18.568345323741
696 18.5603448275862
697 18.545193687231
698 18.5186246418338
699 18.5264663805436
700 18.5014285714286
701 18.5135520684736
702 18.5014245014245
703 18.4850640113798
704 18.4815340909091
705 18.4609929078014
706 18.4801699716714
707 18.4752475247525
708 18.4703389830508
709 18.4555712270804
710 18.4633802816901
711 18.4852320675105
712 18.4620786516854
713 18.4670406732118
714 18.4887955182073
715 18.4629370629371
716 18.4664804469274
717 18.4476987447699
718 18.4651810584958
719 18.4436717663421
720 18.4513888888889
721 18.4438280166436
722 18.4681440443213
723 18.4813278008299
724 18.4779005524862
725 18.4689655172414
726 18.4559228650138
727 18.4580467675378
728 18.4546703296703
729 18.4581618655693
730 18.4438356164384
731 18.4473324213406
732 18.422131147541
733 18.4256480218281
734 18.4237057220708
735 18.4176870748299
736 18.3967391304348
737 18.3880597014925
738 18.3658536585366
739 18.3788903924222
740 18.372972972973
741 18.3954116059379
742 18.4056603773585
743 18.4051144010767
744 18.380376344086
745 18.3557046979866
746 18.3445040214477
747 18.3493975903614
748 18.3435828877005
749 18.324432576769
750 18.3173333333333
751 18.3222370173103
752 18.3231382978723
753 18.3333333333333
754 18.3381962864721
755 18.3390728476821
756 18.3505291005291
757 18.3500660501982
758 18.368073878628
759 18.3557312252964
760 18.3671052631579
761 18.3626806833114
762 18.3503937007874
763 18.3355176933159
764 18.3494764397906
765 18.3267973856209
766 18.3446475195822
767 18.3337679269883
768 18.3502604166667
769 18.3524057217165
770 18.361038961039
771 18.348897535668
772 18.3588082901554
773 18.3428201811125
774 18.3656330749354
775 18.341935483871
776 18.3260309278351
777 18.3397683397683
778 18.3213367609255
779 18.3119383825417
780 18.3282051282051
781 18.3469910371319
782 18.343989769821
783 18.3409961685824
784 18.3214285714286
785 18.2993630573248
786 18.3078880407125
787 18.3138500635324
788 18.3223350253807
789 18.3384030418251
790 18.3405063291139
791 18.3299620733249
792 18.3434343434343
793 18.3644388398487
794 18.3413098236776
795 18.3610062893082
796 18.3542713567839
797 18.3387703889586
798 18.3358395989975
799 18.3279098873592
800 18.33875
801 18.3245942571785
802 18.3104738154613
803 18.3250311332503
804 18.3320895522388
805 18.3130434782609
806 18.3002481389578
807 18.3110285006196
808 18.3168316831683
809 18.3028430160692
810 18.3074074074074
811 18.3033292231813
812 18.3029556650246
813 18.3075030750308
814 18.3194103194103
815 18.3042944785276
816 18.2977941176471
817 18.2778457772338
818 18.2958435207824
819 18.2771672771673
820 18.2926829268293
821 18.2935444579781
822 18.2858880778589
823 18.2843256379101
824 18.2900485436893
825 18.2787878787879
826 18.2808716707022
827 18.26723095526
828 18.2512077294686
829 18.2496984318456
830 18.2385542168675
831 18.2250300842359
832 18.2451923076923
833 18.234093637455
834 18.2398081534772
835 18.2526946107784
836 18.2679425837321
837 18.2652329749104
838 18.2828162291169
839 18.2872467222884
840 18.2702380952381
841 18.2841854934602
842 18.2684085510689
843 18.2728351126928
844 18.2618483412322
845 18.2473372781065
846 18.2612293144208
847 18.2597402597403
848 18.2452830188679
849 18.2449941107185
850 18.2235294117647
851 18.2079905992949
852 18.2288732394366
853 18.2203985932005
854 18.2377049180328
855 18.2187134502924
856 18.2266355140187
857 18.2462077012835
858 18.2412587412587
859 18.256111757858
860 18.2755813953488
861 18.2543554006969
862 18.2517401392111
863 18.252607184241
864 18.2592592592593
865 18.249710982659
866 18.2621247113164
867 18.2733564013841
868 18.2546082949309
869 18.2359033371692
870 18.2333333333333
871 18.2342135476464
872 18.2373853211009
873 18.2531500572738
874 18.254004576659
875 18.2468571428571
876 18.2602739726027
877 18.2611174458381
878 18.255125284738
879 18.2377701934016
880 18.2454545454545
881 18.2270147559591
882 18.2448979591837
883 18.2570781426954
884 18.2647058823529
885 18.2790960451977
886 18.2742663656885
887 18.293122886133
888 18.3108108108108
889 18.3127109111361
890 18.2921348314607
891 18.300785634119
892 18.3105381165919
893 18.2978723404255
894 18.2785234899329
895 18.2983240223464
896 18.2790178571429
897 18.2720178372352
898 18.2873051224944
899 18.2981090100111
900 18.3022222222222
901 18.291897891232
902 18.2882483370288
903 18.2868217054264
904 18.2853982300885
905 18.2762430939227
906 18.257174392936
907 18.2535832414553
908 18.2533039647577
909 18.2662266226623
910 18.2681318681319
911 18.2788144895719
912 18.2872807017544
913 18.2771084337349
914 18.2636761487965
915 18.2579234972678
916 18.2565502183406
917 18.2562704471101
918 18.2472766884532
919 18.2491838955386
920 18.2630434782609
921 18.243213897937
922 18.2494577006508
923 18.2567713976165
924 18.2683982683983
925 18.2681081081081
926 18.2742980561555
927 18.2664509169364
928 18.2693965517241
929 18.2820236813778
930 18.2881720430108
931 18.2867883995704
932 18.2789699570815
933 18.2818863879957
934 18.2655246252677
935 18.2823529411765
936 18.2905982905983
937 18.2977588046958
938 18.2878464818763
939 18.2896698615548
940 18.3053191489362
941 18.3092454835282
942 18.296178343949
943 18.304347826087
944 18.3125
945 18.2941798941799
946 18.3012684989429
947 18.29883843717
948 18.2816455696203
949 18.2971548998946
950 18.2926315789474
951 18.2754994742376
952 18.281512605042
953 18.2654774396642
954 18.2830188679245
955 18.2764397905759
956 18.260460251046
957 18.2424242424242
958 18.2546972860125
959 18.2627737226277
960 18.2614583333333
961 18.2653485952133
962 18.2525987525988
963 18.2336448598131
964 18.2199170124481
965 18.2093264248705
966 18.223602484472
967 18.2388831437435
968 18.2210743801653
969 18.2260061919505
970 18.2432989690722
971 18.2605561277034
972 18.2654320987654
973 18.2816032887975
974 18.2936344969199
975 18.2974358974359
976 18.3104508196721
977 18.3142272262027
978 18.3118609406953
979 18.2972420837589
980 18.2795918367347
981 18.2691131498471
982 18.2596741344196
983 18.2543234994914
984 18.2713414634146
985 18.2629441624365
986 18.2758620689655
987 18.2786220871327
988 18.2793522267206
989 18.2709807886754
990 18.2818181818182
991 18.2724520686176
992 18.2883064516129
993 18.2900302114804
994 18.2756539235412
995 18.2824120603015
996 18.289156626506
997 18.2938816449348
998 18.2775551102204
999 18.2612612612613
1000 18.272
};
\addlegendentry{$v_{p}$ (valor promedio de las tiradas)}
\addplot [semithick, color1]
table {%
1 18
2 18
3 18
4 18
5 18
6 18
7 18
8 18
9 18
10 18
11 18
12 18
13 18
14 18
15 18
16 18
17 18
18 18
19 18
20 18
21 18
22 18
23 18
24 18
25 18
26 18
27 18
28 18
29 18
30 18
31 18
32 18
33 18
34 18
35 18
36 18
37 18
38 18
39 18
40 18
41 18
42 18
43 18
44 18
45 18
46 18
47 18
48 18
49 18
50 18
51 18
52 18
53 18
54 18
55 18
56 18
57 18
58 18
59 18
60 18
61 18
62 18
63 18
64 18
65 18
66 18
67 18
68 18
69 18
70 18
71 18
72 18
73 18
74 18
75 18
76 18
77 18
78 18
79 18
80 18
81 18
82 18
83 18
84 18
85 18
86 18
87 18
88 18
89 18
90 18
91 18
92 18
93 18
94 18
95 18
96 18
97 18
98 18
99 18
100 18
101 18
102 18
103 18
104 18
105 18
106 18
107 18
108 18
109 18
110 18
111 18
112 18
113 18
114 18
115 18
116 18
117 18
118 18
119 18
120 18
121 18
122 18
123 18
124 18
125 18
126 18
127 18
128 18
129 18
130 18
131 18
132 18
133 18
134 18
135 18
136 18
137 18
138 18
139 18
140 18
141 18
142 18
143 18
144 18
145 18
146 18
147 18
148 18
149 18
150 18
151 18
152 18
153 18
154 18
155 18
156 18
157 18
158 18
159 18
160 18
161 18
162 18
163 18
164 18
165 18
166 18
167 18
168 18
169 18
170 18
171 18
172 18
173 18
174 18
175 18
176 18
177 18
178 18
179 18
180 18
181 18
182 18
183 18
184 18
185 18
186 18
187 18
188 18
189 18
190 18
191 18
192 18
193 18
194 18
195 18
196 18
197 18
198 18
199 18
200 18
201 18
202 18
203 18
204 18
205 18
206 18
207 18
208 18
209 18
210 18
211 18
212 18
213 18
214 18
215 18
216 18
217 18
218 18
219 18
220 18
221 18
222 18
223 18
224 18
225 18
226 18
227 18
228 18
229 18
230 18
231 18
232 18
233 18
234 18
235 18
236 18
237 18
238 18
239 18
240 18
241 18
242 18
243 18
244 18
245 18
246 18
247 18
248 18
249 18
250 18
251 18
252 18
253 18
254 18
255 18
256 18
257 18
258 18
259 18
260 18
261 18
262 18
263 18
264 18
265 18
266 18
267 18
268 18
269 18
270 18
271 18
272 18
273 18
274 18
275 18
276 18
277 18
278 18
279 18
280 18
281 18
282 18
283 18
284 18
285 18
286 18
287 18
288 18
289 18
290 18
291 18
292 18
293 18
294 18
295 18
296 18
297 18
298 18
299 18
300 18
301 18
302 18
303 18
304 18
305 18
306 18
307 18
308 18
309 18
310 18
311 18
312 18
313 18
314 18
315 18
316 18
317 18
318 18
319 18
320 18
321 18
322 18
323 18
324 18
325 18
326 18
327 18
328 18
329 18
330 18
331 18
332 18
333 18
334 18
335 18
336 18
337 18
338 18
339 18
340 18
341 18
342 18
343 18
344 18
345 18
346 18
347 18
348 18
349 18
350 18
351 18
352 18
353 18
354 18
355 18
356 18
357 18
358 18
359 18
360 18
361 18
362 18
363 18
364 18
365 18
366 18
367 18
368 18
369 18
370 18
371 18
372 18
373 18
374 18
375 18
376 18
377 18
378 18
379 18
380 18
381 18
382 18
383 18
384 18
385 18
386 18
387 18
388 18
389 18
390 18
391 18
392 18
393 18
394 18
395 18
396 18
397 18
398 18
399 18
400 18
401 18
402 18
403 18
404 18
405 18
406 18
407 18
408 18
409 18
410 18
411 18
412 18
413 18
414 18
415 18
416 18
417 18
418 18
419 18
420 18
421 18
422 18
423 18
424 18
425 18
426 18
427 18
428 18
429 18
430 18
431 18
432 18
433 18
434 18
435 18
436 18
437 18
438 18
439 18
440 18
441 18
442 18
443 18
444 18
445 18
446 18
447 18
448 18
449 18
450 18
451 18
452 18
453 18
454 18
455 18
456 18
457 18
458 18
459 18
460 18
461 18
462 18
463 18
464 18
465 18
466 18
467 18
468 18
469 18
470 18
471 18
472 18
473 18
474 18
475 18
476 18
477 18
478 18
479 18
480 18
481 18
482 18
483 18
484 18
485 18
486 18
487 18
488 18
489 18
490 18
491 18
492 18
493 18
494 18
495 18
496 18
497 18
498 18
499 18
500 18
501 18
502 18
503 18
504 18
505 18
506 18
507 18
508 18
509 18
510 18
511 18
512 18
513 18
514 18
515 18
516 18
517 18
518 18
519 18
520 18
521 18
522 18
523 18
524 18
525 18
526 18
527 18
528 18
529 18
530 18
531 18
532 18
533 18
534 18
535 18
536 18
537 18
538 18
539 18
540 18
541 18
542 18
543 18
544 18
545 18
546 18
547 18
548 18
549 18
550 18
551 18
552 18
553 18
554 18
555 18
556 18
557 18
558 18
559 18
560 18
561 18
562 18
563 18
564 18
565 18
566 18
567 18
568 18
569 18
570 18
571 18
572 18
573 18
574 18
575 18
576 18
577 18
578 18
579 18
580 18
581 18
582 18
583 18
584 18
585 18
586 18
587 18
588 18
589 18
590 18
591 18
592 18
593 18
594 18
595 18
596 18
597 18
598 18
599 18
600 18
601 18
602 18
603 18
604 18
605 18
606 18
607 18
608 18
609 18
610 18
611 18
612 18
613 18
614 18
615 18
616 18
617 18
618 18
619 18
620 18
621 18
622 18
623 18
624 18
625 18
626 18
627 18
628 18
629 18
630 18
631 18
632 18
633 18
634 18
635 18
636 18
637 18
638 18
639 18
640 18
641 18
642 18
643 18
644 18
645 18
646 18
647 18
648 18
649 18
650 18
651 18
652 18
653 18
654 18
655 18
656 18
657 18
658 18
659 18
660 18
661 18
662 18
663 18
664 18
665 18
666 18
667 18
668 18
669 18
670 18
671 18
672 18
673 18
674 18
675 18
676 18
677 18
678 18
679 18
680 18
681 18
682 18
683 18
684 18
685 18
686 18
687 18
688 18
689 18
690 18
691 18
692 18
693 18
694 18
695 18
696 18
697 18
698 18
699 18
700 18
701 18
702 18
703 18
704 18
705 18
706 18
707 18
708 18
709 18
710 18
711 18
712 18
713 18
714 18
715 18
716 18
717 18
718 18
719 18
720 18
721 18
722 18
723 18
724 18
725 18
726 18
727 18
728 18
729 18
730 18
731 18
732 18
733 18
734 18
735 18
736 18
737 18
738 18
739 18
740 18
741 18
742 18
743 18
744 18
745 18
746 18
747 18
748 18
749 18
750 18
751 18
752 18
753 18
754 18
755 18
756 18
757 18
758 18
759 18
760 18
761 18
762 18
763 18
764 18
765 18
766 18
767 18
768 18
769 18
770 18
771 18
772 18
773 18
774 18
775 18
776 18
777 18
778 18
779 18
780 18
781 18
782 18
783 18
784 18
785 18
786 18
787 18
788 18
789 18
790 18
791 18
792 18
793 18
794 18
795 18
796 18
797 18
798 18
799 18
800 18
801 18
802 18
803 18
804 18
805 18
806 18
807 18
808 18
809 18
810 18
811 18
812 18
813 18
814 18
815 18
816 18
817 18
818 18
819 18
820 18
821 18
822 18
823 18
824 18
825 18
826 18
827 18
828 18
829 18
830 18
831 18
832 18
833 18
834 18
835 18
836 18
837 18
838 18
839 18
840 18
841 18
842 18
843 18
844 18
845 18
846 18
847 18
848 18
849 18
850 18
851 18
852 18
853 18
854 18
855 18
856 18
857 18
858 18
859 18
860 18
861 18
862 18
863 18
864 18
865 18
866 18
867 18
868 18
869 18
870 18
871 18
872 18
873 18
874 18
875 18
876 18
877 18
878 18
879 18
880 18
881 18
882 18
883 18
884 18
885 18
886 18
887 18
888 18
889 18
890 18
891 18
892 18
893 18
894 18
895 18
896 18
897 18
898 18
899 18
900 18
901 18
902 18
903 18
904 18
905 18
906 18
907 18
908 18
909 18
910 18
911 18
912 18
913 18
914 18
915 18
916 18
917 18
918 18
919 18
920 18
921 18
922 18
923 18
924 18
925 18
926 18
927 18
928 18
929 18
930 18
931 18
932 18
933 18
934 18
935 18
936 18
937 18
938 18
939 18
940 18
941 18
942 18
943 18
944 18
945 18
946 18
947 18
948 18
949 18
950 18
951 18
952 18
953 18
954 18
955 18
956 18
957 18
958 18
959 18
960 18
961 18
962 18
963 18
964 18
965 18
966 18
967 18
968 18
969 18
970 18
971 18
972 18
973 18
974 18
975 18
976 18
977 18
978 18
979 18
980 18
981 18
982 18
983 18
984 18
985 18
986 18
987 18
988 18
989 18
990 18
991 18
992 18
993 18
994 18
995 18
996 18
997 18
998 18
999 18
1000 18
};
\addlegendentry{$v_{p_{e}}$ (valor promedio esperado)}
\end{axis}

\end{tikzpicture}

    \caption{valor promedio con respecto al número de tiradas}
  \end{mytikzresize}
\end{figure}

\begin{figure}[!htbp]
  \begin{mytikzresize}{0.6\textwidth}
    \centering
    % This file was created by tikzplotlib v0.9.1.
\begin{tikzpicture}

\definecolor{color0}{rgb}{0.12156862745098,0.466666666666667,0.705882352941177}
\definecolor{color1}{rgb}{1,0.498039215686275,0.0549019607843137}

\begin{axis}[
legend cell align={left},
legend style={fill opacity=0.5, draw opacity=1, text opacity=1, at={(0.97,0.03)}, anchor=south east, draw=white!80!black},
scaled ticks=false,
tick align=outside,
tick pos=left,
width=\figW,
x grid style={white!69.0196078431373!black},
xlabel={\(\displaystyle n\) (número de tiradas)},
xmajorgrids,
xmin=-48.95, xmax=1049.95,
xtick style={color=black},
xticklabel style={/pgf/number format/.cd,fixed,precision=2},
y grid style={white!69.0196078431373!black},
ylabel={\(\displaystyle v_{d}\) (valor del desvío)},
ymajorgrids,
ymin=-0.581377674149945, ymax=12.2089311571489,
ytick style={color=black},
yticklabel style={/pgf/number format/.cd,fixed,precision=2}
]
\addplot [semithick, color0]
table {%
1 0
2 11
3 9.0921211313239
4 9.4339811320566
5 11.6275534829989
6 11.3785861257989
7 10.5540397527222
8 10.2583807201722
9 9.94553066693344
10 9.94183081730925
11 9.48770409138164
12 9.0905934288631
13 9.5867265238864
14 9.28681312179715
15 9.21496367630147
16 8.99978298349466
17 9.36865125285267
18 9.31479640672653
19 9.90732681674223
20 10.3469802357983
21 10.1084818788823
22 10.133917356666
23 10.4727569885584
24 10.2610881759955
25 10.6572792024982
26 10.7155254284193
27 10.5174751699549
28 10.3560960062079
29 10.490128987916
30 10.5953868368372
31 10.4614648906685
32 10.7084954592137
33 10.5528535520015
34 10.4267505977501
35 10.3993720375564
36 10.2968291895139
37 10.1644620744157
38 10.2641025206307
39 10.1336239966618
40 10.4264087777144
41 10.3004247846196
42 10.4398721115354
43 10.32259541992
44 10.3365625230726
45 10.2567051239665
46 10.3676281118396
47 10.3866404425644
48 10.3745607336729
49 10.6387815782348
50 10.5578217450381
51 10.5309999897595
52 10.4617505634607
53 10.3665810244261
54 10.3983748862189
55 10.3040391662928
56 10.4615427901921
57 10.4386112339569
58 10.4002681068524
59 10.3182899335654
60 10.293727863769
61 10.2093380381181
62 10.214975947638
63 10.1895366228579
64 10.3695392255394
65 10.3031448062913
66 10.2303250215188
67 10.1739164789369
68 10.2426444195765
69 10.1688537732402
70 10.1961277163521
71 10.1561308894638
72 10.0864224683421
73 10.0181295595733
74 9.97257012297032
75 9.9411602273913
76 9.89067665857816
77 9.96688732630236
78 9.91546285766773
79 9.96569986756296
80 10.0466956632517
81 10.0238551010439
82 10.0632443857299
83 10.0383934187678
84 9.9794786602841
85 9.93355085183297
86 9.98340971026143
87 9.92941131903373
88 9.89714015584617
89 9.99449412875843
90 9.98700390074923
91 9.93229100160252
92 10.0122030929785
93 10.1218726837006
94 10.1241277589297
95 10.161444702834
96 10.1606400768275
97 10.2061334998151
98 10.2485409004041
99 10.28322517144
100 10.2365765761801
101 10.2565655818864
102 10.2181886965533
103 10.2478108171209
104 10.3009270278522
105 10.264616810999
106 10.2264499059492
107 10.2584813624505
108 10.327503167442
109 10.2959734902506
110 10.259847114354
111 10.2208388416856
112 10.203275159457
113 10.2198252096675
114 10.2853860353999
115 10.3475191770343
116 10.3047499523982
117 10.3915315859539
118 10.4188806418616
119 10.4309851688324
120 10.457320668104
121 10.4402312746918
122 10.4371527479743
123 10.4601513537226
124 10.4649954148226
125 10.4367473860394
126 10.4585860984954
127 10.4301496430321
128 10.4036425740504
129 10.3806467550757
130 10.3435341264411
131 10.3042012372307
132 10.279295299082
133 10.3556082811669
134 10.3170636415171
135 10.2827248687798
136 10.3558965534427
137 10.4140351694133
138 10.4824294632597
139 10.5479059515496
140 10.5531212135055
141 10.5247318915622
142 10.5855167117506
143 10.6122423645667
144 10.5969855805819
145 10.5724295386436
146 10.5755398696886
147 10.539841093019
148 10.5198753592841
149 10.4960821353438
150 10.5630151419417
151 10.6064104748208
152 10.6072464516159
153 10.6116822614816
154 10.5815090424021
155 10.5573504385825
156 10.541430347937
157 10.5163884360246
158 10.5007992157361
159 10.469380444975
160 10.4703730085179
161 10.4832172279028
162 10.4907130455578
163 10.4646464859307
164 10.5108015034579
165 10.54925044883
166 10.5383418183254
167 10.5747858753912
168 10.6110636225505
169 10.5829820854561
170 10.5881111103829
171 10.562407262567
172 10.6166802440821
173 10.6010678194271
174 10.6522829595335
175 10.6276558745452
176 10.6031861635162
177 10.5839824126638
178 10.5881827376156
179 10.6318969871101
180 10.6251666653595
181 10.6645777870808
182 10.6549951491224
183 10.6262587891863
184 10.6266775613283
185 10.626776824747
186 10.6119172445654
187 10.597049753004
188 10.6199242929187
189 10.594381657583
190 10.5747621991167
191 10.5936738586952
192 10.6161676933372
193 10.6051405402924
194 10.60294618726
195 10.5757292255524
196 10.5967361166791
197 10.5786095655153
198 10.5709580245009
199 10.5848853075895
200 10.6346179997215
201 10.6609892442516
202 10.6429161672303
203 10.6398444484932
204 10.6654816599548
205 10.7134945233497
206 10.7253576723804
207 10.7560535642697
208 10.7562421673245
209 10.7327943994553
210 10.7079145078104
211 10.7404637516331
212 10.770018698999
213 10.7457804706635
214 10.721704733744
215 10.7268258016685
216 10.7342563040911
217 10.7726579976664
218 10.7655742339561
219 10.7542380132348
220 10.7299341844816
221 10.7063757569621
222 10.6921462256101
223 10.66842626769
224 10.658451991356
225 10.6452692791066
226 10.6316044230709
227 10.6090856571369
228 10.5857968574022
229 10.5633973038018
230 10.5812331641497
231 10.6238529090196
232 10.6216832271483
233 10.6186606615371
234 10.5959502516622
235 10.6176054075442
236 10.6043128364146
237 10.582037897924
238 10.5701158822141
239 10.5855623707479
240 10.5841043026271
241 10.613387639032
242 10.6286839678194
243 10.6074187747242
244 10.614342671943
245 10.6165109164915
246 10.5981378931786
247 10.6036508276068
248 10.6006760811135
249 10.6287521114015
250 10.6076199026926
251 10.5929459828305
252 10.5763155144609
253 10.5560225208593
254 10.5447859335797
255 10.5421101360554
256 10.5567546319584
257 10.5423087909899
258 10.5450813681656
259 10.5545369333294
260 10.5413499691262
261 10.533440197735
262 10.5321410448511
263 10.548201502184
264 10.567841718216
265 10.5480061773902
266 10.5349373929657
267 10.5178195615039
268 10.5130390085746
269 10.493482859158
270 10.5183026842749
271 10.5275750386801
272 10.5248074676848
273 10.531640695003
274 10.547354329919
275 10.5564231357732
276 10.571872096565
277 10.5567468233121
278 10.5656226444206
279 10.571980940311
280 10.5643889447289
281 10.587671271034
282 10.5697037989941
283 10.5518272385826
284 10.5580003905839
285 10.5940138849213
286 10.5779887343873
287 10.591223809364
288 10.572936621948
289 10.5905429785019
290 10.5777125305106
291 10.5597678642341
292 10.5419141938066
293 10.5348294858645
294 10.5475685919727
295 10.5518329129334
296 10.5734062937996
297 10.5945274260846
298 10.5884581148633
299 10.5912651450387
300 10.5785627043039
301 10.5612815189045
302 10.5686302496243
303 10.5792695669268
304 10.5629128408559
305 10.5456069202192
306 10.5283857818972
307 10.51966790704
308 10.5115390097189
309 10.4974859852617
310 10.4888965622039
311 10.4764122412681
312 10.4600300571673
313 10.4522357369075
314 10.4598113488964
315 10.460170531521
316 10.4480823068352
317 10.4324776098203
318 10.4616228184527
319 10.4955421820742
320 10.4811568272543
321 10.4911064527496
322 10.5240547768264
323 10.508216090837
324 10.5032691726681
325 10.5056013236558
326 10.5304474410691
327 10.5267845898686
328 10.514697158526
329 10.5154452206015
330 10.5104888797903
331 10.529980807051
332 10.52654048143
333 10.5153425186315
334 10.5094277162946
335 10.5044670085156
336 10.4888384148349
337 10.5205697050571
338 10.5069514870292
339 10.5125525743062
340 10.5434608869217
341 10.5351627596225
342 10.5208611598327
343 10.5319221796
344 10.5168066933818
345 10.5175557619653
346 10.5455152558371
347 10.5758773021004
348 10.5608804305492
349 10.5467723471435
350 10.5766303804228
351 10.5815333319636
352 10.5669461032196
353 10.5539368057996
354 10.5588126704084
355 10.5817215417692
356 10.572996393921
357 10.5887273313564
358 10.5956035408115
359 10.6135495690669
360 10.6139027014362
361 10.606086427053
362 10.5973421840407
363 10.5829457250629
364 10.5742678984968
365 10.6002779344451
366 10.6032788152519
367 10.5983776761712
368 10.5929810552578
369 10.5989766198664
370 10.5855021656671
371 10.576450532239
372 10.5691605840529
373 10.5552222948478
374 10.5425379816653
375 10.545567915796
376 10.5388218137427
377 10.5272817097302
378 10.5157771783593
379 10.5364093332583
380 10.5489687172363
381 10.540410532826
382 10.5335568289001
383 10.5456754218046
384 10.5699294707641
385 10.5586520961102
386 10.5708193281167
387 10.5596217723826
388 10.5832045258477
389 10.5698875888911
390 10.5563517122496
391 10.5436345196475
392 10.5316998330957
393 10.5234525865593
394 10.5352969779628
395 10.5255021029929
396 10.51572974812
397 10.5032787756697
398 10.5006661477127
399 10.5157077470527
400 10.512941726748
401 10.5241304230296
402 10.519154614164
403 10.5162268119025
404 10.5381339980086
405 10.5521516423374
406 10.5392281225112
407 10.5383186950561
408 10.5351154814924
409 10.5392604023136
410 10.5269083837078
411 10.5363951467059
412 10.5490192302749
413 10.5415976340496
414 10.5319275341478
415 10.5490513848975
416 10.5421445259312
417 10.5296365904223
418 10.5395928120299
419 10.5326660071403
420 10.5219778625784
421 10.5457458490938
422 10.5549333063758
423 10.5539064045192
424 10.5484175435726
425 10.5416428468258
426 10.5648306255258
427 10.5525970572451
428 10.5585316817419
429 10.5715630462322
430 10.591472073903
431 10.5802260130478
432 10.5734616540399
433 10.566479869339
434 10.5597690715269
435 10.5751739029836
436 10.5648672554164
437 10.5544187586061
438 10.542843307444
439 10.5359608244154
440 10.5583727693802
441 10.5492495247506
442 10.5423377735325
443 10.5539476445126
444 10.5531309555438
445 10.5477793653072
446 10.565901735051
447 10.5612343211018
448 10.5604393594341
449 10.5595990180953
450 10.5580217002155
451 10.5468115955338
452 10.5523382873895
453 10.5775086533584
454 10.5985641477616
455 10.5874544401741
456 10.576955812873
457 10.5661425852921
458 10.5655118844208
459 10.5763428735434
460 10.5671138036782
461 10.5617167617257
462 10.5787166289862
463 10.5767640606551
464 10.5665217684438
465 10.5768867352164
466 10.5968674613129
467 10.612758435901
468 10.6325016059746
469 10.6515890401416
470 10.6586749524915
471 10.6741905186205
472 10.6629129080594
473 10.6559477521645
474 10.6657179667439
475 10.6615140833034
476 10.6658552818534
477 10.663612223038
478 10.654413056005
479 10.6474653038472
480 10.6363907487325
481 10.625355152682
482 10.6251189888382
483 10.631837482888
484 10.6208701299273
485 10.6330990400771
486 10.651919174796
487 10.6497088261981
488 10.6390064105594
489 10.6542032176297
490 10.646175935869
491 10.6381637394776
492 10.6312991887326
493 10.620704376153
494 10.6099660137515
495 10.5992755593687
496 10.6013179486165
497 10.5947217335189
498 10.5846914443782
499 10.5781220961784
500 10.5687121258931
501 10.5728465666336
502 10.5634352981861
503 10.5786467058361
504 10.5935702197729
505 10.5977711643346
506 10.6041944896326
507 10.6020228336526
508 10.5968960591492
509 10.5883109042386
510 10.5784740772185
511 10.5700468083231
512 10.5739941779233
513 10.5718443596048
514 10.5785500665161
515 10.5805538381108
516 10.5742250388464
517 10.5705300653724
518 10.5655305400529
519 10.5619923205905
520 10.5518521047367
521 10.5498715112746
522 10.5709110830172
523 10.5727227088854
524 10.5626492015554
525 10.553162425937
526 10.5495836596869
527 10.5413539119502
528 10.5559270684887
529 10.5470278697204
530 10.5469721856802
531 10.5375903606356
532 10.5374784730216
533 10.5488471839035
534 10.5508110838355
535 10.5473970695104
536 10.538621435879
537 10.5306154465996
538 10.5257543060135
539 10.5165204776137
540 10.5067970683976
541 10.5154395271232
542 10.5083208453074
543 10.5121415310905
544 10.5072571055194
545 10.5109873535683
546 10.5168900305849
547 10.5089069787772
548 10.5085679772454
549 10.519274713301
550 10.512527906037
551 10.5032349120062
552 10.5067391181903
553 10.5158597016937
554 10.5079513915937
555 10.5030369997754
556 10.4951552751876
557 10.4890924537357
558 10.4862039802197
559 10.4885241934076
560 10.5001018884755
561 10.4923268727674
562 10.483135310373
563 10.4753982108115
564 10.4672592122642
565 10.4814795884412
566 10.4850416887404
567 10.4776292220442
568 10.471766459352
569 10.4698412560995
570 10.4868337038715
571 10.4834412224526
572 10.4743118328627
573 10.4667224265738
574 10.4591485755507
575 10.4588698800703
576 10.4569393657014
577 10.448121027325
578 10.4405910569983
579 10.4316153666163
580 10.4244662991387
581 10.4187412959273
582 10.4102099591045
583 10.4015263705046
584 10.411765469838
585 10.4172300614681
586 10.4298382674192
587 10.4497833450829
588 10.4458119268466
589 10.4373462290018
590 10.4472999306173
591 10.4462173283926
592 10.4380143439315
593 10.4298305054233
594 10.425913421304
595 10.442507363228
596 10.4589347083509
597 10.4549096309917
598 10.4530979388114
599 10.4444078205639
600 10.4529081546185
601 10.4562846598278
602 10.451832126646
603 10.4571617925475
604 10.4633974961694
605 10.4666690883914
606 10.4582641965217
607 10.4564170334027
608 10.4685750595215
609 10.460022475064
610 10.4596464171209
611 10.4558144704821
612 10.4704274617817
613 10.4727083933279
614 10.4860209872325
615 10.4992027307487
616 10.496551891891
617 10.5084984075271
618 10.4999983791339
619 10.5020928752574
620 10.505192215069
621 10.4969611743284
622 10.5029401769162
623 10.4979452230815
624 10.5123314262404
625 10.5056175449138
626 10.5115133666982
627 10.5134958906439
628 10.5148148385258
629 10.5328281462432
630 10.5244990077951
631 10.5295220051882
632 10.5217307424994
633 10.5164088624226
634 10.5137735692039
635 10.5262910089739
636 10.5188480647733
637 10.518558998133
638 10.5335447500043
639 10.5257101462393
640 10.5214884284497
641 10.5270969503042
642 10.5198751429087
643 10.534637968074
644 10.5467487998206
645 10.5538255708837
646 10.5479554704311
647 10.5480171445093
648 10.565077535929
649 10.5793559923461
650 10.5727017030547
651 10.5667603195902
652 10.5809035115167
653 10.5729616738229
654 10.5869819449797
655 10.5884643998198
656 10.5934754493614
657 10.6025784719055
658 10.6040721696181
659 10.6038506935483
660 10.6009095716621
661 10.5930620211606
662 10.5865403841503
663 10.5789792942993
664 10.586073806171
665 10.5782788316582
666 10.5725057094369
667 10.5650292121123
668 10.5648882072279
669 10.57612985093
670 10.568653343928
671 10.5607883974814
672 10.562127235838
673 10.5542894730085
674 10.5495011485128
675 10.5468407429459
676 10.5559932778891
677 10.5531986985433
678 10.5458676269516
679 10.5508967678897
680 10.5494550237241
681 10.5431231294172
682 10.5520573447941
683 10.5457152454936
684 10.5524848322182
685 10.5540075057413
686 10.5524860882693
687 10.5538286171109
688 10.5536379437306
689 10.5463769855702
690 10.5550254015153
691 10.5596573620547
692 10.553329892027
693 10.5624050749343
694 10.5669975521895
695 10.5601634415584
696 10.5588882997424
697 10.5534589262396
698 10.5463609293823
699 10.5509533703305
700 10.5638003365749
701 10.5566331842595
702 10.5518208171598
703 10.5506091973434
704 10.5438655026753
705 10.5363926770665
706 10.5468442659489
707 10.5451963828257
708 10.5435365007482
709 10.5547459903787
710 10.5574142776963
711 10.5503381467239
712 10.5505497383892
713 10.5507393871314
714 10.549558099464
715 10.5557835560639
716 10.5590400562396
717 10.5714850005198
718 10.5662519797435
719 10.5610250105813
720 10.5678290304427
721 10.5634010844913
722 10.5596266380566
723 10.5697912061214
724 10.563708833018
725 10.5609321103413
726 10.5538434994056
727 10.554025696325
728 10.5476543085022
729 10.5543986473945
730 10.5644827661175
731 10.5586684254128
732 10.5552551777104
733 10.548168907644
734 10.5428209676328
735 10.5444208012037
736 10.5374306809964
737 10.5340434465229
738 10.5289477515766
739 10.5274563308991
740 10.5276050217759
741 10.5239777326589
742 10.5213551505661
743 10.5319788038536
744 10.5441237116332
745 10.5483718308376
746 10.5424900515009
747 10.5421920485918
748 10.5352516070288
749 10.5437955016205
750 10.5435040348706
751 10.5393008942194
752 10.5329879873204
753 10.5288020441509
754 10.542942429796
755 10.5377245379119
756 10.5319114158961
757 10.5261073961196
758 10.5257863396962
759 10.5195200925816
760 10.5129187773347
761 10.5107726925793
762 10.5045436424111
763 10.5062277097693
764 10.4995320356076
765 10.5053640672345
766 10.5170118730759
767 10.5179896488777
768 10.5196987837105
769 10.5137252676178
770 10.5111010396924
771 10.5043175151959
772 10.5118229996728
773 10.5078433325126
774 10.510244755016
775 10.5048642615203
776 10.5022389421171
777 10.4961110444001
778 10.4934867506815
779 10.4983852495536
780 10.491847356245
781 10.4942292514533
782 10.4922275415236
783 10.486917989345
784 10.4872867840048
785 10.4829639301082
786 10.496420692818
787 10.4973397938957
788 10.496489473141
789 10.4918638461673
790 10.4872427670781
791 10.4807007014319
792 10.4872328956825
793 10.4806537238519
794 10.4756969129494
795 10.4795997158905
796 10.4919529877923
797 10.4859935964922
798 10.4883487775135
799 10.4848861916126
800 10.47894309735
801 10.4880141780481
802 10.4814754230493
803 10.4755475915106
804 10.4726613369917
805 10.4681575107413
806 10.479155046629
807 10.4772638421001
808 10.474384193962
809 10.4689452639156
810 10.4727245334043
811 10.4736059421276
812 10.4784065723362
813 10.4816039256514
814 10.4833281899232
815 10.4788584881883
816 10.4724719127933
817 10.4825055409096
818 10.4896113486695
819 10.4834908859754
820 10.4784288066389
821 10.4913065065152
822 10.4935382436655
823 10.4877466913916
824 10.4814609927644
825 10.4786435309125
826 10.4738508612946
827 10.4755744426401
828 10.4727569885584
829 10.4667178855281
830 10.4704286045655
831 10.4784823680898
832 10.4756669142344
833 10.4709330132546
834 10.464837655136
835 10.4607735372142
836 10.4604597703749
837 10.4542111602298
838 10.4492919647659
839 10.4539206346934
840 10.4665512337107
841 10.4625011434972
842 10.4563656508745
843 10.4530844382322
844 10.4478758307778
845 10.4471630059758
846 10.4464389014224
847 10.4415767304777
848 10.4375846692281
849 10.4320092157744
850 10.4324603885079
851 10.4264097901878
852 10.421587213157
853 10.4320968939609
854 10.4269681048896
855 10.4218465052922
856 10.4167320797434
857 10.4153557670466
858 10.4130038224075
859 10.4165923341808
860 10.4183888967486
861 10.4149541300484
862 10.4103892903974
863 10.4226574271124
864 10.4348405632363
865 10.4380083561784
866 10.4411466657593
867 10.4446431581671
868 10.4432084149625
869 10.4385148408676
870 10.4361369996893
871 10.4423327768123
872 10.4519634569803
873 10.4487687928813
874 10.4461746924116
875 10.4576222216435
876 10.455910127179
877 10.4603600583936
878 10.4544033857897
879 10.4505377054889
880 10.4526589206956
881 10.4586555470106
882 10.4535133250696
883 10.4508945631291
884 10.445935035319
885 10.4480654765117
886 10.4436158096793
887 10.4377639273816
888 10.4346601349604
889 10.4329922166468
890 10.4448950495874
891 10.4483219654978
892 10.4527565670298
893 10.4626493056723
894 10.45771072446
895 10.4597236481022
896 10.4583186969282
897 10.4590597445004
898 10.459784473624
899 10.4547800553341
900 10.4491641818265
901 10.4434077710499
902 10.4376804101431
903 10.4453099969046
904 10.4472875435578
905 10.4458852095364
906 10.4518584084629
907 10.4495543870409
908 10.4481520964889
909 10.4577466788728
910 10.4520585106596
911 10.4505115942226
912 10.4461295587852
913 10.4522951338965
914 10.4465758161114
915 10.4496291475187
916 10.4573881365616
917 10.4529630064955
918 10.4548301288541
919 10.4524360995918
920 10.4501191643064
921 10.4470167731941
922 10.4431978383899
923 10.4508657380128
924 10.4452628790881
925 10.4404508605198
926 10.4390120703298
927 10.4408070175444
928 10.4403175754962
929 10.4349090789053
930 10.4441218342168
931 10.441061848011
932 10.4440840954745
933 10.4551835567088
934 10.4528861315085
935 10.4524308760454
936 10.4530101801308
937 10.4515853286648
938 10.4460126468577
939 10.4534589784944
940 10.4511775324746
941 10.4507268036259
942 10.4566129405659
943 10.4515222789286
944 10.4461889414906
945 10.4438963526764
946 10.4388262361548
947 10.4383832755575
948 10.4379198876217
949 10.4328682703087
950 10.4387431450704
951 10.4340575403987
952 10.428778749065
953 10.4346210743653
954 10.4454223427436
955 10.4431645521783
956 10.4461571892615
957 10.4551117704614
958 10.4501120769762
959 10.4543920916859
960 10.4549222524874
961 10.4591503695723
962 10.4538980657166
963 10.4569462360806
964 10.4575803722446
965 10.4522146765847
966 10.4476128543256
967 10.4548011151899
968 10.4501672613853
969 10.4545263387054
970 10.4574016582526
971 10.4524761727108
972 10.4580761538265
973 10.4559001146023
974 10.451789402569
975 10.4482280331897
976 10.4538180767806
977 10.4492233601339
978 10.4455983856104
979 10.4542470989302
980 10.4529460743205
981 10.4632454378249
982 10.4581318646488
983 10.4540762380769
984 10.4491723637227
985 10.4450327952103
986 10.4399091131333
987 10.4471931470221
988 10.442644872571
989 10.4454207183405
990 10.4559768697518
991 10.4546834468892
992 10.4525578723764
993 10.4473487126675
994 10.4501340289894
995 10.4452901956108
996 10.4537779912689
997 10.4525069776869
998 10.4484218136285
999 10.4469834779286
1000 10.4474321725484
};
\addlegendentry{$v_{d}$ (valor del desvío de las tiradas)}
\addplot [semithick, color1]
table {%
1 10.6770782520313
2 10.6770782520313
3 10.6770782520313
4 10.6770782520313
5 10.6770782520313
6 10.6770782520313
7 10.6770782520313
8 10.6770782520313
9 10.6770782520313
10 10.6770782520313
11 10.6770782520313
12 10.6770782520313
13 10.6770782520313
14 10.6770782520313
15 10.6770782520313
16 10.6770782520313
17 10.6770782520313
18 10.6770782520313
19 10.6770782520313
20 10.6770782520313
21 10.6770782520313
22 10.6770782520313
23 10.6770782520313
24 10.6770782520313
25 10.6770782520313
26 10.6770782520313
27 10.6770782520313
28 10.6770782520313
29 10.6770782520313
30 10.6770782520313
31 10.6770782520313
32 10.6770782520313
33 10.6770782520313
34 10.6770782520313
35 10.6770782520313
36 10.6770782520313
37 10.6770782520313
38 10.6770782520313
39 10.6770782520313
40 10.6770782520313
41 10.6770782520313
42 10.6770782520313
43 10.6770782520313
44 10.6770782520313
45 10.6770782520313
46 10.6770782520313
47 10.6770782520313
48 10.6770782520313
49 10.6770782520313
50 10.6770782520313
51 10.6770782520313
52 10.6770782520313
53 10.6770782520313
54 10.6770782520313
55 10.6770782520313
56 10.6770782520313
57 10.6770782520313
58 10.6770782520313
59 10.6770782520313
60 10.6770782520313
61 10.6770782520313
62 10.6770782520313
63 10.6770782520313
64 10.6770782520313
65 10.6770782520313
66 10.6770782520313
67 10.6770782520313
68 10.6770782520313
69 10.6770782520313
70 10.6770782520313
71 10.6770782520313
72 10.6770782520313
73 10.6770782520313
74 10.6770782520313
75 10.6770782520313
76 10.6770782520313
77 10.6770782520313
78 10.6770782520313
79 10.6770782520313
80 10.6770782520313
81 10.6770782520313
82 10.6770782520313
83 10.6770782520313
84 10.6770782520313
85 10.6770782520313
86 10.6770782520313
87 10.6770782520313
88 10.6770782520313
89 10.6770782520313
90 10.6770782520313
91 10.6770782520313
92 10.6770782520313
93 10.6770782520313
94 10.6770782520313
95 10.6770782520313
96 10.6770782520313
97 10.6770782520313
98 10.6770782520313
99 10.6770782520313
100 10.6770782520313
101 10.6770782520313
102 10.6770782520313
103 10.6770782520313
104 10.6770782520313
105 10.6770782520313
106 10.6770782520313
107 10.6770782520313
108 10.6770782520313
109 10.6770782520313
110 10.6770782520313
111 10.6770782520313
112 10.6770782520313
113 10.6770782520313
114 10.6770782520313
115 10.6770782520313
116 10.6770782520313
117 10.6770782520313
118 10.6770782520313
119 10.6770782520313
120 10.6770782520313
121 10.6770782520313
122 10.6770782520313
123 10.6770782520313
124 10.6770782520313
125 10.6770782520313
126 10.6770782520313
127 10.6770782520313
128 10.6770782520313
129 10.6770782520313
130 10.6770782520313
131 10.6770782520313
132 10.6770782520313
133 10.6770782520313
134 10.6770782520313
135 10.6770782520313
136 10.6770782520313
137 10.6770782520313
138 10.6770782520313
139 10.6770782520313
140 10.6770782520313
141 10.6770782520313
142 10.6770782520313
143 10.6770782520313
144 10.6770782520313
145 10.6770782520313
146 10.6770782520313
147 10.6770782520313
148 10.6770782520313
149 10.6770782520313
150 10.6770782520313
151 10.6770782520313
152 10.6770782520313
153 10.6770782520313
154 10.6770782520313
155 10.6770782520313
156 10.6770782520313
157 10.6770782520313
158 10.6770782520313
159 10.6770782520313
160 10.6770782520313
161 10.6770782520313
162 10.6770782520313
163 10.6770782520313
164 10.6770782520313
165 10.6770782520313
166 10.6770782520313
167 10.6770782520313
168 10.6770782520313
169 10.6770782520313
170 10.6770782520313
171 10.6770782520313
172 10.6770782520313
173 10.6770782520313
174 10.6770782520313
175 10.6770782520313
176 10.6770782520313
177 10.6770782520313
178 10.6770782520313
179 10.6770782520313
180 10.6770782520313
181 10.6770782520313
182 10.6770782520313
183 10.6770782520313
184 10.6770782520313
185 10.6770782520313
186 10.6770782520313
187 10.6770782520313
188 10.6770782520313
189 10.6770782520313
190 10.6770782520313
191 10.6770782520313
192 10.6770782520313
193 10.6770782520313
194 10.6770782520313
195 10.6770782520313
196 10.6770782520313
197 10.6770782520313
198 10.6770782520313
199 10.6770782520313
200 10.6770782520313
201 10.6770782520313
202 10.6770782520313
203 10.6770782520313
204 10.6770782520313
205 10.6770782520313
206 10.6770782520313
207 10.6770782520313
208 10.6770782520313
209 10.6770782520313
210 10.6770782520313
211 10.6770782520313
212 10.6770782520313
213 10.6770782520313
214 10.6770782520313
215 10.6770782520313
216 10.6770782520313
217 10.6770782520313
218 10.6770782520313
219 10.6770782520313
220 10.6770782520313
221 10.6770782520313
222 10.6770782520313
223 10.6770782520313
224 10.6770782520313
225 10.6770782520313
226 10.6770782520313
227 10.6770782520313
228 10.6770782520313
229 10.6770782520313
230 10.6770782520313
231 10.6770782520313
232 10.6770782520313
233 10.6770782520313
234 10.6770782520313
235 10.6770782520313
236 10.6770782520313
237 10.6770782520313
238 10.6770782520313
239 10.6770782520313
240 10.6770782520313
241 10.6770782520313
242 10.6770782520313
243 10.6770782520313
244 10.6770782520313
245 10.6770782520313
246 10.6770782520313
247 10.6770782520313
248 10.6770782520313
249 10.6770782520313
250 10.6770782520313
251 10.6770782520313
252 10.6770782520313
253 10.6770782520313
254 10.6770782520313
255 10.6770782520313
256 10.6770782520313
257 10.6770782520313
258 10.6770782520313
259 10.6770782520313
260 10.6770782520313
261 10.6770782520313
262 10.6770782520313
263 10.6770782520313
264 10.6770782520313
265 10.6770782520313
266 10.6770782520313
267 10.6770782520313
268 10.6770782520313
269 10.6770782520313
270 10.6770782520313
271 10.6770782520313
272 10.6770782520313
273 10.6770782520313
274 10.6770782520313
275 10.6770782520313
276 10.6770782520313
277 10.6770782520313
278 10.6770782520313
279 10.6770782520313
280 10.6770782520313
281 10.6770782520313
282 10.6770782520313
283 10.6770782520313
284 10.6770782520313
285 10.6770782520313
286 10.6770782520313
287 10.6770782520313
288 10.6770782520313
289 10.6770782520313
290 10.6770782520313
291 10.6770782520313
292 10.6770782520313
293 10.6770782520313
294 10.6770782520313
295 10.6770782520313
296 10.6770782520313
297 10.6770782520313
298 10.6770782520313
299 10.6770782520313
300 10.6770782520313
301 10.6770782520313
302 10.6770782520313
303 10.6770782520313
304 10.6770782520313
305 10.6770782520313
306 10.6770782520313
307 10.6770782520313
308 10.6770782520313
309 10.6770782520313
310 10.6770782520313
311 10.6770782520313
312 10.6770782520313
313 10.6770782520313
314 10.6770782520313
315 10.6770782520313
316 10.6770782520313
317 10.6770782520313
318 10.6770782520313
319 10.6770782520313
320 10.6770782520313
321 10.6770782520313
322 10.6770782520313
323 10.6770782520313
324 10.6770782520313
325 10.6770782520313
326 10.6770782520313
327 10.6770782520313
328 10.6770782520313
329 10.6770782520313
330 10.6770782520313
331 10.6770782520313
332 10.6770782520313
333 10.6770782520313
334 10.6770782520313
335 10.6770782520313
336 10.6770782520313
337 10.6770782520313
338 10.6770782520313
339 10.6770782520313
340 10.6770782520313
341 10.6770782520313
342 10.6770782520313
343 10.6770782520313
344 10.6770782520313
345 10.6770782520313
346 10.6770782520313
347 10.6770782520313
348 10.6770782520313
349 10.6770782520313
350 10.6770782520313
351 10.6770782520313
352 10.6770782520313
353 10.6770782520313
354 10.6770782520313
355 10.6770782520313
356 10.6770782520313
357 10.6770782520313
358 10.6770782520313
359 10.6770782520313
360 10.6770782520313
361 10.6770782520313
362 10.6770782520313
363 10.6770782520313
364 10.6770782520313
365 10.6770782520313
366 10.6770782520313
367 10.6770782520313
368 10.6770782520313
369 10.6770782520313
370 10.6770782520313
371 10.6770782520313
372 10.6770782520313
373 10.6770782520313
374 10.6770782520313
375 10.6770782520313
376 10.6770782520313
377 10.6770782520313
378 10.6770782520313
379 10.6770782520313
380 10.6770782520313
381 10.6770782520313
382 10.6770782520313
383 10.6770782520313
384 10.6770782520313
385 10.6770782520313
386 10.6770782520313
387 10.6770782520313
388 10.6770782520313
389 10.6770782520313
390 10.6770782520313
391 10.6770782520313
392 10.6770782520313
393 10.6770782520313
394 10.6770782520313
395 10.6770782520313
396 10.6770782520313
397 10.6770782520313
398 10.6770782520313
399 10.6770782520313
400 10.6770782520313
401 10.6770782520313
402 10.6770782520313
403 10.6770782520313
404 10.6770782520313
405 10.6770782520313
406 10.6770782520313
407 10.6770782520313
408 10.6770782520313
409 10.6770782520313
410 10.6770782520313
411 10.6770782520313
412 10.6770782520313
413 10.6770782520313
414 10.6770782520313
415 10.6770782520313
416 10.6770782520313
417 10.6770782520313
418 10.6770782520313
419 10.6770782520313
420 10.6770782520313
421 10.6770782520313
422 10.6770782520313
423 10.6770782520313
424 10.6770782520313
425 10.6770782520313
426 10.6770782520313
427 10.6770782520313
428 10.6770782520313
429 10.6770782520313
430 10.6770782520313
431 10.6770782520313
432 10.6770782520313
433 10.6770782520313
434 10.6770782520313
435 10.6770782520313
436 10.6770782520313
437 10.6770782520313
438 10.6770782520313
439 10.6770782520313
440 10.6770782520313
441 10.6770782520313
442 10.6770782520313
443 10.6770782520313
444 10.6770782520313
445 10.6770782520313
446 10.6770782520313
447 10.6770782520313
448 10.6770782520313
449 10.6770782520313
450 10.6770782520313
451 10.6770782520313
452 10.6770782520313
453 10.6770782520313
454 10.6770782520313
455 10.6770782520313
456 10.6770782520313
457 10.6770782520313
458 10.6770782520313
459 10.6770782520313
460 10.6770782520313
461 10.6770782520313
462 10.6770782520313
463 10.6770782520313
464 10.6770782520313
465 10.6770782520313
466 10.6770782520313
467 10.6770782520313
468 10.6770782520313
469 10.6770782520313
470 10.6770782520313
471 10.6770782520313
472 10.6770782520313
473 10.6770782520313
474 10.6770782520313
475 10.6770782520313
476 10.6770782520313
477 10.6770782520313
478 10.6770782520313
479 10.6770782520313
480 10.6770782520313
481 10.6770782520313
482 10.6770782520313
483 10.6770782520313
484 10.6770782520313
485 10.6770782520313
486 10.6770782520313
487 10.6770782520313
488 10.6770782520313
489 10.6770782520313
490 10.6770782520313
491 10.6770782520313
492 10.6770782520313
493 10.6770782520313
494 10.6770782520313
495 10.6770782520313
496 10.6770782520313
497 10.6770782520313
498 10.6770782520313
499 10.6770782520313
500 10.6770782520313
501 10.6770782520313
502 10.6770782520313
503 10.6770782520313
504 10.6770782520313
505 10.6770782520313
506 10.6770782520313
507 10.6770782520313
508 10.6770782520313
509 10.6770782520313
510 10.6770782520313
511 10.6770782520313
512 10.6770782520313
513 10.6770782520313
514 10.6770782520313
515 10.6770782520313
516 10.6770782520313
517 10.6770782520313
518 10.6770782520313
519 10.6770782520313
520 10.6770782520313
521 10.6770782520313
522 10.6770782520313
523 10.6770782520313
524 10.6770782520313
525 10.6770782520313
526 10.6770782520313
527 10.6770782520313
528 10.6770782520313
529 10.6770782520313
530 10.6770782520313
531 10.6770782520313
532 10.6770782520313
533 10.6770782520313
534 10.6770782520313
535 10.6770782520313
536 10.6770782520313
537 10.6770782520313
538 10.6770782520313
539 10.6770782520313
540 10.6770782520313
541 10.6770782520313
542 10.6770782520313
543 10.6770782520313
544 10.6770782520313
545 10.6770782520313
546 10.6770782520313
547 10.6770782520313
548 10.6770782520313
549 10.6770782520313
550 10.6770782520313
551 10.6770782520313
552 10.6770782520313
553 10.6770782520313
554 10.6770782520313
555 10.6770782520313
556 10.6770782520313
557 10.6770782520313
558 10.6770782520313
559 10.6770782520313
560 10.6770782520313
561 10.6770782520313
562 10.6770782520313
563 10.6770782520313
564 10.6770782520313
565 10.6770782520313
566 10.6770782520313
567 10.6770782520313
568 10.6770782520313
569 10.6770782520313
570 10.6770782520313
571 10.6770782520313
572 10.6770782520313
573 10.6770782520313
574 10.6770782520313
575 10.6770782520313
576 10.6770782520313
577 10.6770782520313
578 10.6770782520313
579 10.6770782520313
580 10.6770782520313
581 10.6770782520313
582 10.6770782520313
583 10.6770782520313
584 10.6770782520313
585 10.6770782520313
586 10.6770782520313
587 10.6770782520313
588 10.6770782520313
589 10.6770782520313
590 10.6770782520313
591 10.6770782520313
592 10.6770782520313
593 10.6770782520313
594 10.6770782520313
595 10.6770782520313
596 10.6770782520313
597 10.6770782520313
598 10.6770782520313
599 10.6770782520313
600 10.6770782520313
601 10.6770782520313
602 10.6770782520313
603 10.6770782520313
604 10.6770782520313
605 10.6770782520313
606 10.6770782520313
607 10.6770782520313
608 10.6770782520313
609 10.6770782520313
610 10.6770782520313
611 10.6770782520313
612 10.6770782520313
613 10.6770782520313
614 10.6770782520313
615 10.6770782520313
616 10.6770782520313
617 10.6770782520313
618 10.6770782520313
619 10.6770782520313
620 10.6770782520313
621 10.6770782520313
622 10.6770782520313
623 10.6770782520313
624 10.6770782520313
625 10.6770782520313
626 10.6770782520313
627 10.6770782520313
628 10.6770782520313
629 10.6770782520313
630 10.6770782520313
631 10.6770782520313
632 10.6770782520313
633 10.6770782520313
634 10.6770782520313
635 10.6770782520313
636 10.6770782520313
637 10.6770782520313
638 10.6770782520313
639 10.6770782520313
640 10.6770782520313
641 10.6770782520313
642 10.6770782520313
643 10.6770782520313
644 10.6770782520313
645 10.6770782520313
646 10.6770782520313
647 10.6770782520313
648 10.6770782520313
649 10.6770782520313
650 10.6770782520313
651 10.6770782520313
652 10.6770782520313
653 10.6770782520313
654 10.6770782520313
655 10.6770782520313
656 10.6770782520313
657 10.6770782520313
658 10.6770782520313
659 10.6770782520313
660 10.6770782520313
661 10.6770782520313
662 10.6770782520313
663 10.6770782520313
664 10.6770782520313
665 10.6770782520313
666 10.6770782520313
667 10.6770782520313
668 10.6770782520313
669 10.6770782520313
670 10.6770782520313
671 10.6770782520313
672 10.6770782520313
673 10.6770782520313
674 10.6770782520313
675 10.6770782520313
676 10.6770782520313
677 10.6770782520313
678 10.6770782520313
679 10.6770782520313
680 10.6770782520313
681 10.6770782520313
682 10.6770782520313
683 10.6770782520313
684 10.6770782520313
685 10.6770782520313
686 10.6770782520313
687 10.6770782520313
688 10.6770782520313
689 10.6770782520313
690 10.6770782520313
691 10.6770782520313
692 10.6770782520313
693 10.6770782520313
694 10.6770782520313
695 10.6770782520313
696 10.6770782520313
697 10.6770782520313
698 10.6770782520313
699 10.6770782520313
700 10.6770782520313
701 10.6770782520313
702 10.6770782520313
703 10.6770782520313
704 10.6770782520313
705 10.6770782520313
706 10.6770782520313
707 10.6770782520313
708 10.6770782520313
709 10.6770782520313
710 10.6770782520313
711 10.6770782520313
712 10.6770782520313
713 10.6770782520313
714 10.6770782520313
715 10.6770782520313
716 10.6770782520313
717 10.6770782520313
718 10.6770782520313
719 10.6770782520313
720 10.6770782520313
721 10.6770782520313
722 10.6770782520313
723 10.6770782520313
724 10.6770782520313
725 10.6770782520313
726 10.6770782520313
727 10.6770782520313
728 10.6770782520313
729 10.6770782520313
730 10.6770782520313
731 10.6770782520313
732 10.6770782520313
733 10.6770782520313
734 10.6770782520313
735 10.6770782520313
736 10.6770782520313
737 10.6770782520313
738 10.6770782520313
739 10.6770782520313
740 10.6770782520313
741 10.6770782520313
742 10.6770782520313
743 10.6770782520313
744 10.6770782520313
745 10.6770782520313
746 10.6770782520313
747 10.6770782520313
748 10.6770782520313
749 10.6770782520313
750 10.6770782520313
751 10.6770782520313
752 10.6770782520313
753 10.6770782520313
754 10.6770782520313
755 10.6770782520313
756 10.6770782520313
757 10.6770782520313
758 10.6770782520313
759 10.6770782520313
760 10.6770782520313
761 10.6770782520313
762 10.6770782520313
763 10.6770782520313
764 10.6770782520313
765 10.6770782520313
766 10.6770782520313
767 10.6770782520313
768 10.6770782520313
769 10.6770782520313
770 10.6770782520313
771 10.6770782520313
772 10.6770782520313
773 10.6770782520313
774 10.6770782520313
775 10.6770782520313
776 10.6770782520313
777 10.6770782520313
778 10.6770782520313
779 10.6770782520313
780 10.6770782520313
781 10.6770782520313
782 10.6770782520313
783 10.6770782520313
784 10.6770782520313
785 10.6770782520313
786 10.6770782520313
787 10.6770782520313
788 10.6770782520313
789 10.6770782520313
790 10.6770782520313
791 10.6770782520313
792 10.6770782520313
793 10.6770782520313
794 10.6770782520313
795 10.6770782520313
796 10.6770782520313
797 10.6770782520313
798 10.6770782520313
799 10.6770782520313
800 10.6770782520313
801 10.6770782520313
802 10.6770782520313
803 10.6770782520313
804 10.6770782520313
805 10.6770782520313
806 10.6770782520313
807 10.6770782520313
808 10.6770782520313
809 10.6770782520313
810 10.6770782520313
811 10.6770782520313
812 10.6770782520313
813 10.6770782520313
814 10.6770782520313
815 10.6770782520313
816 10.6770782520313
817 10.6770782520313
818 10.6770782520313
819 10.6770782520313
820 10.6770782520313
821 10.6770782520313
822 10.6770782520313
823 10.6770782520313
824 10.6770782520313
825 10.6770782520313
826 10.6770782520313
827 10.6770782520313
828 10.6770782520313
829 10.6770782520313
830 10.6770782520313
831 10.6770782520313
832 10.6770782520313
833 10.6770782520313
834 10.6770782520313
835 10.6770782520313
836 10.6770782520313
837 10.6770782520313
838 10.6770782520313
839 10.6770782520313
840 10.6770782520313
841 10.6770782520313
842 10.6770782520313
843 10.6770782520313
844 10.6770782520313
845 10.6770782520313
846 10.6770782520313
847 10.6770782520313
848 10.6770782520313
849 10.6770782520313
850 10.6770782520313
851 10.6770782520313
852 10.6770782520313
853 10.6770782520313
854 10.6770782520313
855 10.6770782520313
856 10.6770782520313
857 10.6770782520313
858 10.6770782520313
859 10.6770782520313
860 10.6770782520313
861 10.6770782520313
862 10.6770782520313
863 10.6770782520313
864 10.6770782520313
865 10.6770782520313
866 10.6770782520313
867 10.6770782520313
868 10.6770782520313
869 10.6770782520313
870 10.6770782520313
871 10.6770782520313
872 10.6770782520313
873 10.6770782520313
874 10.6770782520313
875 10.6770782520313
876 10.6770782520313
877 10.6770782520313
878 10.6770782520313
879 10.6770782520313
880 10.6770782520313
881 10.6770782520313
882 10.6770782520313
883 10.6770782520313
884 10.6770782520313
885 10.6770782520313
886 10.6770782520313
887 10.6770782520313
888 10.6770782520313
889 10.6770782520313
890 10.6770782520313
891 10.6770782520313
892 10.6770782520313
893 10.6770782520313
894 10.6770782520313
895 10.6770782520313
896 10.6770782520313
897 10.6770782520313
898 10.6770782520313
899 10.6770782520313
900 10.6770782520313
901 10.6770782520313
902 10.6770782520313
903 10.6770782520313
904 10.6770782520313
905 10.6770782520313
906 10.6770782520313
907 10.6770782520313
908 10.6770782520313
909 10.6770782520313
910 10.6770782520313
911 10.6770782520313
912 10.6770782520313
913 10.6770782520313
914 10.6770782520313
915 10.6770782520313
916 10.6770782520313
917 10.6770782520313
918 10.6770782520313
919 10.6770782520313
920 10.6770782520313
921 10.6770782520313
922 10.6770782520313
923 10.6770782520313
924 10.6770782520313
925 10.6770782520313
926 10.6770782520313
927 10.6770782520313
928 10.6770782520313
929 10.6770782520313
930 10.6770782520313
931 10.6770782520313
932 10.6770782520313
933 10.6770782520313
934 10.6770782520313
935 10.6770782520313
936 10.6770782520313
937 10.6770782520313
938 10.6770782520313
939 10.6770782520313
940 10.6770782520313
941 10.6770782520313
942 10.6770782520313
943 10.6770782520313
944 10.6770782520313
945 10.6770782520313
946 10.6770782520313
947 10.6770782520313
948 10.6770782520313
949 10.6770782520313
950 10.6770782520313
951 10.6770782520313
952 10.6770782520313
953 10.6770782520313
954 10.6770782520313
955 10.6770782520313
956 10.6770782520313
957 10.6770782520313
958 10.6770782520313
959 10.6770782520313
960 10.6770782520313
961 10.6770782520313
962 10.6770782520313
963 10.6770782520313
964 10.6770782520313
965 10.6770782520313
966 10.6770782520313
967 10.6770782520313
968 10.6770782520313
969 10.6770782520313
970 10.6770782520313
971 10.6770782520313
972 10.6770782520313
973 10.6770782520313
974 10.6770782520313
975 10.6770782520313
976 10.6770782520313
977 10.6770782520313
978 10.6770782520313
979 10.6770782520313
980 10.6770782520313
981 10.6770782520313
982 10.6770782520313
983 10.6770782520313
984 10.6770782520313
985 10.6770782520313
986 10.6770782520313
987 10.6770782520313
988 10.6770782520313
989 10.6770782520313
990 10.6770782520313
991 10.6770782520313
992 10.6770782520313
993 10.6770782520313
994 10.6770782520313
995 10.6770782520313
996 10.6770782520313
997 10.6770782520313
998 10.6770782520313
999 10.6770782520313
1000 10.6770782520313
};
\addlegendentry{$v_{d_{e}}$ (valor del desvío esperado)}
\end{axis}

\end{tikzpicture}

    \caption{valor del desvío con respecto al número de tiradas}
  \end{mytikzresize}
\end{figure}

\begin{figure}[!htbp]
  \begin{mytikzresize}{0.6\textwidth}
    \centering
    % This file was created by tikzplotlib v0.9.1.
\begin{tikzpicture}

\definecolor{color0}{rgb}{0.12156862745098,0.466666666666667,0.705882352941177}
\definecolor{color1}{rgb}{1,0.498039215686275,0.0549019607843137}

\begin{axis}[
legend cell align={left},
legend style={fill opacity=0.5, draw opacity=1, text opacity=1, draw=white!80!black},
scaled ticks=false,
tick align=outside,
tick pos=left,
width=\figW,
x grid style={white!69.0196078431373!black},
xlabel={\(\displaystyle n\) (número de tiradas)},
xmajorgrids,
xmin=-48.95, xmax=1049.95,
xtick style={color=black},
xticklabel style={/pgf/number format/.cd,fixed,precision=2},
y grid style={white!69.0196078431373!black},
ylabel={\(\displaystyle v_{d}\) (valor de la varianza)},
ymajorgrids,
ymin=-9.8, ymax=205.8,
ytick style={color=black},
yticklabel style={/pgf/number format/.cd,fixed,precision=2}
]
\addplot [semithick, color0]
table {%
1 0
2 196
3 174.222222222222
4 154.75
5 144.96
6 157.25
7 134.816326530612
8 126
9 127.432098765432
10 127.89
11 120.198347107438
12 110.305555555556
13 102.390532544379
14 95.454081632653
15 96.3288888888889
16 93.125
17 120.878892733564
18 117.472222222222
19 141.218836565097
20 140.76
21 159.746031746032
22 155.743801652893
23 150.854442344045
24 146.776041666667
25 154.5856
26 150.390532544379
27 148.666666666667
28 154.515306122449
29 155.384066587396
30 164.182222222222
31 163.281997918835
32 158.796875
33 162.855831037649
34 158.06660899654
35 154.599183673469
36 152.045524691358
37 151.601168736304
38 152.98891966759
39 149.096646942801
40 145.3975
41 142.722189173111
42 143.311224489796
43 149.26014061655
44 148.681818181818
45 147.550617283951
46 145.629489603025
47 142.770484382073
48 140.072482638889
49 143.979175343607
50 144.9524
51 142.314494425221
52 139.850591715976
53 143.706657173371
54 141.252743484225
55 139.076363636364
56 137.142857142857
57 137.016312711604
58 134.703032104637
59 133.978741740879
60 136.203055555556
61 133.980112872884
62 134.407127991675
63 136.559838750315
64 139.0302734375
65 137.129940828402
66 135.070018365473
67 133.117398084206
68 133.261894463668
69 135.890359168242
70 138.825306122449
71 138.985518746281
72 141.409722222222
73 142.280728091574
74 140.970964207451
75 140.594488888889
76 141.492209141274
77 140.220610558273
78 140.079717291256
79 138.312770389361
80 139.109375
81 139.111415942692
82 138.945419393218
83 138.947307301495
84 137.354166666667
85 139.000138408304
86 138.432666306111
87 138.108072400581
88 138.605371900826
89 137.177124100492
90 138.004444444444
91 138.177031759449
92 136.794305293006
93 135.519482021043
94 135.07888184699
95 135.597340720222
96 137.33984375
97 135.953023700712
98 134.70179092045
99 136.019793898582
100 136.2811
101 134.932849720616
102 135.169165705498
103 135.347346592516
104 134.151534763314
105 134.300589569161
106 133.284442862229
107 132.989256703642
108 133.070644718793
109 133.191313862469
110 132.121983471074
111 131.161431701972
112 131.731425382653
113 131.460098676482
114 133.448753462604
115 133.177769376181
116 132.103076694411
117 133.690700562495
118 135.057454754381
119 134.232751924299
120 134.521597222222
121 133.544839833345
122 132.973730180059
123 131.923590455417
124 133.435158688866
125 133.878656
126 134.428886369363
127 134.650629301259
128 133.6396484375
129 133.061955411333
130 132.140591715976
131 132.98269331624
132 133.147842056933
133 133.915088473062
134 134.74654711517
135 133.767352537723
136 132.832828719723
137 131.881187063775
138 132.038437303088
139 131.623207908493
140 132.634897959184
141 131.695789950204
142 131.403689744098
143 131.11614259866
144 131.869164737654
145 130.999952437574
146 130.181647588666
147 129.890971354528
148 130.58838568298
149 129.780910769785
150 129.5524
151 131.004780492084
152 130.180055401662
153 131.583237216455
154 131.195985832349
155 130.616607700312
156 131.102399737015
157 130.5497180413
158 130.7277679859
159 131.61884419129
160 131.759375
161 131.011303576251
162 130.993446121018
163 130.360043659904
164 129.765429803688
165 130.576822773186
166 131.689976774568
167 131.862167879809
168 131.247980442177
169 131.518014075137
170 131.36
171 130.637392702028
172 129.896261492699
173 129.409068127903
174 129.670927467301
175 130.535510204082
176 130.569085743802
177 130.223307478694
178 129.648939527837
179 129.106707031616
180 130.044320987654
181 129.696285217179
182 130.721651974399
183 130.023709277673
184 130.003987476371
185 129.378407596786
186 130.37093883686
187 130.171351768709
188 129.539384336804
189 129.793622798914
190 130.035346260388
191 130.407828732765
192 129.850667317708
193 129.7111868775
194 130.335875225848
195 130.099408284024
196 130.452519783424
197 129.790357906671
198 130.748290990715
199 130.511855761218
200 130.590775
201 131.064577609465
202 131.249509851975
203 130.64549006285
204 131.11272106882
205 130.478048780488
206 130.327740597606
207 129.702490139793
208 129.471061390533
209 129.236464366658
210 128.625306122449
211 128.136340154085
212 127.647739409042
213 127.068218387004
214 127.042034238798
215 126.451141157382
216 126.423782578875
217 126.879653422243
218 126.410739836714
219 125.833531410938
220 126.285847107438
221 126.004340615467
222 125.980683386089
223 126.174385167608
224 126.608099489796
225 126.495802469136
226 126.289392278174
227 125.886549321741
228 125.497441520468
229 125.584218455026
230 125.077958412098
231 125.383407357433
232 125.275862068966
233 125.460406343827
234 125.354536489152
235 124.888474422816
236 125.582950301637
237 125.653830404672
238 126.055663441847
239 125.533796677229
240 125.205399305556
241 126.149584201374
242 125.942848849122
243 125.947264136565
244 125.462022977694
245 125.465156184923
246 125.617175622976
247 125.49079643987
248 125.059166883455
249 124.980145481525
250 125.179136
251 124.98827002746
252 124.645549886621
253 124.259432267337
254 123.809923119846
255 123.728350634371
256 123.915771484375
257 124.821329618919
258 124.394702842377
259 123.95298221553
260 123.871301775148
261 123.935790725327
262 124.455800944001
263 123.996270005349
264 123.531206955923
265 123.101231755073
266 122.883599977387
267 122.514413163321
268 122.290529627979
269 121.849172897003
270 121.400768175583
271 121.040399776692
272 120.971764165225
273 121.136041004173
274 120.818530555704
275 121.405699173554
276 122.105899495904
277 121.671102190827
278 122.488289943585
279 122.686347811565
280 123.491517857143
281 123.617254087461
282 123.22418389417
283 123.333017018567
284 123.456791807181
285 123.98325638658
286 123.8636241381
287 123.62517451954
288 124.26384066358
289 124.462949437866
290 125.090951248514
291 125.197954676964
292 125.69585053481
293 125.372712553437
294 125.5695427831
295 125.447905774203
296 126.170745069394
297 126.440975410672
298 126.035043466511
299 125.619467343766
300 125.578888888889
301 125.257050142934
302 124.88628788211
303 124.482959187008
304 124.425802891274
305 124.628325718893
306 124.661081208082
307 124.350221222506
308 124.545285882948
309 124.814989369613
310 124.906347554631
311 124.86791906618
312 124.598115959895
313 124.85753656769
314 124.468832406994
315 124.479778281683
316 124.210743470598
317 124.058454159162
318 124.441052569123
319 124.917365198848
320 125.285
321 125.73507632884
322 125.573637205355
323 125.224300050801
324 125.583095183661
325 125.203446153846
326 125.863750987994
327 125.479860468161
328 125.11745240928
329 124.959562457849
330 124.71015610652
331 124.339974991101
332 124.780474306866
333 124.680265851437
334 125.124242532898
335 124.965613722433
336 125.055236678005
337 125.006278121671
338 125.430167010959
339 125.232063765543
340 125.774429065744
341 126.11448129961
342 125.746528846483
343 125.38058122041
344 125.102994862088
345 125.621373660996
346 126.254368672525
347 125.891137705653
348 126.410787422381
349 126.922422640208
350 126.565681632653
351 126.653663525458
352 126.294477982955
353 126.612636326429
354 126.513230553162
355 126.277294187661
356 126.039854500694
357 126.442875189291
358 126.12199993758
359 126.517485121934
360 126.288510802469
361 126.194888007305
362 126.156039192943
363 125.815313161669
364 126.221916133317
365 125.882049164947
366 125.961897936636
367 125.61919681637
368 126.101717568526
369 126.189834093463
370 126.666975894814
371 126.614947581026
372 127.177094172737
373 127.02026177145
374 127.258743458492
375 126.920632888889
376 126.885779764599
377 126.606519429532
378 126.347330701828
379 126.555398528275
380 126.714542936288
381 126.847955029243
382 126.814423946712
383 127.295939027466
384 127.046440972222
385 127.412879068983
386 127.860915729281
387 127.683005161282
388 127.468354766713
389 127.254221158993
390 126.928967784352
391 126.841203288832
392 127.060853550604
393 127.434738975325
394 127.186999149682
395 126.917186348342
396 127.204519946944
397 126.960249731932
398 127.322851695664
399 127.146261644085
400 126.84359375
401 127.201435314457
402 127.727067399322
403 127.481654341816
404 127.338765807274
405 127.530949550373
406 127.821980635298
407 128.187806748003
408 128.469717176086
409 128.428416855471
410 128.219565734682
411 127.956003102042
412 127.822226411537
413 128.091224079405
414 128.002129571285
415 128.066970532733
416 128.020057091346
417 128.209558971528
418 128.159571667315
419 127.853988072522
420 127.625504535147
421 127.810969245265
422 127.729369286404
423 127.428410151513
424 127.128159487362
425 127.004711418685
426 127.413123498424
427 127.489878187233
428 127.917476417154
429 127.619671703588
430 127.395678745268
431 127.592734750567
432 128.009066358025
433 128.339678594478
434 128.175072734609
435 127.927578279826
436 127.841737858766
437 127.618315014479
438 127.57857947082
439 127.587818660136
440 127.298326446281
441 127.311007244924
442 127.496611453492
443 127.209269856152
444 127.466033601169
445 127.717328620124
446 127.457620302037
447 127.645421377415
448 127.374157963967
449 127.693344775076
450 128.083535802469
451 127.846441266267
452 127.68466598794
453 127.431516161572
454 127.197388848998
455 127.203207342108
456 127.116877308403
457 127.289074881852
458 127.479987033047
459 127.496670321481
460 127.308015122873
461 127.311870356341
462 127.037218193062
463 126.777966963507
464 126.53328905321
465 126.767737310672
466 126.501667004366
467 126.277290463985
468 126.013309043758
469 125.830369929215
470 125.602539610684
471 125.375507683431
472 125.149270145073
473 124.885696534647
474 124.81995406719
475 124.568571745152
476 124.330039368689
477 124.1928721174
478 124.170585248858
479 124.559943514891
480 124.365815972222
481 124.10759808265
482 124.033182107746
483 123.780341121956
484 123.536285089816
485 123.371852481667
486 123.207725787058
487 122.998030939963
488 123.012345471647
489 123.258116183857
490 123.118738025823
491 122.868322265131
492 122.940346354683
493 122.691630082823
494 122.924371813995
495 122.715086215692
496 122.787521949792
497 123.081685282723
498 122.835292979145
499 122.765306163429
500 122.577264
501 122.703431460432
502 122.883751845209
503 123.061669742974
504 122.842576845553
505 122.74384864229
506 122.973019419144
507 122.993763834911
508 123.420965341931
509 123.178835962498
510 123.110976547482
511 123.474917758434
512 123.585678100586
513 123.369842192659
514 123.437254916804
515 123.222109529645
516 123.513190313082
517 123.68125886213
518 123.454450589586
519 123.355964671946
520 123.330469674556
521 123.09816129472
522 122.97030651341
523 122.758559421487
524 123.108064798089
525 122.873999092971
526 122.721251572236
527 122.624731302817
528 122.393021120294
529 122.613183915152
530 122.486094695621
531 122.500842315072
532 122.294826587145
533 122.237545276304
534 122.405448245872
535 122.627373569744
536 122.737270967922
537 122.712108444389
538 122.586752532442
539 122.645323401751
540 122.915624142661
541 122.712680358479
542 122.489784316662
543 122.269039406612
544 122.214042901168
545 121.994492046124
546 122.109101826135
547 122.019992714123
548 122.238198625393
549 122.574218400072
550 122.845292561983
551 122.752224136284
552 122.565686699223
553 122.35482932157
554 122.418238215016
555 122.299858777697
556 122.24275399824
557 122.454241593043
558 122.334862090672
559 122.151631619202
560 121.937292729592
561 121.7422923796
562 121.537113258444
563 121.448324599567
564 121.711998893416
565 121.822307150129
566 121.838133201813
567 121.94764984183
568 121.928573199762
569 122.253514166314
570 122.515878731918
571 122.322388902009
572 122.119150080689
573 122.09669569243
574 122.475555123894
575 122.799649149338
576 122.68546248071
577 122.526034079338
578 122.675399001449
579 122.995200467723
580 122.939464922711
581 122.955643572569
582 122.933704136701
583 122.757700308042
584 122.958408589792
585 123.013911900066
586 122.804345420448
587 122.689673185922
588 122.533446712018
589 122.360462468401
590 122.506868715886
591 122.33281512593
592 122.351987650657
593 122.149717473958
594 122.037365801676
595 121.951715274345
596 121.796368969866
597 121.939805111543
598 121.808827641749
599 122.064665371613
600 122.310330555556
601 122.460823751872
602 122.479553205815
603 122.276764326516
604 122.474277443972
605 122.666768663343
606 122.483895914344
607 122.316326004413
608 122.265016339162
609 122.113162982196
610 122.25930932545
611 122.274444780765
612 122.571860715964
613 122.811541132983
614 122.682288406243
615 122.483082821072
616 122.315810001687
617 122.164853725745
618 122.10983336999
619 121.912799058359
620 121.781737773153
621 121.873063289432
622 121.728375947312
623 121.554235833962
624 121.841870069034
625 121.64762624
626 121.572977166246
627 121.63383520422
628 121.693689399164
629 121.546361474165
630 121.422816830436
631 121.347585524449
632 121.158708540298
633 121.217512834143
634 121.237312044104
635 121.426910533821
636 121.661709880938
637 121.717908465161
638 121.594031603463
639 121.54392255113
640 121.527419433594
641 121.385403559668
642 121.671635562543
643 121.655082005278
644 121.466610084487
645 121.297556637221
646 121.3521575976
647 121.30377989962
648 121.285884392623
649 121.568329609854
650 121.444991715976
651 121.72435647863
652 121.585644924536
653 121.447211480058
654 121.309055541528
655 121.23585805023
656 121.05474559414
657 121.069512497423
658 121.055175488031
659 120.889433339243
660 120.817196969697
661 120.64424003424
662 120.921479814898
663 121.010626318052
664 120.910888282044
665 120.924068064899
666 120.936623560497
667 120.755918892128
668 120.744278658252
669 120.595668523397
670 120.415896636222
671 120.404112464214
672 120.233504730017
673 120.083540687392
674 119.991152955472
675 119.813917146776
676 120.147532912013
677 120.283145949744
678 120.559245046597
679 120.487388323381
680 120.620413062284
681 120.487859048087
682 120.539772189782
683 120.592579889344
684 120.445904295339
685 120.43082955938
686 120.360098683372
687 120.450830457085
688 120.319556179016
689 120.249279050221
690 120.378147448015
691 120.307174526316
692 120.398651809282
693 120.227985898982
694 120.058170485595
695 119.903433569691
696 119.812621796142
697 119.684448003228
698 119.814451441285
699 119.723492174596
700 119.561328571429
701 119.778120109646
702 119.907955292571
703 119.959790291152
704 119.806818181818
705 119.665562094462
706 119.499273728222
707 119.358798357104
708 119.193431006416
709 119.025656430221
710 118.858379289823
711 118.947497730065
712 118.859550561798
713 118.819247436409
714 118.680988081507
715 118.523867181769
716 118.375573483974
717 118.636461779964
718 118.49984093854
719 118.587057824478
720 118.797760416667
721 118.968796228077
722 119.225220800945
723 119.353665092237
724 119.442177818137
725 119.319621403092
726 119.369269327384
727 119.233438467899
728 119.169648517691
729 119.221113914809
730 119.08538937887
731 118.964033677607
732 118.801845382066
733 118.656536798632
734 118.497954547142
735 118.339784349114
736 118.206514354915
737 118.33090990097
738 118.348831530321
739 118.436353848323
740 118.397596785975
741 118.314281499451
742 118.302584259051
743 118.553276973602
744 118.678481616372
745 118.764945723166
746 118.88806251752
747 118.876226010692
748 118.926470588235
749 119.176022859139
750 119.020024888889
751 119.269969379487
752 119.127615366116
753 118.969734166477
754 119.217916470249
755 119.464367352309
756 119.664287114023
757 119.532207542461
758 119.429041847383
759 119.280417163559
760 119.177768351801
761 119.192303508248
762 119.432900365801
763 119.519108712419
764 119.400783969738
765 119.420592079969
766 119.339841433236
767 119.28015312202
768 119.141423543294
769 119.220858325118
770 119.459438353854
771 119.510356788983
772 119.364229576633
773 119.609418120292
774 119.852941863804
775 120.094811238293
776 120.082447656499
777 119.953989289897
778 120.108188883235
779 119.969896562691
780 119.831918145957
781 120.026142655022
782 120.014337949124
783 120.298511309125
784 120.313409451791
785 120.160486835166
786 120.441621506128
787 120.296249893036
788 120.234191811178
789 120.084443737641
790 120.131799391123
791 119.987555959027
792 119.92620650954
793 119.970018239673
794 119.959489940295
795 119.878175705075
796 119.953970543673
797 119.872379012262
798 119.736686641416
799 119.785207103372
800 119.895
801 119.833298264809
802 119.698187511272
803 119.54959375567
804 119.403286737952
805 119.674293430037
806 119.791347770136
807 119.649915777222
808 119.502009606901
809 119.551033567055
810 119.403629019966
811 119.445494366152
812 119.669768739838
813 119.522727994808
814 119.467610731124
815 119.702149121156
816 119.782293769223
817 119.638088417936
818 119.505958536833
819 119.524215033373
820 119.378875669244
821 119.369845454505
822 119.360505798569
823 119.251362336472
824 119.131939862381
825 119.025004958678
826 118.991334885003
827 118.979470091194
828 118.838285315877
829 118.719977416947
830 118.6649862099
831 118.590580701777
832 118.701639931583
833 118.566736178265
834 118.427215062482
835 118.651060991789
836 118.759064295689
837 118.905866517074
838 118.771634930309
839 118.955445284343
840 118.999994331066
841 118.965734976622
842 119.009530808334
843 118.868682007573
844 119.091226612161
845 119.03573964497
846 118.961582527148
847 118.85681947118
848 118.731043075828
849 118.605529126624
850 118.480276816609
851 118.623677680644
852 118.55038131764
853 118.693064544281
854 118.708521409101
855 118.619376902295
856 118.726630055027
857 118.86818826086
858 118.941317423835
859 118.826729513295
860 119.003757436452
861 118.865886707648
862 118.742550912194
863 118.918512988591
864 118.781099965706
865 118.693475892947
866 118.605939548454
867 118.574101789437
868 118.444493406103
869 118.390520696911
870 118.268131853613
871 118.180048270526
872 118.044744497517
873 117.974894014005
874 117.903345569176
875 117.852212244898
876 118.029940993724
877 117.958917164741
878 117.950908307865
879 117.966320710395
880 117.855655991736
881 117.786237649148
882 117.73402543179
883 117.647969895689
884 117.638454372351
885 117.505753774458
886 117.611958532273
887 117.685767995434
888 117.553546790033
889 117.50292602626
890 117.377987627825
891 117.394251783328
892 117.28555269561
893 117.497477581638
894 117.634425876712
895 117.516811585157
896 117.726481534997
897 117.7416235712
898 117.623767987262
899 117.50694196122
900 117.551111111111
901 117.652666109059
902 117.669947542047
903 117.879827178753
904 117.829586058031
905 117.821255761424
906 117.863587854334
907 117.73390352386
908 117.940745745114
909 118.148680654644
910 118.051926095882
911 118.04360897001
912 118.085449369037
913 118.290840950179
914 118.282553423766
915 118.199174654364
916 118.070383859957
917 118.171416203565
918 118.18639079936
919 118.10341940961
920 118.094120982987
921 118.134306170063
922 118.019053411192
923 118.058767808115
924 117.953575645134
925 118.11775485756
926 118.021856005299
927 118.191070009275
928 118.097112812128
929 118.014520746987
930 117.98310671754
931 118.151393521397
932 118.02482547109
933 117.905365834606
934 117.785487805437
935 117.882153907747
936 117.875666593615
937 117.810931821055
938 117.685539709312
939 117.567185084624
940 117.539197600724
941 117.532396516695
942 117.629459838353
943 117.643550906439
944 117.519129515585
945 117.488878810784
946 117.527714332965
947 117.464340790514
948 117.41514447471
949 117.291597499892
950 117.362309141274
951 117.457479591464
952 117.652159760963
953 117.57447900208
954 117.451613860211
955 117.334649817713
956 117.57354125453
957 117.811178480295
958 117.876438823053
959 117.97017444092
960 117.849930555556
961 117.73428000013
962 117.642299263921
963 117.520503704566
964 117.55808590589
965 117.448182770007
966 117.348581587645
967 117.286985516887
968 117.165988277782
969 117.121288317619
970 117.185763630566
971 117.084967666871
972 117.283140908398
973 117.253524503844
974 117.295950356075
975 117.207620249836
976 117.143875134372
977 117.043876089674
978 116.956147724374
979 117.053802998828
980 117.118163265306
981 117.057374519541
982 117.252500197029
983 117.320085398882
984 117.412948641682
985 117.35172769203
986 117.392370468506
987 117.316730464632
988 117.271626932092
989 117.287225200154
990 117.199429650036
991 117.083409616926
992 117.037977204735
993 116.993110890027
994 116.917748948419
995 116.890549228555
996 116.864199770971
997 117.020310681292
998 117.085501664652
999 117.124874624374
1000 117.139879
};
\addlegendentry{$v_{v}$ (valor de la varianza de las tiradas)}
\addplot [semithick, color1, dashed]
table {%
1 114
2 114
3 114
4 114
5 114
6 114
7 114
8 114
9 114
10 114
11 114
12 114
13 114
14 114
15 114
16 114
17 114
18 114
19 114
20 114
21 114
22 114
23 114
24 114
25 114
26 114
27 114
28 114
29 114
30 114
31 114
32 114
33 114
34 114
35 114
36 114
37 114
38 114
39 114
40 114
41 114
42 114
43 114
44 114
45 114
46 114
47 114
48 114
49 114
50 114
51 114
52 114
53 114
54 114
55 114
56 114
57 114
58 114
59 114
60 114
61 114
62 114
63 114
64 114
65 114
66 114
67 114
68 114
69 114
70 114
71 114
72 114
73 114
74 114
75 114
76 114
77 114
78 114
79 114
80 114
81 114
82 114
83 114
84 114
85 114
86 114
87 114
88 114
89 114
90 114
91 114
92 114
93 114
94 114
95 114
96 114
97 114
98 114
99 114
100 114
101 114
102 114
103 114
104 114
105 114
106 114
107 114
108 114
109 114
110 114
111 114
112 114
113 114
114 114
115 114
116 114
117 114
118 114
119 114
120 114
121 114
122 114
123 114
124 114
125 114
126 114
127 114
128 114
129 114
130 114
131 114
132 114
133 114
134 114
135 114
136 114
137 114
138 114
139 114
140 114
141 114
142 114
143 114
144 114
145 114
146 114
147 114
148 114
149 114
150 114
151 114
152 114
153 114
154 114
155 114
156 114
157 114
158 114
159 114
160 114
161 114
162 114
163 114
164 114
165 114
166 114
167 114
168 114
169 114
170 114
171 114
172 114
173 114
174 114
175 114
176 114
177 114
178 114
179 114
180 114
181 114
182 114
183 114
184 114
185 114
186 114
187 114
188 114
189 114
190 114
191 114
192 114
193 114
194 114
195 114
196 114
197 114
198 114
199 114
200 114
201 114
202 114
203 114
204 114
205 114
206 114
207 114
208 114
209 114
210 114
211 114
212 114
213 114
214 114
215 114
216 114
217 114
218 114
219 114
220 114
221 114
222 114
223 114
224 114
225 114
226 114
227 114
228 114
229 114
230 114
231 114
232 114
233 114
234 114
235 114
236 114
237 114
238 114
239 114
240 114
241 114
242 114
243 114
244 114
245 114
246 114
247 114
248 114
249 114
250 114
251 114
252 114
253 114
254 114
255 114
256 114
257 114
258 114
259 114
260 114
261 114
262 114
263 114
264 114
265 114
266 114
267 114
268 114
269 114
270 114
271 114
272 114
273 114
274 114
275 114
276 114
277 114
278 114
279 114
280 114
281 114
282 114
283 114
284 114
285 114
286 114
287 114
288 114
289 114
290 114
291 114
292 114
293 114
294 114
295 114
296 114
297 114
298 114
299 114
300 114
301 114
302 114
303 114
304 114
305 114
306 114
307 114
308 114
309 114
310 114
311 114
312 114
313 114
314 114
315 114
316 114
317 114
318 114
319 114
320 114
321 114
322 114
323 114
324 114
325 114
326 114
327 114
328 114
329 114
330 114
331 114
332 114
333 114
334 114
335 114
336 114
337 114
338 114
339 114
340 114
341 114
342 114
343 114
344 114
345 114
346 114
347 114
348 114
349 114
350 114
351 114
352 114
353 114
354 114
355 114
356 114
357 114
358 114
359 114
360 114
361 114
362 114
363 114
364 114
365 114
366 114
367 114
368 114
369 114
370 114
371 114
372 114
373 114
374 114
375 114
376 114
377 114
378 114
379 114
380 114
381 114
382 114
383 114
384 114
385 114
386 114
387 114
388 114
389 114
390 114
391 114
392 114
393 114
394 114
395 114
396 114
397 114
398 114
399 114
400 114
401 114
402 114
403 114
404 114
405 114
406 114
407 114
408 114
409 114
410 114
411 114
412 114
413 114
414 114
415 114
416 114
417 114
418 114
419 114
420 114
421 114
422 114
423 114
424 114
425 114
426 114
427 114
428 114
429 114
430 114
431 114
432 114
433 114
434 114
435 114
436 114
437 114
438 114
439 114
440 114
441 114
442 114
443 114
444 114
445 114
446 114
447 114
448 114
449 114
450 114
451 114
452 114
453 114
454 114
455 114
456 114
457 114
458 114
459 114
460 114
461 114
462 114
463 114
464 114
465 114
466 114
467 114
468 114
469 114
470 114
471 114
472 114
473 114
474 114
475 114
476 114
477 114
478 114
479 114
480 114
481 114
482 114
483 114
484 114
485 114
486 114
487 114
488 114
489 114
490 114
491 114
492 114
493 114
494 114
495 114
496 114
497 114
498 114
499 114
500 114
501 114
502 114
503 114
504 114
505 114
506 114
507 114
508 114
509 114
510 114
511 114
512 114
513 114
514 114
515 114
516 114
517 114
518 114
519 114
520 114
521 114
522 114
523 114
524 114
525 114
526 114
527 114
528 114
529 114
530 114
531 114
532 114
533 114
534 114
535 114
536 114
537 114
538 114
539 114
540 114
541 114
542 114
543 114
544 114
545 114
546 114
547 114
548 114
549 114
550 114
551 114
552 114
553 114
554 114
555 114
556 114
557 114
558 114
559 114
560 114
561 114
562 114
563 114
564 114
565 114
566 114
567 114
568 114
569 114
570 114
571 114
572 114
573 114
574 114
575 114
576 114
577 114
578 114
579 114
580 114
581 114
582 114
583 114
584 114
585 114
586 114
587 114
588 114
589 114
590 114
591 114
592 114
593 114
594 114
595 114
596 114
597 114
598 114
599 114
600 114
601 114
602 114
603 114
604 114
605 114
606 114
607 114
608 114
609 114
610 114
611 114
612 114
613 114
614 114
615 114
616 114
617 114
618 114
619 114
620 114
621 114
622 114
623 114
624 114
625 114
626 114
627 114
628 114
629 114
630 114
631 114
632 114
633 114
634 114
635 114
636 114
637 114
638 114
639 114
640 114
641 114
642 114
643 114
644 114
645 114
646 114
647 114
648 114
649 114
650 114
651 114
652 114
653 114
654 114
655 114
656 114
657 114
658 114
659 114
660 114
661 114
662 114
663 114
664 114
665 114
666 114
667 114
668 114
669 114
670 114
671 114
672 114
673 114
674 114
675 114
676 114
677 114
678 114
679 114
680 114
681 114
682 114
683 114
684 114
685 114
686 114
687 114
688 114
689 114
690 114
691 114
692 114
693 114
694 114
695 114
696 114
697 114
698 114
699 114
700 114
701 114
702 114
703 114
704 114
705 114
706 114
707 114
708 114
709 114
710 114
711 114
712 114
713 114
714 114
715 114
716 114
717 114
718 114
719 114
720 114
721 114
722 114
723 114
724 114
725 114
726 114
727 114
728 114
729 114
730 114
731 114
732 114
733 114
734 114
735 114
736 114
737 114
738 114
739 114
740 114
741 114
742 114
743 114
744 114
745 114
746 114
747 114
748 114
749 114
750 114
751 114
752 114
753 114
754 114
755 114
756 114
757 114
758 114
759 114
760 114
761 114
762 114
763 114
764 114
765 114
766 114
767 114
768 114
769 114
770 114
771 114
772 114
773 114
774 114
775 114
776 114
777 114
778 114
779 114
780 114
781 114
782 114
783 114
784 114
785 114
786 114
787 114
788 114
789 114
790 114
791 114
792 114
793 114
794 114
795 114
796 114
797 114
798 114
799 114
800 114
801 114
802 114
803 114
804 114
805 114
806 114
807 114
808 114
809 114
810 114
811 114
812 114
813 114
814 114
815 114
816 114
817 114
818 114
819 114
820 114
821 114
822 114
823 114
824 114
825 114
826 114
827 114
828 114
829 114
830 114
831 114
832 114
833 114
834 114
835 114
836 114
837 114
838 114
839 114
840 114
841 114
842 114
843 114
844 114
845 114
846 114
847 114
848 114
849 114
850 114
851 114
852 114
853 114
854 114
855 114
856 114
857 114
858 114
859 114
860 114
861 114
862 114
863 114
864 114
865 114
866 114
867 114
868 114
869 114
870 114
871 114
872 114
873 114
874 114
875 114
876 114
877 114
878 114
879 114
880 114
881 114
882 114
883 114
884 114
885 114
886 114
887 114
888 114
889 114
890 114
891 114
892 114
893 114
894 114
895 114
896 114
897 114
898 114
899 114
900 114
901 114
902 114
903 114
904 114
905 114
906 114
907 114
908 114
909 114
910 114
911 114
912 114
913 114
914 114
915 114
916 114
917 114
918 114
919 114
920 114
921 114
922 114
923 114
924 114
925 114
926 114
927 114
928 114
929 114
930 114
931 114
932 114
933 114
934 114
935 114
936 114
937 114
938 114
939 114
940 114
941 114
942 114
943 114
944 114
945 114
946 114
947 114
948 114
949 114
950 114
951 114
952 114
953 114
954 114
955 114
956 114
957 114
958 114
959 114
960 114
961 114
962 114
963 114
964 114
965 114
966 114
967 114
968 114
969 114
970 114
971 114
972 114
973 114
974 114
975 114
976 114
977 114
978 114
979 114
980 114
981 114
982 114
983 114
984 114
985 114
986 114
987 114
988 114
989 114
990 114
991 114
992 114
993 114
994 114
995 114
996 114
997 114
998 114
999 114
1000 114
};
\addlegendentry{$v_{v_{e}}$ (valor de la varianza esperada)}
\end{axis}

\end{tikzpicture}

    \caption{valor de la varianza con respecto al número de tiradas}
  \end{mytikzresize}
\end{figure}

\begin{figure}[!htbp]
  \begin{mytikzresize}{0.6\textwidth}
    \centering
    % This file was created by tikzplotlib v0.9.1.
\begin{tikzpicture}

\definecolor{color0}{rgb}{0.12156862745098,0.466666666666667,0.705882352941177}
\definecolor{color1}{rgb}{1,0.498039215686275,0.0549019607843137}
\definecolor{color2}{rgb}{0.172549019607843,0.627450980392157,0.172549019607843}
\definecolor{color3}{rgb}{0.83921568627451,0.152941176470588,0.156862745098039}
\definecolor{color4}{rgb}{0.580392156862745,0.403921568627451,0.741176470588235}
\definecolor{color5}{rgb}{0.549019607843137,0.337254901960784,0.294117647058824}
\definecolor{color6}{rgb}{0.890196078431372,0.466666666666667,0.76078431372549}
\definecolor{color7}{rgb}{0.737254901960784,0.741176470588235,0.133333333333333}
\definecolor{color8}{rgb}{0.0901960784313725,0.745098039215686,0.811764705882353}

\begin{axis}[
legend cell align={left},
legend style={fill opacity=0.5, draw opacity=1, text opacity=1, draw=white!80!black},
scaled ticks=false,
tick align=outside,
tick pos=left,
width=\figW,
x grid style={white!69.0196078431373!black},
xlabel={\(\displaystyle n\) (número de tiradas)},
xmajorgrids,
xmin=-48.95, xmax=1049.95,
xtick style={color=black},
xticklabel style={/pgf/number format/.cd,fixed,precision=2},
y grid style={white!69.0196078431373!black},
ylabel={\(\displaystyle f_{r}\) (frecuencia relativa)},
ymajorgrids,
ymin=-0.0108695652173913, ymax=0.228260869565217,
ytick style={color=black},
yticklabel style={/pgf/number format/.cd,fixed,precision=2}
]
\addplot [semithick, color0, forget plot]
table {%
1 0
2 0
3 0
4 0
5 0
6 0
7 0
8 0
9 0
10 0
11 0
12 0
13 0
14 0
15 0
16 0
17 0
18 0
19 0
20 0
21 0
22 0
23 0
24 0
25 0
26 0
27 0
28 0
29 0
30 0
31 0
32 0
33 0
34 0
35 0
36 0
37 0
38 0
39 0
40 0
41 0
42 0
43 0
44 0
45 0
46 0
47 0
48 0
49 0
50 0
51 0
52 0
53 0
54 0
55 0
56 0
57 0
58 0.0172413793103448
59 0.0169491525423729
60 0.0166666666666667
61 0.0163934426229508
62 0.0161290322580645
63 0.0158730158730159
64 0.015625
65 0.0153846153846154
66 0.0303030303030303
67 0.0298507462686567
68 0.0294117647058824
69 0.0289855072463768
70 0.0285714285714286
71 0.028169014084507
72 0.0277777777777778
73 0.0273972602739726
74 0.027027027027027
75 0.0266666666666667
76 0.0263157894736842
77 0.025974025974026
78 0.0256410256410256
79 0.0379746835443038
80 0.0375
81 0.037037037037037
82 0.0365853658536585
83 0.036144578313253
84 0.0357142857142857
85 0.0352941176470588
86 0.0348837209302326
87 0.0344827586206897
88 0.0340909090909091
89 0.0337078651685393
90 0.0333333333333333
91 0.032967032967033
92 0.0326086956521739
93 0.032258064516129
94 0.0319148936170213
95 0.0315789473684211
96 0.03125
97 0.0309278350515464
98 0.0306122448979592
99 0.0303030303030303
100 0.03
101 0.0297029702970297
102 0.0294117647058824
103 0.029126213592233
104 0.0288461538461538
105 0.0285714285714286
106 0.0283018867924528
107 0.0280373831775701
108 0.0277777777777778
109 0.0275229357798165
110 0.0272727272727273
111 0.027027027027027
112 0.0267857142857143
113 0.0265486725663717
114 0.0263157894736842
115 0.0260869565217391
116 0.0258620689655172
117 0.0256410256410256
118 0.0254237288135593
119 0.0252100840336134
120 0.025
121 0.0247933884297521
122 0.0245901639344262
123 0.024390243902439
124 0.0241935483870968
125 0.024
126 0.0238095238095238
127 0.0236220472440945
128 0.0234375
129 0.0232558139534884
130 0.0230769230769231
131 0.0229007633587786
132 0.0227272727272727
133 0.0225563909774436
134 0.0223880597014925
135 0.0222222222222222
136 0.0220588235294118
137 0.0218978102189781
138 0.0217391304347826
139 0.0215827338129496
140 0.0214285714285714
141 0.0283687943262411
142 0.028169014084507
143 0.027972027972028
144 0.0277777777777778
145 0.0275862068965517
146 0.0273972602739726
147 0.0272108843537415
148 0.027027027027027
149 0.0268456375838926
150 0.0266666666666667
151 0.0264900662251656
152 0.0263157894736842
153 0.0261437908496732
154 0.025974025974026
155 0.0258064516129032
156 0.0256410256410256
157 0.0254777070063694
158 0.0253164556962025
159 0.0251572327044025
160 0.025
161 0.0248447204968944
162 0.0246913580246914
163 0.0245398773006135
164 0.024390243902439
165 0.0242424242424242
166 0.0240963855421687
167 0.0239520958083832
168 0.0238095238095238
169 0.0236686390532544
170 0.0235294117647059
171 0.0233918128654971
172 0.0232558139534884
173 0.023121387283237
174 0.0229885057471264
175 0.0228571428571429
176 0.0227272727272727
177 0.0225988700564972
178 0.0224719101123595
179 0.0223463687150838
180 0.0222222222222222
181 0.0220994475138122
182 0.021978021978022
183 0.0218579234972678
184 0.0217391304347826
185 0.0216216216216216
186 0.021505376344086
187 0.0213903743315508
188 0.0212765957446809
189 0.0211640211640212
190 0.0210526315789474
191 0.0209424083769634
192 0.0208333333333333
193 0.0207253886010363
194 0.0206185567010309
195 0.0205128205128205
196 0.0204081632653061
197 0.0203045685279188
198 0.0202020202020202
199 0.0201005025125628
200 0.02
201 0.0199004975124378
202 0.0198019801980198
203 0.0197044334975369
204 0.0196078431372549
205 0.024390243902439
206 0.0242718446601942
207 0.0241545893719807
208 0.0240384615384615
209 0.0239234449760766
210 0.0238095238095238
211 0.023696682464455
212 0.0235849056603774
213 0.0234741784037559
214 0.0233644859813084
215 0.0232558139534884
216 0.0231481481481481
217 0.0230414746543779
218 0.0229357798165138
219 0.0228310502283105
220 0.0227272727272727
221 0.0226244343891403
222 0.0225225225225225
223 0.0224215246636771
224 0.0223214285714286
225 0.0222222222222222
226 0.0221238938053097
227 0.0220264317180617
228 0.0219298245614035
229 0.0218340611353712
230 0.0217391304347826
231 0.0216450216450216
232 0.021551724137931
233 0.0214592274678112
234 0.0213675213675214
235 0.0212765957446809
236 0.0211864406779661
237 0.0210970464135021
238 0.0210084033613445
239 0.0209205020920502
240 0.0208333333333333
241 0.020746887966805
242 0.0206611570247934
243 0.0205761316872428
244 0.0204918032786885
245 0.0204081632653061
246 0.0203252032520325
247 0.0202429149797571
248 0.0201612903225806
249 0.0200803212851406
250 0.02
251 0.0199203187250996
252 0.0198412698412698
253 0.0197628458498024
254 0.0196850393700787
255 0.0196078431372549
256 0.01953125
257 0.0194552529182879
258 0.0193798449612403
259 0.0193050193050193
260 0.0192307692307692
261 0.0191570881226054
262 0.0190839694656489
263 0.0190114068441065
264 0.0189393939393939
265 0.0188679245283019
266 0.018796992481203
267 0.0187265917602996
268 0.0186567164179104
269 0.0185873605947955
270 0.0222222222222222
271 0.022140221402214
272 0.0220588235294118
273 0.021978021978022
274 0.0218978102189781
275 0.0218181818181818
276 0.0217391304347826
277 0.0216606498194946
278 0.0215827338129496
279 0.021505376344086
280 0.0214285714285714
281 0.0213523131672598
282 0.0212765957446809
283 0.0212014134275618
284 0.0211267605633803
285 0.0210526315789474
286 0.020979020979021
287 0.0209059233449477
288 0.0208333333333333
289 0.0207612456747405
290 0.0206896551724138
291 0.0206185567010309
292 0.0205479452054795
293 0.0204778156996587
294 0.0204081632653061
295 0.0203389830508475
296 0.0202702702702703
297 0.0202020202020202
298 0.0201342281879195
299 0.020066889632107
300 0.02
301 0.0199335548172757
302 0.0198675496688742
303 0.0198019801980198
304 0.0197368421052632
305 0.019672131147541
306 0.0196078431372549
307 0.0195439739413681
308 0.0194805194805195
309 0.0194174757281553
310 0.0193548387096774
311 0.0192926045016077
312 0.0192307692307692
313 0.0191693290734824
314 0.0191082802547771
315 0.019047619047619
316 0.0189873417721519
317 0.0189274447949527
318 0.0188679245283019
319 0.0188087774294671
320 0.01875
321 0.0186915887850467
322 0.0186335403726708
323 0.0185758513931889
324 0.0185185185185185
325 0.0184615384615385
326 0.0184049079754601
327 0.018348623853211
328 0.0182926829268293
329 0.0182370820668693
330 0.0181818181818182
331 0.0181268882175227
332 0.0180722891566265
333 0.018018018018018
334 0.0179640718562874
335 0.017910447761194
336 0.0178571428571429
337 0.0178041543026706
338 0.0177514792899408
339 0.0176991150442478
340 0.0176470588235294
341 0.0175953079178886
342 0.0204678362573099
343 0.0204081632653061
344 0.0203488372093023
345 0.0202898550724638
346 0.0202312138728324
347 0.0201729106628242
348 0.0201149425287356
349 0.0200573065902579
350 0.02
351 0.0199430199430199
352 0.0227272727272727
353 0.0226628895184136
354 0.0225988700564972
355 0.0225352112676056
356 0.0224719101123595
357 0.0224089635854342
358 0.0223463687150838
359 0.0222841225626741
360 0.0222222222222222
361 0.0221606648199446
362 0.0220994475138122
363 0.0220385674931129
364 0.021978021978022
365 0.0219178082191781
366 0.0218579234972678
367 0.0245231607629428
368 0.0244565217391304
369 0.024390243902439
370 0.0243243243243243
371 0.0242587601078167
372 0.0241935483870968
373 0.0241286863270777
374 0.0240641711229947
375 0.024
376 0.023936170212766
377 0.0238726790450928
378 0.0238095238095238
379 0.0237467018469657
380 0.0236842105263158
381 0.0236220472440945
382 0.0235602094240838
383 0.0234986945169713
384 0.0234375
385 0.0233766233766234
386 0.0233160621761658
387 0.0232558139534884
388 0.0231958762886598
389 0.0231362467866324
390 0.0230769230769231
391 0.0230179028132992
392 0.0229591836734694
393 0.0229007633587786
394 0.0228426395939086
395 0.0227848101265823
396 0.0227272727272727
397 0.0226700251889169
398 0.0226130653266332
399 0.0225563909774436
400 0.0225
401 0.0224438902743142
402 0.0223880597014925
403 0.0223325062034739
404 0.0222772277227723
405 0.0222222222222222
406 0.0221674876847291
407 0.0221130221130221
408 0.0220588235294118
409 0.0220048899755501
410 0.0219512195121951
411 0.0218978102189781
412 0.0218446601941748
413 0.0217917675544794
414 0.0217391304347826
415 0.0216867469879518
416 0.0216346153846154
417 0.0215827338129496
418 0.0215311004784689
419 0.0238663484486874
420 0.0238095238095238
421 0.0237529691211401
422 0.023696682464455
423 0.0236406619385343
424 0.0259433962264151
425 0.0258823529411765
426 0.0258215962441315
427 0.0257611241217799
428 0.0257009345794393
429 0.027972027972028
430 0.027906976744186
431 0.0278422273781903
432 0.0277777777777778
433 0.0277136258660508
434 0.0276497695852535
435 0.0275862068965517
436 0.0275229357798165
437 0.0274599542334096
438 0.0273972602739726
439 0.0273348519362187
440 0.0295454545454545
441 0.0294784580498866
442 0.0294117647058824
443 0.0316027088036117
444 0.0315315315315315
445 0.0314606741573034
446 0.031390134529148
447 0.0313199105145414
448 0.03125
449 0.0311804008908686
450 0.0311111111111111
451 0.0310421286031042
452 0.0309734513274336
453 0.0309050772626932
454 0.0308370044052863
455 0.0307692307692308
456 0.0307017543859649
457 0.0306345733041575
458 0.0305676855895196
459 0.0305010893246187
460 0.0304347826086957
461 0.0303687635574837
462 0.0303030303030303
463 0.0302375809935205
464 0.0301724137931034
465 0.0301075268817204
466 0.0300429184549356
467 0.0299785867237687
468 0.0299145299145299
469 0.0298507462686567
470 0.0297872340425532
471 0.029723991507431
472 0.0296610169491525
473 0.0295983086680761
474 0.029535864978903
475 0.0294736842105263
476 0.0294117647058824
477 0.0293501048218029
478 0.0292887029288703
479 0.0292275574112735
480 0.0291666666666667
481 0.0311850311850312
482 0.0311203319502075
483 0.031055900621118
484 0.0309917355371901
485 0.0309278350515464
486 0.0308641975308642
487 0.0308008213552361
488 0.0307377049180328
489 0.0306748466257669
490 0.0306122448979592
491 0.0325865580448065
492 0.032520325203252
493 0.0324543610547667
494 0.0323886639676113
495 0.0323232323232323
496 0.032258064516129
497 0.0321931589537223
498 0.0321285140562249
499 0.032064128256513
500 0.032
501 0.031936127744511
502 0.0318725099601594
503 0.0318091451292246
504 0.0317460317460317
505 0.0316831683168317
506 0.0316205533596838
507 0.0315581854043393
508 0.031496062992126
509 0.0333988212180747
510 0.0333333333333333
511 0.0332681017612524
512 0.033203125
513 0.0331384015594542
514 0.0330739299610895
515 0.0330097087378641
516 0.0329457364341085
517 0.0328820116054159
518 0.0328185328185328
519 0.0327552986512524
520 0.0326923076923077
521 0.0326295585412668
522 0.0325670498084291
523 0.0325047801147228
524 0.0324427480916031
525 0.0323809523809524
526 0.032319391634981
527 0.032258064516129
528 0.0340909090909091
529 0.0340264650283554
530 0.0339622641509434
531 0.0338983050847458
532 0.0338345864661654
533 0.0337711069418387
534 0.0337078651685393
535 0.0336448598130841
536 0.0335820895522388
537 0.0335195530726257
538 0.033457249070632
539 0.0333951762523191
540 0.0333333333333333
541 0.033271719038817
542 0.033210332103321
543 0.0331491712707182
544 0.0330882352941176
545 0.0330275229357798
546 0.032967032967033
547 0.0329067641681901
548 0.0328467153284672
549 0.0327868852459016
550 0.0327272727272727
551 0.0326678765880218
552 0.0326086956521739
553 0.0325497287522604
554 0.0324909747292419
555 0.0324324324324324
556 0.0323741007194245
557 0.0323159784560144
558 0.032258064516129
559 0.0322003577817531
560 0.0321428571428571
561 0.0320855614973262
562 0.0320284697508897
563 0.0319715808170515
564 0.0319148936170213
565 0.031858407079646
566 0.0318021201413428
567 0.0317460317460317
568 0.0316901408450704
569 0.031634446397188
570 0.0315789473684211
571 0.031523642732049
572 0.0314685314685315
573 0.031413612565445
574 0.0313588850174216
575 0.031304347826087
576 0.03125
577 0.0311958405545927
578 0.0311418685121107
579 0.0310880829015544
580 0.0310344827586207
581 0.0309810671256454
582 0.0309278350515464
583 0.0308747855917667
584 0.0308219178082192
585 0.0307692307692308
586 0.0307167235494881
587 0.030664395229983
588 0.0306122448979592
589 0.0305602716468591
590 0.0305084745762712
591 0.0304568527918782
592 0.0304054054054054
593 0.03035413153457
594 0.0303030303030303
595 0.0302521008403361
596 0.0302013422818792
597 0.0301507537688442
598 0.0301003344481605
599 0.0300500834724541
600 0.03
601 0.0299500831946755
602 0.0299003322259136
603 0.0298507462686567
604 0.0298013245033113
605 0.0297520661157025
606 0.0297029702970297
607 0.0296540362438221
608 0.0296052631578947
609 0.0295566502463054
610 0.0295081967213115
611 0.0294599018003273
612 0.0294117647058824
613 0.0293637846655791
614 0.0293159609120521
615 0.0292682926829268
616 0.0292207792207792
617 0.0291734197730956
618 0.029126213592233
619 0.0290791599353796
620 0.0290322580645161
621 0.0289855072463768
622 0.0289389067524116
623 0.028892455858748
624 0.0288461538461538
625 0.0304
626 0.0303514376996805
627 0.0303030303030303
628 0.0302547770700637
629 0.0302066772655008
630 0.0301587301587302
631 0.0301109350237718
632 0.0300632911392405
633 0.0300157977883096
634 0.0299684542586751
635 0.0299212598425197
636 0.029874213836478
637 0.0298273155416013
638 0.0297805642633229
639 0.0297339593114241
640 0.0296875
641 0.0296411856474259
642 0.029595015576324
643 0.0295489891135303
644 0.031055900621118
645 0.0310077519379845
646 0.0309597523219814
647 0.0309119010819165
648 0.0308641975308642
649 0.0308166409861325
650 0.0307692307692308
651 0.0307219662058372
652 0.0306748466257669
653 0.0306278713629403
654 0.0305810397553517
655 0.0305343511450382
656 0.0304878048780488
657 0.030441400304414
658 0.0303951367781155
659 0.0303490136570561
660 0.0303030303030303
661 0.0302571860816944
662 0.0302114803625378
663 0.0301659125188537
664 0.0301204819277108
665 0.0300751879699248
666 0.03003003003003
667 0.0314842578710645
668 0.031437125748503
669 0.031390134529148
670 0.0313432835820895
671 0.0312965722801788
672 0.03125
673 0.0312035661218425
674 0.0311572700296736
675 0.0325925925925926
676 0.0325443786982249
677 0.0324963072378139
678 0.0324483775811209
679 0.03240058910162
680 0.0323529411764706
681 0.0323054331864905
682 0.032258064516129
683 0.0322108345534407
684 0.0321637426900585
685 0.0321167883211679
686 0.032069970845481
687 0.0320232896652111
688 0.0319767441860465
689 0.0319303338171263
690 0.0318840579710145
691 0.0318379160636758
692 0.0317919075144509
693 0.0317460317460317
694 0.031700288184438
695 0.0316546762589928
696 0.0316091954022989
697 0.0315638450502152
698 0.0315186246418338
699 0.0314735336194564
700 0.0314285714285714
701 0.0313837375178317
702 0.0313390313390313
703 0.0312944523470839
704 0.03125
705 0.0312056737588652
706 0.0311614730878187
707 0.0311173974540311
708 0.0310734463276836
709 0.0324400564174894
710 0.0323943661971831
711 0.0323488045007032
712 0.0323033707865169
713 0.032258064516129
714 0.0322128851540616
715 0.0321678321678322
716 0.032122905027933
717 0.0320781032078103
718 0.032033426183844
719 0.0319888734353268
720 0.0319444444444444
721 0.0319001386962552
722 0.0318559556786704
723 0.0318118948824343
724 0.031767955801105
725 0.0317241379310345
726 0.0316804407713499
727 0.031636863823934
728 0.0315934065934066
729 0.0315500685871056
730 0.0315068493150685
731 0.0314637482900137
732 0.0314207650273224
733 0.0313778990450205
734 0.0313351498637602
735 0.0312925170068027
736 0.03125
737 0.0312075983717775
738 0.0311653116531165
739 0.0311231393775372
740 0.0310810810810811
741 0.0310391363022942
742 0.0309973045822102
743 0.0309555854643338
744 0.0309139784946237
745 0.0308724832214765
746 0.0308310991957105
747 0.0307898259705489
748 0.0307486631016043
749 0.0307076101468625
750 0.0306666666666667
751 0.0306258322237017
752 0.0305851063829787
753 0.0305444887118194
754 0.0305039787798408
755 0.0304635761589404
756 0.0304232804232804
757 0.0303830911492734
758 0.0303430079155673
759 0.0303030303030303
760 0.0302631578947368
761 0.0302233902759527
762 0.0301837270341207
763 0.0301441677588467
764 0.0301047120418848
765 0.0300653594771242
766 0.0300261096605744
767 0.029986962190352
768 0.0299479166666667
769 0.0299089726918075
770 0.0298701298701299
771 0.0298313878080415
772 0.0297927461139896
773 0.0297542043984476
774 0.0297157622739018
775 0.0296774193548387
776 0.029639175257732
777 0.0296010296010296
778 0.0295629820051414
779 0.0295250320924262
780 0.0294871794871795
781 0.029449423815621
782 0.0294117647058824
783 0.0293742017879949
784 0.0293367346938776
785 0.0292993630573248
786 0.0292620865139949
787 0.0292249047013977
788 0.0291878172588832
789 0.0291508238276299
790 0.0291139240506329
791 0.0290771175726928
792 0.029040404040404
793 0.0290037831021438
794 0.0289672544080605
795 0.0289308176100629
796 0.028894472361809
797 0.0288582183186951
798 0.0288220551378446
799 0.0287859824780976
800 0.02875
801 0.0287141073657928
802 0.0286783042394015
803 0.0286425902864259
804 0.0286069651741294
805 0.0285714285714286
806 0.0285359801488834
807 0.0285006195786865
808 0.0297029702970297
809 0.0296662546353523
810 0.0308641975308642
811 0.030826140567201
812 0.0307881773399015
813 0.031980319803198
814 0.0319410319410319
815 0.0319018404907975
816 0.0318627450980392
817 0.0318237454100367
818 0.0317848410757946
819 0.0317460317460317
820 0.0317073170731707
821 0.0316686967113276
822 0.0316301703163017
823 0.031591737545565
824 0.0315533980582524
825 0.0315151515151515
826 0.0314769975786925
827 0.0314389359129383
828 0.0314009661835749
829 0.0313630880579011
830 0.0313253012048193
831 0.0312876052948255
832 0.03125
833 0.0312124849939976
834 0.0311750599520384
835 0.0311377245508982
836 0.0311004784688995
837 0.031063321385902
838 0.0310262529832936
839 0.0309892729439809
840 0.030952380952381
841 0.0309155766944114
842 0.0308788598574822
843 0.0308422301304864
844 0.0308056872037915
845 0.0307692307692308
846 0.0307328605200946
847 0.0306965761511216
848 0.0306603773584906
849 0.0306242638398115
850 0.0305882352941176
851 0.0305522914218566
852 0.0305164319248826
853 0.0304806565064478
854 0.0304449648711944
855 0.0304093567251462
856 0.0303738317757009
857 0.0303383897316219
858 0.0303030303030303
859 0.030267753201397
860 0.0302325581395349
861 0.0313588850174216
862 0.031322505800464
863 0.0312862108922364
864 0.03125
865 0.0312138728323699
866 0.0311778290993072
867 0.0311418685121107
868 0.0311059907834101
869 0.0310701956271577
870 0.0310344827586207
871 0.0309988518943743
872 0.0309633027522936
873 0.0309278350515464
874 0.0308924485125858
875 0.0308571428571429
876 0.0308219178082192
877 0.0307867730900798
878 0.030751708428246
879 0.0307167235494881
880 0.0306818181818182
881 0.0306469920544835
882 0.0306122448979592
883 0.0305775764439411
884 0.0305429864253394
885 0.0305084745762712
886 0.0304740406320542
887 0.0304396843291995
888 0.0315315315315315
889 0.031496062992126
890 0.0314606741573034
891 0.0314253647586981
892 0.031390134529148
893 0.0313549832026876
894 0.0313199105145414
895 0.0312849162011173
896 0.03125
897 0.0312151616499443
898 0.0311804008908686
899 0.0311457174638487
900 0.0311111111111111
901 0.0310765815760266
902 0.0310421286031042
903 0.0310077519379845
904 0.0309734513274336
905 0.030939226519337
906 0.0309050772626932
907 0.0319735391400221
908 0.0319383259911894
909 0.0319031903190319
910 0.0318681318681319
911 0.0318331503841932
912 0.0317982456140351
913 0.031763417305586
914 0.0317286652078775
915 0.0316939890710383
916 0.0316593886462882
917 0.0316248636859324
918 0.0315904139433551
919 0.0315560391730141
920 0.0315217391304348
921 0.0314875135722041
922 0.0314533622559653
923 0.0314192849404117
924 0.0313852813852814
925 0.0313513513513514
926 0.031317494600432
927 0.0312837108953614
928 0.03125
929 0.031216361679225
930 0.0311827956989247
931 0.0311493018259936
932 0.0311158798283262
933 0.0310825294748124
934 0.0310492505353319
935 0.0310160427807487
936 0.030982905982906
937 0.0309498399146211
938 0.0309168443496802
939 0.0308839190628328
940 0.0308510638297872
941 0.0308182784272051
942 0.0307855626326964
943 0.0307529162248144
944 0.0307203389830508
945 0.0306878306878307
946 0.0306553911205074
947 0.030623020063358
948 0.0305907172995781
949 0.0305584826132771
950 0.0305263157894737
951 0.0304942166140904
952 0.0304621848739496
953 0.0304302203567681
954 0.0314465408805031
955 0.031413612565445
956 0.0313807531380753
957 0.0313479623824451
958 0.0313152400835073
959 0.0312825860271116
960 0.03125
961 0.0312174817898023
962 0.0311850311850312
963 0.0321910695742471
964 0.0321576763485477
965 0.0321243523316062
966 0.0320910973084886
967 0.0320579110651499
968 0.0320247933884297
969 0.0319917440660475
970 0.0319587628865979
971 0.0319258496395469
972 0.0318930041152263
973 0.0318602261048304
974 0.0318275154004107
975 0.0317948717948718
976 0.0317622950819672
977 0.0317297850562948
978 0.0316973415132924
979 0.0316649642492339
980 0.0316326530612245
981 0.0316004077471967
982 0.0315682281059063
983 0.0315361139369278
984 0.0315040650406504
985 0.0314720812182741
986 0.0314401622718053
987 0.0314083080040527
988 0.0313765182186235
989 0.0313447927199191
990 0.0313131313131313
991 0.0312815338042381
992 0.03125
993 0.0312185297079557
994 0.0311871227364185
995 0.0311557788944724
996 0.0311244979919679
997 0.0310932798395186
998 0.031062124248497
999 0.031031031031031
1000 0.031
};
\addplot [semithick, color1, forget plot]
table {%
1 0
2 0
3 0
4 0
5 0
6 0
7 0
8 0
9 0
10 0
11 0
12 0
13 0
14 0
15 0
16 0
17 0
18 0
19 0
20 0
21 0
22 0
23 0
24 0
25 0
26 0
27 0
28 0
29 0
30 0
31 0
32 0
33 0
34 0
35 0
36 0
37 0
38 0
39 0
40 0
41 0
42 0
43 0
44 0
45 0
46 0
47 0
48 0
49 0
50 0
51 0
52 0
53 0
54 0
55 0
56 0
57 0
58 0
59 0.0169491525423729
60 0.0166666666666667
61 0.0163934426229508
62 0.0161290322580645
63 0.0158730158730159
64 0.015625
65 0.0153846153846154
66 0.0151515151515152
67 0.0149253731343284
68 0.0147058823529412
69 0.0144927536231884
70 0.0142857142857143
71 0.0140845070422535
72 0.0138888888888889
73 0.0136986301369863
74 0.0135135135135135
75 0.0133333333333333
76 0.0131578947368421
77 0.012987012987013
78 0.0128205128205128
79 0.0126582278481013
80 0.0125
81 0.0123456790123457
82 0.0121951219512195
83 0.0120481927710843
84 0.0119047619047619
85 0.0117647058823529
86 0.0116279069767442
87 0.0114942528735632
88 0.0227272727272727
89 0.0224719101123595
90 0.0222222222222222
91 0.021978021978022
92 0.0217391304347826
93 0.021505376344086
94 0.0212765957446809
95 0.0210526315789474
96 0.0208333333333333
97 0.0206185567010309
98 0.0306122448979592
99 0.0303030303030303
100 0.03
101 0.0297029702970297
102 0.0294117647058824
103 0.029126213592233
104 0.0288461538461538
105 0.0285714285714286
106 0.0283018867924528
107 0.0280373831775701
108 0.0277777777777778
109 0.0275229357798165
110 0.0272727272727273
111 0.027027027027027
112 0.0267857142857143
113 0.0265486725663717
114 0.0263157894736842
115 0.0260869565217391
116 0.0258620689655172
117 0.0256410256410256
118 0.0254237288135593
119 0.0252100840336134
120 0.025
121 0.0247933884297521
122 0.0245901639344262
123 0.024390243902439
124 0.0241935483870968
125 0.024
126 0.0238095238095238
127 0.0236220472440945
128 0.0234375
129 0.0232558139534884
130 0.0230769230769231
131 0.0229007633587786
132 0.0227272727272727
133 0.0225563909774436
134 0.0223880597014925
135 0.0296296296296296
136 0.0294117647058824
137 0.0291970802919708
138 0.0289855072463768
139 0.0287769784172662
140 0.0285714285714286
141 0.0354609929078014
142 0.0352112676056338
143 0.034965034965035
144 0.0347222222222222
145 0.0344827586206897
146 0.0342465753424658
147 0.0340136054421769
148 0.0337837837837838
149 0.0335570469798658
150 0.0333333333333333
151 0.033112582781457
152 0.0328947368421053
153 0.0326797385620915
154 0.0324675324675325
155 0.032258064516129
156 0.032051282051282
157 0.0318471337579618
158 0.0316455696202532
159 0.0314465408805031
160 0.03125
161 0.031055900621118
162 0.0308641975308642
163 0.0306748466257669
164 0.0304878048780488
165 0.0303030303030303
166 0.0301204819277108
167 0.029940119760479
168 0.0297619047619048
169 0.029585798816568
170 0.0294117647058824
171 0.0292397660818713
172 0.0290697674418605
173 0.0289017341040462
174 0.0344827586206897
175 0.0342857142857143
176 0.0340909090909091
177 0.0338983050847458
178 0.0337078651685393
179 0.0335195530726257
180 0.0333333333333333
181 0.0331491712707182
182 0.0384615384615385
183 0.0382513661202186
184 0.0380434782608696
185 0.0378378378378378
186 0.0376344086021505
187 0.0374331550802139
188 0.0372340425531915
189 0.037037037037037
190 0.0368421052631579
191 0.0366492146596859
192 0.0364583333333333
193 0.0362694300518135
194 0.0360824742268041
195 0.0358974358974359
196 0.0357142857142857
197 0.0355329949238579
198 0.0353535353535354
199 0.0351758793969849
200 0.035
201 0.0348258706467662
202 0.0346534653465347
203 0.0344827586206897
204 0.0343137254901961
205 0.0341463414634146
206 0.0339805825242718
207 0.0338164251207729
208 0.0336538461538462
209 0.0334928229665072
210 0.0333333333333333
211 0.033175355450237
212 0.0330188679245283
213 0.0328638497652582
214 0.0327102803738318
215 0.0372093023255814
216 0.037037037037037
217 0.0368663594470046
218 0.036697247706422
219 0.0365296803652968
220 0.0363636363636364
221 0.0361990950226244
222 0.036036036036036
223 0.0358744394618834
224 0.0357142857142857
225 0.0355555555555556
226 0.0353982300884956
227 0.0352422907488987
228 0.0350877192982456
229 0.0349344978165939
230 0.0347826086956522
231 0.0346320346320346
232 0.0344827586206897
233 0.0343347639484979
234 0.0341880341880342
235 0.0340425531914894
236 0.0338983050847458
237 0.0337552742616034
238 0.0336134453781513
239 0.0334728033472803
240 0.0333333333333333
241 0.033195020746888
242 0.0330578512396694
243 0.0329218106995885
244 0.0327868852459016
245 0.0326530612244898
246 0.032520325203252
247 0.0323886639676113
248 0.032258064516129
249 0.0321285140562249
250 0.032
251 0.0318725099601594
252 0.0317460317460317
253 0.0316205533596838
254 0.031496062992126
255 0.0313725490196078
256 0.03125
257 0.0311284046692607
258 0.0310077519379845
259 0.0308880308880309
260 0.0307692307692308
261 0.0306513409961686
262 0.0305343511450382
263 0.0342205323193916
264 0.0340909090909091
265 0.0339622641509434
266 0.0338345864661654
267 0.0337078651685393
268 0.0335820895522388
269 0.033457249070632
270 0.0333333333333333
271 0.033210332103321
272 0.0330882352941176
273 0.032967032967033
274 0.0328467153284672
275 0.0327272727272727
276 0.0326086956521739
277 0.0324909747292419
278 0.0323741007194245
279 0.032258064516129
280 0.0321428571428571
281 0.0320284697508897
282 0.0319148936170213
283 0.0318021201413428
284 0.0316901408450704
285 0.0315789473684211
286 0.0314685314685315
287 0.0313588850174216
288 0.03125
289 0.0311418685121107
290 0.0310344827586207
291 0.0309278350515464
292 0.0308219178082192
293 0.0307167235494881
294 0.0306122448979592
295 0.0305084745762712
296 0.0304054054054054
297 0.0303030303030303
298 0.0302013422818792
299 0.0301003344481605
300 0.03
301 0.0299003322259136
302 0.0298013245033113
303 0.0297029702970297
304 0.0296052631578947
305 0.0295081967213115
306 0.0294117647058824
307 0.0293159609120521
308 0.0292207792207792
309 0.029126213592233
310 0.0290322580645161
311 0.0289389067524116
312 0.0288461538461538
313 0.0287539936102236
314 0.0286624203821656
315 0.0285714285714286
316 0.0284810126582278
317 0.028391167192429
318 0.0283018867924528
319 0.0282131661442006
320 0.028125
321 0.0280373831775701
322 0.0279503105590062
323 0.0278637770897833
324 0.0277777777777778
325 0.0276923076923077
326 0.0276073619631902
327 0.0275229357798165
328 0.0274390243902439
329 0.027355623100304
330 0.0272727272727273
331 0.027190332326284
332 0.0271084337349398
333 0.027027027027027
334 0.0269461077844311
335 0.026865671641791
336 0.0267857142857143
337 0.0267062314540059
338 0.0266272189349112
339 0.0265486725663717
340 0.0264705882352941
341 0.0263929618768328
342 0.0263157894736842
343 0.0262390670553936
344 0.0261627906976744
345 0.0260869565217391
346 0.0260115606936416
347 0.0259365994236311
348 0.0258620689655172
349 0.0257879656160458
350 0.0257142857142857
351 0.0256410256410256
352 0.0284090909090909
353 0.028328611898017
354 0.0282485875706215
355 0.028169014084507
356 0.0280898876404494
357 0.0280112044817927
358 0.0279329608938547
359 0.0278551532033426
360 0.0277777777777778
361 0.0277008310249307
362 0.0276243093922652
363 0.0275482093663912
364 0.0274725274725275
365 0.0273972602739726
366 0.0273224043715847
367 0.0272479564032698
368 0.0271739130434783
369 0.02710027100271
370 0.027027027027027
371 0.0269541778975741
372 0.0268817204301075
373 0.0268096514745308
374 0.0267379679144385
375 0.0266666666666667
376 0.0265957446808511
377 0.026525198938992
378 0.0264550264550265
379 0.0263852242744063
380 0.0263157894736842
381 0.026246719160105
382 0.0261780104712042
383 0.0261096605744125
384 0.0260416666666667
385 0.025974025974026
386 0.0259067357512953
387 0.0258397932816537
388 0.0257731958762887
389 0.025706940874036
390 0.0256410256410256
391 0.0255754475703325
392 0.0255102040816327
393 0.0254452926208651
394 0.0253807106598985
395 0.0253164556962025
396 0.0252525252525253
397 0.0251889168765743
398 0.0251256281407035
399 0.025062656641604
400 0.025
401 0.0249376558603491
402 0.0248756218905473
403 0.0248138957816377
404 0.0247524752475248
405 0.0246913580246914
406 0.0246305418719212
407 0.0245700245700246
408 0.0245098039215686
409 0.0244498777506112
410 0.024390243902439
411 0.024330900243309
412 0.0242718446601942
413 0.0242130750605327
414 0.0241545893719807
415 0.0240963855421687
416 0.0240384615384615
417 0.0239808153477218
418 0.0239234449760766
419 0.0238663484486874
420 0.0238095238095238
421 0.0237529691211401
422 0.023696682464455
423 0.0236406619385343
424 0.0235849056603774
425 0.0235294117647059
426 0.0234741784037559
427 0.0234192037470726
428 0.0233644859813084
429 0.0233100233100233
430 0.0232558139534884
431 0.0232018561484919
432 0.0231481481481481
433 0.023094688221709
434 0.0230414746543779
435 0.0229885057471264
436 0.0229357798165138
437 0.022883295194508
438 0.0228310502283105
439 0.0227790432801822
440 0.0227272727272727
441 0.0226757369614512
442 0.0226244343891403
443 0.0225733634311512
444 0.0225225225225225
445 0.0224719101123595
446 0.0224215246636771
447 0.0223713646532438
448 0.0223214285714286
449 0.022271714922049
450 0.0222222222222222
451 0.0221729490022173
452 0.0221238938053097
453 0.022075055187638
454 0.0220264317180617
455 0.021978021978022
456 0.0219298245614035
457 0.0218818380743982
458 0.0218340611353712
459 0.0217864923747277
460 0.0217391304347826
461 0.0216919739696312
462 0.0216450216450216
463 0.0215982721382289
464 0.0237068965517241
465 0.0236559139784946
466 0.0236051502145923
467 0.0235546038543897
468 0.0235042735042735
469 0.023454157782516
470 0.0234042553191489
471 0.0233545647558386
472 0.0233050847457627
473 0.0232558139534884
474 0.0232067510548523
475 0.0231578947368421
476 0.023109243697479
477 0.0230607966457023
478 0.0230125523012552
479 0.022964509394572
480 0.0229166666666667
481 0.0228690228690229
482 0.0228215767634855
483 0.0227743271221532
484 0.0227272727272727
485 0.022680412371134
486 0.0226337448559671
487 0.0225872689938398
488 0.0225409836065574
489 0.0224948875255624
490 0.0224489795918367
491 0.0224032586558045
492 0.0223577235772358
493 0.0223123732251521
494 0.0222672064777328
495 0.0222222222222222
496 0.0221774193548387
497 0.0221327967806841
498 0.0220883534136546
499 0.0220440881763527
500 0.022
501 0.0219560878243513
502 0.0219123505976096
503 0.0218687872763419
504 0.0218253968253968
505 0.0217821782178218
506 0.0217391304347826
507 0.0216962524654832
508 0.0216535433070866
509 0.0216110019646365
510 0.0215686274509804
511 0.0215264187866928
512 0.021484375
513 0.0214424951267057
514 0.0233463035019455
515 0.0233009708737864
516 0.0232558139534884
517 0.02321083172147
518 0.0250965250965251
519 0.0250481695568401
520 0.025
521 0.0249520153550864
522 0.024904214559387
523 0.0248565965583174
524 0.0248091603053435
525 0.0247619047619048
526 0.0247148288973384
527 0.0246679316888046
528 0.0246212121212121
529 0.0245746691871456
530 0.0245283018867925
531 0.0244821092278719
532 0.0244360902255639
533 0.024390243902439
534 0.0243445692883895
535 0.0242990654205607
536 0.0242537313432836
537 0.0242085661080074
538 0.0241635687732342
539 0.0241187384044527
540 0.0240740740740741
541 0.0240295748613678
542 0.0239852398523985
543 0.0239410681399632
544 0.0238970588235294
545 0.0238532110091743
546 0.0238095238095238
547 0.0237659963436929
548 0.0237226277372263
549 0.0236794171220401
550 0.0236363636363636
551 0.0235934664246824
552 0.0235507246376812
553 0.0235081374321881
554 0.0234657039711191
555 0.0234234234234234
556 0.0233812949640288
557 0.0233393177737881
558 0.0232974910394265
559 0.0232558139534884
560 0.0232142857142857
561 0.0231729055258467
562 0.0231316725978648
563 0.0230905861456483
564 0.0230496453900709
565 0.0247787610619469
566 0.0247349823321555
567 0.0246913580246914
568 0.0246478873239437
569 0.0246045694200351
570 0.0245614035087719
571 0.0245183887915937
572 0.0244755244755245
573 0.0244328097731239
574 0.024390243902439
575 0.0243478260869565
576 0.0243055555555556
577 0.024263431542461
578 0.0242214532871972
579 0.0241796200345423
580 0.0241379310344828
581 0.0240963855421687
582 0.0240549828178694
583 0.0240137221269297
584 0.023972602739726
585 0.0239316239316239
586 0.0238907849829352
587 0.0255536626916525
588 0.0255102040816327
589 0.0254668930390492
590 0.0254237288135593
591 0.027072758037225
592 0.027027027027027
593 0.0269814502529511
594 0.0269360269360269
595 0.026890756302521
596 0.0268456375838926
597 0.0268006700167504
598 0.0267558528428094
599 0.0267111853088481
600 0.0266666666666667
601 0.0266222961730449
602 0.026578073089701
603 0.0265339966832504
604 0.0264900662251656
605 0.0264462809917355
606 0.0264026402640264
607 0.0263591433278418
608 0.0263157894736842
609 0.0262725779967159
610 0.0262295081967213
611 0.0261865793780687
612 0.0261437908496732
613 0.0261011419249592
614 0.0260586319218241
615 0.0260162601626016
616 0.025974025974026
617 0.0259319286871961
618 0.0258899676375405
619 0.0258481421647819
620 0.0258064516129032
621 0.0257648953301127
622 0.0257234726688103
623 0.0256821829855538
624 0.0256410256410256
625 0.0256
626 0.0255591054313099
627 0.025518341307815
628 0.0254777070063694
629 0.0254372019077901
630 0.0253968253968254
631 0.0253565768621236
632 0.0253164556962025
633 0.0252764612954186
634 0.0252365930599369
635 0.0251968503937008
636 0.0251572327044025
637 0.0251177394034537
638 0.0250783699059561
639 0.02660406885759
640 0.0265625
641 0.0265210608424337
642 0.0264797507788162
643 0.0264385692068429
644 0.0263975155279503
645 0.0263565891472868
646 0.0263157894736842
647 0.0262751159196291
648 0.0262345679012346
649 0.0261941448382126
650 0.0261538461538462
651 0.0261136712749616
652 0.0260736196319018
653 0.0260336906584992
654 0.0259938837920489
655 0.0259541984732824
656 0.0259146341463415
657 0.0258751902587519
658 0.0258358662613982
659 0.0257966616084977
660 0.0257575757575758
661 0.0257186081694402
662 0.0256797583081571
663 0.0256410256410256
664 0.0256024096385542
665 0.0255639097744361
666 0.0255255255255255
667 0.0254872563718141
668 0.0254491017964072
669 0.0254110612855007
670 0.0253731343283582
671 0.0253353204172876
672 0.0267857142857143
673 0.0267459138187221
674 0.0267062314540059
675 0.0266666666666667
676 0.0266272189349112
677 0.0265878877400295
678 0.0265486725663717
679 0.0265095729013255
680 0.0264705882352941
681 0.026431718061674
682 0.0263929618768328
683 0.0263543191800878
684 0.0263157894736842
685 0.0262773722627737
686 0.0262390670553936
687 0.0262008733624454
688 0.0261627906976744
689 0.0261248185776488
690 0.0260869565217391
691 0.0260492040520984
692 0.0260115606936416
693 0.025974025974026
694 0.0259365994236311
695 0.0258992805755396
696 0.0258620689655172
697 0.0258249641319943
698 0.0257879656160458
699 0.0257510729613734
700 0.0257142857142857
701 0.0256776034236805
702 0.0256410256410256
703 0.027027027027027
704 0.0269886363636364
705 0.0269503546099291
706 0.0269121813031161
707 0.0268741159830269
708 0.0268361581920904
709 0.0267983074753173
710 0.0267605633802817
711 0.0267229254571027
712 0.026685393258427
713 0.0266479663394109
714 0.0266106442577031
715 0.0265734265734266
716 0.026536312849162
717 0.0264993026499303
718 0.0264623955431755
719 0.0264255910987483
720 0.0263888888888889
721 0.0277392510402219
722 0.0277008310249307
723 0.0276625172890733
724 0.0276243093922652
725 0.0275862068965517
726 0.0275482093663912
727 0.0275103163686382
728 0.0274725274725275
729 0.0274348422496571
730 0.0273972602739726
731 0.027359781121751
732 0.0273224043715847
733 0.0272851296043656
734 0.0272479564032698
735 0.0272108843537415
736 0.0271739130434783
737 0.0271370420624152
738 0.02710027100271
739 0.027063599458728
740 0.027027027027027
741 0.0269905533063428
742 0.0269541778975741
743 0.0269179004037685
744 0.0268817204301075
745 0.0268456375838926
746 0.0268096514745308
747 0.0267737617135207
748 0.0267379679144385
749 0.0267022696929239
750 0.0266666666666667
751 0.0266311584553928
752 0.0265957446808511
753 0.0265604249667995
754 0.0278514588859416
755 0.0278145695364238
756 0.0277777777777778
757 0.0277410832232497
758 0.0277044854881266
759 0.0276679841897233
760 0.0276315789473684
761 0.0275952693823916
762 0.0275590551181102
763 0.0275229357798165
764 0.0274869109947644
765 0.0274509803921569
766 0.0274151436031332
767 0.0273794002607562
768 0.02734375
769 0.0273081924577373
770 0.0272727272727273
771 0.0272373540856031
772 0.0272020725388601
773 0.0271668822768435
774 0.0271317829457364
775 0.0270967741935484
776 0.0270618556701031
777 0.027027027027027
778 0.0269922879177378
779 0.0269576379974326
780 0.0269230769230769
781 0.0268886043533931
782 0.0268542199488491
783 0.0268199233716475
784 0.0267857142857143
785 0.0267515923566879
786 0.0267175572519084
787 0.0266836086404066
788 0.0266497461928934
789 0.026615969581749
790 0.0265822784810127
791 0.0265486725663717
792 0.0265151515151515
793 0.0264817150063052
794 0.026448362720403
795 0.0264150943396226
796 0.0263819095477387
797 0.0263488080301129
798 0.0263157894736842
799 0.0262828535669587
800 0.02625
801 0.0262172284644195
802 0.0261845386533666
803 0.0261519302615193
804 0.027363184079602
805 0.0273291925465838
806 0.0272952853598015
807 0.0272614622057001
808 0.0272277227722772
809 0.0271940667490729
810 0.0271604938271605
811 0.0271270036991369
812 0.0270935960591133
813 0.027060270602706
814 0.027027027027027
815 0.0269938650306748
816 0.0269607843137255
817 0.0269277845777234
818 0.0268948655256724
819 0.0268620268620269
820 0.0268292682926829
821 0.0267965895249695
822 0.0279805352798054
823 0.0279465370595383
824 0.0279126213592233
825 0.0278787878787879
826 0.0278450363196126
827 0.0278113663845224
828 0.0277777777777778
829 0.0277442702050663
830 0.027710843373494
831 0.0276774969915764
832 0.0276442307692308
833 0.0276110444177671
834 0.0275779376498801
835 0.0275449101796407
836 0.027511961722488
837 0.027479091995221
838 0.0274463007159905
839 0.0274135876042908
840 0.0273809523809524
841 0.0273483947681332
842 0.0273159144893112
843 0.0272835112692764
844 0.0272511848341232
845 0.0272189349112426
846 0.0271867612293144
847 0.0271546635182999
848 0.027122641509434
849 0.0270906949352179
850 0.0270588235294118
851 0.027027027027027
852 0.0269953051643192
853 0.0269636576787808
854 0.0269320843091335
855 0.0269005847953216
856 0.0268691588785047
857 0.0268378063010502
858 0.0268065268065268
859 0.0267753201396973
860 0.0267441860465116
861 0.0267131242740999
862 0.0266821345707657
863 0.0266512166859791
864 0.0266203703703704
865 0.0265895953757225
866 0.0265588914549654
867 0.0265282583621684
868 0.0264976958525346
869 0.0264672036823936
870 0.0264367816091954
871 0.026406429391504
872 0.0263761467889908
873 0.0263459335624284
874 0.0263157894736842
875 0.0262857142857143
876 0.0262557077625571
877 0.0262257696693273
878 0.0261958997722096
879 0.0261660978384528
880 0.0261363636363636
881 0.0261066969353008
882 0.0260770975056689
883 0.0260475651189128
884 0.0260180995475113
885 0.0259887005649717
886 0.0259593679458239
887 0.0259301014656144
888 0.0259009009009009
889 0.0258717660292463
890 0.0258426966292135
891 0.0258136924803591
892 0.0257847533632287
893 0.0257558790593505
894 0.0257270693512304
895 0.0256983240223464
896 0.0256696428571429
897 0.0256410256410256
898 0.0256124721603563
899 0.0255839822024472
900 0.0255555555555556
901 0.025527192008879
902 0.0254988913525499
903 0.0254706533776301
904 0.0254424778761062
905 0.025414364640884
906 0.0253863134657837
907 0.0253583241455347
908 0.0253303964757709
909 0.0253025302530253
910 0.0252747252747253
911 0.0252469813391877
912 0.025219298245614
913 0.0251916757940854
914 0.025164113785558
915 0.0251366120218579
916 0.0251091703056769
917 0.0250817884405671
918 0.0250544662309368
919 0.0250272034820457
920 0.025
921 0.0249728555917481
922 0.0249457700650759
923 0.0260021668472373
924 0.025974025974026
925 0.0259459459459459
926 0.0259179265658747
927 0.0258899676375405
928 0.0258620689655172
929 0.0258342303552207
930 0.0258064516129032
931 0.0257787325456498
932 0.0257510729613734
933 0.0257234726688103
934 0.0256959314775161
935 0.025668449197861
936 0.0256410256410256
937 0.0256136606189968
938 0.0266524520255864
939 0.0266240681576145
940 0.0265957446808511
941 0.026567481402763
942 0.0265392781316348
943 0.0265111346765642
944 0.0275423728813559
945 0.0275132275132275
946 0.0274841437632135
947 0.027455121436114
948 0.0274261603375527
949 0.0273972602739726
950 0.0273684210526316
951 0.0273396424815983
952 0.0273109243697479
953 0.0283315844700944
954 0.0283018867924528
955 0.0282722513089005
956 0.0282426778242678
957 0.0282131661442006
958 0.0281837160751566
959 0.0281543274244004
960 0.028125
961 0.0280957336108221
962 0.0280665280665281
963 0.0280373831775701
964 0.0280082987551867
965 0.027979274611399
966 0.0279503105590062
967 0.0279214064115822
968 0.0278925619834711
969 0.0278637770897833
970 0.0278350515463918
971 0.0278063851699279
972 0.0277777777777778
973 0.0277492291880781
974 0.0277207392197125
975 0.0276923076923077
976 0.0276639344262295
977 0.0276356192425793
978 0.0276073619631902
979 0.0275791624106231
980 0.0275510204081633
981 0.0275229357798165
982 0.0274949083503055
983 0.0274669379450661
984 0.0274390243902439
985 0.0274111675126904
986 0.0273833671399594
987 0.027355623100304
988 0.0273279352226721
989 0.0273003033367037
990 0.0272727272727273
991 0.0272452068617558
992 0.0272177419354839
993 0.027190332326284
994 0.0271629778672032
995 0.0271356783919598
996 0.0271084337349398
997 0.0270812437311936
998 0.0270541082164329
999 0.027027027027027
1000 0.027
};
\addplot [semithick, color2, forget plot]
table {%
1 0
2 0
3 0
4 0
5 0
6 0
7 0
8 0
9 0
10 0
11 0
12 0
13 0
14 0
15 0
16 0
17 0
18 0
19 0
20 0
21 0
22 0
23 0
24 0
25 0
26 0
27 0
28 0
29 0
30 0
31 0
32 0
33 0
34 0
35 0
36 0
37 0
38 0
39 0
40 0
41 0
42 0
43 0
44 0
45 0
46 0
47 0
48 0
49 0
50 0
51 0
52 0
53 0
54 0
55 0.0181818181818182
56 0.0178571428571429
57 0.0175438596491228
58 0.0172413793103448
59 0.0169491525423729
60 0.0166666666666667
61 0.0163934426229508
62 0.0161290322580645
63 0.0158730158730159
64 0.015625
65 0.0153846153846154
66 0.0151515151515152
67 0.0149253731343284
68 0.0147058823529412
69 0.0144927536231884
70 0.0142857142857143
71 0.0140845070422535
72 0.0138888888888889
73 0.0136986301369863
74 0.0135135135135135
75 0.0133333333333333
76 0.0131578947368421
77 0.012987012987013
78 0.0128205128205128
79 0.0126582278481013
80 0.0125
81 0.0123456790123457
82 0.0121951219512195
83 0.0120481927710843
84 0.0119047619047619
85 0.0117647058823529
86 0.0116279069767442
87 0.0114942528735632
88 0.0113636363636364
89 0.0112359550561798
90 0.0111111111111111
91 0.010989010989011
92 0.0108695652173913
93 0.010752688172043
94 0.0106382978723404
95 0.0105263157894737
96 0.0104166666666667
97 0.0103092783505155
98 0.0102040816326531
99 0.0101010101010101
100 0.01
101 0.0099009900990099
102 0.00980392156862745
103 0.00970873786407767
104 0.00961538461538462
105 0.00952380952380952
106 0.00943396226415094
107 0.00934579439252336
108 0.00925925925925926
109 0.00917431192660551
110 0.00909090909090909
111 0.00900900900900901
112 0.00892857142857143
113 0.00884955752212389
114 0.0087719298245614
115 0.00869565217391304
116 0.00862068965517241
117 0.00854700854700855
118 0.0169491525423729
119 0.0168067226890756
120 0.0166666666666667
121 0.0165289256198347
122 0.0163934426229508
123 0.016260162601626
124 0.0161290322580645
125 0.016
126 0.0158730158730159
127 0.015748031496063
128 0.015625
129 0.0155038759689922
130 0.0153846153846154
131 0.0152671755725191
132 0.0151515151515152
133 0.0150375939849624
134 0.0149253731343284
135 0.0148148148148148
136 0.0147058823529412
137 0.0145985401459854
138 0.0144927536231884
139 0.0143884892086331
140 0.0142857142857143
141 0.0141843971631206
142 0.0140845070422535
143 0.013986013986014
144 0.0138888888888889
145 0.0206896551724138
146 0.0205479452054795
147 0.0204081632653061
148 0.0202702702702703
149 0.0201342281879195
150 0.02
151 0.0198675496688742
152 0.0197368421052632
153 0.0196078431372549
154 0.0194805194805195
155 0.0193548387096774
156 0.0192307692307692
157 0.0191082802547771
158 0.0189873417721519
159 0.0251572327044025
160 0.025
161 0.0248447204968944
162 0.0246913580246914
163 0.0245398773006135
164 0.024390243902439
165 0.0242424242424242
166 0.0240963855421687
167 0.0239520958083832
168 0.0238095238095238
169 0.0236686390532544
170 0.0235294117647059
171 0.0233918128654971
172 0.0232558139534884
173 0.023121387283237
174 0.0229885057471264
175 0.0228571428571429
176 0.0227272727272727
177 0.0225988700564972
178 0.0224719101123595
179 0.0223463687150838
180 0.0222222222222222
181 0.0220994475138122
182 0.0274725274725275
183 0.0273224043715847
184 0.0271739130434783
185 0.027027027027027
186 0.0268817204301075
187 0.0267379679144385
188 0.0265957446808511
189 0.0264550264550265
190 0.0263157894736842
191 0.0261780104712042
192 0.0260416666666667
193 0.0259067357512953
194 0.0257731958762887
195 0.0256410256410256
196 0.0255102040816327
197 0.0253807106598985
198 0.0252525252525253
199 0.0251256281407035
200 0.025
201 0.0248756218905473
202 0.0247524752475248
203 0.0246305418719212
204 0.0245098039215686
205 0.024390243902439
206 0.0242718446601942
207 0.0241545893719807
208 0.0240384615384615
209 0.0239234449760766
210 0.0238095238095238
211 0.023696682464455
212 0.0235849056603774
213 0.0234741784037559
214 0.0233644859813084
215 0.0232558139534884
216 0.0231481481481481
217 0.0230414746543779
218 0.0229357798165138
219 0.0228310502283105
220 0.0227272727272727
221 0.0226244343891403
222 0.0225225225225225
223 0.0224215246636771
224 0.0223214285714286
225 0.0222222222222222
226 0.0221238938053097
227 0.0220264317180617
228 0.0219298245614035
229 0.0218340611353712
230 0.0217391304347826
231 0.0216450216450216
232 0.021551724137931
233 0.0214592274678112
234 0.0213675213675214
235 0.025531914893617
236 0.0254237288135593
237 0.0253164556962025
238 0.0252100840336134
239 0.0251046025104603
240 0.025
241 0.024896265560166
242 0.0247933884297521
243 0.0246913580246914
244 0.0245901639344262
245 0.0244897959183673
246 0.024390243902439
247 0.0242914979757085
248 0.0241935483870968
249 0.0240963855421687
250 0.024
251 0.0239043824701195
252 0.0238095238095238
253 0.0237154150197628
254 0.0236220472440945
255 0.0235294117647059
256 0.0234375
257 0.0233463035019455
258 0.0232558139534884
259 0.0231660231660232
260 0.0230769230769231
261 0.0229885057471264
262 0.0229007633587786
263 0.0228136882129278
264 0.0227272727272727
265 0.0226415094339623
266 0.0225563909774436
267 0.0224719101123595
268 0.0223880597014925
269 0.0223048327137546
270 0.0222222222222222
271 0.022140221402214
272 0.0220588235294118
273 0.021978021978022
274 0.0218978102189781
275 0.0218181818181818
276 0.0217391304347826
277 0.0216606498194946
278 0.0215827338129496
279 0.021505376344086
280 0.0214285714285714
281 0.0213523131672598
282 0.0212765957446809
283 0.0212014134275618
284 0.0211267605633803
285 0.0210526315789474
286 0.020979020979021
287 0.0209059233449477
288 0.0208333333333333
289 0.0207612456747405
290 0.0206896551724138
291 0.0206185567010309
292 0.0205479452054795
293 0.0204778156996587
294 0.0204081632653061
295 0.0203389830508475
296 0.0202702702702703
297 0.0202020202020202
298 0.0201342281879195
299 0.020066889632107
300 0.02
301 0.0199335548172757
302 0.0198675496688742
303 0.0198019801980198
304 0.0197368421052632
305 0.019672131147541
306 0.0228758169934641
307 0.0228013029315961
308 0.0227272727272727
309 0.0226537216828479
310 0.0225806451612903
311 0.022508038585209
312 0.0224358974358974
313 0.0223642172523962
314 0.0222929936305732
315 0.0222222222222222
316 0.0221518987341772
317 0.0220820189274448
318 0.0220125786163522
319 0.0219435736677116
320 0.021875
321 0.0218068535825545
322 0.0217391304347826
323 0.021671826625387
324 0.0216049382716049
325 0.0215384615384615
326 0.0214723926380368
327 0.0214067278287462
328 0.0213414634146341
329 0.0212765957446809
330 0.0212121212121212
331 0.0211480362537764
332 0.0210843373493976
333 0.021021021021021
334 0.0209580838323353
335 0.0208955223880597
336 0.0208333333333333
337 0.0207715133531157
338 0.0207100591715976
339 0.0206489675516224
340 0.0205882352941176
341 0.0205278592375367
342 0.0204678362573099
343 0.0204081632653061
344 0.0203488372093023
345 0.0202898550724638
346 0.0202312138728324
347 0.0201729106628242
348 0.0201149425287356
349 0.0200573065902579
350 0.02
351 0.0199430199430199
352 0.0198863636363636
353 0.0198300283286119
354 0.019774011299435
355 0.0197183098591549
356 0.0196629213483146
357 0.0196078431372549
358 0.0195530726256983
359 0.0194986072423398
360 0.0194444444444444
361 0.0193905817174515
362 0.0193370165745856
363 0.0192837465564738
364 0.0192307692307692
365 0.0191780821917808
366 0.0191256830601093
367 0.0190735694822888
368 0.0190217391304348
369 0.018970189701897
370 0.0189189189189189
371 0.0188679245283019
372 0.0188172043010753
373 0.0187667560321716
374 0.018716577540107
375 0.0186666666666667
376 0.0186170212765957
377 0.0185676392572944
378 0.0185185185185185
379 0.0184696569920844
380 0.0184210526315789
381 0.0183727034120735
382 0.0183246073298429
383 0.0182767624020888
384 0.0182291666666667
385 0.0181818181818182
386 0.0181347150259067
387 0.0180878552971576
388 0.0180412371134021
389 0.0179948586118252
390 0.0179487179487179
391 0.0179028132992327
392 0.0178571428571429
393 0.0178117048346056
394 0.0177664974619289
395 0.0177215189873418
396 0.0176767676767677
397 0.017632241813602
398 0.0175879396984925
399 0.0175438596491228
400 0.0175
401 0.0174563591022444
402 0.0174129353233831
403 0.0173697270471464
404 0.0173267326732673
405 0.0172839506172839
406 0.0172413793103448
407 0.0171990171990172
408 0.0196078431372549
409 0.019559902200489
410 0.0195121951219512
411 0.0194647201946472
412 0.0194174757281553
413 0.0193704600484262
414 0.0193236714975845
415 0.0192771084337349
416 0.0216346153846154
417 0.0215827338129496
418 0.0215311004784689
419 0.0214797136038186
420 0.0214285714285714
421 0.0213776722090261
422 0.0213270142180095
423 0.0212765957446809
424 0.0212264150943396
425 0.0211764705882353
426 0.0211267605633803
427 0.0210772833723653
428 0.0210280373831776
429 0.020979020979021
430 0.0209302325581395
431 0.0208816705336427
432 0.0208333333333333
433 0.0207852193995381
434 0.0207373271889401
435 0.0206896551724138
436 0.0206422018348624
437 0.0205949656750572
438 0.0205479452054795
439 0.020501138952164
440 0.0204545454545455
441 0.0204081632653061
442 0.0203619909502262
443 0.0203160270880361
444 0.0202702702702703
445 0.0202247191011236
446 0.0201793721973094
447 0.0201342281879195
448 0.0200892857142857
449 0.0200445434298441
450 0.02
451 0.0199556541019956
452 0.0199115044247788
453 0.0198675496688742
454 0.0198237885462555
455 0.0197802197802198
456 0.0197368421052632
457 0.0196936542669584
458 0.0196506550218341
459 0.0196078431372549
460 0.0195652173913043
461 0.0195227765726681
462 0.0194805194805195
463 0.019438444924406
464 0.0193965517241379
465 0.0193548387096774
466 0.01931330472103
467 0.019271948608137
468 0.0192307692307692
469 0.0191897654584222
470 0.0191489361702128
471 0.0191082802547771
472 0.0190677966101695
473 0.0190274841437632
474 0.0189873417721519
475 0.0189473684210526
476 0.0189075630252101
477 0.0188679245283019
478 0.0188284518828452
479 0.0187891440501044
480 0.01875
481 0.0187110187110187
482 0.0186721991701245
483 0.0186335403726708
484 0.0185950413223141
485 0.0185567010309278
486 0.0185185185185185
487 0.0184804928131417
488 0.0184426229508197
489 0.0184049079754601
490 0.0183673469387755
491 0.0183299389002037
492 0.0182926829268293
493 0.0182555780933063
494 0.0182186234817814
495 0.0181818181818182
496 0.0181451612903226
497 0.0181086519114688
498 0.0180722891566265
499 0.0180360721442886
500 0.018
501 0.0179640718562874
502 0.0179282868525896
503 0.0178926441351889
504 0.0178571428571429
505 0.0178217821782178
506 0.0177865612648221
507 0.0177514792899408
508 0.0177165354330709
509 0.0176817288801572
510 0.0176470588235294
511 0.0176125244618395
512 0.017578125
513 0.0175438596491228
514 0.0175097276264591
515 0.0174757281553398
516 0.0174418604651163
517 0.0174081237911025
518 0.0193050193050193
519 0.0192678227360308
520 0.0192307692307692
521 0.0191938579654511
522 0.0191570881226054
523 0.0191204588910134
524 0.0190839694656489
525 0.019047619047619
526 0.0190114068441065
527 0.0189753320683112
528 0.0189393939393939
529 0.0189035916824197
530 0.0188679245283019
531 0.0188323917137476
532 0.018796992481203
533 0.0187617260787993
534 0.0187265917602996
535 0.0186915887850467
536 0.0186567164179104
537 0.0186219739292365
538 0.0185873605947955
539 0.0185528756957328
540 0.0185185185185185
541 0.0184842883548983
542 0.018450184501845
543 0.0184162062615101
544 0.0183823529411765
545 0.0201834862385321
546 0.0201465201465201
547 0.020109689213894
548 0.0200729927007299
549 0.0200364298724954
550 0.02
551 0.0199637023593466
552 0.0199275362318841
553 0.0198915009041591
554 0.01985559566787
555 0.0198198198198198
556 0.0197841726618705
557 0.0197486535008977
558 0.0197132616487455
559 0.0196779964221825
560 0.0196428571428571
561 0.0196078431372549
562 0.0195729537366548
563 0.019538188277087
564 0.0195035460992908
565 0.0212389380530973
566 0.0212014134275618
567 0.0211640211640212
568 0.0211267605633803
569 0.0210896309314587
570 0.0210526315789474
571 0.021015761821366
572 0.020979020979021
573 0.0209424083769634
574 0.0209059233449477
575 0.0208695652173913
576 0.0208333333333333
577 0.0207972270363951
578 0.0207612456747405
579 0.0207253886010363
580 0.0206896551724138
581 0.0206540447504303
582 0.0206185567010309
583 0.0205831903945112
584 0.0205479452054795
585 0.0205128205128205
586 0.0204778156996587
587 0.020442930153322
588 0.0204081632653061
589 0.0203735144312394
590 0.0203389830508475
591 0.0203045685279188
592 0.0202702702702703
593 0.0202360876897133
594 0.0202020202020202
595 0.0201680672268908
596 0.0201342281879195
597 0.0201005025125628
598 0.020066889632107
599 0.0200333889816361
600 0.02
601 0.0199667221297837
602 0.0199335548172757
603 0.0199004975124378
604 0.0198675496688742
605 0.0198347107438017
606 0.0198019801980198
607 0.0197693574958814
608 0.0197368421052632
609 0.0197044334975369
610 0.019672131147541
611 0.0196399345335516
612 0.0196078431372549
613 0.0195758564437194
614 0.0195439739413681
615 0.0195121951219512
616 0.0194805194805195
617 0.0194489465153971
618 0.0194174757281553
619 0.0193861066235864
620 0.0193548387096774
621 0.0193236714975845
622 0.0192926045016077
623 0.0192616372391653
624 0.0192307692307692
625 0.0192
626 0.0191693290734824
627 0.0191387559808612
628 0.0191082802547771
629 0.0190779014308426
630 0.019047619047619
631 0.0190174326465927
632 0.0189873417721519
633 0.018957345971564
634 0.0189274447949527
635 0.0188976377952756
636 0.0188679245283019
637 0.0188383045525903
638 0.0188087774294671
639 0.0187793427230047
640 0.01875
641 0.0187207488299532
642 0.0186915887850467
643 0.0186625194401244
644 0.0186335403726708
645 0.0186046511627907
646 0.0185758513931889
647 0.0185471406491499
648 0.0185185185185185
649 0.0184899845916795
650 0.0184615384615385
651 0.0184331797235023
652 0.0184049079754601
653 0.0183767228177642
654 0.018348623853211
655 0.0183206106870229
656 0.0182926829268293
657 0.0182648401826484
658 0.0182370820668693
659 0.0182094081942337
660 0.0181818181818182
661 0.0181543116490166
662 0.0181268882175227
663 0.0180995475113122
664 0.0180722891566265
665 0.0180451127819549
666 0.018018018018018
667 0.0179910044977511
668 0.0179640718562874
669 0.0179372197309417
670 0.017910447761194
671 0.0178837555886736
672 0.0178571428571429
673 0.0178306092124814
674 0.0178041543026706
675 0.0177777777777778
676 0.0177514792899408
677 0.0192023633677991
678 0.0191740412979351
679 0.0191458026509573
680 0.0191176470588235
681 0.0190895741556534
682 0.0205278592375367
683 0.020497803806735
684 0.0204678362573099
685 0.0204379562043796
686 0.0204081632653061
687 0.0203784570596798
688 0.0203488372093023
689 0.0203193033381713
690 0.0202898550724638
691 0.020260492040521
692 0.0202312138728324
693 0.0202020202020202
694 0.0201729106628242
695 0.0201438848920863
696 0.0201149425287356
697 0.0200860832137733
698 0.0200573065902579
699 0.0200286123032904
700 0.02
701 0.0199714693295292
702 0.0213675213675214
703 0.0213371266002845
704 0.0213068181818182
705 0.0212765957446809
706 0.0226628895184136
707 0.0226308345120226
708 0.0225988700564972
709 0.0225669957686883
710 0.0225352112676056
711 0.0225035161744023
712 0.0224719101123595
713 0.0224403927068724
714 0.0224089635854342
715 0.0223776223776224
716 0.0223463687150838
717 0.0223152022315202
718 0.0222841225626741
719 0.0222531293463143
720 0.0222222222222222
721 0.0221914008321775
722 0.0221606648199446
723 0.0221300138312586
724 0.0220994475138122
725 0.0220689655172414
726 0.0220385674931129
727 0.0220082530949106
728 0.021978021978022
729 0.0219478737997256
730 0.0219178082191781
731 0.0218878248974008
732 0.0218579234972678
733 0.0218281036834925
734 0.0217983651226158
735 0.0217687074829932
736 0.0217391304347826
737 0.0217096336499322
738 0.021680216802168
739 0.0216508795669824
740 0.0216216216216216
741 0.0215924426450742
742 0.0215633423180593
743 0.0215343203230148
744 0.021505376344086
745 0.0214765100671141
746 0.0214477211796247
747 0.0214190093708166
748 0.0213903743315508
749 0.0213618157543391
750 0.0226666666666667
751 0.0226364846870839
752 0.0226063829787234
753 0.0225763612217795
754 0.0225464190981432
755 0.0225165562913907
756 0.0224867724867725
757 0.0224570673712021
758 0.0224274406332454
759 0.0223978919631094
760 0.0223684210526316
761 0.0223390275952694
762 0.0223097112860892
763 0.0222804718217562
764 0.0222513089005236
765 0.0222222222222222
766 0.0221932114882507
767 0.0221642764015645
768 0.0221354166666667
769 0.0221066319895969
770 0.0220779220779221
771 0.0220492866407263
772 0.022020725388601
773 0.0219922380336352
774 0.0219638242894057
775 0.0219354838709677
776 0.0219072164948454
777 0.0218790218790219
778 0.0218508997429306
779 0.0218228498074454
780 0.0217948717948718
781 0.0217669654289373
782 0.0230179028132992
783 0.0229885057471264
784 0.0229591836734694
785 0.0229299363057325
786 0.0229007633587786
787 0.0228716645489199
788 0.0228426395939086
789 0.0228136882129278
790 0.0227848101265823
791 0.02275600505689
792 0.0227272727272727
793 0.0226986128625473
794 0.0226700251889169
795 0.0226415094339623
796 0.0226130653266332
797 0.0225846925972397
798 0.0225563909774436
799 0.0225281602002503
800 0.0225
801 0.0237203495630462
802 0.0236907730673317
803 0.024906600249066
804 0.0248756218905473
805 0.0248447204968944
806 0.0248138957816377
807 0.0247831474597274
808 0.0247524752475248
809 0.0247218788627936
810 0.0246913580246914
811 0.0246609124537608
812 0.0246305418719212
813 0.02460024600246
814 0.0245700245700246
815 0.0245398773006135
816 0.0245098039215686
817 0.0244798041615667
818 0.0244498777506112
819 0.0244200244200244
820 0.024390243902439
821 0.0243605359317905
822 0.024330900243309
823 0.0243013365735115
824 0.0242718446601942
825 0.0242424242424242
826 0.0242130750605327
827 0.0241837968561064
828 0.0241545893719807
829 0.0241254523522316
830 0.0240963855421687
831 0.0240673886883273
832 0.0240384615384615
833 0.0240096038415366
834 0.0239808153477218
835 0.0239520958083832
836 0.0239234449760766
837 0.02389486260454
838 0.0238663484486874
839 0.0238379022646007
840 0.0238095238095238
841 0.0237812128418549
842 0.0237529691211401
843 0.0237247924080664
844 0.023696682464455
845 0.0236686390532544
846 0.0236406619385343
847 0.0236127508854782
848 0.0235849056603774
849 0.0235571260306243
850 0.0235294117647059
851 0.0235017626321974
852 0.0234741784037559
853 0.0234466588511137
854 0.0234192037470726
855 0.0233918128654971
856 0.0233644859813084
857 0.0233372228704784
858 0.0233100233100233
859 0.0232828870779977
860 0.0232558139534884
861 0.0232288037166086
862 0.0232018561484919
863 0.0231749710312862
864 0.0231481481481481
865 0.023121387283237
866 0.023094688221709
867 0.0230680507497117
868 0.0230414746543779
869 0.0230149597238205
870 0.0229885057471264
871 0.0229621125143513
872 0.0229357798165138
873 0.0229095074455899
874 0.022883295194508
875 0.0228571428571429
876 0.0228310502283105
877 0.0228050171037628
878 0.0227790432801822
879 0.0227531285551763
880 0.0227272727272727
881 0.0227014755959137
882 0.0226757369614512
883 0.0226500566251416
884 0.0226244343891403
885 0.0225988700564972
886 0.0225733634311512
887 0.0225479143179256
888 0.0225225225225225
889 0.0224971878515186
890 0.0224719101123595
891 0.0224466891133558
892 0.0224215246636771
893 0.0223964165733483
894 0.0223713646532438
895 0.0223463687150838
896 0.0223214285714286
897 0.0222965440356745
898 0.022271714922049
899 0.0222469410456062
900 0.0222222222222222
901 0.0221975582685905
902 0.0221729490022173
903 0.0221483942414175
904 0.0221238938053097
905 0.0220994475138122
906 0.022075055187638
907 0.0220507166482911
908 0.0220264317180617
909 0.022002200220022
910 0.021978021978022
911 0.021953896816685
912 0.0219298245614035
913 0.0219058050383352
914 0.0218818380743982
915 0.0218579234972678
916 0.0218340611353712
917 0.0218102508178844
918 0.0217864923747277
919 0.0217627856365615
920 0.0217391304347826
921 0.0217155266015201
922 0.0216919739696312
923 0.0216684723726977
924 0.0216450216450216
925 0.0216216216216216
926 0.0215982721382289
927 0.0215749730312837
928 0.021551724137931
929 0.0215285252960172
930 0.021505376344086
931 0.0214822771213749
932 0.0214592274678112
933 0.0214362272240086
934 0.0214132762312634
935 0.0213903743315508
936 0.0213675213675214
937 0.0213447171824973
938 0.0213219616204691
939 0.0212992545260916
940 0.0212765957446809
941 0.0212539851222104
942 0.0212314225053079
943 0.0212089077412513
944 0.0211864406779661
945 0.0211640211640212
946 0.0211416490486258
947 0.0211193241816262
948 0.0210970464135021
949 0.0210748155953635
950 0.0210526315789474
951 0.0210304942166141
952 0.0210084033613445
953 0.0209863588667366
954 0.0209643605870021
955 0.0209424083769634
956 0.0209205020920502
957 0.0208986415882968
958 0.0208768267223382
959 0.0208550573514077
960 0.0208333333333333
961 0.0208116545265349
962 0.0207900207900208
963 0.0207684319833853
964 0.020746887966805
965 0.0207253886010363
966 0.020703933747412
967 0.0206825232678387
968 0.0206611570247934
969 0.0206398348813209
970 0.0206185567010309
971 0.0205973223480947
972 0.0205761316872428
973 0.0205549845837616
974 0.0205338809034908
975 0.0205128205128205
976 0.0204918032786885
977 0.0204708290685773
978 0.0204498977505112
979 0.0204290091930541
980 0.0204081632653061
981 0.0214067278287462
982 0.0213849287169043
983 0.0213631739572737
984 0.0213414634146341
985 0.0213197969543147
986 0.0212981744421907
987 0.0212765957446809
988 0.0212550607287449
989 0.0212335692618807
990 0.0212121212121212
991 0.0221997981836529
992 0.0221774193548387
993 0.0221550855991944
994 0.0221327967806841
995 0.0221105527638191
996 0.0220883534136546
997 0.0220661985957874
998 0.0220440881763527
999 0.022022022022022
1000 0.022
};
\addplot [semithick, color3, forget plot]
table {%
1 0
2 0
3 0
4 0
5 0
6 0
7 0
8 0
9 0
10 0
11 0
12 0
13 0
14 0
15 0.0666666666666667
16 0.0625
17 0.0588235294117647
18 0.0555555555555556
19 0.0526315789473684
20 0.05
21 0.0476190476190476
22 0.0454545454545455
23 0.0434782608695652
24 0.0416666666666667
25 0.04
26 0.0384615384615385
27 0.037037037037037
28 0.0357142857142857
29 0.0344827586206897
30 0.0333333333333333
31 0.032258064516129
32 0.03125
33 0.0303030303030303
34 0.0294117647058824
35 0.0285714285714286
36 0.0277777777777778
37 0.027027027027027
38 0.0263157894736842
39 0.0256410256410256
40 0.025
41 0.024390243902439
42 0.0238095238095238
43 0.0232558139534884
44 0.0227272727272727
45 0.0222222222222222
46 0.0217391304347826
47 0.0212765957446809
48 0.0208333333333333
49 0.0204081632653061
50 0.02
51 0.0196078431372549
52 0.0192307692307692
53 0.0188679245283019
54 0.0185185185185185
55 0.0181818181818182
56 0.0178571428571429
57 0.0175438596491228
58 0.0172413793103448
59 0.0169491525423729
60 0.0166666666666667
61 0.0163934426229508
62 0.0161290322580645
63 0.0158730158730159
64 0.015625
65 0.0153846153846154
66 0.0151515151515152
67 0.0149253731343284
68 0.0147058823529412
69 0.0144927536231884
70 0.0142857142857143
71 0.0140845070422535
72 0.0138888888888889
73 0.0136986301369863
74 0.0135135135135135
75 0.0133333333333333
76 0.0131578947368421
77 0.012987012987013
78 0.0128205128205128
79 0.0126582278481013
80 0.0125
81 0.0123456790123457
82 0.0121951219512195
83 0.0120481927710843
84 0.0119047619047619
85 0.0117647058823529
86 0.0116279069767442
87 0.0114942528735632
88 0.0113636363636364
89 0.0112359550561798
90 0.0111111111111111
91 0.010989010989011
92 0.0108695652173913
93 0.010752688172043
94 0.0106382978723404
95 0.0105263157894737
96 0.0104166666666667
97 0.0103092783505155
98 0.0102040816326531
99 0.0101010101010101
100 0.01
101 0.0099009900990099
102 0.00980392156862745
103 0.00970873786407767
104 0.00961538461538462
105 0.00952380952380952
106 0.00943396226415094
107 0.00934579439252336
108 0.00925925925925926
109 0.00917431192660551
110 0.00909090909090909
111 0.00900900900900901
112 0.00892857142857143
113 0.00884955752212389
114 0.0087719298245614
115 0.00869565217391304
116 0.00862068965517241
117 0.00854700854700855
118 0.00847457627118644
119 0.00840336134453781
120 0.00833333333333333
121 0.00826446280991736
122 0.00819672131147541
123 0.00813008130081301
124 0.0161290322580645
125 0.016
126 0.0158730158730159
127 0.015748031496063
128 0.015625
129 0.0155038759689922
130 0.0153846153846154
131 0.0152671755725191
132 0.0151515151515152
133 0.0150375939849624
134 0.0149253731343284
135 0.0148148148148148
136 0.0147058823529412
137 0.0145985401459854
138 0.0144927536231884
139 0.0143884892086331
140 0.0142857142857143
141 0.0141843971631206
142 0.0140845070422535
143 0.013986013986014
144 0.0138888888888889
145 0.0137931034482759
146 0.0136986301369863
147 0.0136054421768707
148 0.0202702702702703
149 0.0201342281879195
150 0.02
151 0.0198675496688742
152 0.0197368421052632
153 0.0196078431372549
154 0.0194805194805195
155 0.0193548387096774
156 0.0192307692307692
157 0.0191082802547771
158 0.0189873417721519
159 0.0188679245283019
160 0.01875
161 0.0186335403726708
162 0.0185185185185185
163 0.0184049079754601
164 0.0182926829268293
165 0.0181818181818182
166 0.0180722891566265
167 0.0179640718562874
168 0.0178571428571429
169 0.0177514792899408
170 0.0235294117647059
171 0.0233918128654971
172 0.0232558139534884
173 0.023121387283237
174 0.0229885057471264
175 0.0228571428571429
176 0.0227272727272727
177 0.0225988700564972
178 0.0224719101123595
179 0.0223463687150838
180 0.0222222222222222
181 0.0220994475138122
182 0.021978021978022
183 0.0218579234972678
184 0.0217391304347826
185 0.0216216216216216
186 0.021505376344086
187 0.0213903743315508
188 0.0212765957446809
189 0.0211640211640212
190 0.0210526315789474
191 0.0209424083769634
192 0.0208333333333333
193 0.0207253886010363
194 0.0206185567010309
195 0.0205128205128205
196 0.0204081632653061
197 0.0203045685279188
198 0.0202020202020202
199 0.0251256281407035
200 0.025
201 0.0248756218905473
202 0.0247524752475248
203 0.0246305418719212
204 0.0245098039215686
205 0.024390243902439
206 0.0242718446601942
207 0.0241545893719807
208 0.0240384615384615
209 0.0239234449760766
210 0.0238095238095238
211 0.023696682464455
212 0.0235849056603774
213 0.0234741784037559
214 0.0233644859813084
215 0.0232558139534884
216 0.0231481481481481
217 0.0230414746543779
218 0.0229357798165138
219 0.0228310502283105
220 0.0227272727272727
221 0.0226244343891403
222 0.0225225225225225
223 0.0224215246636771
224 0.0223214285714286
225 0.0222222222222222
226 0.0221238938053097
227 0.0220264317180617
228 0.0219298245614035
229 0.0262008733624454
230 0.0260869565217391
231 0.025974025974026
232 0.0258620689655172
233 0.0257510729613734
234 0.0256410256410256
235 0.025531914893617
236 0.0254237288135593
237 0.0253164556962025
238 0.0252100840336134
239 0.0251046025104603
240 0.025
241 0.024896265560166
242 0.0247933884297521
243 0.0246913580246914
244 0.0245901639344262
245 0.0244897959183673
246 0.024390243902439
247 0.0242914979757085
248 0.0241935483870968
249 0.0240963855421687
250 0.024
251 0.0278884462151394
252 0.0277777777777778
253 0.0276679841897233
254 0.0275590551181102
255 0.0274509803921569
256 0.02734375
257 0.0272373540856031
258 0.0271317829457364
259 0.027027027027027
260 0.0269230769230769
261 0.0268199233716475
262 0.0267175572519084
263 0.026615969581749
264 0.0265151515151515
265 0.0264150943396226
266 0.0263157894736842
267 0.0262172284644195
268 0.0261194029850746
269 0.0260223048327138
270 0.0259259259259259
271 0.025830258302583
272 0.0257352941176471
273 0.0256410256410256
274 0.0255474452554745
275 0.0254545454545455
276 0.0253623188405797
277 0.0252707581227437
278 0.0251798561151079
279 0.025089605734767
280 0.025
281 0.0249110320284698
282 0.024822695035461
283 0.0247349823321555
284 0.0246478873239437
285 0.0245614035087719
286 0.0244755244755245
287 0.024390243902439
288 0.0243055555555556
289 0.0242214532871972
290 0.0241379310344828
291 0.0240549828178694
292 0.023972602739726
293 0.0238907849829352
294 0.0238095238095238
295 0.023728813559322
296 0.0236486486486486
297 0.0235690235690236
298 0.023489932885906
299 0.0234113712374582
300 0.0233333333333333
301 0.0232558139534884
302 0.0231788079470199
303 0.0231023102310231
304 0.0230263157894737
305 0.0229508196721311
306 0.0228758169934641
307 0.0228013029315961
308 0.0227272727272727
309 0.0226537216828479
310 0.0225806451612903
311 0.022508038585209
312 0.0224358974358974
313 0.0223642172523962
314 0.0222929936305732
315 0.0222222222222222
316 0.0221518987341772
317 0.0220820189274448
318 0.0220125786163522
319 0.0219435736677116
320 0.021875
321 0.0218068535825545
322 0.0217391304347826
323 0.021671826625387
324 0.0216049382716049
325 0.0215384615384615
326 0.0214723926380368
327 0.0214067278287462
328 0.0213414634146341
329 0.0212765957446809
330 0.0212121212121212
331 0.0211480362537764
332 0.0210843373493976
333 0.024024024024024
334 0.0239520958083832
335 0.0238805970149254
336 0.0238095238095238
337 0.0237388724035608
338 0.0236686390532544
339 0.023598820058997
340 0.0235294117647059
341 0.0234604105571848
342 0.0233918128654971
343 0.0233236151603499
344 0.0232558139534884
345 0.0231884057971014
346 0.023121387283237
347 0.0230547550432277
348 0.0229885057471264
349 0.0229226361031519
350 0.0228571428571429
351 0.0227920227920228
352 0.0227272727272727
353 0.0226628895184136
354 0.0225988700564972
355 0.0225352112676056
356 0.0224719101123595
357 0.0224089635854342
358 0.0223463687150838
359 0.0222841225626741
360 0.0222222222222222
361 0.0221606648199446
362 0.0220994475138122
363 0.0220385674931129
364 0.021978021978022
365 0.0219178082191781
366 0.0245901639344262
367 0.0272479564032698
368 0.0271739130434783
369 0.02710027100271
370 0.027027027027027
371 0.0269541778975741
372 0.0268817204301075
373 0.0268096514745308
374 0.0267379679144385
375 0.0293333333333333
376 0.0292553191489362
377 0.0291777188328912
378 0.0291005291005291
379 0.029023746701847
380 0.0289473684210526
381 0.0288713910761155
382 0.0287958115183246
383 0.0287206266318538
384 0.0286458333333333
385 0.0285714285714286
386 0.0284974093264249
387 0.0284237726098191
388 0.0283505154639175
389 0.0282776349614396
390 0.0282051282051282
391 0.0281329923273657
392 0.0280612244897959
393 0.0279898218829517
394 0.0279187817258883
395 0.0278481012658228
396 0.0277777777777778
397 0.0277078085642317
398 0.0276381909547739
399 0.0275689223057644
400 0.0275
401 0.027431421446384
402 0.027363184079602
403 0.0272952853598015
404 0.0272277227722772
405 0.0271604938271605
406 0.0270935960591133
407 0.027027027027027
408 0.0269607843137255
409 0.0268948655256724
410 0.0268292682926829
411 0.0267639902676399
412 0.0266990291262136
413 0.026634382566586
414 0.0265700483091787
415 0.0265060240963855
416 0.0264423076923077
417 0.026378896882494
418 0.0263157894736842
419 0.0262529832935561
420 0.0261904761904762
421 0.0261282660332542
422 0.0260663507109005
423 0.0260047281323877
424 0.0259433962264151
425 0.0258823529411765
426 0.0258215962441315
427 0.0257611241217799
428 0.0257009345794393
429 0.0256410256410256
430 0.0255813953488372
431 0.0255220417633411
432 0.025462962962963
433 0.0254041570438799
434 0.0253456221198157
435 0.0252873563218391
436 0.0252293577981651
437 0.0251716247139588
438 0.0251141552511416
439 0.0250569476082005
440 0.025
441 0.0249433106575964
442 0.0248868778280543
443 0.0248306997742664
444 0.0247747747747748
445 0.0247191011235955
446 0.0246636771300448
447 0.0246085011185682
448 0.0245535714285714
449 0.0244988864142539
450 0.0244444444444444
451 0.024390243902439
452 0.0243362831858407
453 0.0242825607064018
454 0.0242290748898678
455 0.0241758241758242
456 0.0241228070175439
457 0.0240700218818381
458 0.0240174672489083
459 0.0239651416122004
460 0.0239130434782609
461 0.0238611713665944
462 0.0238095238095238
463 0.0237580993520518
464 0.0237068965517241
465 0.0236559139784946
466 0.0236051502145923
467 0.0235546038543897
468 0.0235042735042735
469 0.023454157782516
470 0.0234042553191489
471 0.0233545647558386
472 0.0233050847457627
473 0.0232558139534884
474 0.0232067510548523
475 0.0231578947368421
476 0.023109243697479
477 0.0230607966457023
478 0.0230125523012552
479 0.022964509394572
480 0.0229166666666667
481 0.0228690228690229
482 0.0228215767634855
483 0.0227743271221532
484 0.0227272727272727
485 0.022680412371134
486 0.0226337448559671
487 0.0225872689938398
488 0.0225409836065574
489 0.0224948875255624
490 0.0224489795918367
491 0.0224032586558045
492 0.0223577235772358
493 0.0223123732251521
494 0.0222672064777328
495 0.0222222222222222
496 0.0221774193548387
497 0.0221327967806841
498 0.0220883534136546
499 0.0220440881763527
500 0.022
501 0.0219560878243513
502 0.0219123505976096
503 0.0218687872763419
504 0.0218253968253968
505 0.0217821782178218
506 0.0217391304347826
507 0.0216962524654832
508 0.0216535433070866
509 0.0216110019646365
510 0.0215686274509804
511 0.0215264187866928
512 0.021484375
513 0.0214424951267057
514 0.0214007782101167
515 0.0213592233009709
516 0.0213178294573643
517 0.0212765957446809
518 0.0212355212355212
519 0.0211946050096339
520 0.0211538461538462
521 0.0211132437619962
522 0.0229885057471264
523 0.0229445506692161
524 0.0229007633587786
525 0.0228571428571429
526 0.0228136882129278
527 0.0227703984819734
528 0.0227272727272727
529 0.0226843100189036
530 0.0226415094339623
531 0.0225988700564972
532 0.0225563909774436
533 0.0225140712945591
534 0.0224719101123595
535 0.0224299065420561
536 0.0223880597014925
537 0.0223463687150838
538 0.0223048327137546
539 0.0222634508348794
540 0.0222222222222222
541 0.022181146025878
542 0.022140221402214
543 0.0220994475138122
544 0.0220588235294118
545 0.0220183486238532
546 0.021978021978022
547 0.0219378427787934
548 0.0218978102189781
549 0.0218579234972678
550 0.0218181818181818
551 0.0217785843920145
552 0.0217391304347826
553 0.0216998191681736
554 0.0216606498194946
555 0.0216216216216216
556 0.0215827338129496
557 0.0215439856373429
558 0.021505376344086
559 0.0214669051878354
560 0.0214285714285714
561 0.0213903743315508
562 0.0213523131672598
563 0.0213143872113677
564 0.0212765957446809
565 0.0212389380530973
566 0.0212014134275618
567 0.0211640211640212
568 0.0211267605633803
569 0.0210896309314587
570 0.0210526315789474
571 0.021015761821366
572 0.020979020979021
573 0.0209424083769634
574 0.0209059233449477
575 0.0208695652173913
576 0.0208333333333333
577 0.0207972270363951
578 0.0207612456747405
579 0.0207253886010363
580 0.0206896551724138
581 0.0206540447504303
582 0.0206185567010309
583 0.0205831903945112
584 0.0205479452054795
585 0.0205128205128205
586 0.0204778156996587
587 0.020442930153322
588 0.0204081632653061
589 0.0203735144312394
590 0.0203389830508475
591 0.0203045685279188
592 0.0202702702702703
593 0.0202360876897133
594 0.0202020202020202
595 0.0201680672268908
596 0.0201342281879195
597 0.0201005025125628
598 0.020066889632107
599 0.0200333889816361
600 0.02
601 0.0199667221297837
602 0.0199335548172757
603 0.0199004975124378
604 0.0198675496688742
605 0.0198347107438017
606 0.0198019801980198
607 0.0197693574958814
608 0.0197368421052632
609 0.0197044334975369
610 0.019672131147541
611 0.0196399345335516
612 0.0196078431372549
613 0.0195758564437194
614 0.0195439739413681
615 0.0195121951219512
616 0.0194805194805195
617 0.0194489465153971
618 0.0194174757281553
619 0.0193861066235864
620 0.0193548387096774
621 0.0193236714975845
622 0.0192926045016077
623 0.0192616372391653
624 0.0192307692307692
625 0.0192
626 0.0191693290734824
627 0.0191387559808612
628 0.0191082802547771
629 0.0190779014308426
630 0.019047619047619
631 0.0190174326465927
632 0.0189873417721519
633 0.018957345971564
634 0.0189274447949527
635 0.0188976377952756
636 0.0188679245283019
637 0.0188383045525903
638 0.0188087774294671
639 0.0187793427230047
640 0.01875
641 0.0187207488299532
642 0.0186915887850467
643 0.0186625194401244
644 0.0186335403726708
645 0.0186046511627907
646 0.0185758513931889
647 0.0185471406491499
648 0.0200617283950617
649 0.0200308166409861
650 0.02
651 0.0199692780337942
652 0.0199386503067485
653 0.0199081163859112
654 0.0198776758409786
655 0.0198473282442748
656 0.0198170731707317
657 0.0197869101978691
658 0.0197568389057751
659 0.0197268588770865
660 0.0196969696969697
661 0.0196671709531014
662 0.0196374622356495
663 0.0196078431372549
664 0.019578313253012
665 0.0195488721804511
666 0.021021021021021
667 0.0209895052473763
668 0.0209580838323353
669 0.0209267563527653
670 0.0208955223880597
671 0.0208643815201192
672 0.0208333333333333
673 0.0208023774145617
674 0.0207715133531157
675 0.0207407407407407
676 0.0207100591715976
677 0.0206794682422452
678 0.0206489675516224
679 0.0206185567010309
680 0.0205882352941176
681 0.0205580029368576
682 0.0205278592375367
683 0.020497803806735
684 0.0204678362573099
685 0.0204379562043796
686 0.0204081632653061
687 0.0203784570596798
688 0.0203488372093023
689 0.0203193033381713
690 0.0202898550724638
691 0.020260492040521
692 0.0202312138728324
693 0.0202020202020202
694 0.0201729106628242
695 0.0201438848920863
696 0.0201149425287356
697 0.0200860832137733
698 0.0200573065902579
699 0.0200286123032904
700 0.02
701 0.0199714693295292
702 0.0199430199430199
703 0.0199146514935989
704 0.0198863636363636
705 0.0198581560283688
706 0.0198300283286119
707 0.0198019801980198
708 0.019774011299435
709 0.0197461212976023
710 0.0197183098591549
711 0.019690576652602
712 0.0196629213483146
713 0.0196353436185133
714 0.0196078431372549
715 0.0195804195804196
716 0.0195530726256983
717 0.0195258019525802
718 0.0194986072423398
719 0.019471488178025
720 0.0194444444444444
721 0.0194174757281553
722 0.0193905817174515
723 0.020746887966805
724 0.0207182320441989
725 0.0206896551724138
726 0.0206611570247934
727 0.0206327372764787
728 0.0206043956043956
729 0.0205761316872428
730 0.0205479452054795
731 0.0205198358413133
732 0.0204918032786885
733 0.0204638472032742
734 0.0204359673024523
735 0.0204081632653061
736 0.0203804347826087
737 0.0203527815468114
738 0.0203252032520325
739 0.020297699594046
740 0.0202702702702703
741 0.0202429149797571
742 0.0202156334231806
743 0.0201884253028264
744 0.0201612903225806
745 0.0201342281879195
746 0.0201072386058981
747 0.0200803212851406
748 0.0200534759358289
749 0.0200267022696929
750 0.02
751 0.0199733688415446
752 0.0199468085106383
753 0.0199203187250996
754 0.019893899204244
755 0.0198675496688742
756 0.0198412698412698
757 0.0198150594451783
758 0.0197889182058048
759 0.0197628458498024
760 0.0197368421052632
761 0.0197109067017083
762 0.0196850393700787
763 0.0196592398427261
764 0.0196335078534031
765 0.0196078431372549
766 0.0195822454308094
767 0.0195567144719687
768 0.01953125
769 0.0195058517555267
770 0.0194805194805195
771 0.0194552529182879
772 0.0194300518134715
773 0.019404915912031
774 0.0193798449612403
775 0.0193548387096774
776 0.0193298969072165
777 0.0193050193050193
778 0.019280205655527
779 0.0192554557124519
780 0.0205128205128205
781 0.0204865556978233
782 0.020460358056266
783 0.0204342273307791
784 0.0204081632653061
785 0.0203821656050955
786 0.0203562340966921
787 0.0203303684879288
788 0.0203045685279188
789 0.0202788339670469
790 0.020253164556962
791 0.0202275600505689
792 0.0202020202020202
793 0.0201765447667087
794 0.0201511335012594
795 0.020125786163522
796 0.0201005025125628
797 0.0200752823086575
798 0.0200501253132832
799 0.0200250312891114
800 0.02125
801 0.0212234706616729
802 0.0211970074812968
803 0.0211706102117061
804 0.0211442786069652
805 0.0211180124223602
806 0.0210918114143921
807 0.0210656753407683
808 0.021039603960396
809 0.0210135970333745
810 0.0209876543209877
811 0.0209617755856967
812 0.020935960591133
813 0.020910209102091
814 0.0208845208845209
815 0.0208588957055215
816 0.0208333333333333
817 0.0208078335373317
818 0.0207823960880196
819 0.0207570207570208
820 0.0207317073170732
821 0.0207064555420219
822 0.0206812652068127
823 0.0206561360874848
824 0.020631067961165
825 0.0206060606060606
826 0.0205811138014528
827 0.0205562273276904
828 0.0205314009661836
829 0.0205066344993969
830 0.0204819277108434
831 0.0204572803850782
832 0.0204326923076923
833 0.0204081632653061
834 0.0203836930455636
835 0.0203592814371257
836 0.0203349282296651
837 0.020310633213859
838 0.0202863961813842
839 0.0202622169249106
840 0.0202380952380952
841 0.0202140309155767
842 0.0201900237529691
843 0.0201660735468565
844 0.0201421800947867
845 0.0201183431952663
846 0.0200945626477541
847 0.0200708382526564
848 0.0200471698113208
849 0.0200235571260306
850 0.02
851 0.0199764982373678
852 0.0199530516431925
853 0.0199296600234467
854 0.0199063231850117
855 0.0198830409356725
856 0.0198598130841121
857 0.0198366394399067
858 0.0198135198135198
859 0.019790454016298
860 0.0197674418604651
861 0.0197444831591173
862 0.0197215777262181
863 0.0196987253765933
864 0.0196759259259259
865 0.0196531791907514
866 0.0196304849884527
867 0.0196078431372549
868 0.0195852534562212
869 0.0195627157652474
870 0.0195402298850575
871 0.0195177956371986
872 0.0194954128440367
873 0.0194730813287514
874 0.0194508009153318
875 0.0194285714285714
876 0.0194063926940639
877 0.0193842645381984
878 0.0193621867881549
879 0.0193401592718999
880 0.0193181818181818
881 0.0192962542565267
882 0.0192743764172336
883 0.0192525481313703
884 0.0192307692307692
885 0.0192090395480226
886 0.0191873589164786
887 0.0191657271702368
888 0.0191441441441441
889 0.0191226096737908
890 0.0191011235955056
891 0.0190796857463524
892 0.0190582959641256
893 0.019036954087346
894 0.0190156599552573
895 0.0189944134078212
896 0.0189732142857143
897 0.0189520624303233
898 0.0189309576837416
899 0.0189098998887653
900 0.0188888888888889
901 0.0188679245283019
902 0.0188470066518847
903 0.0188261351052049
904 0.0188053097345133
905 0.0187845303867403
906 0.0198675496688742
907 0.019845644983462
908 0.0209251101321586
909 0.0209020902090209
910 0.0208791208791209
911 0.0208562019758507
912 0.0208333333333333
913 0.0208105147864184
914 0.0207877461706783
915 0.0207650273224044
916 0.0207423580786026
917 0.0207197382769902
918 0.0206971677559913
919 0.0206746463547334
920 0.0206521739130435
921 0.0206297502714441
922 0.0206073752711497
923 0.0205850487540628
924 0.0205627705627706
925 0.0205405405405405
926 0.0205183585313175
927 0.0215749730312837
928 0.021551724137931
929 0.0215285252960172
930 0.021505376344086
931 0.0214822771213749
932 0.0214592274678112
933 0.0214362272240086
934 0.0214132762312634
935 0.0213903743315508
936 0.0213675213675214
937 0.0213447171824973
938 0.0213219616204691
939 0.0212992545260916
940 0.0212765957446809
941 0.0212539851222104
942 0.0212314225053079
943 0.0212089077412513
944 0.0211864406779661
945 0.0211640211640212
946 0.0211416490486258
947 0.0211193241816262
948 0.0210970464135021
949 0.0210748155953635
950 0.0210526315789474
951 0.0210304942166141
952 0.0210084033613445
953 0.0209863588667366
954 0.0209643605870021
955 0.0209424083769634
956 0.0209205020920502
957 0.0208986415882968
958 0.0219206680584551
959 0.0218978102189781
960 0.021875
961 0.0218522372528616
962 0.0218295218295218
963 0.0218068535825545
964 0.0217842323651452
965 0.0217616580310881
966 0.0217391304347826
967 0.0217166494312306
968 0.0216942148760331
969 0.021671826625387
970 0.0216494845360825
971 0.0216271884654995
972 0.0216049382716049
973 0.0215827338129496
974 0.0215605749486653
975 0.0215384615384615
976 0.021516393442623
977 0.0214943705220061
978 0.0214723926380368
979 0.0214504596527068
980 0.0214285714285714
981 0.0214067278287462
982 0.0213849287169043
983 0.0213631739572737
984 0.0213414634146341
985 0.0213197969543147
986 0.0212981744421907
987 0.0212765957446809
988 0.0212550607287449
989 0.0212335692618807
990 0.0212121212121212
991 0.0211907164480323
992 0.0211693548387097
993 0.0211480362537764
994 0.0211267605633803
995 0.021105527638191
996 0.0210843373493976
997 0.0220661985957874
998 0.0220440881763527
999 0.022022022022022
1000 0.022
};
\addplot [semithick, color4, forget plot]
table {%
1 0
2 0
3 0
4 0
5 0
6 0
7 0
8 0
9 0
10 0
11 0
12 0
13 0
14 0
15 0
16 0.0625
17 0.0588235294117647
18 0.0555555555555556
19 0.0526315789473684
20 0.05
21 0.0476190476190476
22 0.0454545454545455
23 0.0434782608695652
24 0.0416666666666667
25 0.04
26 0.0384615384615385
27 0.037037037037037
28 0.0357142857142857
29 0.0344827586206897
30 0.0333333333333333
31 0.032258064516129
32 0.03125
33 0.0303030303030303
34 0.0294117647058824
35 0.0285714285714286
36 0.0277777777777778
37 0.027027027027027
38 0.0263157894736842
39 0.0512820512820513
40 0.05
41 0.0487804878048781
42 0.0476190476190476
43 0.0465116279069767
44 0.0454545454545455
45 0.0444444444444444
46 0.0434782608695652
47 0.0425531914893617
48 0.0416666666666667
49 0.0408163265306122
50 0.04
51 0.0392156862745098
52 0.0384615384615385
53 0.0377358490566038
54 0.037037037037037
55 0.0363636363636364
56 0.0357142857142857
57 0.0350877192982456
58 0.0344827586206897
59 0.0338983050847458
60 0.0333333333333333
61 0.0327868852459016
62 0.032258064516129
63 0.0317460317460317
64 0.03125
65 0.0307692307692308
66 0.0303030303030303
67 0.0298507462686567
68 0.0294117647058824
69 0.0289855072463768
70 0.0285714285714286
71 0.028169014084507
72 0.0277777777777778
73 0.0273972602739726
74 0.027027027027027
75 0.0266666666666667
76 0.0263157894736842
77 0.025974025974026
78 0.0256410256410256
79 0.0253164556962025
80 0.025
81 0.0246913580246914
82 0.024390243902439
83 0.0240963855421687
84 0.0238095238095238
85 0.0235294117647059
86 0.0232558139534884
87 0.0229885057471264
88 0.0227272727272727
89 0.0224719101123595
90 0.0222222222222222
91 0.021978021978022
92 0.0217391304347826
93 0.021505376344086
94 0.0212765957446809
95 0.0210526315789474
96 0.0208333333333333
97 0.0206185567010309
98 0.0204081632653061
99 0.0202020202020202
100 0.02
101 0.0198019801980198
102 0.0196078431372549
103 0.0194174757281553
104 0.0192307692307692
105 0.019047619047619
106 0.0188679245283019
107 0.0186915887850467
108 0.0185185185185185
109 0.018348623853211
110 0.0181818181818182
111 0.018018018018018
112 0.0178571428571429
113 0.0176991150442478
114 0.0175438596491228
115 0.0173913043478261
116 0.0172413793103448
117 0.0170940170940171
118 0.0169491525423729
119 0.0168067226890756
120 0.0166666666666667
121 0.0165289256198347
122 0.0163934426229508
123 0.016260162601626
124 0.0161290322580645
125 0.016
126 0.0158730158730159
127 0.015748031496063
128 0.015625
129 0.0155038759689922
130 0.0153846153846154
131 0.0152671755725191
132 0.0151515151515152
133 0.0150375939849624
134 0.0149253731343284
135 0.0148148148148148
136 0.0147058823529412
137 0.0145985401459854
138 0.0144927536231884
139 0.0143884892086331
140 0.0142857142857143
141 0.0141843971631206
142 0.0140845070422535
143 0.013986013986014
144 0.0138888888888889
145 0.0137931034482759
146 0.0136986301369863
147 0.0136054421768707
148 0.0135135135135135
149 0.0134228187919463
150 0.0133333333333333
151 0.0132450331125828
152 0.0131578947368421
153 0.0130718954248366
154 0.012987012987013
155 0.0129032258064516
156 0.0128205128205128
157 0.0127388535031847
158 0.0126582278481013
159 0.0125786163522013
160 0.0125
161 0.0124223602484472
162 0.0123456790123457
163 0.0122699386503067
164 0.0121951219512195
165 0.0121212121212121
166 0.0120481927710843
167 0.0119760479041916
168 0.0119047619047619
169 0.0118343195266272
170 0.0117647058823529
171 0.0116959064327485
172 0.0116279069767442
173 0.0115606936416185
174 0.0114942528735632
175 0.0114285714285714
176 0.0113636363636364
177 0.0112994350282486
178 0.0112359550561798
179 0.0167597765363128
180 0.0166666666666667
181 0.0165745856353591
182 0.0164835164835165
183 0.0163934426229508
184 0.016304347826087
185 0.0162162162162162
186 0.0161290322580645
187 0.0160427807486631
188 0.0159574468085106
189 0.0158730158730159
190 0.0157894736842105
191 0.0157068062827225
192 0.015625
193 0.0155440414507772
194 0.0154639175257732
195 0.0153846153846154
196 0.0153061224489796
197 0.0152284263959391
198 0.0151515151515152
199 0.0150753768844221
200 0.015
201 0.0149253731343284
202 0.0148514851485149
203 0.0147783251231527
204 0.0147058823529412
205 0.0146341463414634
206 0.0145631067961165
207 0.0144927536231884
208 0.0144230769230769
209 0.0143540669856459
210 0.0142857142857143
211 0.014218009478673
212 0.0141509433962264
213 0.0140845070422535
214 0.014018691588785
215 0.013953488372093
216 0.0138888888888889
217 0.0138248847926267
218 0.0137614678899083
219 0.0136986301369863
220 0.0136363636363636
221 0.0135746606334842
222 0.0135135135135135
223 0.0134529147982063
224 0.0133928571428571
225 0.0133333333333333
226 0.0132743362831858
227 0.013215859030837
228 0.0131578947368421
229 0.0131004366812227
230 0.0130434782608696
231 0.012987012987013
232 0.0129310344827586
233 0.0128755364806867
234 0.0128205128205128
235 0.0127659574468085
236 0.0127118644067797
237 0.0126582278481013
238 0.0126050420168067
239 0.0125523012552301
240 0.0125
241 0.012448132780083
242 0.012396694214876
243 0.0123456790123457
244 0.0122950819672131
245 0.0122448979591837
246 0.0121951219512195
247 0.0121457489878543
248 0.0120967741935484
249 0.0120481927710843
250 0.012
251 0.0119521912350598
252 0.0119047619047619
253 0.0118577075098814
254 0.0118110236220472
255 0.0117647058823529
256 0.01171875
257 0.0116731517509728
258 0.0116279069767442
259 0.0115830115830116
260 0.0115384615384615
261 0.0114942528735632
262 0.0114503816793893
263 0.0114068441064639
264 0.0113636363636364
265 0.0113207547169811
266 0.0112781954887218
267 0.0112359550561798
268 0.0111940298507463
269 0.0111524163568773
270 0.0111111111111111
271 0.011070110701107
272 0.0110294117647059
273 0.010989010989011
274 0.0109489051094891
275 0.0109090909090909
276 0.0108695652173913
277 0.0144404332129964
278 0.0143884892086331
279 0.017921146953405
280 0.0178571428571429
281 0.0177935943060498
282 0.0177304964539007
283 0.0176678445229682
284 0.0176056338028169
285 0.0175438596491228
286 0.0174825174825175
287 0.0174216027874564
288 0.0173611111111111
289 0.0173010380622837
290 0.0172413793103448
291 0.0171821305841924
292 0.0171232876712329
293 0.0170648464163823
294 0.0170068027210884
295 0.0169491525423729
296 0.0168918918918919
297 0.0168350168350168
298 0.0167785234899329
299 0.0167224080267559
300 0.0166666666666667
301 0.0166112956810631
302 0.0165562913907285
303 0.0165016501650165
304 0.0164473684210526
305 0.0163934426229508
306 0.0163398692810458
307 0.0162866449511401
308 0.0162337662337662
309 0.0161812297734628
310 0.0161290322580645
311 0.0192926045016077
312 0.0192307692307692
313 0.0191693290734824
314 0.0191082802547771
315 0.019047619047619
316 0.0189873417721519
317 0.0189274447949527
318 0.0188679245283019
319 0.0188087774294671
320 0.01875
321 0.0186915887850467
322 0.0186335403726708
323 0.0185758513931889
324 0.0185185185185185
325 0.0184615384615385
326 0.0184049079754601
327 0.018348623853211
328 0.0182926829268293
329 0.0182370820668693
330 0.0181818181818182
331 0.0181268882175227
332 0.0180722891566265
333 0.018018018018018
334 0.0179640718562874
335 0.017910447761194
336 0.0178571428571429
337 0.0178041543026706
338 0.0177514792899408
339 0.0176991150442478
340 0.0176470588235294
341 0.0175953079178886
342 0.0175438596491228
343 0.0174927113702624
344 0.0174418604651163
345 0.0173913043478261
346 0.0173410404624277
347 0.0172910662824208
348 0.0172413793103448
349 0.0171919770773639
350 0.0171428571428571
351 0.0170940170940171
352 0.0170454545454545
353 0.0169971671388102
354 0.0169491525423729
355 0.0169014084507042
356 0.0168539325842697
357 0.0168067226890756
358 0.0167597765363128
359 0.0167130919220056
360 0.0166666666666667
361 0.0166204986149584
362 0.0165745856353591
363 0.0165289256198347
364 0.0164835164835165
365 0.0164383561643836
366 0.0163934426229508
367 0.0163487738419619
368 0.016304347826087
369 0.016260162601626
370 0.0162162162162162
371 0.0161725067385445
372 0.0161290322580645
373 0.0160857908847185
374 0.0160427807486631
375 0.016
376 0.0159574468085106
377 0.0159151193633952
378 0.0158730158730159
379 0.0158311345646438
380 0.0157894736842105
381 0.015748031496063
382 0.0157068062827225
383 0.0156657963446475
384 0.015625
385 0.0155844155844156
386 0.0155440414507772
387 0.0155038759689922
388 0.0154639175257732
389 0.0154241645244216
390 0.0153846153846154
391 0.0179028132992327
392 0.0178571428571429
393 0.0178117048346056
394 0.0177664974619289
395 0.0177215189873418
396 0.0176767676767677
397 0.017632241813602
398 0.0175879396984925
399 0.0175438596491228
400 0.0175
401 0.0174563591022444
402 0.0174129353233831
403 0.0173697270471464
404 0.0173267326732673
405 0.0172839506172839
406 0.0172413793103448
407 0.0171990171990172
408 0.017156862745098
409 0.0171149144254279
410 0.0170731707317073
411 0.0170316301703163
412 0.0169902912621359
413 0.0169491525423729
414 0.0169082125603865
415 0.0168674698795181
416 0.0168269230769231
417 0.0167865707434053
418 0.0167464114832536
419 0.0167064439140811
420 0.0166666666666667
421 0.0166270783847981
422 0.0165876777251185
423 0.016548463356974
424 0.0165094339622642
425 0.0164705882352941
426 0.0164319248826291
427 0.0163934426229508
428 0.0163551401869159
429 0.0163170163170163
430 0.0162790697674419
431 0.0162412993039443
432 0.0162037037037037
433 0.0161662817551963
434 0.0161290322580645
435 0.0160919540229885
436 0.0160550458715596
437 0.0183066361556064
438 0.0182648401826484
439 0.0182232346241458
440 0.0181818181818182
441 0.018140589569161
442 0.0180995475113122
443 0.018058690744921
444 0.018018018018018
445 0.0179775280898876
446 0.0179372197309417
447 0.0178970917225951
448 0.0178571428571429
449 0.0178173719376392
450 0.0177777777777778
451 0.0177383592017738
452 0.0176991150442478
453 0.0198675496688742
454 0.0198237885462555
455 0.0197802197802198
456 0.0197368421052632
457 0.0196936542669584
458 0.0196506550218341
459 0.0196078431372549
460 0.0217391304347826
461 0.0216919739696312
462 0.0216450216450216
463 0.0215982721382289
464 0.021551724137931
465 0.021505376344086
466 0.0214592274678112
467 0.0214132762312634
468 0.0213675213675214
469 0.0213219616204691
470 0.0212765957446809
471 0.0212314225053079
472 0.0211864406779661
473 0.0211416490486258
474 0.0210970464135021
475 0.0210526315789474
476 0.0210084033613445
477 0.0209643605870021
478 0.0209205020920502
479 0.0208768267223382
480 0.0208333333333333
481 0.0207900207900208
482 0.020746887966805
483 0.020703933747412
484 0.0206611570247934
485 0.0206185567010309
486 0.0205761316872428
487 0.0205338809034908
488 0.0204918032786885
489 0.0204498977505112
490 0.0204081632653061
491 0.0203665987780041
492 0.0203252032520325
493 0.0202839756592292
494 0.0202429149797571
495 0.0202020202020202
496 0.0201612903225806
497 0.0201207243460765
498 0.0200803212851406
499 0.0200400801603206
500 0.02
501 0.0199600798403194
502 0.0199203187250996
503 0.0198807157057654
504 0.0198412698412698
505 0.0198019801980198
506 0.0197628458498024
507 0.019723865877712
508 0.0196850393700787
509 0.0196463654223969
510 0.0196078431372549
511 0.0195694716242661
512 0.01953125
513 0.0194931773879142
514 0.0194552529182879
515 0.0194174757281553
516 0.0193798449612403
517 0.0193423597678917
518 0.0193050193050193
519 0.0192678227360308
520 0.0192307692307692
521 0.0191938579654511
522 0.0191570881226054
523 0.0191204588910134
524 0.0190839694656489
525 0.019047619047619
526 0.0190114068441065
527 0.0189753320683112
528 0.0189393939393939
529 0.0189035916824197
530 0.0188679245283019
531 0.0188323917137476
532 0.018796992481203
533 0.0187617260787993
534 0.0187265917602996
535 0.0186915887850467
536 0.0186567164179104
537 0.0186219739292365
538 0.0185873605947955
539 0.0185528756957328
540 0.0185185185185185
541 0.0184842883548983
542 0.018450184501845
543 0.0184162062615101
544 0.0183823529411765
545 0.018348623853211
546 0.0183150183150183
547 0.0182815356489945
548 0.0182481751824818
549 0.0182149362477231
550 0.0181818181818182
551 0.0181488203266788
552 0.0181159420289855
553 0.0180831826401447
554 0.0180505415162455
555 0.018018018018018
556 0.0179856115107914
557 0.0179533213644524
558 0.017921146953405
559 0.0178890876565295
560 0.0196428571428571
561 0.0196078431372549
562 0.0195729537366548
563 0.019538188277087
564 0.0195035460992908
565 0.0194690265486726
566 0.019434628975265
567 0.0194003527336861
568 0.0193661971830986
569 0.0193321616871705
570 0.0192982456140351
571 0.021015761821366
572 0.020979020979021
573 0.0209424083769634
574 0.0209059233449477
575 0.0208695652173913
576 0.0208333333333333
577 0.0207972270363951
578 0.0207612456747405
579 0.0207253886010363
580 0.0206896551724138
581 0.0206540447504303
582 0.0206185567010309
583 0.0205831903945112
584 0.0205479452054795
585 0.0205128205128205
586 0.0204778156996587
587 0.020442930153322
588 0.0204081632653061
589 0.0203735144312394
590 0.0203389830508475
591 0.0203045685279188
592 0.0202702702702703
593 0.0202360876897133
594 0.0202020202020202
595 0.0201680672268908
596 0.0201342281879195
597 0.0201005025125628
598 0.020066889632107
599 0.0200333889816361
600 0.02
601 0.0199667221297837
602 0.0199335548172757
603 0.0199004975124378
604 0.0198675496688742
605 0.0198347107438017
606 0.0198019801980198
607 0.0197693574958814
608 0.0197368421052632
609 0.0197044334975369
610 0.019672131147541
611 0.0196399345335516
612 0.0196078431372549
613 0.0195758564437194
614 0.0195439739413681
615 0.0195121951219512
616 0.0194805194805195
617 0.0194489465153971
618 0.0194174757281553
619 0.0193861066235864
620 0.0193548387096774
621 0.0193236714975845
622 0.0192926045016077
623 0.0192616372391653
624 0.0192307692307692
625 0.0192
626 0.0191693290734824
627 0.0191387559808612
628 0.0191082802547771
629 0.0190779014308426
630 0.019047619047619
631 0.0190174326465927
632 0.0205696202531646
633 0.0205371248025276
634 0.0205047318611987
635 0.0204724409448819
636 0.020440251572327
637 0.0204081632653061
638 0.0203761755485893
639 0.0203442879499218
640 0.0203125
641 0.0202808112324493
642 0.0202492211838006
643 0.0202177293934681
644 0.0201863354037267
645 0.0201550387596899
646 0.0201238390092879
647 0.0200927357032458
648 0.0200617283950617
649 0.0200308166409861
650 0.02
651 0.0199692780337942
652 0.0199386503067485
653 0.0199081163859112
654 0.0198776758409786
655 0.0198473282442748
656 0.0198170731707317
657 0.0197869101978691
658 0.0197568389057751
659 0.0212443095599393
660 0.0212121212121212
661 0.0211800302571861
662 0.0211480362537764
663 0.0211161387631976
664 0.0210843373493976
665 0.0210526315789474
666 0.021021021021021
667 0.0209895052473763
668 0.0209580838323353
669 0.0209267563527653
670 0.0208955223880597
671 0.0208643815201192
672 0.0208333333333333
673 0.0208023774145617
674 0.0207715133531157
675 0.0207407407407407
676 0.0207100591715976
677 0.0206794682422452
678 0.0206489675516224
679 0.0206185567010309
680 0.0205882352941176
681 0.0205580029368576
682 0.0205278592375367
683 0.020497803806735
684 0.0204678362573099
685 0.0204379562043796
686 0.021865889212828
687 0.0218340611353712
688 0.0218023255813953
689 0.0217706821480406
690 0.0217391304347826
691 0.0217076700434153
692 0.0216763005780347
693 0.0216450216450216
694 0.0216138328530259
695 0.0215827338129496
696 0.021551724137931
697 0.0215208034433286
698 0.0214899713467049
699 0.0214592274678112
700 0.0214285714285714
701 0.021398002853067
702 0.0213675213675214
703 0.0213371266002845
704 0.0213068181818182
705 0.0212765957446809
706 0.0212464589235127
707 0.0212164073550212
708 0.0211864406779661
709 0.0225669957686883
710 0.0225352112676056
711 0.0239099859353024
712 0.023876404494382
713 0.0238429172510519
714 0.0238095238095238
715 0.0237762237762238
716 0.0237430167597765
717 0.0237099023709902
718 0.0236768802228412
719 0.023643949930459
720 0.0236111111111111
721 0.0235783633841886
722 0.0235457063711911
723 0.0235131396957123
724 0.0234806629834254
725 0.023448275862069
726 0.0234159779614325
727 0.0247592847317744
728 0.0247252747252747
729 0.0246913580246914
730 0.0246575342465753
731 0.0246238030095759
732 0.0245901639344262
733 0.0245566166439291
734 0.0245231607629428
735 0.0244897959183673
736 0.0244565217391304
737 0.0244233378561737
738 0.024390243902439
739 0.0243572395128552
740 0.0243243243243243
741 0.0242914979757085
742 0.0242587601078167
743 0.0242261103633917
744 0.0241935483870968
745 0.0241610738255034
746 0.0241286863270777
747 0.0240963855421687
748 0.0240641711229947
749 0.0240320427236315
750 0.024
751 0.0239680426098535
752 0.023936170212766
753 0.0239043824701195
754 0.0238726790450928
755 0.023841059602649
756 0.0238095238095238
757 0.023778071334214
758 0.0237467018469657
759 0.0237154150197628
760 0.0236842105263158
761 0.0236530880420499
762 0.0236220472440945
763 0.0235910878112713
764 0.0235602094240838
765 0.0235294117647059
766 0.0234986945169713
767 0.0234680573663625
768 0.0234375
769 0.023407022106632
770 0.0233766233766234
771 0.0233463035019455
772 0.0246113989637306
773 0.0245795601552393
774 0.0245478036175711
775 0.0245161290322581
776 0.0244845360824742
777 0.0244530244530245
778 0.0244215938303342
779 0.024390243902439
780 0.0243589743589744
781 0.0243277848911652
782 0.0242966751918159
783 0.0242656449553001
784 0.024234693877551
785 0.024203821656051
786 0.0241730279898219
787 0.0241423125794155
788 0.0241116751269036
789 0.0240811153358682
790 0.0240506329113924
791 0.0240202275600506
792 0.023989898989899
793 0.0239596469104666
794 0.0239294710327456
795 0.0238993710691824
796 0.0238693467336683
797 0.0238393977415307
798 0.0238095238095238
799 0.0237797246558198
800 0.02375
801 0.0237203495630462
802 0.0236907730673317
803 0.0236612702366127
804 0.0236318407960199
805 0.0236024844720497
806 0.0235732009925558
807 0.023543990086741
808 0.0235148514851485
809 0.0234857849196539
810 0.0234567901234568
811 0.0234278668310728
812 0.0233990147783251
813 0.023370233702337
814 0.0233415233415233
815 0.0233128834355828
816 0.0232843137254902
817 0.0232558139534884
818 0.0232273838630807
819 0.0231990231990232
820 0.0231707317073171
821 0.023142509135201
822 0.0231143552311436
823 0.023086269744836
824 0.0230582524271845
825 0.023030303030303
826 0.0230024213075061
827 0.0229746070133011
828 0.0229468599033816
829 0.02291917973462
830 0.0228915662650602
831 0.022864019253911
832 0.0228365384615385
833 0.0228091236494598
834 0.0227817745803357
835 0.0227544910179641
836 0.0227272727272727
837 0.022700119474313
838 0.022673031026253
839 0.0226460071513707
840 0.0226190476190476
841 0.0225921521997622
842 0.0225653206650831
843 0.0225385527876631
844 0.0225118483412322
845 0.0224852071005917
846 0.0224586288416076
847 0.0224321133412042
848 0.0224056603773585
849 0.0223792697290931
850 0.0223529411764706
851 0.0223266745005875
852 0.0223004694835681
853 0.022274325908558
854 0.022248243559719
855 0.0222222222222222
856 0.022196261682243
857 0.0221703617269545
858 0.0221445221445221
859 0.0221187427240978
860 0.022093023255814
861 0.0220673635307782
862 0.0220417633410673
863 0.0220162224797219
864 0.0219907407407407
865 0.0219653179190751
866 0.0219399538106236
867 0.0219146482122261
868 0.021889400921659
869 0.0218642117376295
870 0.0218390804597701
871 0.0218140068886338
872 0.0217889908256881
873 0.0217640320733104
874 0.0217391304347826
875 0.0217142857142857
876 0.021689497716895
877 0.0216647662485747
878 0.0216400911161731
879 0.0216154721274175
880 0.0215909090909091
881 0.021566401816118
882 0.0215419501133787
883 0.0215175537938845
884 0.0214932126696833
885 0.0214689265536723
886 0.0214446952595937
887 0.0214205186020293
888 0.0213963963963964
889 0.0213723284589426
890 0.0213483146067416
891 0.021324354657688
892 0.0213004484304933
893 0.0212765957446809
894 0.0212527964205817
895 0.0212290502793296
896 0.0212053571428571
897 0.0211817168338907
898 0.0211581291759465
899 0.0211345939933259
900 0.0211111111111111
901 0.0210876803551609
902 0.0210643015521064
903 0.0210409745293466
904 0.0210176991150442
905 0.0209944751381215
906 0.0209713024282561
907 0.0209481808158765
908 0.0209251101321586
909 0.022002200220022
910 0.021978021978022
911 0.021953896816685
912 0.0219298245614035
913 0.0219058050383352
914 0.0218818380743982
915 0.0218579234972678
916 0.0218340611353712
917 0.0218102508178844
918 0.0217864923747277
919 0.0217627856365615
920 0.0217391304347826
921 0.0217155266015201
922 0.0216919739696312
923 0.0216684723726977
924 0.0216450216450216
925 0.0216216216216216
926 0.0215982721382289
927 0.0215749730312837
928 0.021551724137931
929 0.0215285252960172
930 0.0225806451612903
931 0.0225563909774436
932 0.0225321888412017
933 0.022508038585209
934 0.0224839400428266
935 0.0224598930481283
936 0.0224358974358974
937 0.0224119530416222
938 0.0223880597014925
939 0.0223642172523962
940 0.0223404255319149
941 0.0223166843783209
942 0.0222929936305732
943 0.0222693531283139
944 0.0222457627118644
945 0.0222222222222222
946 0.0221987315010571
947 0.0221752903907075
948 0.0221518987341772
949 0.0221285563751317
950 0.0221052631578947
951 0.0220820189274448
952 0.0220588235294118
953 0.0220356768100735
954 0.0220125786163522
955 0.0219895287958115
956 0.0219665271966527
957 0.0219435736677116
958 0.0219206680584551
959 0.0218978102189781
960 0.021875
961 0.0218522372528616
962 0.0218295218295218
963 0.0218068535825545
964 0.0217842323651452
965 0.0217616580310881
966 0.0217391304347826
967 0.0217166494312306
968 0.0216942148760331
969 0.021671826625387
970 0.0216494845360825
971 0.0216271884654995
972 0.0216049382716049
973 0.0215827338129496
974 0.0215605749486653
975 0.0215384615384615
976 0.021516393442623
977 0.0214943705220061
978 0.0214723926380368
979 0.0214504596527068
980 0.0214285714285714
981 0.0214067278287462
982 0.0213849287169043
983 0.0213631739572737
984 0.0213414634146341
985 0.0213197969543147
986 0.0212981744421907
987 0.0212765957446809
988 0.0212550607287449
989 0.0212335692618807
990 0.0212121212121212
991 0.0211907164480323
992 0.0211693548387097
993 0.0211480362537764
994 0.0211267605633803
995 0.021105527638191
996 0.0210843373493976
997 0.0210631895687061
998 0.0220440881763527
999 0.022022022022022
1000 0.022
};
\addplot [semithick, color5, forget plot]
table {%
1 0
2 0
3 0
4 0
5 0
6 0
7 0
8 0
9 0
10 0
11 0
12 0
13 0
14 0
15 0
16 0
17 0
18 0
19 0
20 0
21 0
22 0
23 0
24 0
25 0
26 0
27 0
28 0
29 0
30 0
31 0
32 0
33 0
34 0
35 0
36 0.0277777777777778
37 0.027027027027027
38 0.0263157894736842
39 0.0256410256410256
40 0.025
41 0.024390243902439
42 0.0238095238095238
43 0.0232558139534884
44 0.0227272727272727
45 0.0222222222222222
46 0.0217391304347826
47 0.0212765957446809
48 0.0208333333333333
49 0.0204081632653061
50 0.02
51 0.0196078431372549
52 0.0192307692307692
53 0.0188679245283019
54 0.037037037037037
55 0.0363636363636364
56 0.0357142857142857
57 0.0350877192982456
58 0.0344827586206897
59 0.0338983050847458
60 0.0333333333333333
61 0.0327868852459016
62 0.032258064516129
63 0.0317460317460317
64 0.03125
65 0.0307692307692308
66 0.0303030303030303
67 0.0298507462686567
68 0.0294117647058824
69 0.0289855072463768
70 0.0285714285714286
71 0.028169014084507
72 0.0277777777777778
73 0.0273972602739726
74 0.027027027027027
75 0.0266666666666667
76 0.0263157894736842
77 0.025974025974026
78 0.0256410256410256
79 0.0253164556962025
80 0.025
81 0.0246913580246914
82 0.024390243902439
83 0.0240963855421687
84 0.0238095238095238
85 0.0235294117647059
86 0.0232558139534884
87 0.0229885057471264
88 0.0227272727272727
89 0.0224719101123595
90 0.0222222222222222
91 0.021978021978022
92 0.0217391304347826
93 0.021505376344086
94 0.0212765957446809
95 0.0210526315789474
96 0.0208333333333333
97 0.0206185567010309
98 0.0204081632653061
99 0.0202020202020202
100 0.02
101 0.0198019801980198
102 0.0196078431372549
103 0.0194174757281553
104 0.0192307692307692
105 0.0285714285714286
106 0.0283018867924528
107 0.0280373831775701
108 0.0277777777777778
109 0.0275229357798165
110 0.0363636363636364
111 0.036036036036036
112 0.0357142857142857
113 0.0353982300884956
114 0.0350877192982456
115 0.0347826086956522
116 0.0344827586206897
117 0.0341880341880342
118 0.0338983050847458
119 0.0336134453781513
120 0.0333333333333333
121 0.0330578512396694
122 0.0327868852459016
123 0.032520325203252
124 0.032258064516129
125 0.032
126 0.0317460317460317
127 0.031496062992126
128 0.03125
129 0.0310077519379845
130 0.0307692307692308
131 0.0305343511450382
132 0.0303030303030303
133 0.0300751879699248
134 0.0298507462686567
135 0.0296296296296296
136 0.0294117647058824
137 0.0291970802919708
138 0.0289855072463768
139 0.0287769784172662
140 0.0285714285714286
141 0.0283687943262411
142 0.028169014084507
143 0.027972027972028
144 0.0277777777777778
145 0.0275862068965517
146 0.0273972602739726
147 0.0272108843537415
148 0.027027027027027
149 0.0268456375838926
150 0.0266666666666667
151 0.0264900662251656
152 0.0263157894736842
153 0.0261437908496732
154 0.025974025974026
155 0.0258064516129032
156 0.032051282051282
157 0.0318471337579618
158 0.0316455696202532
159 0.0314465408805031
160 0.03125
161 0.031055900621118
162 0.0308641975308642
163 0.0306748466257669
164 0.0304878048780488
165 0.0303030303030303
166 0.0301204819277108
167 0.029940119760479
168 0.0297619047619048
169 0.029585798816568
170 0.0294117647058824
171 0.0292397660818713
172 0.0290697674418605
173 0.0289017341040462
174 0.028735632183908
175 0.0285714285714286
176 0.0284090909090909
177 0.0282485875706215
178 0.0280898876404494
179 0.0279329608938547
180 0.0277777777777778
181 0.0276243093922652
182 0.0274725274725275
183 0.0273224043715847
184 0.0271739130434783
185 0.027027027027027
186 0.0268817204301075
187 0.0267379679144385
188 0.0265957446808511
189 0.0264550264550265
190 0.0263157894736842
191 0.0261780104712042
192 0.03125
193 0.0310880829015544
194 0.0309278350515464
195 0.0307692307692308
196 0.0306122448979592
197 0.0304568527918782
198 0.0303030303030303
199 0.0351758793969849
200 0.035
201 0.0348258706467662
202 0.0346534653465347
203 0.0344827586206897
204 0.0343137254901961
205 0.0341463414634146
206 0.0339805825242718
207 0.0386473429951691
208 0.0384615384615385
209 0.0382775119617225
210 0.0380952380952381
211 0.037914691943128
212 0.0377358490566038
213 0.0375586854460094
214 0.0373831775700935
215 0.0372093023255814
216 0.037037037037037
217 0.0368663594470046
218 0.036697247706422
219 0.0365296803652968
220 0.0363636363636364
221 0.0361990950226244
222 0.036036036036036
223 0.0358744394618834
224 0.0357142857142857
225 0.0355555555555556
226 0.0353982300884956
227 0.0352422907488987
228 0.0350877192982456
229 0.0393013100436681
230 0.0391304347826087
231 0.038961038961039
232 0.0387931034482759
233 0.0386266094420601
234 0.0384615384615385
235 0.0382978723404255
236 0.038135593220339
237 0.0379746835443038
238 0.0378151260504202
239 0.0376569037656904
240 0.0375
241 0.037344398340249
242 0.0371900826446281
243 0.037037037037037
244 0.0368852459016393
245 0.036734693877551
246 0.0365853658536585
247 0.0364372469635627
248 0.0362903225806452
249 0.036144578313253
250 0.04
251 0.0398406374501992
252 0.0396825396825397
253 0.0395256916996047
254 0.0393700787401575
255 0.0392156862745098
256 0.0390625
257 0.0389105058365759
258 0.0387596899224806
259 0.0386100386100386
260 0.0384615384615385
261 0.0383141762452107
262 0.0381679389312977
263 0.0380228136882129
264 0.0378787878787879
265 0.0377358490566038
266 0.037593984962406
267 0.0374531835205993
268 0.0373134328358209
269 0.0371747211895911
270 0.037037037037037
271 0.03690036900369
272 0.0367647058823529
273 0.0366300366300366
274 0.0364963503649635
275 0.0363636363636364
276 0.036231884057971
277 0.036101083032491
278 0.0359712230215827
279 0.03584229390681
280 0.0357142857142857
281 0.0355871886120996
282 0.0354609929078014
283 0.0353356890459364
284 0.0352112676056338
285 0.0350877192982456
286 0.034965034965035
287 0.0348432055749129
288 0.0347222222222222
289 0.0346020761245675
290 0.0344827586206897
291 0.0343642611683849
292 0.0342465753424658
293 0.0341296928327645
294 0.0340136054421769
295 0.0338983050847458
296 0.0337837837837838
297 0.0336700336700337
298 0.0335570469798658
299 0.0334448160535117
300 0.0333333333333333
301 0.0332225913621262
302 0.033112582781457
303 0.033003300330033
304 0.0361842105263158
305 0.0360655737704918
306 0.0359477124183007
307 0.0358306188925081
308 0.0357142857142857
309 0.0355987055016181
310 0.0354838709677419
311 0.0353697749196141
312 0.0352564102564103
313 0.0351437699680511
314 0.035031847133758
315 0.0349206349206349
316 0.0348101265822785
317 0.0347003154574132
318 0.0345911949685535
319 0.0344827586206897
320 0.034375
321 0.0342679127725857
322 0.0341614906832298
323 0.0340557275541796
324 0.0339506172839506
325 0.0338461538461538
326 0.0337423312883436
327 0.0336391437308868
328 0.0335365853658537
329 0.033434650455927
330 0.0333333333333333
331 0.0332326283987915
332 0.0331325301204819
333 0.033033033033033
334 0.0329341317365269
335 0.0328358208955224
336 0.0327380952380952
337 0.0326409495548961
338 0.0325443786982249
339 0.0324483775811209
340 0.0323529411764706
341 0.032258064516129
342 0.0321637426900585
343 0.032069970845481
344 0.0319767441860465
345 0.0318840579710145
346 0.0317919075144509
347 0.031700288184438
348 0.0316091954022989
349 0.0315186246418338
350 0.0314285714285714
351 0.0313390313390313
352 0.03125
353 0.0311614730878187
354 0.0310734463276836
355 0.0309859154929577
356 0.0308988764044944
357 0.030812324929972
358 0.0307262569832402
359 0.0306406685236769
360 0.0305555555555556
361 0.0304709141274238
362 0.0303867403314917
363 0.0303030303030303
364 0.0302197802197802
365 0.0328767123287671
366 0.0327868852459016
367 0.0326975476839237
368 0.0326086956521739
369 0.032520325203252
370 0.0324324324324324
371 0.032345013477089
372 0.032258064516129
373 0.032171581769437
374 0.0320855614973262
375 0.032
376 0.0319148936170213
377 0.0318302387267905
378 0.0343915343915344
379 0.0343007915567282
380 0.0342105263157895
381 0.0341207349081365
382 0.0340314136125654
383 0.0339425587467363
384 0.0338541666666667
385 0.0363636363636364
386 0.0362694300518135
387 0.0361757105943152
388 0.0360824742268041
389 0.038560411311054
390 0.0384615384615385
391 0.0383631713554987
392 0.038265306122449
393 0.0381679389312977
394 0.0380710659898477
395 0.0379746835443038
396 0.0378787878787879
397 0.0377833753148615
398 0.0376884422110553
399 0.037593984962406
400 0.0375
401 0.0374064837905237
402 0.0373134328358209
403 0.0372208436724566
404 0.0371287128712871
405 0.037037037037037
406 0.0369458128078818
407 0.0368550368550369
408 0.0367647058823529
409 0.0366748166259169
410 0.0365853658536585
411 0.0364963503649635
412 0.0364077669902913
413 0.036319612590799
414 0.036231884057971
415 0.036144578313253
416 0.0360576923076923
417 0.0359712230215827
418 0.0358851674641148
419 0.035799522673031
420 0.0357142857142857
421 0.0356294536817102
422 0.0355450236966825
423 0.0354609929078014
424 0.035377358490566
425 0.0352941176470588
426 0.0352112676056338
427 0.0351288056206089
428 0.0350467289719626
429 0.034965034965035
430 0.0348837209302326
431 0.0348027842227378
432 0.037037037037037
433 0.0369515011547344
434 0.0368663594470046
435 0.0367816091954023
436 0.036697247706422
437 0.0366132723112128
438 0.0365296803652968
439 0.0364464692482916
440 0.0363636363636364
441 0.036281179138322
442 0.0361990950226244
443 0.036117381489842
444 0.036036036036036
445 0.0359550561797753
446 0.0358744394618834
447 0.0357941834451902
448 0.0357142857142857
449 0.0356347438752784
450 0.0355555555555556
451 0.0354767184035477
452 0.0353982300884956
453 0.0353200883002208
454 0.0352422907488987
455 0.0351648351648352
456 0.0350877192982456
457 0.0350109409190372
458 0.0349344978165939
459 0.0348583877995643
460 0.0347826086956522
461 0.03470715835141
462 0.0346320346320346
463 0.0345572354211663
464 0.0344827586206897
465 0.0344086021505376
466 0.0343347639484979
467 0.0342612419700214
468 0.0341880341880342
469 0.0341151385927505
470 0.0340425531914894
471 0.0339702760084926
472 0.0360169491525424
473 0.0380549682875264
474 0.0379746835443038
475 0.0378947368421053
476 0.0378151260504202
477 0.0377358490566038
478 0.0376569037656904
479 0.0375782881002088
480 0.0375
481 0.0374220374220374
482 0.037344398340249
483 0.0372670807453416
484 0.0371900826446281
485 0.0371134020618557
486 0.037037037037037
487 0.0369609856262834
488 0.0368852459016393
489 0.0368098159509202
490 0.036734693877551
491 0.0366598778004073
492 0.0365853658536585
493 0.0365111561866126
494 0.0364372469635627
495 0.0363636363636364
496 0.0362903225806452
497 0.0362173038229376
498 0.036144578313253
499 0.0360721442885772
500 0.036
501 0.0359281437125748
502 0.0358565737051793
503 0.0357852882703777
504 0.0357142857142857
505 0.0356435643564356
506 0.0355731225296443
507 0.0355029585798817
508 0.0354330708661417
509 0.0353634577603143
510 0.0352941176470588
511 0.0352250489236791
512 0.03515625
513 0.0350877192982456
514 0.0350194552529183
515 0.0349514563106796
516 0.0348837209302326
517 0.034816247582205
518 0.0366795366795367
519 0.0366088631984586
520 0.0365384615384615
521 0.0383877159309021
522 0.0402298850574713
523 0.0401529636711281
524 0.0400763358778626
525 0.04
526 0.0399239543726236
527 0.0398481973434535
528 0.0397727272727273
529 0.0396975425330813
530 0.039622641509434
531 0.0395480225988701
532 0.0394736842105263
533 0.0393996247654784
534 0.0393258426966292
535 0.0392523364485981
536 0.0391791044776119
537 0.0391061452513966
538 0.0390334572490706
539 0.038961038961039
540 0.0388888888888889
541 0.0388170055452865
542 0.0387453874538745
543 0.0386740331491713
544 0.0386029411764706
545 0.0385321100917431
546 0.0384615384615385
547 0.0383912248628885
548 0.0383211678832117
549 0.0382513661202186
550 0.0381818181818182
551 0.0381125226860254
552 0.0380434782608696
553 0.0379746835443038
554 0.0379061371841155
555 0.0378378378378378
556 0.0377697841726619
557 0.0377019748653501
558 0.0376344086021505
559 0.037567084078712
560 0.0375
561 0.0374331550802139
562 0.0373665480427046
563 0.0373001776198934
564 0.0372340425531915
565 0.0371681415929204
566 0.0371024734982332
567 0.037037037037037
568 0.0369718309859155
569 0.0369068541300527
570 0.0368421052631579
571 0.0367775831873905
572 0.0367132867132867
573 0.0366492146596859
574 0.0365853658536585
575 0.0365217391304348
576 0.0364583333333333
577 0.0363951473136915
578 0.0363321799307958
579 0.0362694300518135
580 0.0362068965517241
581 0.036144578313253
582 0.0360824742268041
583 0.0360205831903945
584 0.0376712328767123
585 0.0376068376068376
586 0.037542662116041
587 0.0374787052810903
588 0.0374149659863946
589 0.0373514431239389
590 0.0372881355932203
591 0.0372250423011844
592 0.0371621621621622
593 0.0387858347386172
594 0.0387205387205387
595 0.038655462184874
596 0.0385906040268456
597 0.0385259631490787
598 0.0384615384615385
599 0.0383973288814691
600 0.0383333333333333
601 0.0382695507487521
602 0.0382059800664452
603 0.0381426202321725
604 0.0380794701986755
605 0.0380165289256198
606 0.037953795379538
607 0.0378912685337726
608 0.0378289473684211
609 0.0377668308702791
610 0.0377049180327869
611 0.0376432078559738
612 0.0375816993464052
613 0.0375203915171289
614 0.0374592833876222
615 0.0373983739837398
616 0.0373376623376623
617 0.0372771474878444
618 0.0372168284789644
619 0.037156704361874
620 0.0370967741935484
621 0.037037037037037
622 0.0369774919614148
623 0.0369181380417336
624 0.0368589743589744
625 0.0368
626 0.036741214057508
627 0.0366826156299841
628 0.036624203821656
629 0.0365659777424483
630 0.0380952380952381
631 0.0380348652931854
632 0.0379746835443038
633 0.037914691943128
634 0.0378548895899054
635 0.0377952755905512
636 0.0377358490566038
637 0.0376766091051805
638 0.0376175548589342
639 0.0375586854460094
640 0.0375
641 0.0374414976599064
642 0.0373831775700935
643 0.0373250388802488
644 0.0372670807453416
645 0.0372093023255814
646 0.0371517027863777
647 0.0370942812982998
648 0.037037037037037
649 0.036979969183359
650 0.0369230769230769
651 0.0368663594470046
652 0.0368098159509202
653 0.0367534456355283
654 0.036697247706422
655 0.0366412213740458
656 0.0365853658536585
657 0.0365296803652968
658 0.0364741641337386
659 0.0364188163884674
660 0.0363636363636364
661 0.0363086232980333
662 0.0362537764350453
663 0.0361990950226244
664 0.036144578313253
665 0.0360902255639098
666 0.036036036036036
667 0.0359820089955022
668 0.0359281437125748
669 0.0358744394618834
670 0.0358208955223881
671 0.0357675111773472
672 0.0357142857142857
673 0.0356612184249629
674 0.0356083086053412
675 0.0355555555555556
676 0.0355029585798817
677 0.0354505169867061
678 0.0353982300884956
679 0.0353460972017673
680 0.0352941176470588
681 0.0352422907488987
682 0.0351906158357771
683 0.0351390922401171
684 0.0350877192982456
685 0.035036496350365
686 0.0349854227405248
687 0.0349344978165939
688 0.0348837209302326
689 0.034833091436865
690 0.0347826086956522
691 0.0347322720694645
692 0.0346820809248555
693 0.0346320346320346
694 0.0345821325648415
695 0.0345323741007194
696 0.0344827586206897
697 0.0344332855093257
698 0.0343839541547278
699 0.0343347639484979
700 0.0342857142857143
701 0.0342368045649073
702 0.0341880341880342
703 0.0341394025604552
704 0.0340909090909091
705 0.0340425531914894
706 0.0339943342776204
707 0.0339462517680339
708 0.0338983050847458
709 0.0338504936530324
710 0.0338028169014084
711 0.0337552742616034
712 0.0337078651685393
713 0.0336605890603086
714 0.0336134453781513
715 0.0335664335664336
716 0.0335195530726257
717 0.0334728033472803
718 0.0334261838440111
719 0.0333796940194715
720 0.0333333333333333
721 0.0332871012482663
722 0.0332409972299169
723 0.033195020746888
724 0.0331491712707182
725 0.0331034482758621
726 0.0330578512396694
727 0.0330123796423659
728 0.032967032967033
729 0.0329218106995885
730 0.0328767123287671
731 0.0328317373461012
732 0.0327868852459016
733 0.0327421555252387
734 0.0326975476839237
735 0.0326530612244898
736 0.0326086956521739
737 0.0325644504748982
738 0.032520325203252
739 0.0324763193504736
740 0.0324324324324324
741 0.0323886639676113
742 0.032345013477089
743 0.0323014804845222
744 0.032258064516129
745 0.0322147651006711
746 0.032171581769437
747 0.0321285140562249
748 0.0320855614973262
749 0.0320427236315087
750 0.032
751 0.0319573901464714
752 0.0319148936170213
753 0.0318725099601594
754 0.0318302387267905
755 0.0317880794701987
756 0.0317460317460317
757 0.0317040951122853
758 0.0316622691292876
759 0.0316205533596838
760 0.0315789473684211
761 0.0328515111695138
762 0.0328083989501312
763 0.0327653997378768
764 0.0327225130890052
765 0.0326797385620915
766 0.0326370757180157
767 0.0325945241199478
768 0.0325520833333333
769 0.0325097529258778
770 0.0324675324675325
771 0.0324254215304799
772 0.0323834196891192
773 0.0323415265200517
774 0.0322997416020672
775 0.032258064516129
776 0.0322164948453608
777 0.0321750321750322
778 0.032133676092545
779 0.0320924261874198
780 0.032051282051282
781 0.0320102432778489
782 0.0319693094629156
783 0.0319284802043423
784 0.0318877551020408
785 0.0318471337579618
786 0.0318066157760814
787 0.0317662007623888
788 0.0317258883248731
789 0.0316856780735108
790 0.0316455696202532
791 0.0316055625790139
792 0.0315656565656566
793 0.0315258511979823
794 0.0314861460957179
795 0.0314465408805031
796 0.0314070351758794
797 0.0326223337515684
798 0.0338345864661654
799 0.0337922403003755
800 0.03375
801 0.0337078651685393
802 0.0336658354114713
803 0.0336239103362391
804 0.0335820895522388
805 0.0335403726708075
806 0.0334987593052109
807 0.033457249070632
808 0.0334158415841584
809 0.0333745364647713
810 0.0333333333333333
811 0.0332922318125771
812 0.0332512315270936
813 0.033210332103321
814 0.0331695331695332
815 0.0331288343558282
816 0.0330882352941176
817 0.0330477356181151
818 0.0330073349633252
819 0.032967032967033
820 0.0329268292682927
821 0.0328867235079172
822 0.0328467153284672
823 0.0328068043742406
824 0.0327669902912621
825 0.0327272727272727
826 0.0326876513317191
827 0.0326481257557437
828 0.0326086956521739
829 0.0325693606755127
830 0.0325301204819277
831 0.0324909747292419
832 0.0324519230769231
833 0.0324129651860744
834 0.0323741007194245
835 0.0323353293413174
836 0.0322966507177034
837 0.032258064516129
838 0.0322195704057279
839 0.032181168057211
840 0.0321428571428571
841 0.0321046373365042
842 0.0320665083135392
843 0.0320284697508897
844 0.0319905213270142
845 0.0319526627218935
846 0.0319148936170213
847 0.0318772136953955
848 0.0318396226415094
849 0.0318021201413428
850 0.0317647058823529
851 0.0317273795534665
852 0.0316901408450704
853 0.0316529894490035
854 0.031615925058548
855 0.0315789473684211
856 0.0315420560747664
857 0.0315052508751459
858 0.0314685314685315
859 0.0314318975552969
860 0.0325581395348837
861 0.032520325203252
862 0.0324825986078886
863 0.0324449594438007
864 0.0324074074074074
865 0.0323699421965318
866 0.0323325635103926
867 0.0322952710495963
868 0.032258064516129
869 0.0322209436133487
870 0.032183908045977
871 0.0321469575200918
872 0.0321100917431193
873 0.0320733104238259
874 0.0331807780320366
875 0.0331428571428571
876 0.0331050228310502
877 0.0330672748004561
878 0.0330296127562642
879 0.0329920364050057
880 0.0329545454545455
881 0.0329171396140749
882 0.0328798185941043
883 0.0328425821064553
884 0.0328054298642534
885 0.0327683615819209
886 0.0327313769751693
887 0.0326944757609921
888 0.0326576576576577
889 0.0326209223847019
890 0.0325842696629213
891 0.0325476992143659
892 0.0325112107623318
893 0.032474804031355
894 0.0324384787472036
895 0.0324022346368715
896 0.0323660714285714
897 0.032329988851728
898 0.032293986636971
899 0.032258064516129
900 0.0322222222222222
901 0.0321864594894562
902 0.0321507760532151
903 0.0321151716500554
904 0.0320796460176991
905 0.0320441988950276
906 0.0320088300220751
907 0.0319735391400221
908 0.0319383259911894
909 0.0319031903190319
910 0.0318681318681319
911 0.0318331503841932
912 0.0317982456140351
913 0.031763417305586
914 0.0317286652078775
915 0.0316939890710383
916 0.0316593886462882
917 0.0316248636859324
918 0.0315904139433551
919 0.0315560391730141
920 0.0315217391304348
921 0.0314875135722041
922 0.0314533622559653
923 0.0314192849404117
924 0.0313852813852814
925 0.0313513513513514
926 0.031317494600432
927 0.0312837108953614
928 0.03125
929 0.031216361679225
930 0.0311827956989247
931 0.0311493018259936
932 0.0311158798283262
933 0.0310825294748124
934 0.0310492505353319
935 0.0310160427807487
936 0.030982905982906
937 0.0309498399146211
938 0.0309168443496802
939 0.0308839190628328
940 0.0308510638297872
941 0.0308182784272051
942 0.0307855626326964
943 0.0307529162248144
944 0.0307203389830508
945 0.0306878306878307
946 0.0306553911205074
947 0.030623020063358
948 0.0305907172995781
949 0.0305584826132771
950 0.0305263157894737
951 0.0304942166140904
952 0.0304621848739496
953 0.0304302203567681
954 0.030398322851153
955 0.0303664921465969
956 0.0303347280334728
957 0.0303030303030303
958 0.0302713987473904
959 0.0302398331595412
960 0.0302083333333333
961 0.0301768990634755
962 0.0301455301455301
963 0.0301142263759086
964 0.0300829875518672
965 0.0300518134715026
966 0.0300207039337474
967 0.0299896587383661
968 0.0299586776859504
969 0.0299277605779154
970 0.0298969072164948
971 0.0298661174047374
972 0.0298353909465021
973 0.0298047276464543
974 0.0297741273100616
975 0.0297435897435897
976 0.0297131147540984
977 0.0296827021494371
978 0.0296523517382413
979 0.0296220633299285
980 0.0295918367346939
981 0.0295616717635066
982 0.0295315682281059
983 0.0295015259409969
984 0.0294715447154472
985 0.0294416243654822
986 0.0294117647058824
987 0.0293819655521783
988 0.0293522267206478
989 0.0293225480283114
990 0.0292929292929293
991 0.029263370332997
992 0.0292338709677419
993 0.0292044310171198
994 0.0291750503018109
995 0.0291457286432161
996 0.0291164658634538
997 0.0290872617853561
998 0.0290581162324649
999 0.029029029029029
1000 0.029
};
\addplot [semithick, color6, forget plot]
table {%
1 0
2 0
3 0
4 0
5 0
6 0
7 0
8 0
9 0
10 0
11 0
12 0
13 0
14 0
15 0
16 0
17 0
18 0
19 0
20 0
21 0
22 0
23 0
24 0
25 0
26 0
27 0
28 0
29 0
30 0
31 0
32 0
33 0
34 0
35 0
36 0
37 0
38 0
39 0
40 0
41 0
42 0
43 0
44 0
45 0
46 0
47 0
48 0
49 0.0204081632653061
50 0.02
51 0.0196078431372549
52 0.0192307692307692
53 0.0188679245283019
54 0.0185185185185185
55 0.0181818181818182
56 0.0178571428571429
57 0.0175438596491228
58 0.0172413793103448
59 0.0169491525423729
60 0.0166666666666667
61 0.0163934426229508
62 0.0161290322580645
63 0.0158730158730159
64 0.015625
65 0.0307692307692308
66 0.0303030303030303
67 0.0298507462686567
68 0.0294117647058824
69 0.0289855072463768
70 0.0285714285714286
71 0.028169014084507
72 0.0277777777777778
73 0.0273972602739726
74 0.027027027027027
75 0.0266666666666667
76 0.0263157894736842
77 0.025974025974026
78 0.0256410256410256
79 0.0253164556962025
80 0.025
81 0.0246913580246914
82 0.024390243902439
83 0.0240963855421687
84 0.0238095238095238
85 0.0235294117647059
86 0.0232558139534884
87 0.0229885057471264
88 0.0227272727272727
89 0.0224719101123595
90 0.0222222222222222
91 0.021978021978022
92 0.0217391304347826
93 0.021505376344086
94 0.0212765957446809
95 0.0210526315789474
96 0.0208333333333333
97 0.0206185567010309
98 0.0204081632653061
99 0.0202020202020202
100 0.02
101 0.0198019801980198
102 0.0196078431372549
103 0.0194174757281553
104 0.0192307692307692
105 0.019047619047619
106 0.0188679245283019
107 0.0186915887850467
108 0.0185185185185185
109 0.018348623853211
110 0.0181818181818182
111 0.018018018018018
112 0.0178571428571429
113 0.0176991150442478
114 0.0175438596491228
115 0.0173913043478261
116 0.0172413793103448
117 0.0170940170940171
118 0.0169491525423729
119 0.0168067226890756
120 0.0166666666666667
121 0.0165289256198347
122 0.0163934426229508
123 0.016260162601626
124 0.0161290322580645
125 0.016
126 0.0158730158730159
127 0.015748031496063
128 0.015625
129 0.0155038759689922
130 0.0153846153846154
131 0.0152671755725191
132 0.0151515151515152
133 0.0150375939849624
134 0.0149253731343284
135 0.0148148148148148
136 0.0147058823529412
137 0.0218978102189781
138 0.0217391304347826
139 0.0215827338129496
140 0.0214285714285714
141 0.0212765957446809
142 0.0211267605633803
143 0.020979020979021
144 0.0208333333333333
145 0.0206896551724138
146 0.0205479452054795
147 0.0204081632653061
148 0.0202702702702703
149 0.0201342281879195
150 0.02
151 0.0198675496688742
152 0.0197368421052632
153 0.0196078431372549
154 0.0194805194805195
155 0.0193548387096774
156 0.0192307692307692
157 0.0191082802547771
158 0.0189873417721519
159 0.0188679245283019
160 0.01875
161 0.0186335403726708
162 0.0185185185185185
163 0.0184049079754601
164 0.0182926829268293
165 0.0181818181818182
166 0.0180722891566265
167 0.0179640718562874
168 0.0178571428571429
169 0.0177514792899408
170 0.0176470588235294
171 0.0175438596491228
172 0.0174418604651163
173 0.0173410404624277
174 0.0172413793103448
175 0.0171428571428571
176 0.0170454545454545
177 0.0225988700564972
178 0.0224719101123595
179 0.0223463687150838
180 0.0222222222222222
181 0.0220994475138122
182 0.021978021978022
183 0.0218579234972678
184 0.0217391304347826
185 0.0216216216216216
186 0.0268817204301075
187 0.0267379679144385
188 0.0319148936170213
189 0.0317460317460317
190 0.0315789473684211
191 0.031413612565445
192 0.03125
193 0.0310880829015544
194 0.0309278350515464
195 0.0307692307692308
196 0.0306122448979592
197 0.0304568527918782
198 0.0303030303030303
199 0.0301507537688442
200 0.03
201 0.0298507462686567
202 0.0297029702970297
203 0.0295566502463054
204 0.0294117647058824
205 0.0292682926829268
206 0.029126213592233
207 0.0289855072463768
208 0.0288461538461538
209 0.0287081339712919
210 0.0333333333333333
211 0.033175355450237
212 0.0330188679245283
213 0.0328638497652582
214 0.0327102803738318
215 0.0325581395348837
216 0.0324074074074074
217 0.032258064516129
218 0.0321100917431193
219 0.0319634703196347
220 0.0318181818181818
221 0.0316742081447964
222 0.0315315315315315
223 0.031390134529148
224 0.03125
225 0.0311111111111111
226 0.0309734513274336
227 0.0308370044052863
228 0.0307017543859649
229 0.0305676855895196
230 0.0304347826086957
231 0.0303030303030303
232 0.0301724137931034
233 0.0300429184549356
234 0.0299145299145299
235 0.0297872340425532
236 0.0296610169491525
237 0.029535864978903
238 0.0294117647058824
239 0.0292887029288703
240 0.0291666666666667
241 0.029045643153527
242 0.0289256198347107
243 0.0288065843621399
244 0.0286885245901639
245 0.0285714285714286
246 0.0284552845528455
247 0.0283400809716599
248 0.0282258064516129
249 0.0281124497991968
250 0.028
251 0.0278884462151394
252 0.0277777777777778
253 0.0276679841897233
254 0.0275590551181102
255 0.0274509803921569
256 0.02734375
257 0.0272373540856031
258 0.0310077519379845
259 0.0308880308880309
260 0.0307692307692308
261 0.0306513409961686
262 0.0305343511450382
263 0.0304182509505703
264 0.0303030303030303
265 0.030188679245283
266 0.0300751879699248
267 0.0299625468164794
268 0.0298507462686567
269 0.0297397769516729
270 0.0296296296296296
271 0.029520295202952
272 0.0294117647058824
273 0.0293040293040293
274 0.0291970802919708
275 0.0290909090909091
276 0.0289855072463768
277 0.0288808664259928
278 0.0287769784172662
279 0.028673835125448
280 0.0285714285714286
281 0.0284697508896797
282 0.0283687943262411
283 0.0282685512367491
284 0.028169014084507
285 0.0280701754385965
286 0.027972027972028
287 0.0278745644599303
288 0.0277777777777778
289 0.027681660899654
290 0.0275862068965517
291 0.0274914089347079
292 0.0273972602739726
293 0.0273037542662116
294 0.0272108843537415
295 0.0271186440677966
296 0.027027027027027
297 0.0269360269360269
298 0.0268456375838926
299 0.0267558528428094
300 0.0266666666666667
301 0.026578073089701
302 0.0264900662251656
303 0.0264026402640264
304 0.0263157894736842
305 0.0262295081967213
306 0.0261437908496732
307 0.0260586319218241
308 0.025974025974026
309 0.0258899676375405
310 0.0258064516129032
311 0.0257234726688103
312 0.0256410256410256
313 0.0255591054313099
314 0.0254777070063694
315 0.0253968253968254
316 0.0284810126582278
317 0.028391167192429
318 0.0283018867924528
319 0.0282131661442006
320 0.028125
321 0.0280373831775701
322 0.0279503105590062
323 0.0278637770897833
324 0.0277777777777778
325 0.0276923076923077
326 0.0276073619631902
327 0.0275229357798165
328 0.0274390243902439
329 0.027355623100304
330 0.0272727272727273
331 0.027190332326284
332 0.0271084337349398
333 0.027027027027027
334 0.0269461077844311
335 0.026865671641791
336 0.0267857142857143
337 0.0267062314540059
338 0.0266272189349112
339 0.0265486725663717
340 0.0264705882352941
341 0.0263929618768328
342 0.0263157894736842
343 0.0262390670553936
344 0.0261627906976744
345 0.0260869565217391
346 0.0260115606936416
347 0.0259365994236311
348 0.0258620689655172
349 0.0257879656160458
350 0.0257142857142857
351 0.0256410256410256
352 0.0255681818181818
353 0.0254957507082153
354 0.0254237288135593
355 0.0253521126760563
356 0.0252808988764045
357 0.0252100840336134
358 0.0251396648044693
359 0.0250696378830084
360 0.025
361 0.0249307479224377
362 0.0248618784530387
363 0.0247933884297521
364 0.0247252747252747
365 0.0246575342465753
366 0.0245901639344262
367 0.0245231607629428
368 0.0244565217391304
369 0.02710027100271
370 0.027027027027027
371 0.0269541778975741
372 0.0268817204301075
373 0.0268096514745308
374 0.0267379679144385
375 0.0266666666666667
376 0.0265957446808511
377 0.026525198938992
378 0.0264550264550265
379 0.0263852242744063
380 0.0263157894736842
381 0.026246719160105
382 0.0261780104712042
383 0.0261096605744125
384 0.0260416666666667
385 0.025974025974026
386 0.0259067357512953
387 0.0258397932816537
388 0.0257731958762887
389 0.025706940874036
390 0.0256410256410256
391 0.0255754475703325
392 0.0255102040816327
393 0.0254452926208651
394 0.0253807106598985
395 0.0253164556962025
396 0.0252525252525253
397 0.0251889168765743
398 0.0251256281407035
399 0.025062656641604
400 0.025
401 0.027431421446384
402 0.027363184079602
403 0.0272952853598015
404 0.0272277227722772
405 0.0271604938271605
406 0.0270935960591133
407 0.027027027027027
408 0.0269607843137255
409 0.0268948655256724
410 0.0268292682926829
411 0.0267639902676399
412 0.0266990291262136
413 0.026634382566586
414 0.0265700483091787
415 0.0265060240963855
416 0.0264423076923077
417 0.026378896882494
418 0.0263157894736842
419 0.0262529832935561
420 0.0261904761904762
421 0.0261282660332542
422 0.0260663507109005
423 0.0260047281323877
424 0.0259433962264151
425 0.0258823529411765
426 0.0258215962441315
427 0.0257611241217799
428 0.0257009345794393
429 0.0256410256410256
430 0.0255813953488372
431 0.0255220417633411
432 0.025462962962963
433 0.0254041570438799
434 0.0253456221198157
435 0.0252873563218391
436 0.0252293577981651
437 0.0251716247139588
438 0.0251141552511416
439 0.0250569476082005
440 0.025
441 0.0249433106575964
442 0.0248868778280543
443 0.0248306997742664
444 0.0247747747747748
445 0.0247191011235955
446 0.0246636771300448
447 0.0246085011185682
448 0.0245535714285714
449 0.0244988864142539
450 0.0244444444444444
451 0.024390243902439
452 0.0243362831858407
453 0.0242825607064018
454 0.0242290748898678
455 0.0241758241758242
456 0.0241228070175439
457 0.0262582056892779
458 0.0262008733624454
459 0.0261437908496732
460 0.0260869565217391
461 0.0260303687635575
462 0.025974025974026
463 0.0259179265658747
464 0.0258620689655172
465 0.0258064516129032
466 0.0257510729613734
467 0.0256959314775161
468 0.0256410256410256
469 0.0255863539445629
470 0.025531914893617
471 0.0276008492569002
472 0.0275423728813559
473 0.0274841437632135
474 0.0274261603375527
475 0.0273684210526316
476 0.0273109243697479
477 0.0272536687631027
478 0.0271966527196653
479 0.0271398747390397
480 0.0270833333333333
481 0.027027027027027
482 0.0269709543568465
483 0.0269151138716356
484 0.0268595041322314
485 0.0268041237113402
486 0.0267489711934156
487 0.026694045174538
488 0.0266393442622951
489 0.0265848670756646
490 0.026530612244898
491 0.0264765784114053
492 0.0264227642276423
493 0.026369168356998
494 0.0263157894736842
495 0.0262626262626263
496 0.0262096774193548
497 0.0261569416498994
498 0.0281124497991968
499 0.0280561122244489
500 0.028
501 0.0279441117764471
502 0.0278884462151394
503 0.0278330019880716
504 0.0277777777777778
505 0.0277227722772277
506 0.0276679841897233
507 0.0276134122287968
508 0.0275590551181102
509 0.0275049115913556
510 0.0274509803921569
511 0.0273972602739726
512 0.02734375
513 0.0272904483430799
514 0.0272373540856031
515 0.0271844660194175
516 0.0271317829457364
517 0.0290135396518375
518 0.028957528957529
519 0.0289017341040462
520 0.0288461538461538
521 0.0287907869481766
522 0.028735632183908
523 0.0286806883365201
524 0.0286259541984733
525 0.0285714285714286
526 0.0285171102661597
527 0.0303605313092979
528 0.0303030303030303
529 0.0302457466918715
530 0.030188679245283
531 0.0301318267419962
532 0.0300751879699248
533 0.0300187617260788
534 0.0299625468164794
535 0.0299065420560748
536 0.0298507462686567
537 0.0297951582867784
538 0.0297397769516729
539 0.0296846011131725
540 0.0296296296296296
541 0.0295748613678373
542 0.029520295202952
543 0.0294659300184162
544 0.0294117647058824
545 0.0293577981651376
546 0.0293040293040293
547 0.0292504570383912
548 0.0291970802919708
549 0.029143897996357
550 0.0290909090909091
551 0.029038112522686
552 0.0289855072463768
553 0.0289330922242315
554 0.0288808664259928
555 0.0288288288288288
556 0.0287769784172662
557 0.0287253141831239
558 0.028673835125448
559 0.0286225402504472
560 0.0285714285714286
561 0.0285204991087344
562 0.0284697508896797
563 0.0284191829484902
564 0.0283687943262411
565 0.0283185840707965
566 0.0300353356890459
567 0.0299823633156966
568 0.0299295774647887
569 0.0298769771528998
570 0.0298245614035088
571 0.0297723292469352
572 0.0297202797202797
573 0.0296684118673647
574 0.029616724738676
575 0.0295652173913043
576 0.0295138888888889
577 0.0294627383015598
578 0.0294117647058824
579 0.0293609671848014
580 0.0293103448275862
581 0.0292598967297762
582 0.0292096219931271
583 0.0291595197255575
584 0.0291095890410959
585 0.0290598290598291
586 0.0290102389078498
587 0.0289608177172061
588 0.0289115646258503
589 0.0288624787775891
590 0.0288135593220339
591 0.0287648054145516
592 0.0287162162162162
593 0.0286677908937605
594 0.0286195286195286
595 0.0285714285714286
596 0.0302013422818792
597 0.0301507537688442
598 0.0301003344481605
599 0.0300500834724541
600 0.03
601 0.0299500831946755
602 0.0299003322259136
603 0.0298507462686567
604 0.0298013245033113
605 0.0297520661157025
606 0.0297029702970297
607 0.0296540362438221
608 0.0296052631578947
609 0.0295566502463054
610 0.0295081967213115
611 0.0294599018003273
612 0.0294117647058824
613 0.0293637846655791
614 0.0293159609120521
615 0.0292682926829268
616 0.0292207792207792
617 0.0291734197730956
618 0.029126213592233
619 0.0290791599353796
620 0.0290322580645161
621 0.0289855072463768
622 0.0289389067524116
623 0.028892455858748
624 0.0288461538461538
625 0.0288
626 0.0287539936102236
627 0.0287081339712919
628 0.0286624203821656
629 0.0286168521462639
630 0.0285714285714286
631 0.0285261489698891
632 0.0284810126582278
633 0.028436018957346
634 0.028391167192429
635 0.0283464566929134
636 0.0283018867924528
637 0.0282574568288854
638 0.0282131661442006
639 0.028169014084507
640 0.028125
641 0.0280811232449298
642 0.0280373831775701
643 0.0279937791601866
644 0.0279503105590062
645 0.027906976744186
646 0.0278637770897833
647 0.0278207109737249
648 0.0277777777777778
649 0.0277349768875193
650 0.0276923076923077
651 0.0276497695852535
652 0.0276073619631902
653 0.0275650842266462
654 0.0275229357798165
655 0.0274809160305344
656 0.0289634146341463
657 0.0289193302891933
658 0.0288753799392097
659 0.0288315629742033
660 0.0287878787878788
661 0.0287443267776097
662 0.0302114803625378
663 0.0301659125188537
664 0.0301204819277108
665 0.0300751879699248
666 0.03003003003003
667 0.0299850074962519
668 0.029940119760479
669 0.0298953662182362
670 0.0298507462686567
671 0.029806259314456
672 0.0297619047619048
673 0.0297176820208024
674 0.029673590504451
675 0.0296296296296296
676 0.029585798816568
677 0.0295420974889217
678 0.0294985250737463
679 0.0294550810014728
680 0.0294117647058824
681 0.0293685756240822
682 0.0293255131964809
683 0.0292825768667643
684 0.0292397660818713
685 0.0291970802919708
686 0.0291545189504373
687 0.0291120815138282
688 0.0290697674418605
689 0.0290275761973875
690 0.0289855072463768
691 0.0289435600578871
692 0.0289017341040462
693 0.0288600288600289
694 0.0288184438040346
695 0.0287769784172662
696 0.028735632183908
697 0.0286944045911047
698 0.0286532951289398
699 0.0286123032904149
700 0.0285714285714286
701 0.0285306704707561
702 0.0284900284900285
703 0.0284495021337127
704 0.0284090909090909
705 0.0283687943262411
706 0.028328611898017
707 0.0282885431400283
708 0.0282485875706215
709 0.0282087447108604
710 0.028169014084507
711 0.0281293952180028
712 0.0280898876404494
713 0.0280504908835905
714 0.0280112044817927
715 0.027972027972028
716 0.0279329608938547
717 0.0278940027894003
718 0.0278551532033426
719 0.0278164116828929
720 0.0277777777777778
721 0.0277392510402219
722 0.0277008310249307
723 0.0276625172890733
724 0.0276243093922652
725 0.0275862068965517
726 0.0275482093663912
727 0.0275103163686382
728 0.0274725274725275
729 0.0274348422496571
730 0.0273972602739726
731 0.027359781121751
732 0.0273224043715847
733 0.0272851296043656
734 0.0272479564032698
735 0.0272108843537415
736 0.0271739130434783
737 0.0271370420624152
738 0.02710027100271
739 0.027063599458728
740 0.027027027027027
741 0.0269905533063428
742 0.0283018867924528
743 0.0282637954239569
744 0.0282258064516129
745 0.0281879194630872
746 0.0281501340482574
747 0.0281124497991968
748 0.0280748663101604
749 0.0280373831775701
750 0.028
751 0.0279627163781625
752 0.0279255319148936
753 0.0278884462151394
754 0.0278514588859416
755 0.0278145695364238
756 0.0277777777777778
757 0.0277410832232497
758 0.0277044854881266
759 0.0276679841897233
760 0.0276315789473684
761 0.0275952693823916
762 0.0275590551181102
763 0.0275229357798165
764 0.0287958115183246
765 0.0287581699346405
766 0.0287206266318538
767 0.0286831812255541
768 0.0286458333333333
769 0.0286085825747724
770 0.0285714285714286
771 0.0285343709468223
772 0.0284974093264249
773 0.0284605433376455
774 0.0284237726098191
775 0.0283870967741936
776 0.0283505154639175
777 0.0283140283140283
778 0.0282776349614396
779 0.0282413350449294
780 0.0282051282051282
781 0.028169014084507
782 0.0281329923273657
783 0.0280970625798212
784 0.0280612244897959
785 0.0280254777070064
786 0.0279898218829517
787 0.0279542566709022
788 0.0279187817258883
789 0.0278833967046895
790 0.0278481012658228
791 0.0278128950695322
792 0.0277777777777778
793 0.0277427490542245
794 0.0277078085642317
795 0.0276729559748428
796 0.0276381909547739
797 0.027603513174404
798 0.0275689223057644
799 0.0275344180225282
800 0.0275
801 0.0274656679151061
802 0.027431421446384
803 0.0273972602739726
804 0.027363184079602
805 0.0273291925465838
806 0.0272952853598015
807 0.0272614622057001
808 0.0272277227722772
809 0.0271940667490729
810 0.0271604938271605
811 0.0271270036991369
812 0.0270935960591133
813 0.027060270602706
814 0.027027027027027
815 0.0269938650306748
816 0.0269607843137255
817 0.0269277845777234
818 0.0268948655256724
819 0.0268620268620269
820 0.0268292682926829
821 0.0267965895249695
822 0.0267639902676399
823 0.0267314702308627
824 0.0266990291262136
825 0.0266666666666667
826 0.026634382566586
827 0.026602176541717
828 0.0265700483091787
829 0.0265379975874548
830 0.0265060240963855
831 0.02647412755716
832 0.0264423076923077
833 0.0264105642256903
834 0.026378896882494
835 0.0263473053892216
836 0.0263157894736842
837 0.026284348864994
838 0.0262529832935561
839 0.0262216924910608
840 0.0261904761904762
841 0.0261593341260404
842 0.0261282660332542
843 0.0260972716488731
844 0.0260663507109005
845 0.0260355029585799
846 0.0260047281323877
847 0.025974025974026
848 0.0259433962264151
849 0.0259128386336867
850 0.0270588235294118
851 0.027027027027027
852 0.0269953051643192
853 0.0269636576787808
854 0.0269320843091335
855 0.0269005847953216
856 0.0268691588785047
857 0.0268378063010502
858 0.0268065268065268
859 0.0267753201396973
860 0.0267441860465116
861 0.0267131242740999
862 0.0266821345707657
863 0.0266512166859791
864 0.0266203703703704
865 0.0265895953757225
866 0.0265588914549654
867 0.0265282583621684
868 0.0264976958525346
869 0.0264672036823936
870 0.0264367816091954
871 0.026406429391504
872 0.0263761467889908
873 0.0263459335624284
874 0.0263157894736842
875 0.0262857142857143
876 0.0262557077625571
877 0.0262257696693273
878 0.0261958997722096
879 0.0261660978384528
880 0.0261363636363636
881 0.0261066969353008
882 0.0260770975056689
883 0.0260475651189128
884 0.0260180995475113
885 0.0259887005649717
886 0.0259593679458239
887 0.0259301014656144
888 0.0259009009009009
889 0.0258717660292463
890 0.0258426966292135
891 0.0258136924803591
892 0.0257847533632287
893 0.0257558790593505
894 0.0257270693512304
895 0.0256983240223464
896 0.0256696428571429
897 0.0256410256410256
898 0.0256124721603563
899 0.0255839822024472
900 0.0255555555555556
901 0.025527192008879
902 0.0254988913525499
903 0.0254706533776301
904 0.0254424778761062
905 0.025414364640884
906 0.0253863134657837
907 0.0253583241455347
908 0.0253303964757709
909 0.0264026402640264
910 0.0263736263736264
911 0.026344676180022
912 0.0263157894736842
913 0.0262869660460022
914 0.0262582056892779
915 0.0262295081967213
916 0.0262008733624454
917 0.0261723009814613
918 0.0261437908496732
919 0.0261153427638738
920 0.0260869565217391
921 0.0260586319218241
922 0.0260303687635575
923 0.0270855904658722
924 0.0270562770562771
925 0.027027027027027
926 0.0269978401727862
927 0.0269687162891046
928 0.0269396551724138
929 0.0269106566200215
930 0.0268817204301075
931 0.0268528464017186
932 0.0268240343347639
933 0.0267952840300107
934 0.0267665952890792
935 0.0267379679144385
936 0.0267094017094017
937 0.0266808964781217
938 0.0277185501066098
939 0.0276890308839191
940 0.0276595744680851
941 0.0276301806588735
942 0.0276008492569002
943 0.0275715800636267
944 0.0275423728813559
945 0.0275132275132275
946 0.0274841437632135
947 0.027455121436114
948 0.0274261603375527
949 0.0273972602739726
950 0.0273684210526316
951 0.0273396424815983
952 0.0273109243697479
953 0.0272822665267576
954 0.0272536687631027
955 0.0272251308900524
956 0.0271966527196653
957 0.0271682340647858
958 0.0271398747390397
959 0.02711157455683
960 0.0270833333333333
961 0.0270551508844953
962 0.027027027027027
963 0.0269989615784008
964 0.0269709543568465
965 0.0269430051813472
966 0.0269151138716356
967 0.0268872802481903
968 0.0268595041322314
969 0.0268317853457172
970 0.0268041237113402
971 0.0267765190525232
972 0.0267489711934156
973 0.02672147995889
974 0.026694045174538
975 0.0266666666666667
976 0.0266393442622951
977 0.0266120777891505
978 0.0265848670756646
979 0.0265577119509704
980 0.026530612244898
981 0.0265035677879715
982 0.0264765784114053
983 0.0264496439471007
984 0.0264227642276423
985 0.0263959390862944
986 0.026369168356998
987 0.0263424518743668
988 0.0263157894736842
989 0.0262891809908999
990 0.0262626262626263
991 0.0262361251261352
992 0.0262096774193548
993 0.0261832829808661
994 0.0261569416498994
995 0.0271356783919598
996 0.0271084337349398
997 0.0270812437311936
998 0.0270541082164329
999 0.027027027027027
1000 0.027
};
\addplot [semithick, white!49.8039215686275!black, forget plot]
table {%
1 0
2 0
3 0
4 0
5 0
6 0
7 0
8 0
9 0
10 0.1
11 0.0909090909090909
12 0.0833333333333333
13 0.0769230769230769
14 0.0714285714285714
15 0.0666666666666667
16 0.125
17 0.117647058823529
18 0.166666666666667
19 0.157894736842105
20 0.2
21 0.19047619047619
22 0.181818181818182
23 0.217391304347826
24 0.208333333333333
25 0.2
26 0.192307692307692
27 0.185185185185185
28 0.178571428571429
29 0.172413793103448
30 0.166666666666667
31 0.161290322580645
32 0.15625
33 0.151515151515152
34 0.147058823529412
35 0.142857142857143
36 0.138888888888889
37 0.135135135135135
38 0.131578947368421
39 0.128205128205128
40 0.125
41 0.121951219512195
42 0.119047619047619
43 0.116279069767442
44 0.113636363636364
45 0.111111111111111
46 0.108695652173913
47 0.106382978723404
48 0.104166666666667
49 0.102040816326531
50 0.1
51 0.0980392156862745
52 0.0961538461538462
53 0.0943396226415094
54 0.0925925925925926
55 0.0909090909090909
56 0.0892857142857143
57 0.087719298245614
58 0.0862068965517241
59 0.0847457627118644
60 0.0833333333333333
61 0.0819672131147541
62 0.0806451612903226
63 0.0793650793650794
64 0.078125
65 0.0769230769230769
66 0.0757575757575758
67 0.0746268656716418
68 0.0735294117647059
69 0.072463768115942
70 0.0714285714285714
71 0.0704225352112676
72 0.0694444444444444
73 0.0684931506849315
74 0.0675675675675676
75 0.0666666666666667
76 0.0657894736842105
77 0.0649350649350649
78 0.0641025641025641
79 0.0632911392405063
80 0.0625
81 0.0617283950617284
82 0.0609756097560976
83 0.0602409638554217
84 0.0595238095238095
85 0.0588235294117647
86 0.0581395348837209
87 0.0574712643678161
88 0.0568181818181818
89 0.0561797752808989
90 0.0555555555555556
91 0.0549450549450549
92 0.0543478260869565
93 0.0537634408602151
94 0.0531914893617021
95 0.0526315789473684
96 0.0520833333333333
97 0.0515463917525773
98 0.0510204081632653
99 0.0505050505050505
100 0.05
101 0.0495049504950495
102 0.0490196078431373
103 0.0485436893203883
104 0.0480769230769231
105 0.0476190476190476
106 0.0471698113207547
107 0.0467289719626168
108 0.0462962962962963
109 0.0458715596330275
110 0.0454545454545455
111 0.045045045045045
112 0.0446428571428571
113 0.0442477876106195
114 0.043859649122807
115 0.0434782608695652
116 0.0431034482758621
117 0.0427350427350427
118 0.0423728813559322
119 0.0420168067226891
120 0.0416666666666667
121 0.0413223140495868
122 0.040983606557377
123 0.040650406504065
124 0.0403225806451613
125 0.04
126 0.0396825396825397
127 0.0393700787401575
128 0.0390625
129 0.0387596899224806
130 0.0384615384615385
131 0.0381679389312977
132 0.0454545454545455
133 0.0451127819548872
134 0.0447761194029851
135 0.0444444444444444
136 0.0441176470588235
137 0.0437956204379562
138 0.0434782608695652
139 0.0431654676258993
140 0.0428571428571429
141 0.0425531914893617
142 0.0422535211267606
143 0.041958041958042
144 0.0416666666666667
145 0.0413793103448276
146 0.0410958904109589
147 0.0408163265306122
148 0.0405405405405405
149 0.0402684563758389
150 0.04
151 0.0397350993377483
152 0.0460526315789474
153 0.0457516339869281
154 0.0454545454545455
155 0.0451612903225806
156 0.0448717948717949
157 0.0445859872611465
158 0.0443037974683544
159 0.0440251572327044
160 0.04375
161 0.0434782608695652
162 0.0432098765432099
163 0.0429447852760736
164 0.0426829268292683
165 0.0424242424242424
166 0.0421686746987952
167 0.0419161676646707
168 0.0416666666666667
169 0.0414201183431953
170 0.0411764705882353
171 0.0409356725146199
172 0.0465116279069767
173 0.046242774566474
174 0.0459770114942529
175 0.0457142857142857
176 0.0454545454545455
177 0.0451977401129944
178 0.0449438202247191
179 0.0446927374301676
180 0.0444444444444444
181 0.0441988950276243
182 0.043956043956044
183 0.0437158469945355
184 0.0434782608695652
185 0.0432432432432432
186 0.043010752688172
187 0.0427807486631016
188 0.0425531914893617
189 0.0423280423280423
190 0.0421052631578947
191 0.0418848167539267
192 0.0416666666666667
193 0.0414507772020725
194 0.0412371134020619
195 0.041025641025641
196 0.0408163265306122
197 0.0406091370558376
198 0.0404040404040404
199 0.0402010050251256
200 0.04
201 0.0398009950248756
202 0.0445544554455446
203 0.0443349753694581
204 0.0441176470588235
205 0.0439024390243902
206 0.0436893203883495
207 0.0434782608695652
208 0.0432692307692308
209 0.0430622009569378
210 0.0428571428571429
211 0.042654028436019
212 0.0424528301886792
213 0.0422535211267606
214 0.0420560747663551
215 0.0418604651162791
216 0.0416666666666667
217 0.0414746543778802
218 0.0412844036697248
219 0.0410958904109589
220 0.0409090909090909
221 0.0407239819004525
222 0.0405405405405405
223 0.0403587443946188
224 0.0401785714285714
225 0.04
226 0.0398230088495575
227 0.039647577092511
228 0.0394736842105263
229 0.0393013100436681
230 0.0391304347826087
231 0.038961038961039
232 0.0387931034482759
233 0.0386266094420601
234 0.0384615384615385
235 0.0382978723404255
236 0.038135593220339
237 0.0379746835443038
238 0.0378151260504202
239 0.0376569037656904
240 0.0375
241 0.037344398340249
242 0.0371900826446281
243 0.037037037037037
244 0.0368852459016393
245 0.036734693877551
246 0.0365853658536585
247 0.0364372469635627
248 0.0362903225806452
249 0.036144578313253
250 0.036
251 0.0358565737051793
252 0.0357142857142857
253 0.0355731225296443
254 0.0354330708661417
255 0.0352941176470588
256 0.03515625
257 0.0350194552529183
258 0.0348837209302326
259 0.0347490347490347
260 0.0346153846153846
261 0.0344827586206897
262 0.0343511450381679
263 0.0342205323193916
264 0.0340909090909091
265 0.0339622641509434
266 0.0338345864661654
267 0.0337078651685393
268 0.0335820895522388
269 0.033457249070632
270 0.0333333333333333
271 0.033210332103321
272 0.0330882352941176
273 0.032967032967033
274 0.0328467153284672
275 0.0327272727272727
276 0.0326086956521739
277 0.0324909747292419
278 0.0323741007194245
279 0.032258064516129
280 0.0321428571428571
281 0.0320284697508897
282 0.0354609929078014
283 0.0353356890459364
284 0.0352112676056338
285 0.0350877192982456
286 0.034965034965035
287 0.0348432055749129
288 0.0347222222222222
289 0.0346020761245675
290 0.0344827586206897
291 0.0343642611683849
292 0.0342465753424658
293 0.0341296928327645
294 0.0340136054421769
295 0.0338983050847458
296 0.0337837837837838
297 0.0336700336700337
298 0.0335570469798658
299 0.0334448160535117
300 0.0333333333333333
301 0.0365448504983389
302 0.0364238410596026
303 0.0363036303630363
304 0.0361842105263158
305 0.0360655737704918
306 0.0359477124183007
307 0.0358306188925081
308 0.0357142857142857
309 0.0355987055016181
310 0.0354838709677419
311 0.0353697749196141
312 0.0352564102564103
313 0.0351437699680511
314 0.035031847133758
315 0.0349206349206349
316 0.0348101265822785
317 0.0347003154574132
318 0.0345911949685535
319 0.0344827586206897
320 0.034375
321 0.0342679127725857
322 0.0341614906832298
323 0.0340557275541796
324 0.0339506172839506
325 0.0338461538461538
326 0.0337423312883436
327 0.0336391437308868
328 0.0335365853658537
329 0.033434650455927
330 0.0333333333333333
331 0.0332326283987915
332 0.0331325301204819
333 0.033033033033033
334 0.0329341317365269
335 0.0328358208955224
336 0.0327380952380952
337 0.0326409495548961
338 0.0325443786982249
339 0.0324483775811209
340 0.0323529411764706
341 0.032258064516129
342 0.0321637426900585
343 0.032069970845481
344 0.0319767441860465
345 0.0318840579710145
346 0.0317919075144509
347 0.031700288184438
348 0.0344827586206897
349 0.0343839541547278
350 0.0342857142857143
351 0.0341880341880342
352 0.0340909090909091
353 0.0339943342776204
354 0.0338983050847458
355 0.0338028169014084
356 0.0337078651685393
357 0.0336134453781513
358 0.0335195530726257
359 0.0334261838440111
360 0.0333333333333333
361 0.0332409972299169
362 0.0331491712707182
363 0.0330578512396694
364 0.032967032967033
365 0.0328767123287671
366 0.0327868852459016
367 0.0326975476839237
368 0.0326086956521739
369 0.032520325203252
370 0.0324324324324324
371 0.032345013477089
372 0.032258064516129
373 0.032171581769437
374 0.0320855614973262
375 0.032
376 0.0319148936170213
377 0.0318302387267905
378 0.0317460317460317
379 0.0316622691292876
380 0.0315789473684211
381 0.031496062992126
382 0.031413612565445
383 0.031331592689295
384 0.03125
385 0.0311688311688312
386 0.0310880829015544
387 0.0310077519379845
388 0.0309278350515464
389 0.0308483290488432
390 0.0307692307692308
391 0.030690537084399
392 0.0306122448979592
393 0.0305343511450382
394 0.0304568527918782
395 0.030379746835443
396 0.0303030303030303
397 0.0302267002518892
398 0.0301507537688442
399 0.0300751879699248
400 0.03
401 0.0324189526184539
402 0.0323383084577114
403 0.032258064516129
404 0.0321782178217822
405 0.0320987654320988
406 0.0320197044334975
407 0.0319410319410319
408 0.0318627450980392
409 0.0317848410757946
410 0.0317073170731707
411 0.0316301703163017
412 0.0315533980582524
413 0.0314769975786925
414 0.0314009661835749
415 0.0313253012048193
416 0.03125
417 0.0311750599520384
418 0.0311004784688995
419 0.0310262529832936
420 0.030952380952381
421 0.0308788598574822
422 0.0308056872037915
423 0.0307328605200946
424 0.0306603773584906
425 0.0305882352941176
426 0.0305164319248826
427 0.0304449648711944
428 0.0303738317757009
429 0.0303030303030303
430 0.0302325581395349
431 0.0301624129930394
432 0.0300925925925926
433 0.0300230946882217
434 0.0299539170506912
435 0.0298850574712644
436 0.0298165137614679
437 0.0297482837528604
438 0.0296803652968037
439 0.0296127562642369
440 0.0295454545454545
441 0.0294784580498866
442 0.0294117647058824
443 0.0293453724604966
444 0.0292792792792793
445 0.0314606741573034
446 0.031390134529148
447 0.0313199105145414
448 0.03125
449 0.0311804008908686
450 0.0311111111111111
451 0.0310421286031042
452 0.0309734513274336
453 0.0309050772626932
454 0.0308370044052863
455 0.0307692307692308
456 0.0307017543859649
457 0.0306345733041575
458 0.0305676855895196
459 0.0305010893246187
460 0.0304347826086957
461 0.0303687635574837
462 0.0303030303030303
463 0.0302375809935205
464 0.0301724137931034
465 0.0301075268817204
466 0.0300429184549356
467 0.0299785867237687
468 0.0299145299145299
469 0.0298507462686567
470 0.0297872340425532
471 0.029723991507431
472 0.0296610169491525
473 0.0295983086680761
474 0.029535864978903
475 0.0294736842105263
476 0.0294117647058824
477 0.0293501048218029
478 0.0292887029288703
479 0.0292275574112735
480 0.0291666666666667
481 0.0291060291060291
482 0.029045643153527
483 0.0289855072463768
484 0.0289256198347107
485 0.0288659793814433
486 0.0288065843621399
487 0.0287474332648871
488 0.0286885245901639
489 0.0286298568507157
490 0.0285714285714286
491 0.0285132382892057
492 0.0284552845528455
493 0.0283975659229209
494 0.0283400809716599
495 0.0282828282828283
496 0.0282258064516129
497 0.028169014084507
498 0.0281124497991968
499 0.0280561122244489
500 0.028
501 0.0279441117764471
502 0.0278884462151394
503 0.0278330019880716
504 0.0277777777777778
505 0.0277227722772277
506 0.0276679841897233
507 0.0276134122287968
508 0.0275590551181102
509 0.0275049115913556
510 0.0274509803921569
511 0.0293542074363992
512 0.029296875
513 0.0292397660818713
514 0.0291828793774319
515 0.029126213592233
516 0.0290697674418605
517 0.0290135396518375
518 0.028957528957529
519 0.0289017341040462
520 0.0288461538461538
521 0.0307101727447217
522 0.0306513409961686
523 0.0325047801147228
524 0.0324427480916031
525 0.0323809523809524
526 0.032319391634981
527 0.032258064516129
528 0.0321969696969697
529 0.0321361058601134
530 0.0320754716981132
531 0.032015065913371
532 0.0319548872180451
533 0.0318949343339587
534 0.0318352059925094
535 0.0317757009345794
536 0.0317164179104478
537 0.031657355679702
538 0.0315985130111524
539 0.0315398886827458
540 0.0314814814814815
541 0.0314232902033272
542 0.0313653136531365
543 0.0313075506445672
544 0.03125
545 0.0311926605504587
546 0.0311355311355311
547 0.0310786106032907
548 0.031021897810219
549 0.0309653916211293
550 0.0309090909090909
551 0.0308529945553539
552 0.0307971014492754
553 0.0307414104882459
554 0.0306859205776173
555 0.0306306306306306
556 0.0305755395683453
557 0.0305206463195691
558 0.0304659498207885
559 0.0304114490161002
560 0.0303571428571429
561 0.0303030303030303
562 0.0302491103202847
563 0.0301953818827709
564 0.0301418439716312
565 0.0300884955752212
566 0.0300353356890459
567 0.0299823633156966
568 0.0299295774647887
569 0.0298769771528998
570 0.0298245614035088
571 0.0297723292469352
572 0.0297202797202797
573 0.0296684118673647
574 0.029616724738676
575 0.0295652173913043
576 0.0295138888888889
577 0.0294627383015598
578 0.0294117647058824
579 0.0293609671848014
580 0.0293103448275862
581 0.0292598967297762
582 0.0292096219931271
583 0.0291595197255575
584 0.0291095890410959
585 0.0290598290598291
586 0.0290102389078498
587 0.0289608177172061
588 0.0289115646258503
589 0.0288624787775891
590 0.0288135593220339
591 0.0287648054145516
592 0.0287162162162162
593 0.0286677908937605
594 0.0286195286195286
595 0.0285714285714286
596 0.0285234899328859
597 0.0284757118927973
598 0.0284280936454849
599 0.0283806343906511
600 0.0283333333333333
601 0.0282861896838602
602 0.0282392026578073
603 0.0281923714759536
604 0.0281456953642384
605 0.028099173553719
606 0.0280528052805281
607 0.028006589785832
608 0.0279605263157895
609 0.0279146141215107
610 0.0278688524590164
611 0.027823240589198
612 0.0277777777777778
613 0.0277324632952692
614 0.0276872964169381
615 0.0276422764227642
616 0.0275974025974026
617 0.0275526742301459
618 0.0275080906148867
619 0.0274636510500808
620 0.0274193548387097
621 0.0273752012882448
622 0.0273311897106109
623 0.028892455858748
624 0.0288461538461538
625 0.0288
626 0.0287539936102236
627 0.0287081339712919
628 0.0286624203821656
629 0.0286168521462639
630 0.0285714285714286
631 0.0285261489698891
632 0.0284810126582278
633 0.028436018957346
634 0.028391167192429
635 0.0283464566929134
636 0.0283018867924528
637 0.0282574568288854
638 0.0282131661442006
639 0.028169014084507
640 0.028125
641 0.0280811232449298
642 0.0280373831775701
643 0.0279937791601866
644 0.0279503105590062
645 0.0294573643410853
646 0.0294117647058824
647 0.0293663060278207
648 0.029320987654321
649 0.0292758089368259
650 0.0292307692307692
651 0.0291858678955453
652 0.0291411042944785
653 0.0290964777947933
654 0.0290519877675841
655 0.0290076335877863
656 0.0289634146341463
657 0.0289193302891933
658 0.0288753799392097
659 0.0288315629742033
660 0.0287878787878788
661 0.0287443267776097
662 0.0287009063444109
663 0.028657616892911
664 0.0286144578313253
665 0.0285714285714286
666 0.0285285285285285
667 0.0284857571214393
668 0.0284431137724551
669 0.0284005979073244
670 0.0283582089552239
671 0.029806259314456
672 0.0297619047619048
673 0.0297176820208024
674 0.029673590504451
675 0.0296296296296296
676 0.029585798816568
677 0.0295420974889217
678 0.0294985250737463
679 0.0294550810014728
680 0.0294117647058824
681 0.0293685756240822
682 0.0293255131964809
683 0.0292825768667643
684 0.0292397660818713
685 0.0291970802919708
686 0.0291545189504373
687 0.0291120815138282
688 0.0290697674418605
689 0.0290275761973875
690 0.0289855072463768
691 0.0289435600578871
692 0.0289017341040462
693 0.0288600288600289
694 0.0288184438040346
695 0.0287769784172662
696 0.028735632183908
697 0.0286944045911047
698 0.0286532951289398
699 0.0286123032904149
700 0.03
701 0.0299572039942939
702 0.0299145299145299
703 0.0298719772403983
704 0.0298295454545455
705 0.0297872340425532
706 0.0297450424929178
707 0.0297029702970297
708 0.0296610169491525
709 0.0296191819464034
710 0.0295774647887324
711 0.0309423347398031
712 0.0308988764044944
713 0.0308555399719495
714 0.030812324929972
715 0.0307692307692308
716 0.0307262569832402
717 0.0306834030683403
718 0.0306406685236769
719 0.0305980528511822
720 0.0305555555555556
721 0.0305131761442441
722 0.0318559556786704
723 0.0318118948824343
724 0.031767955801105
725 0.0317241379310345
726 0.0316804407713499
727 0.031636863823934
728 0.0315934065934066
729 0.0315500685871056
730 0.0315068493150685
731 0.0314637482900137
732 0.0314207650273224
733 0.0313778990450205
734 0.0313351498637602
735 0.0312925170068027
736 0.03125
737 0.0312075983717775
738 0.0311653116531165
739 0.0311231393775372
740 0.0310810810810811
741 0.0323886639676113
742 0.032345013477089
743 0.0323014804845222
744 0.0336021505376344
745 0.0335570469798658
746 0.0335120643431635
747 0.0334672021419009
748 0.0334224598930481
749 0.0333778371161549
750 0.0333333333333333
751 0.033288948069241
752 0.0332446808510638
753 0.0332005312084993
754 0.0331564986737401
755 0.033112582781457
756 0.0330687830687831
757 0.0330250990752972
758 0.0329815303430079
759 0.0329380764163373
760 0.0328947368421053
761 0.0328515111695138
762 0.0328083989501312
763 0.0327653997378768
764 0.0327225130890052
765 0.0326797385620915
766 0.0326370757180157
767 0.0325945241199478
768 0.0325520833333333
769 0.0325097529258778
770 0.0324675324675325
771 0.0324254215304799
772 0.0323834196891192
773 0.0323415265200517
774 0.0322997416020672
775 0.032258064516129
776 0.0322164948453608
777 0.0321750321750322
778 0.032133676092545
779 0.0320924261874198
780 0.032051282051282
781 0.0320102432778489
782 0.0319693094629156
783 0.0319284802043423
784 0.0318877551020408
785 0.0318471337579618
786 0.0318066157760814
787 0.0317662007623888
788 0.0317258883248731
789 0.0316856780735108
790 0.0316455696202532
791 0.0316055625790139
792 0.0315656565656566
793 0.0327868852459016
794 0.0327455919395466
795 0.0327044025157233
796 0.0326633165829146
797 0.0326223337515684
798 0.0325814536340852
799 0.032540675844806
800 0.0325
801 0.0324594257178527
802 0.0324189526184539
803 0.0323785803237858
804 0.0323383084577114
805 0.0322981366459627
806 0.032258064516129
807 0.0322180916976456
808 0.0321782178217822
809 0.0321384425216316
810 0.0320987654320988
811 0.0332922318125771
812 0.0332512315270936
813 0.034440344403444
814 0.0343980343980344
815 0.0343558282208589
816 0.0343137254901961
817 0.0342717258261934
818 0.0342298288508557
819 0.0341880341880342
820 0.0341463414634146
821 0.0341047503045067
822 0.0340632603406326
823 0.0340218712029162
824 0.0339805825242718
825 0.0339393939393939
826 0.0338983050847458
827 0.033857315598549
828 0.0338164251207729
829 0.0337756332931242
830 0.0337349397590361
831 0.0336943441636582
832 0.0336538461538462
833 0.0336134453781513
834 0.0335731414868106
835 0.0335329341317365
836 0.0334928229665072
837 0.033452807646356
838 0.0334128878281623
839 0.033373063170441
840 0.0333333333333333
841 0.0332936979785969
842 0.0332541567695962
843 0.033214709371293
844 0.033175355450237
845 0.0331360946745562
846 0.033096926713948
847 0.0330578512396694
848 0.0330188679245283
849 0.032979976442874
850 0.0329411764705882
851 0.0329024676850764
852 0.0328638497652582
853 0.0328253223915592
854 0.0327868852459016
855 0.0327485380116959
856 0.0327102803738318
857 0.0326721120186698
858 0.0326340326340326
859 0.0325960419091967
860 0.0325581395348837
861 0.032520325203252
862 0.0324825986078886
863 0.0324449594438007
864 0.0324074074074074
865 0.0323699421965318
866 0.0323325635103926
867 0.0322952710495963
868 0.032258064516129
869 0.0322209436133487
870 0.032183908045977
871 0.0321469575200918
872 0.0321100917431193
873 0.0320733104238259
874 0.0320366132723112
875 0.032
876 0.0319634703196347
877 0.031927023945268
878 0.0318906605922551
879 0.0318543799772469
880 0.0318181818181818
881 0.0317820658342792
882 0.0317460317460317
883 0.0317100792751982
884 0.0316742081447964
885 0.031638418079096
886 0.0316027088036117
887 0.0315670800450958
888 0.0315315315315315
889 0.031496062992126
890 0.0314606741573034
891 0.0314253647586981
892 0.031390134529148
893 0.0313549832026876
894 0.0313199105145414
895 0.0312849162011173
896 0.03125
897 0.0312151616499443
898 0.0311804008908686
899 0.0311457174638487
900 0.0311111111111111
901 0.0310765815760266
902 0.0321507760532151
903 0.0321151716500554
904 0.0320796460176991
905 0.0320441988950276
906 0.0320088300220751
907 0.0319735391400221
908 0.0319383259911894
909 0.0319031903190319
910 0.0318681318681319
911 0.0318331503841932
912 0.0317982456140351
913 0.031763417305586
914 0.0317286652078775
915 0.0316939890710383
916 0.0316593886462882
917 0.0316248636859324
918 0.0315904139433551
919 0.0315560391730141
920 0.0315217391304348
921 0.0314875135722041
922 0.0314533622559653
923 0.0314192849404117
924 0.0313852813852814
925 0.0313513513513514
926 0.031317494600432
927 0.0312837108953614
928 0.03125
929 0.031216361679225
930 0.0311827956989247
931 0.0322234156820623
932 0.0321888412017167
933 0.0321543408360129
934 0.0321199143468951
935 0.0320855614973262
936 0.032051282051282
937 0.032017075773746
938 0.0319829424307036
939 0.0319488817891374
940 0.0319148936170213
941 0.0318809776833156
942 0.0318471337579618
943 0.031813361611877
944 0.0317796610169492
945 0.0317460317460317
946 0.0317124735729387
947 0.0316789862724393
948 0.0316455696202532
949 0.0316122233930453
950 0.0315789473684211
951 0.0315457413249211
952 0.0315126050420168
953 0.0314795383001049
954 0.0314465408805031
955 0.031413612565445
956 0.0313807531380753
957 0.0313479623824451
958 0.0313152400835073
959 0.0312825860271116
960 0.03125
961 0.0312174817898023
962 0.0311850311850312
963 0.0311526479750779
964 0.0311203319502075
965 0.0310880829015544
966 0.031055900621118
967 0.031023784901758
968 0.0309917355371901
969 0.0309597523219814
970 0.0319587628865979
971 0.0319258496395469
972 0.0318930041152263
973 0.0318602261048304
974 0.0318275154004107
975 0.0317948717948718
976 0.0317622950819672
977 0.0317297850562948
978 0.0316973415132924
979 0.0316649642492339
980 0.0316326530612245
981 0.0316004077471967
982 0.0315682281059063
983 0.0315361139369278
984 0.0315040650406504
985 0.0314720812182741
986 0.0314401622718053
987 0.0314083080040527
988 0.0313765182186235
989 0.0313447927199191
990 0.0313131313131313
991 0.0312815338042381
992 0.03125
993 0.0312185297079557
994 0.0311871227364185
995 0.0311557788944724
996 0.0311244979919679
997 0.0310932798395186
998 0.031062124248497
999 0.031031031031031
1000 0.031
};
\addplot [semithick, color7, forget plot]
table {%
1 0
2 0
3 0
4 0
5 0
6 0
7 0
8 0
9 0
10 0
11 0
12 0
13 0
14 0
15 0
16 0
17 0
18 0
19 0
20 0
21 0
22 0
23 0
24 0
25 0
26 0
27 0
28 0
29 0
30 0
31 0
32 0
33 0
34 0
35 0
36 0
37 0
38 0
39 0
40 0
41 0
42 0
43 0
44 0
45 0
46 0
47 0
48 0
49 0
50 0
51 0
52 0
53 0
54 0
55 0
56 0
57 0
58 0
59 0
60 0
61 0
62 0
63 0
64 0
65 0
66 0
67 0
68 0
69 0
70 0
71 0
72 0
73 0
74 0
75 0
76 0
77 0
78 0
79 0.0126582278481013
80 0.0125
81 0.0123456790123457
82 0.0121951219512195
83 0.0120481927710843
84 0.0119047619047619
85 0.0117647058823529
86 0.0116279069767442
87 0.0114942528735632
88 0.0113636363636364
89 0.0112359550561798
90 0.0111111111111111
91 0.010989010989011
92 0.0108695652173913
93 0.010752688172043
94 0.0106382978723404
95 0.0105263157894737
96 0.0104166666666667
97 0.0103092783505155
98 0.0102040816326531
99 0.0101010101010101
100 0.01
101 0.0099009900990099
102 0.00980392156862745
103 0.00970873786407767
104 0.00961538461538462
105 0.00952380952380952
106 0.00943396226415094
107 0.00934579439252336
108 0.00925925925925926
109 0.00917431192660551
110 0.00909090909090909
111 0.00900900900900901
112 0.00892857142857143
113 0.00884955752212389
114 0.0087719298245614
115 0.00869565217391304
116 0.00862068965517241
117 0.00854700854700855
118 0.00847457627118644
119 0.00840336134453781
120 0.00833333333333333
121 0.00826446280991736
122 0.0163934426229508
123 0.016260162601626
124 0.0161290322580645
125 0.016
126 0.0158730158730159
127 0.015748031496063
128 0.015625
129 0.0155038759689922
130 0.0153846153846154
131 0.0152671755725191
132 0.0151515151515152
133 0.0150375939849624
134 0.0149253731343284
135 0.0148148148148148
136 0.0147058823529412
137 0.0145985401459854
138 0.0144927536231884
139 0.0143884892086331
140 0.0142857142857143
141 0.0141843971631206
142 0.0140845070422535
143 0.013986013986014
144 0.0138888888888889
145 0.0137931034482759
146 0.0136986301369863
147 0.0136054421768707
148 0.0135135135135135
149 0.0134228187919463
150 0.0133333333333333
151 0.0132450331125828
152 0.0131578947368421
153 0.0130718954248366
154 0.012987012987013
155 0.0129032258064516
156 0.0128205128205128
157 0.0127388535031847
158 0.0126582278481013
159 0.0125786163522013
160 0.0125
161 0.0124223602484472
162 0.0123456790123457
163 0.0122699386503067
164 0.0121951219512195
165 0.0121212121212121
166 0.0120481927710843
167 0.0119760479041916
168 0.0119047619047619
169 0.0118343195266272
170 0.0117647058823529
171 0.0116959064327485
172 0.0116279069767442
173 0.0115606936416185
174 0.0114942528735632
175 0.0114285714285714
176 0.0113636363636364
177 0.0112994350282486
178 0.0112359550561798
179 0.0111731843575419
180 0.0111111111111111
181 0.0110497237569061
182 0.010989010989011
183 0.0109289617486339
184 0.0108695652173913
185 0.0108108108108108
186 0.010752688172043
187 0.0106951871657754
188 0.0106382978723404
189 0.0105820105820106
190 0.0105263157894737
191 0.0104712041884817
192 0.0104166666666667
193 0.0103626943005181
194 0.0103092783505155
195 0.0102564102564103
196 0.0102040816326531
197 0.0101522842639594
198 0.0101010101010101
199 0.0100502512562814
200 0.01
201 0.00995024875621891
202 0.0099009900990099
203 0.00985221674876847
204 0.00980392156862745
205 0.00975609756097561
206 0.00970873786407767
207 0.00966183574879227
208 0.00961538461538462
209 0.00956937799043062
210 0.00952380952380952
211 0.00947867298578199
212 0.00943396226415094
213 0.00938967136150235
214 0.00934579439252336
215 0.00930232558139535
216 0.00925925925925926
217 0.00921658986175115
218 0.00917431192660551
219 0.0091324200913242
220 0.00909090909090909
221 0.00904977375565611
222 0.00900900900900901
223 0.00896860986547085
224 0.0133928571428571
225 0.0133333333333333
226 0.0132743362831858
227 0.013215859030837
228 0.0131578947368421
229 0.0131004366812227
230 0.0130434782608696
231 0.012987012987013
232 0.0129310344827586
233 0.0128755364806867
234 0.0128205128205128
235 0.0127659574468085
236 0.0127118644067797
237 0.0168776371308017
238 0.0168067226890756
239 0.0167364016736402
240 0.0166666666666667
241 0.016597510373444
242 0.0165289256198347
243 0.0164609053497942
244 0.0163934426229508
245 0.0163265306122449
246 0.016260162601626
247 0.0161943319838057
248 0.0161290322580645
249 0.0160642570281124
250 0.016
251 0.0159362549800797
252 0.0158730158730159
253 0.0158102766798419
254 0.015748031496063
255 0.0156862745098039
256 0.015625
257 0.0155642023346304
258 0.0155038759689922
259 0.0154440154440154
260 0.0153846153846154
261 0.0153256704980843
262 0.0152671755725191
263 0.0152091254752852
264 0.0151515151515152
265 0.0150943396226415
266 0.0150375939849624
267 0.0149812734082397
268 0.0149253731343284
269 0.0148698884758364
270 0.0148148148148148
271 0.014760147601476
272 0.0147058823529412
273 0.0146520146520147
274 0.0145985401459854
275 0.0145454545454545
276 0.0144927536231884
277 0.0144404332129964
278 0.0143884892086331
279 0.014336917562724
280 0.0142857142857143
281 0.0142348754448399
282 0.0141843971631206
283 0.0141342756183746
284 0.0140845070422535
285 0.0140350877192982
286 0.013986013986014
287 0.0139372822299652
288 0.0138888888888889
289 0.013840830449827
290 0.0137931034482759
291 0.013745704467354
292 0.0136986301369863
293 0.0136518771331058
294 0.0170068027210884
295 0.0169491525423729
296 0.0168918918918919
297 0.0168350168350168
298 0.0167785234899329
299 0.0167224080267559
300 0.0166666666666667
301 0.0166112956810631
302 0.0165562913907285
303 0.0165016501650165
304 0.0164473684210526
305 0.0163934426229508
306 0.0163398692810458
307 0.0162866449511401
308 0.0162337662337662
309 0.0161812297734628
310 0.0161290322580645
311 0.0160771704180064
312 0.016025641025641
313 0.0159744408945687
314 0.0159235668789809
315 0.0158730158730159
316 0.0158227848101266
317 0.0157728706624606
318 0.0157232704402516
319 0.0156739811912226
320 0.015625
321 0.0155763239875389
322 0.015527950310559
323 0.0154798761609907
324 0.0154320987654321
325 0.0153846153846154
326 0.0153374233128834
327 0.0152905198776758
328 0.0152439024390244
329 0.0151975683890578
330 0.0151515151515152
331 0.0151057401812689
332 0.0150602409638554
333 0.015015015015015
334 0.0149700598802395
335 0.0149253731343284
336 0.0148809523809524
337 0.0148367952522255
338 0.014792899408284
339 0.0147492625368732
340 0.0147058823529412
341 0.0146627565982405
342 0.0146198830409357
343 0.0145772594752187
344 0.0145348837209302
345 0.0144927536231884
346 0.0144508670520231
347 0.0144092219020173
348 0.014367816091954
349 0.0143266475644699
350 0.0142857142857143
351 0.0142450142450142
352 0.0142045454545455
353 0.0141643059490085
354 0.0141242937853107
355 0.0140845070422535
356 0.0140449438202247
357 0.0140056022408964
358 0.0139664804469274
359 0.0167130919220056
360 0.0166666666666667
361 0.0166204986149584
362 0.0165745856353591
363 0.0165289256198347
364 0.0164835164835165
365 0.0164383561643836
366 0.0163934426229508
367 0.0163487738419619
368 0.016304347826087
369 0.016260162601626
370 0.0162162162162162
371 0.0161725067385445
372 0.0161290322580645
373 0.0160857908847185
374 0.0160427807486631
375 0.016
376 0.0159574468085106
377 0.0159151193633952
378 0.0158730158730159
379 0.0158311345646438
380 0.0157894736842105
381 0.015748031496063
382 0.0157068062827225
383 0.0156657963446475
384 0.015625
385 0.0155844155844156
386 0.0155440414507772
387 0.0155038759689922
388 0.0154639175257732
389 0.0154241645244216
390 0.0153846153846154
391 0.0153452685421995
392 0.0153061224489796
393 0.0152671755725191
394 0.0152284263959391
395 0.0151898734177215
396 0.0151515151515152
397 0.0151133501259446
398 0.0150753768844221
399 0.0150375939849624
400 0.015
401 0.0149625935162095
402 0.0174129353233831
403 0.0173697270471464
404 0.0173267326732673
405 0.0172839506172839
406 0.0172413793103448
407 0.0171990171990172
408 0.017156862745098
409 0.0171149144254279
410 0.0170731707317073
411 0.0170316301703163
412 0.0169902912621359
413 0.0169491525423729
414 0.0169082125603865
415 0.0168674698795181
416 0.0168269230769231
417 0.0167865707434053
418 0.0167464114832536
419 0.0167064439140811
420 0.0166666666666667
421 0.0166270783847981
422 0.0165876777251185
423 0.016548463356974
424 0.0165094339622642
425 0.0164705882352941
426 0.0164319248826291
427 0.0163934426229508
428 0.0163551401869159
429 0.0163170163170163
430 0.0162790697674419
431 0.0162412993039443
432 0.0162037037037037
433 0.0161662817551963
434 0.0161290322580645
435 0.0160919540229885
436 0.0160550458715596
437 0.0160183066361556
438 0.0159817351598174
439 0.0159453302961276
440 0.0159090909090909
441 0.0158730158730159
442 0.0158371040723982
443 0.0158013544018059
444 0.0157657657657658
445 0.0157303370786517
446 0.015695067264574
447 0.0156599552572707
448 0.015625
449 0.0155902004454343
450 0.0155555555555556
451 0.0155210643015521
452 0.0154867256637168
453 0.0154525386313466
454 0.0154185022026432
455 0.0153846153846154
456 0.0153508771929825
457 0.0153172866520788
458 0.0152838427947598
459 0.0152505446623094
460 0.0152173913043478
461 0.0151843817787419
462 0.0151515151515152
463 0.0151187904967603
464 0.0150862068965517
465 0.0150537634408602
466 0.0150214592274678
467 0.0149892933618844
468 0.014957264957265
469 0.0149253731343284
470 0.0148936170212766
471 0.0148619957537155
472 0.0148305084745763
473 0.0147991543340381
474 0.0147679324894515
475 0.0168421052631579
476 0.0168067226890756
477 0.0167714884696017
478 0.0167364016736402
479 0.0167014613778706
480 0.0166666666666667
481 0.0166320166320166
482 0.016597510373444
483 0.0165631469979296
484 0.0165289256198347
485 0.0164948453608247
486 0.0164609053497942
487 0.0164271047227926
488 0.0163934426229508
489 0.016359918200409
490 0.0163265306122449
491 0.0162932790224033
492 0.016260162601626
493 0.0162271805273834
494 0.0161943319838057
495 0.0161616161616162
496 0.0161290322580645
497 0.0160965794768612
498 0.0160642570281124
499 0.0160320641282565
500 0.016
501 0.0159680638722555
502 0.0159362549800797
503 0.0159045725646123
504 0.0158730158730159
505 0.0158415841584158
506 0.0158102766798419
507 0.0157790927021696
508 0.015748031496063
509 0.0157170923379175
510 0.0156862745098039
511 0.0156555772994129
512 0.015625
513 0.0155945419103314
514 0.0155642023346304
515 0.0155339805825243
516 0.0155038759689922
517 0.0154738878143133
518 0.0154440154440154
519 0.0154142581888247
520 0.0153846153846154
521 0.0153550863723608
522 0.0153256704980843
523 0.0152963671128107
524 0.0152671755725191
525 0.0152380952380952
526 0.0152091254752852
527 0.015180265654649
528 0.0151515151515152
529 0.0151228733459357
530 0.0150943396226415
531 0.0150659133709981
532 0.0150375939849624
533 0.0168855534709193
534 0.0168539325842697
535 0.0168224299065421
536 0.0167910447761194
537 0.0167597765363128
538 0.016728624535316
539 0.0166975881261596
540 0.0166666666666667
541 0.0166358595194085
542 0.0166051660516605
543 0.0165745856353591
544 0.0165441176470588
545 0.0165137614678899
546 0.0164835164835165
547 0.0164533820840951
548 0.0164233576642336
549 0.0163934426229508
550 0.0163636363636364
551 0.0163339382940109
552 0.016304347826087
553 0.0162748643761302
554 0.0162454873646209
555 0.0162162162162162
556 0.0161870503597122
557 0.0161579892280072
558 0.0161290322580645
559 0.0161001788908766
560 0.0160714285714286
561 0.0160427807486631
562 0.0160142348754448
563 0.0159857904085258
564 0.0159574468085106
565 0.015929203539823
566 0.0159010600706714
567 0.0158730158730159
568 0.0158450704225352
569 0.015817223198594
570 0.0157894736842105
571 0.0157618213660245
572 0.0157342657342657
573 0.0157068062827225
574 0.0156794425087108
575 0.0156521739130435
576 0.015625
577 0.0155979202772964
578 0.0155709342560554
579 0.0155440414507772
580 0.0172413793103448
581 0.0172117039586919
582 0.0171821305841924
583 0.0171526586620926
584 0.0171232876712329
585 0.0170940170940171
586 0.0170648464163823
587 0.0170357751277683
588 0.0170068027210884
589 0.0169779286926995
590 0.0169491525423729
591 0.0169204737732657
592 0.0168918918918919
593 0.0168634064080944
594 0.0168350168350168
595 0.0168067226890756
596 0.0167785234899329
597 0.016750418760469
598 0.0167224080267559
599 0.01669449081803
600 0.0166666666666667
601 0.0166389351081531
602 0.0166112956810631
603 0.0165837479270315
604 0.0165562913907285
605 0.0165289256198347
606 0.0165016501650165
607 0.0164744645799012
608 0.0164473684210526
609 0.0164203612479475
610 0.0163934426229508
611 0.016366612111293
612 0.0163398692810458
613 0.0163132137030995
614 0.0162866449511401
615 0.016260162601626
616 0.0162337662337662
617 0.0162074554294976
618 0.0161812297734628
619 0.0161550888529887
620 0.0161290322580645
621 0.0161030595813204
622 0.0160771704180064
623 0.0160513643659711
624 0.016025641025641
625 0.016
626 0.0159744408945687
627 0.0159489633173844
628 0.0159235668789809
629 0.0158982511923688
630 0.0158730158730159
631 0.0158478605388273
632 0.0158227848101266
633 0.0157977883096366
634 0.0157728706624606
635 0.015748031496063
636 0.0157232704402516
637 0.0156985871271586
638 0.0156739811912226
639 0.0156494522691706
640 0.015625
641 0.015600624024961
642 0.0155763239875389
643 0.015552099533437
644 0.015527950310559
645 0.0155038759689922
646 0.0154798761609907
647 0.0154559505409583
648 0.0154320987654321
649 0.0154083204930663
650 0.0153846153846154
651 0.0153609831029186
652 0.0153374233128834
653 0.0153139356814701
654 0.0152905198776758
655 0.0152671755725191
656 0.0152439024390244
657 0.015220700152207
658 0.0151975683890578
659 0.0151745068285281
660 0.0151515151515152
661 0.0151285930408472
662 0.0151057401812689
663 0.0150829562594268
664 0.0150602409638554
665 0.0150375939849624
666 0.015015015015015
667 0.0149925037481259
668 0.0149700598802395
669 0.0149476831091181
670 0.0149253731343284
671 0.014903129657228
672 0.0148809523809524
673 0.0148588410104012
674 0.0148367952522255
675 0.0148148148148148
676 0.014792899408284
677 0.0147710487444609
678 0.0147492625368732
679 0.0147275405007364
680 0.0147058823529412
681 0.0146842878120411
682 0.0146627565982405
683 0.0146412884333821
684 0.0146198830409357
685 0.0145985401459854
686 0.0145772594752187
687 0.0145560407569141
688 0.0145348837209302
689 0.0145137880986938
690 0.0159420289855072
691 0.0159189580318379
692 0.0158959537572254
693 0.0158730158730159
694 0.015850144092219
695 0.0158273381294964
696 0.0158045977011494
697 0.0157819225251076
698 0.0157593123209169
699 0.0157367668097282
700 0.0171428571428571
701 0.0171184022824536
702 0.0170940170940171
703 0.0170697012802276
704 0.0170454545454545
705 0.0170212765957447
706 0.0169971671388102
707 0.016973125884017
708 0.0169491525423729
709 0.0169252468265162
710 0.0169014084507042
711 0.0168776371308017
712 0.0168539325842697
713 0.0168302945301543
714 0.0168067226890756
715 0.0167832167832168
716 0.0167597765363128
717 0.0167364016736402
718 0.0167130919220056
719 0.0166898470097357
720 0.0166666666666667
721 0.0166435506241331
722 0.0166204986149584
723 0.016597510373444
724 0.0165745856353591
725 0.016551724137931
726 0.0165289256198347
727 0.0165061898211829
728 0.0164835164835165
729 0.0164609053497942
730 0.0164383561643836
731 0.0164158686730506
732 0.0163934426229508
733 0.0163710777626194
734 0.0163487738419619
735 0.0163265306122449
736 0.016304347826087
737 0.0162822252374491
738 0.016260162601626
739 0.0162381596752368
740 0.0162162162162162
741 0.0161943319838057
742 0.0161725067385445
743 0.0161507402422611
744 0.0161290322580645
745 0.0161073825503356
746 0.0160857908847185
747 0.0160642570281124
748 0.0160427807486631
749 0.0160213618157543
750 0.016
751 0.0159786950732357
752 0.0159574468085106
753 0.0159362549800797
754 0.0159151193633952
755 0.0158940397350993
756 0.0158730158730159
757 0.0158520475561427
758 0.0171503957783641
759 0.0171277997364954
760 0.0171052631578947
761 0.0170827858081472
762 0.0170603674540682
763 0.0170380078636959
764 0.0170157068062827
765 0.0169934640522876
766 0.0169712793733681
767 0.0169491525423729
768 0.0169270833333333
769 0.0169050715214564
770 0.0168831168831169
771 0.0168612191958495
772 0.016839378238342
773 0.0168175937904269
774 0.0167958656330749
775 0.0167741935483871
776 0.0167525773195876
777 0.0167310167310167
778 0.0167095115681234
779 0.0166880616174583
780 0.0179487179487179
781 0.0179257362355954
782 0.0179028132992327
783 0.0178799489144317
784 0.0178571428571429
785 0.0178343949044586
786 0.0178117048346056
787 0.0177890724269377
788 0.0177664974619289
789 0.017743979721166
790 0.0177215189873418
791 0.0176991150442478
792 0.0176767676767677
793 0.0176544766708701
794 0.017632241813602
795 0.0176100628930818
796 0.0175879396984925
797 0.0188205771643664
798 0.018796992481203
799 0.0187734668335419
800 0.01875
801 0.0187265917602996
802 0.0187032418952618
803 0.0186799501867995
804 0.0186567164179104
805 0.0186335403726708
806 0.0186104218362283
807 0.0185873605947955
808 0.0185643564356436
809 0.0197775030902349
810 0.0209876543209877
811 0.0209617755856967
812 0.020935960591133
813 0.020910209102091
814 0.0208845208845209
815 0.0208588957055215
816 0.0208333333333333
817 0.0208078335373317
818 0.0207823960880196
819 0.0207570207570208
820 0.0207317073170732
821 0.0207064555420219
822 0.0206812652068127
823 0.0206561360874848
824 0.020631067961165
825 0.0206060606060606
826 0.0205811138014528
827 0.0205562273276904
828 0.0205314009661836
829 0.0205066344993969
830 0.0204819277108434
831 0.0204572803850782
832 0.0204326923076923
833 0.0204081632653061
834 0.0215827338129496
835 0.0215568862275449
836 0.0215311004784689
837 0.021505376344086
838 0.0214797136038186
839 0.0214541120381406
840 0.0214285714285714
841 0.0214030915576694
842 0.0213776722090261
843 0.0213523131672598
844 0.0213270142180095
845 0.021301775147929
846 0.0212765957446809
847 0.0212514757969303
848 0.0224056603773585
849 0.0223792697290931
850 0.0223529411764706
851 0.0223266745005875
852 0.0223004694835681
853 0.022274325908558
854 0.022248243559719
855 0.0222222222222222
856 0.022196261682243
857 0.0221703617269545
858 0.0221445221445221
859 0.0221187427240978
860 0.022093023255814
861 0.0220673635307782
862 0.0220417633410673
863 0.0220162224797219
864 0.0219907407407407
865 0.0219653179190751
866 0.0219399538106236
867 0.0219146482122261
868 0.021889400921659
869 0.0218642117376295
870 0.0229885057471264
871 0.0229621125143513
872 0.0229357798165138
873 0.0229095074455899
874 0.022883295194508
875 0.0228571428571429
876 0.0228310502283105
877 0.0228050171037628
878 0.0227790432801822
879 0.0227531285551763
880 0.0227272727272727
881 0.0227014755959137
882 0.0226757369614512
883 0.0226500566251416
884 0.0226244343891403
885 0.0225988700564972
886 0.0225733634311512
887 0.0225479143179256
888 0.0225225225225225
889 0.0224971878515186
890 0.0224719101123595
891 0.0224466891133558
892 0.0224215246636771
893 0.0223964165733483
894 0.0223713646532438
895 0.0223463687150838
896 0.0234375
897 0.0234113712374582
898 0.0233853006681514
899 0.0233592880978865
900 0.0233333333333333
901 0.02330743618202
902 0.0232815964523282
903 0.0243632336655592
904 0.0243362831858407
905 0.0243093922651934
906 0.0242825607064018
907 0.0242557883131202
908 0.0242290748898678
909 0.0242024202420242
910 0.0241758241758242
911 0.0241492864983535
912 0.0241228070175439
913 0.0240963855421687
914 0.0240700218818381
915 0.0240437158469945
916 0.0240174672489083
917 0.0239912758996728
918 0.0239651416122004
919 0.0239390642002176
920 0.0239130434782609
921 0.0238870792616721
922 0.0238611713665944
923 0.0238353196099675
924 0.0238095238095238
925 0.0237837837837838
926 0.0237580993520518
927 0.0237324703344121
928 0.0237068965517241
929 0.0236813778256189
930 0.0236559139784946
931 0.0236305048335124
932 0.0236051502145923
933 0.0235798499464094
934 0.0235546038543897
935 0.0235294117647059
936 0.0235042735042735
937 0.0234791889007471
938 0.023454157782516
939 0.0234291799787007
940 0.0234042553191489
941 0.0233793836344315
942 0.0233545647558386
943 0.0233297985153765
944 0.0233050847457627
945 0.0232804232804233
946 0.0232558139534884
947 0.0232312565997888
948 0.0232067510548523
949 0.0231822971548999
950 0.0231578947368421
951 0.0231335436382755
952 0.023109243697479
953 0.0230849947534103
954 0.0230607966457023
955 0.0230366492146597
956 0.0230125523012552
957 0.0229885057471264
958 0.022964509394572
959 0.0229405630865485
960 0.0229166666666667
961 0.0228928199791883
962 0.0228690228690229
963 0.0228452751817238
964 0.0228215767634855
965 0.0227979274611399
966 0.0227743271221532
967 0.0227507755946225
968 0.0227272727272727
969 0.022703818369453
970 0.022680412371134
971 0.0226570545829042
972 0.0226337448559671
973 0.0226104830421377
974 0.0225872689938398
975 0.0225641025641026
976 0.0225409836065574
977 0.022517911975435
978 0.0224948875255624
979 0.0224719101123595
980 0.0224489795918367
981 0.0224260958205912
982 0.0224032586558045
983 0.0223804679552391
984 0.0223577235772358
985 0.0223350253807107
986 0.0223123732251521
987 0.022289766970618
988 0.0222672064777328
989 0.0222446916076845
990 0.0232323232323232
991 0.0232088799192735
992 0.0231854838709677
993 0.0231621349446123
994 0.0231388329979879
995 0.0231155778894472
996 0.0230923694779116
997 0.0230692076228686
998 0.0230460921843687
999 0.023023023023023
1000 0.023
};
\addplot [semithick, color8, forget plot]
table {%
1 0
2 0
3 0
4 0
5 0
6 0
7 0
8 0
9 0
10 0
11 0.0909090909090909
12 0.0833333333333333
13 0.0769230769230769
14 0.0714285714285714
15 0.0666666666666667
16 0.0625
17 0.0588235294117647
18 0.0555555555555556
19 0.0526315789473684
20 0.05
21 0.0476190476190476
22 0.0454545454545455
23 0.0434782608695652
24 0.0416666666666667
25 0.04
26 0.0384615384615385
27 0.037037037037037
28 0.0357142857142857
29 0.0344827586206897
30 0.0333333333333333
31 0.032258064516129
32 0.03125
33 0.0303030303030303
34 0.0294117647058824
35 0.0285714285714286
36 0.0277777777777778
37 0.027027027027027
38 0.0263157894736842
39 0.0256410256410256
40 0.025
41 0.024390243902439
42 0.0238095238095238
43 0.0232558139534884
44 0.0227272727272727
45 0.0222222222222222
46 0.0217391304347826
47 0.0212765957446809
48 0.0208333333333333
49 0.0204081632653061
50 0.02
51 0.0196078431372549
52 0.0192307692307692
53 0.0188679245283019
54 0.0185185185185185
55 0.0181818181818182
56 0.0178571428571429
57 0.0175438596491228
58 0.0172413793103448
59 0.0169491525423729
60 0.0166666666666667
61 0.0163934426229508
62 0.0161290322580645
63 0.0158730158730159
64 0.015625
65 0.0153846153846154
66 0.0151515151515152
67 0.0149253731343284
68 0.0147058823529412
69 0.0144927536231884
70 0.0142857142857143
71 0.0140845070422535
72 0.0138888888888889
73 0.0136986301369863
74 0.0135135135135135
75 0.0133333333333333
76 0.0131578947368421
77 0.012987012987013
78 0.0128205128205128
79 0.0126582278481013
80 0.0125
81 0.0123456790123457
82 0.0121951219512195
83 0.0120481927710843
84 0.0119047619047619
85 0.0117647058823529
86 0.0116279069767442
87 0.0114942528735632
88 0.0113636363636364
89 0.0112359550561798
90 0.0111111111111111
91 0.010989010989011
92 0.0108695652173913
93 0.010752688172043
94 0.0106382978723404
95 0.0105263157894737
96 0.0104166666666667
97 0.0103092783505155
98 0.0102040816326531
99 0.0101010101010101
100 0.01
101 0.0099009900990099
102 0.00980392156862745
103 0.00970873786407767
104 0.00961538461538462
105 0.00952380952380952
106 0.00943396226415094
107 0.00934579439252336
108 0.00925925925925926
109 0.00917431192660551
110 0.00909090909090909
111 0.00900900900900901
112 0.00892857142857143
113 0.00884955752212389
114 0.0087719298245614
115 0.00869565217391304
116 0.00862068965517241
117 0.00854700854700855
118 0.00847457627118644
119 0.00840336134453781
120 0.00833333333333333
121 0.00826446280991736
122 0.00819672131147541
123 0.00813008130081301
124 0.00806451612903226
125 0.008
126 0.00793650793650794
127 0.0078740157480315
128 0.0078125
129 0.00775193798449612
130 0.00769230769230769
131 0.00763358778625954
132 0.00757575757575758
133 0.0075187969924812
134 0.00746268656716418
135 0.00740740740740741
136 0.00735294117647059
137 0.0072992700729927
138 0.0072463768115942
139 0.00719424460431655
140 0.00714285714285714
141 0.00709219858156028
142 0.00704225352112676
143 0.00699300699300699
144 0.00694444444444444
145 0.00689655172413793
146 0.00684931506849315
147 0.00680272108843537
148 0.00675675675675676
149 0.00671140939597315
150 0.00666666666666667
151 0.00662251655629139
152 0.00657894736842105
153 0.0065359477124183
154 0.00649350649350649
155 0.00645161290322581
156 0.00641025641025641
157 0.00636942675159236
158 0.0126582278481013
159 0.0125786163522013
160 0.0125
161 0.0124223602484472
162 0.0123456790123457
163 0.0122699386503067
164 0.0121951219512195
165 0.0121212121212121
166 0.0120481927710843
167 0.0119760479041916
168 0.0119047619047619
169 0.0118343195266272
170 0.0117647058823529
171 0.0116959064327485
172 0.0116279069767442
173 0.0115606936416185
174 0.0114942528735632
175 0.0114285714285714
176 0.0113636363636364
177 0.0112994350282486
178 0.0112359550561798
179 0.0111731843575419
180 0.0111111111111111
181 0.0110497237569061
182 0.010989010989011
183 0.0109289617486339
184 0.0108695652173913
185 0.0108108108108108
186 0.010752688172043
187 0.0106951871657754
188 0.0106382978723404
189 0.0105820105820106
190 0.0105263157894737
191 0.0104712041884817
192 0.0104166666666667
193 0.0103626943005181
194 0.0103092783505155
195 0.0102564102564103
196 0.0102040816326531
197 0.0101522842639594
198 0.0101010101010101
199 0.0100502512562814
200 0.01
201 0.00995024875621891
202 0.0099009900990099
203 0.00985221674876847
204 0.00980392156862745
205 0.00975609756097561
206 0.00970873786407767
207 0.00966183574879227
208 0.0144230769230769
209 0.0143540669856459
210 0.0142857142857143
211 0.014218009478673
212 0.0141509433962264
213 0.0140845070422535
214 0.014018691588785
215 0.013953488372093
216 0.0138888888888889
217 0.0138248847926267
218 0.018348623853211
219 0.0182648401826484
220 0.0181818181818182
221 0.0180995475113122
222 0.018018018018018
223 0.0179372197309417
224 0.0178571428571429
225 0.0177777777777778
226 0.0176991150442478
227 0.0176211453744493
228 0.0175438596491228
229 0.0174672489082969
230 0.0173913043478261
231 0.0173160173160173
232 0.0172413793103448
233 0.0171673819742489
234 0.0170940170940171
235 0.0170212765957447
236 0.0169491525423729
237 0.0168776371308017
238 0.0168067226890756
239 0.0167364016736402
240 0.0166666666666667
241 0.016597510373444
242 0.0165289256198347
243 0.0164609053497942
244 0.0163934426229508
245 0.0163265306122449
246 0.016260162601626
247 0.0161943319838057
248 0.0161290322580645
249 0.0160642570281124
250 0.016
251 0.0159362549800797
252 0.0158730158730159
253 0.0158102766798419
254 0.015748031496063
255 0.0156862745098039
256 0.015625
257 0.0155642023346304
258 0.0155038759689922
259 0.0154440154440154
260 0.0153846153846154
261 0.0153256704980843
262 0.0152671755725191
263 0.0152091254752852
264 0.0151515151515152
265 0.0150943396226415
266 0.0150375939849624
267 0.0149812734082397
268 0.0149253731343284
269 0.0148698884758364
270 0.0148148148148148
271 0.014760147601476
272 0.0147058823529412
273 0.0146520146520147
274 0.0145985401459854
275 0.0145454545454545
276 0.0144927536231884
277 0.0144404332129964
278 0.0143884892086331
279 0.014336917562724
280 0.0142857142857143
281 0.0142348754448399
282 0.0141843971631206
283 0.0141342756183746
284 0.0140845070422535
285 0.0140350877192982
286 0.013986013986014
287 0.0139372822299652
288 0.0138888888888889
289 0.013840830449827
290 0.0137931034482759
291 0.013745704467354
292 0.0136986301369863
293 0.0136518771331058
294 0.0136054421768707
295 0.0135593220338983
296 0.0135135135135135
297 0.0134680134680135
298 0.0134228187919463
299 0.0133779264214047
300 0.0133333333333333
301 0.0132890365448505
302 0.0132450331125828
303 0.0132013201320132
304 0.0131578947368421
305 0.0131147540983607
306 0.0130718954248366
307 0.0130293159609121
308 0.012987012987013
309 0.0129449838187702
310 0.0129032258064516
311 0.0128617363344051
312 0.0128205128205128
313 0.012779552715655
314 0.0127388535031847
315 0.0126984126984127
316 0.0126582278481013
317 0.0126182965299685
318 0.0125786163522013
319 0.0125391849529781
320 0.0125
321 0.0124610591900312
322 0.0124223602484472
323 0.0123839009287926
324 0.0123456790123457
325 0.0123076923076923
326 0.0122699386503067
327 0.0122324159021407
328 0.0121951219512195
329 0.0121580547112462
330 0.0121212121212121
331 0.0120845921450151
332 0.0120481927710843
333 0.012012012012012
334 0.0119760479041916
335 0.0119402985074627
336 0.0119047619047619
337 0.0118694362017804
338 0.0118343195266272
339 0.0117994100294985
340 0.0117647058823529
341 0.0117302052785924
342 0.0116959064327485
343 0.0116618075801749
344 0.0116279069767442
345 0.0115942028985507
346 0.0115606936416185
347 0.0115273775216138
348 0.0114942528735632
349 0.0114613180515759
350 0.0114285714285714
351 0.0113960113960114
352 0.0113636363636364
353 0.0113314447592068
354 0.0112994350282486
355 0.0112676056338028
356 0.0112359550561798
357 0.0112044817927171
358 0.0111731843575419
359 0.011142061281337
360 0.0111111111111111
361 0.0110803324099723
362 0.0110497237569061
363 0.0110192837465565
364 0.010989010989011
365 0.0136986301369863
366 0.0136612021857923
367 0.0163487738419619
368 0.016304347826087
369 0.016260162601626
370 0.0162162162162162
371 0.0161725067385445
372 0.0161290322580645
373 0.0160857908847185
374 0.0160427807486631
375 0.016
376 0.0159574468085106
377 0.0159151193633952
378 0.0158730158730159
379 0.0158311345646438
380 0.0157894736842105
381 0.015748031496063
382 0.0157068062827225
383 0.0156657963446475
384 0.015625
385 0.0155844155844156
386 0.0155440414507772
387 0.0155038759689922
388 0.0154639175257732
389 0.0154241645244216
390 0.0153846153846154
391 0.0153452685421995
392 0.0153061224489796
393 0.0152671755725191
394 0.0152284263959391
395 0.0151898734177215
396 0.0151515151515152
397 0.017632241813602
398 0.0175879396984925
399 0.0175438596491228
400 0.0175
401 0.0174563591022444
402 0.0174129353233831
403 0.0173697270471464
404 0.0173267326732673
405 0.0172839506172839
406 0.0172413793103448
407 0.0171990171990172
408 0.017156862745098
409 0.0171149144254279
410 0.0170731707317073
411 0.0170316301703163
412 0.0169902912621359
413 0.0169491525423729
414 0.0169082125603865
415 0.0168674698795181
416 0.0168269230769231
417 0.0167865707434053
418 0.0167464114832536
419 0.0167064439140811
420 0.0166666666666667
421 0.0166270783847981
422 0.0165876777251185
423 0.0189125295508274
424 0.0188679245283019
425 0.0188235294117647
426 0.0187793427230047
427 0.0187353629976581
428 0.0186915887850467
429 0.0186480186480186
430 0.0186046511627907
431 0.0185614849187935
432 0.0185185185185185
433 0.0184757505773672
434 0.0184331797235023
435 0.0183908045977011
436 0.018348623853211
437 0.0183066361556064
438 0.0182648401826484
439 0.0182232346241458
440 0.0181818181818182
441 0.018140589569161
442 0.0180995475113122
443 0.018058690744921
444 0.0202702702702703
445 0.0224719101123595
446 0.0224215246636771
447 0.0223713646532438
448 0.0223214285714286
449 0.022271714922049
450 0.0222222222222222
451 0.024390243902439
452 0.0243362831858407
453 0.0242825607064018
454 0.0242290748898678
455 0.0241758241758242
456 0.0241228070175439
457 0.0240700218818381
458 0.0240174672489083
459 0.0239651416122004
460 0.0239130434782609
461 0.0238611713665944
462 0.0238095238095238
463 0.0237580993520518
464 0.0237068965517241
465 0.0236559139784946
466 0.0236051502145923
467 0.0235546038543897
468 0.0235042735042735
469 0.023454157782516
470 0.0234042553191489
471 0.0233545647558386
472 0.0233050847457627
473 0.0232558139534884
474 0.0232067510548523
475 0.0231578947368421
476 0.023109243697479
477 0.0230607966457023
478 0.0230125523012552
479 0.022964509394572
480 0.0229166666666667
481 0.0228690228690229
482 0.0228215767634855
483 0.0227743271221532
484 0.0227272727272727
485 0.022680412371134
486 0.0226337448559671
487 0.0225872689938398
488 0.0225409836065574
489 0.0224948875255624
490 0.0224489795918367
491 0.0224032586558045
492 0.0223577235772358
493 0.0223123732251521
494 0.0222672064777328
495 0.0222222222222222
496 0.0221774193548387
497 0.0241448692152917
498 0.0240963855421687
499 0.0240480961923848
500 0.024
501 0.0239520958083832
502 0.0239043824701195
503 0.0238568588469185
504 0.0238095238095238
505 0.0237623762376238
506 0.0237154150197628
507 0.0236686390532544
508 0.0236220472440945
509 0.0235756385068762
510 0.0235294117647059
511 0.0234833659491194
512 0.0234375
513 0.0233918128654971
514 0.0233463035019455
515 0.0233009708737864
516 0.0232558139534884
517 0.02321083172147
518 0.0231660231660232
519 0.023121387283237
520 0.0230769230769231
521 0.0230326295585413
522 0.0229885057471264
523 0.0229445506692161
524 0.0229007633587786
525 0.0228571428571429
526 0.0228136882129278
527 0.0227703984819734
528 0.0227272727272727
529 0.0226843100189036
530 0.0226415094339623
531 0.0225988700564972
532 0.0225563909774436
533 0.0225140712945591
534 0.0224719101123595
535 0.0224299065420561
536 0.0223880597014925
537 0.0223463687150838
538 0.0223048327137546
539 0.0222634508348794
540 0.0222222222222222
541 0.022181146025878
542 0.022140221402214
543 0.0220994475138122
544 0.0220588235294118
545 0.0220183486238532
546 0.021978021978022
547 0.0219378427787934
548 0.0218978102189781
549 0.0218579234972678
550 0.0218181818181818
551 0.0217785843920145
552 0.0217391304347826
553 0.0216998191681736
554 0.0216606498194946
555 0.0216216216216216
556 0.0215827338129496
557 0.0215439856373429
558 0.021505376344086
559 0.0214669051878354
560 0.0214285714285714
561 0.0213903743315508
562 0.0213523131672598
563 0.0213143872113677
564 0.0212765957446809
565 0.0212389380530973
566 0.0212014134275618
567 0.0211640211640212
568 0.0211267605633803
569 0.0210896309314587
570 0.0210526315789474
571 0.021015761821366
572 0.0227272727272727
573 0.0226876090750436
574 0.0226480836236934
575 0.0226086956521739
576 0.0225694444444444
577 0.0225303292894281
578 0.0224913494809689
579 0.0224525043177893
580 0.0224137931034483
581 0.0223752151462995
582 0.0223367697594502
583 0.0222984562607204
584 0.0222602739726027
585 0.0222222222222222
586 0.0221843003412969
587 0.0221465076660988
588 0.022108843537415
589 0.0220713073005093
590 0.0220338983050847
591 0.0219966159052453
592 0.0219594594594595
593 0.0219224283305228
594 0.0218855218855219
595 0.0218487394957983
596 0.0218120805369128
597 0.0217755443886097
598 0.0217391304347826
599 0.0217028380634391
600 0.0216666666666667
601 0.021630615640599
602 0.0215946843853821
603 0.021558872305141
604 0.021523178807947
605 0.0214876033057851
606 0.0214521452145215
607 0.0214168039538715
608 0.0213815789473684
609 0.0213464696223317
610 0.0213114754098361
611 0.0212765957446809
612 0.0212418300653595
613 0.0212071778140294
614 0.0211726384364821
615 0.0211382113821138
616 0.0211038961038961
617 0.0210696920583468
618 0.0210355987055016
619 0.0210016155088853
620 0.0209677419354839
621 0.0209339774557166
622 0.0209003215434084
623 0.0224719101123595
624 0.0224358974358974
625 0.0224
626 0.0223642172523962
627 0.0223285486443381
628 0.0222929936305732
629 0.0222575516693164
630 0.0222222222222222
631 0.0221870047543582
632 0.0221518987341772
633 0.0221169036334913
634 0.0236593059936909
635 0.0236220472440945
636 0.0235849056603774
637 0.0235478806907378
638 0.0235109717868339
639 0.0234741784037559
640 0.0234375
641 0.0234009360374415
642 0.0233644859813084
643 0.0233281493001555
644 0.0232919254658385
645 0.0232558139534884
646 0.0232198142414861
647 0.0231839258114374
648 0.0231481481481481
649 0.0231124807395994
650 0.0230769230769231
651 0.0230414746543779
652 0.0230061349693252
653 0.0229709035222052
654 0.0229357798165138
655 0.0229007633587786
656 0.0228658536585366
657 0.0228310502283105
658 0.0227963525835866
659 0.0227617602427921
660 0.0227272727272727
661 0.0226928895612708
662 0.0241691842900302
663 0.024132730015083
664 0.0240963855421687
665 0.0240601503759399
666 0.024024024024024
667 0.0239880059970015
668 0.0239520958083832
669 0.0239162929745889
670 0.0238805970149254
671 0.0238450074515648
672 0.0238095238095238
673 0.0237741456166419
674 0.0237388724035608
675 0.0237037037037037
676 0.0236686390532544
677 0.0236336779911374
678 0.023598820058997
679 0.0235640648011782
680 0.0235294117647059
681 0.0234948604992658
682 0.0234604105571848
683 0.0234260614934114
684 0.0233918128654971
685 0.0233576642335766
686 0.0233236151603499
687 0.0232896652110626
688 0.0232558139534884
689 0.02322206095791
690 0.0246376811594203
691 0.0246020260492041
692 0.0245664739884393
693 0.0245310245310245
694 0.0244956772334294
695 0.0244604316546763
696 0.0244252873563218
697 0.024390243902439
698 0.0243553008595989
699 0.0243204577968526
700 0.0242857142857143
701 0.0242510699001427
702 0.0242165242165242
703 0.0241820768136558
704 0.0241477272727273
705 0.024113475177305
706 0.0240793201133144
707 0.024045261669024
708 0.0240112994350282
709 0.0239774330042313
710 0.023943661971831
711 0.0239099859353024
712 0.023876404494382
713 0.0238429172510519
714 0.0238095238095238
715 0.0237762237762238
716 0.0237430167597765
717 0.0237099023709902
718 0.0236768802228412
719 0.023643949930459
720 0.0236111111111111
721 0.0235783633841886
722 0.0235457063711911
723 0.0235131396957123
724 0.0234806629834254
725 0.023448275862069
726 0.0234159779614325
727 0.0233837689133425
728 0.0233516483516484
729 0.0233196159122085
730 0.0232876712328767
731 0.0232558139534884
732 0.0245901639344262
733 0.0245566166439291
734 0.0245231607629428
735 0.0244897959183673
736 0.0244565217391304
737 0.0244233378561737
738 0.024390243902439
739 0.0243572395128552
740 0.0243243243243243
741 0.0242914979757085
742 0.0242587601078167
743 0.0242261103633917
744 0.0241935483870968
745 0.0241610738255034
746 0.0241286863270777
747 0.0240963855421687
748 0.0240641711229947
749 0.0240320427236315
750 0.0253333333333333
751 0.0252996005326232
752 0.0252659574468085
753 0.0252324037184595
754 0.0251989389920424
755 0.0251655629139073
756 0.0251322751322751
757 0.0250990752972259
758 0.025065963060686
759 0.0250329380764163
760 0.025
761 0.0249671484888305
762 0.0249343832020997
763 0.0249017038007864
764 0.024869109947644
765 0.0248366013071895
766 0.0248041775456919
767 0.0247718383311604
768 0.0247395833333333
769 0.0247074122236671
770 0.0246753246753247
771 0.0246433203631647
772 0.0246113989637306
773 0.0245795601552393
774 0.0245478036175711
775 0.0245161290322581
776 0.0244845360824742
777 0.0244530244530245
778 0.0244215938303342
779 0.024390243902439
780 0.0243589743589744
781 0.0243277848911652
782 0.0242966751918159
783 0.0242656449553001
784 0.024234693877551
785 0.024203821656051
786 0.0241730279898219
787 0.0241423125794155
788 0.0241116751269036
789 0.0240811153358682
790 0.0240506329113924
791 0.0240202275600506
792 0.023989898989899
793 0.0239596469104666
794 0.0239294710327456
795 0.0238993710691824
796 0.0238693467336683
797 0.0238393977415307
798 0.0238095238095238
799 0.0237797246558198
800 0.02375
801 0.0237203495630462
802 0.0236907730673317
803 0.0236612702366127
804 0.0236318407960199
805 0.0236024844720497
806 0.0235732009925558
807 0.023543990086741
808 0.0235148514851485
809 0.0234857849196539
810 0.0234567901234568
811 0.0234278668310728
812 0.0233990147783251
813 0.023370233702337
814 0.0233415233415233
815 0.0233128834355828
816 0.0232843137254902
817 0.0232558139534884
818 0.0232273838630807
819 0.0231990231990232
820 0.0231707317073171
821 0.023142509135201
822 0.0231143552311436
823 0.023086269744836
824 0.0242718446601942
825 0.0242424242424242
826 0.0242130750605327
827 0.0241837968561064
828 0.0241545893719807
829 0.0241254523522316
830 0.0240963855421687
831 0.0240673886883273
832 0.0240384615384615
833 0.0240096038415366
834 0.0239808153477218
835 0.0239520958083832
836 0.0239234449760766
837 0.025089605734767
838 0.0250596658711217
839 0.0250297973778308
840 0.025
841 0.0249702734839477
842 0.0249406175771971
843 0.0260972716488731
844 0.0260663507109005
845 0.0260355029585799
846 0.0260047281323877
847 0.025974025974026
848 0.0259433962264151
849 0.0259128386336867
850 0.0258823529411765
851 0.0258519388954172
852 0.0258215962441315
853 0.0257913247362251
854 0.0257611241217799
855 0.0257309941520468
856 0.0257009345794393
857 0.0256709451575263
858 0.0256410256410256
859 0.0256111757857974
860 0.0255813953488372
861 0.0255516840882695
862 0.0255220417633411
863 0.0254924681344148
864 0.025462962962963
865 0.0254335260115607
866 0.0254041570438799
867 0.0253748558246828
868 0.0253456221198157
869 0.0253164556962025
870 0.0252873563218391
871 0.0252583237657865
872 0.0252293577981651
873 0.0252004581901489
874 0.0251716247139588
875 0.0251428571428571
876 0.0251141552511416
877 0.0250855188141391
878 0.0250569476082005
879 0.025028441410694
880 0.025
881 0.0249716231555051
882 0.0260770975056689
883 0.0260475651189128
884 0.0260180995475113
885 0.0259887005649717
886 0.0259593679458239
887 0.0259301014656144
888 0.0259009009009009
889 0.0258717660292463
890 0.0258426966292135
891 0.0258136924803591
892 0.0257847533632287
893 0.0257558790593505
894 0.0257270693512304
895 0.0256983240223464
896 0.0256696428571429
897 0.0256410256410256
898 0.0256124721603563
899 0.0255839822024472
900 0.0255555555555556
901 0.025527192008879
902 0.0254988913525499
903 0.0254706533776301
904 0.0254424778761062
905 0.025414364640884
906 0.0253863134657837
907 0.0253583241455347
908 0.0253303964757709
909 0.0253025302530253
910 0.0263736263736264
911 0.026344676180022
912 0.0263157894736842
913 0.0262869660460022
914 0.0262582056892779
915 0.0262295081967213
916 0.0262008733624454
917 0.0261723009814613
918 0.0261437908496732
919 0.0261153427638738
920 0.0260869565217391
921 0.0260586319218241
922 0.0260303687635575
923 0.0260021668472373
924 0.025974025974026
925 0.0259459459459459
926 0.0259179265658747
927 0.0258899676375405
928 0.0258620689655172
929 0.0258342303552207
930 0.0258064516129032
931 0.0257787325456498
932 0.0257510729613734
933 0.0257234726688103
934 0.0256959314775161
935 0.025668449197861
936 0.0256410256410256
937 0.0256136606189968
938 0.0255863539445629
939 0.0255591054313099
940 0.025531914893617
941 0.0255047821466525
942 0.0254777070063694
943 0.0254506892895016
944 0.0254237288135593
945 0.0253968253968254
946 0.025369978858351
947 0.0253431890179514
948 0.0253164556962025
949 0.0252897787144362
950 0.0252631578947368
951 0.0252365930599369
952 0.0252100840336134
953 0.0251836306400839
954 0.0251572327044025
955 0.025130890052356
956 0.0251046025104603
957 0.0250783699059561
958 0.0250521920668058
959 0.0250260688216893
960 0.025
961 0.0249739854318418
962 0.0249480249480249
963 0.0249221183800623
964 0.024896265560166
965 0.0248704663212435
966 0.0248447204968944
967 0.0258531540847983
968 0.0258264462809917
969 0.0257997936016512
970 0.0257731958762887
971 0.0257466529351184
972 0.0257201646090535
973 0.025693730729702
974 0.0256673511293634
975 0.0256410256410256
976 0.0256147540983607
977 0.0255885363357216
978 0.0255623721881391
979 0.0255362614913177
980 0.0255102040816327
981 0.0254841997961264
982 0.0254582484725051
983 0.0254323499491353
984 0.0254065040650406
985 0.0253807106598985
986 0.0253549695740365
987 0.0253292806484296
988 0.0253036437246964
989 0.0252780586450961
990 0.0252525252525253
991 0.0252270433905146
992 0.0252016129032258
993 0.0251762336354481
994 0.0251509054325956
995 0.0251256281407035
996 0.0251004016064257
997 0.0250752256770311
998 0.0250501002004008
999 0.025025025025025
1000 0.025
};
\addplot [semithick, color0, dashed]
table {%
1 0.027027027027027
2 0.027027027027027
3 0.027027027027027
4 0.027027027027027
5 0.027027027027027
6 0.027027027027027
7 0.027027027027027
8 0.027027027027027
9 0.027027027027027
10 0.027027027027027
11 0.027027027027027
12 0.027027027027027
13 0.027027027027027
14 0.027027027027027
15 0.027027027027027
16 0.027027027027027
17 0.027027027027027
18 0.027027027027027
19 0.027027027027027
20 0.027027027027027
21 0.027027027027027
22 0.027027027027027
23 0.027027027027027
24 0.027027027027027
25 0.027027027027027
26 0.027027027027027
27 0.027027027027027
28 0.027027027027027
29 0.027027027027027
30 0.027027027027027
31 0.027027027027027
32 0.027027027027027
33 0.027027027027027
34 0.027027027027027
35 0.027027027027027
36 0.027027027027027
37 0.027027027027027
38 0.027027027027027
39 0.027027027027027
40 0.027027027027027
41 0.027027027027027
42 0.027027027027027
43 0.027027027027027
44 0.027027027027027
45 0.027027027027027
46 0.027027027027027
47 0.027027027027027
48 0.027027027027027
49 0.027027027027027
50 0.027027027027027
51 0.027027027027027
52 0.027027027027027
53 0.027027027027027
54 0.027027027027027
55 0.027027027027027
56 0.027027027027027
57 0.027027027027027
58 0.027027027027027
59 0.027027027027027
60 0.027027027027027
61 0.027027027027027
62 0.027027027027027
63 0.027027027027027
64 0.027027027027027
65 0.027027027027027
66 0.027027027027027
67 0.027027027027027
68 0.027027027027027
69 0.027027027027027
70 0.027027027027027
71 0.027027027027027
72 0.027027027027027
73 0.027027027027027
74 0.027027027027027
75 0.027027027027027
76 0.027027027027027
77 0.027027027027027
78 0.027027027027027
79 0.027027027027027
80 0.027027027027027
81 0.027027027027027
82 0.027027027027027
83 0.027027027027027
84 0.027027027027027
85 0.027027027027027
86 0.027027027027027
87 0.027027027027027
88 0.027027027027027
89 0.027027027027027
90 0.027027027027027
91 0.027027027027027
92 0.027027027027027
93 0.027027027027027
94 0.027027027027027
95 0.027027027027027
96 0.027027027027027
97 0.027027027027027
98 0.027027027027027
99 0.027027027027027
100 0.027027027027027
101 0.027027027027027
102 0.027027027027027
103 0.027027027027027
104 0.027027027027027
105 0.027027027027027
106 0.027027027027027
107 0.027027027027027
108 0.027027027027027
109 0.027027027027027
110 0.027027027027027
111 0.027027027027027
112 0.027027027027027
113 0.027027027027027
114 0.027027027027027
115 0.027027027027027
116 0.027027027027027
117 0.027027027027027
118 0.027027027027027
119 0.027027027027027
120 0.027027027027027
121 0.027027027027027
122 0.027027027027027
123 0.027027027027027
124 0.027027027027027
125 0.027027027027027
126 0.027027027027027
127 0.027027027027027
128 0.027027027027027
129 0.027027027027027
130 0.027027027027027
131 0.027027027027027
132 0.027027027027027
133 0.027027027027027
134 0.027027027027027
135 0.027027027027027
136 0.027027027027027
137 0.027027027027027
138 0.027027027027027
139 0.027027027027027
140 0.027027027027027
141 0.027027027027027
142 0.027027027027027
143 0.027027027027027
144 0.027027027027027
145 0.027027027027027
146 0.027027027027027
147 0.027027027027027
148 0.027027027027027
149 0.027027027027027
150 0.027027027027027
151 0.027027027027027
152 0.027027027027027
153 0.027027027027027
154 0.027027027027027
155 0.027027027027027
156 0.027027027027027
157 0.027027027027027
158 0.027027027027027
159 0.027027027027027
160 0.027027027027027
161 0.027027027027027
162 0.027027027027027
163 0.027027027027027
164 0.027027027027027
165 0.027027027027027
166 0.027027027027027
167 0.027027027027027
168 0.027027027027027
169 0.027027027027027
170 0.027027027027027
171 0.027027027027027
172 0.027027027027027
173 0.027027027027027
174 0.027027027027027
175 0.027027027027027
176 0.027027027027027
177 0.027027027027027
178 0.027027027027027
179 0.027027027027027
180 0.027027027027027
181 0.027027027027027
182 0.027027027027027
183 0.027027027027027
184 0.027027027027027
185 0.027027027027027
186 0.027027027027027
187 0.027027027027027
188 0.027027027027027
189 0.027027027027027
190 0.027027027027027
191 0.027027027027027
192 0.027027027027027
193 0.027027027027027
194 0.027027027027027
195 0.027027027027027
196 0.027027027027027
197 0.027027027027027
198 0.027027027027027
199 0.027027027027027
200 0.027027027027027
201 0.027027027027027
202 0.027027027027027
203 0.027027027027027
204 0.027027027027027
205 0.027027027027027
206 0.027027027027027
207 0.027027027027027
208 0.027027027027027
209 0.027027027027027
210 0.027027027027027
211 0.027027027027027
212 0.027027027027027
213 0.027027027027027
214 0.027027027027027
215 0.027027027027027
216 0.027027027027027
217 0.027027027027027
218 0.027027027027027
219 0.027027027027027
220 0.027027027027027
221 0.027027027027027
222 0.027027027027027
223 0.027027027027027
224 0.027027027027027
225 0.027027027027027
226 0.027027027027027
227 0.027027027027027
228 0.027027027027027
229 0.027027027027027
230 0.027027027027027
231 0.027027027027027
232 0.027027027027027
233 0.027027027027027
234 0.027027027027027
235 0.027027027027027
236 0.027027027027027
237 0.027027027027027
238 0.027027027027027
239 0.027027027027027
240 0.027027027027027
241 0.027027027027027
242 0.027027027027027
243 0.027027027027027
244 0.027027027027027
245 0.027027027027027
246 0.027027027027027
247 0.027027027027027
248 0.027027027027027
249 0.027027027027027
250 0.027027027027027
251 0.027027027027027
252 0.027027027027027
253 0.027027027027027
254 0.027027027027027
255 0.027027027027027
256 0.027027027027027
257 0.027027027027027
258 0.027027027027027
259 0.027027027027027
260 0.027027027027027
261 0.027027027027027
262 0.027027027027027
263 0.027027027027027
264 0.027027027027027
265 0.027027027027027
266 0.027027027027027
267 0.027027027027027
268 0.027027027027027
269 0.027027027027027
270 0.027027027027027
271 0.027027027027027
272 0.027027027027027
273 0.027027027027027
274 0.027027027027027
275 0.027027027027027
276 0.027027027027027
277 0.027027027027027
278 0.027027027027027
279 0.027027027027027
280 0.027027027027027
281 0.027027027027027
282 0.027027027027027
283 0.027027027027027
284 0.027027027027027
285 0.027027027027027
286 0.027027027027027
287 0.027027027027027
288 0.027027027027027
289 0.027027027027027
290 0.027027027027027
291 0.027027027027027
292 0.027027027027027
293 0.027027027027027
294 0.027027027027027
295 0.027027027027027
296 0.027027027027027
297 0.027027027027027
298 0.027027027027027
299 0.027027027027027
300 0.027027027027027
301 0.027027027027027
302 0.027027027027027
303 0.027027027027027
304 0.027027027027027
305 0.027027027027027
306 0.027027027027027
307 0.027027027027027
308 0.027027027027027
309 0.027027027027027
310 0.027027027027027
311 0.027027027027027
312 0.027027027027027
313 0.027027027027027
314 0.027027027027027
315 0.027027027027027
316 0.027027027027027
317 0.027027027027027
318 0.027027027027027
319 0.027027027027027
320 0.027027027027027
321 0.027027027027027
322 0.027027027027027
323 0.027027027027027
324 0.027027027027027
325 0.027027027027027
326 0.027027027027027
327 0.027027027027027
328 0.027027027027027
329 0.027027027027027
330 0.027027027027027
331 0.027027027027027
332 0.027027027027027
333 0.027027027027027
334 0.027027027027027
335 0.027027027027027
336 0.027027027027027
337 0.027027027027027
338 0.027027027027027
339 0.027027027027027
340 0.027027027027027
341 0.027027027027027
342 0.027027027027027
343 0.027027027027027
344 0.027027027027027
345 0.027027027027027
346 0.027027027027027
347 0.027027027027027
348 0.027027027027027
349 0.027027027027027
350 0.027027027027027
351 0.027027027027027
352 0.027027027027027
353 0.027027027027027
354 0.027027027027027
355 0.027027027027027
356 0.027027027027027
357 0.027027027027027
358 0.027027027027027
359 0.027027027027027
360 0.027027027027027
361 0.027027027027027
362 0.027027027027027
363 0.027027027027027
364 0.027027027027027
365 0.027027027027027
366 0.027027027027027
367 0.027027027027027
368 0.027027027027027
369 0.027027027027027
370 0.027027027027027
371 0.027027027027027
372 0.027027027027027
373 0.027027027027027
374 0.027027027027027
375 0.027027027027027
376 0.027027027027027
377 0.027027027027027
378 0.027027027027027
379 0.027027027027027
380 0.027027027027027
381 0.027027027027027
382 0.027027027027027
383 0.027027027027027
384 0.027027027027027
385 0.027027027027027
386 0.027027027027027
387 0.027027027027027
388 0.027027027027027
389 0.027027027027027
390 0.027027027027027
391 0.027027027027027
392 0.027027027027027
393 0.027027027027027
394 0.027027027027027
395 0.027027027027027
396 0.027027027027027
397 0.027027027027027
398 0.027027027027027
399 0.027027027027027
400 0.027027027027027
401 0.027027027027027
402 0.027027027027027
403 0.027027027027027
404 0.027027027027027
405 0.027027027027027
406 0.027027027027027
407 0.027027027027027
408 0.027027027027027
409 0.027027027027027
410 0.027027027027027
411 0.027027027027027
412 0.027027027027027
413 0.027027027027027
414 0.027027027027027
415 0.027027027027027
416 0.027027027027027
417 0.027027027027027
418 0.027027027027027
419 0.027027027027027
420 0.027027027027027
421 0.027027027027027
422 0.027027027027027
423 0.027027027027027
424 0.027027027027027
425 0.027027027027027
426 0.027027027027027
427 0.027027027027027
428 0.027027027027027
429 0.027027027027027
430 0.027027027027027
431 0.027027027027027
432 0.027027027027027
433 0.027027027027027
434 0.027027027027027
435 0.027027027027027
436 0.027027027027027
437 0.027027027027027
438 0.027027027027027
439 0.027027027027027
440 0.027027027027027
441 0.027027027027027
442 0.027027027027027
443 0.027027027027027
444 0.027027027027027
445 0.027027027027027
446 0.027027027027027
447 0.027027027027027
448 0.027027027027027
449 0.027027027027027
450 0.027027027027027
451 0.027027027027027
452 0.027027027027027
453 0.027027027027027
454 0.027027027027027
455 0.027027027027027
456 0.027027027027027
457 0.027027027027027
458 0.027027027027027
459 0.027027027027027
460 0.027027027027027
461 0.027027027027027
462 0.027027027027027
463 0.027027027027027
464 0.027027027027027
465 0.027027027027027
466 0.027027027027027
467 0.027027027027027
468 0.027027027027027
469 0.027027027027027
470 0.027027027027027
471 0.027027027027027
472 0.027027027027027
473 0.027027027027027
474 0.027027027027027
475 0.027027027027027
476 0.027027027027027
477 0.027027027027027
478 0.027027027027027
479 0.027027027027027
480 0.027027027027027
481 0.027027027027027
482 0.027027027027027
483 0.027027027027027
484 0.027027027027027
485 0.027027027027027
486 0.027027027027027
487 0.027027027027027
488 0.027027027027027
489 0.027027027027027
490 0.027027027027027
491 0.027027027027027
492 0.027027027027027
493 0.027027027027027
494 0.027027027027027
495 0.027027027027027
496 0.027027027027027
497 0.027027027027027
498 0.027027027027027
499 0.027027027027027
500 0.027027027027027
501 0.027027027027027
502 0.027027027027027
503 0.027027027027027
504 0.027027027027027
505 0.027027027027027
506 0.027027027027027
507 0.027027027027027
508 0.027027027027027
509 0.027027027027027
510 0.027027027027027
511 0.027027027027027
512 0.027027027027027
513 0.027027027027027
514 0.027027027027027
515 0.027027027027027
516 0.027027027027027
517 0.027027027027027
518 0.027027027027027
519 0.027027027027027
520 0.027027027027027
521 0.027027027027027
522 0.027027027027027
523 0.027027027027027
524 0.027027027027027
525 0.027027027027027
526 0.027027027027027
527 0.027027027027027
528 0.027027027027027
529 0.027027027027027
530 0.027027027027027
531 0.027027027027027
532 0.027027027027027
533 0.027027027027027
534 0.027027027027027
535 0.027027027027027
536 0.027027027027027
537 0.027027027027027
538 0.027027027027027
539 0.027027027027027
540 0.027027027027027
541 0.027027027027027
542 0.027027027027027
543 0.027027027027027
544 0.027027027027027
545 0.027027027027027
546 0.027027027027027
547 0.027027027027027
548 0.027027027027027
549 0.027027027027027
550 0.027027027027027
551 0.027027027027027
552 0.027027027027027
553 0.027027027027027
554 0.027027027027027
555 0.027027027027027
556 0.027027027027027
557 0.027027027027027
558 0.027027027027027
559 0.027027027027027
560 0.027027027027027
561 0.027027027027027
562 0.027027027027027
563 0.027027027027027
564 0.027027027027027
565 0.027027027027027
566 0.027027027027027
567 0.027027027027027
568 0.027027027027027
569 0.027027027027027
570 0.027027027027027
571 0.027027027027027
572 0.027027027027027
573 0.027027027027027
574 0.027027027027027
575 0.027027027027027
576 0.027027027027027
577 0.027027027027027
578 0.027027027027027
579 0.027027027027027
580 0.027027027027027
581 0.027027027027027
582 0.027027027027027
583 0.027027027027027
584 0.027027027027027
585 0.027027027027027
586 0.027027027027027
587 0.027027027027027
588 0.027027027027027
589 0.027027027027027
590 0.027027027027027
591 0.027027027027027
592 0.027027027027027
593 0.027027027027027
594 0.027027027027027
595 0.027027027027027
596 0.027027027027027
597 0.027027027027027
598 0.027027027027027
599 0.027027027027027
600 0.027027027027027
601 0.027027027027027
602 0.027027027027027
603 0.027027027027027
604 0.027027027027027
605 0.027027027027027
606 0.027027027027027
607 0.027027027027027
608 0.027027027027027
609 0.027027027027027
610 0.027027027027027
611 0.027027027027027
612 0.027027027027027
613 0.027027027027027
614 0.027027027027027
615 0.027027027027027
616 0.027027027027027
617 0.027027027027027
618 0.027027027027027
619 0.027027027027027
620 0.027027027027027
621 0.027027027027027
622 0.027027027027027
623 0.027027027027027
624 0.027027027027027
625 0.027027027027027
626 0.027027027027027
627 0.027027027027027
628 0.027027027027027
629 0.027027027027027
630 0.027027027027027
631 0.027027027027027
632 0.027027027027027
633 0.027027027027027
634 0.027027027027027
635 0.027027027027027
636 0.027027027027027
637 0.027027027027027
638 0.027027027027027
639 0.027027027027027
640 0.027027027027027
641 0.027027027027027
642 0.027027027027027
643 0.027027027027027
644 0.027027027027027
645 0.027027027027027
646 0.027027027027027
647 0.027027027027027
648 0.027027027027027
649 0.027027027027027
650 0.027027027027027
651 0.027027027027027
652 0.027027027027027
653 0.027027027027027
654 0.027027027027027
655 0.027027027027027
656 0.027027027027027
657 0.027027027027027
658 0.027027027027027
659 0.027027027027027
660 0.027027027027027
661 0.027027027027027
662 0.027027027027027
663 0.027027027027027
664 0.027027027027027
665 0.027027027027027
666 0.027027027027027
667 0.027027027027027
668 0.027027027027027
669 0.027027027027027
670 0.027027027027027
671 0.027027027027027
672 0.027027027027027
673 0.027027027027027
674 0.027027027027027
675 0.027027027027027
676 0.027027027027027
677 0.027027027027027
678 0.027027027027027
679 0.027027027027027
680 0.027027027027027
681 0.027027027027027
682 0.027027027027027
683 0.027027027027027
684 0.027027027027027
685 0.027027027027027
686 0.027027027027027
687 0.027027027027027
688 0.027027027027027
689 0.027027027027027
690 0.027027027027027
691 0.027027027027027
692 0.027027027027027
693 0.027027027027027
694 0.027027027027027
695 0.027027027027027
696 0.027027027027027
697 0.027027027027027
698 0.027027027027027
699 0.027027027027027
700 0.027027027027027
701 0.027027027027027
702 0.027027027027027
703 0.027027027027027
704 0.027027027027027
705 0.027027027027027
706 0.027027027027027
707 0.027027027027027
708 0.027027027027027
709 0.027027027027027
710 0.027027027027027
711 0.027027027027027
712 0.027027027027027
713 0.027027027027027
714 0.027027027027027
715 0.027027027027027
716 0.027027027027027
717 0.027027027027027
718 0.027027027027027
719 0.027027027027027
720 0.027027027027027
721 0.027027027027027
722 0.027027027027027
723 0.027027027027027
724 0.027027027027027
725 0.027027027027027
726 0.027027027027027
727 0.027027027027027
728 0.027027027027027
729 0.027027027027027
730 0.027027027027027
731 0.027027027027027
732 0.027027027027027
733 0.027027027027027
734 0.027027027027027
735 0.027027027027027
736 0.027027027027027
737 0.027027027027027
738 0.027027027027027
739 0.027027027027027
740 0.027027027027027
741 0.027027027027027
742 0.027027027027027
743 0.027027027027027
744 0.027027027027027
745 0.027027027027027
746 0.027027027027027
747 0.027027027027027
748 0.027027027027027
749 0.027027027027027
750 0.027027027027027
751 0.027027027027027
752 0.027027027027027
753 0.027027027027027
754 0.027027027027027
755 0.027027027027027
756 0.027027027027027
757 0.027027027027027
758 0.027027027027027
759 0.027027027027027
760 0.027027027027027
761 0.027027027027027
762 0.027027027027027
763 0.027027027027027
764 0.027027027027027
765 0.027027027027027
766 0.027027027027027
767 0.027027027027027
768 0.027027027027027
769 0.027027027027027
770 0.027027027027027
771 0.027027027027027
772 0.027027027027027
773 0.027027027027027
774 0.027027027027027
775 0.027027027027027
776 0.027027027027027
777 0.027027027027027
778 0.027027027027027
779 0.027027027027027
780 0.027027027027027
781 0.027027027027027
782 0.027027027027027
783 0.027027027027027
784 0.027027027027027
785 0.027027027027027
786 0.027027027027027
787 0.027027027027027
788 0.027027027027027
789 0.027027027027027
790 0.027027027027027
791 0.027027027027027
792 0.027027027027027
793 0.027027027027027
794 0.027027027027027
795 0.027027027027027
796 0.027027027027027
797 0.027027027027027
798 0.027027027027027
799 0.027027027027027
800 0.027027027027027
801 0.027027027027027
802 0.027027027027027
803 0.027027027027027
804 0.027027027027027
805 0.027027027027027
806 0.027027027027027
807 0.027027027027027
808 0.027027027027027
809 0.027027027027027
810 0.027027027027027
811 0.027027027027027
812 0.027027027027027
813 0.027027027027027
814 0.027027027027027
815 0.027027027027027
816 0.027027027027027
817 0.027027027027027
818 0.027027027027027
819 0.027027027027027
820 0.027027027027027
821 0.027027027027027
822 0.027027027027027
823 0.027027027027027
824 0.027027027027027
825 0.027027027027027
826 0.027027027027027
827 0.027027027027027
828 0.027027027027027
829 0.027027027027027
830 0.027027027027027
831 0.027027027027027
832 0.027027027027027
833 0.027027027027027
834 0.027027027027027
835 0.027027027027027
836 0.027027027027027
837 0.027027027027027
838 0.027027027027027
839 0.027027027027027
840 0.027027027027027
841 0.027027027027027
842 0.027027027027027
843 0.027027027027027
844 0.027027027027027
845 0.027027027027027
846 0.027027027027027
847 0.027027027027027
848 0.027027027027027
849 0.027027027027027
850 0.027027027027027
851 0.027027027027027
852 0.027027027027027
853 0.027027027027027
854 0.027027027027027
855 0.027027027027027
856 0.027027027027027
857 0.027027027027027
858 0.027027027027027
859 0.027027027027027
860 0.027027027027027
861 0.027027027027027
862 0.027027027027027
863 0.027027027027027
864 0.027027027027027
865 0.027027027027027
866 0.027027027027027
867 0.027027027027027
868 0.027027027027027
869 0.027027027027027
870 0.027027027027027
871 0.027027027027027
872 0.027027027027027
873 0.027027027027027
874 0.027027027027027
875 0.027027027027027
876 0.027027027027027
877 0.027027027027027
878 0.027027027027027
879 0.027027027027027
880 0.027027027027027
881 0.027027027027027
882 0.027027027027027
883 0.027027027027027
884 0.027027027027027
885 0.027027027027027
886 0.027027027027027
887 0.027027027027027
888 0.027027027027027
889 0.027027027027027
890 0.027027027027027
891 0.027027027027027
892 0.027027027027027
893 0.027027027027027
894 0.027027027027027
895 0.027027027027027
896 0.027027027027027
897 0.027027027027027
898 0.027027027027027
899 0.027027027027027
900 0.027027027027027
901 0.027027027027027
902 0.027027027027027
903 0.027027027027027
904 0.027027027027027
905 0.027027027027027
906 0.027027027027027
907 0.027027027027027
908 0.027027027027027
909 0.027027027027027
910 0.027027027027027
911 0.027027027027027
912 0.027027027027027
913 0.027027027027027
914 0.027027027027027
915 0.027027027027027
916 0.027027027027027
917 0.027027027027027
918 0.027027027027027
919 0.027027027027027
920 0.027027027027027
921 0.027027027027027
922 0.027027027027027
923 0.027027027027027
924 0.027027027027027
925 0.027027027027027
926 0.027027027027027
927 0.027027027027027
928 0.027027027027027
929 0.027027027027027
930 0.027027027027027
931 0.027027027027027
932 0.027027027027027
933 0.027027027027027
934 0.027027027027027
935 0.027027027027027
936 0.027027027027027
937 0.027027027027027
938 0.027027027027027
939 0.027027027027027
940 0.027027027027027
941 0.027027027027027
942 0.027027027027027
943 0.027027027027027
944 0.027027027027027
945 0.027027027027027
946 0.027027027027027
947 0.027027027027027
948 0.027027027027027
949 0.027027027027027
950 0.027027027027027
951 0.027027027027027
952 0.027027027027027
953 0.027027027027027
954 0.027027027027027
955 0.027027027027027
956 0.027027027027027
957 0.027027027027027
958 0.027027027027027
959 0.027027027027027
960 0.027027027027027
961 0.027027027027027
962 0.027027027027027
963 0.027027027027027
964 0.027027027027027
965 0.027027027027027
966 0.027027027027027
967 0.027027027027027
968 0.027027027027027
969 0.027027027027027
970 0.027027027027027
971 0.027027027027027
972 0.027027027027027
973 0.027027027027027
974 0.027027027027027
975 0.027027027027027
976 0.027027027027027
977 0.027027027027027
978 0.027027027027027
979 0.027027027027027
980 0.027027027027027
981 0.027027027027027
982 0.027027027027027
983 0.027027027027027
984 0.027027027027027
985 0.027027027027027
986 0.027027027027027
987 0.027027027027027
988 0.027027027027027
989 0.027027027027027
990 0.027027027027027
991 0.027027027027027
992 0.027027027027027
993 0.027027027027027
994 0.027027027027027
995 0.027027027027027
996 0.027027027027027
997 0.027027027027027
998 0.027027027027027
999 0.027027027027027
1000 0.027027027027027
};
\addlegendentry{$f_{r_{e}}$ (frecuencia relativa esperada de $18$)}
\end{axis}

\end{tikzpicture}

    \caption{frecuencia relativa para 10 corridas del experimento}
  \end{mytikzresize}
\end{figure}

\begin{figure}[!htbp]
  \begin{mytikzresize}{0.6\textwidth}
    \centering
    % This file was created by tikzplotlib v0.9.1.
\begin{tikzpicture}

\definecolor{color0}{rgb}{0.12156862745098,0.466666666666667,0.705882352941177}
\definecolor{color1}{rgb}{1,0.498039215686275,0.0549019607843137}
\definecolor{color2}{rgb}{0.172549019607843,0.627450980392157,0.172549019607843}
\definecolor{color3}{rgb}{0.83921568627451,0.152941176470588,0.156862745098039}
\definecolor{color4}{rgb}{0.580392156862745,0.403921568627451,0.741176470588235}
\definecolor{color5}{rgb}{0.549019607843137,0.337254901960784,0.294117647058824}
\definecolor{color6}{rgb}{0.890196078431372,0.466666666666667,0.76078431372549}
\definecolor{color7}{rgb}{0.737254901960784,0.741176470588235,0.133333333333333}
\definecolor{color8}{rgb}{0.0901960784313725,0.745098039215686,0.811764705882353}

\begin{axis}[
legend cell align={left},
legend style={fill opacity=0.5, draw opacity=1, text opacity=1, draw=white!80!black},
scaled ticks=false,
tick align=outside,
tick pos=left,
width=\figW,
x grid style={white!69.0196078431373!black},
xlabel={\(\displaystyle n\) (número de tiradas)},
xmajorgrids,
xmin=-48.95, xmax=1049.95,
xtick style={color=black},
xticklabel style={/pgf/number format/.cd,fixed,precision=2},
y grid style={white!69.0196078431373!black},
ylabel={\(\displaystyle v_{p}\) (valor promedio)},
ymajorgrids,
ymin=-0.7, ymax=36.7,
ytick style={color=black},
yticklabel style={/pgf/number format/.cd,fixed,precision=2}
]
\addplot [semithick, color0, forget plot]
table {%
1 33
2 19
3 23.6666666666667
4 26.5
5 24.2
6 21.5
7 21.4285714285714
8 22.5
9 23.8888888888889
10 25.1
11 25.7272727272727
12 25.8333333333333
13 25.6153846153846
14 25.7857142857143
15 25.0666666666667
16 25.5
17 24.0588235294118
18 24.5
19 23.2105263157895
20 23.8
21 22.6666666666667
22 22.2727272727273
23 22.5652173913043
24 22.875
25 22.12
26 22.3846153846154
27 22
28 21.3571428571429
29 21.8275862068966
30 21.1333333333333
31 21.5161290322581
32 21.375
33 20.8484848484848
34 20.8529411764706
35 21.0285714285714
36 20.8055555555556
37 20.4864864864865
38 20.1052631578947
39 20.0769230769231
40 20.05
41 19.9024390243902
42 20.2142857142857
43 19.7441860465116
44 20
45 20.2222222222222
46 20.3913043478261
47 20.3191489361702
48 20.3958333333333
49 20.0204081632653
50 19.74
51 19.8039215686275
52 19.7307692307692
53 19.377358490566
54 19.3148148148148
55 19.4
56 19.5
57 19.7017543859649
58 19.6724137931034
59 19.5084745762712
60 19.7833333333333
61 19.7704918032787
62 19.5645161290323
63 19.3015873015873
64 19.03125
65 19.0923076923077
66 19.0757575757576
67 19.044776119403
68 18.8676470588235
69 18.6086956521739
70 18.3428571428571
71 18.169014084507
72 18.4166666666667
73 18.2191780821918
74 18.3108108108108
75 18.4533333333333
76 18.6447368421053
77 18.5584415584416
78 18.7051282051282
79 18.6962025316456
80 18.875
81 18.7283950617284
82 18.8658536585366
83 18.7228915662651
84 18.75
85 18.5529411764706
86 18.4418604651163
87 18.5632183908046
88 18.4090909090909
89 18.3707865168539
90 18.5333333333333
91 18.6703296703297
92 18.7065217391304
93 18.752688172043
94 18.6489361702128
95 18.5052631578947
96 18.6875
97 18.6701030927835
98 18.6326530612245
99 18.7979797979798
100 18.67
101 18.6732673267327
102 18.5490196078431
103 18.6699029126214
104 18.7019230769231
105 18.8190476190476
106 18.8679245283019
107 18.9626168224299
108 18.8518518518519
109 18.9633027522936
110 18.9272727272727
111 18.972972972973
112 19.0982142857143
113 19.0088495575221
114 18.8421052631579
115 18.9304347826087
116 18.9051724137931
117 18.7521367521368
118 18.8983050847458
119 18.9495798319328
120 19.0583333333333
121 19.0247933884298
122 18.9590163934426
123 18.9430894308943
124 18.7983870967742
125 18.688
126 18.8015873015873
127 18.7007874015748
128 18.71875
129 18.6589147286822
130 18.6307692307692
131 18.5114503816794
132 18.6060606060606
133 18.7218045112782
134 18.6044776119403
135 18.5925925925926
136 18.5735294117647
137 18.5620437956204
138 18.6521739130435
139 18.589928057554
140 18.4714285714286
141 18.468085106383
142 18.5352112676056
143 18.4685314685315
144 18.5763888888889
145 18.5931034482759
146 18.6164383561644
147 18.6802721088435
148 18.7837837837838
149 18.8053691275168
150 18.74
151 18.6158940397351
152 18.6315789473684
153 18.5098039215686
154 18.4545454545455
155 18.4129032258065
156 18.3205128205128
157 18.3630573248408
158 18.4430379746835
159 18.5471698113208
160 18.625
161 18.6459627329193
162 18.7160493827161
163 18.7484662576687
164 18.7134146341463
165 18.8121212121212
166 18.7048192771084
167 18.6287425149701
168 18.6607142857143
169 18.7396449704142
170 18.8
171 18.7836257309942
172 18.7732558139535
173 18.7341040462428
174 18.8103448275862
175 18.7142857142857
176 18.6477272727273
177 18.6949152542373
178 18.7247191011236
179 18.6927374301676
180 18.7888888888889
181 18.8342541436464
182 18.7362637362637
183 18.7267759562842
184 18.7880434782609
185 18.7675675675676
186 18.6720430107527
187 18.620320855615
188 18.6382978723404
189 18.7089947089947
190 18.7789473684211
191 18.8534031413613
192 18.828125
193 18.880829015544
194 18.7989690721649
195 18.8461538461538
196 18.9183673469388
197 18.9187817258883
198 18.8282828282828
199 18.8743718592965
200 18.935
201 19.0099502487562
202 18.9455445544554
203 18.9310344827586
204 19.0049019607843
205 19
206 19.0485436893204
207 19.0531400966184
208 19.0096153846154
209 19.0526315789474
210 19.0571428571429
211 19.0331753554502
212 19.0566037735849
213 19.0469483568075
214 18.9953271028037
215 18.9953488372093
216 19.0462962962963
217 18.9769585253456
218 18.954128440367
219 18.9543378995434
220 19.0227272727273
221 18.9864253393665
222 19.036036036036
223 18.9775784753363
224 18.9107142857143
225 18.9555555555556
226 18.9159292035398
227 18.8898678414097
228 18.9166666666667
229 18.9694323144105
230 18.9826086956522
231 19.04329004329
232 19
233 19.0557939914163
234 19.0128205128205
235 19.0297872340426
236 18.9576271186441
237 18.9071729957806
238 18.844537815126
239 18.8493723849372
240 18.8208333333333
241 18.7427385892116
242 18.7066115702479
243 18.7530864197531
244 18.7418032786885
245 18.7877551020408
246 18.7357723577236
247 18.6963562753036
248 18.7137096774194
249 18.7550200803213
250 18.808
251 18.7729083665339
252 18.797619047619
253 18.8181818181818
254 18.8307086614173
255 18.8705882352941
256 18.921875
257 18.8482490272374
258 18.8333333333333
259 18.8455598455598
260 18.8846153846154
261 18.8390804597701
262 18.9007633587786
263 18.893536121673
264 18.8977272727273
265 18.9094339622641
266 18.9398496240602
267 18.9213483146067
268 18.8917910447761
269 18.8847583643123
270 18.8814814814815
271 18.8634686346863
272 18.9007352941176
273 18.8534798534799
274 18.8321167883212
275 18.7709090909091
276 18.7065217391304
277 18.7111913357401
278 18.6438848920863
279 18.6917562724014
280 18.625
281 18.5800711743772
282 18.5673758865248
283 18.6113074204947
284 18.5669014084507
285 18.5087719298246
286 18.541958041958
287 18.5679442508711
288 18.5069444444444
289 18.4602076124567
290 18.5206896551724
291 18.4776632302405
292 18.4212328767123
293 18.4402730375427
294 18.4863945578231
295 18.4542372881356
296 18.3918918918919
297 18.3434343434343
298 18.3355704697987
299 18.3311036789298
300 18.3666666666667
301 18.3488372093023
302 18.3609271523179
303 18.3663366336634
304 18.3322368421053
305 18.3770491803279
306 18.4150326797386
307 18.3973941368078
308 18.4415584415584
309 18.3948220064725
310 18.3548387096774
311 18.3890675241158
312 18.3685897435897
313 18.3226837060703
314 18.328025477707
315 18.2920634920635
316 18.2721518987342
317 18.2996845425867
318 18.3490566037736
319 18.4012539184953
320 18.45
321 18.398753894081
322 18.4254658385093
323 18.4365325077399
324 18.4845679012346
325 18.48
326 18.4233128834356
327 18.4250764525994
328 18.4329268292683
329 18.4589665653495
330 18.4787878787879
331 18.4743202416918
332 18.4246987951807
333 18.4534534534535
334 18.502994011976
335 18.5283582089552
336 18.5654761904762
337 18.5964391691395
338 18.6449704142012
339 18.622418879056
340 18.5705882352941
341 18.524926686217
342 18.5233918128655
343 18.5247813411079
344 18.5406976744186
345 18.5913043478261
346 18.5375722543353
347 18.5389048991354
348 18.4885057471264
349 18.4383954154728
350 18.4342857142857
351 18.4700854700855
352 18.46875
353 18.4249291784703
354 18.4519774011299
355 18.4704225352113
356 18.4522471910112
357 18.406162464986
358 18.3966480446927
359 18.3509749303621
360 18.3694444444444
361 18.3961218836565
362 18.4254143646409
363 18.4297520661157
364 18.4752747252747
365 18.4712328767123
366 18.4371584699454
367 18.4359673024523
368 18.3885869565217
369 18.4227642276423
370 18.3756756756757
371 18.3477088948787
372 18.2983870967742
373 18.2761394101877
374 18.3155080213904
375 18.3173333333333
376 18.3457446808511
377 18.3580901856764
378 18.3439153439153
379 18.3060686015831
380 18.3421052631579
381 18.3070866141732
382 18.3350785340314
383 18.3812010443864
384 18.3958333333333
385 18.3532467532468
386 18.3082901554404
387 18.328165374677
388 18.3453608247423
389 18.3624678663239
390 18.3641025641026
391 18.3887468030691
392 18.4260204081633
393 18.4681933842239
394 18.4543147208122
395 18.4658227848101
396 18.5050505050505
397 18.4911838790932
398 18.4497487437186
399 18.468671679198
400 18.4625
401 18.4214463840399
402 18.3756218905473
403 18.3622828784119
404 18.3415841584158
405 18.3061728395062
406 18.3448275862069
407 18.3857493857494
408 18.4240196078431
409 18.4498777506112
410 18.4658536585366
411 18.4549878345499
412 18.4757281553398
413 18.4382566585956
414 18.4613526570048
415 18.4313253012048
416 18.40625
417 18.3717026378897
418 18.3468899521531
419 18.346062052506
420 18.3595238095238
421 18.3254156769596
422 18.3483412322275
423 18.3498817966903
424 18.3490566037736
425 18.3694117647059
426 18.3286384976526
427 18.3583138173302
428 18.3995327102804
429 18.3986013986014
430 18.4116279069767
431 18.445475638051
432 18.4861111111111
433 18.4480369515012
434 18.4654377880184
435 18.4758620689655
436 18.4977064220184
437 18.5102974828375
438 18.486301369863
439 18.4601366742597
440 18.4590909090909
441 18.4852607709751
442 18.4524886877828
443 18.451467268623
444 18.4864864864865
445 18.4516853932584
446 18.4439461883408
447 18.4765100671141
448 18.4709821428571
449 18.43429844098
450 18.3955555555556
451 18.4057649667406
452 18.3893805309735
453 18.3973509933775
454 18.4074889867841
455 18.3824175824176
456 18.3618421052632
457 18.3304157549234
458 18.3624454148472
459 18.3877995642702
460 18.3739130434783
461 18.3492407809111
462 18.3506493506494
463 18.3563714902808
464 18.364224137931
465 18.3311827956989
466 18.3347639484979
467 18.3447537473233
468 18.3482905982906
469 18.3347547974414
470 18.3255319148936
471 18.3163481953291
472 18.3072033898305
473 18.3086680761099
474 18.3291139240506
475 18.3242105263158
476 18.3172268907563
477 18.3333333333333
478 18.3556485355649
479 18.39248434238
480 18.4041666666667
481 18.4033264033264
482 18.3838174273859
483 18.3809523809524
484 18.3760330578512
485 18.3896907216495
486 18.40329218107
487 18.4127310061602
488 18.3893442622951
489 18.4212678936605
490 18.4061224489796
491 18.4052953156823
492 18.4308943089431
493 18.4320486815416
494 18.4008097165992
495 18.3919191919192
496 18.4173387096774
497 18.3843058350101
498 18.3855421686747
499 18.3667334669339
500 18.356
501 18.3832335329341
502 18.4123505976096
503 18.441351888668
504 18.4484126984127
505 18.4653465346535
506 18.4347826086957
507 18.4575936883629
508 18.4212598425197
509 18.4204322200393
510 18.4019607843137
511 18.4363992172211
512 18.41015625
513 18.4171539961014
514 18.4416342412451
515 18.4485436893204
516 18.4806201550388
517 18.5087040618955
518 18.5135135135135
519 18.4971098265896
520 18.5173076923077
521 18.5143953934741
522 18.5
523 18.4933078393881
524 18.5267175572519
525 18.527619047619
526 18.5152091254753
527 18.5313092979127
528 18.530303030303
529 18.5595463137996
530 18.5735849056604
531 18.5951035781544
532 18.5883458646617
533 18.5703564727955
534 18.5430711610487
535 18.5140186915888
536 18.5391791044776
537 18.5586592178771
538 18.5724907063197
539 18.595547309833
540 18.6259259259259
541 18.6192236598891
542 18.6217712177122
543 18.6187845303867
544 18.6011029411765
545 18.5981651376147
546 18.5732600732601
547 18.5575868372943
548 18.529197080292
549 18.4972677595628
550 18.4672727272727
551 18.4519056261343
552 18.4438405797101
553 18.4394213381555
554 18.4620938628159
555 18.4756756756757
556 18.4928057553957
557 18.5206463195691
558 18.5340501792115
559 18.5420393559928
560 18.5446428571429
561 18.5383244206774
562 18.5338078291815
563 18.5488454706927
564 18.5780141843972
565 18.553982300885
566 18.5742049469965
567 18.5502645502646
568 18.5316901408451
569 18.5008787346221
570 18.4719298245614
571 18.4658493870403
572 18.4615384615385
573 18.4432809773124
574 18.411149825784
575 18.4417391304348
576 18.4548611111111
577 18.4644714038128
578 18.439446366782
579 18.4697754749568
580 18.4862068965517
581 18.5060240963855
582 18.4879725085911
583 18.4957118353345
584 18.5222602739726
585 18.5435897435897
586 18.5443686006826
587 18.557069846678
588 18.547619047619
589 18.5398981324278
590 18.564406779661
591 18.5719120135364
592 18.5523648648649
593 18.5497470489039
594 18.5622895622896
595 18.5764705882353
596 18.5855704697987
597 18.6097152428811
598 18.5986622073579
599 18.5709515859766
600 18.5983333333333
601 18.5740432612313
602 18.5548172757475
603 18.5555555555556
604 18.5298013245033
605 18.5553719008264
606 18.5610561056106
607 18.5535420098847
608 18.5378289473684
609 18.5467980295567
610 18.5229508196721
611 18.5417348608838
612 18.5702614379085
613 18.5970636215334
614 18.586319218241
615 18.5869918699187
616 18.5941558441558
617 18.6029173419773
618 18.6181229773463
619 18.6187399030695
620 18.6290322580645
621 18.6505636070853
622 18.6414790996785
623 18.6356340288925
624 18.6634615384615
625 18.6624
626 18.6485623003195
627 18.6283891547049
628 18.6082802547771
629 18.6168521462639
630 18.6063492063492
631 18.5927099841521
632 18.5949367088608
633 18.5750394944708
634 18.5567823343849
635 18.5322834645669
636 18.5581761006289
637 18.5384615384615
638 18.5282131661442
639 18.5430359937402
640 18.5265625
641 18.5179407176287
642 18.5451713395639
643 18.5287713841369
644 18.527950310559
645 18.522480620155
646 18.5030959752322
647 18.5177743431221
648 18.5339506172839
649 18.5608628659476
650 18.5707692307692
651 18.5975422427035
652 18.5889570552147
653 18.5803981623277
654 18.5718654434251
655 18.5587786259542
656 18.5564024390244
657 18.5738203957382
658 18.5577507598784
659 18.5629742033384
660 18.55
661 18.5461422087746
662 18.5725075528701
663 18.5927601809955
664 18.6039156626506
665 18.6210526315789
666 18.6381381381381
667 18.6371814092954
668 18.6212574850299
669 18.6143497757848
670 18.6149253731343
671 18.5991058122206
672 18.6026785714286
673 18.6092124814265
674 18.5979228486647
675 18.597037037037
676 18.5695266272189
677 18.5480059084195
678 18.5221238938053
679 18.5346097201767
680 18.5132352941176
681 18.5051395007342
682 18.5234604105572
683 18.5051244509517
684 18.4985380116959
685 18.5138686131387
686 18.5262390670554
687 18.5065502183406
688 18.5145348837209
689 18.5268505079826
690 18.5478260869565
691 18.5600578871201
692 18.5404624277457
693 18.5425685425685
694 18.5403458213256
695 18.5352517985612
696 18.5244252873563
697 18.5164992826399
698 18.4957020057307
699 18.5064377682403
700 18.51
701 18.5335235378031
702 18.5128205128205
703 18.4950213371266
704 18.5
705 18.4936170212766
706 18.4957507082153
707 18.4893917963225
708 18.4915254237288
709 18.490832157969
710 18.4915492957746
711 18.5105485232068
712 18.5
713 18.5133239831697
714 18.5196078431373
715 18.5160839160839
716 18.5111731843575
717 18.5355648535565
718 18.5292479108635
719 18.547983310153
720 18.5708333333333
721 18.5492371705964
722 18.5734072022161
723 18.5532503457815
724 18.5345303867403
725 18.5268965517241
726 18.5440771349862
727 18.5378266850069
728 18.5260989010989
729 18.5089163237311
730 18.5150684931507
731 18.5075239398085
732 18.5081967213115
733 18.5034106412005
734 18.5013623978202
735 18.5034013605442
736 18.4972826086957
737 18.4776119402985
738 18.4620596205962
739 18.4803788903924
740 18.4675675675676
741 18.4777327935223
742 18.4636118598383
743 18.4401076716016
744 18.4596774193548
745 18.4778523489933
746 18.4973190348525
747 18.4832663989291
748 18.5
749 18.5233644859813
750 18.5253333333333
751 18.5019973368842
752 18.5066489361702
753 18.5073041168659
754 18.4840848806366
755 18.4609271523179
756 18.4391534391534
757 18.4332892998679
758 18.4248021108179
759 18.4281949934124
760 18.4197368421053
761 18.4047306176084
762 18.3818897637795
763 18.3997378768021
764 18.3926701570681
765 18.4078431372549
766 18.4177545691906
767 18.4289439374185
768 18.43359375
769 18.4161248374512
770 18.3935064935065
771 18.4098573281453
772 18.4132124352332
773 18.4359637774903
774 18.4586563307494
775 18.4812903225806
776 18.4948453608247
777 18.5006435006435
778 18.5205655526992
779 18.5160462130937
780 18.5115384615385
781 18.5326504481434
782 18.5191815856777
783 18.4955300127714
784 18.4808673469388
785 18.4815286624204
786 18.4580152671756
787 18.4548919949174
788 18.4441624365482
789 18.4423320659062
790 18.4582278481013
791 18.4551201011378
792 18.4444444444444
793 18.4287515762926
794 18.4420654911839
795 18.4327044025157
796 18.4158291457286
797 18.4065244667503
798 18.4022556390977
799 18.4180225281602
800 18.4
801 18.3895131086142
802 18.3852867830424
803 18.3860523038605
804 18.384328358209
805 18.3614906832298
806 18.3796526054591
807 18.3767038413879
808 18.3762376237624
809 18.3918417799753
810 18.3913580246914
811 18.3760789149199
812 18.3546798029557
813 18.3542435424354
814 18.3648648648649
815 18.3865030674847
816 18.4031862745098
817 18.4014687882497
818 18.3973105134474
819 18.4114774114774
820 18.4121951219512
821 18.4250913520097
822 18.4379562043796
823 18.4313487241798
824 18.4368932038835
825 18.4436363636364
826 18.455205811138
827 18.4425634824667
828 18.4408212560386
829 18.4463208685163
830 18.4566265060241
831 18.4657039711191
832 18.4831730769231
833 18.4861944777911
834 18.484412470024
835 18.4634730538922
836 18.4461722488038
837 18.4647550776583
838 18.4677804295943
839 18.4874851013111
840 18.502380952381
841 18.4910820451843
842 18.4762470308789
843 18.4768683274021
844 18.4976303317536
845 18.5076923076923
846 18.4988179669031
847 18.5053128689492
848 18.5094339622642
849 18.5135453474676
850 18.5176470588235
851 18.4994124559342
852 18.4906103286385
853 18.5087924970692
854 18.5222482435597
855 18.5146198830409
856 18.4976635514019
857 18.5157526254376
858 18.5314685314685
859 18.5261932479627
860 18.5453488372093
861 18.5447154471545
862 18.5406032482599
863 18.5596755504056
864 18.5601851851852
865 18.5526011560694
866 18.5450346420323
867 18.5340253748558
868 18.536866359447
869 18.5466052934407
870 18.5505747126437
871 18.5579793340987
872 18.5584862385321
873 18.5498281786942
874 18.558352402746
875 18.5485714285714
876 18.5296803652968
877 18.5381984036488
878 18.5261958997722
879 18.5392491467577
880 18.5340909090909
881 18.5255391600454
882 18.5351473922902
883 18.5424688561721
884 18.5542986425339
885 18.554802259887
886 18.538374717833
887 18.5231116121759
888 18.5225225225225
889 18.5129358830146
890 18.5101123595506
891 18.5230078563412
892 18.5179372197309
893 18.498320268757
894 18.4809843400447
895 18.4849162011173
896 18.4654017857143
897 18.4526198439242
898 18.4487750556793
899 18.4527252502781
900 18.4666666666667
901 18.4506104328524
902 18.4634146341463
903 18.4828349944629
904 18.4922566371681
905 18.5038674033149
906 18.4900662251656
907 18.4895259095921
908 18.4702643171806
909 18.4895489548955
910 18.4835164835165
911 18.4950603732162
912 18.5087719298246
913 18.5279299014239
914 18.5164113785558
915 18.5234972677596
916 18.5240174672489
917 18.5081788440567
918 18.520697167756
919 18.5277475516866
920 18.5391304347826
921 18.5526601520087
922 18.556399132321
923 18.5698808234019
924 18.5649350649351
925 18.5827027027027
926 18.5885529157667
927 18.5706580366775
928 18.5646551724138
929 18.5715823466093
930 18.5817204301075
931 18.5639097744361
932 18.5643776824034
933 18.561629153269
934 18.5642398286938
935 18.579679144385
936 18.5683760683761
937 18.5602988260406
938 18.5607675906183
939 18.5580404685836
940 18.5478723404255
941 18.5366631243358
942 18.552016985138
943 18.5641569459173
944 18.5646186440678
945 18.5746031746032
946 18.5877378435518
947 18.5797254487856
948 18.5886075949367
949 18.5890410958904
950 18.5747368421053
951 18.589905362776
952 18.6081932773109
953 18.601259181532
954 18.6006289308176
955 18.6031413612565
956 18.5836820083682
957 18.564263322884
958 18.5782881002088
959 18.5933263816476
960 18.5916666666667
961 18.5889698231009
962 18.5945945945946
963 18.5939771547248
964 18.606846473029
965 18.6103626943005
966 18.6055900621118
967 18.5977249224405
968 18.5981404958678
969 18.5892672858617
970 18.6030927835052
971 18.6076210092688
972 18.5895061728395
973 18.5991778006167
974 18.5862422997947
975 18.5805128205128
976 18.5881147540984
977 18.5926305015353
978 18.5869120654397
979 18.5720122574055
980 18.5857142857143
981 18.5779816513761
982 18.5600814663951
983 18.5462868769074
984 18.5609756097561
985 18.5532994923858
986 18.5405679513185
987 18.5339412360689
988 18.5253036437247
989 18.5136501516684
990 18.5080808080808
991 18.5095862764884
992 18.5181451612903
993 18.5095669687815
994 18.5160965794769
995 18.5256281407035
996 18.5160642570281
997 18.4994984954865
998 18.5130260521042
999 18.5005005005005
1000 18.489
};
\addplot [semithick, color1, forget plot]
table {%
1 13
2 13
3 19
4 22.5
5 22.4
6 20.3333333333333
7 17.5714285714286
8 17.5
9 18.6666666666667
10 20
11 21.3636363636364
12 22.4166666666667
13 21.0769230769231
14 21.9285714285714
15 20.9333333333333
16 21.625
17 21.5294117647059
18 20.6111111111111
19 19.5263157894737
20 20.15
21 19.6190476190476
22 18.8636363636364
23 19.5217391304348
24 20.0416666666667
25 19.4
26 19.8076923076923
27 20.2592592592593
28 20.2857142857143
29 20.6551724137931
30 20.8666666666667
31 20.7096774193548
32 20.46875
33 20.0606060606061
34 19.8823529411765
35 19.5428571428571
36 19.5833333333333
37 19.3513513513514
38 19.4210526315789
39 19.3076923076923
40 19.55
41 19.4390243902439
42 19.1666666666667
43 18.7209302325581
44 19.0454545454545
45 19.1555555555556
46 19.1739130434783
47 18.7872340425532
48 18.8333333333333
49 18.5918367346939
50 18.44
51 18.4901960784314
52 18.2115384615385
53 18.377358490566
54 18.5740740740741
55 18.4545454545455
56 18.3035714285714
57 18.1228070175439
58 17.948275862069
59 17.9491525423729
60 17.7666666666667
61 17.6393442622951
62 17.4032258064516
63 17.3174603174603
64 17.4375
65 17.6923076923077
66 17.6060606060606
67 17.4179104477612
68 17.3823529411765
69 17.5072463768116
70 17.5285714285714
71 17.5774647887324
72 17.3472222222222
73 17.3013698630137
74 17.3243243243243
75 17.1066666666667
76 17.1842105263158
77 17.1298701298701
78 16.9487179487179
79 16.8987341772152
80 17.05
81 17.0493827160494
82 17.1585365853659
83 17.2289156626506
84 17.3095238095238
85 17.4470588235294
86 17.3139534883721
87 17.1149425287356
88 17.125
89 17.1460674157303
90 17.1
91 17.021978021978
92 16.9021739130435
93 16.8279569892473
94 16.7446808510638
95 16.8
96 16.96875
97 16.8041237113402
98 16.8163265306122
99 16.9191919191919
100 16.96
101 17.0792079207921
102 17.0392156862745
103 17.0194174757282
104 16.9903846153846
105 16.9428571428571
106 16.9056603773585
107 16.9065420560748
108 17.0555555555556
109 16.9724770642202
110 16.9454545454545
111 17.009009009009
112 17.1339285714286
113 17.1592920353982
114 17.0350877192982
115 16.9565217391304
116 16.9568965517241
117 17.025641025641
118 17
119 16.9915966386555
120 16.9833333333333
121 17.0495867768595
122 16.9180327868852
123 17.0650406504065
124 16.9354838709677
125 16.952
126 16.8333333333333
127 16.7244094488189
128 16.71875
129 16.8062015503876
130 16.8846153846154
131 16.7862595419847
132 16.8484848484848
133 16.8045112781955
134 16.6791044776119
135 16.6888888888889
136 16.8235294117647
137 16.8029197080292
138 16.7391304347826
139 16.8057553956835
140 16.8785714285714
141 16.886524822695
142 16.9718309859155
143 16.8881118881119
144 17
145 17.1103448275862
146 17.2260273972603
147 17.2244897959184
148 17.2567567567568
149 17.3355704697987
150 17.3466666666667
151 17.3708609271523
152 17.4736842105263
153 17.3725490196078
154 17.4350649350649
155 17.3548387096774
156 17.2948717948718
157 17.4140127388535
158 17.5316455696203
159 17.6415094339623
160 17.65625
161 17.5962732919255
162 17.7098765432099
163 17.6134969325153
164 17.5121951219512
165 17.5212121212121
166 17.5843373493976
167 17.622754491018
168 17.6785714285714
169 17.7633136094675
170 17.8352941176471
171 17.7368421052632
172 17.656976744186
173 17.5780346820809
174 17.5804597701149
175 17.52
176 17.6022727272727
177 17.6497175141243
178 17.7022471910112
179 17.7653631284916
180 17.7222222222222
181 17.7734806629834
182 17.7747252747253
183 17.7267759562842
184 17.7554347826087
185 17.7243243243243
186 17.8225806451613
187 17.8449197860963
188 17.8829787234043
189 17.9206349206349
190 17.9157894736842
191 17.9842931937173
192 17.9166666666667
193 18.0103626943005
194 17.9484536082474
195 17.9641025641026
196 17.9489795918367
197 17.8781725888325
198 17.8030303030303
199 17.7638190954774
200 17.785
201 17.7512437810945
202 17.8267326732673
203 17.7635467980296
204 17.7254901960784
205 17.7414634146341
206 17.7669902912621
207 17.7149758454106
208 17.6298076923077
209 17.5789473684211
210 17.5190476190476
211 17.4549763033175
212 17.4292452830189
213 17.3708920187793
214 17.3224299065421
215 17.3255813953488
216 17.3703703703704
217 17.4423963133641
218 17.3899082568807
219 17.4018264840183
220 17.3772727272727
221 17.3348416289593
222 17.3558558558559
223 17.3318385650224
224 17.2544642857143
225 17.2
226 17.2743362831858
227 17.215859030837
228 17.1929824561403
229 17.1397379912664
230 17.1086956521739
231 17.1168831168831
232 17.198275862069
233 17.1287553648069
234 17.0726495726496
235 17.1148936170213
236 17.1949152542373
237 17.2067510548523
238 17.2058823529412
239 17.2426778242678
240 17.275
241 17.253112033195
242 17.3181818181818
243 17.3786008230453
244 17.3729508196721
245 17.3224489795918
246 17.369918699187
247 17.4089068825911
248 17.3991935483871
249 17.3493975903614
250 17.36
251 17.3107569721116
252 17.3412698412698
253 17.2727272727273
254 17.3385826771654
255 17.3921568627451
256 17.35546875
257 17.3190661478599
258 17.3914728682171
259 17.4594594594595
260 17.4576923076923
261 17.4750957854406
262 17.4580152671756
263 17.4600760456274
264 17.3939393939394
265 17.4264150943396
266 17.4699248120301
267 17.4569288389513
268 17.4067164179104
269 17.3977695167286
270 17.3740740740741
271 17.3616236162362
272 17.4264705882353
273 17.4542124542125
274 17.4489051094891
275 17.4
276 17.4275362318841
277 17.4548736462094
278 17.4784172661871
279 17.426523297491
280 17.4464285714286
281 17.3879003558719
282 17.4042553191489
283 17.4028268551237
284 17.4401408450704
285 17.4350877192982
286 17.4545454545455
287 17.4947735191638
288 17.5347222222222
289 17.5397923875433
290 17.5310344827586
291 17.5601374570447
292 17.5719178082192
293 17.5529010238908
294 17.5646258503401
295 17.5762711864407
296 17.6385135135135
297 17.6228956228956
298 17.6442953020134
299 17.6354515050167
300 17.6033333333333
301 17.5913621262458
302 17.6026490066225
303 17.6006600660066
304 17.5493421052632
305 17.5213114754098
306 17.4640522875817
307 17.4234527687296
308 17.4805194805195
309 17.4724919093851
310 17.4677419354839
311 17.4244372990354
312 17.4807692307692
313 17.5335463258786
314 17.5031847133758
315 17.5365079365079
316 17.5221518987342
317 17.5362776025237
318 17.4968553459119
319 17.4451410658307
320 17.484375
321 17.4984423676012
322 17.4968944099379
323 17.5356037151703
324 17.5061728395062
325 17.4615384615385
326 17.4877300613497
327 17.4464831804281
328 17.4329268292683
329 17.4042553191489
330 17.4242424242424
331 17.3867069486405
332 17.3674698795181
333 17.3843843843844
334 17.3502994011976
335 17.3791044776119
336 17.3958333333333
337 17.3501483679525
338 17.3579881656805
339 17.3569321533923
340 17.3911764705882
341 17.4105571847507
342 17.374269005848
343 17.399416909621
344 17.3488372093023
345 17.368115942029
346 17.4017341040462
347 17.4380403458213
348 17.4051724137931
349 17.4040114613181
350 17.4171428571429
351 17.4245014245014
352 17.4261363636364
353 17.4730878186969
354 17.4632768361582
355 17.4507042253521
356 17.435393258427
357 17.4453781512605
358 17.4664804469274
359 17.5097493036212
360 17.4916666666667
361 17.5069252077562
362 17.5303867403315
363 17.5261707988981
364 17.4917582417582
365 17.4958904109589
366 17.4863387978142
367 17.4850136239782
368 17.5217391304348
369 17.550135501355
370 17.5081081081081
371 17.5525606469003
372 17.5698924731183
373 17.6166219839142
374 17.6283422459893
375 17.6533333333333
376 17.6356382978723
377 17.5915119363395
378 17.6137566137566
379 17.622691292876
380 17.5921052631579
381 17.6351706036745
382 17.6780104712042
383 17.6527415143603
384 17.671875
385 17.7142857142857
386 17.7227979274611
387 17.7674418604651
388 17.7242268041237
389 17.7712082262211
390 17.7461538461538
391 17.7237851662404
392 17.6989795918367
393 17.6692111959288
394 17.6954314720812
395 17.6759493670886
396 17.7020202020202
397 17.7329974811083
398 17.7010050251256
399 17.7167919799499
400 17.71
401 17.7531172069825
402 17.7636815920398
403 17.7915632754342
404 17.8019801980198
405 17.841975308642
406 17.8546798029557
407 17.8845208845209
408 17.8529411764706
409 17.8924205378973
410 17.9195121951219
411 17.963503649635
412 17.9757281553398
413 17.9588377723971
414 17.9468599033816
415 17.9397590361446
416 17.9567307692308
417 17.9424460431655
418 17.9210526315789
419 17.9427207637232
420 17.947619047619
421 17.9904988123515
422 18.0071090047393
423 18.016548463357
424 18.0117924528302
425 17.9717647058824
426 18.0140845070423
427 17.9742388758782
428 17.9369158878505
429 17.9044289044289
430 17.9232558139535
431 17.9419953596288
432 17.900462962963
433 17.8614318706697
434 17.8940092165899
435 17.9057471264368
436 17.8967889908257
437 17.8649885583524
438 17.8949771689498
439 17.9316628701595
440 17.9522727272727
441 17.9546485260771
442 17.9841628959276
443 18
444 17.9842342342342
445 17.9977528089888
446 18.02466367713
447 17.9865771812081
448 18.0022321428571
449 18.0423162583519
450 18.0155555555556
451 18.0443458980044
452 18.0176991150442
453 18.0022075055188
454 18.0088105726872
455 18.0197802197802
456 18.015350877193
457 18.0459518599562
458 18.0829694323144
459 18.1045751633987
460 18.1
461 18.119305856833
462 18.1017316017316
463 18.1058315334773
464 18.1056034482759
465 18.0838709677419
466 18.068669527897
467 18.0963597430407
468 18.0576923076923
469 18.0277185501066
470 18.0489361702128
471 18.031847133758
472 18.0423728813559
473 18.0401691331924
474 18.0379746835443
475 18.0589473684211
476 18.0252100840336
477 18.0062893081761
478 17.9874476987448
479 17.9686847599165
480 17.95
481 17.981288981289
482 18.0124481327801
483 18.0227743271222
484 18.0247933884298
485 18.0309278350515
486 18.0288065843621
487 18.0164271047228
488 18.0450819672131
489 18.0654396728016
490 18.0367346938776
491 18.0386965376782
492 18.0630081300813
493 18.0730223123732
494 18.1052631578947
495 18.0767676767677
496 18.0866935483871
497 18.0824949698189
498 18.0481927710843
499 18.0420841683367
500 18.006
501 17.9780439121757
502 18.0019920318725
503 18.0318091451292
504 18.0496031746032
505 18.0376237623762
506 18.0256916996047
507 18.0414201183432
508 18.007874015748
509 18.0353634577603
510 18
511 17.9706457925636
512 17.94921875
513 17.9805068226121
514 17.9805447470817
515 18.0058252427184
516 17.9980620155039
517 17.9806576402321
518 17.980694980695
519 17.9961464354528
520 17.9769230769231
521 17.9884836852207
522 17.9789272030651
523 17.9674952198853
524 17.9751908396947
525 17.9885714285714
526 17.9790874524715
527 17.965844402277
528 17.9867424242424
529 17.9905482041588
530 17.9679245283019
531 17.9604519774011
532 17.9266917293233
533 17.9193245778612
534 17.8913857677903
535 17.8859813084112
536 17.8712686567164
537 17.852886405959
538 17.860594795539
539 17.8571428571429
540 17.8722222222222
541 17.8706099815157
542 17.8929889298893
543 17.8950276243094
544 17.8713235294118
545 17.8422018348624
546 17.8461538461538
547 17.872029250457
548 17.8905109489051
549 17.9216757741348
550 17.9290909090909
551 17.9074410163339
552 17.9039855072464
553 17.9077757685353
554 17.8808664259928
555 17.8720720720721
556 17.8489208633094
557 17.8366247755835
558 17.8297491039427
559 17.8228980322004
560 17.8482142857143
561 17.8217468805704
562 17.8167259786477
563 17.797513321492
564 17.7765957446809
565 17.7769911504425
566 17.7508833922261
567 17.7654320987654
568 17.7676056338028
569 17.7504393673111
570 17.759649122807
571 17.7513134851138
572 17.75
573 17.7417102966841
574 17.7578397212544
575 17.7826086956522
576 17.7864583333333
577 17.7660311958406
578 17.7629757785467
579 17.7616580310881
580 17.7551724137931
581 17.7693631669535
582 17.7594501718213
583 17.7804459691252
584 17.8082191780822
585 17.8239316239316
586 17.8412969283276
587 17.8415672913118
588 17.8469387755102
589 17.8590831918506
590 17.8474576271186
591 17.8477157360406
592 17.8783783783784
593 17.8650927487352
594 17.8720538720539
595 17.8689075630252
596 17.8674496644295
597 17.8978224455611
598 17.9230769230769
599 17.89816360601
600 17.8783333333333
601 17.8535773710483
602 17.8455149501661
603 17.8723051409619
604 17.8609271523179
605 17.8710743801653
606 17.8613861386139
607 17.8780889621087
608 17.8700657894737
609 17.8817733990148
610 17.9032786885246
611 17.9263502454992
612 17.9477124183007
613 17.9592169657423
614 17.9462540716612
615 17.9723577235772
616 18
617 18.0064829821718
618 18.0355987055016
619 18.064620355412
620 18.0516129032258
621 18.0322061191626
622 18.0192926045016
623 18.0337078651685
624 18.0320512820513
625 18.0048
626 18.0015974440895
627 17.9728867623604
628 17.9601910828025
629 17.9411764705882
630 17.9190476190476
631 17.9334389857369
632 17.9588607594937
633 17.9541864139021
634 17.9321766561514
635 17.9275590551181
636 17.9245283018868
637 17.8963893249608
638 17.8793103448276
639 17.8794992175274
640 17.875
641 17.8829953198128
642 17.8707165109034
643 17.8740279937792
644 17.888198757764
645 17.8899224806202
646 17.8839009287926
647 17.9010819165379
648 17.8873456790123
649 17.8659476117103
650 17.84
651 17.8371735791091
652 17.8251533742331
653 17.8116385911179
654 17.7951070336391
655 17.8030534351145
656 17.7759146341463
657 17.7686453576865
658 17.7507598784195
659 17.763277693475
660 17.7818181818182
661 17.8093797276853
662 17.8066465256798
663 17.8084464555053
664 17.8012048192771
665 17.8285714285714
666 17.8408408408408
667 17.8305847076462
668 17.8038922155689
669 17.8191330343797
670 17.8328358208955
671 17.8450074515648
672 17.8452380952381
673 17.8261515601783
674 17.8041543026706
675 17.8162962962963
676 17.8210059171598
677 17.8404726735598
678 17.8392330383481
679 17.8527245949926
680 17.8544117647059
681 17.8678414096916
682 17.8782991202346
683 17.8901903367496
684 17.9078947368421
685 17.9138686131387
686 17.9387755102041
687 17.9213973799127
688 17.9476744186047
689 17.9245283018868
690 17.9014492753623
691 17.9146164978292
692 17.9161849710983
693 17.9033189033189
694 17.9193083573487
695 17.936690647482
696 17.9324712643678
697 17.9196556671449
698 17.9011461318052
699 17.9127324749642
700 17.9185714285714
701 17.9343794579173
702 17.9444444444444
703 17.9445234708393
704 17.9232954545455
705 17.9205673758865
706 17.9277620396601
707 17.9052333804809
708 17.9209039548023
709 17.9421720733427
710 17.9183098591549
711 17.9395218002813
712 17.9325842696629
713 17.9438990182328
714 17.9509803921569
715 17.9706293706294
716 17.9525139664804
717 17.9274755927476
718 17.9052924791086
719 17.8929068150209
720 17.8763888888889
721 17.876560332871
722 17.8836565096953
723 17.8616874135546
724 17.8439226519337
725 17.8593103448276
726 17.8402203856749
727 17.8198074277854
728 17.8241758241758
729 17.8230452674897
730 17.8219178082192
731 17.8084815321477
732 17.8073770491803
733 17.7912687585266
734 17.8133514986376
735 17.8054421768707
736 17.820652173913
737 17.8276797829037
738 17.8048780487805
739 17.8186738836265
740 17.8418918918919
741 17.8461538461538
742 17.8369272237197
743 17.8156123822342
744 17.8319892473118
745 17.8201342281879
746 17.8257372654156
747 17.8112449799197
748 17.798128342246
749 17.8224299065421
750 17.808
751 17.8282290279627
752 17.8364361702128
753 17.8326693227092
754 17.8328912466844
755 17.8370860927152
756 17.8518518518519
757 17.8665785997358
758 17.8588390501319
759 17.8563899868248
760 17.8539473684211
761 17.862023653088
762 17.8438320209974
763 17.8361730013106
764 17.848167539267
765 17.8444444444444
766 17.8433420365535
767 17.8461538461538
768 17.8515625
769 17.8699609882965
770 17.8831168831169
771 17.8975356679637
772 17.8950777202073
773 17.9042690815006
774 17.906976744186
775 17.9045161290323
776 17.9239690721649
777 17.9060489060489
778 17.8856041131105
779 17.9037227214377
780 17.9205128205128
781 17.9359795134443
782 17.957800511509
783 17.9425287356322
784 17.9375
785 17.9452229299363
786 17.9262086513995
787 17.9237611181703
788 17.9073604060914
789 17.9252217997465
790 17.9075949367089
791 17.9026548672566
792 17.8964646464646
793 17.8953341740227
794 17.9130982367758
795 17.9295597484277
796 17.9309045226131
797 17.9473023839398
798 17.9573934837093
799 17.9674593241552
800 17.96625
801 17.956304619226
802 17.9389027431421
803 17.9439601494396
804 17.9440298507463
805 17.9341614906832
806 17.9441687344913
807 17.9306071871128
808 17.9517326732673
809 17.9653893695921
810 17.9641975308642
811 17.9728729963009
812 17.9642857142857
813 17.9717097170972
814 17.9643734643735
815 17.9472392638037
816 17.9558823529412
817 17.9583843329253
818 17.979217603912
819 17.959706959707
820 17.9731707317073
821 17.9902557856273
822 17.9902676399027
823 18.0072904009721
824 18.001213592233
825 17.9927272727273
826 17.9830508474576
827 17.9975816203144
828 17.9915458937198
829 18.0120627261761
830 18.0204819277108
831 18.0216606498195
832 18.0096153846154
833 17.9903961584634
834 17.9916067146283
835 18.0035928143713
836 17.9844497607655
837 17.9832735961768
838 17.9653937947494
839 17.9475566150179
840 17.9345238095238
841 17.9548156956005
842 17.9429928741093
843 17.9395017793594
844 17.9206161137441
845 17.901775147929
846 17.9219858156028
847 17.9055489964581
848 17.8950471698113
849 17.9081272084806
850 17.9035294117647
851 17.8895417156287
852 17.8978873239437
853 17.9144196951934
854 17.9086651053864
855 17.9111111111111
856 17.9158878504673
857 17.9346557759627
858 17.9370629370629
859 17.9406286379511
860 17.9255813953488
861 17.9291521486643
862 17.9211136890951
863 17.936268829664
864 17.931712962963
865 17.9387283236994
866 17.9376443418014
867 17.9423298731257
868 17.9377880184332
869 17.95166858458
870 17.9712643678161
871 17.9862227324914
872 17.9816513761468
873 17.9610538373425
874 17.9508009153318
875 17.9405714285714
876 17.9200913242009
877 17.9361459521095
878 17.9248291571754
879 17.9351535836177
880 17.9522727272727
881 17.9625425652667
882 17.9659863945578
883 17.9716874292186
884 17.9581447963801
885 17.9570621468927
886 17.9401805869074
887 17.9537767756483
888 17.9527027027027
889 17.955005624297
890 17.9505617977528
891 17.9349046015713
892 17.9293721973094
893 17.9305711086226
894 17.9250559284116
895 17.9418994413408
896 17.9486607142857
897 17.9620958751394
898 17.9498886414254
899 17.9332591768632
900 17.9188888888889
901 17.9245283018868
902 17.9079822616408
903 17.8970099667774
904 17.9070796460177
905 17.892817679558
906 17.8863134657837
907 17.8665931642778
908 17.8656387665198
909 17.8470847084708
910 17.8307692307692
911 17.847420417124
912 17.8585526315789
913 17.8707557502738
914 17.8851203501094
915 17.8983606557377
916 17.9028384279476
917 17.907306434024
918 17.9270152505447
919 17.9314472252448
920 17.95
921 17.9695982627579
922 17.9566160520607
923 17.9566630552546
924 17.9718614718615
925 17.9783783783784
926 17.9730021598272
927 17.9600862998921
928 17.9536637931034
929 17.9386437029064
930 17.9569892473118
931 17.937701396348
932 17.9399141630901
933 17.9239013933548
934 17.9261241970021
935 17.916577540107
936 17.9102564102564
937 17.8964781216649
938 17.8965884861407
939 17.8871139510117
940 17.9063829787234
941 17.8894792773645
942 17.9033970276008
943 17.9119830328738
944 17.9120762711864
945 17.9269841269841
946 17.9376321353066
947 17.9313621964097
948 17.9208860759494
949 17.9251844046365
950 17.9347368421053
951 17.9337539432177
952 17.9390756302521
953 17.9391395592865
954 17.9517819706499
955 17.9664921465969
956 17.9686192468619
957 17.9770114942529
958 17.9864300626305
959 17.9916579770594
960 17.9854166666667
961 18
962 18.006237006237
963 17.9885773624091
964 17.9782157676349
965 17.9699481865285
966 17.9544513457557
967 17.9513960703206
968 17.9359504132231
969 17.9236326109391
970 17.9278350515464
971 17.9309989701339
972 17.9300411522634
973 17.929085303186
974 17.9127310061602
975 17.9241025641026
976 17.9129098360656
977 17.9263050153531
978 17.9366053169734
979 17.9550561797753
980 17.9602040816327
981 17.9714576962283
982 17.9826883910387
983 17.970498474059
984 17.9684959349593
985 17.9746192893401
986 17.9878296146045
987 18.0050658561297
988 17.9908906882591
989 17.9757330637007
990 17.9808080808081
991 17.9798183652876
992 17.9637096774194
993 17.9587109768379
994 17.9738430583501
995 17.9557788944724
996 17.9658634538153
997 17.9598796389167
998 17.9418837675351
999 17.95995995996
1000 17.944
};
\addplot [semithick, color2, forget plot]
table {%
1 1
2 17
3 13
4 18
5 16.2
6 13.5
7 13.2857142857143
8 15.75
9 16.8888888888889
10 15.7
11 14.9090909090909
12 15.75
13 16.3846153846154
14 16.2857142857143
15 16.8
16 16.0625
17 16
18 15.1666666666667
19 15.8421052631579
20 15.15
21 15.6666666666667
22 16
23 15.9565217391304
24 16.125
25 15.88
26 15.7692307692308
27 15.6666666666667
28 15.8928571428571
29 15.6206896551724
30 16.2333333333333
31 16.7741935483871
32 16.3125
33 16.5151515151515
34 16.3529411764706
35 16.8285714285714
36 16.9722222222222
37 17.3243243243243
38 16.9736842105263
39 17.2820512820513
40 16.975
41 16.5853658536585
42 16.2857142857143
43 16.3720930232558
44 16.0454545454545
45 15.8666666666667
46 15.695652173913
47 16.0851063829787
48 15.8541666666667
49 16.0204081632653
50 15.78
51 15.8627450980392
52 16.0576923076923
53 15.9622641509434
54 16.2407407407407
55 16.2727272727273
56 16.2857142857143
57 16.2456140350877
58 15.9655172413793
59 16.0338983050847
60 16.2333333333333
61 16.5245901639344
62 16.5161290322581
63 16.4920634920635
64 16.453125
65 16.3846153846154
66 16.3636363636364
67 16.134328358209
68 16.3823529411765
69 16.3768115942029
70 16.5428571428571
71 16.5070422535211
72 16.2916666666667
73 16.1506849315068
74 16.2837837837838
75 16.44
76 16.4868421052632
77 16.3636363636364
78 16.1794871794872
79 16.253164556962
80 16.3625
81 16.4197530864198
82 16.6585365853659
83 16.7469879518072
84 16.75
85 16.8823529411765
86 16.6976744186047
87 16.6436781609195
88 16.5681818181818
89 16.5730337078652
90 16.6111111111111
91 16.4835164835165
92 16.5217391304348
93 16.6666666666667
94 16.8510638297872
95 16.7578947368421
96 16.65625
97 16.5257731958763
98 16.6122448979592
99 16.6666666666667
100 16.62
101 16.5643564356436
102 16.7352941176471
103 16.8252427184466
104 16.9807692307692
105 16.9047619047619
106 16.9622641509434
107 16.8878504672897
108 16.8425925925926
109 16.9082568807339
110 17.0090909090909
111 17.1621621621622
112 17.0357142857143
113 17.1238938053097
114 17.140350877193
115 17.2434782608696
116 17.2413793103448
117 17.1709401709402
118 17.1779661016949
119 17.327731092437
120 17.3083333333333
121 17.2066115702479
122 17.1311475409836
123 17.0243902439024
124 16.9193548387097
125 16.936
126 16.9603174603175
127 16.9133858267717
128 16.78125
129 16.6666666666667
130 16.7538461538462
131 16.8625954198473
132 16.9015151515152
133 16.9924812030075
134 16.9179104477612
135 16.9407407407407
136 16.9117647058824
137 16.8978102189781
138 16.963768115942
139 16.9280575539568
140 16.85
141 16.7446808510638
142 16.830985915493
143 16.7342657342657
144 16.8125
145 16.8206896551724
146 16.8698630136986
147 16.8571428571429
148 16.9527027027027
149 16.8523489932886
150 16.7933333333333
151 16.7417218543046
152 16.8486842105263
153 16.7712418300654
154 16.7012987012987
155 16.6258064516129
156 16.525641025641
157 16.5987261146497
158 16.5949367088608
159 16.6037735849057
160 16.675
161 16.6894409937888
162 16.7654320987654
163 16.7423312883436
164 16.7926829268293
165 16.7757575757576
166 16.8614457831325
167 16.8802395209581
168 16.9880952380952
169 17
170 17.0352941176471
171 17.1052631578947
172 17.1046511627907
173 17.0751445086705
174 17.051724137931
175 17.1142857142857
176 17.1534090909091
177 17.2146892655367
178 17.3089887640449
179 17.391061452514
180 17.3666666666667
181 17.4033149171271
182 17.4065934065934
183 17.448087431694
184 17.4891304347826
185 17.4054054054054
186 17.3870967741935
187 17.3048128342246
188 17.3776595744681
189 17.3015873015873
190 17.3684210526316
191 17.4188481675393
192 17.4947916666667
193 17.419689119171
194 17.3505154639175
195 17.3589743589744
196 17.4438775510204
197 17.5380710659898
198 17.6010101010101
199 17.5879396984925
200 17.52
201 17.5174129353234
202 17.5841584158416
203 17.5566502463054
204 17.5441176470588
205 17.5512195121951
206 17.5582524271845
207 17.5845410628019
208 17.5384615384615
209 17.4688995215311
210 17.552380952381
211 17.5781990521327
212 17.5047169811321
213 17.5492957746479
214 17.4719626168224
215 17.4186046511628
216 17.4166666666667
217 17.4331797235023
218 17.3761467889908
219 17.4611872146119
220 17.4590909090909
221 17.4570135746606
222 17.3918918918919
223 17.3497757847534
224 17.4196428571429
225 17.4
226 17.358407079646
227 17.4140969162996
228 17.3947368421053
229 17.3362445414847
230 17.295652173913
231 17.3593073593074
232 17.3189655172414
233 17.3948497854077
234 17.4444444444444
235 17.4468085106383
236 17.4449152542373
237 17.4388185654008
238 17.390756302521
239 17.4686192468619
240 17.4666666666667
241 17.4813278008299
242 17.5413223140496
243 17.4897119341564
244 17.5409836065574
245 17.5469387755102
246 17.5569105691057
247 17.6032388663968
248 17.5725806451613
249 17.6425702811245
250 17.712
251 17.7290836653386
252 17.797619047619
253 17.8379446640316
254 17.8425196850394
255 17.8274509803922
256 17.890625
257 17.8677042801556
258 17.9302325581395
259 17.957528957529
260 17.9961538461538
261 18.0536398467433
262 18.0343511450382
263 18.0722433460076
264 18.1022727272727
265 18.1245283018868
266 18.109022556391
267 18.0898876404494
268 18.0335820895522
269 18.0223048327138
270 17.9555555555556
271 17.929889298893
272 17.8970588235294
273 17.8681318681319
274 17.8503649635037
275 17.8618181818182
276 17.8840579710145
277 17.9494584837545
278 17.8992805755396
279 17.9390681003584
280 17.9214285714286
281 17.9715302491103
282 17.936170212766
283 17.9328621908127
284 17.9119718309859
285 17.9719298245614
286 17.9370629370629
287 18
288 18.0416666666667
289 18.0968858131488
290 18.1413793103448
291 18.1340206185567
292 18.1027397260274
293 18.0477815699659
294 18.0986394557823
295 18.0881355932203
296 18.0540540540541
297 18.003367003367
298 17.9530201342282
299 17.9765886287625
300 17.9233333333333
301 17.8770764119601
302 17.864238410596
303 17.8778877887789
304 17.8289473684211
305 17.8852459016393
306 17.8856209150327
307 17.928338762215
308 17.9090909090909
309 17.9288025889968
310 17.9548387096774
311 17.9099678456592
312 17.9455128205128
313 17.9329073482428
314 17.9203821656051
315 17.8825396825397
316 17.9177215189873
317 17.9211356466877
318 17.9056603773585
319 17.9122257053292
320 17.85625
321 17.8504672897196
322 17.8385093167702
323 17.8606811145511
324 17.9012345679012
325 17.9169230769231
326 17.8773006134969
327 17.9266055045872
328 17.890243902439
329 17.8601823708207
330 17.8757575757576
331 17.9244712990937
332 17.8825301204819
333 17.8558558558559
334 17.8203592814371
335 17.8417910447761
336 17.8392857142857
337 17.8575667655786
338 17.8698224852071
339 17.8200589970501
340 17.8558823529412
341 17.8152492668622
342 17.7631578947368
343 17.7551020408163
344 17.7645348837209
345 17.8086956521739
346 17.8583815028902
347 17.8184438040346
348 17.8362068965517
349 17.8137535816619
350 17.7657142857143
351 17.8034188034188
352 17.8068181818182
353 17.7903682719547
354 17.8022598870057
355 17.8225352112676
356 17.8398876404494
357 17.8851540616247
358 17.9134078212291
359 17.941504178273
360 17.9694444444444
361 18
362 17.9806629834254
363 18
364 18.0054945054945
365 18.0164383561644
366 18.0655737704918
367 18.0708446866485
368 18.0271739130435
369 18.0569105691057
370 18.0648648648649
371 18.0592991913747
372 18.0994623655914
373 18.1394101876676
374 18.1470588235294
375 18.1066666666667
376 18.1223404255319
377 18.1511936339523
378 18.1878306878307
379 18.1715039577836
380 18.1289473684211
381 18.1364829396325
382 18.1832460732984
383 18.1592689295039
384 18.2005208333333
385 18.1766233766234
386 18.1865284974093
387 18.1550387596899
388 18.1984536082474
389 18.1902313624679
390 18.1461538461538
391 18.1841432225064
392 18.1785714285714
393 18.1959287531807
394 18.2055837563452
395 18.2303797468354
396 18.2449494949495
397 18.2141057934509
398 18.1683417085427
399 18.1854636591479
400 18.155
401 18.1970074812968
402 18.2114427860697
403 18.2506203473945
404 18.2821782178218
405 18.2444444444444
406 18.2142857142857
407 18.2260442260442
408 18.2254901960784
409 18.2469437652812
410 18.209756097561
411 18.1873479318735
412 18.1650485436893
413 18.1404358353511
414 18.1473429951691
415 18.1325301204819
416 18.1322115384615
417 18.1750599520384
418 18.1961722488038
419 18.1837708830549
420 18.1404761904762
421 18.1472684085511
422 18.1492890995261
423 18.1536643026005
424 18.1132075471698
425 18.1411764705882
426 18.1150234741784
427 18.1334894613583
428 18.1658878504673
429 18.1841491841492
430 18.1813953488372
431 18.1902552204176
432 18.2291666666667
433 18.256351039261
434 18.2903225806452
435 18.3310344827586
436 18.348623853211
437 18.3798627002288
438 18.3630136986301
439 18.3644646924829
440 18.3340909090909
441 18.3265306122449
442 18.3371040723982
443 18.3295711060948
444 18.3445945945946
445 18.3573033707865
446 18.3744394618834
447 18.3870246085011
448 18.4084821428571
449 18.4097995545657
450 18.3888888888889
451 18.370288248337
452 18.3407079646018
453 18.3046357615894
454 18.3436123348018
455 18.3648351648352
456 18.3728070175439
457 18.3391684901532
458 18.353711790393
459 18.363834422658
460 18.3717391304348
461 18.3947939262473
462 18.3679653679654
463 18.3390928725702
464 18.3189655172414
465 18.3010752688172
466 18.3218884120172
467 18.3297644539615
468 18.3440170940171
469 18.3134328358209
470 18.3191489361702
471 18.2802547770701
472 18.3135593220339
473 18.323467230444
474 18.2932489451477
475 18.3073684210526
476 18.3046218487395
477 18.3333333333333
478 18.2991631799163
479 18.2693110647182
480 18.23125
481 18.2141372141372
482 18.1970954356846
483 18.184265010352
484 18.202479338843
485 18.2103092783505
486 18.2448559670782
487 18.2546201232033
488 18.2766393442623
489 18.2658486707566
490 18.2918367346939
491 18.2586558044807
492 18.2926829268293
493 18.290060851927
494 18.3036437246964
495 18.3353535353535
496 18.3427419354839
497 18.3782696177062
498 18.3955823293173
499 18.4168336673347
500 18.446
501 18.4610778443114
502 18.4880478087649
503 18.5089463220676
504 18.5059523809524
505 18.5405940594059
506 18.5671936758893
507 18.5897435897436
508 18.5669291338583
509 18.5913555992141
510 18.6156862745098
511 18.5929549902153
512 18.615234375
513 18.6081871345029
514 18.6400778210117
515 18.6582524271845
516 18.6356589147287
517 18.6247582205029
518 18.6235521235521
519 18.6531791907514
520 18.625
521 18.6429942418426
522 18.6475095785441
523 18.65965583174
524 18.6641221374046
525 18.6704761904762
526 18.6577946768061
527 18.6261859582543
528 18.6382575757576
529 18.6465028355388
530 18.6301886792453
531 18.623352165725
532 18.6334586466165
533 18.5984990619137
534 18.563670411985
535 18.5345794392523
536 18.5279850746269
537 18.5567970204842
538 18.5501858736059
539 18.5751391465677
540 18.5666666666667
541 18.543438077634
542 18.5682656826568
543 18.5506445672192
544 18.5735294117647
545 18.5724770642202
546 18.5732600732601
547 18.5393053016453
548 18.5237226277372
549 18.5245901639344
550 18.5054545454545
551 18.508166969147
552 18.4963768115942
553 18.4792043399638
554 18.4458483754513
555 18.4738738738739
556 18.4478417266187
557 18.4631956912029
558 18.4713261648746
559 18.4758497316637
560 18.4482142857143
561 18.4295900178253
562 18.4501779359431
563 18.4280639431616
564 18.4131205673759
565 18.412389380531
566 18.3922261484099
567 18.4021164021164
568 18.3943661971831
569 18.4165202108963
570 18.4210526315789
571 18.3905429071804
572 18.3951048951049
573 18.4013961605585
574 18.4163763066202
575 18.4452173913043
576 18.4322916666667
577 18.4506065857886
578 18.4740484429066
579 18.4991364421416
580 18.5086206896552
581 18.487091222031
582 18.5051546391753
583 18.4819897084048
584 18.5068493150685
585 18.4974358974359
586 18.4795221843003
587 18.5059625212947
588 18.4778911564626
589 18.4482173174873
590 18.4254237288136
591 18.4416243654822
592 18.4239864864865
593 18.397976391231
594 18.4208754208754
595 18.4420168067227
596 18.4328859060403
597 18.4522613065327
598 18.4397993311037
599 18.4641068447412
600 18.45
601 18.4292845257904
602 18.4385382059801
603 18.4145936981758
604 18.4056291390728
605 18.3801652892562
606 18.3646864686469
607 18.3789126853377
608 18.3536184210526
609 18.3793103448276
610 18.3655737704918
611 18.3567921440262
612 18.3692810457516
613 18.3882544861338
614 18.4039087947883
615 18.380487804878
616 18.375
617 18.3970826580227
618 18.378640776699
619 18.389337641357
620 18.3967741935484
621 18.389694041868
622 18.3649517684887
623 18.3739967897271
624 18.3782051282051
625 18.3504
626 18.3466453674121
627 18.3700159489633
628 18.3805732484076
629 18.3831478537361
630 18.3857142857143
631 18.4136291600634
632 18.4319620253165
633 18.4123222748815
634 18.4006309148265
635 18.3732283464567
636 18.3647798742138
637 18.3720565149137
638 18.3448275862069
639 18.3286384976526
640 18.3125
641 18.2854914196568
642 18.2881619937695
643 18.2597200622084
644 18.2437888198758
645 18.2589147286822
646 18.2755417956656
647 18.2689335394127
648 18.2854938271605
649 18.3127889060092
650 18.2923076923077
651 18.3133640552995
652 18.308282208589
653 18.3047473200613
654 18.2859327217125
655 18.3022900763359
656 18.3033536585366
657 18.2800608828006
658 18.273556231003
659 18.2594840667678
660 18.2530303030303
661 18.231467473525
662 18.2114803625378
663 18.2081447963801
664 18.210843373494
665 18.2210526315789
666 18.2327327327327
667 18.2578710644678
668 18.252994011976
669 18.2690582959641
670 18.2477611940299
671 18.25782414307
672 18.2410714285714
673 18.222882615156
674 18.2255192878338
675 18.237037037037
676 18.2159763313609
677 18.2156573116691
678 18.2241887905605
679 18.2120765832106
680 18.2176470588235
681 18.2393538913363
682 18.2390029325513
683 18.2459736456808
684 18.2690058479532
685 18.2744525547445
686 18.2827988338192
687 18.2561863173217
688 18.2456395348837
689 18.2307692307692
690 18.2434782608696
691 18.2532561505065
692 18.2268786127168
693 18.2294372294372
694 18.2478386167147
695 18.2489208633094
696 18.2313218390805
697 18.2223816355811
698 18.1977077363897
699 18.2103004291845
700 18.2114285714286
701 18.1954350927247
702 18.1951566951567
703 18.1778093883357
704 18.1917613636364
705 18.1787234042553
706 18.1784702549575
707 18.1683168316832
708 18.1694915254237
709 18.1847672778561
710 18.1605633802817
711 18.1533052039381
712 18.1601123595506
713 18.1697054698457
714 18.1750700280112
715 18.1636363636364
716 18.1606145251397
717 18.1715481171548
718 18.1699164345404
719 18.1682892906815
720 18.1555555555556
721 18.1789181692094
722 18.1980609418283
723 18.2060857538036
724 18.1809392265193
725 18.1558620689655
726 18.1308539944904
727 18.1361760660248
728 18.114010989011
729 18.1207133058985
730 18.1315068493151
731 18.1518467852257
732 18.155737704918
733 18.1637107776262
734 18.1825613079019
735 18.2
736 18.1779891304348
737 18.1736770691995
738 18.180216802168
739 18.1989174560217
740 18.1743243243243
741 18.1929824561403
742 18.1859838274933
743 18.1709286675639
744 18.1747311827957
745 18.1543624161074
746 18.1474530831099
747 18.1526104417671
748 18.1483957219251
749 18.1401869158878
750 18.14
751 18.1264980026631
752 18.1196808510638
753 18.1420982735724
754 18.131299734748
755 18.153642384106
756 18.1759259259259
757 18.1902245706737
758 18.1754617414248
759 18.1646903820817
760 18.1710526315789
761 18.1576872536137
762 18.1509186351706
763 18.1441677588467
764 18.1374345549738
765 18.1228758169935
766 18.11227154047
767 18.122555410691
768 18.0989583333333
769 18.0975292587776
770 18.0805194805195
771 18.1037613488975
772 18.119170984456
773 18.1384217335058
774 18.1524547803618
775 18.1406451612903
776 18.1327319587629
777 18.1222651222651
778 18.133676092545
779 18.1129653401797
780 18.1025641025641
781 18.0921895006402
782 18.0920716112532
783 18.1034482758621
784 18.094387755102
785 18.0815286624204
786 18.0648854961832
787 18.0444726810673
788 18.0317258883249
789 18.0152091254753
790 17.9974683544304
791 18.0012642225032
792 18.0063131313131
793 18.0252206809584
794 18.0277078085642
795 18.0477987421384
796 18.0414572864322
797 18.0589711417817
798 18.0513784461153
799 18.0488110137672
800 18.06375
801 18.063670411985
802 18.0486284289277
803 18.0485678704857
804 18.0435323383085
805 18.0223602484472
806 18.0198511166253
807 18.0247831474597
808 18.0222772277228
809 18.0407911001236
810 18.0185185185185
811 18.0160295930949
812 18.0049261083744
813 18.0110701107011
814 18.022113022113
815 18.0159509202454
816 18.0049019607843
817 18.0195838433293
818 18.0048899755501
819 17.992673992674
820 17.9987804878049
821 18.0121802679659
822 18.021897810219
823 18.0437424058323
824 18.0485436893204
825 18.0339393939394
826 18.0326876513317
827 18.0266021765417
828 18.048309178744
829 18.0349819059107
830 18.044578313253
831 18.0397111913357
832 18.0192307692308
833 18.0396158463385
834 18.0503597122302
835 18.0694610778443
836 18.0753588516746
837 18.0908004778973
838 18.0978520286396
839 18.0834326579261
840 18.0797619047619
841 18.077288941736
842 18.0724465558195
843 18.0818505338078
844 18.0936018957346
845 18.0745562130178
846 18.0957446808511
847 18.1097992916175
848 18.1308962264151
849 18.1154299175501
850 18.1364705882353
851 18.1280846063455
852 18.1338028169014
853 18.1395076201641
854 18.1252927400468
855 18.1438596491228
856 18.1483644859813
857 18.1633605600933
858 18.1526806526807
859 18.1571594877765
860 18.1651162790698
861 18.1788617886179
862 18.1983758700696
863 18.1842410196987
864 18.1967592592593
865 18.1976878612717
866 18.2124711316397
867 18.2006920415225
868 18.1854838709677
869 18.1956271576525
870 18.1965517241379
871 18.203214695752
872 18.2178899082569
873 18.2153493699885
874 18.2231121281465
875 18.2194285714286
876 18.2100456621005
877 18.1961231470924
878 18.1970387243736
879 18.2150170648464
880 18.1988636363636
881 18.2077185017026
882 18.1984126984127
883 18.1778029445074
884 18.1685520361991
885 18.1864406779661
886 18.1839729119639
887 18.1927846674183
888 18.1880630630631
889 18.1721034870641
890 18.152808988764
891 18.1649831649832
892 18.1838565022422
893 18.1780515117581
894 18.1812080536913
895 18.1687150837989
896 18.1607142857143
897 18.1415830546265
898 18.1447661469933
899 18.1624026696329
900 18.16
901 18.1487236403996
902 18.1574279379157
903 18.1771871539313
904 18.1570796460177
905 18.171270718232
906 18.1699779249448
907 18.1753031973539
908 18.1828193832599
909 18.1639163916392
910 18.1527472527473
911 18.1580680570801
912 18.1436403508772
913 18.1237677984666
914 18.1137855579869
915 18.1333333333333
916 18.1386462882096
917 18.1352235550709
918 18.1252723311547
919 18.1055495103373
920 18.0934782608696
921 18.0814332247557
922 18.0661605206074
923 18.0747562296858
924 18.0757575757576
925 18.0875675675676
926 18.1047516198704
927 18.091693635383
928 18.0862068965517
929 18.0839612486545
930 18.0978494623656
931 18.0859291084855
932 18.0761802575107
933 18.064308681672
934 18.0481798715203
935 18.0417112299465
936 18.0405982905983
937 18.0320170757737
938 18.0351812366738
939 18.0330138445154
940 18.0425531914894
941 18.0563230605739
942 18.0700636942675
943 18.0668080593849
944 18.0688559322034
945 18.0804232804233
946 18.0993657505285
947 18.1161562829989
948 18.0981012658228
949 18.0874604847208
950 18.1010526315789
951 18.1093585699264
952 18.1144957983193
953 18.1238195173137
954 18.1058700209644
955 18.117277486911
956 18.1140167364017
957 18.1212121212121
958 18.1221294363257
959 18.1157455683003
960 18.10625
961 18.1175858480749
962 18.1299376299376
963 18.116303219107
964 18.1172199170124
965 18.1015544041451
966 18.1055900621118
967 18.1147880041365
968 18.1012396694215
969 18.0877192982456
970 18.0824742268041
971 18.0978372811534
972 18.1111111111111
973 18.1294964028777
974 18.1190965092402
975 18.1312820512821
976 18.1372950819672
977 18.1330603889458
978 18.1216768916155
979 18.1348314606742
980 18.1367346938775
981 18.1365953109072
982 18.1486761710794
983 18.1505595116989
984 18.145325203252
985 18.141116751269
986 18.1450304259635
987 18.1317122593718
988 18.1133603238866
989 18.1314459049545
990 18.1212121212121
991 18.1210898082745
992 18.1219758064516
993 18.1127895266868
994 18.1227364185111
995 18.1095477386935
996 18.1114457831325
997 18.1293881644935
998 18.1162324649299
999 18.1141141141141
1000 18.1
};
\addplot [semithick, color3, forget plot]
table {%
1 26
2 27.5
3 19.3333333333333
4 22.75
5 19.4
6 20
7 17.2857142857143
8 16.5
9 16.5555555555556
10 16.1
11 16.0909090909091
12 15.6666666666667
13 17
14 18.3571428571429
15 18.3333333333333
16 18.75
17 18.1176470588235
18 17.3888888888889
19 16.8947368421053
20 16.6
21 16.7619047619048
22 16.5454545454545
23 16.2173913043478
24 15.7916666666667
25 16.44
26 16.1923076923077
27 16.2962962962963
28 16.4642857142857
29 16.6896551724138
30 17.1
31 17.6129032258065
32 17.6875
33 18.2424242424242
34 17.8529411764706
35 17.4857142857143
36 17.8055555555556
37 17.4054054054054
38 17.0526315789474
39 17.5384615384615
40 17.375
41 16.9512195121951
42 16.7857142857143
43 16.7906976744186
44 16.8409090909091
45 16.4888888888889
46 16.9130434782609
47 16.8936170212766
48 16.7291666666667
49 16.6938775510204
50 16.42
51 16.5098039215686
52 16.8076923076923
53 17
54 17.2592592592593
55 17.4181818181818
56 17.6071428571429
57 17.4035087719298
58 17.7241379310345
59 17.864406779661
60 18.1166666666667
61 17.9672131147541
62 17.8870967741935
63 18.015873015873
64 18.21875
65 18.0307692307692
66 18.2878787878788
67 18.1492537313433
68 17.9117647058824
69 17.7391304347826
70 17.8285714285714
71 17.9577464788732
72 17.7083333333333
73 17.7534246575342
74 17.5945945945946
75 17.3866666666667
76 17.2236842105263
77 17.2987012987013
78 17.4487179487179
79 17.493670886076
80 17.65
81 17.7037037037037
82 17.4878048780488
83 17.710843373494
84 17.5119047619048
85 17.3882352941176
86 17.5232558139535
87 17.5862068965517
88 17.4545454545455
89 17.6179775280899
90 17.6888888888889
91 17.8131868131868
92 17.6304347826087
93 17.6021505376344
94 17.7127659574468
95 17.7894736842105
96 17.8125
97 17.7628865979381
98 17.6224489795918
99 17.7171717171717
100 17.88
101 17.8613861386139
102 17.9803921568627
103 17.9029126213592
104 18.0192307692308
105 18.152380952381
106 18.1698113207547
107 18
108 18.1018518518519
109 17.9449541284404
110 17.9545454545455
111 17.9459459459459
112 17.8125
113 17.929203539823
114 17.8421052631579
115 17.8
116 17.7068965517241
117 17.8547008547009
118 17.9406779661017
119 18.0336134453782
120 17.9333333333333
121 17.8181818181818
122 17.8524590163934
123 17.7723577235772
124 17.7741935483871
125 17.744
126 17.8730158730159
127 17.9606299212598
128 17.9765625
129 18.031007751938
130 18.0461538461538
131 17.9923664122137
132 18.0378787878788
133 17.9323308270677
134 18.0671641791045
135 18.2
136 18.2132352941176
137 18.1751824817518
138 18.1304347826087
139 18.2230215827338
140 18.35
141 18.3049645390071
142 18.3802816901408
143 18.4055944055944
144 18.3333333333333
145 18.3793103448276
146 18.4109589041096
147 18.3537414965986
148 18.3513513513514
149 18.3020134228188
150 18.4066666666667
151 18.4635761589404
152 18.3815789473684
153 18.4640522875817
154 18.3831168831169
155 18.3290322580645
156 18.3910256410256
157 18.4012738853503
158 18.3227848101266
159 18.4088050314465
160 18.30625
161 18.2111801242236
162 18.320987654321
163 18.4294478527607
164 18.3963414634146
165 18.2909090909091
166 18.1927710843373
167 18.1197604790419
168 18.1607142857143
169 18.0650887573964
170 18.0647058823529
171 18.0526315789474
172 18.0348837209302
173 18.0635838150289
174 18.1264367816092
175 18.04
176 17.9659090909091
177 17.864406779661
178 17.7808988764045
179 17.8268156424581
180 17.8555555555556
181 17.767955801105
182 17.6868131868132
183 17.6830601092896
184 17.7717391304348
185 17.7837837837838
186 17.7741935483871
187 17.8502673796791
188 17.8351063829787
189 17.8677248677249
190 17.8157894736842
191 17.8743455497382
192 17.9375
193 17.8911917098446
194 17.9381443298969
195 17.9692307692308
196 18.0102040816327
197 18.0609137055838
198 18.0808080808081
199 18.0804020100502
200 18.135
201 18.2238805970149
202 18.1782178217822
203 18.1379310344828
204 18.1470588235294
205 18.0585365853659
206 18.1116504854369
207 18.1739130434783
208 18.1009615384615
209 18.1148325358852
210 18.0857142857143
211 18.1137440758294
212 18.0754716981132
213 18.0516431924883
214 18.0981308411215
215 18.1441860465116
216 18.1296296296296
217 18.0691244239631
218 18.1284403669725
219 18.0913242009132
220 18.1681818181818
221 18.1719457013575
222 18.2522522522523
223 18.2735426008969
224 18.2991071428571
225 18.2533333333333
226 18.212389380531
227 18.136563876652
228 18.0745614035088
229 18.0742358078603
230 18.1260869565217
231 18.0952380952381
232 18.1724137931034
233 18.1845493562232
234 18.2051282051282
235 18.1957446808511
236 18.1822033898305
237 18.1476793248945
238 18.1680672268908
239 18.2301255230126
240 18.175
241 18.195020746888
242 18.1900826446281
243 18.2057613168724
244 18.2254098360656
245 18.2081632653061
246 18.2235772357724
247 18.2105263157895
248 18.1935483870968
249 18.1726907630522
250 18.148
251 18.1474103585657
252 18.1547619047619
253 18.2094861660079
254 18.1929133858268
255 18.243137254902
256 18.203125
257 18.272373540856
258 18.3255813953488
259 18.3166023166023
260 18.2769230769231
261 18.2835249042146
262 18.2137404580153
263 18.2433460076046
264 18.2954545454545
265 18.2377358490566
266 18.2142857142857
267 18.1947565543071
268 18.2574626865672
269 18.2304832713755
270 18.2444444444444
271 18.2730627306273
272 18.2867647058824
273 18.3516483516484
274 18.2956204379562
275 18.2690909090909
276 18.3260869565217
277 18.2707581227437
278 18.2841726618705
279 18.3118279569892
280 18.2928571428571
281 18.288256227758
282 18.2695035460993
283 18.2402826855124
284 18.2183098591549
285 18.2035087719298
286 18.1888111888112
287 18.191637630662
288 18.1458333333333
289 18.1695501730104
290 18.1586206896552
291 18.1718213058419
292 18.1198630136986
293 18.0716723549488
294 18.0816326530612
295 18.0440677966102
296 18.0135135135135
297 17.973063973064
298 17.9328859060403
299 17.943143812709
300 17.9133333333333
301 17.9435215946844
302 17.9602649006623
303 17.957095709571
304 17.8980263157895
305 17.8852459016393
306 17.8725490196078
307 17.814332247557
308 17.7662337662338
309 17.7346278317152
310 17.6903225806452
311 17.6430868167203
312 17.6378205128205
313 17.6453674121406
314 17.6687898089172
315 17.6380952380952
316 17.6962025316456
317 17.7413249211356
318 17.7327044025157
319 17.6833855799373
320 17.659375
321 17.6386292834891
322 17.5869565217391
323 17.6191950464396
324 17.570987654321
325 17.5353846153846
326 17.4815950920245
327 17.434250764526
328 17.4207317073171
329 17.3920972644377
330 17.369696969697
331 17.4018126888218
332 17.4096385542169
333 17.4114114114114
334 17.3922155688623
335 17.3761194029851
336 17.4077380952381
337 17.4362017804154
338 17.4497041420118
339 17.4041297935103
340 17.3558823529412
341 17.3079178885631
342 17.3538011695906
343 17.3206997084548
344 17.3546511627907
345 17.3159420289855
346 17.3005780346821
347 17.2795389048991
348 17.2327586206897
349 17.2091690544413
350 17.1914285714286
351 17.1680911680912
352 17.1931818181818
353 17.1529745042493
354 17.1751412429379
355 17.2197183098592
356 17.1825842696629
357 17.1428571428571
358 17.1312849162011
359 17.08356545961
360 17.1111111111111
361 17.1301939058172
362 17.1740331491713
363 17.1570247933884
364 17.1785714285714
365 17.1315068493151
366 17.1338797814208
367 17.1362397820164
368 17.1413043478261
369 17.1788617886179
370 17.1432432432432
371 17.1105121293801
372 17.1586021505376
373 17.1528150134048
374 17.120320855615
375 17.1226666666667
376 17.093085106383
377 17.1140583554377
378 17.1507936507937
379 17.1609498680739
380 17.1552631578947
381 17.1837270341207
382 17.1570680628272
383 17.1984334203655
384 17.1692708333333
385 17.1532467532468
386 17.1165803108808
387 17.1627906976744
388 17.1520618556701
389 17.146529562982
390 17.1589743589744
391 17.1355498721228
392 17.1811224489796
393 17.1908396946565
394 17.1598984771574
395 17.1493670886076
396 17.1439393939394
397 17.1612090680101
398 17.1306532663317
399 17.1428571428571
400 17.1525
401 17.1645885286783
402 17.2039800995025
403 17.2109181141439
404 17.2450495049505
405 17.2814814814815
406 17.320197044335
407 17.3095823095823
408 17.3088235294118
409 17.3227383863081
410 17.2853658536585
411 17.3041362530414
412 17.2864077669903
413 17.2687651331719
414 17.2946859903382
415 17.2771084337349
416 17.3028846153846
417 17.2757793764988
418 17.2799043062201
419 17.2649164677804
420 17.2595238095238
421 17.2874109263658
422 17.2748815165877
423 17.2978723404255
424 17.2712264150943
425 17.24
426 17.2018779342723
427 17.231850117096
428 17.2429906542056
429 17.2517482517483
430 17.2627906976744
431 17.2505800464037
432 17.2939814814815
433 17.3117782909931
434 17.3179723502304
435 17.3103448275862
436 17.3440366972477
437 17.3684210526316
438 17.3584474885845
439 17.373576309795
440 17.35
441 17.3401360544218
442 17.3235294117647
443 17.3273137697517
444 17.3040540540541
445 17.3370786516854
446 17.3004484304933
447 17.3109619686801
448 17.3459821428571
449 17.3140311804009
450 17.3288888888889
451 17.3658536585366
452 17.3451327433628
453 17.3355408388521
454 17.3303964757709
455 17.356043956044
456 17.3618421052632
457 17.328227571116
458 17.3406113537118
459 17.3246187363834
460 17.35
461 17.3275488069414
462 17.3528138528139
463 17.3650107991361
464 17.385775862069
465 17.4021505376344
466 17.4420600858369
467 17.4047109207709
468 17.3952991452991
469 17.3773987206823
470 17.3914893617021
471 17.3736730360934
472 17.4067796610169
473 17.4207188160677
474 17.4008438818565
475 17.3915789473684
476 17.4264705882353
477 17.4570230607966
478 17.4811715481172
479 17.4592901878914
480 17.44375
481 17.4365904365904
482 17.4460580912863
483 17.4409937888199
484 17.4566115702479
485 17.4474226804124
486 17.4362139917695
487 17.4312114989733
488 17.4180327868852
489 17.398773006135
490 17.4163265306122
491 17.3808553971487
492 17.3739837398374
493 17.3853955375254
494 17.414979757085
495 17.3838383838384
496 17.3931451612903
497 17.3702213279678
498 17.3534136546185
499 17.3226452905812
500 17.292
501 17.3213572854291
502 17.3306772908367
503 17.3121272365805
504 17.3392857142857
505 17.3346534653465
506 17.3122529644269
507 17.3313609467456
508 17.3681102362205
509 17.3673870333988
510 17.3627450980392
511 17.3444227005871
512 17.3203125
513 17.3255360623782
514 17.3560311284047
515 17.3553398058252
516 17.3798449612403
517 17.3829787234043
518 17.4131274131274
519 17.3988439306358
520 17.3980769230769
521 17.3685220729367
522 17.3697318007663
523 17.395793499044
524 17.3950381679389
525 17.4247619047619
526 17.3992395437262
527 17.3851992409867
528 17.3901515151515
529 17.3950850661626
530 17.4
531 17.4067796610169
532 17.3759398496241
533 17.3508442776735
534 17.3295880149813
535 17.3289719626168
536 17.3097014925373
537 17.3258845437616
538 17.3531598513011
539 17.3747680890538
540 17.35
541 17.3807763401109
542 17.4077490774908
543 17.4235727440147
544 17.3915441176471
545 17.3963302752294
546 17.3937728937729
547 17.3802559414991
548 17.3540145985401
549 17.3879781420765
550 17.3836363636364
551 17.3811252268603
552 17.4130434782609
553 17.3978300180832
554 17.3664259927798
555 17.3423423423423
556 17.3291366906475
557 17.3034111310592
558 17.2939068100358
559 17.3076923076923
560 17.3392857142857
561 17.3440285204991
562 17.3523131672598
563 17.3392539964476
564 17.3705673758865
565 17.353982300885
566 17.3533568904594
567 17.3386243386243
568 17.3433098591549
569 17.3356766256591
570 17.3543859649123
571 17.3274956217163
572 17.3339160839161
573 17.3577661431065
574 17.3763066202091
575 17.3478260869565
576 17.3576388888889
577 17.3604852686308
578 17.3685121107266
579 17.3419689119171
580 17.3241379310345
581 17.3459552495697
582 17.3367697594502
583 17.3602058319039
584 17.3630136986301
585 17.3401709401709
586 17.3668941979522
587 17.3373083475298
588 17.3248299319728
589 17.2971137521222
590 17.2779661016949
591 17.2673434856176
592 17.285472972973
593 17.2984822934233
594 17.2929292929293
595 17.3226890756303
596 17.3523489932886
597 17.356783919598
598 17.3595317725753
599 17.3706176961603
600 17.3433333333333
601 17.3178036605657
602 17.3488372093023
603 17.3615257048093
604 17.3741721854305
605 17.3917355371901
606 17.4125412541254
607 17.4233937397035
608 17.4161184210526
609 17.4269293924466
610 17.4475409836066
611 17.4713584288052
612 17.5
613 17.4991843393148
614 17.485342019544
615 17.4878048780488
616 17.4691558441558
617 17.482982171799
618 17.4805825242718
619 17.5072697899838
620 17.5241935483871
621 17.4991948470209
622 17.5128617363344
623 17.5329052969502
624 17.5464743589744
625 17.5296
626 17.555910543131
627 17.5374800637959
628 17.5429936305732
629 17.5182829888712
630 17.5253968253968
631 17.5007923930269
632 17.5253164556962
633 17.5308056872038
634 17.5520504731861
635 17.5275590551181
636 17.5503144654088
637 17.5761381475667
638 17.564263322884
639 17.5915492957746
640 17.6
641 17.6177847113885
642 17.6230529595016
643 17.6174183514775
644 17.5931677018634
645 17.5984496124031
646 17.5789473684211
647 17.5950540958269
648 17.5956790123457
649 17.597842835131
650 17.6123076923077
651 17.594470046083
652 17.6165644171779
653 17.6339969372129
654 17.6483180428135
655 17.6519083969466
656 17.6448170731707
657 17.6605783866058
658 17.6823708206687
659 17.6555386949924
660 17.6363636363636
661 17.6384266263238
662 17.6359516616314
663 17.657616892911
664 17.6325301204819
665 17.6541353383459
666 17.6546546546547
667 17.6746626686657
668 17.6736526946108
669 17.6980568011958
670 17.7119402985075
671 17.7302533532042
672 17.7127976190476
673 17.7102526002972
674 17.700296735905
675 17.6874074074074
676 17.6923076923077
677 17.6971935007386
678 17.7094395280236
679 17.7187039764359
680 17.7220588235294
681 17.7209985315712
682 17.6979472140762
683 17.6998535871157
684 17.7090643274854
685 17.6948905109489
686 17.698250728863
687 17.6928675400291
688 17.6889534883721
689 17.6734397677794
690 17.6492753623188
691 17.6512301013025
692 17.6300578034682
693 17.6291486291486
694 17.6541786743516
695 17.6719424460432
696 17.676724137931
697 17.6556671449067
698 17.6747851002865
699 17.6494992846924
700 17.6471428571429
701 17.6519258202568
702 17.6752136752137
703 17.6628733997155
704 17.6875
705 17.7049645390071
706 17.7039660056657
707 17.7171145685997
708 17.7429378531073
709 17.7348377997179
710 17.7295774647887
711 17.7552742616034
712 17.7387640449438
713 17.7279102384292
714 17.7521008403361
715 17.7594405594406
716 17.7402234636872
717 17.7308228730823
718 17.7381615598886
719 17.7190542420028
720 17.6972222222222
721 17.7128987517337
722 17.6883656509695
723 17.6887966804979
724 17.6726519337017
725 17.6910344827586
726 17.6955922865014
727 17.7001375515818
728 17.7115384615385
729 17.6872427983539
730 17.6684931506849
731 17.671682626539
732 17.6803278688525
733 17.668485675307
734 17.6621253405995
735 17.6857142857143
736 17.6820652173913
737 17.7055630936228
738 17.710027100271
739 17.6874154262517
740 17.6959459459459
741 17.7112010796221
742 17.7277628032345
743 17.7160161507402
744 17.7204301075269
745 17.7073825503356
746 17.7319034852547
747 17.7536813922356
748 17.7312834224599
749 17.7209612817089
750 17.7346666666667
751 17.7549933422104
752 17.7579787234043
753 17.7662682602922
754 17.7586206896552
755 17.7496688741722
756 17.7738095238095
757 17.7529722589168
758 17.7467018469657
759 17.7430830039526
760 17.7513157894737
761 17.7687253613666
762 17.758530183727
763 17.7614678899083
764 17.7526178010471
765 17.7607843137255
766 17.7493472584856
767 17.7275097783572
768 17.7434895833333
769 17.7607282184655
770 17.7402597402597
771 17.7315175097276
772 17.7098445595855
773 17.7024579560155
774 17.7260981912145
775 17.7212903225806
776 17.7409793814433
777 17.7477477477477
778 17.7583547557841
779 17.7560975609756
780 17.7564102564103
781 17.752880921895
782 17.7416879795396
783 17.7292464878672
784 17.7525510204082
785 17.743949044586
786 17.7493638676845
787 17.7712833545108
788 17.753807106599
789 17.7465145754119
790 17.7481012658228
791 17.7547408343869
792 17.7765151515152
793 17.7944514501892
794 17.808564231738
795 17.7899371069182
796 17.7939698492462
797 17.8030112923463
798 17.7882205513784
799 17.7984981226533
800 17.79875
801 17.7915106117353
802 17.7693266832918
803 17.7509339975093
804 17.75
805 17.767701863354
806 17.7803970223325
807 17.7657992565056
808 17.7462871287129
809 17.761433868974
810 17.7543209876543
811 17.7644882860666
812 17.7783251231527
813 17.7773677736777
814 17.7727272727273
815 17.7521472392638
816 17.7377450980392
817 17.7576499388005
818 17.7689486552567
819 17.7667887667888
820 17.7646341463415
821 17.7466504263094
822 17.7518248175182
823 17.7533414337789
824 17.7657766990291
825 17.7648484848485
826 17.7845036319613
827 17.7992744860943
828 17.7958937198068
829 17.8009650180941
830 17.8156626506024
831 17.7990373044525
832 17.796875
833 17.7863145258103
834 17.7997601918465
835 17.8059880239521
836 17.8122009569378
837 17.8267622461171
838 17.8078758949881
839 17.8295589988081
840 17.8369047619048
841 17.8192627824019
842 17.8313539192399
843 17.8422301304864
844 17.8388625592417
845 17.8378698224852
846 17.8451536643026
847 17.8429752066116
848 17.8325471698113
849 17.8504122497055
850 17.8717647058824
851 17.8566392479436
852 17.8638497652582
853 17.873388042204
854 17.8922716627635
855 17.8842105263158
856 17.9053738317757
857 17.9218203033839
858 17.9382284382284
859 17.9441210710128
860 17.9383720930233
861 17.9279907084785
862 17.9222737819026
863 17.901506373117
864 17.9224537037037
865 17.9052023121387
866 17.8856812933025
867 17.8685121107266
868 17.8767281105991
869 17.867663981588
870 17.8540229885057
871 17.8599311136625
872 17.8520642201835
873 17.8705612829324
874 17.8604118993135
875 17.8754285714286
876 17.8561643835616
877 17.8415051311288
878 17.8302961275626
879 17.8248009101251
880 17.8159090909091
881 17.7956867196368
882 17.7879818594104
883 17.8006795016988
884 17.7918552036199
885 17.7966101694915
886 17.793453724605
887 17.782412626832
888 17.777027027027
889 17.7626546681665
890 17.7696629213483
891 17.7744107744108
892 17.7881165919283
893 17.7917133258679
894 17.7796420581655
895 17.7787709497207
896 17.7801339285714
897 17.7915273132664
898 17.78285077951
899 17.7853170189099
900 17.8055555555556
901 17.8002219755827
902 17.8059866962306
903 17.8050941306755
904 17.8163716814159
905 17.8132596685083
906 17.8134657836645
907 17.8070562293275
908 17.807268722467
909 17.8019801980198
910 17.8076923076923
911 17.7925356750823
912 17.8059210526316
913 17.7973713033954
914 17.8140043763676
915 17.824043715847
916 17.8198689956332
917 17.835332606325
918 17.8398692810458
919 17.822633297062
920 17.804347826087
921 17.8067318132465
922 17.795010845987
923 17.8006500541712
924 17.7987012987013
925 17.8140540540541
926 17.804535637149
927 17.8047464940669
928 17.801724137931
929 17.8062432723358
930 17.7978494623656
931 17.8163265306122
932 17.8240343347639
933 17.8252947481243
934 17.8190578158458
935 17.8096256684492
936 17.8119658119658
937 17.8249733191035
938 17.8102345415778
939 17.828541001065
940 17.8404255319149
941 17.8512221041445
942 17.859872611465
943 17.8748674443266
944 17.8622881355932
945 17.8539682539683
946 17.845665961945
947 17.8426610348469
948 17.8270042194093
949 17.8113804004215
950 17.8221052631579
951 17.8380651945321
952 17.8497899159664
953 17.8331584470094
954 17.8218029350105
955 17.8293193717277
956 17.8242677824268
957 17.8307210031348
958 17.8308977035491
959 17.8258602711157
960 17.8177083333333
961 17.8303850156087
962 17.8357588357588
963 17.8494288681205
964 17.8630705394191
965 17.8569948186529
966 17.8633540372671
967 17.8676318510858
968 17.8564049586777
969 17.8627450980392
970 17.8690721649485
971 17.8527291452111
972 17.8497942386831
973 17.8550873586845
974 17.8624229979466
975 17.8615384615385
976 17.8668032786885
977 17.85670419652
978 17.8609406952965
979 17.8447395301328
980 17.8612244897959
981 17.8440366972477
982 17.8309572301426
983 17.8382502543235
984 17.8404471544715
985 17.853807106599
986 17.869168356998
987 17.8844984802432
988 17.8937246963563
989 17.8827098078868
990 17.8939393939394
991 17.9101917255298
992 17.9163306451613
993 17.8992950654582
994 17.897384305835
995 17.8984924623116
996 17.9056224899598
997 17.9057171514544
998 17.8897795591182
999 17.8998998998999
1000 17.917
};
\addplot [semithick, color4, forget plot]
table {%
1 10
2 8
3 16
4 15.75
5 14.8
6 16.6666666666667
7 18.5714285714286
8 19.25
9 17.4444444444444
10 19.2
11 19.8181818181818
12 21.1666666666667
13 22.1538461538462
14 22.5
15 22.8666666666667
16 22.5625
17 23.2941176470588
18 23.6111111111111
19 23.3684210526316
20 22.55
21 22.7619047619048
22 22.6818181818182
23 23.1304347826087
24 23.4166666666667
25 23.92
26 24.1153846153846
27 24.5555555555556
28 24.5357142857143
29 24.8620689655172
30 24.0666666666667
31 24.0322580645161
32 23.375
33 23.3939393939394
34 23.4411764705882
35 23.1428571428571
36 22.5555555555556
37 21.9459459459459
38 22.1052631578947
39 22
40 21.5
41 21.2439024390244
42 21.0238095238095
43 20.9302325581395
44 21.1363636363636
45 21.0444444444444
46 20.6521739130435
47 20.2127659574468
48 20.0833333333333
49 20.1020408163265
50 20.42
51 20.3921568627451
52 20.5769230769231
53 20.2641509433962
54 19.9259259259259
55 19.6545454545455
56 19.875
57 19.6491228070175
58 19.8275862068966
59 19.5254237288136
60 19.6833333333333
61 19.8852459016393
62 20
63 19.9047619047619
64 19.59375
65 19.2923076923077
66 19.2878787878788
67 19
68 18.9411764705882
69 18.9855072463768
70 18.9857142857143
71 18.9577464788732
72 19.1944444444444
73 19.027397260274
74 19.1621621621622
75 19.3733333333333
76 19.4736842105263
77 19.2987012987013
78 19.3717948717949
79 19.3164556962025
80 19.0875
81 18.8888888888889
82 18.6707317073171
83 18.7469879518072
84 18.9404761904762
85 18.7764705882353
86 18.8837209302326
87 18.9195402298851
88 18.7613636363636
89 18.8089887640449
90 18.9333333333333
91 19
92 19.1847826086957
93 19.0967741935484
94 19.1276595744681
95 19.2421052631579
96 19.2083333333333
97 19.2577319587629
98 19.3775510204082
99 19.2121212121212
100 19.28
101 19.3168316831683
102 19.2647058823529
103 19.1941747572816
104 19.0192307692308
105 18.9428571428571
106 18.8679245283019
107 18.7196261682243
108 18.7407407407407
109 18.6146788990826
110 18.6181818181818
111 18.6576576576577
112 18.5803571428571
113 18.6548672566372
114 18.6052631578947
115 18.5652173913043
116 18.5862068965517
117 18.5213675213675
118 18.5254237288136
119 18.5042016806723
120 18.55
121 18.6115702479339
122 18.5327868852459
123 18.5691056910569
124 18.5887096774194
125 18.56
126 18.531746031746
127 18.5826771653543
128 18.4921875
129 18.5658914728682
130 18.4384615384615
131 18.381679389313
132 18.2954545454545
133 18.1578947368421
134 18.1417910447761
135 18.0592592592593
136 18.1911764705882
137 18.1751824817518
138 18.2391304347826
139 18.1151079136691
140 18.1785714285714
141 18.2695035460993
142 18.1408450704225
143 18.1468531468531
144 18.125
145 18
146 18.1232876712329
147 18.047619047619
148 17.9391891891892
149 18.0335570469799
150 18.0133333333333
151 18.112582781457
152 18.0657894736842
153 17.9803921568627
154 18.0194805194805
155 18.0387096774194
156 17.9487179487179
157 17.968152866242
158 17.9746835443038
159 17.9308176100629
160 18.0125
161 17.9813664596273
162 18.037037037037
163 18.0736196319018
164 18.0609756097561
165 18.1151515151515
166 18.210843373494
167 18.2574850299401
168 18.2738095238095
169 18.207100591716
170 18.3058823529412
171 18.3391812865497
172 18.25
173 18.3526011560694
174 18.2988505747126
175 18.2285714285714
176 18.2897727272727
177 18.2768361581921
178 18.1797752808989
179 18.1787709497207
180 18.1888888888889
181 18.1436464088398
182 18.1153846153846
183 18.0382513661202
184 18.0108695652174
185 18.0324324324324
186 18.1290322580645
187 18.1229946524064
188 18.1648936170213
189 18.1798941798942
190 18.0842105263158
191 18.0052356020942
192 17.9166666666667
193 17.9740932642487
194 17.9278350515464
195 17.9897435897436
196 17.9948979591837
197 18.0203045685279
198 18.0555555555556
199 18.0954773869347
200 18.135
201 18.0945273631841
202 18.049504950495
203 18.1133004926108
204 18.1323529411765
205 18.0536585365854
206 18.0291262135922
207 17.9420289855072
208 17.9086538461538
209 17.9952153110048
210 18.0714285714286
211 18.1469194312796
212 18.1933962264151
213 18.2723004694836
214 18.214953271028
215 18.2232558139535
216 18.3009259259259
217 18.2442396313364
218 18.2706422018349
219 18.337899543379
220 18.3818181818182
221 18.3891402714932
222 18.3783783783784
223 18.4125560538117
224 18.3482142857143
225 18.28
226 18.3185840707965
227 18.2775330396476
228 18.2105263157895
229 18.2794759825328
230 18.3434782608696
231 18.2900432900433
232 18.2629310344828
233 18.3175965665236
234 18.3376068376068
235 18.3021276595745
236 18.2754237288136
237 18.253164556962
238 18.2941176470588
239 18.2594142259414
240 18.2041666666667
241 18.2489626556017
242 18.3099173553719
243 18.3703703703704
244 18.3073770491803
245 18.3224489795918
246 18.3048780487805
247 18.3684210526316
248 18.3306451612903
249 18.3815261044177
250 18.324
251 18.2669322709163
252 18.2380952380952
253 18.2213438735178
254 18.1968503937008
255 18.156862745098
256 18.08984375
257 18.1206225680934
258 18.1705426356589
259 18.1081081081081
260 18.1115384615385
261 18.0689655172414
262 18.1259541984733
263 18.1787072243346
264 18.1212121212121
265 18.0754716981132
266 18.093984962406
267 18.1573033707865
268 18.1044776119403
269 18.092936802974
270 18.1148148148148
271 18.1660516605166
272 18.2095588235294
273 18.2307692307692
274 18.2408759124088
275 18.2072727272727
276 18.213768115942
277 18.2129963898917
278 18.158273381295
279 18.15770609319
280 18.1464285714286
281 18.135231316726
282 18.1808510638298
283 18.1625441696113
284 18.1866197183099
285 18.1473684210526
286 18.1328671328671
287 18.1114982578397
288 18.09375
289 18.0449826989619
290 18.0965517241379
291 18.0996563573883
292 18.1301369863014
293 18.1535836177474
294 18.1836734693878
295 18.1322033898305
296 18.1824324324324
297 18.1481481481481
298 18.1241610738255
299 18.0936454849498
300 18.1366666666667
301 18.093023255814
302 18.1423841059603
303 18.1683168316832
304 18.1118421052632
305 18.1147540983607
306 18.1437908496732
307 18.1596091205212
308 18.211038961039
309 18.1877022653722
310 18.2290322580645
311 18.2282958199357
312 18.1762820512821
313 18.2108626198083
314 18.2292993630573
315 18.1746031746032
316 18.126582278481
317 18.0883280757098
318 18.0314465408805
319 17.9843260188088
320 18.0375
321 18.0342679127726
322 18.0590062111801
323 18.0185758513932
324 18.0123456790123
325 17.9969230769231
326 17.9631901840491
327 18.0183486238532
328 18.0579268292683
329 18.1094224924012
330 18.1151515151515
331 18.1299093655589
332 18.105421686747
333 18.1141141141141
334 18.1047904191617
335 18.1074626865672
336 18.0982142857143
337 18.1097922848665
338 18.0917159763314
339 18.0884955752212
340 18.1323529411765
341 18.1348973607038
342 18.1286549707602
343 18.0962099125364
344 18.0639534883721
345 18.0608695652174
346 18.014450867052
347 18.0028818443804
348 18.0488505747126
349 18.0916905444126
350 18.1
351 18.0797720797721
352 18.0625
353 18.1048158640227
354 18.135593220339
355 18.169014084507
356 18.2134831460674
357 18.1624649859944
358 18.122905027933
359 18.1392757660167
360 18.0944444444444
361 18.1357340720222
362 18.1270718232044
363 18.1129476584022
364 18.1318681318681
365 18.0821917808219
366 18.0546448087432
367 18.0681198910082
368 18.0489130434783
369 18.0027100271003
370 18.027027027027
371 18.0080862533693
372 18.0295698924731
373 18.0268096514745
374 17.9946524064171
375 17.9706666666667
376 17.9627659574468
377 17.9416445623342
378 17.9708994708995
379 18.0052770448549
380 18.0342105263158
381 18.0629921259843
382 18.0916230366492
383 18.0731070496084
384 18.0859375
385 18.1246753246753
386 18.1295336787565
387 18.0852713178295
388 18.1237113402062
389 18.1491002570694
390 18.1205128205128
391 18.1202046035806
392 18.0918367346939
393 18.1297709923664
394 18.0862944162437
395 18.0607594936709
396 18.0959595959596
397 18.1083123425693
398 18.0929648241206
399 18.047619047619
400 18.0075
401 17.9950124688279
402 17.952736318408
403 17.9280397022332
404 17.9356435643564
405 17.9654320987654
406 18
407 18.017199017199
408 18.031862745098
409 17.9902200488998
410 18.0219512195122
411 18.036496350365
412 18.0776699029126
413 18.0677966101695
414 18.0386473429952
415 18.0433734939759
416 18.0600961538462
417 18.0911270983213
418 18.1124401913876
419 18.1431980906921
420 18.1642857142857
421 18.1852731591449
422 18.1729857819905
423 18.1631205673759
424 18.1957547169811
425 18.2305882352941
426 18.2558685446009
427 18.2154566744731
428 18.1822429906542
429 18.2027972027972
430 18.1604651162791
431 18.1716937354988
432 18.150462962963
433 18.1639722863741
434 18.1958525345622
435 18.167816091954
436 18.1811926605505
437 18.1807780320366
438 18.2009132420091
439 18.1662870159453
440 18.1727272727273
441 18.1700680272109
442 18.2081447963801
443 18.1918735891648
444 18.1666666666667
445 18.161797752809
446 18.1345291479821
447 18.165548098434
448 18.1808035714286
449 18.1536748329621
450 18.1644444444444
451 18.1862527716186
452 18.1946902654867
453 18.1942604856512
454 18.1629955947137
455 18.1252747252747
456 18.1008771929825
457 18.1203501094092
458 18.1397379912664
459 18.1350762527233
460 18.1347826086957
461 18.1084598698482
462 18.0692640692641
463 18.0345572354212
464 18.0366379310345
465 18.0645161290323
466 18.0729613733906
467 18.1070663811563
468 18.1111111111111
469 18.0831556503198
470 18.1212765957447
471 18.0955414012739
472 18.0656779661017
473 18.0507399577167
474 18.0232067510549
475 17.9936842105263
476 17.9831932773109
477 17.9559748427673
478 17.9435146443515
479 17.9415448851775
480 17.9208333333333
481 17.9147609147609
482 17.9294605809129
483 17.9026915113872
484 17.8842975206612
485 17.8824742268041
486 17.8950617283951
487 17.8973305954825
488 17.8954918032787
489 17.8936605316973
490 17.8816326530612
491 17.8635437881874
492 17.8434959349593
493 17.815415821501
494 17.8178137651822
495 17.7959595959596
496 17.8266129032258
497 17.8008048289738
498 17.781124497992
499 17.811623246493
500 17.808
501 17.7864271457086
502 17.8167330677291
503 17.8051689860835
504 17.7896825396825
505 17.8138613861386
506 17.8162055335968
507 17.8520710059172
508 17.8582677165354
509 17.8271119842829
510 17.8254901960784
511 17.8199608610568
512 17.814453125
513 17.8304093567251
514 17.8560311284047
515 17.8446601941748
516 17.8488372093023
517 17.8510638297872
518 17.8416988416988
519 17.8092485549133
520 17.7980769230769
521 17.8157389635317
522 17.8371647509579
523 17.810707456979
524 17.8129770992366
525 17.8152380952381
526 17.8231939163498
527 17.8576850094877
528 17.8503787878788
529 17.8449905482042
530 17.8320754716981
531 17.84934086629
532 17.8214285714286
533 17.8067542213884
534 17.8052434456929
535 17.7869158878505
536 17.7798507462687
537 17.7467411545624
538 17.7397769516729
539 17.7588126159555
540 17.75
541 17.7338262476895
542 17.7269372693727
543 17.7127071823204
544 17.7389705882353
545 17.7559633027523
546 17.7417582417582
547 17.7714808043876
548 17.7737226277372
549 17.7413479052823
550 17.7236363636364
551 17.7495462794918
552 17.7826086956522
553 17.7757685352622
554 17.7617328519856
555 17.7495495495495
556 17.726618705036
557 17.7414721723519
558 17.7329749103943
559 17.7567084078712
560 17.7571428571429
561 17.7504456327986
562 17.7384341637011
563 17.7708703374778
564 17.8014184397163
565 17.8159292035398
566 17.7932862190813
567 17.7707231040564
568 17.7693661971831
569 17.7504393673111
570 17.7719298245614
571 17.7723292469352
572 17.7814685314685
573 17.8115183246073
574 17.7926829268293
575 17.8086956521739
576 17.8315972222222
577 17.8284228769497
578 17.8235294117647
579 17.8169257340242
580 17.848275862069
581 17.8623063683305
582 17.8505154639175
583 17.8319039451115
584 17.8202054794521
585 17.8478632478632
586 17.8310580204778
587 17.809199318569
588 17.7891156462585
589 17.8030560271647
590 17.8135593220339
591 17.8121827411168
592 17.8344594594595
593 17.8634064080944
594 17.8552188552189
595 17.853781512605
596 17.8825503355705
597 17.8726968174204
598 17.8561872909699
599 17.8631051752922
600 17.845
601 17.8602329450915
602 17.8538205980066
603 17.8673300165837
604 17.8973509933775
605 17.9173553719008
606 17.8943894389439
607 17.9110378912685
608 17.9128289473684
609 17.9244663382594
610 17.9491803278689
611 17.9525368248772
612 17.9738562091503
613 17.9869494290375
614 18.0114006514658
615 17.9821138211382
616 18.0081168831169
617 18.0356564019449
618 18.0097087378641
619 17.983844911147
620 17.9870967741935
621 17.9597423510467
622 17.9581993569132
623 17.9486356340289
624 17.9519230769231
625 17.9808
626 18.0079872204473
627 17.993620414673
628 18.015923566879
629 17.9984101748808
630 18.0015873015873
631 17.9873217115689
632 17.9873417721519
633 17.9620853080569
634 17.9495268138801
635 17.9700787401575
636 17.9465408805031
637 17.9607535321821
638 17.9623824451411
639 17.943661971831
640 17.9546875
641 17.9407176287051
642 17.9610591900312
643 17.9766718506998
644 17.9704968944099
645 17.9829457364341
646 17.9876160990712
647 17.9799072642968
648 17.9598765432099
649 17.9645608628659
650 17.9369230769231
651 17.9124423963134
652 17.9340490797546
653 17.9326186830015
654 17.9556574923547
655 17.963358778626
656 17.9481707317073
657 17.9649923896499
658 17.9574468085106
659 17.9575113808801
660 17.9833333333333
661 18.0045385779123
662 18.0166163141994
663 18.0150829562594
664 18.0075301204819
665 17.9834586466165
666 17.9654654654655
667 17.9505247376312
668 17.937125748503
669 17.9297458893871
670 17.9044776119403
671 17.9076005961252
672 17.8854166666667
673 17.8588410104012
674 17.8827893175074
675 17.9037037037037
676 17.9023668639053
677 17.9113737075332
678 17.8923303834808
679 17.9042709867452
680 17.9279411764706
681 17.9544787077827
682 17.9765395894428
683 17.9838945827233
684 18.0102339181287
685 18.0277372262774
686 18.0276967930029
687 18.0014556040757
688 18.0145348837209
689 18.033381712627
690 18.0478260869565
691 18.0709117221418
692 18.0462427745665
693 18.02886002886
694 18.0230547550432
695 18.0129496402878
696 18.0387931034483
697 18.0286944045911
698 18.0114613180516
699 18.0143061516452
700 18.0357142857143
701 18.0442225392297
702 18.0541310541311
703 18.0640113798009
704 18.0809659090909
705 18.0624113475177
706 18.0467422096317
707 18.033946251768
708 18.0254237288136
709 18.0253878702398
710 18.0394366197183
711 18.0393811533052
712 18.0533707865169
713 18.0771388499299
714 18.0952380952381
715 18.0923076923077
716 18.0824022346369
717 18.0906555090656
718 18.0793871866295
719 18.0848400556328
720 18.1083333333333
721 18.1178918169209
722 18.101108033241
723 18.0871369294606
724 18.0732044198895
725 18.0496551724138
726 18.0289256198347
727 18.0288858321871
728 18.0521978021978
729 18.0438957475994
730 18.0232876712329
731 18.015047879617
732 18.0396174863388
733 18.0572987721692
734 18.0395095367847
735 18.0258503401361
736 18.0339673913043
737 18.0352781546811
738 18.0325203252033
739 18.0284167794317
740 18.0135135135135
741 18.0364372469636
742 18.0161725067385
743 18.0201884253028
744 18.0430107526882
745 18.061744966443
746 18.0522788203753
747 18.0348058902276
748 18.0187165775401
749 18.0400534045394
750 18.0413333333333
751 18.0452729693742
752 18.0212765957447
753 18.0225763612218
754 18.0119363395225
755 17.9933774834437
756 17.9722222222222
757 17.9762219286658
758 17.9828496042216
759 17.9973649538867
760 17.9855263157895
761 17.9894875164258
762 17.992125984252
763 18.002621231979
764 18.0052356020942
765 17.9921568627451
766 18.0117493472585
767 18.013037809648
768 18.02734375
769 18.0481144343303
770 18.0519480519481
771 18.0609597924773
772 18.0608808290155
773 18.0776196636481
774 18.0749354005168
775 18.0774193548387
776 18.0579896907216
777 18.0514800514801
778 18.0385604113111
779 18.0577663671374
780 18.048717948718
781 18.0422535211268
782 18.0358056265985
783 18.0127713920817
784 18.0165816326531
785 18.0140127388535
786 18.0038167938931
787 18.0139771283355
788 18
789 17.9860583016477
790 18.0037974683544
791 18.0063211125158
792 17.9987373737374
793 18.0037831021438
794 18.0188916876574
795 18.0100628930818
796 17.9899497487437
797 17.9811794228356
798 17.9586466165414
799 17.973717146433
800 17.95125
801 17.9463171036205
802 17.927680798005
803 17.947696139477
804 17.9328358208955
805 17.9540372670807
806 17.970223325062
807 17.9628252788104
808 17.980198019802
809 17.9715698393078
810 17.9703703703704
811 17.9543773119605
812 17.9679802955665
813 17.990159901599
814 18.0012285012285
815 18.0134969325153
816 18.0036764705882
817 17.9889840881273
818 17.979217603912
819 17.970695970696
820 17.9609756097561
821 17.9634591961023
822 17.978102189781
823 17.9659781287971
824 17.9490291262136
825 17.9624242424242
826 17.9757869249395
827 17.9854897218863
828 17.9685990338164
829 17.9589867310012
830 17.9807228915663
831 17.9939831528279
832 17.9963942307692
833 18.015606242497
834 18.0251798561151
835 18.0359281437126
836 18.0430622009569
837 18.0227001194743
838 18.0262529832936
839 18.0393325387366
840 18.0190476190476
841 17.9976218787158
842 18.0071258907363
843 18.0083036773428
844 17.9976303317536
845 18
846 17.983451536643
847 17.9846517119244
848 17.998820754717
849 17.9882214369847
850 18.0047058823529
851 17.9988249118684
852 18.0023474178404
853 17.9917936694021
854 17.9812646370023
855 17.9695906432749
856 17.9509345794393
857 17.9684947491249
858 17.9755244755245
859 17.988358556461
860 18.0081395348837
861 18.0209059233449
862 18.0313225058005
863 18.0162224797219
864 18.0289351851852
865 18.0104046242775
866 18.0288683602771
867 18.0219146482122
868 18.0264976958525
869 18.0184119677791
870 18.0068965517241
871 17.9988518943743
872 17.9862385321101
873 17.9816723940435
874 17.9713958810069
875 17.9851428571429
876 18
877 17.9954389965792
878 17.9931662870159
879 18.0136518771331
880 18.0306818181818
881 18.0317820658343
882 18.046485260771
883 18.0668176670442
884 18.0542986425339
885 18.0632768361582
886 18.0722347629797
887 18.0698985343856
888 18.0709459459459
889 18.0731158605174
890 18.0707865168539
891 18.0606060606061
892 18.0784753363229
893 18.0806270996641
894 18.0961968680089
895 18.0905027932961
896 18.1082589285714
897 18.1192865105909
898 18.1280623608018
899 18.1323692992214
900 18.1177777777778
901 18.1009988901221
902 18.0986696230599
903 18.1173864894795
904 18.0973451327434
905 18.0917127071823
906 18.1114790286976
907 18.1014332965821
908 18.0881057268722
909 18.0880088008801
910 18.0956043956044
911 18.0867178924259
912 18.1041666666667
913 18.1161007667032
914 18.1269146608315
915 18.1355191256831
916 18.1484716157205
917 18.1570338058888
918 18.1644880174292
919 18.1545157780196
920 18.1521739130435
921 18.1400651465798
922 18.1496746203905
923 18.1462621885157
924 18.1569264069264
925 18.1372972972973
926 18.1263498920086
927 18.1391585760518
928 18.1433189655172
929 18.1270182992465
930 18.1268817204301
931 18.1278195488722
932 18.1330472103004
933 18.1296891747053
934 18.1359743040685
935 18.1347593582888
936 18.1153846153846
937 18.1205976520811
938 18.1012793176972
939 18.1139510117146
940 18.1212765957447
941 18.1264612114772
942 18.135881104034
943 18.1410392364793
944 18.1292372881356
945 18.1470899470899
946 18.1553911205074
947 18.1605068637804
948 18.1413502109705
949 18.145416227608
950 18.1547368421053
951 18.1409043112513
952 18.1313025210084
953 18.1196222455404
954 18.1341719077568
955 18.1434554973822
956 18.1589958158996
957 18.1556948798328
958 18.1409185803758
959 18.1491136600626
960 18.153125
961 18.1415192507804
962 18.1288981288981
963 18.1443406022845
964 18.1296680497925
965 18.1471502590674
966 18.1501035196687
967 18.1613236814891
968 18.146694214876
969 18.1341589267286
970 18.1268041237113
971 18.1318228630278
972 18.1460905349794
973 18.1623843782117
974 18.1519507186858
975 18.1415384615385
976 18.156762295082
977 18.1637666325486
978 18.1513292433538
979 18.1562819203269
980 18.1428571428571
981 18.1437308868502
982 18.1568228105906
983 18.1576805696846
984 18.145325203252
985 18.1583756345178
986 18.1643002028398
987 18.1691995947315
988 18.1791497975709
989 18.1830131445905
990 18.1939393939394
991 18.1836528758829
992 18.1774193548387
993 18.1893252769386
994 18.1871227364185
995 18.1809045226131
996 18.1907630522088
997 18.1815446339017
998 18.1813627254509
999 18.1951951951952
1000 18.196
};
\addplot [semithick, color5, forget plot]
table {%
1 21
2 28
3 26.3333333333333
4 20
5 22.8
6 24.6666666666667
7 25.8571428571429
8 25.75
9 26.6666666666667
10 27
11 27.1818181818182
12 24.9166666666667
13 25
14 23.7857142857143
15 22.2666666666667
16 21.5625
17 20.3529411764706
18 19.6111111111111
19 19.4736842105263
20 18.7
21 18.047619047619
22 18.6818181818182
23 18
24 18.625
25 18.64
26 19.2692307692308
27 19.5925925925926
28 19.25
29 19.7931034482759
30 20.3333333333333
31 20.0322580645161
32 19.8125
33 19.9090909090909
34 19.5
35 19.4857142857143
36 19.4444444444444
37 19.1891891891892
38 19.0263157894737
39 19.2820512820513
40 19.175
41 19.4634146341463
42 19.5238095238095
43 19.4651162790698
44 19.0909090909091
45 19.3777777777778
46 19.695652173913
47 19.9574468085106
48 20
49 19.8979591836735
50 20.12
51 20.4117647058824
52 20.6153846153846
53 20.6603773584906
54 20.6111111111111
55 20.5454545454545
56 20.2857142857143
57 20.3684210526316
58 20.4310344827586
59 20.1186440677966
60 20.2666666666667
61 20.0983606557377
62 19.7741935483871
63 19.8888888888889
64 19.78125
65 19.5230769230769
66 19.5757575757576
67 19.6119402985075
68 19.4705882352941
69 19.2463768115942
70 19.0285714285714
71 18.9718309859155
72 18.7083333333333
73 18.6438356164384
74 18.4054054054054
75 18.28
76 18.1710526315789
77 18.2857142857143
78 18.1153846153846
79 18.0126582278481
80 17.7875
81 17.9259259259259
82 17.9634146341463
83 18.1686746987952
84 18.1547619047619
85 17.9764705882353
86 18.0348837209302
87 18
88 17.875
89 17.7078651685393
90 17.9111111111111
91 17.7912087912088
92 17.6086956521739
93 17.6774193548387
94 17.5
95 17.4526315789474
96 17.5208333333333
97 17.5051546391753
98 17.6020408163265
99 17.7272727272727
100 17.67
101 17.5049504950495
102 17.3823529411765
103 17.5436893203883
104 17.4615384615385
105 17.4666666666667
106 17.5188679245283
107 17.4672897196262
108 17.5277777777778
109 17.697247706422
110 17.7
111 17.6936936936937
112 17.8571428571429
113 17.8495575221239
114 17.9298245614035
115 17.8521739130435
116 17.7931034482759
117 17.7264957264957
118 17.5762711864407
119 17.5294117647059
120 17.6083333333333
121 17.6859504132231
122 17.6229508196721
123 17.5365853658537
124 17.6854838709677
125 17.544
126 17.4365079365079
127 17.4488188976378
128 17.3671875
129 17.3178294573643
130 17.3615384615385
131 17.2900763358779
132 17.2878787878788
133 17.203007518797
134 17.0970149253731
135 17.1333333333333
136 17.0882352941176
137 17.0729927007299
138 17.1594202898551
139 17.1942446043165
140 17.1714285714286
141 17.1276595744681
142 17.1760563380282
143 17.0629370629371
144 17.1111111111111
145 17.0275862068966
146 17.0958904109589
147 16.9863945578231
148 17.0675675675676
149 17.0805369127517
150 17.2066666666667
151 17.2384105960265
152 17.2302631578947
153 17.2287581699346
154 17.1753246753247
155 17.2838709677419
156 17.2884615384615
157 17.1974522292994
158 17.2278481012658
159 17.2201257861635
160 17.175
161 17.1614906832298
162 17.141975308642
163 17.079754601227
164 17.0243902439024
165 17.0727272727273
166 17.0963855421687
167 17.0179640718563
168 17.0416666666667
169 17.0650887573964
170 17.0058823529412
171 17.1169590643275
172 17.1511627906977
173 17.0635838150289
174 17.0574712643678
175 17.0285714285714
176 17.0454545454545
177 17.0960451977401
178 17.0730337078652
179 17.0614525139665
180 17.1277777777778
181 17.1491712707182
182 17.0549450549451
183 17.0382513661202
184 17.054347826087
185 16.9837837837838
186 16.9139784946237
187 17
188 16.9734042553191
189 16.9259259259259
190 16.9052631578947
191 16.9214659685864
192 16.9270833333333
193 16.9896373056995
194 17.0309278350515
195 17.0871794871795
196 17.1581632653061
197 17.253807106599
198 17.3232323232323
199 17.3266331658291
200 17.39
201 17.4676616915423
202 17.5
203 17.4581280788177
204 17.421568627451
205 17.3365853658537
206 17.373786407767
207 17.3768115942029
208 17.3076923076923
209 17.3301435406699
210 17.3761904761905
211 17.303317535545
212 17.3254716981132
213 17.4037558685446
214 17.3831775700935
215 17.4651162790698
216 17.4722222222222
217 17.4331797235023
218 17.4220183486239
219 17.4109589041096
220 17.3863636363636
221 17.3212669683258
222 17.4009009009009
223 17.3408071748879
224 17.3348214285714
225 17.4
226 17.3982300884956
227 17.4449339207048
228 17.4649122807018
229 17.4672489082969
230 17.4391304347826
231 17.4891774891775
232 17.5215517241379
233 17.4892703862661
234 17.5042735042735
235 17.531914893617
236 17.6016949152542
237 17.6582278481013
238 17.6806722689076
239 17.6485355648536
240 17.6791666666667
241 17.7261410788382
242 17.7438016528926
243 17.7818930041152
244 17.8114754098361
245 17.8204081632653
246 17.8414634146341
247 17.7894736842105
248 17.7822580645161
249 17.7991967871486
250 17.8
251 17.7450199203187
252 17.797619047619
253 17.8300395256917
254 17.7716535433071
255 17.8313725490196
256 17.8671875
257 17.8832684824903
258 17.8759689922481
259 17.8880308880309
260 17.9269230769231
261 17.911877394636
262 17.9656488549618
263 17.9049429657795
264 17.875
265 17.9207547169811
266 17.984962406015
267 18.0524344569288
268 18.0111940298507
269 17.9591078066914
270 17.9074074074074
271 17.9594095940959
272 17.9191176470588
273 17.9047619047619
274 17.8795620437956
275 17.9454545454545
276 17.9746376811594
277 17.9819494584838
278 17.9424460431655
279 17.9139784946237
280 17.8714285714286
281 17.846975088968
282 17.9042553191489
283 17.8480565371025
284 17.8908450704225
285 17.9263157894737
286 17.9055944055944
287 17.8745644599303
288 17.8472222222222
289 17.8685121107266
290 17.8206896551724
291 17.8316151202749
292 17.8527397260274
293 17.839590443686
294 17.8571428571429
295 17.9152542372881
296 17.8851351351351
297 17.9225589225589
298 17.8624161073826
299 17.8996655518395
300 17.8666666666667
301 17.8837209302326
302 17.8907284768212
303 17.8316831683168
304 17.8322368421053
305 17.8918032786885
306 17.8790849673203
307 17.8599348534202
308 17.9058441558442
309 17.8770226537217
310 17.8677419354839
311 17.8745980707395
312 17.8621794871795
313 17.8849840255591
314 17.8917197452229
315 17.8634920634921
316 17.8955696202532
317 17.8738170347003
318 17.9150943396226
319 17.8965517241379
320 17.94375
321 17.9470404984424
322 17.8975155279503
323 17.8699690402477
324 17.8827160493827
325 17.9107692307692
326 17.8957055214724
327 17.9113149847095
328 17.9573170731707
329 17.9483282674772
330 17.9787878787879
331 18.0271903323263
332 17.9879518072289
333 18.036036036036
334 18.0149700598802
335 18.0208955223881
336 18.0327380952381
337 18.0356083086053
338 18.0532544378698
339 18.0412979351032
340 18.0147058823529
341 18.058651026393
342 18.0497076023392
343 18.067055393586
344 18.0348837209302
345 18.0115942028986
346 18.0260115606936
347 17.9740634005764
348 17.9310344827586
349 17.9713467048711
350 18.0171428571429
351 17.968660968661
352 18.0028409090909
353 18.0368271954674
354 18.0875706214689
355 18.0450704225352
356 18.0477528089888
357 18.0196078431373
358 18.0307262569832
359 18.066852367688
360 18.0555555555556
361 18.016620498615
362 17.975138121547
363 17.9586776859504
364 17.9532967032967
365 17.9534246575342
366 17.9426229508197
367 17.9100817438692
368 17.9130434782609
369 17.9430894308943
370 17.9135135135135
371 17.8679245283019
372 17.8951612903226
373 17.8981233243968
374 17.8529411764706
375 17.88
376 17.8351063829787
377 17.7931034482759
378 17.7936507936508
379 17.8364116094987
380 17.8026315789474
381 17.7742782152231
382 17.7486910994764
383 17.7023498694517
384 17.6796875
385 17.6805194805195
386 17.6735751295337
387 17.6976744186047
388 17.7010309278351
389 17.7017994858612
390 17.7102564102564
391 17.7570332480818
392 17.7117346938776
393 17.7302798982188
394 17.7131979695431
395 17.6936708860759
396 17.6742424242424
397 17.6372795969773
398 17.678391959799
399 17.6842105263158
400 17.64
401 17.6334164588529
402 17.636815920398
403 17.6178660049628
404 17.6287128712871
405 17.6
406 17.6428571428571
407 17.6142506142506
408 17.5857843137255
409 17.6088019559902
410 17.6487804878049
411 17.6155717761557
412 17.5946601941748
413 17.5617433414044
414 17.5797101449275
415 17.6048192771084
416 17.6225961538462
417 17.6282973621103
418 17.6076555023923
419 17.6252983293556
420 17.6142857142857
421 17.5819477434679
422 17.5805687203791
423 17.6099290780142
424 17.6320754716981
425 17.6235294117647
426 17.6431924882629
427 17.6463700234192
428 17.6051401869159
429 17.5850815850816
430 17.5581395348837
431 17.5800464037123
432 17.5810185185185
433 17.5889145496536
434 17.5806451612903
435 17.5402298850575
436 17.5
437 17.5263157894737
438 17.5182648401826
439 17.4897494305239
440 17.5204545454545
441 17.5442176870748
442 17.5859728506787
443 17.6207674943567
444 17.6599099099099
445 17.6539325842697
446 17.6502242152466
447 17.6241610738255
448 17.6540178571429
449 17.6347438752784
450 17.64
451 17.6274944567628
452 17.6216814159292
453 17.6534216335541
454 17.625550660793
455 17.6131868131868
456 17.6491228070175
457 17.6849015317287
458 17.6768558951965
459 17.681917211329
460 17.7
461 17.7396963123644
462 17.7575757575758
463 17.7861771058315
464 17.7974137931034
465 17.7849462365591
466 17.7618025751073
467 17.7558886509636
468 17.767094017094
469 17.7547974413646
470 17.768085106383
471 17.7303609341826
472 17.7309322033898
473 17.7315010570825
474 17.710970464135
475 17.7031578947368
476 17.672268907563
477 17.643605870021
478 17.6464435146443
479 17.643006263048
480 17.65
481 17.6340956340956
482 17.6390041493776
483 17.6501035196687
484 17.6177685950413
485 17.6494845360825
486 17.6399176954733
487 17.6344969199179
488 17.6147540983607
489 17.601226993865
490 17.569387755102
491 17.5845213849287
492 17.5711382113821
493 17.5983772819473
494 17.6012145748988
495 17.6343434343434
496 17.6491935483871
497 17.6680080482897
498 17.6666666666667
499 17.6472945891784
500 17.658
501 17.6946107784431
502 17.6752988047809
503 17.6938369781312
504 17.7281746031746
505 17.7326732673267
506 17.7371541501976
507 17.7554240631164
508 17.744094488189
509 17.7347740667976
510 17.7176470588235
511 17.7260273972603
512 17.759765625
513 17.7543859649123
514 17.727626459144
515 17.7126213592233
516 17.7170542635659
517 17.7098646034816
518 17.7104247104247
519 17.7321772639692
520 17.7288461538462
521 17.7293666026871
522 17.7298850574713
523 17.7456978967495
524 17.7690839694656
525 17.7638095238095
526 17.754752851711
527 17.7514231499051
528 17.7386363636364
529 17.7183364839319
530 17.7471698113208
531 17.7231638418079
532 17.7161654135338
533 17.6829268292683
534 17.7153558052434
535 17.7196261682243
536 17.740671641791
537 17.7616387337058
538 17.7342007434944
539 17.7142857142857
540 17.6833333333333
541 17.6524953789279
542 17.6309963099631
543 17.6243093922652
544 17.6084558823529
545 17.6330275229358
546 17.6007326007326
547 17.6142595978062
548 17.5839416058394
549 17.5956284153005
550 17.6090909090909
551 17.5898366606171
552 17.588768115942
553 17.6057866184448
554 17.629963898917
555 17.6018018018018
556 17.6330935251799
557 17.6499102333932
558 17.6487455197133
559 17.6386404293381
560 17.6357142857143
561 17.6149732620321
562 17.6120996441281
563 17.5914742451155
564 17.6081560283688
565 17.5840707964602
566 17.6024734982332
567 17.5767195767196
568 17.5739436619718
569 17.5764499121265
570 17.5456140350877
571 17.5323992994746
572 17.5559440559441
573 17.5392670157068
574 17.5104529616725
575 17.4973913043478
576 17.4809027777778
577 17.5025996533795
578 17.4723183391003
579 17.4766839378238
580 17.4724137931034
581 17.4664371772806
582 17.4828178694158
583 17.5145797598628
584 17.5154109589041
585 17.5094017094017
586 17.481228668942
587 17.4701873935264
588 17.4795918367347
589 17.4499151103565
590 17.4610169491525
591 17.4906937394247
592 17.4695945945946
593 17.4704890387858
594 17.4478114478114
595 17.4773109243697
596 17.4714765100671
597 17.499162479062
598 17.5150501672241
599 17.5242070116861
600 17.5516666666667
601 17.5540765391015
602 17.5681063122924
603 17.5820895522388
604 17.5745033112583
605 17.6049586776859
606 17.5973597359736
607 17.5831960461285
608 17.5855263157895
609 17.5730706075534
610 17.5672131147541
611 17.5450081833061
612 17.5539215686275
613 17.5481239804241
614 17.5211726384365
615 17.5414634146341
616 17.5600649350649
617 17.5769854132901
618 17.5711974110032
619 17.562197092084
620 17.5903225806452
621 17.5877616747182
622 17.5707395498392
623 17.5762439807384
624 17.5977564102564
625 17.5888
626 17.611821086262
627 17.591706539075
628 17.6210191082803
629 17.6152623211447
630 17.615873015873
631 17.5879556259905
632 17.5712025316456
633 17.6003159557662
634 17.5867507886435
635 17.5748031496063
636 17.5676100628931
637 17.5871271585557
638 17.5987460815047
639 17.5978090766823
640 17.571875
641 17.588143525741
642 17.6168224299065
643 17.6314152410575
644 17.6164596273292
645 17.6217054263566
646 17.640866873065
647 17.613601236476
648 17.6064814814815
649 17.6055469953775
650 17.6030769230769
651 17.6190476190476
652 17.6472392638037
653 17.646248085758
654 17.6345565749235
655 17.6320610687023
656 17.6524390243902
657 17.6286149162861
658 17.6109422492401
659 17.6069802731411
660 17.6318181818182
661 17.6399394856278
662 17.6238670694864
663 17.6168929110106
664 17.6009036144578
665 17.5924812030075
666 17.6096096096096
667 17.6221889055472
668 17.6362275449102
669 17.6233183856502
670 17.6149253731343
671 17.6110283159463
672 17.5967261904762
673 17.5839524517088
674 17.6083086053412
675 17.5822222222222
676 17.5724852071006
677 17.5937961595273
678 17.6106194690265
679 17.6038291605302
680 17.5911764705882
681 17.6079295154185
682 17.599706744868
683 17.6207906295754
684 17.624269005848
685 17.607299270073
686 17.6224489795918
687 17.6113537117904
688 17.6235465116279
689 17.611030478955
690 17.5913043478261
691 17.5716353111433
692 17.5765895953757
693 17.5815295815296
694 17.5634005763689
695 17.589928057554
696 17.6063218390805
697 17.6284074605452
698 17.6260744985673
699 17.6194563662375
700 17.64
701 17.6590584878745
702 17.6680911680912
703 17.6827880512091
704 17.671875
705 17.6595744680851
706 17.6586402266289
707 17.6577086280057
708 17.635593220339
709 17.6488011283498
710 17.6507042253521
711 17.6413502109705
712 17.6292134831461
713 17.6549789621318
714 17.6358543417367
715 17.6391608391608
716 17.627094972067
717 17.6248256624826
718 17.6114206128134
719 17.6008344923505
720 17.6194444444444
721 17.6338418862691
722 17.6592797783934
723 17.6528354080221
724 17.6367403314917
725 17.6344827586207
726 17.6322314049587
727 17.6286107290234
728 17.6222527472527
729 17.636488340192
730 17.6123287671233
731 17.6279069767442
732 17.6379781420765
733 17.656207366985
734 17.6471389645777
735 17.6312925170068
736 17.6548913043478
737 17.6675712347354
738 17.6639566395664
739 17.6454668470907
740 17.6337837837838
741 17.6315789473684
742 17.6401617250674
743 17.6635262449529
744 17.6478494623656
745 17.6577181208054
746 17.6635388739946
747 17.6720214190094
748 17.6483957219251
749 17.6608811748999
750 17.6426666666667
751 17.6657789613848
752 17.688829787234
753 17.6958831341301
754 17.6909814323607
755 17.6927152317881
756 17.7156084656085
757 17.7173051519155
758 17.7018469656992
759 17.6982872200264
760 17.7210526315789
761 17.7214191852825
762 17.7204724409449
763 17.7090432503277
764 17.7225130890052
765 17.7006535947712
766 17.7062663185379
767 17.6962190352021
768 17.7200520833333
769 17.7009102730819
770 17.7051948051948
771 17.715953307393
772 17.7085492227979
773 17.732212160414
774 17.7519379844961
775 17.7690322580645
776 17.7538659793814
777 17.7760617760618
778 17.7866323907455
779 17.7663671373556
780 17.7717948717949
781 17.7567221510883
782 17.7468030690537
783 17.735632183908
784 17.7474489795918
785 17.7630573248408
786 17.7760814249364
787 17.7979669631512
788 17.8147208121827
789 17.7921419518378
790 17.8025316455696
791 17.8242730720607
792 17.8282828282828
793 17.8247162673392
794 17.8274559193955
795 17.8088050314465
796 17.8015075376884
797 17.801756587202
798 17.8020050125313
799 17.8235294117647
800 17.8425
801 17.8451935081149
802 17.8653366583541
803 17.8717310087173
804 17.8905472636816
805 17.8770186335404
806 17.8585607940447
807 17.8550185873606
808 17.8589108910891
809 17.8504326328801
810 17.8395061728395
811 17.8594327990136
812 17.871921182266
813 17.8610086100861
814 17.8452088452088
815 17.8613496932515
816 17.8823529411765
817 17.8776009791922
818 17.8740831295844
819 17.8681318681319
820 17.8487804878049
821 17.8355663824604
822 17.845498783455
823 17.8262454434994
824 17.8300970873786
825 17.8157575757576
826 17.8050847457627
827 17.8089480048368
828 17.8272946859903
829 17.8407720144753
830 17.8469879518072
831 17.8315282791817
832 17.8509615384615
833 17.8703481392557
834 17.8597122302158
835 17.8610778443114
836 17.8421052631579
837 17.8291517323775
838 17.8198090692124
839 17.8152562574493
840 17.8095238095238
841 17.7883472057075
842 17.7767220902613
843 17.7651245551601
844 17.7642180094787
845 17.7633136094675
846 17.7718676122931
847 17.7804014167651
848 17.7830188679245
849 17.8021201413428
850 17.8058823529412
851 17.813160987074
852 17.8298122065728
853 17.8241500586166
854 17.8395784543326
855 17.8187134502924
856 17.8364485981308
857 17.8448074679113
858 17.8659673659674
859 17.8509895227008
860 17.8511627906977
861 17.8304297328688
862 17.8329466357309
863 17.8470451911935
864 17.8518518518519
865 17.8682080924855
866 17.8521939953811
867 17.8419838523645
868 17.8617511520737
869 17.8515535097814
870 17.8425287356322
871 17.8266360505166
872 17.8119266055046
873 17.8167239404353
874 17.8169336384439
875 17.8068571428571
876 17.7945205479452
877 17.7981755986317
878 17.7870159453303
879 17.773606370876
880 17.7795454545455
881 17.7786606129398
882 17.7936507936508
883 17.8052095130238
884 17.7941176470588
885 17.8011299435028
886 17.7821670428894
887 17.7733934611048
888 17.768018018018
889 17.7514060742407
890 17.7314606741573
891 17.7171717171717
892 17.737668161435
893 17.7346024636058
894 17.7326621923937
895 17.7206703910615
896 17.71875
897 17.7391304347826
898 17.7349665924276
899 17.7285873192436
900 17.7144444444444
901 17.7225305216426
902 17.7206208425721
903 17.718715393134
904 17.7201327433628
905 17.7127071823204
906 17.719646799117
907 17.7364939360529
908 17.7455947136564
909 17.7381738173817
910 17.7461538461538
911 17.7431394072448
912 17.7609649122807
913 17.7437020810515
914 17.7560175054705
915 17.7661202185792
916 17.7489082969432
917 17.7382769901854
918 17.7254901960784
919 17.727965179543
920 17.7434782608696
921 17.756786102063
922 17.7603036876356
923 17.7529794149512
924 17.7467532467532
925 17.7578378378378
926 17.7537796976242
927 17.7637540453074
928 17.7553879310345
929 17.7395048439182
930 17.7344086021505
931 17.7411385606874
932 17.730686695279
933 17.7363344051447
934 17.7237687366167
935 17.7251336898396
936 17.7403846153846
937 17.7513340448239
938 17.7388059701493
939 17.7444089456869
940 17.7404255319149
941 17.7343251859724
942 17.7261146496815
943 17.7306468716861
944 17.7468220338983
945 17.7544973544974
946 17.7515856236786
947 17.7645195353749
948 17.7742616033755
949 17.7608008429926
950 17.7663157894737
951 17.7770767613039
952 17.7741596638655
953 17.7932843651626
954 17.7798742138365
955 17.7979057591623
956 17.8054393305439
957 17.8150470219436
958 17.7964509394572
959 17.7977059436913
960 17.784375
961 17.7689906347555
962 17.7629937629938
963 17.7715472481828
964 17.753112033195
965 17.7709844559585
966 17.7857142857143
967 17.8045501551189
968 17.8150826446281
969 17.8028895768834
970 17.8072164948454
971 17.7981462409887
972 17.8055555555556
973 17.8191161356629
974 17.8069815195072
975 17.8082051282051
976 17.7981557377049
977 17.8045035823951
978 17.7985685071575
979 17.7814096016343
980 17.7918367346939
981 17.8012232415902
982 17.806517311609
983 17.7884028484232
984 17.7774390243902
985 17.7939086294416
986 17.7758620689655
987 17.7781155015198
988 17.7904858299595
989 17.7724974721941
990 17.7545454545455
991 17.7467204843592
992 17.7298387096774
993 17.7432024169184
994 17.7344064386318
995 17.7306532663317
996 17.7489959839357
997 17.765295887663
998 17.749498997996
999 17.7357357357357
1000 17.728
};
\addplot [semithick, color6, forget plot]
table {%
1 8
2 16.5
3 17.3333333333333
4 18
5 16.2
6 14.1666666666667
7 15.2857142857143
8 16.375
9 18.1111111111111
10 17.5
11 18.1818181818182
12 16.9166666666667
13 16.3076923076923
14 17.2857142857143
15 16.8
16 15.875
17 15.4117647058824
18 16.1111111111111
19 15.8947368421053
20 16.4
21 17.1428571428571
22 16.5454545454545
23 16.2608695652174
24 15.5833333333333
25 15.72
26 16.5
27 17.1851851851852
28 17.5714285714286
29 17.551724137931
30 17.9666666666667
31 18.4838709677419
32 18.875
33 18.7575757575758
34 19.2352941176471
35 18.9428571428571
36 18.6111111111111
37 18.4864864864865
38 18.6315789473684
39 18.6410256410256
40 18.275
41 18.4390243902439
42 18.5952380952381
43 18.5581395348837
44 18.2727272727273
45 17.9333333333333
46 18.1304347826087
47 17.9787234042553
48 18.0625
49 18.0612244897959
50 18.14
51 18.4509803921569
52 18.2115384615385
53 18.1698113207547
54 17.8888888888889
55 18.0181818181818
56 18.2142857142857
57 18.3157894736842
58 18.448275862069
59 18.5423728813559
60 18.5
61 18.344262295082
62 18.0967741935484
63 18.3650793650794
64 18.3125
65 18.3076923076923
66 18.0606060606061
67 18.2985074626866
68 18.3823529411765
69 18.6086956521739
70 18.4571428571429
71 18.1971830985915
72 18.3194444444444
73 18.0821917808219
74 17.972972972973
75 17.7333333333333
76 17.7894736842105
77 17.6103896103896
78 17.7692307692308
79 17.9620253164557
80 18.1625
81 18.3827160493827
82 18.1707317073171
83 18.3132530120482
84 18.4642857142857
85 18.3529411764706
86 18.4883720930233
87 18.3563218390805
88 18.3068181818182
89 18.247191011236
90 18.3777777777778
91 18.4175824175824
92 18.5326086956522
93 18.4193548387097
94 18.3510638297872
95 18.2526315789474
96 18.3645833333333
97 18.4742268041237
98 18.4183673469388
99 18.5050505050505
100 18.47
101 18.3960396039604
102 18.2941176470588
103 18.3592233009709
104 18.5
105 18.447619047619
106 18.4528301886792
107 18.5981308411215
108 18.6481481481481
109 18.605504587156
110 18.5181818181818
111 18.5045045045045
112 18.5803571428571
113 18.5929203539823
114 18.640350877193
115 18.7826086956522
116 18.698275862069
117 18.5811965811966
118 18.635593220339
119 18.7563025210084
120 18.6916666666667
121 18.5619834710744
122 18.5737704918033
123 18.4227642276423
124 18.5645161290323
125 18.592
126 18.5952380952381
127 18.5354330708661
128 18.5703125
129 18.6821705426357
130 18.6307692307692
131 18.6488549618321
132 18.7651515151515
133 18.781954887218
134 18.9029850746269
135 18.8148148148148
136 18.8235294117647
137 18.8175182481752
138 18.695652173913
139 18.6043165467626
140 18.7214285714286
141 18.8297872340426
142 18.9366197183099
143 18.8671328671329
144 18.9791666666667
145 18.848275862069
146 18.8287671232877
147 18.9455782312925
148 19.0472972972973
149 18.9731543624161
150 19.02
151 19.0993377483444
152 19.2039473684211
153 19.0915032679739
154 19.1168831168831
155 19.2
156 19.2628205128205
157 19.3694267515924
158 19.4177215189873
159 19.4025157232704
160 19.3125
161 19.2732919254658
162 19.2407407407407
163 19.2208588957055
164 19.2621951219512
165 19.2060606060606
166 19.0903614457831
167 19.0119760479042
168 18.9285714285714
169 18.9171597633136
170 18.9235294117647
171 18.8888888888889
172 18.9360465116279
173 18.9017341040462
174 18.8850574712644
175 18.8742857142857
176 18.9090909090909
177 18.9039548022599
178 18.9101123595506
179 18.9832402234637
180 19.0555555555556
181 19.1325966850829
182 19.0879120879121
183 19.0983606557377
184 19.1630434782609
185 19.1675675675676
186 19.1612903225806
187 19.2245989304813
188 19.218085106383
189 19.2222222222222
190 19.2368421052632
191 19.3089005235602
192 19.2395833333333
193 19.2849740932642
194 19.2835051546392
195 19.2102564102564
196 19.1377551020408
197 19.0761421319797
198 19.0505050505051
199 19.0653266331658
200 19.12
201 19.0298507462687
202 19.0742574257426
203 19.0591133004926
204 18.9803921568627
205 19.009756097561
206 18.9466019417476
207 19
208 19.0144230769231
209 18.9712918660287
210 18.9666666666667
211 18.8957345971564
212 18.8679245283019
213 18.8262910798122
214 18.7663551401869
215 18.846511627907
216 18.8611111111111
217 18.8801843317972
218 18.8990825688073
219 18.8538812785388
220 18.7681818181818
221 18.7963800904977
222 18.8243243243243
223 18.8789237668161
224 18.9553571428571
225 18.9866666666667
226 18.9247787610619
227 18.8458149779736
228 18.9078947368421
229 18.9650655021834
230 18.9391304347826
231 18.987012987013
232 18.9741379310345
233 18.9356223175966
234 18.8632478632479
235 18.9021276595745
236 18.9110169491525
237 18.8565400843882
238 18.8781512605042
239 18.8619246861925
240 18.8375
241 18.8838174273859
242 18.9338842975207
243 18.917695473251
244 18.8852459016393
245 18.869387755102
246 18.9186991869919
247 18.8704453441296
248 18.9395161290323
249 18.9759036144578
250 18.976
251 19.00796812749
252 18.9444444444444
253 18.99604743083
254 19.0354330708661
255 19.0901960784314
256 19.046875
257 18.9844357976654
258 18.9806201550388
259 18.992277992278
260 18.9346153846154
261 18.9233716475096
262 18.8931297709924
263 18.9239543726236
264 18.8901515151515
265 18.8830188679245
266 18.8872180451128
267 18.9026217228464
268 18.8955223880597
269 18.9219330855019
270 18.9851851851852
271 18.9188191881919
272 18.9080882352941
273 18.9120879120879
274 18.8467153284672
275 18.8981818181818
276 18.945652173913
277 18.9205776173285
278 18.863309352518
279 18.8351254480287
280 18.7714285714286
281 18.7259786476868
282 18.7517730496454
283 18.7279151943463
284 18.6619718309859
285 18.7087719298246
286 18.7552447552448
287 18.8048780487805
288 18.7465277777778
289 18.681660899654
290 18.7206896551724
291 18.6872852233677
292 18.6952054794521
293 18.7030716723549
294 18.6632653061224
295 18.6915254237288
296 18.6925675675676
297 18.7239057239057
298 18.7651006711409
299 18.819397993311
300 18.7933333333333
301 18.8006644518272
302 18.817880794702
303 18.8712871287129
304 18.9013157894737
305 18.8688524590164
306 18.8137254901961
307 18.8403908794788
308 18.8116883116883
309 18.7540453074434
310 18.7677419354839
311 18.7427652733119
312 18.7435897435897
313 18.7348242811502
314 18.7165605095541
315 18.6920634920635
316 18.6898734177215
317 18.6845425867508
318 18.6949685534591
319 18.6614420062696
320 18.653125
321 18.6542056074766
322 18.611801242236
323 18.5572755417957
324 18.537037037037
325 18.5076923076923
326 18.5398773006135
327 18.519877675841
328 18.5731707317073
329 18.5927051671733
330 18.5787878787879
331 18.607250755287
332 18.5692771084337
333 18.5645645645646
334 18.5568862275449
335 18.5671641791045
336 18.5446428571429
337 18.5252225519288
338 18.4733727810651
339 18.4660766961652
340 18.4764705882353
341 18.4604105571848
342 18.4619883040936
343 18.4169096209913
344 18.4331395348837
345 18.4289855072464
346 18.3988439306358
347 18.3631123919308
348 18.3534482758621
349 18.3094555873925
350 18.3428571428571
351 18.3447293447293
352 18.3522727272727
353 18.342776203966
354 18.3757062146893
355 18.4056338028169
356 18.4466292134831
357 18.4313725490196
358 18.4217877094972
359 18.4345403899721
360 18.4388888888889
361 18.4819944598338
362 18.4337016574586
363 18.4490358126722
364 18.4340659340659
365 18.3972602739726
366 18.3551912568306
367 18.3242506811989
368 18.2771739130435
369 18.2764227642276
370 18.272972972973
371 18.2533692722372
372 18.239247311828
373 18.2466487935657
374 18.2299465240642
375 18.1813333333333
376 18.1648936170213
377 18.1273209549072
378 18.0952380952381
379 18.1398416886544
380 18.1868421052632
381 18.1522309711286
382 18.1780104712042
383 18.1618798955614
384 18.1822916666667
385 18.161038961039
386 18.1217616580311
387 18.1550387596899
388 18.1649484536082
389 18.1696658097686
390 18.1974358974359
391 18.1764705882353
392 18.1428571428571
393 18.1094147582697
394 18.1065989847716
395 18.1493670886076
396 18.1919191919192
397 18.1738035264484
398 18.1331658291457
399 18.1278195488722
400 18.165
401 18.1645885286783
402 18.1393034825871
403 18.1091811414392
404 18.1262376237624
405 18.1407407407407
406 18.1379310344828
407 18.1179361179361
408 18.0857843137255
409 18.0904645476773
410 18.0731707317073
411 18.036496350365
412 18.0461165048544
413 18.0605326876513
414 18.0797101449275
415 18.0915662650602
416 18.0721153846154
417 18.0887290167866
418 18.0980861244019
419 18.1002386634845
420 18.1309523809524
421 18.1520190023753
422 18.1255924170616
423 18.1347517730496
424 18.1721698113208
425 18.2070588235294
426 18.1924882629108
427 18.1686182669789
428 18.1542056074766
429 18.1888111888112
430 18.2279069767442
431 18.199535962877
432 18.2037037037037
433 18.2101616628176
434 18.2188940092166
435 18.2206896551724
436 18.2247706422018
437 18.1853546910755
438 18.1438356164384
439 18.1640091116173
440 18.175
441 18.2154195011338
442 18.2443438914027
443 18.2505643340858
444 18.2387387387387
445 18.2359550561798
446 18.2511210762332
447 18.2170022371365
448 18.2098214285714
449 18.2204899777283
450 18.1911111111111
451 18.1707317073171
452 18.1570796460177
453 18.1898454746137
454 18.1607929515419
455 18.1428571428571
456 18.140350877193
457 18.1400437636762
458 18.1135371179039
459 18.1220043572985
460 18.104347826087
461 18.0802603036876
462 18.0714285714286
463 18.0907127429806
464 18.1271551724138
465 18.1591397849462
466 18.1673819742489
467 18.1584582441113
468 18.1410256410256
469 18.1769722814499
470 18.2106382978723
471 18.2101910828025
472 18.2033898305085
473 18.1670190274841
474 18.1350210970464
475 18.0989473684211
476 18.109243697479
477 18.0922431865828
478 18.0564853556485
479 18.0688935281837
480 18.05625
481 18.039501039501
482 18.0560165975104
483 18.0434782608696
484 18.0392561983471
485 18.0350515463918
486 18.0658436213992
487 18.0759753593429
488 18.0450819672131
489 18.0613496932515
490 18.0734693877551
491 18.071283095723
492 18.0873983739837
493 18.079107505071
494 18.1012145748988
495 18.0646464646465
496 18.0302419354839
497 18.0160965794769
498 18.0160642570281
499 18
500 17.972
501 17.9560878243513
502 17.9482071713147
503 17.9821073558648
504 17.9920634920635
505 18.0217821782178
506 18.0395256916996
507 18.0157790927022
508 17.9803149606299
509 17.9823182711198
510 17.9666666666667
511 17.9510763209393
512 17.9453125
513 17.9746588693957
514 17.9688715953307
515 17.947572815534
516 17.9728682170543
517 17.972920696325
518 18.007722007722
519 18.0423892100193
520 18.0134615384615
521 18.0403071017274
522 18.0651340996169
523 18.0325047801147
524 18.0209923664122
525 18.007619047619
526 17.9847908745247
527 17.9848197343454
528 17.9791666666667
529 17.9735349716446
530 18.0056603773585
531 17.9792843691149
532 18.0093984962406
533 17.9868667917448
534 17.9606741573034
535 17.985046728972
536 17.9794776119403
537 17.9739292364991
538 17.9776951672862
539 17.9795918367347
540 18.0037037037037
541 17.9944547134935
542 18.0055350553506
543 17.9815837937385
544 17.9540441176471
545 17.9816513761468
546 17.9615384615385
547 17.945155393053
548 17.9324817518248
549 17.9071038251366
550 17.9054545454545
551 17.8729582577132
552 17.8677536231884
553 17.8462929475588
554 17.826714801444
555 17.8126126126126
556 17.8057553956835
557 17.7773788150808
558 17.7508960573477
559 17.7388193202147
560 17.7625
561 17.7860962566845
562 17.788256227758
563 17.7815275310835
564 17.7712765957447
565 17.7769911504425
566 17.7773851590106
567 17.8077601410935
568 17.8221830985915
569 17.792618629174
570 17.8035087719298
571 17.816112084063
572 17.8356643356643
573 17.8446771378709
574 17.8693379790941
575 17.8678260869565
576 17.8732638888889
577 17.8509532062392
578 17.856401384083
579 17.839378238342
580 17.8568965517241
581 17.8623063683305
582 17.8333333333333
583 17.8456260720412
584 17.847602739726
585 17.8410256410256
586 17.8532423208191
587 17.8603066439523
588 17.8367346938776
589 17.8404074702886
590 17.8610169491525
591 17.8544839255499
592 17.875
593 17.8768971332209
594 17.8653198653199
595 17.8924369747899
596 17.8926174496644
597 17.8844221105528
598 17.8946488294314
599 17.906510851419
600 17.9116666666667
601 17.9068219633943
602 17.9019933554817
603 17.9154228855721
604 17.9172185430464
605 17.897520661157
606 17.9273927392739
607 17.9044481054366
608 17.9309210526316
609 17.9343185550082
610 17.9377049180328
611 17.9509001636661
612 17.9591503267974
613 17.9380097879282
614 17.9397394136808
615 17.9609756097561
616 17.9399350649351
617 17.9643435980551
618 17.962783171521
619 17.9757673667205
620 17.9516129032258
621 17.9533011272142
622 17.9549839228296
623 17.9261637239165
624 17.9535256410256
625 17.9504
626 17.9568690095847
627 17.9282296650718
628 17.9187898089172
629 17.9252782193959
630 17.9079365079365
631 17.9096671949287
632 17.881329113924
633 17.86887835703
634 17.8769716088328
635 17.903937007874
636 17.9103773584906
637 17.9042386185243
638 17.8965517241379
639 17.9014084507042
640 17.8765625
641 17.8876755070203
642 17.9096573208723
643 17.9331259720062
644 17.9316770186335
645 17.9472868217054
646 17.969040247678
647 17.952086553323
648 17.962962962963
649 17.9522342064715
650 17.9646153846154
651 17.9769585253456
652 17.9585889570552
653 17.9601837672282
654 17.980122324159
655 17.9893129770992
656 17.9893292682927
657 18.0152207001522
658 18.0395136778115
659 18.0349013657056
660 18.0075757575758
661 18.0030257186082
662 18.0030211480363
663 17.9924585218703
664 17.9819277108434
665 17.9789473684211
666 17.9924924924925
667 17.9970014992504
668 17.9940119760479
669 18.0194319880419
670 17.9955223880597
671 17.9776453055142
672 17.9627976190476
673 17.9569093610698
674 17.9836795252226
675 18.0103703703704
676 18.0177514792899
677 18.0118168389956
678 18.0353982300885
679 18.0574374079529
680 18.0647058823529
681 18.0631424375918
682 18.0542521994135
683 18.0307467057101
684 18.0248538011696
685 18.0087591240876
686 17.9927113702624
687 17.9868995633188
688 17.9767441860465
689 17.9956458635704
690 17.9768115942029
691 17.9898697539797
692 17.9754335260116
693 17.987012987013
694 18.007204610951
695 18.0230215827338
696 18.0057471264368
697 18.0229555236729
698 18.0171919770774
699 18.0114449213162
700 18.0071428571429
701 18.0285306704708
702 18.014245014245
703 18.0071123755334
704 17.9957386363636
705 18.0070921985816
706 17.9844192634561
707 17.995756718529
708 17.9745762711864
709 17.9816643159379
710 17.9957746478873
711 17.9887482419128
712 17.997191011236
713 17.9915848527349
714 17.9803921568627
715 17.9986013986014
716 18
717 17.9818688981869
718 17.9693593314763
719 17.9860917941586
720 18.0013888888889
721 18.0194174757282
722 18.0332409972299
723 18.0193637621024
724 18.0124309392265
725 18.0248275862069
726 18.004132231405
727 18.0137551581843
728 17.9917582417582
729 18.0013717421125
730 18.0260273972603
731 18.0478796169631
732 18.0341530054645
733 18.0204638472033
734 18.0272479564033
735 18.0517006802721
736 18.0298913043478
737 18.0067842605156
738 17.9878048780488
739 18.0108254397835
740 17.9986486486486
741 18.0175438596491
742 18.0175202156334
743 18.0148048452221
744 18.0255376344086
745 18.0053691275168
746 18.0040214477212
747 18.0107095046854
748 18.0294117647059
749 18.0080106809079
750 18.0026666666667
751 17.9986684420772
752 17.9747340425532
753 17.9721115537849
754 17.9960212201592
755 18.0172185430464
756 18.0013227513227
757 18.0184940554822
758 18.0171503957784
759 18.030303030303
760 18.0210526315789
761 18.0197109067017
762 18.0393700787402
763 18.0498034076016
764 18.0497382198953
765 18.0509803921569
766 18.0613577023499
767 18.0769230769231
768 18.0611979166667
769 18.0533159947984
770 18.0350649350649
771 18.0402075226978
772 18.050518134715
773 18.0517464424321
774 18.0426356589147
775 18.0593548387097
776 18.0373711340206
777 18.023166023166
778 18.0218508997429
779 18.0385109114249
780 18.0564102564103
781 18.0653008962868
782 18.0703324808184
783 18.0574712643678
784 18.0701530612245
785 18.087898089172
786 18.1094147582697
787 18.1232528589581
788 18.1395939086294
789 18.1166032953105
790 18.1303797468354
791 18.1213653603034
792 18.1035353535354
793 18.0907944514502
794 18.0843828715365
795 18.0704402515723
796 18.0540201005025
797 18.0627352572146
798 18.0614035087719
799 18.0713391739675
800 18.05875
801 18.0661672908864
802 18.0885286783042
803 18.0859277708593
804 18.1019900497512
805 18.0795031055901
806 18.0980148883375
807 18.0830235439901
808 18.1039603960396
809 18.1075401730532
810 18.0888888888889
811 18.0678175092478
812 18.0455665024631
813 18.0233702337023
814 18.0184275184275
815 18.039263803681
816 18.0232843137255
817 18.0330477356181
818 18.0366748166259
819 18.046398046398
820 18.0585365853659
821 18.0706455542022
822 18.0669099756691
823 18.0595382746051
824 18.0424757281553
825 18.0606060606061
826 18.0508474576271
827 18.0725513905683
828 18.0736714975845
829 18.0579010856454
830 18.0457831325301
831 18.0373044524669
832 18.0180288461538
833 18.0012004801921
834 18.0023980815348
835 18.0167664670659
836 18
837 18.0215053763441
838 18.0047732696897
839 17.9916567342074
840 18.0035714285714
841 17.9988109393579
842 18.0059382422803
843 17.9916963226572
844 18.0047393364929
845 18.0177514792899
846 18.031914893617
847 18.0200708382527
848 18.0117924528302
849 18.0294464075383
850 18.0294117647059
851 18.0458284371328
852 18.056338028169
853 18.0398593200469
854 18.0351288056206
855 18.0526315789474
856 18.0490654205607
857 18.0536756126021
858 18.0431235431235
859 18.0325960419092
860 18.0279069767442
861 18.0394889663182
862 18.0522041763341
863 18.0405561993047
864 18.0416666666667
865 18.0208092485549
866 18.0311778290993
867 18.0161476355248
868 18.0115207373272
869 18.02416570771
870 18.0114942528736
871 18.0011481056257
872 18.0183486238532
873 18.0263459335624
874 18.0274599542334
875 18.0114285714286
876 18.0011415525114
877 17.9874572405929
878 17.9886104783599
879 17.9931740614334
880 18.0056818181818
881 18.0068104426788
882 17.9909297052154
883 18.0011325028313
884 17.9977375565611
885 17.9853107344633
886 17.9808126410835
887 17.9729425028185
888 17.9898648648649
889 17.9977502812148
890 17.9820224719101
891 17.9652076318743
892 17.9517937219731
893 17.9608062709966
894 17.9519015659955
895 17.9642458100559
896 17.9765625
897 17.9877369007804
898 18.0055679287305
899 18.0144605116796
900 18.0022222222222
901 18.0055493895671
902 17.9955654101996
903 17.9867109634551
904 17.9878318584071
905 17.9988950276243
906 17.9878587196468
907 17.986769570011
908 17.9977973568282
909 17.997799779978
910 18.0098901098901
911 17.993413830955
912 18.0120614035088
913 18.0186199342826
914 17.9989059080963
915 18.0065573770492
916 17.9945414847162
917 18.0032715376227
918 17.9956427015251
919 18.0076169749728
920 18.0130434782609
921 18.0293159609121
922 18.0119305856833
923 18.011917659805
924 17.9989177489177
925 18.0086486486486
926 18.024838012959
927 18.0183387270766
928 18.0161637931034
929 18.0064585575888
930 18.0172043010753
931 18.0085929108485
932 18.0096566523605
933 18.0128617363344
934 18.0267665952891
935 18.0245989304813
936 18.0213675213675
937 18.0394877267876
938 18.0394456289979
939 18.0436634717785
940 18.0255319148936
941 18.0085015940489
942 18.0180467091295
943 18.0021208907741
944 17.9862288135593
945 17.9703703703704
946 17.9788583509514
947 17.9598732840549
948 17.9440928270042
949 17.9251844046365
950 17.9157894736842
951 17.9211356466877
952 17.9128151260504
953 17.9202518363064
954 17.9318658280922
955 17.9371727748691
956 17.9414225941423
957 17.9425287356322
958 17.9488517745303
959 17.9416058394161
960 17.9385416666667
961 17.9292403746098
962 17.9178794178794
963 17.9231568016615
964 17.9221991701245
965 17.9347150259067
966 17.935817805383
967 17.9389865563599
968 17.9318181818182
969 17.9287925696594
970 17.9329896907216
971 17.9176107106076
972 17.9166666666667
973 17.9198355601233
974 17.911704312115
975 17.9179487179487
976 17.9354508196721
977 17.9416581371546
978 17.9274028629857
979 17.9213483146067
980 17.9255102040816
981 17.9092762487258
982 17.9063136456212
983 17.9165818921668
984 17.9085365853659
985 17.9177664974619
986 17.9056795131846
987 17.8895643363728
988 17.9048582995951
989 17.906976744186
990 17.9121212121212
991 17.9081735620585
992 17.898185483871
993 17.8972809667674
994 17.8802816901408
995 17.8804020100503
996 17.89859437751
997 17.8966900702106
998 17.9068136272545
999 17.9229229229229
1000 17.91
};
\addplot [semithick, white!49.8039215686275!black, forget plot]
table {%
1 35
2 27
3 26.6666666666667
4 20.25
5 21.2
6 21.3333333333333
7 19.4285714285714
8 19
9 16.8888888888889
10 17
11 17.8181818181818
12 18.9166666666667
13 18.6153846153846
14 19.1428571428571
15 18.6
16 18.5625
17 19.1176470588235
18 19.0555555555556
19 18.3684210526316
20 18.35
21 18.2380952380952
22 18.9090909090909
23 18.8695652173913
24 19.4166666666667
25 18.8
26 18.6923076923077
27 18.3333333333333
28 18.7857142857143
29 19.1724137931034
30 18.6666666666667
31 18.258064516129
32 18.0625
33 17.8787878787879
34 17.9411764705882
35 17.5428571428571
36 17.6666666666667
37 17.5135135135135
38 17.1842105263158
39 17.0769230769231
40 16.65
41 17.0243902439024
42 16.6190476190476
43 16.3953488372093
44 16.5909090909091
45 16.3555555555556
46 16.0652173913043
47 16.468085106383
48 16.25
49 16.1020408163265
50 15.84
51 15.5882352941176
52 15.7307692307692
53 15.6415094339623
54 15.9444444444444
55 15.8363636363636
56 16.125
57 16.0350877192982
58 15.8965517241379
59 15.8135593220339
60 15.8
61 15.8196721311475
62 15.6612903225806
63 15.7936507936508
64 15.625
65 15.4307692307692
66 15.5909090909091
67 15.6119402985075
68 15.6029411764706
69 15.8260869565217
70 15.6857142857143
71 15.9577464788732
72 16.0277777777778
73 16.0684931506849
74 16.0405405405405
75 16.3066666666667
76 16.4210526315789
77 16.3246753246753
78 16.1538461538462
79 16.1518987341772
80 16.0875
81 16.0740740740741
82 15.9878048780488
83 15.7951807228916
84 15.952380952381
85 15.7882352941176
86 15.7325581395349
87 15.9540229885057
88 16.1477272727273
89 16.0337078651685
90 16.1333333333333
91 16.3406593406593
92 16.2826086956522
93 16.4516129032258
94 16.6382978723404
95 16.4736842105263
96 16.4583333333333
97 16.3505154639175
98 16.3877551020408
99 16.3030303030303
100 16.25
101 16.3564356435644
102 16.5196078431373
103 16.5728155339806
104 16.7211538461538
105 16.7142857142857
106 16.6415094339623
107 16.5233644859813
108 16.5833333333333
109 16.7522935779816
110 16.6909090909091
111 16.8378378378378
112 16.8035714285714
113 16.8318584070796
114 16.6929824561403
115 16.8521739130435
116 16.7068965517241
117 16.8547008547009
118 16.8305084745763
119 16.8823529411765
120 17.0166666666667
121 16.900826446281
122 16.9344262295082
123 16.7967479674797
124 16.7258064516129
125 16.84
126 16.9126984126984
127 16.9763779527559
128 16.9921875
129 16.937984496124
130 16.8769230769231
131 16.9847328244275
132 16.9924242424242
133 17.0451127819549
134 17.0820895522388
135 17.1703703703704
136 17.1397058823529
137 17.021897810219
138 17.1231884057971
139 17.0143884892086
140 17.0785714285714
141 17.2127659574468
142 17.1408450704225
143 17.041958041958
144 16.9791666666667
145 16.9793103448276
146 17.0616438356164
147 16.9795918367347
148 17.0945945945946
149 17.1140939597315
150 17.2133333333333
151 17.3311258278146
152 17.3355263157895
153 17.3006535947712
154 17.2337662337662
155 17.2709677419355
156 17.2628205128205
157 17.3375796178344
158 17.4050632911392
159 17.4025157232704
160 17.38125
161 17.4534161490683
162 17.5432098765432
163 17.6564417177914
164 17.6951219512195
165 17.8060606060606
166 17.8614457831325
167 17.8023952095808
168 17.8214285714286
169 17.7810650887574
170 17.8764705882353
171 17.7894736842105
172 17.7906976744186
173 17.7456647398844
174 17.7126436781609
175 17.7942857142857
176 17.8295454545455
177 17.728813559322
178 17.7247191011236
179 17.7988826815642
180 17.8444444444444
181 17.7624309392265
182 17.7197802197802
183 17.6939890710383
184 17.7554347826087
185 17.7297297297297
186 17.6612903225806
187 17.6737967914438
188 17.75
189 17.8253968253968
190 17.8052631578947
191 17.7539267015707
192 17.8333333333333
193 17.9170984455959
194 18.0103092783505
195 18.0205128205128
196 18.0102040816327
197 18.0609137055838
198 18.0454545454545
199 18.0552763819095
200 18.12
201 18.0696517412935
202 18.0693069306931
203 18.1527093596059
204 18.2352941176471
205 18.2780487804878
206 18.2815533980583
207 18.3574879227053
208 18.2932692307692
209 18.2822966507177
210 18.2
211 18.1232227488152
212 18.127358490566
213 18.112676056338
214 18.0373831775701
215 17.9953488372093
216 17.9351851851852
217 18.0138248847926
218 17.9816513761468
219 17.9223744292237
220 17.8545454545455
221 17.8461538461538
222 17.8738738738739
223 17.8699551569507
224 17.8616071428571
225 17.7955555555556
226 17.8761061946903
227 17.8810572687225
228 17.9166666666667
229 17.9039301310044
230 17.9304347826087
231 17.978354978355
232 17.9655172413793
233 17.9442060085837
234 18.017094017094
235 18.0382978723404
236 18.0720338983051
237 18
238 18.0756302521008
239 18.1255230125523
240 18.1166666666667
241 18.1784232365145
242 18.2066115702479
243 18.238683127572
244 18.266393442623
245 18.3224489795918
246 18.2764227642276
247 18.2348178137652
248 18.1693548387097
249 18.2208835341365
250 18.208
251 18.2151394422311
252 18.265873015873
253 18.2055335968379
254 18.259842519685
255 18.2117647058824
256 18.1953125
257 18.2451361867704
258 18.1821705426357
259 18.1351351351351
260 18.15
261 18.2068965517241
262 18.236641221374
263 18.2889733840304
264 18.2386363636364
265 18.2943396226415
266 18.3533834586466
267 18.2958801498127
268 18.3283582089552
269 18.3159851301115
270 18.2777777777778
271 18.3062730627306
272 18.2536764705882
273 18.2710622710623
274 18.3175182481752
275 18.2763636363636
276 18.3115942028986
277 18.3718411552347
278 18.431654676259
279 18.3978494623656
280 18.4357142857143
281 18.4377224199288
282 18.436170212766
283 18.4628975265018
284 18.5
285 18.5087719298246
286 18.4825174825175
287 18.5296167247387
288 18.5138888888889
289 18.560553633218
290 18.5344827586207
291 18.5360824742268
292 18.4897260273973
293 18.5085324232082
294 18.5612244897959
295 18.5322033898305
296 18.4864864864865
297 18.4377104377104
298 18.3926174496644
299 18.438127090301
300 18.4666666666667
301 18.4651162790698
302 18.4403973509934
303 18.4521452145215
304 18.4440789473684
305 18.3901639344262
306 18.4019607843137
307 18.4071661237785
308 18.3863636363636
309 18.3462783171521
310 18.3548387096774
311 18.4083601286174
312 18.4519230769231
313 18.3961661341853
314 18.3917197452229
315 18.4253968253968
316 18.376582278481
317 18.3690851735016
318 18.4182389937107
319 18.4326018808777
320 18.440625
321 18.4548286604361
322 18.4937888198758
323 18.4643962848297
324 18.4475308641975
325 18.4184615384615
326 18.3711656441718
327 18.3669724770642
328 18.3841463414634
329 18.4255319148936
330 18.4151515151515
331 18.404833836858
332 18.4096385542169
333 18.3963963963964
334 18.4341317365269
335 18.4298507462687
336 18.422619047619
337 18.4569732937685
338 18.4408284023669
339 18.4808259587021
340 18.4382352941176
341 18.4310850439883
342 18.3771929824561
343 18.3848396501458
344 18.343023255814
345 18.3797101449275
346 18.4306358381503
347 18.3948126801153
348 18.3936781609195
349 18.4154727793696
350 18.4342857142857
351 18.3960113960114
352 18.4289772727273
353 18.4730878186969
354 18.4661016949153
355 18.4338028169014
356 18.4662921348315
357 18.4397759103641
358 18.3966480446927
359 18.4401114206128
360 18.4611111111111
361 18.4764542936288
362 18.4861878453039
363 18.4931129476584
364 18.4972527472527
365 18.4739726027397
366 18.4371584699454
367 18.4441416893733
368 18.4592391304348
369 18.4092140921409
370 18.4243243243243
371 18.3935309973046
372 18.3817204301075
373 18.3512064343164
374 18.3101604278075
375 18.328
376 18.2898936170213
377 18.3262599469496
378 18.3465608465608
379 18.3166226912929
380 18.3447368421053
381 18.3175853018373
382 18.3010471204188
383 18.3394255874674
384 18.3255208333333
385 18.2831168831169
386 18.2616580310881
387 18.266149870801
388 18.2319587628866
389 18.2236503856041
390 18.2282051282051
391 18.1994884910486
392 18.1811224489796
393 18.1908396946565
394 18.2157360406091
395 18.2607594936709
396 18.280303030303
397 18.264483627204
398 18.2562814070352
399 18.2155388471178
400 18.21
401 18.2094763092269
402 18.1940298507463
403 18.1985111662531
404 18.2227722772277
405 18.2592592592593
406 18.2142857142857
407 18.1867321867322
408 18.171568627451
409 18.2053789731051
410 18.2146341463415
411 18.2189781021898
412 18.1966019417476
413 18.2130750605327
414 18.2367149758454
415 18.2746987951807
416 18.2860576923077
417 18.3213429256595
418 18.311004784689
419 18.2935560859189
420 18.3261904761905
421 18.3301662707838
422 18.3507109004739
423 18.3238770685579
424 18.3018867924528
425 18.3341176470588
426 18.3380281690141
427 18.3231850117096
428 18.3434579439252
429 18.3473193473193
430 18.3279069767442
431 18.3039443155452
432 18.2800925925926
433 18.2817551963049
434 18.2926267281106
435 18.2620689655172
436 18.2798165137615
437 18.2768878718535
438 18.2671232876712
439 18.250569476082
440 18.2840909090909
441 18.3174603174603
442 18.3212669683258
443 18.2957110609481
444 18.2837837837838
445 18.2831460674157
446 18.2443946188341
447 18.2505592841163
448 18.2589285714286
449 18.2561247216036
450 18.2288888888889
451 18.2084257206208
452 18.2300884955752
453 18.2229580573951
454 18.2070484581498
455 18.189010989011
456 18.2127192982456
457 18.1728665207877
458 18.1550218340611
459 18.1503267973856
460 18.1869565217391
461 18.1952277657267
462 18.1753246753247
463 18.1727861771058
464 18.135775862069
465 18.1225806451613
466 18.1008583690987
467 18.1134903640257
468 18.1346153846154
469 18.1044776119403
470 18.1021276595745
471 18.0976645435244
472 18.1292372881356
473 18.1458773784355
474 18.1371308016878
475 18.1705263157895
476 18.1974789915966
477 18.2180293501048
478 18.2217573221757
479 18.2359081419624
480 18.2229166666667
481 18.2141372141372
482 18.1991701244813
483 18.184265010352
484 18.2107438016529
485 18.2329896907216
486 18.2057613168724
487 18.2258726899384
488 18.2561475409836
489 18.2188139059305
490 18.2367346938776
491 18.2443991853361
492 18.2113821138211
493 18.2210953346856
494 18.1902834008097
495 18.1757575757576
496 18.1673387096774
497 18.1851106639839
498 18.2088353413655
499 18.2004008016032
500 18.18
501 18.1437125748503
502 18.1474103585657
503 18.155069582505
504 18.1527777777778
505 18.1326732673267
506 18.1245059288538
507 18.1025641025641
508 18.1122047244094
509 18.147347740668
510 18.1705882352941
511 18.1702544031311
512 18.19140625
513 18.2066276803119
514 18.1712062256809
515 18.1825242718447
516 18.1937984496124
517 18.1624758220503
518 18.1718146718147
519 18.1618497109827
520 18.1461538461538
521 18.1458733205374
522 18.1781609195402
523 18.1778202676864
524 18.1870229007634
525 18.1847619047619
526 18.1825095057034
527 18.157495256167
528 18.1628787878788
529 18.1512287334594
530 18.1471698113208
531 18.1789077212806
532 18.2011278195489
533 18.2326454033771
534 18.2116104868914
535 18.1775700934579
536 18.1548507462687
537 18.1601489757914
538 18.1468401486989
539 18.1447124304267
540 18.1481481481481
541 18.173752310536
542 18.1531365313653
543 18.121546961326
544 18.1121323529412
545 18.1192660550459
546 18.1355311355311
547 18.1627056672761
548 18.1532846715328
549 18.1821493624772
550 18.1963636363636
551 18.1941923774955
552 18.2264492753623
553 18.2043399638336
554 18.2328519855596
555 18.2234234234234
556 18.2140287769784
557 18.1867145421903
558 18.173835125448
559 18.1932021466905
560 18.1767857142857
561 18.1836007130125
562 18.1992882562278
563 18.1705150976909
564 18.1932624113475
565 18.1787610619469
566 18.1943462897527
567 18.1693121693122
568 18.1760563380282
569 18.1616871704745
570 18.1298245614035
571 18.1593695271454
572 18.1783216783217
573 18.2006980802792
574 18.1689895470383
575 18.1478260869565
576 18.140625
577 18.1629116117851
578 18.1349480968858
579 18.1174438687392
580 18.1431034482759
581 18.1239242685026
582 18.1494845360825
583 18.1715265866209
584 18.1952054794521
585 18.2222222222222
586 18.2354948805461
587 18.2436115843271
588 18.2159863945578
589 18.2054329371817
590 18.2220338983051
591 18.2402707275804
592 18.2652027027027
593 18.2529510961214
594 18.2255892255892
595 18.2134453781513
596 18.2348993288591
597 18.2479061976549
598 18.2190635451505
599 18.2220367278798
600 18.2483333333333
601 18.2495840266223
602 18.2408637873754
603 18.2321724709784
604 18.203642384106
605 18.2231404958678
606 18.2475247524752
607 18.2257001647446
608 18.2105263157895
609 18.2036124794745
610 18.2
611 18.1816693944354
612 18.1977124183007
613 18.1729200652529
614 18.1693811074919
615 18.1414634146341
616 18.1233766233766
617 18.095623987034
618 18.1148867313916
619 18.124394184168
620 18.1370967741935
621 18.1545893719807
622 18.1720257234727
623 18.1717495987159
624 18.1955128205128
625 18.2
626 18.1789137380192
627 18.1802232854864
628 18.1576433121019
629 18.1367249602544
630 18.1571428571429
631 18.1315372424723
632 18.1139240506329
633 18.1090047393365
634 18.115141955836
635 18.1338582677165
636 18.1540880503145
637 18.1773940345369
638 18.1630094043887
639 18.1658841940532
640 18.153125
641 18.1669266770671
642 18.1619937694704
643 18.1741835147745
644 18.1599378881988
645 18.1596899224806
646 18.1439628482972
647 18.1514683153014
648 18.162037037037
649 18.1895223420647
650 18.2015384615385
651 18.1920122887865
652 18.1656441717791
653 18.1699846860643
654 18.1681957186544
655 18.1404580152672
656 18.1158536585366
657 18.1095890410959
658 18.1003039513678
659 18.1183611532625
660 18.1166666666667
661 18.1089258698941
662 18.0815709969789
663 18.0678733031674
664 18.085843373494
665 18.0812030075188
666 18.0975975975976
667 18.0749625187406
668 18.0688622754491
669 18.0627802690583
670 18.0611940298507
671 18.0611028315946
672 18.0357142857143
673 18.0312035661218
674 18.0178041543027
675 18.0207407407407
676 18.0295857988166
677 18.0029542097489
678 18.0014749262537
679 17.9896907216495
680 18.0161764705882
681 18.0132158590308
682 17.9912023460411
683 17.9736456808199
684 17.9897660818713
685 18.0058394160584
686 17.9941690962099
687 18.0145560407569
688 18.030523255814
689 18.0522496371553
690 18.0391304347826
691 18.0578871201158
692 18.0534682080925
693 18.0562770562771
694 18.0475504322767
695 18.0258992805755
696 18.0172413793103
697 18.0157819225251
698 18.0401146131805
699 18.0557939914163
700 18.0557142857143
701 18.0798858773181
702 18.0754985754986
703 18.0810810810811
704 18.0965909090909
705 18.0992907801418
706 18.1203966005666
707 18.1159830268741
708 18.1073446327684
709 18.0874471086037
710 18.1098591549296
711 18.1097046413502
712 18.1207865168539
713 18.1178120617111
714 18.1288515406162
715 18.1342657342657
716 18.1578212290503
717 18.1380753138075
718 18.1573816155989
719 18.1682892906815
720 18.1722222222222
721 18.1567267683773
722 18.1565096952909
723 18.1424619640387
724 18.1270718232044
725 18.1337931034483
726 18.1157024793388
727 18.0962861072902
728 18.0769230769231
729 18.0987654320988
730 18.1041095890411
731 18.0848153214774
732 18.1010928961749
733 18.0832196452933
734 18.0681198910082
735 18.0748299319728
736 18.0502717391304
737 18.044776119403
738 18.029810298103
739 18.0284167794317
740 18.0486486486486
741 18.0485829959514
742 18.0673854447439
743 18.0551816958277
744 18.0551075268817
745 18.0563758389262
746 18.0428954423592
747 18.0254350736278
748 18.024064171123
749 18.0373831775701
750 18.048
751 18.0532623169108
752 18.0438829787234
753 18.0265604249668
754 18.0066312997347
755 18.0278145695364
756 18.0291005291005
757 18.0066050198151
758 18.0145118733509
759 18.0276679841897
760 18.0197368421053
761 18.0078843626807
762 17.990813648294
763 17.9764089121887
764 17.988219895288
765 17.9816993464052
766 17.9712793733681
767 17.9543676662321
768 17.9622395833333
769 17.9830949284785
770 17.9701298701299
771 17.9896238651102
772 18.0103626943005
773 18.0245795601552
774 18.0271317829457
775 18.0258064516129
776 18.0167525773196
777 18.006435006435
778 18.0192802056555
779 18.0320924261874
780 18.0294871794872
781 18.0268886043534
782 18.0166240409207
783 18.0280970625798
784 18.0102040816327
785 18.0076433121019
786 17.9847328244275
787 17.9758576874206
788 17.988578680203
789 17.9683143219265
790 17.9746835443038
791 17.9595448798989
792 17.9646464646465
793 17.9646910466583
794 17.9697732997481
795 17.9748427672956
796 17.9899497487437
797 17.9749058971142
798 17.9573934837093
799 17.9386733416771
800 17.9225
801 17.9138576779026
802 17.9326683291771
803 17.9165628891656
804 17.9203980099502
805 17.9378881987578
806 17.9553349875931
807 17.9454770755886
808 17.9529702970297
809 17.9307787391842
810 17.9271604938272
811 17.9272503082614
812 17.9113300492611
813 17.9114391143911
814 17.9312039312039
815 17.9263803680982
816 17.9424019607843
817 17.937576499388
818 17.9535452322738
819 17.9487179487179
820 17.9475609756098
821 17.9342265529842
822 17.9245742092457
823 17.9113001215067
824 17.9247572815534
825 17.9284848484848
826 17.9067796610169
827 17.8887545344619
828 17.8997584541063
829 17.8986731001206
830 17.9156626506024
831 17.9253910950662
832 17.9158653846154
833 17.921968787515
834 17.9088729016787
835 17.9041916167665
836 17.9174641148325
837 17.9115890083632
838 17.9307875894988
839 17.9141835518474
840 17.9214285714286
841 17.9274673008323
842 17.9418052256532
843 17.9572953736655
844 17.9478672985782
845 17.9668639053254
846 17.9586288416076
847 17.9704840613932
848 17.9681603773585
849 17.9587750294464
850 17.9411764705882
851 17.957696827262
852 17.9565727699531
853 17.947245017585
854 17.9461358313817
855 17.9251461988304
856 17.9135514018692
857 17.9206534422404
858 17.9079254079254
859 17.918509895227
860 17.9232558139535
861 17.9361207897793
862 17.915313225058
863 17.9223638470452
864 17.912037037037
865 17.8959537572254
866 17.8868360277136
867 17.8961937716263
868 17.8997695852535
869 17.9159953970081
870 17.9022988505747
871 17.8897818599311
872 17.8692660550459
873 17.8499427262314
874 17.8672768878719
875 17.8685714285714
876 17.8755707762557
877 17.8574686431015
878 17.8496583143508
879 17.844141069397
880 17.8397727272727
881 17.8433598183882
882 17.8560090702948
883 17.8550396375991
884 17.8574660633484
885 17.8723163841808
886 17.8589164785553
887 17.8421645997745
888 17.8524774774775
889 17.8335208098988
890 17.8438202247191
891 17.8271604938272
892 17.8340807174888
893 17.8398656215006
894 17.8579418344519
895 17.8413407821229
896 17.8426339285714
897 17.8394648829431
898 17.8240534521158
899 17.8142380422692
900 17.7955555555556
901 17.7980022197558
902 17.7982261640798
903 17.8139534883721
904 17.8196902654867
905 17.8
906 17.8200883002207
907 17.8390297684675
908 17.8193832599119
909 17.8063806380638
910 17.8043956043956
911 17.8068057080132
912 17.8092105263158
913 17.8192771084337
914 17.8326039387309
915 17.824043715847
916 17.8406113537118
917 17.8386041439477
918 17.8278867102396
919 17.8476605005441
920 17.8347826086957
921 17.8154180238871
922 17.7971800433839
923 17.8136511375948
924 17.8051948051948
925 17.7978378378378
926 17.792656587473
927 17.8122977346278
928 17.801724137931
929 17.805166846071
930 17.8032258064516
931 17.8034371643394
932 17.7907725321888
933 17.7877813504823
934 17.7869379014989
935 17.8064171122995
936 17.7938034188034
937 17.7908217716115
938 17.7771855010661
939 17.7667731629393
940 17.7478723404255
941 17.7630180658874
942 17.7802547770701
943 17.7751855779427
944 17.791313559322
945 17.7883597883598
946 17.7790697674419
947 17.7719112988384
948 17.785864978903
949 17.7881981032666
950 17.7726315789474
951 17.7781282860147
952 17.7689075630252
953 17.7670514165792
954 17.7840670859539
955 17.7832460732984
956 17.7866108786611
957 17.8045977011494
958 17.7881002087683
959 17.7799791449426
960 17.7895833333333
961 17.7731529656608
962 17.7671517671518
963 17.7798546209761
964 17.792531120332
965 17.7917098445596
966 17.7846790890269
967 17.7900723888314
968 17.779958677686
969 17.7925696594427
970 17.7927835051546
971 17.8084449021627
972 17.8024691358025
973 17.8129496402878
974 17.8100616016427
975 17.8061538461538
976 17.8073770491803
977 17.7911975435005
978 17.8006134969325
979 17.8049029622063
980 17.8234693877551
981 17.8318042813456
982 17.8421588594705
983 17.8555442522889
984 17.8587398373984
985 17.8568527918782
986 17.8519269776876
987 17.8409321175279
988 17.8390688259109
989 17.8513650151668
990 17.8646464646465
991 17.8597376387487
992 17.8639112903226
993 17.8559919436052
994 17.8551307847083
995 17.8422110552764
996 17.8293172690763
997 17.8164493480441
998 17.8036072144289
999 17.8168168168168
1000 17.805
};
\addplot [semithick, color7, forget plot]
table {%
1 30
2 23
3 17
4 19
5 21.2
6 20
7 19
8 19.25
9 19.2222222222222
10 18.4
11 17.5454545454545
12 18.9166666666667
13 19
14 18.5
15 19.6
16 20.4375
17 19.5294117647059
18 20.3888888888889
19 19.4736842105263
20 18.9
21 18.6190476190476
22 17.8181818181818
23 18.2173913043478
24 17.9166666666667
25 18.2
26 17.7307692307692
27 17.5555555555556
28 17.8571428571429
29 18.3448275862069
30 18.8333333333333
31 18.2903225806452
32 17.84375
33 18.2727272727273
34 18.5
35 18.3428571428571
36 18.1111111111111
37 17.7027027027027
38 17.5526315789474
39 17.2820512820513
40 17.175
41 17.3414634146341
42 17.1190476190476
43 17.3953488372093
44 17.6590909090909
45 17.2666666666667
46 16.8913043478261
47 16.9787234042553
48 16.8125
49 16.6530612244898
50 16.6
51 16.6078431372549
52 16.6538461538462
53 16.7358490566038
54 16.8703703703704
55 16.7636363636364
56 16.6785714285714
57 16.4210526315789
58 16.2931034482759
59 16.1864406779661
60 16.2
61 16.5245901639344
62 16.258064516129
63 16.1904761904762
64 16.078125
65 16.3230769230769
66 16.2878787878788
67 16.3731343283582
68 16.5588235294118
69 16.3333333333333
70 16.2142857142857
71 16.4788732394366
72 16.6666666666667
73 16.6027397260274
74 16.7567567567568
75 16.5466666666667
76 16.4078947368421
77 16.5844155844156
78 16.7179487179487
79 16.7341772151899
80 16.825
81 16.9012345679012
82 16.9756097560976
83 16.8192771084337
84 16.9404761904762
85 17.1294117647059
86 17.0697674418605
87 16.8735632183908
88 16.9318181818182
89 16.9775280898876
90 16.8111111111111
91 16.6923076923077
92 16.8586956521739
93 16.7634408602151
94 16.6595744680851
95 16.7368421052632
96 16.7916666666667
97 16.639175257732
98 16.7142857142857
99 16.5959595959596
100 16.76
101 16.8415841584158
102 16.921568627451
103 16.9902912621359
104 17.1634615384615
105 17.2761904761905
106 17.3867924528302
107 17.5233644859813
108 17.3981481481481
109 17.4678899082569
110 17.5363636363636
111 17.6216216216216
112 17.6785714285714
113 17.5221238938053
114 17.640350877193
115 17.6521739130435
116 17.8103448275862
117 17.9401709401709
118 17.8813559322034
119 17.9495798319328
120 18.0083333333333
121 17.9586776859504
122 17.9590163934426
123 17.8536585365854
124 17.7661290322581
125 17.872
126 17.8015873015873
127 17.7716535433071
128 17.8203125
129 17.7984496124031
130 17.8307692307692
131 17.7251908396947
132 17.6893939393939
133 17.6165413533835
134 17.5597014925373
135 17.6814814814815
136 17.5882352941176
137 17.7080291970803
138 17.7898550724638
139 17.7194244604317
140 17.8357142857143
141 17.7304964539007
142 17.8521126760563
143 17.7832167832168
144 17.8333333333333
145 17.8275862068966
146 17.9178082191781
147 17.8979591836735
148 17.777027027027
149 17.8187919463087
150 17.8866666666667
151 17.8609271523179
152 17.8026315789474
153 17.7581699346405
154 17.6883116883117
155 17.5870967741935
156 17.6153846153846
157 17.5477707006369
158 17.5189873417722
159 17.4528301886792
160 17.55
161 17.4658385093168
162 17.5617283950617
163 17.4907975460123
164 17.5243902439024
165 17.5515151515152
166 17.5421686746988
167 17.5209580838323
168 17.4821428571429
169 17.4674556213018
170 17.4588235294118
171 17.374269005848
172 17.4244186046512
173 17.4335260115607
174 17.3390804597701
175 17.2742857142857
176 17.2670454545455
177 17.1864406779661
178 17.1797752808989
179 17.2513966480447
180 17.2722222222222
181 17.2044198895028
182 17.2967032967033
183 17.2568306010929
184 17.1684782608696
185 17.2162162162162
186 17.2150537634409
187 17.2620320855615
188 17.3191489361702
189 17.2804232804233
190 17.2736842105263
191 17.2565445026178
192 17.2916666666667
193 17.3782383419689
194 17.3917525773196
195 17.3282051282051
196 17.3928571428571
197 17.4619289340102
198 17.510101010101
199 17.4522613065327
200 17.525
201 17.4975124378109
202 17.4257425742574
203 17.3793103448276
204 17.4117647058824
205 17.3756097560976
206 17.3446601941748
207 17.2898550724638
208 17.2403846153846
209 17.1722488038278
210 17.2333333333333
211 17.2890995260664
212 17.3349056603774
213 17.4131455399061
214 17.4065420560748
215 17.3860465116279
216 17.3425925925926
217 17.4285714285714
218 17.4403669724771
219 17.3744292237443
220 17.3863636363636
221 17.4705882352941
222 17.4504504504505
223 17.4484304932735
224 17.4508928571429
225 17.5111111111111
226 17.4734513274336
227 17.5330396475771
228 17.5701754385965
229 17.5851528384279
230 17.5913043478261
231 17.5151515151515
232 17.448275862069
233 17.4849785407725
234 17.482905982906
235 17.531914893617
236 17.4703389830508
237 17.4725738396624
238 17.4495798319328
239 17.489539748954
240 17.475
241 17.5435684647303
242 17.6074380165289
243 17.6049382716049
244 17.6598360655738
245 17.6163265306122
246 17.5691056910569
247 17.6072874493927
248 17.6048387096774
249 17.5622489959839
250 17.544
251 17.5139442231076
252 17.5674603174603
253 17.5059288537549
254 17.511811023622
255 17.5372549019608
256 17.515625
257 17.4747081712062
258 17.453488372093
259 17.4517374517375
260 17.4884615384615
261 17.4444444444444
262 17.5152671755725
263 17.5247148288973
264 17.5530303030303
265 17.5056603773585
266 17.5112781954887
267 17.501872659176
268 17.5261194029851
269 17.5018587360595
270 17.5111111111111
271 17.4575645756458
272 17.4044117647059
273 17.4505494505494
274 17.4160583941606
275 17.4509090909091
276 17.4094202898551
277 17.4151624548736
278 17.4280575539568
279 17.4659498207885
280 17.4214285714286
281 17.4555160142349
282 17.4716312056738
283 17.4911660777385
284 17.4577464788732
285 17.4035087719298
286 17.4615384615385
287 17.4947735191638
288 17.5
289 17.4913494809689
290 17.4413793103448
291 17.4089347079038
292 17.3972602739726
293 17.3856655290102
294 17.3877551020408
295 17.4237288135593
296 17.3851351351351
297 17.4410774410774
298 17.3993288590604
299 17.3612040133779
300 17.31
301 17.2724252491694
302 17.317880794702
303 17.3729372937294
304 17.4111842105263
305 17.3672131147541
306 17.3235294117647
307 17.3778501628664
308 17.3831168831169
309 17.3754045307443
310 17.3870967741935
311 17.4083601286174
312 17.4166666666667
313 17.4408945686901
314 17.5
315 17.4761904761905
316 17.503164556962
317 17.5583596214511
318 17.5251572327044
319 17.5673981191223
320 17.565625
321 17.5732087227414
322 17.6211180124224
323 17.5789473684211
324 17.537037037037
325 17.5230769230769
326 17.521472392638
327 17.5137614678899
328 17.5670731707317
329 17.6109422492401
330 17.5787878787879
331 17.583081570997
332 17.5331325301205
333 17.5075075075075
334 17.4730538922156
335 17.4805970149254
336 17.5119047619048
337 17.4747774480712
338 17.4940828402367
339 17.4601769911504
340 17.45
341 17.4545454545455
342 17.4356725146199
343 17.463556851312
344 17.4941860465116
345 17.4985507246377
346 17.5231213872832
347 17.4956772334294
348 17.4712643678161
349 17.4212034383954
350 17.3857142857143
351 17.3703703703704
352 17.3693181818182
353 17.4079320113314
354 17.4435028248588
355 17.4225352112676
356 17.3932584269663
357 17.3893557422969
358 17.377094972067
359 17.3788300835655
360 17.4222222222222
361 17.4432132963989
362 17.4834254143646
363 17.4986225895317
364 17.5467032967033
365 17.5616438356164
366 17.5819672131148
367 17.5776566757493
368 17.6086956521739
369 17.5718157181572
370 17.5837837837838
371 17.6172506738544
372 17.6129032258065
373 17.6380697050938
374 17.620320855615
375 17.6373333333333
376 17.593085106383
377 17.6153846153846
378 17.6481481481481
379 17.688654353562
380 17.7368421052632
381 17.7086614173228
382 17.738219895288
383 17.6971279373368
384 17.6744791666667
385 17.7194805194805
386 17.7124352331606
387 17.6795865633075
388 17.7010309278351
389 17.681233933162
390 17.7
391 17.6624040920716
392 17.6760204081633
393 17.6615776081425
394 17.6725888324873
395 17.7189873417722
396 17.7020202020202
397 17.6700251889169
398 17.6733668341709
399 17.6390977443609
400 17.6175
401 17.6034912718204
402 17.6044776119403
403 17.5732009925558
404 17.539603960396
405 17.5654320987654
406 17.5394088669951
407 17.5036855036855
408 17.4828431372549
409 17.5061124694377
410 17.5
411 17.4817518248175
412 17.4733009708738
413 17.4866828087167
414 17.4661835748792
415 17.4698795180723
416 17.4423076923077
417 17.4292565947242
418 17.4210526315789
419 17.3866348448687
420 17.3619047619048
421 17.3515439429929
422 17.3199052132701
423 17.3475177304965
424 17.3301886792453
425 17.32
426 17.3544600938967
427 17.3255269320843
428 17.3481308411215
429 17.3869463869464
430 17.3790697674419
431 17.4176334106729
432 17.4282407407407
433 17.4110854503464
434 17.4193548387097
435 17.4390804597701
436 17.4449541284404
437 17.4508009153318
438 17.486301369863
439 17.5216400911162
440 17.5590909090909
441 17.6009070294785
442 17.5656108597285
443 17.5259593679458
444 17.5472972972973
445 17.5213483146067
446 17.5448430493274
447 17.5816554809843
448 17.5446428571429
449 17.5167037861915
450 17.5288888888889
451 17.5609756097561
452 17.5575221238938
453 17.5916114790287
454 17.5969162995595
455 17.5604395604396
456 17.5350877192982
457 17.5251641137856
458 17.561135371179
459 17.5969498910675
460 17.5652173913043
461 17.5379609544469
462 17.5584415584416
463 17.5745140388769
464 17.5625
465 17.5956989247312
466 17.5772532188841
467 17.6038543897216
468 17.6153846153846
469 17.6417910447761
470 17.6340425531915
471 17.6114649681529
472 17.6483050847458
473 17.6828752642706
474 17.7088607594937
475 17.7094736842105
476 17.7436974789916
477 17.7169811320755
478 17.7426778242678
479 17.7181628392484
480 17.7
481 17.6777546777547
482 17.7033195020747
483 17.7246376811594
484 17.7293388429752
485 17.760824742268
486 17.7263374485597
487 17.7022587268994
488 17.7110655737705
489 17.721881390593
490 17.7387755102041
491 17.7535641547862
492 17.7682926829268
493 17.7464503042596
494 17.7206477732794
495 17.7515151515152
496 17.7782258064516
497 17.7967806841046
498 17.781124497992
499 17.74749498998
500 17.74
501 17.7085828343313
502 17.7370517928287
503 17.7017892644135
504 17.7202380952381
505 17.7445544554455
506 17.7312252964427
507 17.7337278106509
508 17.7185039370079
509 17.7445972495088
510 17.756862745098
511 17.7906066536204
512 17.78125
513 17.8089668615984
514 17.7801556420233
515 17.8019417475728
516 17.7751937984496
517 17.7911025145068
518 17.7992277992278
519 17.8285163776493
520 17.8269230769231
521 17.8560460652591
522 17.867816091954
523 17.8948374760994
524 17.8625954198473
525 17.8895238095238
526 17.8726235741445
527 17.8557874762808
528 17.8276515151515
529 17.8582230623819
530 17.8320754716981
531 17.8079096045198
532 17.8402255639098
533 17.8405253283302
534 17.874531835206
535 17.8504672897196
536 17.8432835820896
537 17.8379888268156
538 17.814126394052
539 17.8237476808905
540 17.8555555555556
541 17.8410351201479
542 17.8394833948339
543 17.8489871086556
544 17.8400735294118
545 17.8348623853211
546 17.8131868131868
547 17.7861060329068
548 17.7755474452555
549 17.7941712204007
550 17.8236363636364
551 17.8003629764065
552 17.8079710144928
553 17.7848101265823
554 17.7635379061372
555 17.7711711711712
556 17.7625899280576
557 17.7504488330341
558 17.7293906810036
559 17.7513416815742
560 17.7625
561 17.7326203208556
562 17.7633451957295
563 17.765541740675
564 17.7801418439716
565 17.7982300884956
566 17.8286219081272
567 17.8218694885362
568 17.7957746478873
569 17.8137082601054
570 17.8105263157895
571 17.8353765323993
572 17.8479020979021
573 17.8429319371728
574 17.8728222996516
575 17.8434782608696
576 17.8715277777778
577 17.8769497400347
578 17.9083044982699
579 17.9170984455959
580 17.9172413793103
581 17.9414802065404
582 17.9278350515464
583 17.8987993138937
584 17.8852739726027
585 17.8905982905983
586 17.8839590443686
587 17.8722316865417
588 17.8418367346939
589 17.8200339558574
590 17.8016949152542
591 17.7986463620981
592 17.8192567567568
593 17.8414839797639
594 17.8552188552189
595 17.8521008403361
596 17.8607382550336
597 17.8793969849246
598 17.9080267558528
599 17.9115191986644
600 17.8966666666667
601 17.9151414309484
602 17.9302325581395
603 17.9187396351575
604 17.8940397350993
605 17.8743801652893
606 17.8943894389439
607 17.9060955518946
608 17.8799342105263
609 17.8669950738916
610 17.855737704918
611 17.847790507365
612 17.8496732026144
613 17.8727569331158
614 17.8501628664495
615 17.8634146341463
616 17.8344155844156
617 17.8152350081037
618 17.8139158576052
619 17.7867528271405
620 17.7645161290323
621 17.7407407407407
622 17.7331189710611
623 17.7158908507223
624 17.7275641025641
625 17.704
626 17.7284345047923
627 17.7240829346092
628 17.7388535031847
629 17.724960254372
630 17.731746031746
631 17.7147385103011
632 17.7325949367089
633 17.7251184834123
634 17.698738170347
635 17.711811023622
636 17.7311320754717
637 17.7488226059655
638 17.7351097178683
639 17.7151799687011
640 17.7421875
641 17.7410296411856
642 17.7367601246106
643 17.7589424572317
644 17.7655279503106
645 17.737984496124
646 17.7244582043344
647 17.7341576506955
648 17.7191358024691
649 17.7473035439137
650 17.7523076923077
651 17.7757296466974
652 17.7745398773006
653 17.7626339969372
654 17.7385321100917
655 17.7648854961832
656 17.7926829268293
657 17.7914764079148
658 17.8024316109423
659 17.8118361153263
660 17.8333333333333
661 17.8593040847201
662 17.8383685800604
663 17.8280542986425
664 17.8207831325301
665 17.806015037594
666 17.8288288288288
667 17.8530734632684
668 17.8323353293413
669 17.8295964125561
670 17.8358208955224
671 17.8286140089419
672 17.8244047619048
673 17.8291233283804
674 17.8516320474777
675 17.8577777777778
676 17.8698224852071
677 17.8522895125554
678 17.8613569321534
679 17.8674521354934
680 17.8470588235294
681 17.8693098384728
682 17.8563049853372
683 17.8404099560761
684 17.8523391812865
685 17.8554744525547
686 17.8454810495627
687 17.8398835516739
688 17.8328488372093
689 17.8200290275762
690 17.8202898550725
691 17.8364688856729
692 17.8612716763006
693 17.8860028860029
694 17.8631123919308
695 17.8647482014389
696 17.8764367816092
697 17.8680057388809
698 17.852435530086
699 17.862660944206
700 17.8628571428571
701 17.8701854493581
702 17.8660968660969
703 17.8904694167852
704 17.8678977272727
705 17.8425531914894
706 17.8640226628895
707 17.8868458274399
708 17.909604519774
709 17.9083215796897
710 17.9
711 17.9226441631505
712 17.9185393258427
713 17.8948106591865
714 17.8795518207283
715 17.9048951048951
716 17.895251396648
717 17.8716875871688
718 17.8871866295265
719 17.8623087621697
720 17.8388888888889
721 17.8515950069348
722 17.8531855955679
723 17.8713692946058
724 17.8701657458564
725 17.8868965517241
726 17.8911845730028
727 17.8858321870702
728 17.8626373626374
729 17.8683127572016
730 17.8657534246575
731 17.8686730506156
732 17.8524590163934
733 17.8635743519782
734 17.8460490463215
735 17.8353741496599
736 17.8301630434783
737 17.8276797829037
738 17.8252032520325
739 17.8470906630582
740 17.8378378378378
741 17.8151147098516
742 17.7924528301887
743 17.8115746971736
744 17.8346774193548
745 17.8107382550336
746 17.8096514745308
747 17.8246318607764
748 17.807486631016
749 17.8277703604806
750 17.8093333333333
751 17.8242343541944
752 17.8204787234043
753 17.800796812749
754 17.8196286472149
755 17.8437086092715
756 17.8505291005291
757 17.8295904887715
758 17.8298153034301
759 17.8076416337286
760 17.8157894736842
761 17.80814717477
762 17.8136482939633
763 17.8099606815203
764 17.7918848167539
765 17.8
766 17.7924281984334
767 17.8044328552803
768 17.828125
769 17.8335500650195
770 17.8545454545455
771 17.8378728923476
772 17.8549222797927
773 17.8576972833118
774 17.8475452196382
775 17.8503225806452
776 17.8737113402062
777 17.8854568854569
778 17.9023136246787
779 17.8793324775353
780 17.8794871794872
781 17.8860435339309
782 17.9066496163683
783 17.9029374201788
784 17.906887755102
785 17.8993630573248
786 17.8765903307888
787 17.8742058449809
788 17.8591370558376
789 17.8580481622307
790 17.8670886075949
791 17.8482932996207
792 17.8295454545455
793 17.828499369483
794 17.8501259445844
795 17.8691823899371
796 17.8856783919598
797 17.8858218318695
798 17.8822055137845
799 17.8698372966208
800 17.865
801 17.8439450686642
802 17.8241895261845
803 17.8393524283935
804 17.8594527363184
805 17.8409937888199
806 17.8523573200993
807 17.8438661710037
808 17.8329207920792
809 17.8331273176761
810 17.8333333333333
811 17.8298397040691
812 17.8189655172414
813 17.830258302583
814 17.8157248157248
815 17.8208588957055
816 17.8100490196078
817 17.8041615667075
818 17.8264058679707
819 17.8290598290598
820 17.8475609756098
821 17.8355663824604
822 17.8479318734793
823 17.865127582017
824 17.8604368932039
825 17.8763636363636
826 17.8668280871671
827 17.8778718258767
828 17.8659420289855
829 17.8841978287093
830 17.9036144578313
831 17.9085439229844
832 17.9050480769231
833 17.9195678271309
834 17.9196642685851
835 17.9077844311377
836 17.9090909090909
837 17.9032258064516
838 17.9188544152745
839 17.8986889153754
840 17.8797619047619
841 17.8727705112961
842 17.8847980997625
843 17.8695136417556
844 17.8708530805687
845 17.8816568047337
846 17.8770685579196
847 17.8783943329398
848 17.8785377358491
849 17.8657243816254
850 17.8458823529412
851 17.8331374853114
852 17.8368544600939
853 17.8487690504103
854 17.8536299765808
855 17.8350877192982
856 17.8247663551402
857 17.8319719953326
858 17.8426573426573
859 17.860302677532
860 17.8662790697674
861 17.8478513356562
862 17.8491879350348
863 17.8574739281576
864 17.8761574074074
865 17.8867052023121
866 17.8822170900693
867 17.883506343714
868 17.8790322580645
869 17.8584579976985
870 17.8586206896552
871 17.8541905855339
872 17.848623853211
873 17.836197021764
874 17.8398169336384
875 17.848
876 17.8310502283105
877 17.811858608894
878 17.8234624145786
879 17.8122866894198
880 17.8090909090909
881 17.8013620885358
882 17.7913832199546
883 17.8086070215176
884 17.8156108597285
885 17.8180790960452
886 17.823927765237
887 17.8049605411499
888 17.8186936936937
889 17.8267716535433
890 17.814606741573
891 17.7979797979798
892 17.8015695067265
893 17.7939529675252
894 17.8020134228188
895 17.7843575418994
896 17.7845982142857
897 17.7859531772575
898 17.7750556792873
899 17.7808676307008
900 17.7677777777778
901 17.7580466148724
902 17.7649667405765
903 17.765227021041
904 17.7809734513274
905 17.7635359116022
906 17.7472406181015
907 17.7276736493936
908 17.7235682819383
909 17.7161716171617
910 17.710989010989
911 17.6926454445664
912 17.7127192982456
913 17.7097480832421
914 17.6969365426696
915 17.7114754098361
916 17.7139737991266
917 17.7001090512541
918 17.7015250544662
919 17.7083786724701
920 17.7
921 17.7100977198697
922 17.704989154013
923 17.715059588299
924 17.6969696969697
925 17.6951351351351
926 17.682505399568
927 17.6688241639698
928 17.6821120689655
929 17.697524219591
930 17.6860215053763
931 17.6852846401719
932 17.6834763948498
933 17.6645230439443
934 17.6595289079229
935 17.6491978609626
936 17.642094017094
937 17.6435432230523
938 17.6396588486141
939 17.6389776357827
940 17.6510638297872
941 17.6620616365569
942 17.6539278131635
943 17.644750795334
944 17.6345338983051
945 17.6507936507937
946 17.6532769556025
947 17.6557550158395
948 17.6381856540084
949 17.654373024236
950 17.6368421052632
951 17.6340694006309
952 17.640756302521
953 17.6253934942288
954 17.6174004192872
955 17.5989528795812
956 17.586820083682
957 17.5914315569488
958 17.580375782881
959 17.5943691345151
960 17.6104166666667
961 17.6056191467222
962 17.6185031185031
963 17.6002076843198
964 17.5995850622407
965 17.6020725388601
966 17.5942028985507
967 17.5811789038263
968 17.5919421487603
969 17.5861713106295
970 17.5938144329897
971 17.5993820803296
972 17.6111111111111
973 17.6269270298047
974 17.6406570841889
975 17.6389743589744
976 17.625
977 17.6386898669396
978 17.6329243353783
979 17.6496424923391
980 17.6663265306122
981 17.6493374108053
982 17.6639511201629
983 17.6693794506612
984 17.6727642276423
985 17.6903553299492
986 17.6926977687627
987 17.6828774062817
988 17.6761133603239
989 17.675429726997
990 17.6757575757576
991 17.6589303733602
992 17.648185483871
993 17.6596173212487
994 17.6700201207243
995 17.686432160804
996 17.6857429718875
997 17.6990972918756
998 17.7134268537074
999 17.7177177177177
1000 17.708
};
\addplot [semithick, color8, forget plot]
table {%
1 30
2 20.5
3 25
4 20.75
5 20.4
6 20.5
7 18.1428571428571
8 16.375
9 15.1111111111111
10 16.8
11 16.9090909090909
12 18.3333333333333
13 18.5384615384615
14 18.3571428571429
15 19.3333333333333
16 20.3125
17 20.1176470588235
18 19.3333333333333
19 20.2105263157895
20 19.95
21 20.5238095238095
22 21.1818181818182
23 20.5652173913043
24 20.8333333333333
25 20.36
26 20.8076923076923
27 20.962962962963
28 20.2857142857143
29 19.9310344827586
30 19.8
31 19.9032258064516
32 19.6875
33 19.1515151515152
34 18.6764705882353
35 18.6285714285714
36 18.6666666666667
37 18.5675675675676
38 18.1842105263158
39 18.3333333333333
40 18.65
41 18.9756097560976
42 19.0714285714286
43 18.7674418604651
44 18.4772727272727
45 18.5111111111111
46 18.4782608695652
47 18.1489361702128
48 17.8333333333333
49 17.4897959183673
50 17.36
51 17.6862745098039
52 17.7884615384615
53 17.9811320754717
54 17.9074074074074
55 17.8727272727273
56 17.9285714285714
57 18.1052631578947
58 18.3620689655172
59 18.3050847457627
60 18.4333333333333
61 18.4426229508197
62 18.3225806451613
63 18.1587301587302
64 18.125
65 17.8769230769231
66 17.9848484848485
67 17.8059701492537
68 17.7941176470588
69 18.0289855072464
70 18.2142857142857
71 17.9859154929577
72 18.0138888888889
73 18.0684931506849
74 17.9864864864865
75 17.92
76 18.1052631578947
77 18.2987012987013
78 18.3717948717949
79 18.3164556962025
80 18.5
81 18.5802469135802
82 18.7439024390244
83 18.8433734939759
84 18.6547619047619
85 18.7058823529412
86 18.5348837209302
87 18.5632183908046
88 18.7045454545455
89 18.8988764044944
90 18.7222222222222
91 18.8131868131868
92 18.7391304347826
93 18.8387096774194
94 18.9787234042553
95 19.1052631578947
96 19.1770833333333
97 19.0103092783505
98 19.1836734693878
99 19.3535353535354
100 19.33
101 19.2772277227723
102 19.4411764705882
103 19.4174757281553
104 19.4423076923077
105 19.5809523809524
106 19.6320754716981
107 19.7289719626168
108 19.8425925925926
109 19.8440366972477
110 19.7181818181818
111 19.7567567567568
112 19.6339285714286
113 19.7610619469027
114 19.6228070175439
115 19.5130434782609
116 19.5948275862069
117 19.4529914529915
118 19.5338983050847
119 19.6050420168067
120 19.4416666666667
121 19.3388429752066
122 19.2540983606557
123 19.260162601626
124 19.2338709677419
125 19.208
126 19.1507936507937
127 19.1496062992126
128 19.28125
129 19.3488372093023
130 19.4615384615385
131 19.3511450381679
132 19.3106060606061
133 19.3157894736842
134 19.2238805970149
135 19.1925925925926
136 19.1102941176471
137 18.978102189781
138 19.0507246376812
139 18.9496402877698
140 18.8285714285714
141 18.8581560283688
142 18.8943661971831
143 19.013986013986
144 18.9722222222222
145 18.9931034482759
146 19.0753424657534
147 18.9795918367347
148 18.9256756756757
149 18.8456375838926
150 18.74
151 18.7417218543046
152 18.8092105263158
153 18.7254901960784
154 18.7532467532468
155 18.8193548387097
156 18.7948717948718
157 18.7834394904459
158 18.7784810126582
159 18.7672955974843
160 18.8375
161 18.7577639751553
162 18.8333333333333
163 18.8834355828221
164 18.7743902439024
165 18.7818181818182
166 18.7289156626506
167 18.7065868263473
168 18.7678571428571
169 18.7337278106509
170 18.7941176470588
171 18.7426900584795
172 18.6395348837209
173 18.5491329479769
174 18.632183908046
175 18.68
176 18.7215909090909
177 18.7231638418079
178 18.6685393258427
179 18.6256983240223
180 18.6444444444444
181 18.7016574585635
182 18.7142857142857
183 18.792349726776
184 18.820652173913
185 18.8648648648649
186 18.8333333333333
187 18.8502673796791
188 18.8404255319149
189 18.7460317460317
190 18.7789473684211
191 18.7486910994764
192 18.6927083333333
193 18.6580310880829
194 18.7164948453608
195 18.6717948717949
196 18.5918367346939
197 18.6751269035533
198 18.5959595959596
199 18.6231155778894
200 18.67
201 18.6218905472637
202 18.549504950495
203 18.6354679802956
204 18.5441176470588
205 18.4975609756098
206 18.4271844660194
207 18.4927536231884
208 18.4903846153846
209 18.4928229665072
210 18.4666666666667
211 18.436018957346
212 18.3584905660377
213 18.3380281690141
214 18.3271028037383
215 18.353488372093
216 18.3425925925926
217 18.2626728110599
218 18.2614678899083
219 18.3196347031963
220 18.3227272727273
221 18.2805429864253
222 18.3378378378378
223 18.3273542600897
224 18.2455357142857
225 18.2666666666667
226 18.3451327433628
227 18.4096916299559
228 18.4736842105263
229 18.4672489082969
230 18.5347826086957
231 18.5108225108225
232 18.4568965517241
233 18.3991416309013
234 18.3888888888889
235 18.3489361702128
236 18.271186440678
237 18.2067510548523
238 18.2436974789916
239 18.18410041841
240 18.1166666666667
241 18.1867219917012
242 18.1363636363636
243 18.0905349794239
244 18.1024590163934
245 18.0408163265306
246 17.9959349593496
247 18.0283400809717
248 18.0846774193548
249 18.0682730923695
250 18.12
251 18.1673306772908
252 18.1507936507937
253 18.1422924901186
254 18.1338582677165
255 18.1607843137255
256 18.19921875
257 18.1906614785992
258 18.2441860465116
259 18.1891891891892
260 18.2076923076923
261 18.176245210728
262 18.1068702290076
263 18.1026615969582
264 18.0719696969697
265 18.0188679245283
266 17.9812030075188
267 17.9850187265918
268 17.9813432835821
269 17.9182156133829
270 17.8592592592593
271 17.8560885608856
272 17.9080882352941
273 17.8974358974359
274 17.8941605839416
275 17.9454545454545
276 17.981884057971
277 17.9169675090253
278 17.8992805755396
279 17.9534050179211
280 18.0142857142857
281 18.0604982206406
282 18.0035460992908
283 18.0636042402827
284 18.1161971830986
285 18.1473684210526
286 18.2027972027972
287 18.191637630662
288 18.1319444444444
289 18.1072664359862
290 18.1586206896552
291 18.192439862543
292 18.2123287671233
293 18.2696245733788
294 18.2176870748299
295 18.2
296 18.2060810810811
297 18.2626262626263
298 18.2315436241611
299 18.180602006689
300 18.2133333333333
301 18.1860465116279
302 18.2384105960265
303 18.2409240924092
304 18.1875
305 18.2360655737705
306 18.2679738562091
307 18.2703583061889
308 18.2597402597403
309 18.2783171521036
310 18.3064516129032
311 18.2508038585209
312 18.2724358974359
313 18.2523961661342
314 18.203821656051
315 18.2444444444444
316 18.2278481012658
317 18.2492113564669
318 18.2547169811321
319 18.2476489028213
320 18.221875
321 18.196261682243
322 18.1894409937888
323 18.1486068111455
324 18.1574074074074
325 18.1723076923077
326 18.2177914110429
327 18.1651376146789
328 18.2103658536585
329 18.1945288753799
330 18.2060606060606
331 18.1722054380665
332 18.1415662650602
333 18.1051051051051
334 18.125748502994
335 18.1164179104478
336 18.0803571428571
337 18.0474777448071
338 18.0325443786982
339 18.0766961651917
340 18.0823529411765
341 18.0674486803519
342 18.0526315789474
343 18.0233236151604
344 18.0058139534884
345 18.0492753623188
346 18.0231213872832
347 18.0634005763689
348 18.0862068965517
349 18.0458452722063
350 18.0028571428571
351 18.0541310541311
352 18.0142045454545
353 18.0594900849858
354 18.0423728813559
355 18.0619718309859
356 18.0252808988764
357 18.0532212885154
358 18.0670391061453
359 18.0863509749304
360 18.1166666666667
361 18.1662049861496
362 18.1270718232044
363 18.0909090909091
364 18.0824175824176
365 18.0821917808219
366 18.1311475409836
367 18.1307901907357
368 18.0923913043478
369 18.0623306233062
370 18.1054054054054
371 18.1266846361186
372 18.1102150537634
373 18.0777479892761
374 18.120320855615
375 18.0986666666667
376 18.1382978723404
377 18.1564986737401
378 18.1772486772487
379 18.1477572559367
380 18.1342105263158
381 18.1391076115486
382 18.0916230366492
383 18.1279373368146
384 18.1484375
385 18.1428571428571
386 18.1787564766839
387 18.1627906976744
388 18.1649484536082
389 18.1928020565553
390 18.1692307692308
391 18.1636828644501
392 18.1352040816327
393 18.1145038167939
394 18.0786802030457
395 18.0810126582278
396 18.040404040404
397 18.0403022670025
398 18.0276381909548
399 18.062656641604
400 18.0175
401 18.0448877805486
402 18.0199004975124
403 18.0645161290323
404 18.0816831683168
405 18.0888888888889
406 18.0492610837438
407 18.012285012285
408 18.0441176470588
409 18.0537897310513
410 18.0804878048781
411 18.0997566909976
412 18.0898058252427
413 18.0968523002421
414 18.1256038647343
415 18.1373493975904
416 18.1586538461538
417 18.1654676258993
418 18.133971291866
419 18.1479713603819
420 18.1357142857143
421 18.1591448931116
422 18.1374407582938
423 18.1371158392435
424 18.1391509433962
425 18.1082352941176
426 18.1150234741784
427 18.1170960187354
428 18.0864485981308
429 18.0606060606061
430 18.0627906976744
431 18.0788863109049
432 18.0393518518519
433 18.0739030023095
434 18.0460829493088
435 18.0344827586207
436 18.0229357798165
437 17.9862700228833
438 18.0182648401826
439 18.0501138952164
440 18.0613636363636
441 18.0544217687075
442 18.0633484162896
443 18.058690744921
444 18.0585585585586
445 18.0584269662921
446 18.0448430493274
447 18.076062639821
448 18.0357142857143
449 18.075723830735
450 18.1133333333333
451 18.1130820399113
452 18.1482300884956
453 18.1655629139073
454 18.1718061674009
455 18.1406593406593
456 18.1271929824561
457 18.1203501094092
458 18.1222707423581
459 18.1525054466231
460 18.1130434782609
461 18.0911062906724
462 18.0822510822511
463 18.0907127429806
464 18.1077586206897
465 18.0752688172043
466 18.0364806866953
467 18.0192719486081
468 18.0021367521368
469 17.9936034115139
470 18.031914893617
471 18.0339702760085
472 18.0508474576271
473 18.0422832980973
474 18.0717299578059
475 18.0652631578947
476 18.0441176470588
477 18.0146750524109
478 18.0062761506276
479 18.0187891440501
480 18.05
481 18.0561330561331
482 18.0539419087137
483 18.0414078674948
484 18.0268595041322
485 17.9938144329897
486 17.9732510288066
487 17.9568788501027
488 17.9200819672131
489 17.9222903885481
490 17.9510204081633
491 17.9429735234216
492 17.9512195121951
493 17.9553752535497
494 17.9331983805668
495 17.9252525252525
496 17.9415322580645
497 17.9416498993964
498 17.9678714859438
499 17.9779559118236
500 17.948
501 17.9660678642715
502 17.9402390438247
503 17.9443339960239
504 17.9246031746032
505 17.9544554455446
506 17.9387351778656
507 17.9349112426036
508 17.9665354330709
509 17.9901768172888
510 17.9725490196078
511 17.9510763209393
512 17.9296875
513 17.9220272904483
514 17.908560311284
515 17.9320388349515
516 17.9205426356589
517 17.9110251450677
518 17.8938223938224
519 17.868978805395
520 17.8480769230769
521 17.8560460652591
522 17.8773946360153
523 17.8623326959847
524 17.8664122137405
525 17.8552380952381
526 17.8688212927757
527 17.8500948766603
528 17.844696969697
529 17.8185255198488
530 17.8018867924528
531 17.7702448210923
532 17.7518796992481
533 17.7392120075047
534 17.7172284644195
535 17.7233644859813
536 17.6996268656716
537 17.6815642458101
538 17.6840148698885
539 17.6938775510204
540 17.662962962963
541 17.6783733826248
542 17.6660516605166
543 17.6685082872928
544 17.6875
545 17.6623853211009
546 17.6483516483516
547 17.6819012797075
548 17.6642335766423
549 17.6794171220401
550 17.6690909090909
551 17.6932849364791
552 17.7264492753623
553 17.7287522603978
554 17.7003610108303
555 17.7099099099099
556 17.7410071942446
557 17.7127468581688
558 17.7311827956989
559 17.7584973166369
560 17.7607142857143
561 17.7629233511586
562 17.7953736654804
563 17.8010657193606
564 17.8049645390071
565 17.8141592920354
566 17.7826855123675
567 17.7530864197531
568 17.7834507042254
569 17.7574692442882
570 17.7543859649123
571 17.7250437828371
572 17.7255244755245
573 17.7050610820244
574 17.7299651567944
575 17.735652173913
576 17.7586805555556
577 17.7452339688042
578 17.7214532871972
579 17.7150259067358
580 17.701724137931
581 17.7039586919105
582 17.7027491408935
583 17.6912521440823
584 17.7157534246575
585 17.7264957264957
586 17.69795221843
587 17.6678023850085
588 17.6802721088435
589 17.6621392190153
590 17.6389830508475
591 17.6159052453469
592 17.6317567567568
593 17.6576728499157
594 17.6868686868687
595 17.6890756302521
596 17.7147651006711
597 17.6850921273032
598 17.6571906354515
599 17.6560934891486
600 17.655
601 17.6705490848586
602 17.6960132890365
603 17.6915422885572
604 17.6903973509934
605 17.6776859504132
606 17.6584158415842
607 17.6869851729819
608 17.7088815789474
609 17.7389162561576
610 17.7459016393443
611 17.7348608837971
612 17.7532679738562
613 17.742251223491
614 17.7703583061889
615 17.7691056910569
616 17.7824675324675
617 17.7974068071313
618 17.7766990291262
619 17.7867528271405
620 17.7677419354839
621 17.7713365539453
622 17.7733118971061
623 17.7736757624398
624 17.7852564102564
625 17.7696
626 17.7667731629393
627 17.7400318979266
628 17.7245222929936
629 17.7186009538951
630 17.7063492063492
631 17.6973058637084
632 17.6882911392405
633 17.6903633491311
634 17.6908517350158
635 17.7165354330709
636 17.690251572327
637 17.7174254317111
638 17.7460815047022
639 17.7276995305164
640 17.7484375
641 17.7581903276131
642 17.7710280373832
643 17.7527216174184
644 17.7267080745342
645 17.7317829457364
646 17.7538699690402
647 17.7372488408037
648 17.712962962963
649 17.7149460708783
650 17.7369230769231
651 17.7434715821813
652 17.7530674846626
653 17.7611026033691
654 17.7706422018349
655 17.7847328244275
656 17.7743902439024
657 17.7686453576865
658 17.7750759878419
659 17.7708649468892
660 17.7893939393939
661 17.7881996974281
662 17.7885196374622
663 17.7948717948718
664 17.7921686746988
665 17.7684210526316
666 17.7747747747748
667 17.7646176911544
668 17.7380239520958
669 17.7144992526158
670 17.7194029850746
671 17.7213114754098
672 17.7068452380952
673 17.7325408618128
674 17.7359050445104
675 17.7348148148148
676 17.7130177514793
677 17.7355982274741
678 17.7507374631268
679 17.7275405007364
680 17.7205882352941
681 17.6945668135095
682 17.6994134897361
683 17.7115666178624
684 17.7266081871345
685 17.7445255474453
686 17.7419825072886
687 17.7438136826783
688 17.7630813953488
689 17.7706821480406
690 17.7710144927536
691 17.7684515195369
692 17.7919075144509
693 17.7965367965368
694 17.792507204611
695 17.8187050359712
696 17.8204022988506
697 17.8235294117647
698 17.8295128939828
699 17.8426323319027
700 17.8214285714286
701 17.8473609129815
702 17.8433048433048
703 17.8321479374111
704 17.8366477272727
705 17.8340425531915
706 17.8215297450425
707 17.8458274398868
708 17.8587570621469
709 17.8730606488011
710 17.8845070422535
711 17.8902953586498
712 17.9101123595506
713 17.9200561009818
714 17.9397759103641
715 17.965034965035
716 17.9497206703911
717 17.9246861924686
718 17.9317548746518
719 17.9429763560501
720 17.9222222222222
721 17.9417475728155
722 17.9445983379501
723 17.9612724757953
724 17.9433701657459
725 17.928275862069
726 17.931129476584
727 17.9257221458047
728 17.9395604395604
729 17.9643347050754
730 17.9465753424658
731 17.9233926128591
732 17.9234972677596
733 17.9222373806276
734 17.908719346049
735 17.9197278911565
736 17.9279891304348
737 17.9470827679783
738 17.9512195121951
739 17.9282814614344
740 17.9391891891892
741 17.9298245614035
742 17.9272237196766
743 17.950201884253
744 17.9543010752688
745 17.9771812080537
746 17.985254691689
747 17.9812583668005
748 17.9719251336898
749 17.9759679572764
750 17.976
751 17.9667110519308
752 17.9428191489362
753 17.9349269588313
754 17.9111405835544
755 17.9245033112583
756 17.9365079365079
757 17.9154557463672
758 17.9129287598945
759 17.897233201581
760 17.875
761 17.8777923784494
762 17.8910761154856
763 17.9003931847969
764 17.9201570680628
765 17.9424836601307
766 17.943864229765
767 17.9517601043025
768 17.9544270833333
769 17.9596879063719
770 17.9467532467532
771 17.9403372243839
772 17.9611398963731
773 17.9469598965071
774 17.9483204134367
775 17.9393548387097
776 17.9510309278351
777 17.9369369369369
778 17.9254498714653
779 17.9460847240051
780 17.9564102564103
781 17.9513444302177
782 17.960358056266
783 17.9757343550447
784 17.9668367346939
785 17.9617834394904
786 17.9465648854962
787 17.9313850063532
788 17.9479695431472
789 17.93536121673
790 17.9316455696203
791 17.9127686472819
792 17.915404040404
793 17.9192938209332
794 17.9382871536524
795 17.9396226415094
796 17.928391959799
797 17.9347553324969
798 17.953634085213
799 17.9662077596996
800 17.98125
801 17.9637952559301
802 17.9451371571072
803 17.9227895392279
804 17.9179104477612
805 17.895652173913
806 17.8734491315136
807 17.8847583643123
808 17.8861386138614
809 17.902348578492
810 17.8913580246914
811 17.9001233045623
812 17.8990147783251
813 17.8794587945879
814 17.8648648648649
815 17.8871165644172
816 17.8737745098039
817 17.8812729498164
818 17.8679706601467
819 17.8827838827839
820 17.8926829268293
821 17.9074299634592
822 17.8880778588808
823 17.8979343863913
824 17.8980582524272
825 17.9042424242424
826 17.909200968523
827 17.9286577992745
828 17.9227053140097
829 17.9372738238842
830 17.9265060240964
831 17.927797833935
832 17.9459134615385
833 17.937575030012
834 17.9568345323741
835 17.9461077844311
836 17.9354066985646
837 17.9354838709677
838 17.9534606205251
839 17.9356376638856
840 17.9428571428571
841 17.9274673008323
842 17.91567695962
843 17.9157769869514
844 17.936018957346
845 17.9218934911243
846 17.903073286052
847 17.9114521841795
848 17.9056603773585
849 17.9175500588928
850 17.9341176470588
851 17.9142185663925
852 17.919014084507
853 17.9308323563892
854 17.9426229508197
855 17.9614035087719
856 17.9404205607477
857 17.9358226371062
858 17.9335664335664
859 17.9394644935972
860 17.9558139534884
861 17.9663182346109
862 17.9756380510441
863 17.981460023175
864 17.9884259259259
865 17.9687861271676
866 17.973441108545
867 17.9711649365629
868 17.9573732718894
869 17.9367088607595
870 17.9310344827586
871 17.9425947187141
872 17.9610091743119
873 17.9667812142039
874 17.9473684210526
875 17.9394285714286
876 17.9474885844749
877 17.9498289623717
878 17.9419134396355
879 17.9283276450512
880 17.9102272727273
881 17.9205448354143
882 17.9206349206349
883 17.9014722536806
884 17.893665158371
885 17.8802259887006
886 17.8984198645598
887 17.8838782412627
888 17.8704954954955
889 17.8740157480315
890 17.8876404494382
891 17.8900112233446
892 17.8934977578475
893 17.9025755879059
894 17.8970917225951
895 17.9005586592179
896 17.9051339285714
897 17.8996655518395
898 17.9064587973274
899 17.9265850945495
900 17.9366666666667
901 17.9345172031077
902 17.9368070953437
903 17.9202657807309
904 17.9358407079646
905 17.9524861878453
906 17.9679911699779
907 17.9625137816979
908 17.9449339207048
909 17.9262926292629
910 17.9263736263736
911 17.9308452250274
912 17.9265350877193
913 17.9430449069003
914 17.9584245076586
915 17.9704918032787
916 17.9585152838428
917 17.9749182115594
918 17.9945533769063
919 17.9956474428727
920 18.004347826087
921 18.0119435396308
922 17.9945770065076
923 17.9924160346696
924 17.9772727272727
925 17.9654054054054
926 17.9686825053996
927 17.9805825242718
928 17.9967672413793
929 17.9956942949408
930 18.0086021505376
931 17.9946294307197
932 18.0021459227468
933 18.0182207931404
934 18.0364025695931
935 18.0449197860963
936 18.034188034188
937 18.0373532550694
938 18.0319829424307
939 18.0511182108626
940 18.0702127659574
941 18.0892667375133
942 18.0828025477707
943 18.0795334040297
944 18.0646186440678
945 18.0814814814815
946 18.0919661733615
947 18.0897571277719
948 18.081223628692
949 18.086406743941
950 18.0957894736842
951 18.1083070452156
952 18.1197478991597
953 18.1374606505771
954 18.1320754716981
955 18.1151832460733
956 18.1004184100418
957 18.1086729362591
958 18.1106471816284
959 18.104275286757
960 18.109375
961 18.1134235171696
962 18.1008316008316
963 18.0851505711319
964 18.0684647302905
965 18.0632124352332
966 18.0734989648033
967 18.0734229576008
968 18.0909090909091
969 18.0835913312693
970 18.0711340206186
971 18.0679711637487
972 18.0627572016461
973 18.0534429599178
974 18.0431211498973
975 18.0287179487179
976 18.0215163934426
977 18.0276356192426
978 18.0092024539877
979 17.9989785495403
980 18.0173469387755
981 18.0071355759429
982 18.0224032586558
983 18.0162767039674
984 18.0243902439024
985 18.0223350253807
986 18.0385395537525
987 18.048632218845
988 18.0657894736842
989 18.0626895854398
990 18.0525252525253
991 18.0686175580222
992 18.0635080645161
993 18.0604229607251
994 18.0724346076459
995 18.0824120603015
996 18.0953815261044
997 18.1053159478435
998 18.0961923847695
999 18.0780780780781
1000 18.07
};
\addplot [semithick, color0, dashed]
table {%
1 18
2 18
3 18
4 18
5 18
6 18
7 18
8 18
9 18
10 18
11 18
12 18
13 18
14 18
15 18
16 18
17 18
18 18
19 18
20 18
21 18
22 18
23 18
24 18
25 18
26 18
27 18
28 18
29 18
30 18
31 18
32 18
33 18
34 18
35 18
36 18
37 18
38 18
39 18
40 18
41 18
42 18
43 18
44 18
45 18
46 18
47 18
48 18
49 18
50 18
51 18
52 18
53 18
54 18
55 18
56 18
57 18
58 18
59 18
60 18
61 18
62 18
63 18
64 18
65 18
66 18
67 18
68 18
69 18
70 18
71 18
72 18
73 18
74 18
75 18
76 18
77 18
78 18
79 18
80 18
81 18
82 18
83 18
84 18
85 18
86 18
87 18
88 18
89 18
90 18
91 18
92 18
93 18
94 18
95 18
96 18
97 18
98 18
99 18
100 18
101 18
102 18
103 18
104 18
105 18
106 18
107 18
108 18
109 18
110 18
111 18
112 18
113 18
114 18
115 18
116 18
117 18
118 18
119 18
120 18
121 18
122 18
123 18
124 18
125 18
126 18
127 18
128 18
129 18
130 18
131 18
132 18
133 18
134 18
135 18
136 18
137 18
138 18
139 18
140 18
141 18
142 18
143 18
144 18
145 18
146 18
147 18
148 18
149 18
150 18
151 18
152 18
153 18
154 18
155 18
156 18
157 18
158 18
159 18
160 18
161 18
162 18
163 18
164 18
165 18
166 18
167 18
168 18
169 18
170 18
171 18
172 18
173 18
174 18
175 18
176 18
177 18
178 18
179 18
180 18
181 18
182 18
183 18
184 18
185 18
186 18
187 18
188 18
189 18
190 18
191 18
192 18
193 18
194 18
195 18
196 18
197 18
198 18
199 18
200 18
201 18
202 18
203 18
204 18
205 18
206 18
207 18
208 18
209 18
210 18
211 18
212 18
213 18
214 18
215 18
216 18
217 18
218 18
219 18
220 18
221 18
222 18
223 18
224 18
225 18
226 18
227 18
228 18
229 18
230 18
231 18
232 18
233 18
234 18
235 18
236 18
237 18
238 18
239 18
240 18
241 18
242 18
243 18
244 18
245 18
246 18
247 18
248 18
249 18
250 18
251 18
252 18
253 18
254 18
255 18
256 18
257 18
258 18
259 18
260 18
261 18
262 18
263 18
264 18
265 18
266 18
267 18
268 18
269 18
270 18
271 18
272 18
273 18
274 18
275 18
276 18
277 18
278 18
279 18
280 18
281 18
282 18
283 18
284 18
285 18
286 18
287 18
288 18
289 18
290 18
291 18
292 18
293 18
294 18
295 18
296 18
297 18
298 18
299 18
300 18
301 18
302 18
303 18
304 18
305 18
306 18
307 18
308 18
309 18
310 18
311 18
312 18
313 18
314 18
315 18
316 18
317 18
318 18
319 18
320 18
321 18
322 18
323 18
324 18
325 18
326 18
327 18
328 18
329 18
330 18
331 18
332 18
333 18
334 18
335 18
336 18
337 18
338 18
339 18
340 18
341 18
342 18
343 18
344 18
345 18
346 18
347 18
348 18
349 18
350 18
351 18
352 18
353 18
354 18
355 18
356 18
357 18
358 18
359 18
360 18
361 18
362 18
363 18
364 18
365 18
366 18
367 18
368 18
369 18
370 18
371 18
372 18
373 18
374 18
375 18
376 18
377 18
378 18
379 18
380 18
381 18
382 18
383 18
384 18
385 18
386 18
387 18
388 18
389 18
390 18
391 18
392 18
393 18
394 18
395 18
396 18
397 18
398 18
399 18
400 18
401 18
402 18
403 18
404 18
405 18
406 18
407 18
408 18
409 18
410 18
411 18
412 18
413 18
414 18
415 18
416 18
417 18
418 18
419 18
420 18
421 18
422 18
423 18
424 18
425 18
426 18
427 18
428 18
429 18
430 18
431 18
432 18
433 18
434 18
435 18
436 18
437 18
438 18
439 18
440 18
441 18
442 18
443 18
444 18
445 18
446 18
447 18
448 18
449 18
450 18
451 18
452 18
453 18
454 18
455 18
456 18
457 18
458 18
459 18
460 18
461 18
462 18
463 18
464 18
465 18
466 18
467 18
468 18
469 18
470 18
471 18
472 18
473 18
474 18
475 18
476 18
477 18
478 18
479 18
480 18
481 18
482 18
483 18
484 18
485 18
486 18
487 18
488 18
489 18
490 18
491 18
492 18
493 18
494 18
495 18
496 18
497 18
498 18
499 18
500 18
501 18
502 18
503 18
504 18
505 18
506 18
507 18
508 18
509 18
510 18
511 18
512 18
513 18
514 18
515 18
516 18
517 18
518 18
519 18
520 18
521 18
522 18
523 18
524 18
525 18
526 18
527 18
528 18
529 18
530 18
531 18
532 18
533 18
534 18
535 18
536 18
537 18
538 18
539 18
540 18
541 18
542 18
543 18
544 18
545 18
546 18
547 18
548 18
549 18
550 18
551 18
552 18
553 18
554 18
555 18
556 18
557 18
558 18
559 18
560 18
561 18
562 18
563 18
564 18
565 18
566 18
567 18
568 18
569 18
570 18
571 18
572 18
573 18
574 18
575 18
576 18
577 18
578 18
579 18
580 18
581 18
582 18
583 18
584 18
585 18
586 18
587 18
588 18
589 18
590 18
591 18
592 18
593 18
594 18
595 18
596 18
597 18
598 18
599 18
600 18
601 18
602 18
603 18
604 18
605 18
606 18
607 18
608 18
609 18
610 18
611 18
612 18
613 18
614 18
615 18
616 18
617 18
618 18
619 18
620 18
621 18
622 18
623 18
624 18
625 18
626 18
627 18
628 18
629 18
630 18
631 18
632 18
633 18
634 18
635 18
636 18
637 18
638 18
639 18
640 18
641 18
642 18
643 18
644 18
645 18
646 18
647 18
648 18
649 18
650 18
651 18
652 18
653 18
654 18
655 18
656 18
657 18
658 18
659 18
660 18
661 18
662 18
663 18
664 18
665 18
666 18
667 18
668 18
669 18
670 18
671 18
672 18
673 18
674 18
675 18
676 18
677 18
678 18
679 18
680 18
681 18
682 18
683 18
684 18
685 18
686 18
687 18
688 18
689 18
690 18
691 18
692 18
693 18
694 18
695 18
696 18
697 18
698 18
699 18
700 18
701 18
702 18
703 18
704 18
705 18
706 18
707 18
708 18
709 18
710 18
711 18
712 18
713 18
714 18
715 18
716 18
717 18
718 18
719 18
720 18
721 18
722 18
723 18
724 18
725 18
726 18
727 18
728 18
729 18
730 18
731 18
732 18
733 18
734 18
735 18
736 18
737 18
738 18
739 18
740 18
741 18
742 18
743 18
744 18
745 18
746 18
747 18
748 18
749 18
750 18
751 18
752 18
753 18
754 18
755 18
756 18
757 18
758 18
759 18
760 18
761 18
762 18
763 18
764 18
765 18
766 18
767 18
768 18
769 18
770 18
771 18
772 18
773 18
774 18
775 18
776 18
777 18
778 18
779 18
780 18
781 18
782 18
783 18
784 18
785 18
786 18
787 18
788 18
789 18
790 18
791 18
792 18
793 18
794 18
795 18
796 18
797 18
798 18
799 18
800 18
801 18
802 18
803 18
804 18
805 18
806 18
807 18
808 18
809 18
810 18
811 18
812 18
813 18
814 18
815 18
816 18
817 18
818 18
819 18
820 18
821 18
822 18
823 18
824 18
825 18
826 18
827 18
828 18
829 18
830 18
831 18
832 18
833 18
834 18
835 18
836 18
837 18
838 18
839 18
840 18
841 18
842 18
843 18
844 18
845 18
846 18
847 18
848 18
849 18
850 18
851 18
852 18
853 18
854 18
855 18
856 18
857 18
858 18
859 18
860 18
861 18
862 18
863 18
864 18
865 18
866 18
867 18
868 18
869 18
870 18
871 18
872 18
873 18
874 18
875 18
876 18
877 18
878 18
879 18
880 18
881 18
882 18
883 18
884 18
885 18
886 18
887 18
888 18
889 18
890 18
891 18
892 18
893 18
894 18
895 18
896 18
897 18
898 18
899 18
900 18
901 18
902 18
903 18
904 18
905 18
906 18
907 18
908 18
909 18
910 18
911 18
912 18
913 18
914 18
915 18
916 18
917 18
918 18
919 18
920 18
921 18
922 18
923 18
924 18
925 18
926 18
927 18
928 18
929 18
930 18
931 18
932 18
933 18
934 18
935 18
936 18
937 18
938 18
939 18
940 18
941 18
942 18
943 18
944 18
945 18
946 18
947 18
948 18
949 18
950 18
951 18
952 18
953 18
954 18
955 18
956 18
957 18
958 18
959 18
960 18
961 18
962 18
963 18
964 18
965 18
966 18
967 18
968 18
969 18
970 18
971 18
972 18
973 18
974 18
975 18
976 18
977 18
978 18
979 18
980 18
981 18
982 18
983 18
984 18
985 18
986 18
987 18
988 18
989 18
990 18
991 18
992 18
993 18
994 18
995 18
996 18
997 18
998 18
999 18
1000 18
};
\addlegendentry{$v_{p_{e}}$ (valor promedio esperado)}
\end{axis}

\end{tikzpicture}

    \caption{valor promedio para 10 corridas del experimento}
  \end{mytikzresize}
\end{figure}

\begin{figure}[!htbp]
  \begin{mytikzresize}{0.6\textwidth}
    \centering
    % This file was created by tikzplotlib v0.9.1.
\begin{tikzpicture}

\definecolor{color0}{rgb}{0.12156862745098,0.466666666666667,0.705882352941177}
\definecolor{color1}{rgb}{1,0.498039215686275,0.0549019607843137}
\definecolor{color2}{rgb}{0.172549019607843,0.627450980392157,0.172549019607843}
\definecolor{color3}{rgb}{0.83921568627451,0.152941176470588,0.156862745098039}
\definecolor{color4}{rgb}{0.580392156862745,0.403921568627451,0.741176470588235}
\definecolor{color5}{rgb}{0.549019607843137,0.337254901960784,0.294117647058824}
\definecolor{color6}{rgb}{0.890196078431372,0.466666666666667,0.76078431372549}
\definecolor{color7}{rgb}{0.737254901960784,0.741176470588235,0.133333333333333}
\definecolor{color8}{rgb}{0.0901960784313725,0.745098039215686,0.811764705882353}

\begin{axis}[
legend cell align={left},
legend style={fill opacity=0.5, draw opacity=1, text opacity=1, draw=white!80!black},
scaled ticks=false,
tick align=outside,
tick pos=left,
width=\figW,
x grid style={white!69.0196078431373!black},
xlabel={\(\displaystyle n\) (número de tiradas)},
xmajorgrids,
xmin=-48.95, xmax=1049.95,
xtick style={color=black},
xticklabel style={/pgf/number format/.cd,fixed,precision=2},
y grid style={white!69.0196078431373!black},
ylabel={\(\displaystyle v_{d}\) (valor del desvío)},
ymajorgrids,
ymin=-0.8, ymax=16.8,
ytick style={color=black},
yticklabel style={/pgf/number format/.cd,fixed,precision=2}
]
\addplot [semithick, color0, forget plot]
table {%
1 0
2 14
3 13.1993265821489
4 12.4398553046247
5 12.0399335546339
6 12.5399362039845
7 11.6110433006949
8 11.2249721603218
9 11.2885826730122
10 11.3088460949825
11 10.9635006775864
12 10.5026451694588
13 10.1188207091725
14 9.77006047231301
15 9.81472816174187
16 9.65012953280939
17 10.9944937461242
18 10.8384603252594
19 11.8835531961235
20 11.8642319599711
21 12.6390676770888
22 12.4797356403448
23 12.2822816424329
24 12.1151162465189
25 12.4332457548301
26 12.2633817743875
27 12.1928941054479
28 12.4304185819484
29 12.4653145402511
30 12.813361082176
31 12.7781844531543
32 12.6014632086913
33 12.7614979934822
34 12.5724543744068
35 12.4337920070053
36 12.3306741377492
37 12.3126426382115
38 12.3688689728524
39 12.2105137870116
40 12.0580885715772
41 11.9466392417747
42 11.9712666201115
43 12.217206743628
44 12.193515415245
45 12.1470415033435
46 12.0677044048578
47 11.948660359307
48 11.8352221203866
49 11.9991322746108
50 12.0396179341373
51 11.9295638824402
52 11.8258442284674
53 11.9877711511928
54 11.8849797426931
55 11.7930642174273
56 11.7108008753824
57 11.7053967344812
58 11.6061635394577
59 11.5749186494281
60 11.6706064776238
61 11.5749778778572
62 11.5934088167232
63 11.6858820270579
64 11.7911099323813
65 11.7102493922377
66 11.6219627587371
67 11.5376513244337
68 11.5439115755305
69 11.6572020300003
70 11.7824151226499
71 11.7892119646005
72 11.8915819898877
73 11.9281485609282
74 11.8731193966645
75 11.8572546944429
76 11.8950497746447
77 11.8414783941142
78 11.8355277571917
79 11.7606449818605
80 11.7944637436384
81 11.7945502645371
82 11.7875111619552
83 11.7875912425523
84 11.7198193956505
85 11.7898319923697
86 11.76574121363
87 11.7519390910854
88 11.7730782678459
89 11.7122638332857
90 11.7475292910656
91 11.75487268155
92 11.6959097676498
93 11.6412835212034
94 11.622344077121
95 11.6446271181271
96 11.7192083243707
97 11.6598895235209
98 11.6061100684273
99 11.6627524152141
100 11.6739496315514
101 11.6160599912628
102 11.6262274924198
103 11.633887853702
104 11.582380358256
105 11.5888131216773
106 11.5448881701915
107 11.5320968042955
108 11.5356250250601
109 11.540854121878
110 11.4944327163664
111 11.4525731476368
112 11.4774311316885
113 11.4656050288017
114 11.5520021408674
115 11.5402673008983
116 11.4936102550248
117 11.5624694837433
118 11.6214222345796
119 11.5858858929432
120 11.5983445897344
121 11.5561602547449
122 11.5314235972866
123 11.4857995131126
124 11.5514137095364
125 11.5705944531817
126 11.5943471730565
127 11.6039057778516
128 11.5602616076584
129 11.5352483896678
130 11.4952421338559
131 11.5318122303583
132 11.5389705804692
133 11.5721687022382
134 11.6080380390129
135 11.5657836975158
136 11.5253125215641
137 11.483953459666
138 11.4907979402254
139 11.4727158035268
140 11.5167225354779
141 11.475878613431
142 11.46314484529
143 11.4505957311688
144 11.4834300075219
145 11.4455210644852
146 11.4097172440278
147 11.3969720256974
148 11.4275275402416
149 11.3921425012938
150 11.3821087677108
151 11.4457319771207
152 11.4096474705252
153 11.4709736821447
154 11.4540816232621
155 11.4287622995805
156 11.4499956217029
157 11.4258355511227
158 11.4336244466005
159 11.4725256239108
160 11.4786486573987
161 11.4460169306292
162 11.4452368311459
163 11.4175322929215
164 11.3914630229698
165 11.4270216055272
166 11.4756253326156
167 11.4831253533091
168 11.4563510963211
169 11.4681303652835
170 11.4612390255155
171 11.4296715920462
172 11.3972041085829
173 11.3758106580543
174 11.3873143219681
175 11.425213792489
176 11.4266830595673
177 11.4115427300034
178 11.3863488233866
179 11.362513235707
180 11.4036976892434
181 11.3884276885433
182 11.4333569862223
183 11.4027939241957
184 11.4019291120569
185 11.3744629586098
186 11.418009407811
187 11.4092660486426
188 11.3815369936052
189 11.3927004173248
190 11.4033041817005
191 11.4196247194365
192 11.3952036979471
193 11.3890819154794
194 11.4164738525452
195 11.4061127595699
196 11.4215813171129
197 11.3925571276457
198 11.4345218960268
199 11.4241785595822
200 11.4276320819319
201 11.4483438806434
202 11.4564178455561
203 11.4300258119941
204 11.450446326184
205 11.4226988396126
206 11.4161175798783
207 11.3887001075537
208 11.3785351161972
209 11.368221688842
210 11.3413097181255
211 11.3197323357968
212 11.2981299076016
213 11.2724539647321
214 11.271292483065
215 11.2450496289426
216 11.2438330910271
217 11.2640868880812
218 11.243253080702
219 11.2175546092247
220 11.2376975892501
221 11.2251655050367
222 11.2241116969713
223 11.2327372072709
224 11.2520264614778
225 11.2470352746462
226 11.2378553237784
227 11.2199175274037
228 11.2025640600921
229 11.2064364744117
230 11.1838257502564
231 11.1974732577235
232 11.1926700151914
233 11.2009109604454
234 11.1961840146164
235 11.1753511990817
236 11.2063798927949
237 11.2095419355419
238 11.2274513333101
239 11.2041865691905
240 11.1895218533035
241 11.231633193858
242 11.2224261569913
243 11.22262287242
244 11.2009831254981
245 11.2011229876706
246 11.2079068350418
247 11.202267468681
248 11.1829855979275
249 11.1794519311782
250 11.1883482248275
251 11.1798152948723
252 11.1644771434502
253 11.1471714917883
254 11.1269907486187
255 11.1233246214597
256 11.1317461112071
257 11.1723466478139
258 11.1532373256547
259 11.1334173646518
260 11.1297485045776
261 11.1326452707938
262 11.155976019336
263 11.1353612426966
264 11.1144593640862
265 11.0950994477325
266 11.0852875459948
267 11.0686229117863
268 11.0585048549964
269 11.0385312835088
270 11.0182016761168
271 11.00183620023
272 10.9987164780817
273 11.0061819448968
274 10.9917482938659
275 11.0184254398509
276 11.0501538222734
277 11.0304624649571
278 11.0674427915208
279 11.0763869475368
280 11.1126737492443
281 11.1183296446661
282 11.100638895765
283 11.1055399246757
284 11.1111111868787
285 11.1347768898429
286 11.1294035841145
287 11.1186858270004
288 11.147369226126
289 11.1562964032812
290 11.1844066113725
291 11.1891891876473
292 11.2114160807103
293 11.1969956931954
294 11.2057816676526
295 11.2003529307876
296 11.232575175328
297 11.2445976099935
298 11.2265330118657
299 11.2080090713635
300 11.2061986814838
301 11.1918296155246
302 11.175253369929
303 11.1571931589898
304 11.1546314547489
305 11.1637057341589
306 11.1651726904729
307 11.1512430348597
308 11.1599859266465
309 11.1720628967802
310 11.1761508380404
311 11.1744314873814
312 11.1623526176113
313 11.1739669127705
314 11.156560061551
315 11.1570506085472
316 11.1449873696922
317 11.1381530856404
318 11.1553149919275
319 11.1766437358828
320 11.1930782182561
321 11.213165312651
322 11.2059643585617
323 11.1903663948416
324 11.2063863570582
325 11.1894345770395
326 11.2189015054057
327 11.2017793438436
328 11.1855912856353
329 11.1785313193572
330 11.1673701517645
331 11.1507836043527
332 11.1705180858752
333 11.1660317862451
334 11.1858948025135
335 11.1788019806432
336 11.1828098739988
337 11.1806206501102
338 11.1995610186721
339 11.1907132822507
340 11.2149199313122
341 11.2300704049267
342 11.2136759738492
343 11.1973470617111
344 11.1849450093457
345 11.208094113675
346 11.2362969288162
347 11.2201220004799
348 11.2432551968894
349 11.2659852050412
350 11.250141405007
351 11.2540509828887
352 11.2380815970945
353 11.2522280605411
354 11.2478100336537
355 11.2373170368937
356 11.2267472805214
357 11.2446820848475
358 11.2304051546496
359 11.2479991608256
360 11.2378161046739
361 11.2336498079344
362 11.2319205478379
363 11.216742537906
364 11.2348527419507
365 11.2197169823907
366 11.2232748311995
367 11.2079970028712
368 11.2295021068846
369 11.2334248603648
370 11.2546424152353
371 11.2523307621588
372 11.2772822157086
373 11.2703266044711
374 11.2809017130056
375 11.2659057731231
376 11.2643588261649
377 11.2519562490054
378 11.240432852067
379 11.2496843746069
380 11.2567554355724
381 11.262679744592
382 11.2611910536458
383 11.2825502005294
384 11.2714879662014
385 11.2877313517368
386 11.307560113892
387 11.2996904896232
388 11.2901884291943
389 11.2807012707098
390 11.266275683843
391 11.2623800010847
392 11.2721272859476
393 11.2886996140089
394 11.2777213633642
395 11.2657528087714
396 11.278498124615
397 11.2676638986052
398 11.2837428052781
399 11.2759151133771
400 11.2624861265175
401 11.2783613754152
402 11.3016400313991
403 11.2907774020134
404 11.2844479620083
405 11.2929601766044
406 11.3058383428783
407 11.3220054207726
408 11.334448251948
409 11.33262621176
410 11.3234078675407
411 11.3117639253143
412 11.3058492123121
413 11.3177393537493
414 11.3138026132369
415 11.3166678193156
416 11.3145948708447
417 11.322965997102
418 11.320758440463
419 11.3072537812027
420 11.2971458579213
421 11.3053513543483
422 11.3017418695706
423 11.2884192937502
424 11.2751123935579
425 11.2696367030479
426 11.287742178949
427 11.2911415803378
428 11.310060849401
429 11.296887699875
430 11.2869694225362
431 11.2956954080113
432 11.3141091720924
433 11.3287103676666
434 11.3214430500095
435 11.3105074280434
436 11.3067120711003
437 11.2968276526855
438 11.295068812133
439 11.2954777968945
440 11.2826560014157
441 11.2832179472402
442 11.2914397422779
443 11.278708696307
444 11.2900856330308
445 11.301209166285
446 11.2897130301012
447 11.2980273223875
448 11.2860160359609
449 11.300147997928
450 11.317399692618
451 11.3069200610187
452 11.2997639793024
453 11.2885568679779
454 11.278181983325
455 11.2784399338786
456 11.274612069087
457 11.2822460034274
458 11.2907035667866
459 11.2914423490306
460 11.2830853547633
461 11.283256194749
462 11.2710788389161
463 11.2595722371459
464 11.2487016607789
465 11.259117963263
466 11.2472959863411
467 11.2373168712102
468 11.2255649765951
469 11.2174136916321
470 11.207253883565
471 11.1971205085696
472 11.1870134595911
473 11.1752269119981
474 11.1722850870889
475 11.1610291525984
476 11.1503380831564
477 11.144185574433
478 11.1431855969852
479 11.1606426120941
480 11.1519422511158
481 11.1403589745865
482 11.1370185466195
483 11.1256613790802
484 11.1146878089228
485 11.1072882595918
486 11.0998975575029
487 11.0904477339719
488 11.0910930692898
489 11.1021671841067
490 11.0958883387416
491 11.0845984259752
492 11.0878467862197
493 11.0766253923667
494 11.0871264001992
495 11.0776841539959
496 11.080953115585
497 11.0942185521434
498 11.0831084529181
499 11.079950639034
500 11.0714616921164
501 11.0771580949462
502 11.0852943959648
503 11.0933164447325
504 11.0834370501913
505 11.0789822927149
506 11.0893200611735
507 11.0902553548109
508 11.1094988789743
509 11.0985961257493
510 11.0955385875352
511 11.1119268247426
512 11.116909557093
513 11.1071977650828
514 11.1102319920335
515 11.1005454609062
516 11.1136488298435
517 11.121207617077
518 11.1110058315881
519 11.1065730390587
520 11.1054252360977
521 11.0949610767555
522 11.0891977398462
523 11.0796461776307
524 11.0954073741386
525 11.0848544912854
526 11.0779624287247
527 11.0736051628554
528 11.0631379418452
529 11.0730837581566
530 11.0673436151419
531 11.0680098624401
532 11.0586991362974
533 11.0561089573278
534 11.0636995731931
535 11.0737244669417
536 11.0786854350109
537 11.0775497491272
538 11.0718901969104
539 11.0745349067918
540 11.0867318964004
541 11.0775755632033
542 11.067510303436
543 11.0575331519562
544 11.0550460379488
545 11.0451116810164
546 11.0502987211267
547 11.0462660077568
548 11.0561385042606
549 11.0713241484509
550 11.0835595618909
551 11.0793602764909
552 11.0709388354928
553 11.0614117237164
554 11.0642775731187
555 11.0589266557698
556 11.0563445133661
557 11.0659044633976
558 11.0605091243881
559 11.0522229265973
560 11.0425220275801
561 11.0336889742098
562 11.0243872055749
563 11.0203595494687
564 11.032316116456
565 11.0373143087496
566 11.0380312194618
567 11.042990982602
568 11.042127204473
569 11.0568311086999
570 11.0686891153342
571 11.0599452485991
572 11.0507533716344
573 11.0497373585271
574 11.0668674485553
575 11.0815003112998
576 11.0763469826793
577 11.0691478479302
578 11.0758926954647
579 11.0903201246728
580 11.0878070384865
581 11.0885365839036
582 11.0875472552184
583 11.0796074076676
584 11.0886612622891
585 11.0911636855682
586 11.0817122061732
587 11.0765370574888
588 11.0694826758986
589 11.0616663513415
590 11.0682821031941
591 11.0604165891674
592 11.0612832732309
593 11.0521363307715
594 11.047052358058
595 11.0431750540479
596 11.0361392239255
597 11.0426357864209
598 11.0367036583279
599 11.0482878932264
600 11.059400099262
601 11.0662018665788
602 11.0670480800354
603 11.0578824521929
604 11.0668097229496
605 11.0755030884986
606 11.0672442782449
607 11.0596711526344
608 11.0573512352264
609 11.050482477349
610 11.0570931679827
611 11.0577775696912
612 11.0712176708781
613 11.082036867516
614 11.0762037001061
615 11.0672075439594
616 11.0596478244873
617 11.0528210754424
618 11.0503318217142
619 11.0414129104186
620 11.0354763274248
621 11.0396133668454
622 11.033058322483
623 11.0251637554261
624 11.0382004905253
625 11.0293982718914
626 11.026013657086
627 11.0287730597841
628 11.0314862733525
629 11.0248066411237
630 11.0192021866574
631 11.0157880119603
632 11.0072116605568
633 11.0098825077356
634 11.0107816273008
635 11.0193879382578
636 11.0300367125834
637 11.0325839432638
638 11.0269683777303
639 11.0246960298745
640 11.023947543126
641 11.0175044161402
642 11.0304866421452
643 11.0297362618187
644 11.0211891411266
645 11.0135169967282
646 11.0159955336592
647 11.0137995214921
648 11.0129870785642
649 11.0258029009163
650 11.0202083336013
651 11.0328761652903
652 11.0265880908165
653 11.0203090464858
654 11.014039020338
655 11.0107156011873
656 11.0024881546921
657 11.0031592053111
658 11.0025076908872
659 10.9949730940664
660 10.991687630646
661 10.9838171886753
662 10.9964303214679
663 11.0004830038527
664 10.9959487213266
665 10.9965480067564
666 10.9971188754372
667 10.9888998035348
668 10.9883701547706
669 10.9816059173236
670 10.9734177281384
671 10.9728807732616
672 10.9651039543644
673 10.9582635799379
674 10.9540473321723
675 10.9459543735015
676 10.9611830069575
677 10.9673673208179
678 10.9799474063675
679 10.9766747388898
680 10.9827324952529
681 10.9766961809138
682 10.9790606241965
683 10.9814652888102
684 10.9747849316212
685 10.9740981205464
686 10.9708750190389
687 10.9750093602277
688 10.9690271300155
689 10.965823227201
690 10.9716975645529
691 10.9684627239334
692 10.9726319454032
693 10.9648522971804
694 10.9571059356746
695 10.95004262867
696 10.9458952030495
697 10.9400387569344
698 10.945978779501
699 10.9418230736288
700 10.9344102982936
701 10.9443190793053
702 10.9502490972841
703 10.952615682619
704 10.9456300952398
705 10.9391755674028
706 10.9315723355893
707 10.9251452327694
708 10.9175744103906
709 10.9098880118093
710 10.9022190076068
711 10.9063054115527
712 10.902272724611
713 10.9004241860769
714 10.8940804146797
715 10.8868667293106
716 10.8800539283578
717 10.8920366222284
718 10.8857632226014
719 10.8897684926943
720 10.8994385367626
721 10.9072817983253
722 10.919030213391
723 10.924910301336
724 10.9289605094967
725 10.9233521138473
726 10.9256244364971
727 10.9194065071275
728 10.9164851723296
729 10.9188421508331
730 10.9126252285539
731 10.907063476372
732 10.8996259285384
733 10.8929581289304
734 10.8856765773718
735 10.8784090909063
736 10.8722819295176
737 10.8780011905207
738 10.8788249149585
739 10.8828467713335
740 10.8810659765473
741 10.8772368503886
742 10.8766991435385
743 10.8882173459939
744 10.8939653761324
745 10.8979330940856
746 10.9035802614334
747 10.9030374671782
748 10.9053413788031
749 10.9167771278496
750 10.9096299153037
751 10.9210791307218
752 10.9145597880133
753 10.9073247942141
754 10.9186957311874
755 10.9299756336558
756 10.939117291355
757 10.9330785939945
758 10.9283595222423
759 10.9215574513692
760 10.9168570729767
761 10.9175227734247
762 10.9285360577618
763 10.9324795317631
764 10.9270665766132
765 10.9279729172418
766 10.9242776160823
767 10.9215453632725
768 10.9151923273616
769 10.9188304467611
770 10.9297501505686
771 10.9320792527764
772 10.925393795037
773 10.9366090777851
774 10.9477368375296
775 10.9587778168139
776 10.9582137073749
777 10.9523508567748
778 10.9593881619019
779 10.9530770362803
780 10.9467766098499
781 10.9556443286108
782 10.955105565403
783 10.9680678019934
784 10.968746940822
785 10.9617738908977
786 10.9745898103814
787 10.9679647106032
788 10.9651352846729
789 10.9583047839363
790 10.9604652908133
791 10.9538831452151
792 10.9510824355193
793 10.9530825907446
794 10.9526019712347
795 10.9488892452648
796 10.9523500009666
797 10.9486245260426
798 10.9424259943312
799 10.9446428495119
800 10.949657528891
801 10.9468396473507
802 10.9406666849544
803 10.9338736848232
804 10.9271810975179
805 10.9395746457546
806 10.9449233789066
807 10.9384603933653
808 10.9316974714315
809 10.9339395264038
810 10.9271967594606
811 10.9291122405323
812 10.9393678400463
813 10.9326450593993
814 10.9301240034651
815 10.9408477332041
816 10.9445097546315
817 10.9379197481942
818 10.9318780882716
819 10.9327130682815
820 10.9260640520383
821 10.9256508023323
822 10.9252233752253
823 10.9202272108446
824 10.9147578929805
825 10.9098581548377
826 10.9083149425107
827 10.9077710872201
828 10.9012974143391
829 10.8958697411885
830 10.8933459602594
831 10.8899302432007
832 10.8950282207796
833 10.8888353912742
834 10.8824268921267
835 10.8927067798499
836 10.897663249325
837 10.9043966599291
838 10.8982399923249
839 10.9066697614049
840 10.9087118548005
841 10.9071414667924
842 10.909148949773
843 10.9026915029076
844 10.9128926784863
845 10.9103501155999
846 10.9069511105142
847 10.9021474706216
848 10.8963775208015
849 10.890616563199
850 10.8848645750238
851 10.8914497510958
852 10.888084373187
853 10.8946346677748
854 10.8953440243574
855 10.8912523110198
856 10.8961750194748
857 10.9026688595435
858 10.9060220714904
859 10.9007673818542
860 10.9088843350937
861 10.902563309041
862 10.8969055659024
863 10.9049765239817
864 10.8986742297266
865 10.8946535462559
866 10.8906354060933
867 10.8891736045228
868 10.8832207276203
869 10.8807408156297
870 10.8751152570266
871 10.8710647257077
872 10.8648398284336
873 10.8616248330535
874 10.8583306990152
875 10.8559758771332
876 10.8641585497324
877 10.8608893358114
878 10.8605206278458
879 10.8612301656117
880 10.8561344866272
881 10.852936821393
882 10.8505311128898
883 10.8465648892029
884 10.8461262380792
885 10.8400070929155
886 10.8449047267495
887 10.8483071488336
888 10.8422113422509
889 10.8398766610262
890 10.8341122214893
891 10.8348627948548
892 10.8298454603752
893 10.839625343232
894 10.8459405252247
895 10.8405171272018
896 10.8501834793241
897 10.8508812347754
898 10.8454491832871
899 10.8400618984035
900 10.8420990177692
901 10.846781370944
902 10.8475779574082
903 10.8572476797185
904 10.8549337196517
905 10.8545500027142
906 10.856499797556
907 10.8505254952864
908 10.8600527505678
909 10.8696219186614
910 10.8651703206108
911 10.8647875713246
912 10.8667129054299
913 10.87615929224
914 10.8757782904841
915 10.8719443824168
916 10.8660196879979
917 10.8706676981483
918 10.8713564378766
919 10.867539712815
920 10.8671118970491
921 10.8689606757069
922 10.8636574601371
923 10.8654851621138
924 10.8606434268479
925 10.8681992463131
926 10.8637864488078
927 10.8715716439379
928 10.8672495513873
929 10.8634488421949
930 10.8620028870158
931 10.8697467091647
932 10.8639231160336
933 10.8584237269783
934 10.8529022756789
935 10.8573548301484
936 10.8570560739832
937 10.8540744341033
938 10.8482966270891
939 10.8428402683349
940 10.8415495940721
941 10.8412359312348
942 10.8457115874595
943 10.8463611827395
944 10.8406240371846
945 10.8392286999945
946 10.8410199858208
947 10.8380967328454
948 10.8358268939066
949 10.8301245375984
950 10.8333886268921
951 10.8377801966761
952 10.8467580299813
953 10.8431766102964
954 10.8375095783215
955 10.8321119740203
956 10.8431333688436
957 10.8540857966157
958 10.857091637407
959 10.8614075718076
960 10.8558707875304
961 10.8505428435692
962 10.8463034838566
963 10.8406874184512
964 10.8424206663406
965 10.8373512801795
966 10.8327550321995
967 10.8299116116839
968 10.824323917815
969 10.822258928598
970 10.8252373475396
971 10.8205807453607
972 10.8297341106972
973 10.828366659097
974 10.8303254963124
975 10.8262468219525
976 10.8233024135137
977 10.8186818092443
978 10.8146265642589
979 10.8191405850386
980 10.8221145468576
981 10.8193056394364
982 10.8283193616105
983 10.8314396734175
984 10.8357255706151
985 10.8329002437958
986 10.8347759768491
987 10.8312848021198
988 10.829202506745
989 10.8299226774781
990 10.8258685402159
991 10.8205087503742
992 10.8184091808701
993 10.816335372483
994 10.8128511017409
995 10.8115932789092
996 10.8103746360138
997 10.8175926472248
998 10.8206054204306
999 10.8224246185582
1000 10.8231178040341
};
\addplot [semithick, color1, forget plot]
table {%
1 0
2 0
3 8.48528137423857
4 9.52627944162882
5 8.52291030106501
6 9.04924797366549
7 10.7684345971673
8 10.0747208398049
9 10.0554020859989
10 10.3440804327886
11 10.7641891749932
12 10.8816231428139
13 11.4385417202715
14 11.442179398513
15 11.6645712403938
16 11.6075137303386
17 11.2674317082926
18 11.586097458635
19 12.1800891062881
20 12.1789777896176
21 12.120333919949
22 12.3372881009544
23 12.4546815546444
24 12.4448087124266
25 12.5920609909578
26 12.5146659526064
27 12.4947176767494
28 12.2703392165741
29 12.2143947358589
30 12.0629828631046
31 11.8979357070666
32 11.7871338092642
33 11.8345651785785
34 11.7041086448115
35 11.7043163152814
36 11.543095579417
37 11.4708000998451
38 11.3268000287798
39 11.2024581692382
40 11.1645644787426
41 11.0498841854383
42 11.0559543678566
43 11.3020384689344
44 11.3737227963378
45 11.2703259453123
46 11.147829478396
47 11.3361334738399
48 11.2218784325778
49 11.2320938992291
50 11.169888092546
51 11.0655312712266
52 11.1378366537422
53 11.0968743093505
54 11.0865314547976
55 11.0203418224798
56 10.9787451189904
57 10.9657683876315
58 10.9503934959892
59 10.8571990157801
60 10.8572044693323
61 10.8129139350468
62 10.8827470402725
63 10.8171313229501
64 10.7744996983619
65 10.8838950563497
66 10.8234856703805
67 10.8506128222717
68 10.7744652000005
69 10.7455729858566
70 10.6700132930315
71 10.602500332045
72 10.7058651140161
73 10.6394007086181
74 10.5690881604546
75 10.6640496789707
76 10.6149231178461
77 10.5564046532857
78 10.6082907053718
79 10.5501753980367
80 10.5698864705351
81 10.5044390787753
82 10.4863087418381
83 10.4424127078737
84 10.4060149621404
85 10.42113901684
86 10.4327989375937
87 10.5355722420741
88 10.4759600341145
89 10.4188146167367
90 10.3698816022385
91 10.3392754034143
92 10.3462446270518
93 10.3150622261493
94 10.2914301987839
95 10.2511616799987
96 10.3294210924024
97 10.4018626159743
98 10.3493536609539
99 10.3471820861906
100 10.3033198533288
101 10.3212586233768
102 10.2784007001809
103 10.2303380753714
104 10.1852978434488
105 10.1482614710281
106 10.1074680982111
107 10.060130162366
108 10.1313899208546
109 10.121698660423
110 10.0795349492617
111 10.0561446371799
112 10.0972905981491
113 10.0560960877139
114 10.0985753941982
115 10.0895049254575
116 10.0459223511756
117 10.0302632141256
118 9.99152182976028
119 9.94987082245383
120 9.90873632485775
121 9.89436039356975
122 9.95941778857829
123 10.0508785664586
124 10.1128646395313
125 10.074010919192
126 10.1212882673342
127 10.1552338356903
128 10.1156882335064
129 10.1248619624085
130 10.1250904006484
131 10.1485212201354
132 10.1350612924287
133 10.1095195565075
134 10.1750358437659
135 10.1379131950623
136 10.2210008380534
137 10.1864654844821
138 10.1769164401288
139 10.1704026064644
140 10.1703124387622
141 10.1346202513688
142 10.149546345216
143 10.1630792750582
144 10.2157286138146
145 10.2661930904668
146 10.325371430478
147 10.2902079591101
148 10.2628440434293
149 10.2731883644372
150 10.2397829838115
151 10.2101208676967
152 10.2546185797457
153 10.2968250407978
154 10.2924290197044
155 10.3073678223685
156 10.3013678135341
157 10.3757707631662
158 10.4473792280714
159 10.5056337200558
160 10.47440145963
161 10.4693453033411
162 10.5360533869382
163 10.575074962895
164 10.6218182064422
165 10.5902115346646
166 10.5893556686211
167 10.5691998385474
168 10.5623553405543
169 10.5881849259403
170 10.5983872899979
171 10.6450323078288
172 10.6652995323966
173 10.6847080321786
174 10.6540083556136
175 10.6534179625937
176 10.6787169095688
177 10.6670945763063
178 10.6600216869321
179 10.663503583111
180 10.6494942788407
181 10.6422780106118
182 10.6130139208881
183 10.6037262235556
184 10.5819767214213
185 10.5617721026761
186 10.6177841180177
187 10.5937381044488
188 10.5783363360145
189 10.5629404886229
190 10.5353172039221
191 10.5500435173369
192 10.5639586435305
193 10.6162401926012
194 10.6237150917624
195 10.5986812720937
196 10.5737183321162
197 10.5933309501761
198 10.6190504868141
199 10.606696353015
200 10.5843646479134
201 10.5687896361602
202 10.5967806720581
203 10.6087263658251
204 10.5965744212736
205 10.5731591976757
206 10.5537955105979
207 10.5547076235555
208 10.6003662836542
209 10.6003853338874
210 10.6105122648033
211 10.6259813653018
212 10.6074774888135
213 10.6166004148513
214 10.6153548495345
215 10.5907395490996
216 10.5865853206648
217 10.6150770858618
218 10.6188895647291
219 10.5960790692182
220 10.5782122487765
221 10.5730002162207
222 10.5537848954232
223 10.5361738501198
224 10.5759361996338
225 10.5838451319819
226 10.6191076634682
227 10.6320981281062
228 10.6143540621927
229 10.6216242280263
230 10.6089139321806
231 10.586654213107
232 10.6359992377759
233 10.6658448371598
234 10.6774314585679
235 10.6742677006964
236 10.7220337643409
237 10.700934380574
238 10.6784381361778
239 10.6711837144205
240 10.6606460873626
241 10.6439081531025
242 10.6698191794573
243 10.6892447627444
244 10.6676816729911
245 10.6750761657871
246 10.6792362547731
247 10.6751253498141
248 10.6546748483391
249 10.6621355354007
250 10.6421050549222
251 10.6493852498584
252 10.639222655108
253 10.6737796681345
254 10.7041242124963
255 10.7171814252108
256 10.7122615827739
257 10.7072535463279
258 10.7493394858032
259 10.7840015073675
260 10.7632806364589
261 10.7463061679441
262 10.7293273922387
263 10.7089619550129
264 10.7423387610396
265 10.7350273253846
266 10.7382142770113
267 10.7201820030472
268 10.7315735709001
269 10.7126091604681
270 10.6998128260746
271 10.6820124846306
272 10.7156649713654
273 10.705802345782
274 10.6866081471965
275 10.697833255554
276 10.6881945447067
277 10.6785465824708
278 10.666523053072
279 10.682488891458
280 10.6685780998908
281 10.694515649926
282 10.6790566716262
283 10.6601993727197
284 10.659912796324
285 10.6415354918303
286 10.6279926416323
287 10.6312507188788
288 10.6343345887301
289 10.6162688347054
290 10.5989948595748
291 10.592368645427
292 10.5761248512845
293 10.5630610857909
294 10.5469910770029
295 10.5309928223678
296 10.5674028595376
297 10.5530189897103
298 10.5417506954432
299 10.5252148611354
300 10.522324943767
301 10.506877599085
302 10.4912953323684
303 10.4740256867054
304 10.4948702802252
305 10.4890436944602
306 10.5195281485366
307 10.5263669472285
308 10.5567241938096
309 10.5405698250611
310 10.5238864125249
311 10.5345815887146
312 10.564497363173
313 10.5887241496256
314 10.5854871228631
315 10.5851543478306
316 10.571463399516
317 10.5577625257768
318 10.5644915011177
319 10.5881565658456
320 10.5947985284938
321 10.5812755947669
322 10.5648686684994
323 10.5713469500492
324 10.5682653969793
325 10.5825355738404
326 10.5768370201325
327 10.5868785926421
328 10.5735698436502
329 10.570250453902
330 10.5604473706611
331 10.5665066112448
332 10.5563845631103
333 10.5450269791209
334 10.5475846441982
335 10.5449788194437
336 10.5337261878593
337 10.5513696767357
338 10.5367325128427
339 10.5211980443024
340 10.5246172949021
341 10.5152482486435
342 10.5212252148314
343 10.5161655212028
344 10.5425686842378
345 10.5333492857193
346 10.5366356411571
347 10.5430936553491
348 10.5457229290699
349 10.5306258754825
350 10.5184324121118
351 10.5043403439704
352 10.4894535163808
353 10.5115603882973
354 10.4983214422189
355 10.4861931120374
356 10.4754279155501
357 10.4624424838634
358 10.4554251728807
359 10.4729012134046
360 10.4639559918799
361 10.4534627176558
362 10.4485275795068
363 10.4344340652878
364 10.440697957113
365 10.4266838705283
366 10.4140289592407
367 10.3998621316228
368 10.4095255925706
369 10.4096737494369
370 10.4268981259459
371 10.4478844074451
372 10.4391713195014
373 10.4640552813988
374 10.4525077681017
375 10.449744281826
376 10.4414632807686
377 10.4626521377942
378 10.4577265005263
379 10.4453654434734
380 10.4485929946509
381 10.4685867699702
382 10.4882626104745
383 10.4861981428892
384 10.4792274994092
385 10.4985554267355
386 10.4862776577109
387 10.5093866948267
388 10.5302084191445
389 10.5573032424742
390 10.5553327427261
391 10.5510776649812
392 10.5490206035187
393 10.5520637557311
394 10.5514754395244
395 10.5452036633359
396 10.544618808557
397 10.5493559597845
398 10.555360113735
399 10.5468280867313
400 10.53450995538
401 10.5566466993497
402 10.5456305057729
403 10.5473634957982
404 10.5363771574279
405 10.5540216920344
406 10.5441163418038
407 10.5483061604215
408 10.5546171095816
409 10.5718251869783
410 10.5731302218821
411 10.5977609252874
412 10.5877926228808
413 10.5805225869704
414 10.5705395588359
415 10.5587848397121
416 10.5517521733367
417 10.543119035887
418 10.5395580580701
419 10.536290881188
420 10.5242178330616
421 10.5483798888029
422 10.5413852288312
423 10.5307030741461
424 10.5187322840057
425 10.5386304698443
426 10.5623473673399
427 10.5819780325378
428 10.5977092315057
429 10.6066656247937
430 10.6014991745749
431 10.5963209791098
432 10.6191127379467
433 10.6378215302986
434 10.6471610226968
435 10.6377267937447
436 10.6271630830625
437 10.6357449989078
438 10.6420773530174
439 10.6576407922974
440 10.6542776009614
441 10.642307748739
442 10.6483156890744
443 10.6415005985393
444 10.6346884433667
445 10.6265512023304
446 10.6298008093111
447 10.6483258983014
448 10.6415834624818
449 10.6635315075159
450 10.6667594903369
451 10.6724164304866
452 10.675613017059
453 10.6689082005113
454 10.6580784398594
455 10.6489252388636
456 10.6376619644482
457 10.6460906017433
458 10.6638645388525
459 10.6622724192338
460 10.651127721473
461 10.6476233615387
462 10.6427849485193
463 10.6316506823164
464 10.6201891168298
465 10.6190869980203
466 10.6127507098295
467 10.6182203523418
468 10.6397338573339
469 10.6481466458322
470 10.6467330388331
471 10.6418756122046
472 10.6330505435157
473 10.6219124887442
474 10.6108093575838
475 10.6094644414349
476 10.623789844973
477 10.6206733582654
478 10.6175354961947
479 10.6143765774623
480 10.6111969164652
481 10.6223034098758
482 10.6332607926959
483 10.6246665501106
484 10.6137777369735
485 10.6036889206894
486 10.5928771693014
487 10.5855145477478
488 10.5935535799984
489 10.5922673043678
490 10.6004754201731
491 10.5897641819493
492 10.5927040725699
493 10.5842865656099
494 10.5977736486144
495 10.6059905721593
496 10.597594844432
497 10.5873408431603
498 10.6043148160482
499 10.5945609587537
500 10.6146108736967
501 10.6224218178115
502 10.6253659892877
503 10.6358008887009
504 10.6327362051171
505 10.6256075095758
506 10.6184887984802
507 10.6139101705038
508 10.6303280762049
509 10.6379388761144
510 10.6574101993337
511 10.6675945167022
512 10.6681732862121
513 10.6812587569078
514 10.6708634033657
515 10.6758944987341
516 10.6669995765045
517 10.6640093722981
518 10.6537109881093
519 10.649250499907
520 10.6480156220016
521 10.6410578917641
522 10.6330980587989
523 10.6261382176287
524 10.61745263875
525 10.6117573095731
526 10.6038921366083
527 10.5981797512736
528 10.5990018109348
529 10.5893402138424
530 10.5921343829635
531 10.5835541586042
532 10.6021827956694
533 10.5935952682284
534 10.6033084701948
535 10.5941303498121
536 10.5897124763049
537 10.5884039401664
538 10.5800668835127
539 10.5705510341288
540 10.5665600532068
541 10.5568562489419
542 10.5599495202736
543 10.5503280891777
544 10.5550894337354
545 10.5672533261177
546 10.5579750357468
547 10.5656338192652
548 10.5648355109754
549 10.5803915328202
550 10.5721963440522
551 10.5747944930183
552 10.5655228914997
553 10.5563412688335
554 10.565776198942
555 10.5582824426585
556 10.5628735387969
557 10.5573693793903
558 10.5491532695319
559 10.5409557248977
560 10.5485356307917
561 10.5577247562447
562 10.5489979264532
563 10.5494619931889
564 10.5517847523863
565 10.5424469364091
566 10.5513948769153
567 10.5477667562119
568 10.5386047483364
569 10.537285243318
570 10.530329790226
571 10.5229868069648
572 10.5138312166948
573 10.5065236215725
574 10.5044655943931
575 10.5120904376161
576 10.5033670506672
577 10.5057064803217
578 10.4968711503719
579 10.4878504269158
580 10.4799673007813
581 10.4765203148611
582 10.4702428136798
583 10.4735145079193
584 10.4860083349803
585 10.4839205295795
586 10.4833885025526
587 10.4744571084788
588 10.4663555965026
589 10.461612576432
590 10.4565502288701
591 10.4477018754159
592 10.4654550174225
593 10.4616224224438
594 10.4541870604433
595 10.4456798025631
596 10.4369735549019
597 10.454557257606
598 10.4640219361888
599 10.4730186749995
600 10.4755364487404
601 10.4843687475386
602 10.4775216511718
603 10.4894456546749
604 10.4844822167005
605 10.4787816938734
606 10.4728437477329
607 10.4722885740848
608 10.4655398807549
609 10.4609279827671
610 10.4658142511095
611 10.4727599084441
612 10.4775145793472
613 10.4728329059957
614 10.4692217108945
615 10.4806853001051
616 10.4945873495037
617 10.4873137893129
618 10.5037531301695
619 10.5200335989825
620 10.51652680823
621 10.5191609449086
622 10.5156267492936
623 10.5133326752433
624 10.5049865276491
625 10.5186299944432
626 10.5105301224326
627 10.5266835758087
628 10.5231020329848
629 10.5255253016963
630 10.5318015333282
631 10.5296505689995
632 10.5406784648617
633 10.5330047296617
634 10.5392524229382
635 10.5315923463109
636 10.5235866675686
637 10.5392412941927
638 10.5397967193337
639 10.5315474684865
640 10.5239310621079
641 10.5176639660445
642 10.5140663434942
643 10.5062224225338
644 10.5042102111131
645 10.4961554051264
646 10.4891431833508
647 10.4901270461612
648 10.4878513077439
649 10.4939146972748
650 10.5066543016978
651 10.4988288718098
652 10.4952565596345
653 10.4928935002319
654 10.4933752468456
655 10.4873310624743
656 10.5023269380963
657 10.4959827130037
658 10.4980186739746
659 10.4949638417543
660 10.4978050965776
661 10.5137315187014
662 10.5060226401305
663 10.4981987034859
664 10.4919475029106
665 10.5077455958952
666 10.5046199709665
667 10.5000789653605
668 10.5148390923534
669 10.5143588273519
670 10.5124856549099
671 10.5093727789577
672 10.5015520934709
673 10.5054050853238
674 10.5131081361518
675 10.5100460400543
676 10.502982227659
677 10.5074195526522
678 10.4997174004097
679 10.4978623575846
680 10.4902326254632
681 10.4883758975118
682 10.484236094784
683 10.4811595786494
684 10.4837103778751
685 10.4772202147322
686 10.489855475095
687 10.4920955524055
688 10.5070654715876
689 10.5169760806302
690 10.5267980013141
691 10.5248628780197
692 10.5173362751301
693 10.5151935951835
694 10.5160424495705
695 10.5184465841962
696 10.5114760868793
697 10.509372787218
698 10.5132048965294
699 10.5101406635783
700 10.5037652141253
701 10.5046000014873
702 10.500497433159
703 10.4930266361862
704 10.5006667777024
705 10.4934665099002
706 10.4877721774689
707 10.4974335542355
708 10.4982895177588
709 10.5061355607916
710 10.5179432180341
711 10.5257302740602
712 10.519962567551
713 10.5169172447072
714 10.5112507626949
715 10.5170114088805
716 10.5208216984558
717 10.5348083945128
718 10.5442138411201
719 10.5421040609382
720 10.5440872458817
721 10.536773585877
722 10.5311980221764
723 10.5404554730163
724 10.5439990565494
725 10.5448565435655
726 10.550120804391
727 10.5571995925361
728 10.5506037602286
729 10.5434090503316
730 10.5362290383898
731 10.5352763820723
732 10.5281200458279
733 10.5299588273621
734 10.5397540646323
735 10.5347612631947
736 10.5356747791933
737 10.5302507525883
738 10.5413049000571
739 10.5408351062276
740 10.5526031981963
741 10.546117573229
742 10.5420009783782
743 10.5508917865621
744 10.5532443322122
745 10.5511154654792
746 10.5451503309349
747 10.5455209816295
748 10.5445653044336
749 10.5584635454224
750 10.5588100339637
751 10.5663109057955
752 10.5616781216777
753 10.5551681880318
754 10.5481681752281
755 10.5418096382709
756 10.5426450331742
757 10.5434575579763
758 10.5386520505564
759 10.5319231458278
760 10.5252070962182
761 10.5206456487575
762 10.5257100564406
763 10.5209347332243
764 10.5192660093313
765 10.5128921096124
766 10.5060719136761
767 10.4995092797633
768 10.4937405775345
769 10.4993030977365
770 10.498823713919
771 10.4996390103945
772 10.493058481172
773 10.4893783467347
774 10.4828703836344
775 10.4763287212433
776 10.4835729487727
777 10.4887107031238
778 10.4974485066223
779 10.5028744280059
780 10.5065957180796
781 10.5087488077954
782 10.5197176583965
783 10.5216685613218
784 10.5158976770451
785 10.5114220304947
786 10.5182332965976
787 10.5117726503656
788 10.5151713763092
789 10.5204603651303
790 10.5254516383077
791 10.5197126633438
792 10.5145107825436
793 10.5079272745445
794 10.5132160905774
795 10.5168361783573
796 10.5102964440896
797 10.5138843913125
798 10.5111559889299
799 10.5084240563569
800 10.5019098709473
801 10.4991213389968
802 10.504126147743
803 10.4985605493029
804 10.4920297482382
805 10.489243858318
806 10.486579370302
807 10.4871499821946
808 10.4978261289415
809 10.4985154582037
810 10.4920876553901
811 10.48852365639
812 10.4849155244015
813 10.4806005599764
814 10.4762494244377
815 10.4812266462584
816 10.4777081157431
817 10.4715377522872
818 10.4820633093679
819 10.4905137873894
820 10.4911931565909
821 10.4962102113958
822 10.4898237178483
823 10.4948032188403
824 10.4898817861031
825 10.4863522480925
826 10.4836874734512
827 10.4856667970159
828 10.4807703383661
829 10.4910714737225
830 10.487551531084
831 10.4812944440555
832 10.4807471771116
833 10.4891141861305
834 10.4828820866653
835 10.4823198294964
836 10.4906427801759
837 10.4844292485133
838 10.4909323512785
839 10.497385592621
840 10.4979248975307
841 10.5081522622781
842 10.5075057246424
843 10.5017602654173
844 10.5098510142049
845 10.5178826213747
846 10.5280695991446
847 10.5327085936352
848 10.5309326038679
849 10.5316189985481
850 10.5262746559021
851 10.527989493474
852 10.5246255047614
853 10.5295181934124
854 10.5246936060595
855 10.5187798885793
856 10.5135617761569
857 10.5217638934703
858 10.5158666553699
859 10.5102628518981
860 10.5134043569915
861 10.507819016763
862 10.5043707470833
863 10.5077080787363
864 10.5024782715098
865 10.4984310747006
866 10.4924163086298
867 10.4872700606982
868 10.4820804162953
869 10.4840264316377
870 10.4939107471519
871 10.497161317269
872 10.4920080125466
873 10.5036226680553
874 10.5019821840126
875 10.5003352171952
876 10.5118114905842
877 10.5165571891236
878 10.5159082615505
879 10.5143762979768
880 10.5206504430003
881 10.5190904722778
882 10.5136224984841
883 10.5090314491724
884 10.5107923073982
885 10.5049016464057
886 10.510976181279
887 10.5128420339122
888 10.506969668478
889 10.5012828090436
890 10.4962178698817
891 10.500720111424
892 10.4961315983381
893 10.4903141700886
894 10.4857407570977
895 10.4919750536398
896 10.4880692417909
897 10.4899331292713
898 10.4904636399832
899 10.4964634941963
900 10.4994750603642
901 10.495010596422
902 10.5009429693626
903 10.5002991006764
904 10.4988513220329
905 10.5018074379602
906 10.4978338082452
907 10.5088222386386
908 10.5030731777837
909 10.512172460006
910 10.5179040688576
911 10.5241237403749
912 10.5237176412542
913 10.5244069943763
914 10.5275993238634
915 10.5294563423029
916 10.5245788752937
917 10.5197078923589
918 10.5309022398139
919 10.5260277012642
920 10.5353287643962
921 10.5463739260164
922 10.548013591103
923 10.5422981556803
924 10.5467044358416
925 10.5428632229538
926 10.5384375675432
927 10.5400823895206
928 10.5362166633294
929 10.540480310629
930 10.5496409544923
931 10.5603674443459
932 10.5549164343844
933 10.5605789347565
934 10.5551423765115
935 10.5535301161477
936 10.5496618377803
937 10.5524536428513
938 10.5468277086296
939 10.5452033763191
940 10.5561196177972
941 10.5632302850044
942 10.5662508606417
943 10.5639342521678
944 10.5583378536335
945 10.562685740755
946 10.5621747380262
947 10.5583578920056
948 10.5577109249289
949 10.5529768018872
950 10.5515253915556
951 10.546019851046
952 10.5417570244408
953 10.5362249300167
954 10.5379309870133
955 10.5422077497774
956 10.5368976411965
957 10.5345872228073
958 10.533118244312
959 10.5288685654156
960 10.5251581457504
961 10.5293801334056
962 10.5256820154218
963 10.534464742215
964 10.5339080304167
965 10.5315774124887
966 10.5371272941961
967 10.532105621697
968 10.5376160584778
969 10.5391475644941
970 10.5345258801326
971 10.5295610033747
972 10.5241854908216
973 10.5188182020535
974 10.525786373659
975 10.52637145409
976 10.5267807056411
977 10.5297110453795
978 10.5292498198381
979 10.5396775729666
980 10.5355301659061
981 10.5360504767594
982 10.5365577195322
983 10.5381226481694
984 10.5329536780973
985 10.5293578210684
986 10.5321806654342
987 10.5407480990227
988 10.5448203805706
989 10.5502513444597
990 10.5461293274571
991 10.5408530305096
992 10.5477358423305
993 10.5435989826338
994 10.5490766433695
995 10.5591444696534
996 10.558635256098
997 10.5550282353342
998 10.5650304777842
999 10.5751705721955
1000 10.5819121145472
};
\addplot [semithick, color2, forget plot]
table {%
1 0
2 16
3 14.2361043360417
4 15.0665191733194
5 13.9484766193302
6 14.0919598826186
7 13.0571741168741
8 13.8451254960004
9 13.4449035734276
10 13.2442440327865
11 12.8731895140429
12 12.6367519561001
13 12.3384231727775
14 11.8949483360949
15 11.6516093309036
16 11.6375939845829
17 11.2928920903684
18 11.5
19 11.5542799447456
20 11.65879496346
21 11.6100667867868
22 11.4455231422596
23 11.1957999146014
24 10.9898153608997
25 10.8344635307891
26 10.6384921296629
27 10.4527154012371
28 10.3314336085639
29 10.2533889876115
30 10.6071778632312
31 10.8470589961411
32 10.9813407082196
33 10.8742710298938
34 10.7536104824521
35 10.9557179181485
36 10.8358615283571
37 10.8952110830154
38 10.9604236426606
39 10.9847183312948
40 11.0147344498177
41 11.1551707571981
42 11.1873349187836
43 11.0706468284671
44 11.1517518914265
45 11.0907368756293
46 11.0293473440426
47 11.226544886788
48 11.2212402451284
49 11.1657087949949
50 11.1808586432349
51 11.0861503561578
52 11.0669526441746
53 10.9836280636071
54 11.0686999222954
55 10.9701322532088
56 10.8721700516803
57 10.7805554356051
58 10.8944236245784
59 10.8142500177907
60 10.8326153608238
61 10.9777816705291
62 10.8890918124277
63 10.8039866479014
64 10.7237028462362
65 10.6549985491712
66 10.5753234260111
67 10.6601441036269
68 10.7744652000005
69 10.6962020203507
70 10.7087223651506
71 10.6372627521073
72 10.7178945641805
73 10.7112426314107
74 10.6992299170602
75 10.7122857815999
76 10.649306156114
77 10.6343099360429
78 10.6887714745171
79 10.6408198437717
80 10.6186672304014
81 10.5653337137188
82 10.7183675406503
83 10.6836699771463
84 10.6199217556081
85 10.6267274191135
86 10.7010864123987
87 10.651185257551
88 10.6138796935315
89 10.5541808537302
90 10.501528401401
91 10.5135837172393
92 10.4626439638629
93 10.4986771653491
94 10.5930099863256
95 10.5757575709472
96 10.5670755622121
97 10.5899123206412
98 10.5701086560758
99 10.5303793326209
100 10.4878787178342
101 10.4506533422028
102 10.5402370686107
103 10.5282119232238
104 10.5957001159313
105 10.5735733560012
106 10.5400624110267
107 10.5186324248825
108 10.4802831077793
109 10.4543931193372
110 10.4598761113343
111 10.5356884189022
112 10.572816028642
113 10.5672165802641
114 10.5222214259411
115 10.5340782839794
116 10.4885986454711
117 10.4711982663482
118 10.4270113337803
119 10.5097867505071
120 10.4680433011884
121 10.4840826658793
122 10.4739730665065
123 10.4977453364293
124 10.520024511036
125 10.4794992246767
126 10.4413713093565
127 10.4135160672385
128 10.4790993619442
129 10.5185943392349
130 10.5247412089226
131 10.5575582409896
132 10.5269206471017
133 10.5392187299447
134 10.5349797635708
135 10.4992154648445
136 10.4659606429573
137 10.428963509208
138 10.4197481264517
139 10.3906711672558
140 10.3943150671069
141 10.43208651965
142 10.4456823285825
143 10.4727091551454
144 10.4781311615616
145 10.4423996221751
146 10.4234086519929
147 10.3890313691468
148 10.4184960653141
149 10.4550015896995
150 10.4449647624532
151 10.4294943310836
152 10.4778964519684
153 10.4871516360855
154 10.4887879966205
155 10.4967880722394
156 10.5371435322793
157 10.5431233800291
158 10.5098133615554
159 10.4773003315761
160 10.4830517980214
161 10.4520413322228
162 10.4642502576757
163 10.4362446275975
164 10.4242187575061
165 10.3948422613429
166 10.4217720589864
167 10.3933433298132
168 10.4556820364849
169 10.4258440834338
170 10.4052555259616
171 10.4148188573461
172 10.384502171189
173 10.361674357909
174 10.336447755383
175 10.339857497143
176 10.3234227153555
177 10.3262710898445
178 10.3733684565191
179 10.4021446269411
180 10.3783428349617
181 10.3613065823861
182 10.3328963865792
183 10.319819339769
184 10.3067037758013
185 10.3413613497149
186 10.3165304850371
187 10.3499271919858
188 10.370320316103
189 10.3953106857352
190 10.4085517791143
191 10.4045127509306
192 10.4303230793648
193 10.4551856834479
194 10.4723900491981
195 10.4461676925503
196 10.4867220839802
197 10.5428698129791
198 10.5532511621209
199 10.5283085861313
200 10.5455962372926
201 10.5193943261334
202 10.5359049729326
203 10.517191772013
204 10.4929021157013
205 10.4677698594239
206 10.4428172394927
207 10.4243931996854
208 10.4204154989903
209 10.4437540792292
210 10.4885252060398
211 10.4703281306829
212 10.4999989404804
213 10.4954120077769
214 10.5315129638155
215 10.535946350227
216 10.5115677020869
217 10.4901272668014
218 10.4997064098978
219 10.5506865519916
220 10.5267261202103
221 10.5029281909154
222 10.5238693525944
223 10.5189808132172
224 10.5472054464484
225 10.5278466723046
226 10.5230402557277
227 10.5331603270428
228 10.514082804099
229 10.5282131193309
230 10.5232445922625
231 10.5447259111792
232 10.5398252438655
233 10.5805055615887
234 10.5849792335705
235 10.5624958983012
236 10.5401339316872
237 10.5182908538445
238 10.5222173238914
239 10.5686666137741
240 10.546668773704
241 10.5272154654905
242 10.5466470283565
243 10.5555018087979
244 10.5641270041772
245 10.5429559043151
246 10.522662970174
247 10.5264497032902
248 10.5162496282418
249 10.5528291863093
250 10.5885341761738
251 10.5708721214166
252 10.6056062165232
253 10.6039659467253
254 10.5833215964267
255 10.5652793998368
256 10.5927705115034
257 10.5785006891732
258 10.6054586255897
259 10.5940416011111
260 10.5919047741925
261 10.6121539953967
262 10.5964654267741
263 10.5940702577052
264 10.585195249549
265 10.5713908272651
266 10.5545198146713
267 10.5393577856619
268 10.5598326820605
269 10.5418031673239
270 10.5790615467477
271 10.5679435725717
272 10.5623355620787
273 10.5537613292638
274 10.5385744273925
275 10.5211041748407
276 10.5085006747596
277 10.545636156488
278 10.5597273264534
279 10.5616409061173
280 10.5468802815557
281 10.56142383673
282 10.5593310326721
283 10.5408048421136
284 10.5280977669642
285 10.5580725644304
286 10.5560222318672
287 10.5912329795862
288 10.5963666466913
289 10.6194453743887
290 10.6280700957216
291 10.610533117815
292 10.6057812175071
293 10.6292357239434
294 10.6467936880544
295 10.6302587743071
296 10.628419185159
297 10.6462868177935
298 10.6637663936145
299 10.6536906351038
300 10.6757102287805
301 10.6880334592814
302 10.6726477432006
303 10.6576615011867
304 10.6741672396674
305 10.7017669150223
306 10.6842680653187
307 10.6929949752145
308 10.6809477401633
309 10.6692604537958
310 10.6618656816101
311 10.6739878394947
312 10.6752880400023
313 10.6605466916373
314 10.6458642360147
315 10.6500844274549
316 10.651537709083
317 10.6348970815087
318 10.6217366148312
319 10.6057212808164
320 10.6362275237746
321 10.6201510731208
322 10.6058114979202
323 10.5968525396275
324 10.6055598489853
325 10.59299578794
326 10.6008293495067
327 10.6219780422193
328 10.6261367800034
329 10.6239347242202
330 10.6115868719636
331 10.6324350964392
332 10.6437971464546
333 10.6389111131223
334 10.6427031533271
335 10.6340224098435
336 10.618285193108
337 10.6078134805913
338 10.5944989897696
339 10.6183488429186
340 10.6232179454048
341 10.6340570442043
342 10.6619803013244
343 10.6474690146607
344 10.6334169568872
345 10.6495388636702
346 10.6741084686319
347 10.6845739700371
348 10.6743413203748
349 10.6672642777504
350 10.6897532473765
351 10.6977959091877
352 10.6827792120878
353 10.6721006176114
354 10.6593581324987
355 10.6511679870424
356 10.6412217337755
357 10.6605757272395
358 10.6590528107571
359 10.6574638848138
360 10.655810085594
361 10.6568225581555
362 10.6484331637204
363 10.6401184773536
364 10.6260085508387
365 10.6134963665567
366 10.6404766714875
367 10.6264486799723
368 10.6449273820799
369 10.6457881130313
370 10.6324901412303
371 10.6186906642304
372 10.6325883581948
373 10.6462432385551
374 10.6330269163092
375 10.6475328389048
376 10.6376954616672
377 10.6383000031584
378 10.6480074919232
379 10.6386873338143
380 10.6569327051193
381 10.643951726929
382 10.6691281378644
383 10.665491141678
384 10.6821453320669
385 10.6785365029656
386 10.6664660162162
387 10.6706264058187
388 10.6910357323953
389 10.6785134768378
390 10.7001880202471
391 10.712798198314
392 10.6996924426521
393 10.6915954110031
394 10.6797340833328
395 10.6775566639325
396 10.6679970641675
397 10.6722173984788
398 10.6977338869219
399 10.6897786526271
400 10.6937353155948
401 10.7133864580228
402 10.7039569754971
403 10.7194875235029
404 10.7249398995718
405 10.7385080693528
406 10.7424342883973
407 10.7318447263953
408 10.7186907149176
409 10.7143459663124
410 10.7276666450738
411 10.7242107575738
412 10.7207240439613
413 10.7193851149896
414 10.7073513142784
415 10.6986893514792
416 10.685824583633
417 10.708724857141
418 10.7045929725129
419 10.6948172537311
420 10.7187765720633
421 10.7069437542844
422 10.6943306487796
423 10.6820602377429
424 10.7018520304409
425 10.7047575362562
426 10.705770881341
427 10.7000176301265
428 10.708458308828
429 10.702640265766
430 10.6903402563804
431 10.6795116931659
432 10.69768845099
433 10.7002563085403
434 10.7112735471527
435 10.732519177498
436 10.7264794080603
437 10.7340369257129
438 10.7275602921508
439 10.715378177976
440 10.7220978947435
441 10.7111084804257
442 10.7012888272315
443 10.6903769535844
444 10.6830122185118
445 10.6743616775231
446 10.6685141511142
447 10.6598878847424
448 10.6576439460992
449 10.6458056487437
450 10.6431975609301
451 10.6387112508412
452 10.6454871170352
453 10.661349419548
454 10.681862893735
455 10.6796959831032
456 10.6693345218058
457 10.6818348111279
458 10.674695488734
459 10.6652613097302
460 10.6550082741991
461 10.6549253838702
462 10.6589642666113
463 10.665517641621
464 10.6628173946624
465 10.6583148885343
466 10.6563282633474
467 10.6462704987889
468 10.6393493599527
469 10.648575725171
470 10.637961696816
471 10.6600636113655
472 10.6732670585959
473 10.6641512364339
474 10.673149242881
475 10.6663390069712
476 10.6552971234863
477 10.6625385093472
478 10.6774916413742
479 10.6862894689943
480 10.7076035338212
481 10.7030358709271
482 10.6984580202775
483 10.6910888004587
484 10.6875378327873
485 10.6779036181051
486 10.6940102980582
487 10.6851935692032
488 10.6852945609157
489 10.6770245919268
490 10.6815946362584
491 10.695960247055
492 10.711654398662
493 10.70094319156
494 10.6943601468246
495 10.7067741525541
496 10.6972386653021
497 10.7157235411241
498 10.7119149103433
499 10.7116794831197
500 10.7207781433999
501 10.7153788243454
502 10.7217086524487
503 10.7212753445205
504 10.7108443547361
505 10.7284592216616
506 10.7345085930023
507 10.7359069280331
508 10.7376301665406
509 10.7411956877632
510 10.7446908467286
511 10.7464402368923
512 10.7477469633788
513 10.7384504904451
514 10.7522881825804
515 10.7497439506363
516 10.7515551314275
517 10.7440058494039
518 10.7336652104302
519 10.7444991755055
520 10.7533425036125
521 10.7508511406106
522 10.7410429370869
523 10.7343570865367
524 10.7245958923302
525 10.7153643478221
526 10.7091165315165
527 10.7234831275417
528 10.7169070614594
529 10.7084490457091
530 10.7049201332256
531 10.6959934446799
532 10.6884735573014
533 10.708843183325
534 10.7289847897163
535 10.7400125550644
536 10.7310732572986
537 10.7418079904145
538 10.7329137128939
539 10.7385618368507
540 10.7304171331711
541 10.7340759593399
542 10.7397058039772
543 10.7376514603854
544 10.7410238536932
545 10.7311932487982
546 10.7213772279947
547 10.7409163982866
548 10.737298692983
549 10.7275344985105
550 10.7271519253599
551 10.7176020261867
552 10.7114655448675
553 10.7093789360982
554 10.7284226781538
555 10.7390313763864
556 10.7468823829689
557 10.743332909103
558 10.7354169835726
559 10.7263426063632
560 10.7366610111509
561 10.73613758656
562 10.7376598859337
563 10.7409209535399
564 10.7372506376152
565 10.7277584916637
566 10.7289876650292
567 10.7221044043463
568 10.7142512350547
569 10.7178451712211
570 10.7089851902152
571 10.7243694315914
572 10.7155453944671
573 10.7072481919867
574 10.7039253328331
575 10.716912627109
576 10.7120907298164
577 10.7118265265547
578 10.7173589687054
579 10.7250734881774
580 10.7182535968538
581 10.7215703553615
582 10.7212000963183
583 10.7265689350438
584 10.7341769981033
585 10.7274108643572
586 10.7270076576392
587 10.7369610421703
588 10.7493642103581
589 10.7643118004271
590 10.769402564679
591 10.7674806553168
592 10.7669240908532
593 10.7764402927053
594 10.7817951336661
595 10.7850464155047
596 10.7782962565999
597 10.779648410065
598 10.7749347417434
599 10.7823340875664
600 10.7788759463437
601 10.7818517476697
602 10.7752813218103
603 10.7823601086003
604 10.7756793904625
605 10.784942210558
606 10.7827637402753
607 10.7795683605677
608 10.7887133427133
609 10.7984505449318
610 10.79491964975
611 10.7882626491134
612 10.7838646173073
613 10.7852835684273
614 10.7834647380679
615 10.7903122654739
616 10.7824092403309
617 10.7875997865717
618 10.7885980549235
619 10.7831593726341
620 10.7760482274637
621 10.7688115033477
622 10.7778023003452
623 10.7715112892409
624 10.7633893650926
625 10.7771805144017
626 10.7689781921198
627 10.7762628489
628 10.7709241625173
629 10.7625522064066
630 10.7541997280775
631 10.7684933096705
632 10.7698208768798
633 10.7726310434889
634 10.7681502375689
635 10.7817682860028
636 10.7753920582106
637 10.7684945581363
638 10.7819756952009
639 10.7812932752886
640 10.7805887478375
641 10.7938240756019
642 10.7856263263204
643 10.8013036156391
644 10.8004719498963
645 10.798920505935
646 10.7988184166427
647 10.79177699177
648 10.7916708595713
649 10.8057155346529
650 10.8099997536797
651 10.8150257974527
652 10.807506712636
653 10.7996054663422
654 10.8020506166434
655 10.8019044521658
656 10.7937024761302
657 10.8019720842787
658 10.7950483891478
659 10.7928929170234
660 10.7859858528214
661 10.7920507443493
662 10.796132850029
663 10.7883292632715
664 10.7804263953067
665 10.7755295920763
666 10.7716488086874
667 10.783103987237
668 10.7757659474608
669 10.7757110410658
670 10.781747403705
671 10.7768585503145
672 10.7775771990096
673 10.7798838307859
674 10.7721010867114
675 10.7682712305609
676 10.7742067800412
677 10.7662497178655
678 10.7605969792939
679 10.7572944030942
680 10.7503617063152
681 10.7573684964651
682 10.7494828711459
683 10.7431531144127
684 10.7521589647603
685 10.7452520779436
686 10.7396391881886
687 10.7544318211812
688 10.750168124605
689 10.7494427365811
690 10.7468293707907
691 10.7421212789256
692 10.7567279697556
693 10.7491749057847
694 10.7523452141884
695 10.7446447554954
696 10.7469427077233
697 10.7418202069268
698 10.7538702523581
699 10.7513239585891
700 10.7436830989476
701 10.7443529534561
702 10.7367000833194
703 10.7389013683007
704 10.7376456636266
705 10.7356026532165
706 10.7279989466469
707 10.7238033359118
708 10.7162728717231
709 10.7164239640052
710 10.728249911401
711 10.7224470917511
712 10.7164519109829
713 10.711993097759
714 10.7054474631962
715 10.7023201553182
716 10.6951490942155
717 10.6916917640601
718 10.6843330356375
719 10.6769894776449
720 10.675034333654
721 10.6860324377011
722 10.6909932635034
723 10.6857729916798
724 10.6997764835258
725 10.7136643389263
726 10.727437650923
727 10.7210163046462
728 10.7303063340176
729 10.7244689790283
730 10.7210825040279
731 10.727832105244
732 10.7210179853213
733 10.7158738258187
734 10.7207263663043
735 10.7238433561168
736 10.7331568900286
737 10.7265106937129
738 10.7207111035352
739 10.7254934583064
740 10.7390744078834
741 10.7438212170448
742 10.7382691106621
743 10.738873678645
744 10.7321547898413
745 10.739330483059
746 10.733786960866
747 10.7275248326751
748 10.7209705090649
749 10.716163276671
750 10.7090180066459
751 10.7082719179942
752 10.7027803061225
753 10.7133231449632
754 10.7103163861079
755 10.7207898223572
756 10.7311790362355
757 10.7312927554513
758 10.7319009809736
759 10.7289281606292
760 10.7232999608265
761 10.7225846077147
762 10.7171731810856
763 10.7117689558358
764 10.7063719205534
765 10.7069368310848
766 10.7039648011438
767 10.7007706524918
768 10.7137518209821
769 10.7068567718004
770 10.7102941010105
771 10.7227589934086
772 10.7243510298859
773 10.7307509225402
774 10.730911803489
775 10.7290182142557
776 10.7243657808926
777 10.7214278528488
778 10.7192555331782
779 10.7279378679535
780 10.7249884867427
781 10.7220357988656
782 10.7151785906799
783 10.7130588773896
784 10.7092259068112
785 10.7084574091497
786 10.7117976647623
787 10.7202763301545
788 10.7194381323883
789 10.7226716618185
790 10.7274635680308
791 10.7212113406836
792 10.7153816819009
793 10.7218351877685
794 10.7153101717045
795 10.7235228211323
796 10.7182763130341
797 10.7229411050656
798 10.7183639221457
799 10.7118999932276
800 10.7135281741124
801 10.7068387265728
802 10.7086270190211
803 10.7019571919018
804 10.6962515152651
805 10.7064499927782
806 10.7000430405511
807 10.6943281552771
808 10.6879454055703
809 10.694294187797
810 10.7064490904408
811 10.7000807593026
812 10.6981640784628
813 10.6930159852019
814 10.6910834490246
815 10.6859688197759
816 10.6840762714753
817 10.6857691807707
818 10.6874912854079
819 10.6866774319591
820 10.6815888320139
821 10.6819755637981
822 10.6791065219121
823 10.6909771789947
824 10.685375761325
825 10.6871233591316
826 10.6807127108352
827 10.6756860092444
828 10.687483515983
829 10.6879177991533
830 10.6850503943881
831 10.6795400016027
832 10.6894365265171
833 10.6991878318211
834 10.6972667844257
835 10.7050814089869
836 10.7000342301662
837 10.7029568131624
838 10.698514168336
839 10.70028124534
840 10.6944386945239
841 10.6883189497881
842 10.6828931089972
843 10.6800416023403
844 10.679164584398
845 10.6871765750477
846 10.6986027169705
847 10.7000970251192
848 10.7113979210556
849 10.7145579701075
850 10.725789120113
851 10.7222732244508
852 10.7172772231799
853 10.7122875723657
854 10.7140605420344
855 10.7215313377295
856 10.7160765390377
857 10.7188058993762
858 10.717119109082
859 10.7116825819249
860 10.7079927487559
861 10.7093615242067
862 10.7184532924154
863 10.7202770625952
864 10.7203807568422
865 10.7142169812388
866 10.7168526126481
867 10.7162780590107
868 10.7194608060873
869 10.7174584877368
870 10.7113319334037
871 10.7069851229691
872 10.7096051866507
873 10.703732545198
874 10.7000659301531
875 10.694504314395
876 10.6920014271633
877 10.6938460344621
878 10.6877888030986
879 10.6949831292569
880 10.6996282254676
881 10.6967798051557
882 10.6942817374318
883 10.7057358562553
884 10.7032095318877
885 10.7103749947874
886 10.7045808012361
887 10.7017596368855
888 10.6966565389911
889 10.7012119498279
890 10.7106592909429
891 10.7108065820088
892 10.7196148706436
893 10.7150138937449
894 10.7094348990058
895 10.7099663302887
896 10.7066639557438
897 10.7160065811853
898 10.7104626116295
899 10.7175429884291
900 10.7118293903931
901 10.7112267506617
902 10.7084754770711
903 10.7189842070598
904 10.7300800565609
905 10.7326348522375
906 10.7267806324554
907 10.7220638672787
908 10.718548498697
909 10.7277836778001
910 10.7271744962214
911 10.7224867173995
912 10.7254505034628
913 10.7363614412687
914 10.7347248505509
915 10.7451213142657
916 10.7404569158556
917 10.7350988392016
918 10.7334811869164
919 10.7442705179805
920 10.744663026539
921 10.7450411514576
922 10.7492099396022
923 10.7465553865516
924 10.74078165963
925 10.7409751948939
926 10.747888390373
927 10.7494364540073
928 10.744941845703
929 10.7393751044773
930 10.7419435026394
931 10.7423254983292
932 10.7406807195558
933 10.7410393611848
934 10.7465861272075
935 10.7426569006739
936 10.7369707000154
937 10.7344506434456
938 10.7291643184089
939 10.7236551679275
940 10.7219350563951
941 10.7245492352573
942 10.7271396094096
943 10.7219159453915
944 10.7164199789708
945 10.716643231389
946 10.7267945920155
947 10.7335602941445
948 10.7422760290873
949 10.7416122976081
950 10.7441195151173
951 10.7415203495745
952 10.7370461310237
953 10.7352665963647
954 10.743937256196
955 10.744089568258
956 10.7389415815809
957 10.7356348282755
958 10.7300677403196
959 10.7262919570065
960 10.7247359378914
961 10.7249072045627
962 10.7261681778673
963 10.7289349765728
964 10.7234064649741
965 10.7288796100737
966 10.7240577155189
967 10.7223229499566
968 10.7250613337165
969 10.7277763111318
970 10.7234881452778
971 10.7286398550483
972 10.7310940076407
973 10.7408836295776
974 10.7402687345177
975 10.7414937555228
976 10.737631202037
977 10.7329499717766
978 10.7333606635976
979 10.7357622296164
980 10.7304486445482
981 10.7249789997318
982 10.7261929477217
983 10.7208981604862
984 10.7167057916285
985 10.7120779488059
986 10.7073490264172
987 10.7100913224888
988 10.7201852646786
989 10.729833862377
990 10.7292413875812
991 10.7238273710633
992 10.7184571424969
993 10.7169651002385
994 10.7161580060864
995 10.718839812738
996 10.713624808824
997 10.7232117734845
998 10.7258848700963
999 10.7207240854294
1000 10.7246445162532
};
\addplot [semithick, color3, forget plot]
table {%
1 0
2 1.5
3 11.6141675934562
4 11.6699400169838
5 12.4032253869709
6 11.4017542509914
7 12.4752816826405
8 11.8532695911297
9 11.1764744104263
10 10.6906501205493
11 10.1931754839344
12 9.86013297183269
13 10.5393037282794
14 11.2732752425317
15 10.8913829343303
16 10.6682941466759
17 10.6543686314233
18 10.7813567940676
19 10.7011791619926
20 10.5090437243357
21 10.2813041691947
22 10.0937751867506
23 9.99111136151456
24 9.99157631318614
25 10.292055188348
26 10.1678952922695
27 9.99190344382526
28 9.85060601057164
29 9.75246546467685
30 9.83988482317417
31 10.0792902315232
32 9.92924185172262
33 10.2691957816489
34 10.3615021422378
35 10.4344797353265
36 10.4610904584678
37 10.5943877530829
38 10.6720179328468
39 10.9517500155166
40 10.8620612684702
41 11.05849462589
42 10.9773266636405
43 10.8489805319679
44 10.7300410526584
45 10.8640533099294
46 11.1156491467944
47 10.997551066958
48 10.9406344714961
49 10.8311796361955
50 10.8923642979842
51 10.8037258442824
52 10.9087799175683
53 10.8940004260611
54 10.9564545245074
55 10.9190257792049
56 10.911459525636
57 10.9221481871403
58 11.0948778232124
59 11.0521991478629
60 11.129677543497
61 11.0986149085824
62 11.0265147283786
63 10.9855490408949
64 11.0176970568944
65 11.0355638115618
66 11.1460893420819
67 11.1197736089691
68 11.2075805413878
69 11.216773379228
70 11.1611205640338
71 11.1348161247912
72 11.2551685657943
73 11.184358968309
74 11.1911148623662
75 11.2592401559292
76 11.2736286179768
77 11.2192609602947
78 11.2245694748086
79 11.1603654694043
80 11.1770971186619
81 11.1182692991537
82 11.2198038138135
83 11.3334244130922
84 11.4106195812947
85 11.3997875291315
86 11.4014755717683
87 11.3507827712635
88 11.3527220313307
89 11.3923940899452
90 11.3486606595418
91 11.3475679612607
92 11.4195840471956
93 11.3612620313537
94 11.3509043040496
95 11.315471231755
96 11.25861938324
97 11.2109787831304
98 11.2390675280674
99 11.2214084749835
100 11.2820920045885
101 11.2276441927157
102 11.236303551136
103 11.2089722757174
104 11.217243567581
105 11.2459778650231
106 11.1942300653887
107 11.278132012095
108 11.2751281628708
109 11.3411120778799
110 11.2898879643659
111 11.2392794468135
112 11.2769765847437
113 11.2946982739764
114 11.2831025248116
115 11.2429301456902
116 11.2388007408596
117 11.303327516475
118 11.2936853309237
119 11.2913537034147
120 11.2972956449271
121 11.3210109644059
122 11.2808209002158
123 11.2696534801692
124 11.2241378070178
125 11.1842060066864
126 11.2327356962572
127 11.2315651757051
128 11.1890465716965
129 11.1626022809249
130 11.1209169091136
131 11.0953509163679
132 11.0655110543706
133 11.0903300619519
134 11.1577546564282
135 11.2222002200004
136 11.1819234956372
137 11.1498733652715
138 11.1217413966978
139 11.1349105338416
140 11.1956145751042
141 11.168562300672
142 11.1650437857427
143 11.1300247627806
144 11.1249219722607
145 11.1002137074906
146 11.0686965938173
147 11.0526277979175
148 11.0152626699214
149 10.9946324292744
150 11.0321328652059
151 11.0176107278976
152 11.0274383927966
153 11.0382733705866
154 11.0478287529799
155 11.0325674357679
156 11.0242004872897
157 10.9897809350969
158 10.9990039752409
159 11.0175468034296
160 11.0589312746531
161 11.0899260076771
162 11.1430952550916
163 11.1943069038282
164 11.1681269596044
165 11.2157995604243
166 11.2527992033739
167 11.2584246008724
168 11.2373369775467
169 11.2723896921114
170 11.2391878069794
171 11.2073822568609
172 11.177164921578
173 11.1511684703683
174 11.1497689246724
175 11.176178748954
176 11.1874004450993
177 11.2367299671348
178 11.2600654238474
179 11.2452674800494
180 11.2205774372218
181 11.2510905901455
182 11.2731199186988
183 11.2423908660198
184 11.2757948344823
185 11.2464652348004
186 11.2169505298163
187 11.2349263213972
188 11.2069242221315
189 11.1861812714407
190 11.1795284317219
191 11.1793995973413
192 11.1843571004327
193 11.1737838591091
194 11.1640202944015
195 11.1437726394185
196 11.1300245685769
197 11.124416150888
198 11.0998014066107
199 11.071878808084
200 11.0709879866252
201 11.1147173672972
202 11.1060558269694
203 11.0934538747899
204 11.0669947909278
205 11.1121330219469
206 11.111183913315
207 11.1202776723132
208 11.1430557471071
209 11.1181656646713
210 11.0996476011965
211 11.0807612922587
212 11.0685668425191
213 11.0480027989008
214 11.0430210659608
215 11.0378903934348
216 11.0143783071245
217 11.0248908024396
218 11.0342260840517
219 11.0226362185004
220 11.056215297027
221 11.0313141415409
222 11.0709981226565
223 11.0507013151927
224 11.0326139527937
225 11.0293668600393
226 11.0220626110708
227 11.0566754847266
228 11.0718808637135
229 11.0476810962203
230 11.051528219773
231 11.0375009644717
232 11.0759726451302
233 11.0537245150804
234 11.0345521532539
235 11.0119849109728
236 10.9905902075901
237 10.9801952945881
238 10.9615978468846
239 10.9804586974316
240 10.9906494348605
241 10.9722082007406
242 10.9497832346459
243 10.9299512862586
244 10.9118304085438
245 10.8928704774568
246 10.8733849846551
247 10.8532822385913
248 10.8346646637859
249 10.8178742274051
250 10.8032446977748
251 10.781706843404
252 10.7609236278427
253 10.7747137069715
254 10.7567132021709
255 10.7653994562092
256 10.7633342991089
257 10.7993611736667
258 10.8121112042462
259 10.7921819640215
260 10.7903200099515
261 10.7701551512121
262 10.8085403281036
263 10.7986102921392
264 10.8112165634246
265 10.8314745144251
266 10.8178329717223
267 10.8022526190075
268 10.8306563028332
269 10.8195249016444
270 10.8018974099147
271 10.7921991367097
272 10.7747035500609
273 10.8080559017111
274 10.827960350176
275 10.817172756607
276 10.8388479008853
277 10.8582419977106
278 10.8409943993764
279 10.8313680546161
280 10.8166514679725
281 10.7976620715749
282 10.7830833053045
283 10.7751941235717
284 10.7625565037094
285 10.7465533424048
286 10.7306182728116
287 10.7120141497269
288 10.7215183095078
289 10.7105180722274
290 10.6936499765815
291 10.6776268221108
292 10.6961145157791
293 10.7095528280413
294 10.6926831134368
295 10.6939595049593
296 10.6887706208409
297 10.6934298894518
298 10.6979044736772
299 10.6814679394043
300 10.676102076861
301 10.6711708168173
302 10.6574482207907
303 10.6399896778134
304 10.6721228736782
305 10.65694318528
306 10.6418260712131
307 10.6731751391444
308 10.6891085530852
309 10.6862037122911
310 10.6973422939535
311 10.7124627118414
312 10.6956847453967
313 10.6794173332382
314 10.6704476323894
315 10.6673724256314
316 10.7002950750509
317 10.7134734363362
318 10.6977161899126
319 10.717083006195
320 10.7089144925793
321 10.6986593326816
322 10.7220770123021
323 10.721085527774
324 10.7395322684708
325 10.7421302968084
326 10.7693882188848
327 10.7868331210395
328 10.7731513360069
329 10.7692599013751
330 10.7606039271906
331 10.7601647176246
332 10.7448907943548
333 10.7287938244165
334 10.7184462075074
335 10.7064785736688
336 10.7061870020041
337 10.7030152353635
338 10.6900447204624
339 10.7070999268948
340 10.7281841886761
341 10.7488891360432
342 10.7665539696562
343 10.7682617322806
344 10.7709682682873
345 10.7792826411366
346 10.7674767043709
347 10.7590702605274
348 10.7788834844626
349 10.7724218967493
350 10.7621259564289
351 10.7556494955656
352 10.7506426388733
353 10.7618752350332
354 10.7547310891188
355 10.7722729592869
356 10.7798621062338
357 10.7908190935945
358 10.7779556742727
359 10.8007393015571
360 10.7983480675358
361 10.7894583936254
362 10.8066937399957
363 10.796649004003
364 10.7896206811315
365 10.8121808111865
366 10.7974951419641
367 10.7828691478507
368 10.7686456313438
369 10.7781515328925
370 10.7853013224716
371 10.7891416167679
372 10.814372464894
373 10.8004430383058
374 10.8042358450988
375 10.7899159506561
376 10.790773987443
377 10.7841242351496
378 10.7934436254628
379 10.7810032842695
380 10.7673775668725
381 10.767543715155
382 10.7660237226385
383 10.7823130314525
384 10.783378098644
385 10.7739414391354
386 10.7840020023438
387 10.8082588990577
388 10.7963849847272
389 10.7830496107188
390 10.7720130716003
391 10.7681703756878
392 10.7921149770971
393 10.7800927430781
394 10.783862543646
395 10.7722319234261
396 10.7591628150696
397 10.7510977261158
398 10.7548289995436
399 10.7441022048197
400 10.7323922659396
401 10.7217281376243
402 10.7373984489885
403 10.72497051079
404 10.7335806104171
405 10.7453017136013
406 10.7603054492049
407 10.7492063165087
408 10.7360361011379
409 10.7265863216091
410 10.7401244707237
411 10.7337817418956
412 10.7267703275016
413 10.7197592070976
414 10.7197556020228
415 10.7128042440518
416 10.712797493677
417 10.7142171591125
418 10.7017249421415
419 10.6933381300205
420 10.6811707830674
421 10.6837749960422
422 10.674205270713
423 10.6720363351733
424 10.6735222743567
425 10.6803304641537
426 10.6966975836642
427 10.7020590325999
428 10.6920278813173
429 10.6810957755712
430 10.6711199746949
431 10.6617404057805
432 10.6874433452962
433 10.6815017768807
434 10.6699673073655
435 10.6588804320624
436 10.6698145018196
437 10.6697549626176
438 10.6596070828377
439 10.6521660330818
440 10.6515150437519
441 10.6414433482859
442 10.6351180510393
443 10.6234056834257
444 10.6227226179025
445 10.633573944658
446 10.6497162965974
447 10.6401140729904
448 10.6539912004695
449 10.6635863967475
450 10.6563830057591
451 10.6734054299947
452 10.6706693451917
453 10.6608356501952
454 10.6496510385875
455 10.6519689021595
456 10.6410014752429
457 10.6535624828608
458 10.6452179464974
459 10.6391220959384
460 10.6414540114714
461 10.6408067447071
462 10.6431177427338
463 10.6348497147636
464 10.6327756660035
465 10.6271915234583
466 10.6506095679104
467 10.6697063246126
468 10.6602414200024
469 10.6559092414552
470 10.6489402087267
471 10.644639565385
472 10.6576043418483
473 10.6506386428334
474 10.6481748661377
475 10.6388727455684
476 10.6548630226131
477 10.6645407642456
478 10.6664266763383
479 10.6660208293084
480 10.6603315741507
481 10.6503995423787
482 10.6413716933156
483 10.6309315021049
484 10.6254886949329
485 10.6164538003784
486 10.6083982021374
487 10.5980748305237
488 10.5912043340135
489 10.5889203259025
490 10.5852293495783
491 10.603555876763
492 10.5938687391115
493 10.5861456829973
494 10.5958063232237
495 10.6077035764486
496 10.5990277146094
497 10.6006604778527
498 10.596638778745
499 10.6082599220409
500 10.6197333299853
501 10.629419261954
502 10.6208758173201
503 10.6184501293183
504 10.6253834798077
505 10.6153674858722
506 10.6168134265993
507 10.6150437850289
508 10.6368253679896
509 10.6263839847275
510 10.6164774037548
511 10.6141527204938
512 10.6177796008555
513 10.6080843309561
514 10.6202440766707
515 10.6099397265506
516 10.6142319249114
517 10.6042006760656
518 10.6161157026998
519 10.610864329706
520 10.600671067013
521 10.611915564233
522 10.6017819801259
523 10.6083655745764
524 10.5982523313733
525 10.6099933212347
526 10.6160220127917
527 10.6108322935668
528 10.6013889873871
529 10.5919707098483
530 10.582577351299
531 10.5737599027328
532 10.5876943067518
533 10.5935829739178
534 10.5950303265508
535 10.5851333707162
536 10.5846436879304
537 10.5814188539374
538 10.5904582526192
539 10.5924936531597
540 10.598292140543
541 10.6126176145212
542 10.6213674019334
543 10.6179751574207
544 10.6344336237551
545 10.6252592119962
546 10.6156925471735
547 10.6106864455093
548 10.6187515905992
549 10.6388264290865
550 10.6296371893334
551 10.6201503250303
552 10.6369456366182
553 10.6333330174369
554 10.6493686804174
555 10.654860156937
556 10.6498191278601
557 10.6575319173274
558 10.6503402677695
559 10.6457914050003
560 10.6624789159301
561 10.653562792339
562 10.6458888856343
563 10.6409346074085
564 10.6574276880639
565 10.6552744461046
566 10.6458678775176
567 10.6422491793836
568 10.6334622123886
569 10.625671550327
570 10.6257230123527
571 10.6358082612852
572 10.6276146127245
573 10.633646912881
574 10.6336457154708
575 10.6462840859766
576 10.6396407616172
577 10.6306364651227
578 10.6231863439179
579 10.6331747100074
580 10.6326645433457
581 10.636496048224
582 10.629660338535
583 10.6355787232069
584 10.6266852890583
585 10.6319391404066
586 10.6425090824021
587 10.6575319038421
588 10.6527564606149
589 10.6649072353445
590 10.6659932876601
591 10.6600888979313
592 10.6601964787468
593 10.6559065819262
594 10.6477918786302
595 10.663535896263
596 10.6791214723791
597 10.670723031985
598 10.6620086856184
599 10.6565539311997
600 10.6685888892997
601 10.6780365732479
602 10.6962551420205
603 10.6919156543059
604 10.6875737932538
605 10.6874575966631
606 10.6908912747394
607 10.6854215463202
608 10.6781350967232
609 10.6726941886575
610 10.6760664850207
611 10.6835334369319
612 10.6982528657293
613 10.6895422048094
614 10.6863308894945
615 10.6778136777871
616 10.6791621011906
617 10.6760210351192
618 10.6675465092953
619 10.679553130452
620 10.6792410247997
621 10.6887791328911
622 10.6856123109527
623 10.6887285137865
624 10.6855291137898
625 10.6852947474555
626 10.6969990066928
627 10.6984079080929
628 10.6907781538572
629 10.7002102222657
630 10.6932031261217
631 10.7025590200725
632 10.7118173044142
633 10.7042423629203
634 10.7091445294504
635 10.7184635545829
636 10.7253732791257
637 10.7367206359974
638 10.7324884857523
639 10.7462112088941
640 10.7399371506541
641 10.7409837568993
642 10.7334440225857
643 10.7260445890543
644 10.7353402814934
645 10.7278525018628
646 10.7309824355129
647 10.7304982284356
648 10.7222271086082
649 10.7141049394369
650 10.7122002005235
651 10.713625997561
652 10.720239237119
653 10.7212720431661
654 10.7193210016665
655 10.7115287192379
656 10.7048998908843
657 10.7043646754652
658 10.7108029649642
659 10.7247821363501
660 10.7279532398428
661 10.7199662403003
662 10.712055519829
663 10.7184789219008
664 10.7298660815786
665 10.7362397421645
666 10.7281848327507
667 10.7325676654892
668 10.7245630180107
669 10.7350902948242
670 10.733084885559
671 10.7355542461985
672 10.7370887262882
673 10.7293115660841
674 10.7244596843037
675 10.7217358298174
676 10.7145590272269
677 10.7073964046402
678 10.7042405780391
679 10.6990751846973
680 10.6915627205897
681 10.6837457147667
682 10.6928441776351
683 10.68512944207
684 10.6800288975813
685 10.6786663726779
686 10.6712426493157
687 10.6644053542086
688 10.6571460389502
689 10.6571809806989
690 10.6683280842211
691 10.6607294596246
692 10.667552145773
693 10.6598795505062
694 10.6725565056255
695 10.6751377176902
696 10.6682108435323
697 10.6750194536984
698 10.6793038348922
699 10.6925513445446
700 10.6850927053746
701 10.6782185211203
702 10.6884092451033
703 10.6858077796155
704 10.6981605523821
705 10.7006086274914
706 10.6930604760496
707 10.6912053186295
708 10.7056940529015
709 10.7003123681615
710 10.6936916294858
711 10.7080826841255
712 10.7096125382577
713 10.7060176670167
714 10.7179998505577
715 10.7122976044706
716 10.7171403441761
717 10.712617778004
718 10.7069585561281
719 10.7117530603402
720 10.7203074715211
721 10.7211257897284
722 10.7339316405797
723 10.7265121353938
724 10.7278886377845
725 10.7318919965865
726 10.7252004865011
727 10.718521294615
728 10.7155673367033
729 10.7282616751587
730 10.7328566932527
731 10.7258591491549
732 10.7210785640047
733 10.718552595363
734 10.712632735004
735 10.7244015737777
736 10.7175700967316
737 10.7292513589547
738 10.7226646088381
739 10.7329997725871
740 10.728251906534
741 10.7290389455468
742 10.7312808301901
743 10.7288293152695
744 10.7222917139926
745 10.7210017434872
746 10.7346986117725
747 10.7439891589243
748 10.7542422898981
749 10.750768029687
750 10.7501441023933
751 10.7573972799327
752 10.7505536799311
753 10.7458175077304
754 10.7407395784307
755 10.7364383834621
756 10.7498198230607
757 10.7579840900081
758 10.7522696109749
759 10.7456460028303
760 10.7409691984164
761 10.7446344317407
762 10.7412645062847
763 10.734529674567
764 10.7302872201887
765 10.725647201059
766 10.7233107072546
767 10.7333479495252
768 10.7354836246972
769 10.7391324081287
770 10.747156189723
771 10.7429235976428
772 10.7528165476032
773 10.747818776266
774 10.7609647714349
775 10.7548517909242
776 10.7618872693182
777 10.7566123184567
778 10.7537624607302
779 10.7470423752723
780 10.7401545820847
781 10.7337290744501
782 10.7314236673527
783 10.7302106458828
784 10.7431750343567
785 10.7390313620982
786 10.733269997399
787 10.7440377368918
788 10.7484055350805
789 10.7435424596291
790 10.7368330993084
791 10.7316667957384
792 10.7423594636248
793 10.7474443770863
794 10.7480243491352
795 10.7540789100504
796 10.7479231986704
797 10.7442069695203
798 10.7455888993327
799 10.742786280501
800 10.7360723003108
801 10.7313222039455
802 10.7429919995667
803 10.7489283429753
804 10.7422742109797
805 10.7473272857437
806 10.746696086028
807 10.7480286114623
808 10.7556679083719
809 10.7576377407439
810 10.7528985398393
811 10.7501622883766
812 10.750764580564
813 10.7441853931627
814 10.7383989482745
815 10.7478594668625
816 10.7491379870405
817 10.7575947205319
818 10.7558667277827
819 10.7494757605346
820 10.743096160328
821 10.748894699355
822 10.7433775370513
823 10.7369366393491
824 10.736348009053
825 10.7298722411022
826 10.7382258087056
827 10.7401246541785
828 10.734077413596
829 10.7285938148881
830 10.7304766052283
831 10.7347092287609
832 10.7284372223719
833 10.7263217541738
834 10.7269109555141
835 10.7219942990266
836 10.7170835512645
837 10.7189511896713
838 10.7264794040098
839 10.7384456816489
840 10.7341608646005
841 10.7399555121519
842 10.7393018222827
843 10.7375692484744
844 10.7316516679118
845 10.7253384749933
846 10.7210887344761
847 10.7149453548799
848 10.7129254114092
849 10.7192462748863
850 10.730989886522
851 10.7337453963172
852 10.7295064277907
853 10.726828999531
854 10.7347238550743
855 10.7310304188561
856 10.7425988041003
857 10.7471068512101
858 10.751577436562
859 10.7467036292857
860 10.7417753134574
861 10.7398514029697
862 10.73493075047
863 10.7460212821826
864 10.7574158528519
865 10.76314771145
866 10.7722422805384
867 10.7778774223324
868 10.7743834442392
869 10.7714931580408
870 10.7728085175498
871 10.7680327804484
872 10.7643608091193
873 10.7720510099526
874 10.7700624482527
875 10.7730574895649
876 10.781975737923
877 10.7845580519278
878 10.7835250470698
879 10.7786193021933
880 10.7757185624528
881 10.7862959892911
882 10.7826050710199
883 10.7830935786557
884 10.7801823663766
885 10.7750176546585
886 10.7693446101135
887 10.76828853264
888 10.7634187856008
889 10.7658858026043
890 10.7618646779753
891 10.7567563695256
892 10.7585065955168
893 10.753017685178
894 10.7530542711469
895 10.7470768611408
896 10.7411553307457
897 10.7405822275113
898 10.7377451721988
899 10.73202593307
900 10.7432134765381
901 10.738442149369
902 10.7338827599353
903 10.727971148777
904 10.727390181663
905 10.7218700894172
906 10.7159531028393
907 10.7117816228149
908 10.7058833523956
909 10.7011795426284
910 10.6966846128847
911 10.7005846985004
912 10.7023448009948
913 10.6995978925134
914 10.7055467710611
915 10.7039991014161
916 10.6988999957103
917 10.7033018694701
918 10.6983526914058
919 10.7052756271847
920 10.71380573234
921 10.7082319021339
922 10.7083328111449
923 10.7039001007691
924 10.698270202531
925 10.7026653562002
926 10.7008013743993
927 10.6950300042208
928 10.6896621245806
929 10.6847941706084
930 10.6821122563334
931 10.6912329360833
932 10.6880835749567
933 10.6824235362922
934 10.6784028657498
935 10.6765830705842
936 10.6711181608232
937 10.6728440657991
938 10.6766899187387
939 10.6857222179725
940 10.6862440822246
941 10.6856927147186
942 10.6833155226746
943 10.6875629823167
944 10.6888830980173
945 10.6862839260025
946 10.6836831601876
947 10.6784408378147
948 10.6836772010318
949 10.6888771103204
950 10.6883574529275
951 10.6940563520951
952 10.6945521334003
953 10.7012503899679
954 10.7013834848335
955 10.6982984925745
956 10.6938411960632
957 10.6901148100742
958 10.6845353597541
959 10.680101397729
960 10.6775220946434
961 10.6791906121503
962 10.6749385862991
963 10.6778157847667
964 10.6806687996339
965 10.676799955517
966 10.6731005473048
967 10.6684089832872
968 10.6686108044659
969 10.6649288331203
970 10.6612494270281
971 10.6679081324076
972 10.6628113190261
973 10.6586081340948
974 10.6555923118882
975 10.6501622808056
976 10.6459742076596
977 10.6452010604167
978 10.640581342353
979 10.6472073291875
980 10.6542664890171
981 10.6624197209875
982 10.6648603045652
983 10.6618839741562
984 10.6566875737236
985 10.6595181948122
986 10.6650137350886
987 10.6704732708754
988 10.6690100002646
989 10.6692339188093
990 10.6696900845984
991 10.6765587727802
992 10.6729258506398
993 10.6810355843723
994 10.675831276291
995 10.6705223854592
996 10.6675355058764
997 10.662184765053
998 10.668716829548
999 10.6681675928802
1000 10.676521484079
};
\addplot [semithick, color4, forget plot]
table {%
1 0
2 2
3 11.4309521329882
4 9.90896059130321
5 9.06421535489973
6 9.26762584963856
7 9.76666550579269
8 9.31061222476804
9 10.1555798686799
10 10.9799817850486
11 10.6499524695327
12 11.1342813967594
13 11.2307692307692
14 10.8939694982394
15 10.6136180866323
16 10.3438916153448
17 10.453058690458
18 10.2422810441646
19 10.0221361643314
20 10.3993990210973
21 10.1929236046375
22 9.96533247010201
23 9.97084597093855
24 9.85696313385731
25 9.96762760139041
26 9.8227636839401
27 9.89700041850815
28 9.71920831159008
29 9.70504361563876
30 10.4592330290302
31 10.290878739142
32 10.769604217426
33 10.6057142800599
34 10.4521069546952
35 10.4475365712767
36 10.8715848901939
37 11.3302893404193
38 11.2221338473858
39 11.096314934617
40 11.3929802949009
41 11.3691505738542
42 11.3210467613751
43 11.2050556464009
44 11.1591603404617
45 11.0513057158279
46 11.2428090859743
47 11.514904624863
48 11.4288256420139
49 11.3123463308254
50 11.4176880321718
51 11.3069099666974
52 11.2751362840922
53 11.3937270781748
54 11.5531811425698
55 11.6200738251734
56 11.6313346795874
57 11.6521080512305
58 11.629537202869
59 11.7579485299951
60 11.722473667656
61 11.7307195992651
62 11.6701991579877
63 11.6014699201605
64 11.7722113868848
65 11.9276318842788
66 11.8369799991577
67 11.9788370601421
68 11.9001759168178
69 11.8192826404146
70 11.7345556336134
71 11.6539742598138
72 11.7433656079707
73 11.7484742264859
74 11.7254943156643
75 11.7878732414102
76 11.7422692196437
77 11.7650869542323
78 11.7070097063945
79 11.6429413654988
80 11.7475462863527
81 11.8091824494102
82 11.9000459909429
83 11.848281227081
84 11.9087314810406
85 11.9335182629218
86 11.9050688878496
87 11.8411113089098
88 11.865720602358
89 11.8073262388225
90 11.8
91 11.7520164274005
92 11.8201461660812
93 11.7866924379332
94 11.7276124817632
95 11.7183758435191
96 11.6618293543032
97 11.6116528304082
98 11.6123749140887
99 11.6690670494041
100 11.6302020618732
101 11.5783432979906
102 11.5333501676644
103 11.4993104499282
104 11.5808037196431
105 11.5518117081909
106 11.5228038645099
107 11.5700179666282
108 11.5183994277516
109 11.5400445651921
110 11.4875282713525
111 11.4431579899826
112 11.4210319718062
113 11.3976941554871
114 11.3598387845713
115 11.3184193356267
116 11.2717750686648
117 11.2452066070019
118 11.1975420005621
119 11.1527770589837
120 11.1174412523746
121 11.0919313245948
122 11.0803210776624
123 11.042476231083
124 11.0000088685992
125 10.9605839260507
126 10.9215721497051
127 10.8935008760601
128 10.8986783127517
129 10.8883302591896
130 10.9425097994043
131 10.919873342712
132 10.9231049506578
133 10.996131996122
134 10.9565989010608
135 10.9576713956144
136 11.0243812057121
137 10.9856560826034
138 10.9713423457225
139 11.0284648948979
140 11.0144501562674
141 11.027933219831
142 11.094721209768
143 11.0560922306211
144 11.0207348766869
145 11.0846274415174
146 11.1459129501619
147 11.1455023315337
148 11.1853103180088
149 11.2066711360441
150 11.1719808250621
151 11.2010779083699
152 11.1789693601726
153 11.1920084632288
154 11.1660843043199
155 11.1325641894328
156 11.1532419523945
157 11.120315019919
158 11.0853703152596
159 11.0642034256751
160 11.0775603699551
161 11.0501241103114
162 11.0385904947026
163 11.0145237538457
164 10.9820780398145
165 10.9707081402367
166 11.0064659205308
167 10.9899050422013
168 10.9591787910063
169 10.9608640762293
170 11.0037678353551
171 10.9801329057509
172 11.0101037107254
173 11.0603941983811
174 11.0512026105122
175 11.0585085904428
176 11.0567291293246
177 11.0267868578378
178 11.0713333624009
179 11.0403726683945
180 11.0104944327819
181 10.9968014418885
182 10.9731382618469
183 10.9924793566675
184 10.9688239681315
185 10.9430479980981
186 10.9923979659568
187 10.9632763563276
188 10.9490814483751
189 10.9220139812578
190 10.9723702356081
191 10.9976186771286
192 11.0370273272391
193 11.0371183599155
194 11.0273768251065
195 11.0328134095128
196 11.0048678695531
197 10.9826625671828
198 10.9660607940054
199 10.9528881323806
200 10.9396880668509
201 10.927441375975
202 10.9190325593504
203 10.929779018915
204 10.9063361846897
205 10.9376079241722
206 10.9166803079157
207 10.9617923064254
208 10.9459477097597
209 10.9908607565082
210 11.019879994902
211 11.0480307476533
212 11.0425999200346
213 11.0763900181693
214 11.0821301280659
215 11.0569947809394
216 11.0900021321373
217 11.0957407238029
218 11.0770926761666
219 11.0962981679029
220 11.0901117448111
221 11.0655255147225
222 11.0417341421463
223 11.0287119866929
224 11.0459347898968
225 11.0685460251612
226 11.0591854960414
227 11.0520425720963
228 11.0738935007906
229 11.0986276222146
230 11.1167451479428
231 11.122218918803
232 11.1058700708106
233 11.1132480686577
234 11.0936820896834
235 11.0833493931635
236 11.067416188598
237 11.049335113086
238 11.0441078561874
239 11.0339749791709
240 11.044039839308
241 11.0429306052018
242 11.0606435405161
243 11.0778514553564
244 11.0986535000229
245 11.0784819063551
246 11.0593620249887
247 11.0818584075973
248 11.075417252894
249 11.0821603123704
250 11.0971628806646
251 11.1117316306715
252 11.0990694386344
253 11.0803042050752
254 11.0653316662055
255 11.0619867538316
256 11.0921096663613
257 11.0814565724931
258 11.0888756540705
259 11.112790115122
260 11.0915361675565
261 11.0915311693563
262 11.1085627366567
263 11.1202553449593
264 11.1382702964791
265 11.1420486185894
266 11.1251679679708
267 11.1522311397701
268 11.1648225929963
269 11.1456522238112
270 11.1307782731356
271 11.1420760444523
272 11.1446134355972
273 11.1296820478038
274 11.110608823546
275 11.1043292875175
276 11.0847178520372
277 11.0646986730601
278 11.0822685473453
279 11.0623940663822
280 11.044228684924
281 11.0261515743617
282 11.0331185183623
283 11.0178978980826
284 11.0059377627935
285 11.0065069313192
286 10.9899749372875
287 10.9767622867026
288 10.9618132544935
289 10.9740832194927
290 10.990167478835
291 10.9713951669512
292 10.9649277109912
293 10.9535302547463
294 10.9470092882012
295 10.9640157290924
296 10.9794261912305
297 10.9767862618339
298 10.9661477332874
299 10.9604606997413
300 10.9674361432176
301 10.9752659627431
302 10.9904952137979
303 10.98159514007
304 11.0075030285
305 10.9895604122406
306 10.98330182426
307 10.9688898468633
308 10.988081274244
309 10.9779292069075
310 10.9842612786509
311 10.9665951412859
312 10.9873622752051
313 10.9867889105867
314 10.9741285865996
315 10.9994800490962
316 11.0150841209951
317 11.018700309504
318 11.0478782980558
319 11.0625071602753
320 11.0859638169173
321 11.0688334946072
322 11.0605166056479
323 11.067186953963
324 11.0506620405669
325 11.0371396689091
326 11.036964979454
327 11.0649859508509
328 11.0712629810084
329 11.0936961193039
330 11.0773621521448
331 11.0638648504875
332 11.0561696019898
333 11.0406923641079
334 11.025464854394
335 11.0091049622348
336 10.9940133774959
337 10.9797409456152
338 10.9685074749039
339 10.9524778156853
340 10.9661302249575
341 10.9501395592825
342 10.9347264812677
343 10.9352487367928
344 10.9356725082246
345 10.9199620081001
346 10.9382036844704
347 10.9245509299294
348 10.9424000734974
349 10.9558982953132
350 10.9413370820415
351 10.932291797087
352 10.9215468070732
353 10.9349249215415
354 10.9347695494723
355 10.9374481606729
356 10.954165910239
357 10.9810859839102
358 10.9911838945761
359 10.9802349919421
360 10.9978266652109
361 11.0104897139463
362 10.9965030473821
363 10.984633537626
364 10.9754559255559
365 11.0013117831248
366 10.9988706702725
367 10.9869004243169
368 10.9781303801974
369 10.9990141575939
370 10.9940683920958
371 10.9852849065345
372 10.9783113077488
373 10.9637144524014
374 10.9666472124523
375 10.9618340720074
376 10.9483165328132
377 10.9414545438046
378 10.9417263437461
379 10.9477036357796
380 10.9477894023307
381 10.9477988522168
382 10.947733179232
383 10.9394193432821
384 10.9280511641415
385 10.9402171901344
386 10.9264525855278
387 10.9469222187489
388 10.9589277562251
389 10.9562523597916
390 10.9567138880321
391 10.942695467816
392 10.9431151915619
393 10.9549600323973
394 10.9749444675408
395 10.9727558787694
396 10.981199904924
397 10.9701154125381
398 10.9605918463319
399 10.9841644090322
400 10.9996565287285
401 10.9887711929731
402 11.0076976673348
403 11.0051773459267
404 10.9926085351752
405 10.9953431275383
406 11.0038058798929
407 10.9957418167417
408 10.986242040718
409 11.0049956293692
410 11.0102837117058
411 11.0008241844239
412 11.0191266374519
413 11.0076026660022
414 11.0102479319485
415 10.9973950121505
416 10.9894505352942
417 10.9944979885312
418 10.9899601105768
419 10.9948358581596
420 10.9902190590537
421 10.9855820748816
422 10.9754543070546
423 10.9643463324754
424 10.9719574325687
425 10.9824889874683
426 10.9819645497694
427 11.0007640767149
428 11.009319018421
429 11.0046987891109
430 11.026809602792
431 11.0164709698751
432 11.0125370377533
433 11.0033962810342
434 11.010714569595
435 11.0135495886154
436 11.0044492440429
437 10.9918545393345
438 10.987365059322
439 10.9987430476545
440 10.9870660248635
441 10.9747437222295
442 10.9914457358279
443 10.9843610093222
444 10.9848038633339
445 10.9729340725871
446 10.9757098559081
447 10.9829794534534
448 10.9754550538913
449 10.9782530969779
450 10.9684224644214
451 10.9660183404047
452 10.9553465405901
453 10.9432516805998
454 10.9514284392042
455 10.968872976522
456 10.9691912703385
457 10.9650709355392
458 10.9609326859624
459 10.9494406769468
460 10.9375344449903
461 10.940241643394
462 10.9607508698796
463 10.9742922730658
464 10.962551572164
465 10.9672106133163
466 10.9569504185655
467 10.9699459670601
468 10.958568265223
469 10.96357187077
470 10.9829739833635
471 10.9854855637821
472 10.9929642351804
473 10.9861321617435
474 10.9908616361856
475 10.9980841444156
476 10.9889044064948
477 10.9934300440554
478 10.985295886893
479 10.9739074914103
480 10.9718381613515
481 10.9612343946415
482 10.9546028200726
483 10.9590265036136
484 10.9551603619532
485 10.9439340662887
486 10.9361830210665
487 10.9250636364988
488 10.9139396203579
489 10.9028495128706
490 10.8949655752308
491 10.8912283165862
492 10.8892194382132
493 10.8959864616811
494 10.8850827649276
495 10.8849253482397
496 10.8953127601037
497 10.8995117087451
498 10.8973986957392
499 10.9077285561592
500 10.897115948727
501 10.8969174443786
502 10.9071725589878
503 10.899405072556
504 10.8941248864733
505 10.8968614765992
506 10.8862159566577
507 10.9053580849703
508 10.8955126278845
509 10.9074320704079
510 10.8967946921849
511 10.8868433855998
512 10.8769191057109
513 10.8723091595165
514 10.877219466123
515 10.869711441721
516 10.8595873854766
517 10.8491977004973
518 10.8408119295348
519 10.8555159528675
520 10.8480588547979
521 10.8451242342109
522 10.8457628963699
523 10.8522371748113
524 10.8420012866739
525 10.8317943019822
526 10.8230282790768
527 10.8416519162302
528 10.8326788780385
529 10.8231434120357
530 10.8170074609107
531 10.8141244102108
532 10.8230848133273
533 10.8182230809014
534 10.8081451909578
535 10.8063418911161
536 10.7974933509723
537 10.8146356447423
538 10.8057853819038
539 10.8047819448536
540 10.7967116049972
541 10.79327432975
542 10.7845032591158
543 10.7796601251309
544 10.787122447855
545 10.784506650982
546 10.7797282880196
547 10.7922406871957
548 10.7825167360844
549 10.7993178799085
550 10.7974738398532
551 10.8047711940785
552 10.8228416065784
553 10.814245786552
554 10.8095215362193
555 10.8035852730494
556 10.8073754162141
557 10.8033484396991
558 10.7955265245389
559 10.8004267078851
560 10.7907840497419
561 10.7823271251054
562 10.7764860195214
563 10.794334599949
564 10.8090911684904
565 10.8050182242718
566 10.8088772643908
567 10.8126741560453
568 10.8032000811176
569 10.80312413238
570 10.8058099120207
571 10.7963477967315
572 10.7891167942542
573 10.8036290828412
574 10.8036264130607
575 10.80104314783
576 10.8056268240205
577 10.7965279309432
578 10.7878247404079
579 10.7796740002013
580 10.7967626274845
581 10.7927578597591
582 10.7872263493271
583 10.7873191695416
584 10.781780164677
585 10.7932758292651
586 10.7917199602151
587 10.7954995926224
588 10.7972856834245
589 10.7934107500017
590 10.7872721238318
591 10.7781938380195
592 10.7826951755281
593 10.7965967968157
594 10.7893472856838
595 10.7803337120824
596 10.7941211413394
597 10.7877594383032
598 10.7862813981328
599 10.7786017088777
600 10.7787278934019
601 10.7762186224721
602 10.7684120476334
603 10.7645837211091
604 10.7809031985945
605 10.783203020722
606 10.789100425594
607 10.7879971777571
608 10.7792121402948
609 10.7741804614868
610 10.7826077771418
611 10.7740993576621
612 10.7781840373378
613 10.7742590647271
614 10.7824897407259
615 10.7981331759062
616 10.8086184509172
617 10.8214637217565
618 10.8318974712553
619 10.842225719877
620 10.8337805612214
621 10.846461280845
622 10.837806963597
623 10.8317318408189
624 10.8233601301492
625 10.8387283091699
626 10.8513746765303
627 10.8486745367901
628 10.8544100612423
629 10.8546546183755
630 10.8463290851482
631 10.8436444664552
632 10.8350622493128
633 10.8451028579282
634 10.8411519445986
635 10.8449655576473
636 10.8526568632048
637 10.8500568864886
638 10.8416283218335
639 10.8434565797177
640 10.838565831204
641 10.8358729406876
642 10.8396717484119
643 10.8384611280046
644 10.8311747727596
645 10.8273850624991
646 10.8196516634721
647 10.8130623104312
648 10.8167220787061
649 10.8090432449099
650 10.8236503046412
651 10.8333280786888
652 10.839045819619
653 10.8308048016966
654 10.8385224505101
655 10.8320362151321
656 10.8307543853702
657 10.8310812429024
658 10.8245758118057
659 10.8163599406268
660 10.8284709367612
661 10.833982116873
662 10.8302486504991
663 10.8221498723165
664 10.8157461593749
665 10.8253960697906
666 10.8272128036561
667 10.82596186255
668 10.823388906775
669 10.8169784369957
670 10.8286440605367
671 10.820873962803
672 10.8280786535003
673 10.8419408717375
674 10.8516938165685
675 10.8572379504277
676 10.8492600716407
677 10.8437732660124
678 10.8470963257042
679 10.8435641319702
680 10.8531283859192
681 10.8672127309764
682 10.8744922444198
683 10.8682259385718
684 10.8820717655323
685 10.8837569319076
686 10.8758213228898
687 10.889614093969
688 10.8870959628634
689 10.8904181676994
690 10.8891264564357
691 10.8981288191694
692 10.909541366147
693 10.9112530973766
694 10.9044600718283
695 10.8998636024228
696 10.9133178543181
697 10.908740135844
698 10.9104132791502
699 10.9028652465479
700 10.9097667346319
701 10.9043061748316
702 10.8996943826534
703 10.8950847941298
704 10.896620680231
705 10.900013267004
706 10.9002337414891
707 10.8978272291669
708 10.8924857718236
709 10.8848015193723
710 10.8835640247019
711 10.8759077302075
712 10.8746672369412
713 10.8855294248881
714 10.888634362336
715 10.8812990079829
716 10.8769230797033
717 10.8715787123407
718 10.8681945651683
719 10.8616168914838
720 10.8723368386623
721 10.8678213633935
722 10.869639227868
723 10.8686048030268
724 10.8675552510677
725 10.8785275635106
726 10.8853524906503
727 10.8778634799902
728 10.8885472953297
729 10.8833820375946
730 10.8901491309957
731 10.8849746760439
732 10.8978021480004
733 10.9008675482111
734 10.9040810688307
735 10.9029427996247
736 10.897755478868
737 10.890417704826
738 10.88329439554
739 10.8764996786956
740 10.87669619359
741 10.8872281024526
742 10.8938647097922
743 10.8870808378747
744 10.8975325001251
745 10.9021985179853
746 10.8979522590111
747 10.9011068206613
748 10.9026892850775
749 10.9110249358475
750 10.9038047589923
751 10.8970769434211
752 10.9096665348771
753 10.9024782568269
754 10.899157523044
755 10.9038523411395
756 10.9121320199335
757 10.905476669696
758 10.8998061664191
759 10.8999519362847
760 10.8976603470094
761 10.89104539412
762 10.8841400638273
763 10.8808629219376
764 10.8739794043328
765 10.8728812339846
766 10.8792862718901
767 10.8722503378076
768 10.8723910772195
769 10.880556127616
770 10.8740082143914
771 10.8698308608694
772 10.8627887549662
773 10.8657182406456
774 10.858953232513
775 10.8521652270164
776 10.8586507904797
777 10.8531760143462
778 10.8521758795598
779 10.8584308514838
780 10.8544064500085
781 10.848957515476
782 10.8435159501754
783 10.855716475045
784 10.8493148729643
785 10.8426408668733
786 10.8395062790498
787 10.8363620453605
788 10.8365802761396
789 10.8367799903541
790 10.8413757714235
791 10.8347528361064
792 10.8300110720126
793 10.8241118825644
794 10.8256573569429
795 10.8217065521316
796 10.8297653986797
797 10.8257973932026
798 10.8376973283713
799 10.8392768429537
800 10.8511001026394
801 10.8452220352134
802 10.8512847582944
803 10.8593294068028
804 10.8607406735462
805 10.8706282355086
806 10.8735847520289
807 10.8688751504646
808 10.8733530201765
809 10.869398084407
810 10.862740081814
811 10.8655788569727
812 10.86579390572
813 10.8774862155967
814 10.8753829479582
815 10.8743437307795
816 10.8712941043453
817 10.8727422905408
818 10.8696796107419
819 10.8657754136463
820 10.8627103936801
821 10.8563257925893
822 10.8578296415461
823 10.8567971993788
824 10.8610966935895
825 10.8613205933781
826 10.8615274867041
827 10.8585400042444
828 10.86284628623
829 10.8598154558745
830 10.8713003160507
831 10.8714715102765
832 10.8651585211968
833 10.8727661319446
834 10.8697582379878
835 10.867681150992
836 10.8631355981981
837 10.8725959975015
838 10.8665929803429
839 10.8667134541527
840 10.8761258441912
841 10.8873813028423
842 10.8844043479779
843 10.8780003632577
844 10.8759700368515
845 10.8697506544511
846 10.8739700646235
847 10.8676051258542
848 10.8690207386373
849 10.8670020731098
850 10.871223816408
851 10.8661874257469
852 10.860294833903
853 10.8582977036875
854 10.8562946608726
855 10.8553061570023
856 10.8626696803551
857 10.8684801863703
858 10.864094001703
859 10.8642745049406
860 10.8734230625323
861 10.8735538744197
862 10.871542402707
863 10.874282804778
864 10.8744027411753
865 10.8817557639748
866 10.8890201648845
867 10.8846623740094
868 10.879227583013
869 10.8755754795827
870 10.8746229222876
871 10.8709684505954
872 10.8711086501126
873 10.8657172206766
874 10.8637434126001
875 10.8651372672753
876 10.8678235650518
877 10.8624646232849
878 10.8564855750232
879 10.8672743629227
880 10.8728274523167
881 10.8667039854574
882 10.8693067879344
883 10.8799199832618
884 10.88012601743
885 10.8772533348177
886 10.8743789929417
887 10.8684698696046
888 10.8623933041653
889 10.8564748202172
890 10.8505962374071
891 10.8487575911391
892 10.8557865542426
893 10.8498969114275
894 10.8538041181866
895 10.8490747886419
896 10.8560229993091
897 10.8549900850873
898 10.8521278089943
899 10.8468583464615
900 10.8496551688959
901 10.8553095741234
902 10.8495158323083
903 10.8580673618799
904 10.8687581837165
905 10.8640716523658
906 10.8743445474039
907 10.8725536800005
908 10.8739752943658
909 10.8679927563547
910 10.8644334755475
911 10.86177747766
912 10.8685882574079
913 10.8686115840164
914 10.8675775947667
915 10.8647520336346
916 10.8658858741467
917 10.8630508944289
918 10.8594788725221
919 10.8577737133521
920 10.8521033651553
921 10.85242691882
922 10.8504598152407
923 10.8450754013039
924 10.8440462683804
925 10.8545949041517
926 10.853840332327
927 10.8549845153341
928 10.8498737964209
929 10.8553961342815
930 10.8495591286967
931 10.8437684478246
932 10.8391231326766
933 10.833797876586
934 10.8296984039115
935 10.8239692582522
936 10.8343953355953
937 10.8297868052362
938 10.8401537011827
939 10.8413286089967
940 10.8378854059466
941 10.8332914249851
942 10.8313949214033
943 10.826807867107
944 10.8271391323448
945 10.8353016281441
946 10.8325793500701
947 10.82800170218
948 10.8383334325279
949 10.833344913913
950 10.831448058132
951 10.8341438905193
952 10.8324998853277
953 10.8328114469287
954 10.8364449327107
955 10.8345649420557
956 10.8395406190769
957 10.8343565852046
958 10.8383441327697
959 10.8356610462533
960 10.8307284027611
961 10.8310626529144
962 10.8324998330041
963 10.837463226999
964 10.8414063092008
965 10.8493740256958
966 10.844145026499
967 10.8441451711967
968 10.8480855521223
969 10.8494985849777
970 10.8463212012556
971 10.8418614406846
972 10.8453995075055
973 10.8517214981791
974 10.8510312009911
975 10.8503322738822
976 10.8551857068583
977 10.8518353724454
978 10.8532506826458
979 10.8488119158687
980 10.8514082599674
981 10.8459105511216
982 10.8481393604468
983 10.8426534007487
984 10.8440637209339
985 10.8462861032951
986 10.8423790720641
987 10.8379770423496
988 10.8370003882443
989 10.8322009290574
990 10.8321800615211
991 10.8315500956167
992 10.827867561109
993 10.8289087028099
994 10.8236827289945
995 10.8200185393505
996 10.8190555443424
997 10.817541218563
998 10.8121217761037
999 10.8155403401673
1000 10.8101611458849
};
\addplot [semithick, color5, forget plot]
table {%
1 0
2 7
3 6.18241233033047
4 12.2065556157337
5 12.2702893201424
6 11.9536140513607
7 11.4446315727827
8 10.7092249953019
9 10.4243305140746
10 9.93981891183134
11 9.49467010119775
12 11.7930652880788
13 11.3340874560566
14 11.7666512497249
15 12.7094015944453
16 12.6044077111937
17 13.1504469907058
18 13.1408560895184
19 12.8036525259607
20 12.9271033104868
21 12.9485288389442
22 12.9803570606316
23 13.0916501766518
24 13.1618530230359
25 12.8961389570677
26 13.0311983400421
27 12.8934644922068
28 12.7856644053297
29 12.8877897581923
30 13.0008546727599
31 12.8953201587986
32 12.7510722588338
33 12.5682708638415
34 12.6031041927819
35 12.4220344056644
36 12.2507243424044
37 12.1807065827931
38 12.0601275242129
39 12.0084344673377
40 11.8762104646221
41 11.871461087005
42 11.7356560780711
43 11.6046278603485
44 11.7315002950839
45 11.7554505351327
46 11.8208908021891
47 11.8284860756952
48 11.7082591931223
49 11.6097160188848
50 11.5976549353738
51 11.6672433536825
52 11.6456559186311
53 11.5398299523563
54 11.4381050403282
55 11.3439101512953
56 11.4060045053257
57 11.3224378674078
58 11.234356278791
59 11.3899818391884
60 11.3517497427587
61 11.3335493811447
62 11.5233554628775
63 11.4671526905734
64 11.4092461818255
65 11.5080008185713
66 11.4283812131619
67 11.3465823954236
68 11.3221160309672
69 11.3908249336739
70 11.4529739476216
71 11.3819375451323
72 11.5186292346114
73 11.452546361607
74 11.5558794195842
75 11.5291630225268
76 11.4918600000343
77 11.4606692115038
78 11.4846394284187
79 11.4477278392116
80 11.5506425687059
81 11.5456985499734
82 11.4800411339783
83 11.5610681095623
84 11.4927451729331
85 11.5412064519608
86 11.4865418464885
87 11.4249173140196
88 11.4194934014366
89 11.4628880875486
90 11.5591661026068
91 11.5516199173383
92 11.619844833338
93 11.5759867229143
94 11.640675201591
95 11.5883502491012
96 11.5469865898231
97 11.4883389657766
98 11.4693378239889
99 11.4784102846602
100 11.4350819848395
101 11.4974157771216
102 11.507068686585
103 11.5664200090524
104 11.5408330896156
105 11.4858643281679
106 11.4440650615723
107 11.4028343394148
108 11.3671540571079
109 11.4511363861833
110 11.3990031462086
111 11.3477328710003
112 11.4274553026419
113 11.3770621889836
114 11.3591444716192
115 11.3399972994866
116 11.3087677622119
117 11.2831649282012
118 11.3521496356499
119 11.315805507158
120 11.3013980206974
121 11.2866725017974
122 11.2616627116234
123 11.256284736264
124 11.3317758590923
125 11.3957915038842
126 11.4139263529085
127 11.3697406541225
128 11.3625420060761
129 11.332182679475
130 11.2994239786159
131 11.2856651287708
132 11.2428632602348
133 11.2428820579898
134 11.2673543776442
135 11.2334157596712
136 11.2042997184086
137 11.1647483792609
138 11.170124736119
139 11.1373877423901
140 11.1007997505604
141 11.0734818781351
142 11.0493765650239
143 11.0928798326698
144 11.0692963185405
145 11.076501892457
146 11.0691034688645
147 11.1104451649199
148 11.1164974567034
149 11.0802543876762
150 11.1500652713585
151 11.1198817286528
152 11.0836949916348
153 11.0474299687925
154 11.031320885506
155 11.0778796379846
156 11.0424644310072
157 11.0657787108281
158 11.037277878366
159 11.0029428874931
160 10.9832542991592
161 10.9504250583248
162 10.9193831983079
163 10.9146052850635
164 10.9042122200731
165 10.8887284709234
166 10.8601343019401
167 10.8746109043985
168 10.8465235634603
169 10.8186459749715
170 10.8142047429043
171 10.8793653115942
172 10.8569101796007
173 10.8862485717311
174 10.8552189067537
175 10.8308704683438
176 10.8023662084971
177 10.7926966391037
178 10.7666908736924
179 10.7376859502493
180 10.7445234399026
181 10.7186449769519
182 10.764065816831
183 10.7369776864094
184 10.709975165841
185 10.7237930339191
186 10.7369885387504
187 10.7723154775871
188 10.7497815468345
189 10.7410508986417
190 10.7165132627033
191 10.6907559005564
192 10.6631616555811
193 10.6707627463392
194 10.6586720979538
195 10.6601385348957
196 10.679012637511
197 10.7357057945416
198 10.7528039874486
199 10.7258596179432
200 10.7362889305383
201 10.7657185327937
202 10.7488198753571
203 10.7388145699529
204 10.7251182858414
205 10.7675609507482
206 10.7545922561088
207 10.7286713918392
208 10.7489506849926
209 10.7280921966747
210 10.7232015813466
211 10.7497575587618
212 10.7292015680075
213 10.76450369892
214 10.7435221602471
215 10.7853234411707
216 10.7608328995184
217 10.7513328090272
218 10.7279054600279
219 10.7046300098001
220 10.6864738403505
221 10.7058976526323
222 10.7471596540583
223 10.7603529570839
224 10.7366795364355
225 10.7571165074826
226 10.7333240263687
227 10.7326464119733
228 10.7133134296271
229 10.6899545480491
230 10.6751739271213
231 10.6790491461303
232 10.6673632623748
233 10.6557975869686
234 10.6354703516304
235 10.6212373839974
236 10.6525558566855
237 10.6654765532733
238 10.648653817243
239 10.6379121861246
240 10.6262834192184
241 10.629155338032
242 10.6107141592318
243 10.6054261151168
244 10.5937129261119
245 10.5729917684373
246 10.5566256837068
247 10.566744225572
248 10.5460285170196
249 10.5282103040476
250 10.507140429251
251 10.5221605030996
252 10.5342748545886
253 10.5260249094183
254 10.5462528154268
255 10.5684972162662
256 10.5633291788074
257 10.5458971044315
258 10.526089957695
259 10.5075359784687
260 10.5059711114243
261 10.4886315358366
262 10.5045774899855
263 10.5305320304653
264 10.5217802604162
265 10.5281896254243
266 10.5602353602204
267 10.5977288778917
268 10.5993815677646
269 10.6139681737596
270 10.628174503576
271 10.6429044508337
272 10.6440090489072
273 10.6271343574018
274 10.6158924980815
275 10.6525595418842
276 10.6442510958012
277 10.6257146282999
278 10.6269440264177
279 10.6184959825503
280 10.6233185944498
281 10.6122905864136
282 10.6368846579058
283 10.6599323168741
284 10.665466220864
285 10.6635059952533
286 10.6505935763131
287 10.6449648368724
288 10.6365587091333
289 10.6242855994645
290 10.637065297584
291 10.6204026465198
292 10.6083237879409
293 10.5925888108726
294 10.5788262404603
295 10.6077811791879
296 10.6024752646962
297 10.6041759856233
298 10.6369873649776
299 10.6386356479587
300 10.6362065083792
301 10.6226314901695
302 10.6057266237475
303 10.6378140887161
304 10.6203076448011
305 10.6536270292954
306 10.6385238643521
307 10.6264645970751
308 10.6396510111023
309 10.6344568591352
310 10.6185439466203
311 10.6021458309669
312 10.5874069060099
313 10.578152695845
314 10.56196735307
315 10.5570453656841
316 10.5556922368731
317 10.5461211314495
318 10.5551423497055
319 10.5437714616753
320 10.5609817695847
321 10.5446831013853
322 10.5656213293798
323 10.5608275985173
324 10.5470057786407
325 10.5428666817817
326 10.5301865217678
327 10.5178497085193
328 10.5346991169578
329 10.5199364330235
330 10.5185048360167
331 10.5393458304949
332 10.5476474575729
333 10.5681781349386
334 10.559345429396
335 10.5441295446687
336 10.5306581727095
337 10.5151540638282
338 10.5045835895995
339 10.4913817460206
340 10.4873770722079
341 10.5032918983652
342 10.4892251831936
343 10.478835775076
344 10.4805441834777
345 10.4742546513819
346 10.4625351286285
347 10.4920398769659
348 10.5075704547188
349 10.5194203631495
350 10.5391646650356
351 10.5631536222054
352 10.5675582507872
353 10.571826534659
354 10.5998461387851
355 10.6150675361892
356 10.6002687323163
357 10.5987239262951
358 10.5859955463427
359 10.5933173268132
360 10.5807593837022
361 10.5918879058455
362 10.6065725796319
363 10.5965819330909
364 10.5825127944092
365 10.5680065407022
366 10.5555769686923
367 10.5595537920689
368 10.5453494529611
369 10.5468120098894
370 10.5478616293787
371 10.5700752778658
372 10.5688870954491
373 10.5548647902672
374 10.576802353924
375 10.5756449763911
376 10.5972919809312
377 10.6145216165614
378 10.6004772731107
379 10.6190769639703
380 10.6254656081784
381 10.6258966490898
382 10.6237255897919
383 10.6484370303015
384 10.6438070759015
385 10.6299874645445
386 10.6170835142228
387 10.6139233683249
388 10.6004424615431
389 10.5868192703539
390 10.5745532698706
391 10.601346022555
392 10.6256363579432
393 10.6184593302235
394 10.6103808199117
395 10.6040276772692
396 10.597667133313
397 10.6098391033229
398 10.6281169440318
399 10.6154248349681
400 10.6388627211747
401 10.6264048177289
402 10.6133979821407
403 10.6070288233073
404 10.5961307774211
405 10.5987653601962
406 10.6207824762662
407 10.6233757071991
408 10.6258791503661
409 10.6230602501416
410 10.6408582532824
411 10.649156134466
412 10.6446700565441
413 10.6527486869765
414 10.6461384853985
415 10.6455704189889
416 10.6389329004363
417 10.6268049493702
418 10.6224524671663
419 10.6158988073816
420 10.6056492454902
421 10.6137569048958
422 10.6012116654693
423 10.6058370005374
424 10.6031105009894
425 10.5920908013127
426 10.5874144944157
427 10.5752131511991
428 10.5971549580369
429 10.5929281641691
430 10.5953090225276
431 10.592755524575
432 10.5805075352443
433 10.5695570063936
434 10.5587753072644
435 10.5801859898272
436 10.6013024991696
437 10.6034133108533
438 10.59263915628
439 10.5973848014356
440 10.6048676547709
441 10.604558528114
442 10.6287871112501
443 10.641955486033
444 10.6618422081347
445 10.650600617972
446 10.6389413919224
447 10.6412791131617
448 10.6481232383778
449 10.6440796718809
450 10.6328296641424
451 10.6243475415933
452 10.6133064253811
453 10.6230399772687
454 10.6279018957451
455 10.6194845676042
456 10.6354937497376
457 10.6512885178724
458 10.6410442358359
459 10.6299982387222
460 10.62550254566
461 10.6480639485014
462 10.6434591101144
463 10.6497174983042
464 10.640982593394
465 10.632926606674
466 10.6332299694562
467 10.6226064245544
468 10.6140140034138
469 10.6060289514448
470 10.59864717146
471 10.6189307220004
472 10.6076831355333
473 10.5964712138794
474 10.5947008242843
475 10.5849093438335
476 10.5951941251464
477 10.6025404374264
478 10.5916254273089
479 10.5808305484964
480 10.5709113451332
481 10.565664473291
482 10.5552475319418
483 10.5471304954005
484 10.5601667739463
485 10.5723245374917
486 10.563543332133
487 10.5533688629161
488 10.5515492773643
489 10.544989725478
490 10.5577267204028
491 10.5522888090221
492 10.5457298955396
493 10.55234020379
494 10.5418425272311
495 10.5568988560946
496 10.5514254958653
497 10.5491301793647
498 10.5385757885941
499 10.5368828167717
500 10.5290567478763
501 10.5503521759291
502 10.5486988402112
503 10.5463900803499
504 10.5640300075899
505 10.5540486240396
506 10.5440953918468
507 10.5417057128785
508 10.5344141957569
509 10.5261573801919
510 10.5229293223176
511 10.5143312723599
512 10.5317091120135
513 10.522143422602
514 10.5293611817378
515 10.5246329605226
516 10.5149109631488
517 10.5060063589436
518 10.4958682536878
519 10.4974327203415
520 10.4876087422337
521 10.4775457459238
522 10.4675116608452
523 10.4637385168075
524 10.4674212103305
525 10.4581444571936
526 10.4502590720402
527 10.4406187912902
528 10.4348567074077
529 10.4354195581058
530 10.4466407668105
531 10.4514215763035
532 10.4428394880962
533 10.4611686713414
534 10.4781504502666
535 10.4688183129518
536 10.4703698275188
537 10.4718732757328
538 10.4814396683395
539 10.4818953376825
540 10.4968116852478
541 10.511561202396
542 10.5137582733058
543 10.5052261935464
544 10.5020657196766
545 10.5080664083096
546 10.525475784445
547 10.52059951781
548 10.5348863173372
549 10.5288422939733
550 10.5239946222866
551 10.5241320721324
552 10.5146249392404
553 10.5127203997831
554 10.5186050226846
555 10.5300084441842
556 10.5463307301142
557 10.5443180727116
558 10.5349013840567
559 10.5281805469514
560 10.5190036969054
561 10.5210792761369
562 10.5119350676037
563 10.5139710043974
564 10.5121006330959
565 10.5183578782605
566 10.5181616896951
567 10.5267286153446
568 10.5176657533045
569 10.5085892083966
570 10.5251006608586
571 10.5206119607988
572 10.526457657828
573 10.5248286594558
574 10.5382526437912
575 10.5337343051787
576 10.5320105586357
577 10.5357562433657
578 10.5517391516078
579 10.5431456040527
580 10.5345538684885
581 10.5264681990301
582 10.5248297631909
583 10.5436791460519
584 10.5346672732052
585 10.5266611690766
586 10.5397260453739
587 10.5341359835008
588 10.527640578968
589 10.543287156646
590 10.5377934355935
591 10.5535215424432
592 10.5570724262944
593 10.5481897105113
594 10.553765038587
595 10.5693741574322
596 10.5614624188104
597 10.5742370131744
598 10.5725210610389
599 10.5660652352145
600 10.5786261185258
601 10.5699864316622
602 10.5668028117835
603 10.5636102558305
604 10.5565057426699
605 10.5743010728894
606 10.5672259205559
607 10.5642732448668
608 10.5557380747127
609 10.5515388770158
610 10.5438774295638
611 10.5495101363114
612 10.543190085089
613 10.5355631986549
614 10.5481079362767
615 10.5515145664988
616 10.5530337853011
617 10.5528378965459
618 10.5452766550364
619 10.5391305177898
620 10.5538513025007
621 10.5455432038056
622 10.5455974767279
623 10.5380247364757
624 10.5432592436649
625 10.5371967125987
626 10.544495143619
627 10.5480957316003
628 10.5652207457903
629 10.5578046600424
630 10.5494332406135
631 10.5643352755484
632 10.56435940761
633 10.5813541284862
634 10.5785128854907
635 10.5744601582499
636 10.5676982588196
637 10.5708653282311
638 10.5666476384346
639 10.5584028150207
640 10.5704994671196
641 10.5702663282838
642 10.5869591736623
643 10.5851832690803
644 10.5837583239978
645 10.5763885136064
646 10.579397666034
647 10.5939092113749
648 10.5872807437015
649 10.5791477334193
650 10.5711940848089
651 10.5709165262576
652 10.5872698931017
653 10.5791904225534
654 10.5753202842893
655 10.5674371449782
656 10.5722511345106
657 10.5818101618164
658 10.5834648528494
659 10.5759201891064
660 10.5871228112971
661 10.5811685547597
662 10.5812453721589
663 10.5747850870407
664 10.5748365404946
665 10.5691110733697
666 10.5704059159985
667 10.5674666656882
668 10.5657765295682
669 10.5631474502433
670 10.557493677745
671 10.5501060087809
672 10.5487609855276
673 10.546120790335
674 10.5572196366606
675 10.5711127461284
676 10.5663197357764
677 10.5730415769956
678 10.5743053873301
679 10.5679950073084
680 10.5653670532007
681 10.5666416881042
682 10.5610722111115
683 10.5676919147774
684 10.5603554353908
685 10.5619730726702
686 10.5617174290956
687 10.5580278748949
688 10.55519122711
689 10.55263647976
690 10.5576916000547
691 10.5626930009751
692 10.5558616398064
693 10.5490432725578
694 10.5522379979136
695 10.567775743622
696 10.5690213789127
697 10.577496858334
698 10.5700965888221
699 10.5639801123056
700 10.5703953163
701 10.5748817524307
702 10.5700529262385
703 10.5697077651653
704 10.5661608183351
705 10.5637072978978
706 10.5562524129797
707 10.548813288397
708 10.5577496539478
709 10.5561532588039
710 10.5488384413552
711 10.5443637368856
712 10.5419248849661
713 10.5569400360058
714 10.5618971993374
715 10.5548784661724
716 10.552438514387
717 10.5452520335352
718 10.5440174558843
719 10.5405000435142
720 10.5449914241692
721 10.544755304675
722 10.5595647801198
723 10.5536803205805
724 10.5552651564862
725 10.5481580631675
726 10.541065295086
727 10.534264841889
728 10.5284230361639
729 10.5282082005989
730 10.5411970052206
731 10.5423898874279
732 10.5387046341399
733 10.5430556376227
734 10.5387315359432
735 10.5403068455063
736 10.5525562527023
737 10.5510039232778
738 10.5443097362434
739 10.5491383427884
740 10.5467911993127
741 10.5398428603286
742 10.5353290373546
743 10.5474561469861
744 10.5490238400213
745 10.5453776967059
746 10.5395049164502
747 10.5349958811766
748 10.5477351444577
749 10.5462212253526
750 10.550970550407
751 10.5629247931383
752 10.5747834700817
753 10.5695293107538
754 10.5633744077184
755 10.5564838324919
756 10.5682373163023
757 10.5613577010461
758 10.5629547201459
759 10.5564489136819
760 10.5681287965461
761 10.561187768391
762 10.5542878941524
763 10.5520868471772
764 10.5517406801304
765 10.5621379342796
766 10.5563828662517
767 10.5531632976546
768 10.5669253055917
769 10.5733680066927
770 10.5671679247121
771 10.5645317192286
772 10.5596886959979
773 10.5733174824039
774 10.580708158394
775 10.5845692172657
776 10.5861700044672
777 10.5974083637025
778 10.5946936430123
779 10.6029689703951
780 10.5972528786147
781 10.5988292790011
782 10.5956770492769
783 10.5935156834965
784 10.5919199865982
785 10.5941895300802
786 10.5937346654194
787 10.6047672593069
788 10.6084530297059
789 10.6206575428006
790 10.6179448616218
791 10.6288121863077
792 10.6226985900076
793 10.6164731714065
794 10.6100661213511
795 10.6164070462681
796 10.6117313360467
797 10.6050742773248
798 10.5984297314227
799 10.609233801795
800 10.6161524927819
801 10.609797135181
802 10.6184951547054
803 10.6134262753593
804 10.6202172823059
805 10.6205487479568
806 10.6268700546989
807 10.620759941185
808 10.6147615691676
809 10.610936258994
810 10.608937287656
811 10.6175514452178
812 10.6169698604837
813 10.6149940352173
814 10.618032984738
815 10.6215045504168
816 10.6319156203285
817 10.6262739950681
818 10.6202527471632
819 10.6151318170907
820 10.6231023430464
821 10.6233719229325
822 10.6207217432396
823 10.6286113495808
824 10.6227346866795
825 10.6242715344724
826 10.6222628517699
827 10.6164193623986
828 10.6231166640079
829 10.6237881963273
830 10.6188947006035
831 10.6218455476839
832 10.6302317001334
833 10.6385557603612
834 10.6366063047467
835 10.6303083269337
836 10.6380848443873
837 10.6383230446151
838 10.6354089078219
839 10.6298859463451
840 10.624854274044
841 10.6362585061222
842 10.6352852273991
843 10.6343015059514
844 10.6280322901304
845 10.6217741484844
846 10.6184064758556
847 10.6150388218562
848 10.6090515972021
849 10.617382251296
850 10.6117011214999
851 10.6075872784295
852 10.612482837843
853 10.6075479391383
854 10.6109076400853
855 10.6222155255989
856 10.6286677453833
857 10.6252797007996
858 10.637137863151
859 10.6399934494379
860 10.6338068200023
861 10.6450079220119
862 10.6390878660382
863 10.6409760247477
864 10.635753655943
865 10.6404710944942
866 10.6447506732375
867 10.6428521528152
868 10.6526326843922
869 10.6507400134459
870 10.6479411661016
871 10.6521463932196
872 10.6548840915229
873 10.6497221422578
874 10.6436296835238
875 10.6417162341496
876 10.6418990616199
877 10.6363802602672
878 10.6354573076692
879 10.636829768188
880 10.6322425485676
881 10.6262390648028
882 10.6295295504123
883 10.6290535422017
884 10.628151883406
885 10.6241915010466
886 10.6331691866433
887 10.6303818963537
888 10.6256007626507
889 10.6311542484871
890 10.6418095899888
891 10.6443752382099
892 10.6559849890657
893 10.650410502836
894 10.6446101444579
895 10.6447021783973
896 10.6389155198028
897 10.6504696837062
898 10.6452684078655
899 10.6410634147204
900 10.6436006972485
901 10.6404580771443
902 10.6347126704727
903 10.6289765516644
904 10.6231814468725
905 10.6196577790999
906 10.615848365526
907 10.6221057826441
908 10.6197924464967
909 10.6163046535902
910 10.6131973379643
911 10.6077604827908
912 10.6155861470002
913 10.6225713047654
914 10.6232781796378
915 10.6218637146422
916 10.6288234275706
917 10.6278982308079
918 10.6291632125247
919 10.623643294032
920 10.6282775440744
921 10.6301723738613
922 10.6249423688608
923 10.621513732122
924 10.6174497115052
925 10.6170569381424
926 10.6120404013849
927 10.6106570624336
928 10.6079971997964
929 10.6133210451972
930 10.6087506447105
931 10.6050377341009
932 10.6041433699283
933 10.5998613847468
934 10.6011358697123
935 10.595547413047
936 10.6001488389374
937 10.5997855857397
938 10.6010724536267
939 10.5968155919122
940 10.5918808643439
941 10.587903454943
942 10.5852788846308
943 10.5805792752575
944 10.5866326444242
945 10.5836573321856
946 10.5784406495809
947 10.5803352019003
948 10.5790021694878
949 10.5815465449423
950 10.5773403335553
951 10.5769793609068
952 10.5718055114927
953 10.5827215202107
954 10.585271876384
955 10.5943774385516
956 10.5913940214425
957 10.5900262280037
958 10.6001195454122
959 10.5946626612758
960 10.5971874032394
961 10.6023928699662
962 10.5985113779358
963 10.5963286611505
964 10.6062711751379
965 10.6152880057658
966 10.6196545611951
967 10.6302946527682
968 10.6298493937433
969 10.6311336649604
970 10.6265059325525
971 10.6247886833404
972 10.6218314084312
973 10.6247865314328
974 10.626074653983
975 10.6206926404567
976 10.619887230824
977 10.6163033014014
978 10.6124959140379
979 10.6206393036547
980 10.6202316884427
981 10.618883730125
982 10.6147707816347
983 10.6245453636188
984 10.624707511342
985 10.6318725615124
986 10.6415631199626
987 10.6364062582248
988 10.6381232633199
989 10.6477667249884
990 10.6573517350939
991 10.6548183100679
992 10.6626986222752
993 10.6656367492609
994 10.6638732388624
995 10.6591699769902
996 10.6695174871979
997 10.6765652893074
998 10.6828658136984
999 10.6863666728641
1000 10.6838202905141
};
\addplot [semithick, color6, forget plot]
table {%
1 0
2 8.5
3 7.03957069398096
4 6.20483682299543
5 6.61513416341649
6 7.55902698029904
7 7.51596939312337
8 7.59831395771457
9 8.68516544260569
10 8.44097150806707
11 8.33195580901062
12 9.01349297195907
13 8.91312442313292
14 9.28461531958395
15 9.15204895091804
16 9.55820982192795
17 9.45614536273144
18 9.63148130345304
19 9.41943532133711
20 9.44139820153774
21 9.79448685685553
22 9.95319625559947
23 9.82550979451536
24 10.1526542780146
25 9.97003510525414
26 10.5256097941384
27 10.9037439612783
28 10.8937353275979
29 10.7047724029875
30 10.759440299363
31 10.9570156277712
32 11.0021304755034
33 10.8544930036125
34 11.0401908112491
35 11.0141282182551
36 11.0359960980667
37 10.911490649575
38 10.8030723033374
39 10.6638309840204
40 10.7749419952035
41 10.6931680502377
42 10.6123457390988
43 10.4909758022243
44 10.5385949476709
45 10.6612486239756
46 10.6273001513374
47 10.5638672836857
48 10.4690143001463
49 10.3616407243556
50 10.2723123005485
51 10.4060949815155
52 10.4464505169156
53 10.3518038698879
54 10.4574387252022
55 10.4054022841126
56 10.4141289425786
57 10.3502824687903
58 10.3093068327689
59 10.2466563937116
60 10.1661202038929
61 10.1543582598265
62 10.2559340453495
63 10.3912380479087
64 10.3181802538044
65 10.2385742350614
66 10.3541512504666
67 10.4567551905749
68 10.4022472285965
69 10.4939145806612
70 10.4944689027611
71 10.644868787434
72 10.6207689469591
73 10.7381708407119
74 10.7061146622587
75 10.8324615033806
76 10.7719369678244
77 10.8150398455665
78 10.8355103259454
79 10.90052029646
80 10.9777544948865
81 11.0861585385378
82 11.1823084117647
83 11.1894180979048
84 11.2074023460809
85 11.187919832257
86 11.1925475476199
87 11.1952134799626
88 11.1409950944346
89 11.092340609604
90 11.0991268703421
91 11.0444316774731
92 11.0389144541092
93 11.0330126133211
94 10.9939126742731
95 10.9774586466062
96 10.9745157369117
97 10.9705254443841
98 10.9282665404602
99 10.9067431068798
100 10.8576747050186
101 10.8290763782676
102 10.8244351709673
103 10.7918113229313
104 10.8344180916473
105 10.7959259973654
106 10.7450138286887
107 10.7988055873657
108 10.7611396329698
109 10.7208262969097
110 10.7108541533444
111 10.6634628132187
112 10.645789235368
113 10.5994132148002
114 10.5648599869356
115 10.6279227048748
116 10.6205881936488
117 10.6500185114717
118 10.6211056705523
119 10.6573574117094
120 10.6362554762264
121 10.6870529238923
122 10.643953130412
123 10.7310113102708
124 10.802659421287
125 10.7637138572149
126 10.7209767467161
127 10.6997647156342
128 10.6651327301794
129 10.6988259126364
130 10.673575178945
131 10.6347576939003
132 10.6776883269435
133 10.6392226141082
134 10.6909568446191
135 10.7000762788456
136 10.6611460266744
137 10.6223967892815
138 10.6795271401401
139 10.6949989691518
140 10.7458083902135
141 10.7841211393909
142 10.8206992489241
143 10.8145445718732
144 10.8598833314277
145 10.9357566217165
146 10.900772618741
147 10.9549369561521
148 10.9872988463335
149 10.9874524584413
150 10.9656858730618
151 10.9724247785842
152 11.0115602029183
153 11.0627204617443
154 11.0312119478454
155 11.0438422486426
156 11.0361365519563
157 11.0812210600159
158 11.0626611065282
159 11.0294742325841
160 11.0533860762212
161 11.0301605590425
162 11.0038183920124
163 10.9729307424338
164 10.9521480268006
165 10.9425480217565
166 11.010303109731
167 11.0236479741558
168 11.0435131943358
169 11.011785036248
170 10.9796619154939
171 10.9568235272075
172 10.9423161188273
173 10.9199212027289
174 10.8907060317621
175 10.8604746767387
176 10.839360476393
177 10.8089122130581
178 10.7788186480732
179 10.7928565555854
180 10.8062339118689
181 10.8257973254798
182 10.8127400961956
183 10.7840778980108
184 10.7902705509791
185 10.7612431225548
186 10.7326156735105
187 10.7386469947403
188 10.7104190947062
189 10.6821976878675
190 10.6559452408066
191 10.6743257248973
192 10.6895049695042
193 10.6803112704418
194 10.6527686775503
195 10.6742872186653
196 10.6950495443617
197 10.7026866658855
198 10.6816880379874
199 10.6568567951628
200 10.6581236622588
201 10.7077460443764
202 10.699746952364
203 10.6755303834263
204 10.7082343175313
205 10.6903148526051
206 10.7026021292345
207 10.7041912419822
208 10.680445091824
209 10.6730057360207
210 10.6477734463633
211 10.6721292156679
212 10.6545901807077
213 10.646821203386
214 10.6578736102736
215 10.6975187046301
216 10.6748738680495
217 10.6539373722977
218 10.6331185532512
219 10.6297858844558
220 10.681159554163
221 10.6651708495081
222 10.6492288450344
223 10.6564220814093
224 10.6936967355751
225 10.6801914256669
226 10.6968941896719
227 10.7391177424624
228 10.756284533768
229 10.7674344302477
230 10.75116734331
231 10.7524206554716
232 10.7310066427704
233 10.7240122717664
234 10.7579475554229
235 10.7514964690562
236 10.7295590752307
237 10.7395562776887
238 10.7221333922556
239 10.7026066340963
240 10.6869590506374
241 10.6888756085847
242 10.6950483152696
243 10.675990120805
244 10.6660920688648
245 10.647184241771
246 10.6535184431461
247 10.6588341743136
248 10.6925684061739
249 10.6864504199035
250 10.66505621176
251 10.6557849660179
252 10.6821357738684
253 10.6924294535107
254 10.6897333007328
255 10.7043926278038
256 10.7058390719446
257 10.7315922831005
258 10.7109491000172
259 10.6918914457265
260 10.7115840352573
261 10.6925812059835
262 10.6833336052981
263 10.674670472016
264 10.668527488127
265 10.6490098015901
266 10.6291938280219
267 10.6122443881718
268 10.5930621175191
269 10.5821903753667
270 10.6133983702538
271 10.649777646474
272 10.6316505493728
273 10.6123658104513
274 10.6479088868889
275 10.6626193423412
276 10.6723577632519
277 10.6612176659719
278 10.6846231464661
279 10.6758051865942
280 10.70970378218
281 10.7176477268427
282 10.7073621046117
283 10.6959338844559
284 10.7345613501199
285 10.7446972756088
286 10.7545513430568
287 10.7685619942568
288 10.7952047816151
289 10.8325909642706
290 10.8342330794998
291 10.8305509969728
292 10.8128337762516
293 10.7952029599763
294 10.7983469485273
295 10.7909140709507
296 10.772685624753
297 10.768041046432
298 10.7733758717697
299 10.7961109515289
300 10.7875215977639
301 10.7703357630516
302 10.7566371692592
303 10.7789031505426
304 10.7738473973963
305 10.7710532079152
306 10.7964503899515
307 10.7889404033189
308 10.7831455067496
309 10.8131089451386
310 10.7983387938884
311 10.7899293916176
312 10.7726337724714
313 10.75652568339
314 10.7442435906634
315 10.7359550419437
316 10.7190247872202
317 10.7025239893893
318 10.6872951007885
319 10.6872665769192
320 10.6715885759513
321 10.6549707327604
322 10.6655064780134
323 10.6938381613278
324 10.6835161415508
325 10.6801369581006
326 10.6795173478248
327 10.6692878270919
328 10.696512410657
329 10.6861018867094
330 10.6728842932369
331 10.6692859854178
332 10.6755839300954
333 10.6598883110655
334 10.6448406385718
335 10.6306005511224
336 10.6227701241505
337 10.6129694477762
338 10.6399187437155
339 10.6250608024644
340 10.6111500408915
341 10.5997171907332
342 10.5842492831921
343 10.6016366419274
344 10.5904825374427
345 10.5754035308526
346 10.5749403225434
347 10.5805879342066
348 10.5669085864746
349 10.5836250765473
350 10.5868998757706
351 10.571866087029
352 10.5577844748095
353 10.544324907746
354 10.5475826712795
355 10.5477570851756
356 10.561216141648
357 10.5503419870296
358 10.5371529520464
359 10.5252332031627
360 10.5109275882499
361 10.5281752257671
362 10.5535871254687
363 10.5430779927259
364 10.5324482379816
365 10.5414247770175
366 10.5576514010218
367 10.5598610459824
368 10.5839973879855
369 10.5696560206942
370 10.5555710575326
371 10.5480780033353
372 10.5374022769034
373 10.5242358253582
374 10.5151055844288
375 10.5430760006535
376 10.5338583528161
377 10.5450767415817
378 10.5495268617509
379 10.5712294862484
380 10.596888099632
381 10.604457365662
382 10.6025156661121
383 10.5933576220579
384 10.5870940968205
385 10.5815344656544
386 10.5958831092775
387 10.6023616006117
388 10.5904843618204
389 10.577271317587
390 10.5778914863584
391 10.5724662342667
392 10.5798711461618
393 10.587127327651
394 10.5738306937313
395 10.5945038702929
396 10.6148616682411
397 10.6076118713776
398 10.6251742180284
399 10.6123870975048
400 10.6251011759889
401 10.6118478412693
402 10.6107285194688
403 10.6147511983055
404 10.6071340042973
405 10.598040557048
406 10.5851317629302
407 10.5797938079167
408 10.5867098460874
409 10.574182335199
410 10.5670685966804
411 10.5802980556244
412 10.5692496572143
413 10.5605009743878
414 10.5549367487947
415 10.5449720196979
416 10.5397412456449
417 10.532548276877
418 10.5216771532387
419 10.5092060989814
420 10.5154984958765
421 10.5118721236315
422 10.5134019189654
423 10.5026529519078
424 10.5184509499383
425 10.5306028902354
426 10.5225240151269
427 10.5217361151903
428 10.5136562762045
429 10.5257710483596
430 10.544663051756
431 10.5488412078651
432 10.5369800818613
433 10.5256615146168
434 10.5150983250753
435 10.503071668179
436 10.4913652216215
437 10.511624410979
438 10.5354303685518
439 10.5318901063276
440 10.5224354465719
441 10.5446394961104
442 10.5502043534634
443 10.5391013703765
444 10.5301683393983
445 10.5184935819696
446 10.5115646797578
447 10.5244947891094
448 10.5138383284452
449 10.5045510862598
450 10.5113234650854
451 10.5085598789017
452 10.500932043034
453 10.5124412038509
454 10.5190473466222
455 10.5144290482347
456 10.5030297992819
457 10.491534275985
458 10.4953822316049
459 10.485509012467
460 10.4809341709796
461 10.4822989282701
462 10.4726651880504
463 10.46955785163
464 10.4876259029091
465 10.4989731864458
466 10.4892080454446
467 10.4797422702686
468 10.4753161537354
469 10.4929981020587
470 10.5071552933481
471 10.4959997518823
472 10.4859141370083
473 10.5045852765647
474 10.5165490089966
475 10.5347894314813
476 10.5261099264785
477 10.5216101170157
478 10.5395722618698
479 10.5320592701959
480 10.5247210226289
481 10.5201765832877
482 10.5154981011399
483 10.5082129918771
484 10.4977618760113
485 10.4873418176106
486 10.4984706905217
487 10.4900646170961
488 10.5014643979532
489 10.4968745479044
490 10.4895823054562
491 10.4790067597738
492 10.4744406170293
493 10.4654279494397
494 10.4663466157143
495 10.4873113629036
496 10.5046597159508
497 10.4988138859971
498 10.488267635978
499 10.4838839962723
500 10.492054898827
501 10.4876159201648
502 10.4786496448441
503 10.495747407903
504 10.4877071137549
505 10.4985393039656
506 10.4957368991255
507 10.4989784200376
508 10.5189932407259
509 10.5087521634254
510 10.5043812832451
511 10.5000025075035
512 10.4905527393624
513 10.5013384986747
514 10.4919370511207
515 10.4928625270556
516 10.4983959789065
517 10.4882379446343
518 10.5079459832826
519 10.5274270138778
520 10.5379265375148
521 10.5455918646215
522 10.5507154558128
523 10.5669537387508
524 10.5601484313939
525 10.5545268576194
526 10.5574544625175
527 10.5474331668719
528 10.5382394267091
529 10.5290694452545
530 10.5450500789553
531 10.5526009260844
532 10.565491513242
533 10.568361227671
534 10.5757632388796
535 10.5808750387953
536 10.5717850211344
537 10.5627181594
538 10.55325777693
539 10.5435553485593
540 10.5486520634938
541 10.5410896892872
542 10.5345139481283
543 10.5395699044849
544 10.5494155036496
545 10.5593835406806
546 10.5601532057245
547 10.5574388428732
548 10.5519657615133
549 10.5590767845079
550 10.5495440346416
551 10.5674831765231
552 10.558613663262
553 10.5611056619974
554 10.5616092319533
555 10.5573092148966
556 10.549047962664
557 10.560792209304
558 10.5698203081991
559 10.5642143787234
560 10.5696172133013
561 10.5749452956029
562 10.5656566534209
563 10.5574742564465
564 10.5509146086737
565 10.5424469364091
566 10.5331338671673
567 10.5486231082728
568 10.5449273978359
569 10.5591920157288
570 10.5531231747361
571 10.5481708616633
572 10.549297564488
573 10.5422921451703
574 10.5496338384888
575 10.5405184908861
576 10.5321719645795
577 10.5366556653236
578 10.5283503667207
579 10.5272130436601
580 10.5265773995121
581 10.5183214066802
582 10.5324595096245
583 10.5276004432035
584 10.5186915044349
585 10.5108991364898
586 10.5060829803379
587 10.4985230684161
588 10.5051272329902
589 10.4965835093507
590 10.4996047859303
591 10.4919182190254
592 10.4949110898465
593 10.4861599536999
594 10.4811218451737
595 10.493144297799
596 10.4843385530001
597 10.4774644603055
598 10.4716820621678
599 10.4669576772732
600 10.4589928078929
601 10.4509616328593
602 10.4429487574529
603 10.4394873659083
604 10.4309350256116
605 10.433547789353
606 10.4507966503963
607 10.4574495167597
608 10.4691824289488
609 10.4609189615853
610 10.4526749879985
611 10.4492011773796
612 10.4426521965912
613 10.4472296547126
614 10.4388065105796
615 10.443581559798
616 10.4481385899969
617 10.4572306035062
618 10.4488385267533
619 10.4453834953971
620 10.4542434961102
621 10.4459074213867
622 10.4375912534155
623 10.453950349
624 10.4678730301329
625 10.4597867970624
626 10.4526801876001
627 10.4688929611694
628 10.4632248229059
629 10.4561684942804
630 10.4569154089045
631 10.4487164363178
632 10.4646859197022
633 10.4611005090478
634 10.4548302713315
635 10.4686363345325
636 10.4616619280998
637 10.4545933476874
638 10.4481982678145
639 10.4407403496654
640 10.4514688290017
641 10.4470966965049
642 10.4537819867499
643 10.4625618884807
644 10.4545001892124
645 10.4539008521346
646 10.4604060210546
647 10.46119750779
648 10.4567828355682
649 10.4522923268083
650 10.449010711645
651 10.4457235443701
652 10.4482277463517
653 10.4403039257091
654 10.4447535657505
655 10.4394236036696
656 10.4314636995952
657 10.444595275126
658 10.4552144204737
659 10.447948682256
660 10.4635704348115
661 10.4563058845478
662 10.4484053885124
663 10.4440592921841
664 10.4397138466336
665 10.432144169569
666 10.4301597267549
667 10.4229876555939
668 10.4154692679549
669 10.428398288243
670 10.4389473483698
671 10.4414244849583
672 10.4407390854057
673 10.4340958958669
674 10.4494558853969
675 10.4646797411602
676 10.4586949601563
677 10.4521068151999
678 10.4624028697505
679 10.4704338585227
680 10.4644463149371
681 10.4568398229342
682 10.4517458916073
683 10.462115600547
684 10.4555993480605
685 10.456440577707
686 10.457254624476
687 10.4507496855447
688 10.4465435968365
689 10.4507266667431
690 10.4548462506114
691 10.4529079099857
692 10.4522436102664
693 10.4491404202546
694 10.4551301457454
695 10.4559116633493
696 10.4583174436497
697 10.4606684753944
698 10.4542798913287
699 10.4479025309886
700 10.4410566163251
701 10.448940413187
702 10.4483438040515
703 10.4426200528288
704 10.439557310575
705 10.4364992730611
706 10.4464660256368
707 10.4434211656805
708 10.4512280232198
709 10.4455578073678
710 10.4449588436236
711 10.4392900310091
712 10.434385300724
713 10.4281384840632
714 10.4251181730397
715 10.4291816654878
716 10.4219632716564
717 10.4259870244165
718 10.4241073380732
719 10.4265002134413
720 10.4273277850873
721 10.4313174152221
722 10.4306974384613
723 10.4301489340975
724 10.4246101800236
725 10.4227571376092
726 10.4304721256942
727 10.4265204162065
728 10.4362238016562
729 10.4322886070412
730 10.4463735255286
731 10.4559086126334
732 10.4553530472533
733 10.454781019685
734 10.4492711921959
735 10.4631543593939
736 10.4727481749544
737 10.4843985713979
738 10.4899545600519
739 10.5014924769875
740 10.4996138055522
741 10.505109092451
742 10.4980278110105
743 10.4912215529245
744 10.4882496198028
745 10.495635303983
746 10.4886628399964
747 10.4832316111015
748 10.4886845663341
749 10.4980099796388
750 10.4920283813739
751 10.4856124030962
752 10.4991463651743
753 10.492418963266
754 10.5059657211652
755 10.5151279327598
756 10.517244511409
757 10.520894545551
758 10.5140173456767
759 10.5133269481257
760 10.50949840795
761 10.5026562147038
762 10.5097641700604
763 10.5068228142205
764 10.4999445241394
765 10.4931357435059
766 10.4902115376395
767 10.4922186066026
768 10.4944258330313
769 10.4898746178554
770 10.4952712448167
771 10.4894334805521
772 10.4865463891574
773 10.4798167467599
774 10.476107492548
775 10.4796743163762
776 10.4907861044377
777 10.491498244171
778 10.4848175542002
779 10.4883849610884
780 10.493558338377
781 10.4897773329375
782 10.4840111923965
783 10.4834853885325
784 10.4828055023959
785 10.4879023442236
786 10.4985513559501
787 10.499049661881
788 10.5023954646066
789 10.5155610744654
790 10.5160257982742
791 10.5124301191795
792 10.5177525552946
793 10.5172327690872
794 10.5121583887149
795 10.5128884867329
796 10.5164788812557
797 10.5127551872726
798 10.5062334539339
799 10.5034074968632
800 10.5028709616704
801 10.4984091972255
802 10.5109321111492
803 10.5046435194986
804 10.5079711993452
805 10.5207816704875
806 10.5273634246527
807 10.529444019218
808 10.5397213857635
809 10.5336968276358
810 10.540550738768
811 10.5511069093399
812 10.5636304472408
813 10.5760617608097
814 10.5705029521985
815 10.5807293031648
816 10.5840796555232
817 10.5812764887388
818 10.5753149250824
819 10.5725146962069
820 10.5717750225168
821 10.5710232115518
822 10.5651333988938
823 10.5608278367974
824 10.5657622410713
825 10.5721744385894
826 10.5694901276194
827 10.5814996158004
828 10.5751569487701
829 10.5785145812322
830 10.5778958119463
831 10.5743510246837
832 10.5825923808473
833 10.5873715120249
834 10.581078719257
835 10.5828787990793
836 10.5876383638846
837 10.5995656514347
838 10.6042939982621
839 10.6047722181448
840 10.6040754343001
841 10.598667204195
842 10.5943880131822
843 10.5961642757999
844 10.5966540504423
845 10.5971266468325
846 10.5988613007337
847 10.5982032067251
848 10.5946921827626
849 10.6009235679953
850 10.5946859435386
851 10.5992712368905
852 10.5974848516764
853 10.602187656263
854 10.596879160122
855 10.6030245921479
856 10.5973424664328
857 10.5920167049735
858 10.5903485786527
859 10.5886735954442
860 10.5834079353595
861 10.5827120749265
862 10.583150532238
863 10.5825443044117
864 10.5764686835831
865 10.5881177503561
866 10.5863958083415
867 10.5895300914802
868 10.5843052224601
869 10.5847716414723
870 10.5852795779904
871 10.5836018369805
872 10.5897056221061
873 10.5862731822627
874 10.5802664292033
875 10.5848347427162
876 10.5831669813148
877 10.5848831575563
878 10.5789087298422
879 10.5737541346954
880 10.5742489305016
881 10.5682989833845
882 10.572818840244
883 10.5711738038748
884 10.5656745664555
885 10.5661654360552
886 10.5610486802565
887 10.5576930397938
888 10.563776112144
889 10.5604476494282
890 10.5649256664008
891 10.5709043875008
892 10.5725619548569
893 10.5700684666771
894 10.5675060059099
895 10.5680479672936
896 10.568574321241
897 10.5679763372128
898 10.5755828873859
899 10.5730580865103
900 10.573551676789
901 10.5681537533656
902 10.5665446118908
903 10.5640398269667
904 10.5582489953512
905 10.5576553825566
906 10.5570491837397
907 10.5512787457073
908 10.5506955109653
909 10.5448904513503
910 10.5453969432744
911 10.5513204277696
912 10.5605432540562
913 10.5566164351946
914 10.5676418139317
915 10.5643984389957
916 10.5648844021923
917 10.5624275047559
918 10.5592003977618
919 10.5596882002749
920 10.5552296680809
921 10.5610375475847
922 10.5684869413486
923 10.562760322947
924 10.564428128672
925 10.5628584793573
926 10.5686293978723
927 10.5647787682696
928 10.5592926441872
929 10.5577483802036
930 10.5571524310718
931 10.554748640205
932 10.5491346333722
933 10.5439337928315
934 10.5468431415155
935 10.5414097741388
936 10.5362405032275
937 10.5451986957414
938 10.5395761675104
939 10.5347545817457
940 10.54379858243
941 10.5511220443384
942 10.5495843291567
943 10.555312863585
944 10.561002224294
945 10.5666526648733
946 10.5642891002408
947 10.574843891084
948 10.5804152387463
949 10.5908527689602
950 10.5892330201804
951 10.5849468126091
952 10.5824972259505
953 10.5794321797593
954 10.5799626325874
955 10.5756922893713
956 10.5709754830109
957 10.5655064191877
958 10.5618021004235
959 10.5586761109952
960 10.5536019699404
961 10.5520457473509
962 10.5524387262234
963 10.5482284370529
964 10.5427978276651
965 10.5444967375066
966 10.5390931868335
967 10.5341028012413
968 10.531019705283
969 10.5260052767897
970 10.5213893250745
971 10.5268725254314
972 10.5214971982214
973 10.5165531524992
974 10.5142128589821
975 10.5106264126889
976 10.5194459568641
977 10.5158492514801
978 10.5199122518478
979 10.5162427906207
980 10.5116826283173
981 10.5186076365534
982 10.5136600532
983 10.5132363267557
984 10.5109200035883
985 10.5095720974436
986 10.5110889091237
987 10.5179425273085
988 10.5235929598968
989 10.5184820661154
990 10.5144131257644
991 10.5098408449137
992 10.5092469235945
993 10.5039925588485
994 10.5123647385115
995 10.50708149999
996 10.5174724596686
997 10.5123683698519
998 10.5119615838346
999 10.5190168162165
1000 10.5216871270724
};
\addplot [semithick, white!49.8039215686275!black, forget plot]
table {%
1 0
2 8
3 6.54896090146283
4 12.4774797134678
5 11.3207773584679
6 10.3387082795139
7 10.648368659191
8 10.0249688278817
9 11.1797877583158
10 10.6113147159058
11 10.4430767132905
12 10.6415720434321
13 10.2772224737595
14 10.0843382306932
15 9.95188424369978
16 9.6369649656933
17 9.60932361542792
18 9.34209245073205
19 9.54882481783293
20 9.30738953735149
21 9.09685849439905
22 9.40458829464297
23 9.19973698974755
24 9.38046137221169
25 9.67470929795826
26 9.50210191633422
27 9.5024363347668
28 9.6227303181327
29 9.67424225141794
30 9.89388138643722
31 9.98698424542756
32 9.88982273602515
33 9.79411565997999
34 9.65566330761562
35 9.7960425162201
36 9.68676072448026
37 9.59904731089729
38 9.681385457502
39 9.5793169698819
40 9.82738520665594
41 9.99142999818771
42 10.2072625427575
43 10.1915138772839
44 10.156319532502
45 10.1634542587221
46 10.2393150541835
47 10.4918429588709
48 10.489081624877
49 10.4319855428236
50 10.4887749523002
51 10.5369126530669
52 10.4846323155798
53 10.4051771214697
54 10.541657517197
55 10.4755362394549
56 10.5999705188269
57 10.5280992641175
58 10.4892221452709
59 10.4191391301097
60 10.3324730824716
61 10.2485635886307
62 10.2405637892728
63 10.2122844328961
64 10.2202311617693
65 10.2596584600062
66 10.2631693266797
67 10.1877234049389
68 10.1128045011704
69 10.2065005170251
70 10.2002000780697
71 10.3806925597589
72 10.3252283028596
73 10.2600819339345
74 10.1933194725751
75 10.3807171664047
76 10.3596676916511
77 10.3264149477835
78 10.3689340290798
79 10.3031132171023
80 10.2545035837919
81 10.191715078505
82 10.1590928902226
83 10.2472553317856
84 10.2862654173333
85 10.335654016299
86 10.2882009847414
87 10.4330465745657
88 10.5297627668253
89 10.5249296757129
90 10.5084093309438
91 10.6339904744843
92 10.5905270293751
93 10.6574383424066
94 10.7523882194567
95 10.8140669648995
96 10.7586367114467
97 10.7550431585626
98 10.7063138467412
99 10.6850736496325
100 10.6445995697349
101 10.6451161659445
102 10.7189866291556
103 10.6803530192807
104 10.7349686778673
105 10.6839570747692
106 10.6595591604528
107 10.6791313609375
108 10.6476610194533
109 10.7431701908084
110 10.7134116920169
111 10.7757991618862
112 10.7336580817691
113 10.6902508351303
114 10.745155736736
115 10.8325142436703
116 10.8976565349994
117 10.9671342408817
118 10.9236992281095
119 10.8922734704204
120 10.9453059446606
121 10.9736008981719
122 10.9347826028882
123 10.9959041521592
124 10.9797017334756
125 11.009377820749
126 10.9956844646231
127 10.975609729214
128 10.9341038254099
129 10.9088911450827
130 10.8889608069651
131 10.9167458261349
132 10.8756721924762
133 10.8516064862833
134 10.819446634857
135 10.8276331949123
136 10.7936342855391
137 10.8415712178313
138 10.8670842579254
139 10.9030954693199
140 10.8904072186377
141 10.9672682005185
142 10.9619004400554
143 10.9868797315708
144 10.9743822592537
145 10.9364742257225
146 10.9439559867659
147 10.9516372756349
148 11.0032782816647
149 10.9688578087582
150 10.9991434009906
151 11.0571794149008
152 11.0208797522006
153 10.9932153570527
154 10.9886554297214
155 10.9628755807479
156 10.928152411164
157 10.9332392972204
158 10.9313380506813
159 10.8969556055753
160 10.866158402927
161 10.8707541497056
162 10.8968784551844
163 10.9585836978955
164 10.9362777637819
165 10.9952590167222
166 10.9851524255845
167 10.9786071348898
168 10.9486471378732
169 10.9287361851785
170 10.9669036550225
171 10.9934645797256
172 10.9614719337829
173 10.9456907477511
174 10.9228307201934
175 10.9446906334179
176 10.9235167402813
177 10.9742850560145
178 10.9435505700115
179 10.9577043064595
180 10.9442131955265
181 10.9692650807142
182 10.9541273247211
183 10.9296966965997
184 10.931604203553
185 10.9075938056776
186 10.9179888283281
187 10.8900930311034
188 10.9109670573996
189 10.9310581486852
190 10.9057673628614
191 10.9001740115256
192 10.926998927224
193 10.9602847754146
194 11.0084268053655
195 10.9810834897477
196 10.9539806369687
197 10.949183275184
198 10.9236540000396
199 10.8970495199479
200 10.9080520717496
201 10.9041561467062
202 10.8771332727206
203 10.9148667110263
204 10.951476282065
205 10.9417863658394
206 10.9153116609154
207 10.9433203533541
208 10.9560113848963
209 10.9309149618149
210 10.96956829467
211 10.9999560982603
212 10.9741465737627
213 10.9504423077534
214 10.9799521930465
215 10.9716326293187
216 10.9816961755292
217 11.0171543084815
218 11.002069572932
219 11.0117580092104
220 11.0324614937724
221 11.0081765642839
222 10.9910832960236
223 10.9665673549
224 10.9427711423677
225 10.9630886879678
226 11.0053347830876
227 10.9813194496723
228 10.9703381312269
229 10.9480485116508
230 10.9315830741305
231 10.9320791253933
232 10.9102380220589
233 10.8916384588397
234 10.9251399145985
235 10.9066942783542
236 10.8958426653785
237 10.9289998615923
238 10.9679898374028
239 10.9720516592608
240 10.9500253678042
241 10.9690866640804
242 10.9551432167137
243 10.9439568191052
244 10.9300466503867
245 10.9428061413605
246 10.9442792815243
247 10.9415784942834
248 10.9678575658041
249 10.9758499786351
250 10.9557626845419
251 10.9344993837571
252 10.9423428521951
253 10.9626228528618
254 10.9750702459863
255 10.9802969183545
256 10.9619787459812
257 10.9696355618081
258 10.9947905184482
259 10.9995209215208
260 10.9809537907163
261 10.9982282770685
262 10.9877322716897
263 10.9994880666934
264 11.008943662529
265 11.0253638128368
266 11.0465151097984
267 11.0656240430123
268 11.0577021980851
269 11.0389882328155
270 11.0363316950856
271 11.0258969455148
272 11.039617214753
273 11.023109489959
274 11.0297169779932
275 11.0307003591442
276 11.0261880129172
277 11.0516836950134
278 11.0766137461857
279 11.0711026718933
280 11.0693984313807
281 11.0497355312627
282 11.030157092017
283 11.019795939962
284 11.0180709438605
285 10.9997173919319
286 10.9894120351848
287 10.9991286094258
288 10.9832486789296
289 10.9927925918234
290 10.9827694472961
291 10.9639163140573
292 10.9736560599556
293 10.9596262904546
294 10.9780854617454
295 10.9707537351795
296 10.9803180715551
297 10.9938914234205
298 11.0029074495602
299 11.0125506617447
300 11.0052512718045
301 10.9869877311793
302 10.9771628007136
303 10.960935149815
304 10.9437932033632
305 10.9662029969286
306 10.9502079879267
307 10.9327384261024
308 10.9210601275392
309 10.9260456343589
310 10.909446593388
311 10.9325818039128
312 10.9420499834463
313 10.9688602801113
314 10.9516625372178
315 10.950537687081
316 10.9674701265792
317 10.9509685892718
318 10.9687053715463
319 10.954494209545
320 10.938303095516
321 10.9242071670767
322 10.9295439230254
323 10.9253504959364
324 10.9126877148862
325 10.9084426344946
326 10.9250217892351
327 10.9085668019234
328 10.896351699942
329 10.9055666792462
330 10.89065824908
331 10.8758098307161
332 10.8597700663252
333 10.8461359798188
334 10.8517570665652
335 10.8358307635332
336 10.8205035882113
337 10.8227732905935
338 10.8108148470383
339 10.8198748307543
340 10.832372905701
341 10.8172815057474
342 10.8472032810876
343 10.8323025673283
344 10.8442358571289
345 10.8498658204506
346 10.8753892043988
347 10.8801316241117
348 10.8645085595828
349 10.8565478699848
350 10.8467228192836
351 10.8549034984188
352 10.8570547382977
353 10.8732061897574
354 10.8586310303861
355 10.8603419091507
356 10.862340300945
357 10.8586481021003
358 10.8740473130069
359 10.8899870182558
360 10.8821280031317
361 10.870943999444
362 10.8574936491073
363 10.8433286522161
364 10.8287109788558
365 10.8229845147939
366 10.8310491048925
367 10.8171078680221
368 10.8062719040637
369 10.834203676369
370 10.8234456572209
371 10.825066308244
372 10.8128999644994
373 10.8144219015453
374 10.8290090438903
375 10.8200623534864
376 10.8308318765282
377 10.8394199080503
378 10.8322466523479
379 10.8335944423361
380 10.833165386865
381 10.8318782198787
382 10.8225065543153
383 10.8343659153429
384 10.823670758438
385 10.841495590954
386 10.8356268709367
387 10.8219781351609
388 10.8289326695287
389 10.8162429526346
390 10.8027405999933
391 10.8038119513223
392 10.796132658168
393 10.7841046704989
394 10.7817130426186
395 10.8050787135239
396 10.7984153302948
397 10.7894002022023
398 10.7770763714836
399 10.7942089341533
400 10.7812754347526
401 10.7678291504615
402 10.7588752754709
403 10.7458941442559
404 10.7436316037572
405 10.7553923056685
406 10.7801995492805
407 10.7812523094819
408 10.7723764146004
409 10.7808519205424
410 10.7693232111924
411 10.7565734975539
412 10.7530843670389
413 10.745261852898
414 10.7430240181258
415 10.7578701185499
416 10.7474235658768
417 10.7586271275183
418 10.7478237897394
419 10.7409164155087
420 10.7488993794007
421 10.7364350572979
422 10.7319886350732
423 10.7334598514203
424 10.7303306771047
425 10.7382280599972
426 10.7259200710288
427 10.7177325178456
428 10.7133979882126
429 10.701202404746
430 10.6963115916546
431 10.6954446680141
432 10.6945283686782
433 10.6822277851005
434 10.6723118660268
435 10.6790292014255
436 10.6731961268714
437 10.6611526211608
438 10.6509315920872
439 10.6444331711213
440 10.6555031275312
441 10.6664068373494
442 10.6546338240318
443 10.6561549580455
444 10.6471079737881
445 10.6351466919098
446 10.6546228820674
447 10.6434945481862
448 10.6330814104197
449 10.6213997602201
450 10.6252764088955
451 10.6223635552297
452 10.6205751636306
453 10.6099292530825
454 10.6036459365367
455 10.5989575534417
456 10.5994006604054
457 10.6219441268866
458 10.6171972067918
459 10.6061012829783
460 10.6235916446789
461 10.613545708053
462 10.6106618254992
463 10.5993374732672
464 10.6178166509184
465 10.6102013098194
466 10.6091566710925
467 10.6012993638892
468 10.5998023718443
469 10.6085495349069
470 10.5973800327331
471 10.5865663615244
472 10.5975209490582
473 10.5924835192447
474 10.5830138658607
475 10.5968403799803
476 10.6019893660797
477 10.6003564818413
478 10.5895754498889
479 10.5830390144186
480 10.5758321103537
481 10.5665836461964
482 10.5607194826775
483 10.5548552714571
484 10.5599923432612
485 10.5604467420487
486 10.5666047305418
487 10.5650574742158
488 10.5753523327518
489 10.5966760725925
490 10.5932726799828
491 10.5838396677937
492 10.598360076028
493 10.5897976667046
494 10.6011718625327
495 10.5953781701029
496 10.5863491383396
497 10.583097436285
498 10.5856879515043
499 10.5767506896012
500 10.5759916792706
501 10.5965433951535
502 10.5863073662313
503 10.5771711646382
504 10.5667977413298
505 10.5659748034297
506 10.5571245005598
507 10.5582508064568
508 10.5500871671128
509 10.5694400376802
510 10.5720829957802
511 10.5617361166757
512 10.5622450216543
513 10.5575649794863
514 10.5777585700389
515 10.5705987851491
516 10.5634498706
517 10.5771872351968
518 10.5691059675864
519 10.5613543374781
520 10.5572516371534
521 10.5471169913148
522 10.5627509183267
523 10.552650724523
524 10.5446770113781
525 10.5347568141741
526 10.5248645563978
527 10.5305129876556
528 10.5212620606534
529 10.5147211324882
530 10.5052117123886
531 10.5207179484843
532 10.5232895421619
533 10.5385161500739
534 10.5398377503326
535 10.5593233676881
536 10.5625488948846
537 10.5534224185821
538 10.5481194811128
539 10.5384456006553
540 10.5289854025771
541 10.5360631708494
542 10.5372550872947
543 10.5532044918621
544 10.5457825350516
545 10.5374167496072
546 10.5346081231168
547 10.5441111557639
548 10.5367902638777
549 10.5488528769523
550 10.5445197077804
551 10.5350698938821
552 10.5527225212144
553 10.5559655137663
554 10.5677257195779
555 10.5605329765424
556 10.553352852569
557 10.5635277218873
558 10.558434260691
559 10.5589015061408
560 10.5566074615351
561 10.5484274221901
562 10.5455864944162
563 10.5582735945669
564 10.5627082544549
565 10.5589743009462
566 10.5561448542529
567 10.5636348388408
568 10.555553438413
569 10.5518324631772
570 10.5699336071921
571 10.5842047148043
572 10.5846414651359
573 10.588933579303
574 10.606898026217
575 10.6097932432921
576 10.6019856446505
577 10.6062901375861
578 10.6183781765914
579 10.6175477831434
580 10.62634351861
581 10.6272372315192
582 10.6359626965986
583 10.6401330403909
584 10.6463823181693
585 10.6572964541342
586 10.6530373263815
587 10.6457726926663
588 10.6577532162676
589 10.6517765653258
590 10.6503690999398
591 10.6505706778684
592 10.6588184509223
593 10.6539985554817
594 10.6658594268229
595 10.6610019265991
596 10.6649015147222
597 10.6606958175573
598 10.6750657087249
599 10.666399047463
600 10.6769220231717
601 10.6680796828582
602 10.6613590127177
603 10.6546493136465
604 10.668852975795
605 10.6707971022995
606 10.6788455086635
607 10.6835628784738
608 10.6813176664404
609 10.6739060039053
610 10.6655258952829
611 10.666406777683
612 10.6650639725326
613 10.6739968887426
614 10.6656610825148
615 10.6794150964136
616 10.6801660586328
617 10.6937142191227
618 10.6957665643082
619 10.6897367207824
620 10.6857869707469
621 10.6860602246319
622 10.6863040969906
623 10.6777263852905
624 10.6856412340659
625 10.677677650126
626 10.6821610619865
627 10.673689482043
628 10.6801644611569
629 10.6845386750481
630 10.6883294058299
631 10.699177391019
632 10.6998608080053
633 10.6921209916525
634 10.6848011571808
635 10.6867805636009
636 10.6905368259722
637 10.6982997573792
638 10.6960754398498
639 10.6879494451383
640 10.6844654866013
641 10.681836015587
642 10.6742442405015
643 10.6704116596568
644 10.6682414855335
645 10.6599701803024
646 10.6592022958456
647 10.652669887969
648 10.6478411791502
649 10.6626147829288
650 10.6588062877271
651 10.6533854382687
652 10.6664509329621
653 10.6588567735001
654 10.6508027917375
655 10.6662826009368
656 10.676735244951
657 10.6698132908051
658 10.6643584401992
659 10.6663261079171
660 10.6583312602173
661 10.6521223966993
662 10.6672831243022
663 10.6650601425958
664 10.6670664015006
665 10.6597137061524
666 10.6600948333535
667 10.6681054350952
668 10.6612814735227
669 10.6544700843508
670 10.6465950723221
671 10.6386589968718
672 10.6510634709039
673 10.6437897208309
674 10.6415697363794
675 10.6339574619779
676 10.6285737661179
677 10.6432684873546
678 10.635486220485
679 10.6320802683286
680 10.6466524743712
681 10.6391127958409
682 10.6468193048892
683 10.6488972492034
684 10.6494466513697
685 10.6499701272557
686 10.6465872573312
687 10.6522273838914
688 10.6527073076292
689 10.6602172017361
690 10.6580542775399
691 10.6617297474977
692 10.654656616931
693 10.647222889536
694 10.6420290445035
695 10.6496552322315
696 10.6444492781914
697 10.6368803205581
698 10.6486527922826
699 10.6490930006854
700 10.6414839957899
701 10.6531040255539
702 10.6461473694126
703 10.639600895892
704 10.6399915505738
705 10.6326840980165
706 10.6399193729075
707 10.6330387273252
708 10.6280091656307
709 10.6336996760344
710 10.6429525042787
711 10.6354661722177
712 10.6321018802788
713 10.6249398174562
714 10.6215879779832
715 10.61514358885
716 10.6264115855691
717 10.6321353459149
718 10.6372981179392
719 10.6339156814508
720 10.6270517263756
721 10.6278159277501
722 10.6204549936463
723 10.6198180052353
724 10.6205464759531
725 10.6147601898596
726 10.6186255220531
727 10.6242086830491
728 10.6297382652392
729 10.6387809934824
730 10.6324707930975
731 10.6379763569659
732 10.6398133076139
733 10.6435438354621
734 10.6441444934069
735 10.6384544651209
736 10.6520525135884
737 10.6458674808916
738 10.6464076013913
739 10.6392692607505
740 10.646294155946
741 10.639108146299
742 10.6442492178678
743 10.6422769468926
744 10.6351226663458
745 10.6280389078386
746 10.6272846366219
747 10.6308709574549
748 10.6238284566392
749 10.6229814399852
750 10.619872692269
751 10.613778291617
752 10.609832845412
753 10.6134213555535
754 10.6204701277338
755 10.6293618016317
756 10.6223882434782
757 10.6333742800576
758 10.6285844718471
759 10.6277546361637
760 10.6230077975868
761 10.6210531940645
762 10.6245231911255
763 10.6250017568679
764 10.6230568890938
765 10.6176412450673
766 10.6146216796628
767 10.6180213138331
768 10.6133456271306
769 10.6221779077241
770 10.621364954168
771 10.6282493766907
772 10.636962458088
773 10.6374167527891
774 10.6307796399257
775 10.6239828429083
776 10.6201266699213
777 10.6171813733813
778 10.6163955509744
779 10.6155962048668
780 10.6090383451475
781 10.6024926006155
782 10.5995936735969
783 10.5976805255617
784 10.6027479090874
785 10.5962350085703
786 10.6089295046183
787 10.6051066129481
788 10.6043818885748
789 10.6129155752036
790 10.6077052047077
791 10.609533760711
792 10.6038044738413
793 10.5971165597637
794 10.5914081998767
795 10.5857087031933
796 10.5876290519737
797 10.5894942207771
798 10.5943991077452
799 10.6009654904088
800 10.604196987514
801 10.600394321675
802 10.6071521117788
803 10.6103529010303
804 10.6043092639722
805 10.6093182065285
806 10.6142836519293
807 10.6113965173643
808 10.606964183834
809 10.6191586784251
810 10.6131005970397
811 10.606555668172
812 10.6097139106317
813 10.6031873190934
814 10.6116474080471
815 10.6060280823474
816 10.6093912391748
817 10.6037923287899
818 10.6071338912695
819 10.6015552890517
820 10.5951406902543
821 10.5955686870744
822 10.592732875996
823 10.5931340764291
824 10.5937409937217
825 10.5878592912346
826 10.5997979578887
827 10.6060468149692
828 10.6043628766159
829 10.5980110875273
830 10.6029148291471
831 10.6002392094285
832 10.5974252234443
833 10.592525393083
834 10.5929184492101
835 10.5874366475402
836 10.5880510100441
837 10.5830875193206
838 10.5913452351645
839 10.5959389760393
840 10.5917091501366
841 10.5868569667501
842 10.5887353567895
843 10.5919944832593
844 10.5892565022994
845 10.5973688924665
846 10.5938087995373
847 10.5931669240115
848 10.5871351004253
849 10.5844273639127
850 10.5906206463427
851 10.5953494601668
852 10.5891804774341
853 10.5864733765828
854 10.5803229857467
855 10.5919095990871
856 10.5911488131334
857 10.5870070979936
858 10.5873944325737
859 10.5857711786455
860 10.5805292259887
861 10.5811112271226
862 10.5925825438509
863 10.588467369061
864 10.586685537282
865 10.5911204949666
866 10.5884000066391
867 10.5858743249802
868 10.5802986177574
869 10.5850094657531
870 10.586626534546
871 10.5869869573057
872 10.5982242935749
873 10.6075111437923
874 10.613805402621
875 10.6078076751735
876 10.6037727635183
877 10.6112601055043
878 10.6077374786653
879 10.6029621790262
880 10.5977274555007
881 10.592245662124
882 10.5928950406544
883 10.5869342471656
884 10.581190128213
885 10.5844236771714
886 10.5859571233073
887 10.591731850344
888 10.5902212750678
889 10.5993272426885
890 10.5978210083481
891 10.6035264402817
892 10.599594075176
893 10.595066399453
894 10.6029078053854
895 10.6086014638097
896 10.6027504109408
897 10.5972632162043
898 10.6014139728771
899 10.599597960404
900 10.6085071848662
901 10.6028725404375
902 10.5969956187578
903 10.6016539060561
904 10.5971908070681
905 10.6078674024862
906 10.619220900369
907 10.6286676448837
908 10.6392786334364
909 10.6406408460812
910 10.6349611311504
911 10.6293712028285
912 10.6237900565761
913 10.6223214894735
914 10.6241430805234
915 10.6214892515874
916 10.6275127651697
917 10.6218901876995
918 10.6210629563466
919 10.6321761259442
920 10.6335649343769
921 10.6440086303097
922 10.6526234722612
923 10.6585917203021
924 10.6559199988859
925 10.6525061472382
926 10.647918811485
927 10.658944403606
928 10.6580630215896
929 10.6528414216252
930 10.6472769158694
931 10.6415591377357
932 10.6428662280946
933 10.6375530829258
934 10.631888151014
935 10.6428636680155
936 10.644167179057
937 10.6388768288048
938 10.6413940007417
939 10.6405058914443
940 10.6506041055656
941 10.6550668487944
942 10.6625280289929
943 10.6580086707406
944 10.6638690216942
945 10.6586116390274
946 10.6568038783732
947 10.6534511782293
948 10.6564856438287
949 10.6511118274062
950 10.6562997636956
951 10.65204299892
952 10.6502435992116
953 10.6448084491787
954 10.6521873926003
955 10.6466390703556
956 10.6415773272464
957 10.6505458658934
958 10.6572127722714
959 10.6546203071265
960 10.6532221178879
961 10.6598406777122
962 10.655922874316
963 10.6576738366863
964 10.6594057886781
965 10.6539118641499
966 10.650635594255
967 10.6464468292398
968 10.6455928877718
969 10.6473302468021
970 10.6418425989856
971 10.6475397985106
972 10.6436902459399
973 10.6432361468603
974 10.6381525138489
975 10.6333950521602
976 10.6280148512084
977 10.6345935338579
978 10.6332291424482
979 10.6286436504665
980 10.6390913857514
981 10.6368681491714
982 10.6363963500279
983 10.639256626782
984 10.6343211068624
985 10.62908643393
986 10.624819833277
987 10.6250466995792
988 10.6198296275952
989 10.6214936833741
990 10.6243413403098
991 10.620102766771
992 10.6155616534377
993 10.6131465174538
994 10.6078412795759
995 10.6103309443874
996 10.6127993468098
997 10.6152466110086
998 10.6176728602818
999 10.620559013123
1000 10.6218159935107
};
\addplot [semithick, color7, forget plot]
table {%
1 0
2 7
3 10.2306728354819
4 9.51314879522022
5 9.57914401186244
6 9.1469484893415
7 8.81557064030943
8 8.27269605388715
9 7.79996834434855
10 7.8
11 7.9127473261778
12 8.83608447723815
13 8.49434200829271
14 8.38152730712011
15 9.08331804280059
16 9.37395827545653
17 9.79266019892584
18 10.1551239879125
19 10.6195867748561
20 10.6484740690861
21 10.4675288868831
22 10.8654429840513
23 10.7903198233583
24 10.6611313762763
25 10.5375518978556
26 10.5959269834712
27 10.4361669326872
28 10.3672364609251
29 10.5087020166014
30 10.6617176018793
31 10.9018895917238
32 11.0145057055458
33 11.1144806644652
34 11.0273723604706
35 10.9072527826812
36 10.8417346341138
37 10.9713751998771
38 10.8644697800265
39 10.8532134387051
40 10.7375218276844
41 10.6578947175151
42 10.6261187219497
43 10.6533967394261
44 10.6726992023791
45 10.8697337185007
46 11.0418733856123
47 10.9398536722528
48 10.8851126353076
49 10.8299490650397
50 10.7275346655231
51 10.6219868740205
52 10.524485398318
53 10.4414828401786
54 10.3906052586268
55 10.325544029451
56 10.2523640877485
57 10.3431431111767
58 10.2989932897542
59 10.2436000297022
60 10.1584119493813
61 10.3837934212715
62 10.5079623939781
63 10.4378084771176
64 10.3942662792703
65 10.498509333465
66 10.4225355367881
67 10.3676244359372
68 10.4027461772208
69 10.493153967861
70 10.4647610813528
71 10.6239951040015
72 10.6679686705368
73 10.608525553144
74 10.6184576815763
75 10.7011442794165
76 10.6982261584007
77 10.7393567358579
78 10.7344370780207
79 10.6672441492987
80 10.6310570970153
81 10.5872097750787
82 10.5437249975196
83 10.5751978286925
84 10.5698934428068
85 10.6492620404853
86 10.601437701617
87 10.6962283584719
88 10.6491521798104
89 10.5978347672923
90 10.6550920997578
91 10.6561567693113
92 10.7162833451463
93 10.6976008258533
94 10.6875879406504
95 10.6575503518109
96 10.6153550995192
97 10.6656646976882
98 10.636863125135
99 10.6476337693995
100 10.7192537053659
101 10.6972122520631
102 10.6749535284012
103 10.6456565240378
104 10.7391363505737
105 10.7495251173916
106 10.7585602202864
107 10.8000915489305
108 10.8277248856939
109 10.8022838286394
110 10.7768078194268
111 10.7653548868029
112 10.7339700132289
113 10.813869613605
114 10.8394400005724
115 10.7929473576686
116 10.8793649924853
117 10.9226422074191
118 10.8948514442353
119 10.8742612920137
120 10.8478076382076
121 10.8165748953336
122 10.7721540998331
123 10.7912060906802
124 10.7913562116435
125 10.8125675026795
126 10.7983095688152
127 10.7609596038877
128 10.732859586445
129 10.6940393779177
130 10.6591516919922
131 10.6864067434156
132 10.6537321499839
133 10.6465581324595
134 10.6269940668078
135 10.6809986021226
136 10.6966670099198
137 10.7487297392039
138 10.7524536455307
139 10.745605611436
140 10.7945824469537
141 10.8280430722643
142 10.8860598176522
143 10.8789524512024
144 10.8576649832682
145 10.8203797674358
146 10.8378502451281
147 10.8035863873191
148 10.8664005738654
149 10.8417869767549
150 10.8373038877553
151 10.8059584587965
152 10.794150141874
153 10.7727725925409
154 10.7724513245228
155 10.8108589953084
156 10.7819064595902
157 10.7806419221285
158 10.7525219408494
159 10.750865469451
160 10.7870292481294
161 10.8060433148766
162 10.8411318699907
163 10.8454667231832
164 10.8208534104587
165 10.7936042008085
166 10.7617139370602
167 10.7329245102699
168 10.7126834119353
169 10.6826384076511
170 10.6517636663678
171 10.6776387602731
172 10.6667319818962
173 10.6365292562596
174 10.6784222899603
175 10.6821170415698
176 10.6521575022377
177 10.6757149451975
178 10.6460540748782
179 10.6591915252125
180 10.6331925568296
181 10.6427254391894
182 10.6858173495737
183 10.6701485851523
184 10.7080265439958
185 10.6986615357227
186 10.6698746476756
187 10.6605776919566
188 10.6608378238053
189 10.6458471338121
190 10.6181989802644
191 10.5930010775592
192 10.576523399597
193 10.6170721355164
194 10.5913373482946
195 10.6011597991035
196 10.6125527329764
197 10.6296597944244
198 10.6243193003532
199 10.6287976322629
200 10.6517310799701
201 10.6323098954334
202 10.6546568953437
203 10.6488494957666
204 10.6327766845292
205 10.61937438991
206 10.6028319001108
207 10.6063987376439
208 10.6047841093498
209 10.6249230885057
210 10.6363184345557
211 10.6418125814417
212 10.637514095087
213 10.673481506822
214 10.648950376323
215 10.6283863308518
216 10.6228806487913
217 10.6734398070173
218 10.6503488271775
219 10.6705108948169
220 10.6476969534836
221 10.6967789847904
222 10.6768578050795
223 10.6529342787521
224 10.6291924126748
225 10.6437717254369
226 10.6352104938093
227 10.6495025224844
228 10.6408426145756
229 10.6199921709135
230 10.5972889495986
231 10.6372084369924
232 10.6628139098281
233 10.6545840191843
234 10.6318405076875
235 10.6356507344849
236 10.6549885612807
237 10.6325413504694
238 10.6160839917202
239 10.6117728593827
240 10.592027268973
241 10.6232721379767
242 10.6475672563062
243 10.62570725297
244 10.6383867247
245 10.6383853872888
246 10.6424379052013
247 10.6377425390108
248 10.6163435692538
249 10.616211960384
250 10.5988708832592
251 10.5884062724903
252 10.6013351163156
253 10.6253558148688
254 10.6048318668859
255 10.5917829931622
256 10.5767170407161
257 10.5764008653715
258 10.5613640365192
259 10.5409931027062
260 10.5372900808757
261 10.5410063069609
262 10.5829040681849
263 10.5638722437275
264 10.5538406006214
265 10.5619897906354
266 10.5425143891015
267 10.5238713956568
268 10.5116880867682
269 10.4996457845426
270 10.4812825527517
271 10.4988598911195
272 10.5160092775048
273 10.524275516917
274 10.5204995133876
275 10.5171871926994
276 10.5206380178018
277 10.5020638022811
278 10.485354834772
279 10.4855979951779
280 10.4932412649714
281 10.4900722343109
282 10.4749402154128
283 10.4615614633558
284 10.4582491461684
285 10.4798212940482
286 10.5072530598585
287 10.5039800255559
288 10.4861019131674
289 10.468973499698
290 10.4853762611974
291 10.4819164799171
292 10.465847557491
293 10.4498509542781
294 10.432125278081
295 10.4326790842818
296 10.4361143026749
297 10.4628923992192
298 10.4700723806931
299 10.4732482857054
300 10.4931993849985
301 10.4959509879487
302 10.508193498804
303 10.5343784342536
304 10.5380891190887
305 10.5486961561098
306 10.5590419989414
307 10.5845701386449
308 10.5677763402972
309 10.5515306753872
310 10.5365031211922
311 10.5262095582655
312 10.5103479737787
313 10.5022676008019
314 10.5375428310967
315 10.5292595239385
316 10.5234814045379
317 10.5525828236134
318 10.5525487609694
319 10.5628883880572
320 10.5464185086396
321 10.5308521065672
322 10.5494659129354
323 10.5602703141863
324 10.5708304034443
325 10.5575458866125
326 10.5413805822827
327 10.5261706872726
328 10.5542333995977
329 10.5680887736606
330 10.5681699793569
331 10.5524821691345
332 10.575693252814
333 10.5701192573598
334 10.5729937133701
335 10.5581013502043
336 10.557939968189
337 10.5642074738182
338 10.554520073471
339 10.5573601651166
340 10.5434883723948
341 10.5283509797834
342 10.5187224886551
343 10.5160289233502
344 10.5160437190167
345 10.5011040517366
346 10.4958449016331
347 10.4931353319525
348 10.4879121230217
349 10.5144304025122
350 10.5203107737134
351 10.5092350991853
352 10.4943151055762
353 10.5044517462024
354 10.5108727747642
355 10.5034695432482
356 10.5032022698784
357 10.4887400679196
358 10.4766423065271
359 10.462092217222
360 10.4798513324673
361 10.472902096238
362 10.4862973668226
363 10.4758346230932
364 10.5014658898196
365 10.4909436500092
366 10.4837945440876
367 10.4698264762811
368 10.4724862312583
369 10.4821886463176
370 10.4705381650975
371 10.4762148258483
372 10.4624595250455
373 10.4596940438581
374 10.4513240930574
375 10.4425638401475
376 10.4638107381092
377 10.4588660964985
378 10.4643769237032
379 10.4801938104851
380 10.5083530947786
381 10.5089215213327
382 10.5110042049593
383 10.5279519580072
384 10.5235734751588
385 10.5468286482399
386 10.5340651840968
387 10.5402230354891
388 10.5350812565218
389 10.5287552158433
390 10.5217601140416
391 10.5344930143527
392 10.524492157974
393 10.5149826246964
394 10.5038987153249
395 10.5309438665942
396 10.5230433033048
397 10.5290498110608
398 10.5160248246029
399 10.5250661628467
400 10.5207506267376
401 10.5113589090688
402 10.498295519117
403 10.5039978883996
404 10.5126475380369
405 10.5124871132914
406 10.5125856378637
407 10.524307242195
408 10.5198085680647
409 10.5174479225702
410 10.505341266325
411 10.4990572092883
412 10.4877073571132
413 10.4785237250431
414 10.4741488268718
415 10.4617920432672
416 10.4642954511969
417 10.4551300122021
418 10.4439601300554
419 10.4551965080429
420 10.4550045519553
421 10.4447388131608
422 10.4525346476219
423 10.4555701776157
424 10.4493131471604
425 10.4391209907168
426 10.4510345564471
427 10.4558569981343
428 10.4540749673098
429 10.4727158999533
430 10.4618033515845
431 10.4802129308879
432 10.4703920589427
433 10.4643711793765
434 10.453724825028
435 10.4497853062898
436 10.4385136050254
437 10.4272781035682
438 10.4417735210711
439 10.4560632342166
440 10.4736100874752
441 10.4984352183761
442 10.5127155591577
443 10.5338807851447
444 10.5315919799589
445 10.5339522967895
446 10.5338024276722
447 10.5506946636599
448 10.5679251856546
449 10.5727014036226
450 10.5641032058197
451 10.5743144820065
452 10.5628653704183
453 10.5760619497811
454 10.5650112094796
455 10.5819759587424
456 10.5841903158167
457 10.5747273775135
458 10.5911296214583
459 10.6073140107448
460 10.6175657856973
461 10.6221421506363
462 10.6197482134331
463 10.6138972497682
464 10.6056047621383
465 10.6183034707775
466 10.6143597451067
467 10.6185278120503
468 10.6101033561478
469 10.6141697237398
470 10.6041998222652
471 10.6042392365472
472 10.6231298098899
473 10.6384391479449
474 10.6422276847724
475 10.631027812708
476 10.6460166429638
477 10.650812948454
478 10.6544576260946
479 10.6568170408998
480 10.6531294306728
481 10.6532038630116
482 10.6569065413326
483 10.6561519874336
484 10.6456392527674
485 10.6571940437814
486 10.6732814058486
487 10.6755230135451
488 10.6663501299208
489 10.6581167039535
490 10.6537876052303
491 10.64796637428
492 10.6421452164799
493 10.6423801925292
494 10.647028190861
495 10.6583714301341
496 10.6641928787632
497 10.6614704148456
498 10.6564781822871
499 10.6722144402928
500 10.662851401009
501 10.6753445015686
502 10.6837265791872
503 10.7023035358507
504 10.6996841641776
505 10.7030159092643
506 10.6966293165559
507 10.6862234353643
508 10.6812023106561
509 10.6868993890162
510 10.6800024767193
511 10.6967261769191
512 10.688368020306
513 10.6963474803297
514 10.7058438519707
515 10.706843656465
516 10.7136732965439
517 10.7094057497013
518 10.7006584875791
519 10.7111072630607
520 10.7008647275423
521 10.7111977440024
522 10.7043049411296
523 10.7118718704885
524 10.7270175836947
525 10.7345097593343
526 10.7312898260082
527 10.7280546795403
528 10.7373354123519
529 10.7501586333465
530 10.7568367656041
531 10.7610939365487
532 10.776734475905
533 10.7666224426915
534 10.7851501487987
535 10.7894061388183
536 10.7806172653088
537 10.7712723164251
538 10.7754550908651
539 10.767767474347
540 10.7831086472671
541 10.7784210569641
542 10.768533757681
543 10.7608882665552
544 10.7529994445
545 10.7438173058817
546 10.7458949530688
547 10.7547000401695
548 10.7477202077374
549 10.7467741038496
550 10.759173061099
551 10.7632533359011
552 10.7549824085331
553 10.7590234978645
554 10.7609421656231
555 10.7527443585798
556 10.7449722114413
557 10.7391390074256
558 10.7410159168591
559 10.7439242300008
560 10.7375686011446
561 10.7512709565086
562 10.7663246000339
563 10.7568848216065
564 10.7529261718137
565 10.7519908203982
566 10.7667509662578
567 10.7584517301623
568 10.7669216522045
569 10.7659435923351
570 10.7567634245676
571 10.7637034715704
572 10.7584547564072
573 10.7497200640808
574 10.764158169307
575 10.7777478524628
576 10.7893733933649
577 10.7808051576259
578 10.7977746760952
579 10.7905175480959
580 10.7812119133726
581 10.7877353008342
582 10.7834805148444
583 10.7969747652529
584 10.7926688521578
585 10.7842080088012
586 10.7761990626208
587 10.7707580476384
588 10.7867620759053
589 10.7905608313412
590 10.7905952699471
591 10.7817165942753
592 10.7842525077213
593 10.788719007905
594 10.7848213593666
595 10.7760226268182
596 10.7690397392081
597 10.7696543543772
598 10.783359336553
599 10.7746929209971
600 10.7718455036988
601 10.7723896941252
602 10.7697951691657
603 10.7645553620586
604 10.7727288246547
605 10.7746605777731
606 10.7770108063853
607 10.771985100667
608 10.7824048921764
609 10.778271839399
610 10.7730161475238
611 10.7659860597287
612 10.7572873866671
613 10.7636688045966
614 10.7694385258657
615 10.7656882559745
616 10.7809589233894
617 10.7827325186503
618 10.7740549201933
619 10.7865059130045
620 10.7919936986488
621 10.7995392085199
622 10.7925258278444
623 10.792416982993
624 10.7877011745092
625 10.7951277898874
626 10.8037854170824
627 10.795715550103
628 10.793455493404
629 10.7904905820634
630 10.7832663631809
631 10.783171496859
632 10.7839696244583
633 10.7770872311666
634 10.7890190546559
635 10.7855445266311
636 10.7880541598049
637 10.7888112588118
638 10.785906973977
639 10.7892140798377
640 10.8023768317368
641 10.7939870966283
642 10.7861189539555
643 10.7923736163585
644 10.7852840505228
645 10.7995635412701
646 10.796667984482
647 10.7911374647667
648 10.7895756386912
649 10.8050776870467
650 10.7975154933162
651 10.8057315769495
652 10.7974844663263
653 10.7934958769542
654 10.8028120270786
655 10.8155805317513
656 10.83072391887
657 10.8225223378494
658 10.8179404709287
659 10.8124210368094
660 10.8183112631251
661 10.8306951092946
662 10.8358882179266
663 10.8309649243095
664 10.8244252743152
665 10.8229758349811
666 10.8308373068058
667 10.8407859714534
668 10.8459008480126
669 10.8380229466299
670 10.8311284605583
671 10.8246620919735
672 10.8171545694339
673 10.8098071414347
674 10.817556562245
675 10.8107180469827
676 10.8072504959447
677 10.8088828390971
678 10.8034851494355
679 10.7966933445173
680 10.8018308713389
681 10.8094813833156
682 10.8068837278368
683 10.8069445628087
684 10.8035411035542
685 10.7959638288516
686 10.7912623243459
687 10.7844021273357
688 10.778139051757
689 10.7755625528355
690 10.7677535195608
691 10.768348927139
692 10.7802995551351
693 10.7921457616827
694 10.8011897410935
695 10.7935023306847
696 10.7901464710491
697 10.7846972014689
698 10.7848057755099
699 10.7804740202397
700 10.7727721784211
701 10.7668314887889
702 10.7597046306999
703 10.7714235330808
704 10.7803953540462
705 10.7937152323319
706 10.8011216876089
707 10.8105027554174
708 10.8198012848479
709 10.8122221563676
710 10.8068770488655
711 10.8161171496735
712 10.8090730771782
713 10.8200317317109
714 10.820126066292
715 10.8337424760767
716 10.8292450006068
717 10.8400437488515
718 10.8404394881686
719 10.8533894238313
720 10.8640149570798
721 10.8618305510897
722 10.8543899280755
723 10.8578796714847
724 10.8504268050147
725 10.8522824878546
726 10.8454204818782
727 10.8389183687823
728 10.8495115436883
729 10.8431489577781
730 10.835939935943
731 10.8288129999951
732 10.8302895413219
733 10.8270767039179
734 10.8300975526686
735 10.8265911163278
736 10.8201559588306
737 10.8130226609893
738 10.8059034364522
739 10.8149473847517
740 10.8105641863761
741 10.8209367907786
742 10.8312240646356
743 10.8364583553493
744 10.847467949593
745 10.8598338878838
746 10.8525932773544
747 10.8530421138121
748 10.8559034233693
749 10.8628285709302
750 10.8673047665412
751 10.8677315232629
752 10.8609908928734
753 10.8671880566005
754 10.8722671492907
755 10.8851656523605
756 10.8795783276641
757 10.8876220225787
758 10.8804396018244
759 10.890393905564
760 10.885541504694
761 10.8804270061621
762 10.8743442230096
763 10.867692588599
764 10.8720491384287
765 10.8672560889104
766 10.8621793419578
767 10.8601795822352
768 10.8729232032777
769 10.8668914382025
770 10.8754285833392
771 10.8782159615097
772 10.8814711135487
773 10.8747037075019
774 10.8713412174178
775 10.8645999537031
776 10.8771030500789
777 10.8750245587856
778 10.8781859875527
779 10.8900831837748
780 10.8831009788802
781 10.8776726225385
782 10.8859576257517
783 10.8794992304051
784 10.8731204572191
785 10.8682351201173
786 10.8800439013487
787 10.8733348564481
788 10.8746529329762
789 10.8678023188819
790 10.8638900359709
791 10.8698655110127
792 10.8757902671393
793 10.8689706025898
794 10.8791833777405
795 10.8855911323211
796 10.8886897162039
797 10.8818572771437
798 10.8755161383752
799 10.8743226209333
800 10.8683841945342
801 10.8779113446907
802 10.8854962373406
803 10.8871876475655
804 10.8953135983766
805 10.901116656288
806 10.8991218327864
807 10.8950341187173
808 10.8927287666509
809 10.8859960498517
810 10.879275801969
811 10.8730210512404
812 10.8707355565612
813 10.8688126929842
814 10.8700362855009
815 10.8643530011439
816 10.8620786205211
817 10.8567317389826
818 10.86870688181
819 10.8623347157734
820 10.868613634859
821 10.8674216996979
822 10.8665870769508
823 10.8711680815632
824 10.8654028204159
825 10.868435759932
826 10.8653073802269
827 10.8633740649703
828 10.8622312214273
829 10.8683803940742
830 10.876208652216
831 10.8705903456193
832 10.8645229592608
833 10.8660738943766
834 10.8595578643694
835 10.8584744024536
836 10.8520438238341
837 10.8468848954836
838 10.8498365142946
839 10.8590705980048
840 10.8664433196654
841 10.8618711774389
842 10.8610214945282
843 10.8636347738188
844 10.8572667142555
845 10.8553788200138
846 10.8497810291564
847 10.8434428750306
848 10.8370482532337
849 10.8370896403144
850 10.8461329833283
851 10.8461252008178
852 10.84030054598
853 10.8395249139663
854 10.8341069424356
855 10.8413194514527
856 10.8391874584073
857 10.8349128831175
858 10.8331141491862
859 10.8391368407287
860 10.8342492089724
861 10.841432790619
862 10.8352134122304
863 10.8316662331892
864 10.8393011790104
865 10.8374696356597
866 10.8320149330332
867 10.8258327760474
868 10.8203968931471
869 10.8311441121231
870 10.824918589316
871 10.8194918189213
872 10.8145341777798
873 10.8145661484113
874 10.8089067418979
875 10.805436938346
876 10.8109003123554
877 10.81965548861
878 10.818950969691
879 10.817864722769
880 10.8121316160434
881 10.8084256013073
882 10.8063564959281
883 10.8123421878067
884 10.8082288373119
885 10.802370044116
886 10.7976741042922
887 10.8063438179657
888 10.8079992942271
889 10.8046006758466
890 10.8046187643572
891 10.809940366792
892 10.8044106279088
893 10.8007551966285
894 10.7973998638954
895 10.804270881461
896 10.7982424297064
897 10.7922978755085
898 10.7912239175325
899 10.7866266087454
900 10.7877742300952
901 10.7857376004125
902 10.7817582737416
903 10.7757894895657
904 10.7802174986351
905 10.7870085912712
906 10.7921931405606
907 10.8023097386915
908 10.7970676206871
909 10.7934285245645
910 10.7886280233519
911 10.7968945368787
912 10.8079695891738
913 10.8024217013329
914 10.8034484436162
915 10.8064860865599
916 10.8008501371375
917 10.8031120556187
918 10.7973115483453
919 10.793433172832
920 10.7905554517143
921 10.7890439910207
922 10.7843059551742
923 10.7827990239362
924 10.7909670174754
925 10.785276656132
926 10.7862932272743
927 10.7885096602175
928 10.7902825386225
929 10.7946885389064
930 10.7945784022007
931 10.7888029436842
932 10.7831545592856
933 10.7928957079584
934 10.7881949416606
935 10.7870459496166
936 10.7834701322777
937 10.7778055395397
938 10.7727151140458
939 10.7669975296032
940 10.767640100614
941 10.7671981698946
942 10.7643737148287
943 10.7623509908821
944 10.7612236538654
945 10.7671242668869
946 10.7617026542473
947 10.7562891961169
948 10.7642015740469
949 10.7700671306097
950 10.777936064812
951 10.7726069312924
952 10.7689221123388
953 10.7737032402202
954 10.7708819733237
955 10.780309709724
956 10.7811916814531
957 10.7765006998624
958 10.7763035028972
959 10.7793883565014
960 10.7852279295829
961 10.7806398359385
962 10.7824350234932
963 10.7917644823368
964 10.7861829470122
965 10.7808694437717
966 10.7780606628657
967 10.780088958276
968 10.7797166216434
969 10.7756488217165
970 10.7727205538115
971 10.7685681197832
972 10.7692310518699
973 10.7749840895154
974 10.7779639619817
975 10.7725633915509
976 10.7758814443014
977 10.7788534739995
978 10.7748485741523
979 10.7820277283783
980 10.7891616428063
981 10.7967683119994
982 10.8009722666029
983 10.7968171113988
984 10.7918513222093
985 10.8004774464701
986 10.7952494728219
987 10.7941849249815
988 10.7908135072891
989 10.7853781173795
990 10.7799345110461
991 10.787495024627
992 10.7873608433943
993 10.7879380749962
994 10.7874921329617
995 10.7944787131018
996 10.789080339328
997 10.791900878031
998 10.7959782499274
999 10.7914248942225
1000 10.7904001779359
};
\addplot [semithick, color8, forget plot]
table {%
1 0
2 9.5
3 10.0332779621949
4 11.3880419739304
5 10.2097992144802
6 9.322910847298
7 10.3844467687921
8 10.7812047100498
9 10.7749137088006
10 11.4087685575613
11 10.8833028530743
12 11.4406682011537
13 11.01478296595
14 10.634224687013
15 10.9036181558641
16 11.2178359655506
17 10.910773698458
18 11.0855260988773
19 11.4136527082282
20 11.1824639503108
21 11.2106201672855
22 11.3603631649551
23 11.4808956132863
24 11.3124808164355
25 11.3238862587011
26 11.3273638414961
27 11.1437791364991
28 11.4948968269065
29 11.4498337120368
30 11.2794798934466
31 11.110456680924
32 11.0012428275173
33 11.2495689132793
34 11.4139249016672
35 11.2531537529599
36 11.098047876391
37 10.9631829579625
38 11.0664387443294
39 10.9622506891122
40 11.0035221633802
41 11.0618831477529
42 10.9466082095102
43 10.9964840243499
44 11.0360743933941
45 10.9150705646871
46 10.7980254443655
47 10.9135456013007
48 11.0138801317045
49 11.1577235059801
50 11.0828877103398
51 11.213595033338
52 11.1292002243259
53 11.1109168924128
54 11.020634717418
55 10.9229608797324
56 10.8329146962545
57 10.8185743629435
58 10.8987475962909
59 10.8147016027041
60 10.7693495109449
61 10.6809537705206
62 10.6358708237257
63 10.6297073220565
64 10.549733408954
65 10.654731982057
66 10.6094475111664
67 10.6297790740583
68 10.5517754618796
69 10.6525781250332
70 10.6876335174856
71 10.7827365385817
72 10.7101886272884
73 10.6466648802276
74 10.597670885424
75 10.5423084126137
76 10.594908095983
77 10.6601149272148
78 10.6109630950294
79 10.5549128327175
80 10.6148480912352
81 10.5735101072641
82 10.6115571782175
83 10.5858304108373
84 10.6620087087385
85 10.6094558594312
86 10.6647623142308
87 10.6065487788574
88 10.6281776969067
89 10.7243762489268
90 10.7940598661978
91 10.7692195559484
92 10.7338043472757
93 10.7185803130109
94 10.7465766180957
95 10.7600368657268
96 10.7267139186731
97 10.7956604925941
98 10.8753108693785
99 10.9501285738618
100 10.8977566498798
101 10.8565067516735
102 10.9280835639477
103 10.8775392275767
104 10.8280502214218
105 10.8687239820382
106 10.8300119964551
107 10.8253510818792
108 10.8390258864601
109 10.7892014583446
110 10.8201258757439
111 10.7788717390949
112 10.8083925193361
113 10.8442508207538
114 10.8961528305461
115 10.911792757014
116 10.8999987727584
117 10.9602979921556
118 10.9487884968603
119 10.9300437753473
120 11.0293516229297
121 11.0412851981363
122 11.0353838319378
123 10.9906371079479
124 10.9501133231489
125 10.9100291475321
126 10.8854551715008
127 10.8425225441587
128 10.9015032650318
129 10.886056199476
130 10.9193926614384
131 10.9502160487049
132 10.9185223329772
133 10.8775608313578
134 10.8886095428566
135 10.8542507776967
136 10.8564651340854
137 10.9260735184101
138 10.9195493243043
139 10.9448083951186
140 10.9986641118319
141 10.9651812313636
142 10.9349599241129
143 10.9894965835858
144 10.9626540559582
145 10.927659655252
146 10.9351047225747
147 10.9590886629865
148 10.9415469138411
149 10.9481541039922
150 10.9875262608712
151 10.9511035920084
152 10.946480780689
153 10.959363617633
154 10.9291173633423
155 10.92465159568
156 10.8938455623994
157 10.8600351146853
158 10.8257916468628
159 10.7926104282522
160 10.7951884536584
161 10.8087705344332
162 10.8179377784205
163 10.8035398868307
164 10.8601566598727
165 10.8276149349274
166 10.8163199999947
167 10.7877238819799
168 10.7846745399188
169 10.761815510669
170 10.7587978001172
171 10.7482293829658
172 10.8014994552303
173 10.8352967341732
174 10.8591980637093
175 10.8464819839693
176 10.8296095462188
177 10.7989942585296
178 10.7931115725259
179 10.7780870670631
180 10.751032106641
181 10.7487348613795
182 10.7205110119774
183 10.7429248008444
184 10.720531277834
185 10.7083250315695
186 10.6881084894586
187 10.6619939458701
188 10.6344514322898
189 10.6849569189015
190 10.6664046597554
191 10.6466171341869
192 10.6470041247103
193 10.6302506201753
194 10.633880695293
195 10.6248366193882
196 10.6563545617051
197 10.6930425033673
198 10.7237291794188
199 10.7035742089238
200 10.6972473094717
201 10.6922726739124
202 10.7150317636263
203 10.7582079171615
204 10.8104440561913
205 10.8045267795128
206 10.8252688405797
207 10.8400178107495
208 10.8139823948759
209 10.7881379200516
210 10.7690620953126
211 10.7526886874455
212 10.7862497016929
213 10.7650237787454
214 10.7410259877632
215 10.7229670667675
216 10.6993094594041
217 10.7390556548695
218 10.7144111839901
219 10.7243642658973
220 10.7000608247286
221 10.6941448721623
222 10.7039737847453
223 10.681089097115
224 10.7270301807501
225 10.7078372128912
226 10.7487557719997
227 10.768877223044
228 10.7884038836344
229 10.7652612027471
230 10.7903382183222
231 10.7730871472751
232 10.7810435818355
233 10.7937909007903
234 10.7718394954557
235 10.7662568287362
236 10.8093344916179
237 10.8318309633319
238 10.8240056787536
239 10.8403977874778
240 10.8679063710031
241 10.8995027315048
242 10.9050181791505
243 10.9058839841213
244 10.8851000817312
245 10.9054548860828
246 10.9059161864416
247 10.8956778600551
248 10.909677151094
249 10.8908124865036
250 10.8996146720882
251 10.9035927613696
252 10.8850906189529
253 10.8643954648032
254 10.8438176477089
255 10.8310388839862
256 10.8272731165168
257 10.8070551944041
258 10.8201678401685
259 10.8353293392886
260 10.8185710373386
261 10.8097254210889
262 10.8471348511643
263 10.8267076010503
264 10.8176400159292
265 10.8316283311103
266 10.828621572911
267 10.8085033931114
268 10.7884865932024
269 10.8178911328954
270 10.8410488511049
271 10.821153851531
272 10.8351114651767
273 10.8166755343707
274 10.7970546566591
275 10.8107996202756
276 10.808093689636
277 10.842337501699
278 10.8268218524586
279 10.8450134330411
280 10.8732868695229
281 10.8814333906126
282 10.9039969986124
283 10.9313397036304
284 10.9478863953048
285 10.941280358432
286 10.9621468474167
287 10.9446596415824
288 10.9723430900516
289 10.9613467999829
290 10.9772026221551
291 10.9734486441924
292 10.9598950750587
293 10.9848949436898
294 11.0021748046779
295 10.987697280387
296 10.9696185251822
297 10.994261611811
298 10.9888631069062
299 11.005660977453
300 11.0018705480275
301 10.9937435235814
302 11.0130618471263
303 10.9949602228098
304 11.016183146554
305 11.0306582118142
306 11.026709424454
307 11.0088149845237
308 10.9925034637278
309 10.9795433091753
310 10.9729707649729
311 10.9990407794334
312 10.9880242135557
313 10.9761665565696
314 10.9923185997083
315 10.9984382409039
316 10.9849717446196
317 10.9742043829432
318 10.9573742327774
319 10.9409122244474
320 10.9334988217119
321 10.9260663934899
322 10.909771689685
323 10.9174877287178
324 10.901774153697
325 10.8882930454588
326 10.9024588899047
327 10.9272101759218
328 10.9411514470874
329 10.9282754407705
330 10.913709460039
331 10.9145519827882
332 10.9123488114594
333 10.9161864719807
334 10.9063403197536
335 10.8913850251311
336 10.8951756412369
337 10.8956801877605
338 10.8830036876674
339 10.897214320189
340 10.8816756545995
341 10.8691833037345
342 10.8567294774318
343 10.8544321819008
344 10.843493973198
345 10.8577311424559
346 10.8529071022045
347 10.863126057652
348 10.8558228711631
349 10.8663760029449
350 10.8805195704272
351 10.9072718770776
352 10.9174236493771
353 10.9350063837434
354 10.9242854433839
355 10.9151189395258
356 10.9216788426194
357 10.9191052459073
358 10.9069696160379
359 10.8978957259125
360 10.8978973507125
361 10.9233070217032
362 10.9335201925762
363 10.9401074223804
364 10.9262673055449
365 10.9112904007636
366 10.9364414295586
367 10.9215336217403
368 10.931463700232
369 10.9318616890377
370 10.9483911182826
371 10.9412848334282
372 10.9311728928769
373 10.9344556271185
374 10.9507386952624
375 10.9441429490339
376 10.9564913857375
377 10.9476408294255
378 10.9405711063878
379 10.9411625756004
380 10.9299389971763
381 10.9160032412371
382 10.9410358241356
383 10.9497703244385
384 10.9428604795666
385 10.9291868089198
386 10.9377259625048
387 10.9280881947325
388 10.9140790680918
389 10.9138411031157
390 10.9097499053841
391 10.8963407237461
392 10.896993849606
393 10.8908355301671
394 10.9001654290182
395 10.8864574557819
396 10.9026168593694
397 10.8888771344287
398 10.8781159666861
399 10.88691398884
400 10.9106458905969
401 10.9107912927199
402 10.9086939626244
403 10.9318124084078
404 10.9237121106268
405 10.9111789995625
406 10.9268745816886
407 10.9388448081573
408 10.944289168234
409 10.9326473937684
410 10.9326478954528
411 10.926308122442
412 10.9149044337203
413 10.9026204624911
414 10.9051098910922
415 10.8945848253376
416 10.8901341251976
417 10.8779563619947
418 10.8839570282355
419 10.874728842054
420 10.8646723577923
421 10.862380033003
422 10.8586380357099
423 10.84579722519
424 10.8330806797952
425 10.8390385048372
426 10.8272135675781
427 10.8146125014965
428 10.8205197719522
429 10.8211163936837
430 10.8086210908897
431 10.8012328360358
432 10.8198988299806
433 10.8312305170064
434 10.8342219305793
435 10.8244596115412
436 10.8147209647931
437 10.8294367055532
438 10.8377251949004
439 10.8458758938605
440 10.8361079038616
441 10.8247945289084
442 10.8141672459093
443 10.8023985757858
444 10.790227213903
445 10.7780969009591
446 10.7698198913009
447 10.7779514215916
448 10.7996598585893
449 10.820814861905
450 10.8381240689214
451 10.8261030547687
452 10.8398507427017
453 10.8341483254103
454 10.8230256177236
455 10.8314760901063
456 10.8234053165798
457 10.8125444554791
458 10.800811957836
459 10.8084253243888
460 10.8297217575493
461 10.8281962548727
462 10.8181419753315
463 10.8079834168782
464 10.8025591704647
465 10.8136080727661
466 10.8343338203681
467 10.8291013418709
468 10.8238615262887
469 10.813891879944
470 10.8341973631366
471 10.8227817118397
472 10.8175136226737
473 10.8076742695346
474 10.8152456448042
475 10.8047725168684
476 10.8032512933453
477 10.8110217533512
478 10.8012649549333
479 10.7934518633628
480 10.8038187693056
481 10.7934187624241
482 10.7823235463821
483 10.7746704582436
484 10.7682816705956
485 10.7817124133184
486 10.7801307343148
487 10.7751038676683
488 10.7946445431498
489 10.7837117795263
490 10.7914199184689
491 10.7818965774031
492 10.7724835090064
493 10.7619472980592
494 10.7623195152954
495 10.752893370871
496 10.7481529204125
497 10.7373347643264
498 10.7424659266058
499 10.7340558476254
500 10.7441749799601
501 10.7410477186909
502 10.7459069546279
503 10.73561188422
504 10.7340815068133
505 10.7443701560396
506 10.7395597607523
507 10.7293080588686
508 10.7423687528391
509 10.7450313718891
510 10.7418565312423
511 10.7422913614723
512 10.7426818068089
513 10.7336058491809
514 10.7274967919576
515 10.7302875357762
516 10.7230591066529
517 10.7148649574969
518 10.7116614755561
519 10.7162644972977
520 10.7165398833302
521 10.707792510637
522 10.7086237723218
523 10.7039143279002
524 10.6941027530446
525 10.6869745458343
526 10.681346224413
527 10.6798465523569
528 10.6704478222029
529 10.677306504752
530 10.6740911957463
531 10.6888866260022
532 10.687218150907
533 10.6811849506234
534 10.6832415841615
535 10.6741943988018
536 10.6783572375907
537 10.6766027528754
538 10.6668268144628
539 10.6593822545252
540 10.6736660420123
541 10.6698078854121
542 10.6638122540998
543 10.65414188921
544 10.6535408191499
545 10.6598689166151
546 10.655140594683
547 10.6742228092998
548 10.6724814101076
549 10.6686795755669
550 10.6617220633147
551 10.6671438791124
552 10.6858715676575
553 10.6763425782526
554 10.6875767084289
555 10.6803090260902
556 10.6958192124702
557 10.7069701200367
558 10.7062167833677
559 10.7160786637645
560 10.7066347682745
561 10.6972157909718
562 10.7152955474034
563 10.7066254417134
564 10.6975295500372
565 10.6902889292364
566 10.7070095528774
567 10.7207155861257
568 10.7356491531964
569 10.7440694194602
570 10.7348926105389
571 10.7483417752471
572 10.7389484046071
573 10.7407296220866
574 10.7479148451199
575 10.7394291104391
576 10.7443021557764
577 10.7398373544977
578 10.7457365794701
579 10.7375649470641
580 10.7330779622897
581 10.7239722852535
582 10.7147949551168
583 10.7091939937821
584 10.7163630379489
585 10.7103464089029
586 10.7234502323221
587 10.7391418531878
588 10.7342584600829
589 10.7341517168702
590 10.7397648074042
591 10.745306336968
592 10.7431407227813
593 10.7525837193536
594 10.7670276681948
595 10.7581104047911
596 10.7673311259427
597 10.7826709196141
598 10.7951992984578
599 10.7862178847996
600 10.7772588506231
601 10.7750226299936
602 10.7841530742132
603 10.7757656718081
604 10.7668783488315
605 10.7625113632238
606 10.7640683966397
607 10.778168120987
608 10.7828042960065
609 10.7993711187036
610 10.7918924299603
611 10.7865048333141
612 10.787288533061
613 10.7819312701412
614 10.7956002409394
615 10.7868644153739
616 10.7831978072483
617 10.7808338737219
618 10.7843816459484
619 10.7785651118059
620 10.7802503740382
621 10.7719389874886
622 10.7633889535061
623 10.7547509590048
624 10.750016723226
625 10.7485308689141
626 10.7401748608296
627 10.7524430812671
628 10.7508955601403
629 10.7433709637288
630 10.7392378467398
631 10.733125215601
632 10.7270208737855
633 10.7186709593569
634 10.7102214614845
635 10.7213067996788
636 10.7333298829843
637 10.7467739365587
638 10.7626768007834
639 10.7642702690963
640 10.7686247408197
641 10.763049946506
642 10.7595745322238
643 10.761205793756
644 10.7730613985137
645 10.7654772988599
646 10.7717571118741
647 10.7717166650355
648 10.7811142538322
649 10.7729233873578
650 10.7791832034039
651 10.772194945983
652 10.7667150592567
653 10.7604240571611
654 10.7549573471329
655 10.7527839340888
656 10.7478450588332
657 10.7406703768264
658 10.7337713259937
659 10.7261681775865
660 10.7285886669313
661 10.7205140745983
662 10.7124171075378
663 10.7055829420853
664 10.6977448978399
665 10.7071992393921
666 10.7004122886677
667 10.6956004754993
668 10.7096376507427
669 10.7188886098452
670 10.7116373987816
671 10.7037665728636
672 10.7023618132079
673 10.7151319419604
674 10.7075357624312
675 10.6996387513855
676 10.7067089957837
677 10.7148947313065
678 10.7142335396007
679 10.7233654698264
680 10.7170090047046
681 10.7306134242142
682 10.72348942567
683 10.7203353938731
684 10.7197061070685
685 10.7221233850569
686 10.7145123039843
687 10.706818842929
688 10.7109473358141
689 10.7050283058754
690 10.6972717834262
691 10.6897405569005
692 10.6997943555735
693 10.6927651288832
694 10.6855851782452
695 10.7001752715109
696 10.6925792217251
697 10.6852245238139
698 10.6787360703747
699 10.6767224745711
700 10.6838114233925
701 10.698211858645
702 10.6911287033996
703 10.6876108768085
704 10.6806839301991
705 10.6733301202739
706 10.6709417684876
707 10.6829184061844
708 10.6809056426766
709 10.6801541075388
710 10.6769812673033
711 10.6705849236668
712 10.6761736711726
713 10.6719831653393
714 10.6774986818564
715 10.6913549727162
716 10.6917311248981
717 10.7052518462297
718 10.6994686555429
719 10.6962527091277
720 10.7032993643993
721 10.7086982544664
722 10.701553459443
723 10.7035312763263
724 10.7069630734043
725 10.7072821075564
726 10.7001812807324
727 10.6938121816517
728 10.6929768377632
729 10.7065274182846
730 10.7099311509469
731 10.7209161808558
732 10.7135910215885
733 10.7063347474927
734 10.7052970282691
735 10.7021686368417
736 10.6972405665673
737 10.7025235885801
738 10.6958596897094
739 10.7067693810236
740 10.703640673477
741 10.699448878891
742 10.692470968815
743 10.7035897595717
744 10.6969776555398
745 10.7079981448064
746 10.7030875379014
747 10.6964780211874
748 10.6923688632984
749 10.6858007818213
750 10.6786745744342
751 10.6745941883965
752 10.6875686649159
753 10.6826621904565
754 10.6955112438833
755 10.6947221425777
756 10.6927355291571
757 10.7013369186498
758 10.6945016634109
759 10.6961867652523
760 10.7066829228537
761 10.6999229178696
762 10.6991769569386
763 10.6952562261583
764 10.7021875718704
765 10.7129796835761
766 10.7060526890166
767 10.7013027953337
768 10.6945885938918
769 10.6886271077001
770 10.6877048477743
771 10.6822552785914
772 10.6909502549346
773 10.6912947961803
774 10.6844530013791
775 10.6804705512758
776 10.6785348800578
777 10.6788807509284
778 10.6768179846964
779 10.6854751005165
780 10.6825113457559
781 10.6766076155838
782 10.6727520198598
783 10.6745982814783
784 10.6706933287487
785 10.6648331858983
786 10.6665725817967
787 10.66828564795
788 10.6716610079745
789 10.6707673966793
790 10.6645223302597
791 10.6709775141876
792 10.6644962322304
793 10.6583321345185
794 10.6650383454532
795 10.6583950959383
796 10.6564038324539
797 10.6512295858484
798 10.6578882538484
799 10.6571374080499
800 10.6589586000463
801 10.6637373120872
802 10.6701617450009
803 10.6822796989181
804 10.6765296881072
805 10.6885458389955
806 10.7004726021771
807 10.6986595720799
808 10.6921089594424
809 10.6954286838536
810 10.6933947251732
811 10.6897112216401
812 10.6831735019924
813 10.6911343161813
814 10.6926652358803
815 10.7049449455392
816 10.7051617780514
817 10.700752297198
818 10.7009665737648
819 10.7028203267161
820 10.7000430873056
821 10.701859587587
822 10.7097122146877
823 10.7069336552083
824 10.7004353471233
825 10.6954215722381
826 10.6898941931311
827 10.6980538642058
828 10.6929619918682
829 10.6947298971968
830 10.6927808586414
831 10.6864100450557
832 10.6927459069676
833 10.6890320817467
834 10.6970740549521
835 10.6951539112809
836 10.6932272959131
837 10.6868377898364
838 10.6931147389917
839 10.6991876109816
840 10.6948617582094
841 10.6978042078559
842 10.6969157243763
843 10.6905696831157
844 10.7003867263958
845 10.7019240083165
846 10.7095796807294
847 10.7060299747536
848 10.7010432412309
849 10.7003422610806
850 10.7049362219292
851 10.714363112934
852 10.7089872732038
853 10.7082660880818
854 10.7075335450774
855 10.7153344955629
856 10.7266351959441
857 10.7212191179817
858 10.7151730690922
859 10.7103277300818
860 10.71481917341
861 10.7130248007687
862 10.7103008074654
863 10.7054584446571
864 10.7012181430206
865 10.7105996456448
866 10.7052893805951
867 10.6993235200688
868 10.7008668632801
869 10.7120228092866
870 10.7071714107649
871 10.7064542556266
872 10.7141055481624
873 10.709323889105
874 10.7185535451368
875 10.7149982848761
876 10.7115343708844
877 10.705649808726
878 10.7021189642557
879 10.7036023962355
880 10.7109707964728
881 10.7092647754202
882 10.703192375853
883 10.7122577773615
884 10.7087102728053
885 10.7101147585364
886 10.7177442892569
887 10.7204426755578
888 10.7218155030952
889 10.7162969866747
890 10.7179762912985
891 10.7121935148159
892 10.7066930518483
893 10.7041307175771
894 10.6993974426545
895 10.6939208816027
896 10.6888280494799
897 10.684122243571
898 10.680109863791
899 10.6911934805281
900 10.689527065726
901 10.6837879876848
902 10.6780852941037
903 10.6837276591878
904 10.6880691258217
905 10.6938799509114
906 10.6981498546893
907 10.6935216953916
908 10.7007372783186
909 10.7095909025745
910 10.7037051730619
911 10.6986792607524
912 10.6936034870497
913 10.6993688247349
914 10.7036068345381
915 10.7039752218734
916 10.7042630970394
917 10.7099370442951
918 10.7206036622625
919 10.7148205959019
920 10.7122432295692
921 10.7089046643818
922 10.7160639691201
923 10.7104583771261
924 10.714542991795
925 10.7148239305494
926 10.7095006262559
927 10.7098463847598
928 10.7154109885118
929 10.7096921375217
930 10.7111604535209
931 10.7138833582172
932 10.7105897873292
933 10.7160911275088
934 10.7247418677168
935 10.7221652259515
936 10.7214591301358
937 10.7161739853246
938 10.7117216802802
939 10.7220447284345
940 10.7323019063969
941 10.7424936892314
942 10.7386211639304
943 10.7333947805462
944 10.7374807996273
945 10.7442971670899
946 10.7434526471287
947 10.7379937444526
948 10.7355410455509
949 10.7310700163341
950 10.729314647161
951 10.7306103410096
952 10.7307747151296
953 10.7390585942551
954 10.7347160370836
955 10.7417729309181
956 10.7458447662923
957 10.7432610169149
958 10.7378261115569
959 10.7340381519866
960 10.7296083545817
961 10.7247579810919
962 10.7262874542725
963 10.7317435430453
964 10.7386667716709
965 10.7343400356754
966 10.7335401250686
967 10.7279890322618
968 10.7362250780067
969 10.7330988460067
970 10.734571366633
971 10.7294945572165
972 10.7252045272305
973 10.723624205501
974 10.7229526466225
975 10.7268747520463
976 10.7237359373797
977 10.7199511638941
978 10.7299495411363
979 10.7292331650048
980 10.7391475881678
981 10.7384316557421
982 10.7436101943649
983 10.7398602288027
984 10.7374153334418
985 10.7321571260448
986 10.7387629602193
987 10.7379991520185
988 10.746090696966
989 10.7410984729762
990 10.7404300219983
991 10.7469439960081
992 10.7427300526359
993 10.7377591199835
994 10.7390290612944
995 10.7382396679758
996 10.7406417076536
997 10.7398311786879
998 10.7383140346927
999 10.7481827926051
1000 10.7458410559621
};
\addplot [semithick, color0, dashed]
table {%
1 10.6770782520313
2 10.6770782520313
3 10.6770782520313
4 10.6770782520313
5 10.6770782520313
6 10.6770782520313
7 10.6770782520313
8 10.6770782520313
9 10.6770782520313
10 10.6770782520313
11 10.6770782520313
12 10.6770782520313
13 10.6770782520313
14 10.6770782520313
15 10.6770782520313
16 10.6770782520313
17 10.6770782520313
18 10.6770782520313
19 10.6770782520313
20 10.6770782520313
21 10.6770782520313
22 10.6770782520313
23 10.6770782520313
24 10.6770782520313
25 10.6770782520313
26 10.6770782520313
27 10.6770782520313
28 10.6770782520313
29 10.6770782520313
30 10.6770782520313
31 10.6770782520313
32 10.6770782520313
33 10.6770782520313
34 10.6770782520313
35 10.6770782520313
36 10.6770782520313
37 10.6770782520313
38 10.6770782520313
39 10.6770782520313
40 10.6770782520313
41 10.6770782520313
42 10.6770782520313
43 10.6770782520313
44 10.6770782520313
45 10.6770782520313
46 10.6770782520313
47 10.6770782520313
48 10.6770782520313
49 10.6770782520313
50 10.6770782520313
51 10.6770782520313
52 10.6770782520313
53 10.6770782520313
54 10.6770782520313
55 10.6770782520313
56 10.6770782520313
57 10.6770782520313
58 10.6770782520313
59 10.6770782520313
60 10.6770782520313
61 10.6770782520313
62 10.6770782520313
63 10.6770782520313
64 10.6770782520313
65 10.6770782520313
66 10.6770782520313
67 10.6770782520313
68 10.6770782520313
69 10.6770782520313
70 10.6770782520313
71 10.6770782520313
72 10.6770782520313
73 10.6770782520313
74 10.6770782520313
75 10.6770782520313
76 10.6770782520313
77 10.6770782520313
78 10.6770782520313
79 10.6770782520313
80 10.6770782520313
81 10.6770782520313
82 10.6770782520313
83 10.6770782520313
84 10.6770782520313
85 10.6770782520313
86 10.6770782520313
87 10.6770782520313
88 10.6770782520313
89 10.6770782520313
90 10.6770782520313
91 10.6770782520313
92 10.6770782520313
93 10.6770782520313
94 10.6770782520313
95 10.6770782520313
96 10.6770782520313
97 10.6770782520313
98 10.6770782520313
99 10.6770782520313
100 10.6770782520313
101 10.6770782520313
102 10.6770782520313
103 10.6770782520313
104 10.6770782520313
105 10.6770782520313
106 10.6770782520313
107 10.6770782520313
108 10.6770782520313
109 10.6770782520313
110 10.6770782520313
111 10.6770782520313
112 10.6770782520313
113 10.6770782520313
114 10.6770782520313
115 10.6770782520313
116 10.6770782520313
117 10.6770782520313
118 10.6770782520313
119 10.6770782520313
120 10.6770782520313
121 10.6770782520313
122 10.6770782520313
123 10.6770782520313
124 10.6770782520313
125 10.6770782520313
126 10.6770782520313
127 10.6770782520313
128 10.6770782520313
129 10.6770782520313
130 10.6770782520313
131 10.6770782520313
132 10.6770782520313
133 10.6770782520313
134 10.6770782520313
135 10.6770782520313
136 10.6770782520313
137 10.6770782520313
138 10.6770782520313
139 10.6770782520313
140 10.6770782520313
141 10.6770782520313
142 10.6770782520313
143 10.6770782520313
144 10.6770782520313
145 10.6770782520313
146 10.6770782520313
147 10.6770782520313
148 10.6770782520313
149 10.6770782520313
150 10.6770782520313
151 10.6770782520313
152 10.6770782520313
153 10.6770782520313
154 10.6770782520313
155 10.6770782520313
156 10.6770782520313
157 10.6770782520313
158 10.6770782520313
159 10.6770782520313
160 10.6770782520313
161 10.6770782520313
162 10.6770782520313
163 10.6770782520313
164 10.6770782520313
165 10.6770782520313
166 10.6770782520313
167 10.6770782520313
168 10.6770782520313
169 10.6770782520313
170 10.6770782520313
171 10.6770782520313
172 10.6770782520313
173 10.6770782520313
174 10.6770782520313
175 10.6770782520313
176 10.6770782520313
177 10.6770782520313
178 10.6770782520313
179 10.6770782520313
180 10.6770782520313
181 10.6770782520313
182 10.6770782520313
183 10.6770782520313
184 10.6770782520313
185 10.6770782520313
186 10.6770782520313
187 10.6770782520313
188 10.6770782520313
189 10.6770782520313
190 10.6770782520313
191 10.6770782520313
192 10.6770782520313
193 10.6770782520313
194 10.6770782520313
195 10.6770782520313
196 10.6770782520313
197 10.6770782520313
198 10.6770782520313
199 10.6770782520313
200 10.6770782520313
201 10.6770782520313
202 10.6770782520313
203 10.6770782520313
204 10.6770782520313
205 10.6770782520313
206 10.6770782520313
207 10.6770782520313
208 10.6770782520313
209 10.6770782520313
210 10.6770782520313
211 10.6770782520313
212 10.6770782520313
213 10.6770782520313
214 10.6770782520313
215 10.6770782520313
216 10.6770782520313
217 10.6770782520313
218 10.6770782520313
219 10.6770782520313
220 10.6770782520313
221 10.6770782520313
222 10.6770782520313
223 10.6770782520313
224 10.6770782520313
225 10.6770782520313
226 10.6770782520313
227 10.6770782520313
228 10.6770782520313
229 10.6770782520313
230 10.6770782520313
231 10.6770782520313
232 10.6770782520313
233 10.6770782520313
234 10.6770782520313
235 10.6770782520313
236 10.6770782520313
237 10.6770782520313
238 10.6770782520313
239 10.6770782520313
240 10.6770782520313
241 10.6770782520313
242 10.6770782520313
243 10.6770782520313
244 10.6770782520313
245 10.6770782520313
246 10.6770782520313
247 10.6770782520313
248 10.6770782520313
249 10.6770782520313
250 10.6770782520313
251 10.6770782520313
252 10.6770782520313
253 10.6770782520313
254 10.6770782520313
255 10.6770782520313
256 10.6770782520313
257 10.6770782520313
258 10.6770782520313
259 10.6770782520313
260 10.6770782520313
261 10.6770782520313
262 10.6770782520313
263 10.6770782520313
264 10.6770782520313
265 10.6770782520313
266 10.6770782520313
267 10.6770782520313
268 10.6770782520313
269 10.6770782520313
270 10.6770782520313
271 10.6770782520313
272 10.6770782520313
273 10.6770782520313
274 10.6770782520313
275 10.6770782520313
276 10.6770782520313
277 10.6770782520313
278 10.6770782520313
279 10.6770782520313
280 10.6770782520313
281 10.6770782520313
282 10.6770782520313
283 10.6770782520313
284 10.6770782520313
285 10.6770782520313
286 10.6770782520313
287 10.6770782520313
288 10.6770782520313
289 10.6770782520313
290 10.6770782520313
291 10.6770782520313
292 10.6770782520313
293 10.6770782520313
294 10.6770782520313
295 10.6770782520313
296 10.6770782520313
297 10.6770782520313
298 10.6770782520313
299 10.6770782520313
300 10.6770782520313
301 10.6770782520313
302 10.6770782520313
303 10.6770782520313
304 10.6770782520313
305 10.6770782520313
306 10.6770782520313
307 10.6770782520313
308 10.6770782520313
309 10.6770782520313
310 10.6770782520313
311 10.6770782520313
312 10.6770782520313
313 10.6770782520313
314 10.6770782520313
315 10.6770782520313
316 10.6770782520313
317 10.6770782520313
318 10.6770782520313
319 10.6770782520313
320 10.6770782520313
321 10.6770782520313
322 10.6770782520313
323 10.6770782520313
324 10.6770782520313
325 10.6770782520313
326 10.6770782520313
327 10.6770782520313
328 10.6770782520313
329 10.6770782520313
330 10.6770782520313
331 10.6770782520313
332 10.6770782520313
333 10.6770782520313
334 10.6770782520313
335 10.6770782520313
336 10.6770782520313
337 10.6770782520313
338 10.6770782520313
339 10.6770782520313
340 10.6770782520313
341 10.6770782520313
342 10.6770782520313
343 10.6770782520313
344 10.6770782520313
345 10.6770782520313
346 10.6770782520313
347 10.6770782520313
348 10.6770782520313
349 10.6770782520313
350 10.6770782520313
351 10.6770782520313
352 10.6770782520313
353 10.6770782520313
354 10.6770782520313
355 10.6770782520313
356 10.6770782520313
357 10.6770782520313
358 10.6770782520313
359 10.6770782520313
360 10.6770782520313
361 10.6770782520313
362 10.6770782520313
363 10.6770782520313
364 10.6770782520313
365 10.6770782520313
366 10.6770782520313
367 10.6770782520313
368 10.6770782520313
369 10.6770782520313
370 10.6770782520313
371 10.6770782520313
372 10.6770782520313
373 10.6770782520313
374 10.6770782520313
375 10.6770782520313
376 10.6770782520313
377 10.6770782520313
378 10.6770782520313
379 10.6770782520313
380 10.6770782520313
381 10.6770782520313
382 10.6770782520313
383 10.6770782520313
384 10.6770782520313
385 10.6770782520313
386 10.6770782520313
387 10.6770782520313
388 10.6770782520313
389 10.6770782520313
390 10.6770782520313
391 10.6770782520313
392 10.6770782520313
393 10.6770782520313
394 10.6770782520313
395 10.6770782520313
396 10.6770782520313
397 10.6770782520313
398 10.6770782520313
399 10.6770782520313
400 10.6770782520313
401 10.6770782520313
402 10.6770782520313
403 10.6770782520313
404 10.6770782520313
405 10.6770782520313
406 10.6770782520313
407 10.6770782520313
408 10.6770782520313
409 10.6770782520313
410 10.6770782520313
411 10.6770782520313
412 10.6770782520313
413 10.6770782520313
414 10.6770782520313
415 10.6770782520313
416 10.6770782520313
417 10.6770782520313
418 10.6770782520313
419 10.6770782520313
420 10.6770782520313
421 10.6770782520313
422 10.6770782520313
423 10.6770782520313
424 10.6770782520313
425 10.6770782520313
426 10.6770782520313
427 10.6770782520313
428 10.6770782520313
429 10.6770782520313
430 10.6770782520313
431 10.6770782520313
432 10.6770782520313
433 10.6770782520313
434 10.6770782520313
435 10.6770782520313
436 10.6770782520313
437 10.6770782520313
438 10.6770782520313
439 10.6770782520313
440 10.6770782520313
441 10.6770782520313
442 10.6770782520313
443 10.6770782520313
444 10.6770782520313
445 10.6770782520313
446 10.6770782520313
447 10.6770782520313
448 10.6770782520313
449 10.6770782520313
450 10.6770782520313
451 10.6770782520313
452 10.6770782520313
453 10.6770782520313
454 10.6770782520313
455 10.6770782520313
456 10.6770782520313
457 10.6770782520313
458 10.6770782520313
459 10.6770782520313
460 10.6770782520313
461 10.6770782520313
462 10.6770782520313
463 10.6770782520313
464 10.6770782520313
465 10.6770782520313
466 10.6770782520313
467 10.6770782520313
468 10.6770782520313
469 10.6770782520313
470 10.6770782520313
471 10.6770782520313
472 10.6770782520313
473 10.6770782520313
474 10.6770782520313
475 10.6770782520313
476 10.6770782520313
477 10.6770782520313
478 10.6770782520313
479 10.6770782520313
480 10.6770782520313
481 10.6770782520313
482 10.6770782520313
483 10.6770782520313
484 10.6770782520313
485 10.6770782520313
486 10.6770782520313
487 10.6770782520313
488 10.6770782520313
489 10.6770782520313
490 10.6770782520313
491 10.6770782520313
492 10.6770782520313
493 10.6770782520313
494 10.6770782520313
495 10.6770782520313
496 10.6770782520313
497 10.6770782520313
498 10.6770782520313
499 10.6770782520313
500 10.6770782520313
501 10.6770782520313
502 10.6770782520313
503 10.6770782520313
504 10.6770782520313
505 10.6770782520313
506 10.6770782520313
507 10.6770782520313
508 10.6770782520313
509 10.6770782520313
510 10.6770782520313
511 10.6770782520313
512 10.6770782520313
513 10.6770782520313
514 10.6770782520313
515 10.6770782520313
516 10.6770782520313
517 10.6770782520313
518 10.6770782520313
519 10.6770782520313
520 10.6770782520313
521 10.6770782520313
522 10.6770782520313
523 10.6770782520313
524 10.6770782520313
525 10.6770782520313
526 10.6770782520313
527 10.6770782520313
528 10.6770782520313
529 10.6770782520313
530 10.6770782520313
531 10.6770782520313
532 10.6770782520313
533 10.6770782520313
534 10.6770782520313
535 10.6770782520313
536 10.6770782520313
537 10.6770782520313
538 10.6770782520313
539 10.6770782520313
540 10.6770782520313
541 10.6770782520313
542 10.6770782520313
543 10.6770782520313
544 10.6770782520313
545 10.6770782520313
546 10.6770782520313
547 10.6770782520313
548 10.6770782520313
549 10.6770782520313
550 10.6770782520313
551 10.6770782520313
552 10.6770782520313
553 10.6770782520313
554 10.6770782520313
555 10.6770782520313
556 10.6770782520313
557 10.6770782520313
558 10.6770782520313
559 10.6770782520313
560 10.6770782520313
561 10.6770782520313
562 10.6770782520313
563 10.6770782520313
564 10.6770782520313
565 10.6770782520313
566 10.6770782520313
567 10.6770782520313
568 10.6770782520313
569 10.6770782520313
570 10.6770782520313
571 10.6770782520313
572 10.6770782520313
573 10.6770782520313
574 10.6770782520313
575 10.6770782520313
576 10.6770782520313
577 10.6770782520313
578 10.6770782520313
579 10.6770782520313
580 10.6770782520313
581 10.6770782520313
582 10.6770782520313
583 10.6770782520313
584 10.6770782520313
585 10.6770782520313
586 10.6770782520313
587 10.6770782520313
588 10.6770782520313
589 10.6770782520313
590 10.6770782520313
591 10.6770782520313
592 10.6770782520313
593 10.6770782520313
594 10.6770782520313
595 10.6770782520313
596 10.6770782520313
597 10.6770782520313
598 10.6770782520313
599 10.6770782520313
600 10.6770782520313
601 10.6770782520313
602 10.6770782520313
603 10.6770782520313
604 10.6770782520313
605 10.6770782520313
606 10.6770782520313
607 10.6770782520313
608 10.6770782520313
609 10.6770782520313
610 10.6770782520313
611 10.6770782520313
612 10.6770782520313
613 10.6770782520313
614 10.6770782520313
615 10.6770782520313
616 10.6770782520313
617 10.6770782520313
618 10.6770782520313
619 10.6770782520313
620 10.6770782520313
621 10.6770782520313
622 10.6770782520313
623 10.6770782520313
624 10.6770782520313
625 10.6770782520313
626 10.6770782520313
627 10.6770782520313
628 10.6770782520313
629 10.6770782520313
630 10.6770782520313
631 10.6770782520313
632 10.6770782520313
633 10.6770782520313
634 10.6770782520313
635 10.6770782520313
636 10.6770782520313
637 10.6770782520313
638 10.6770782520313
639 10.6770782520313
640 10.6770782520313
641 10.6770782520313
642 10.6770782520313
643 10.6770782520313
644 10.6770782520313
645 10.6770782520313
646 10.6770782520313
647 10.6770782520313
648 10.6770782520313
649 10.6770782520313
650 10.6770782520313
651 10.6770782520313
652 10.6770782520313
653 10.6770782520313
654 10.6770782520313
655 10.6770782520313
656 10.6770782520313
657 10.6770782520313
658 10.6770782520313
659 10.6770782520313
660 10.6770782520313
661 10.6770782520313
662 10.6770782520313
663 10.6770782520313
664 10.6770782520313
665 10.6770782520313
666 10.6770782520313
667 10.6770782520313
668 10.6770782520313
669 10.6770782520313
670 10.6770782520313
671 10.6770782520313
672 10.6770782520313
673 10.6770782520313
674 10.6770782520313
675 10.6770782520313
676 10.6770782520313
677 10.6770782520313
678 10.6770782520313
679 10.6770782520313
680 10.6770782520313
681 10.6770782520313
682 10.6770782520313
683 10.6770782520313
684 10.6770782520313
685 10.6770782520313
686 10.6770782520313
687 10.6770782520313
688 10.6770782520313
689 10.6770782520313
690 10.6770782520313
691 10.6770782520313
692 10.6770782520313
693 10.6770782520313
694 10.6770782520313
695 10.6770782520313
696 10.6770782520313
697 10.6770782520313
698 10.6770782520313
699 10.6770782520313
700 10.6770782520313
701 10.6770782520313
702 10.6770782520313
703 10.6770782520313
704 10.6770782520313
705 10.6770782520313
706 10.6770782520313
707 10.6770782520313
708 10.6770782520313
709 10.6770782520313
710 10.6770782520313
711 10.6770782520313
712 10.6770782520313
713 10.6770782520313
714 10.6770782520313
715 10.6770782520313
716 10.6770782520313
717 10.6770782520313
718 10.6770782520313
719 10.6770782520313
720 10.6770782520313
721 10.6770782520313
722 10.6770782520313
723 10.6770782520313
724 10.6770782520313
725 10.6770782520313
726 10.6770782520313
727 10.6770782520313
728 10.6770782520313
729 10.6770782520313
730 10.6770782520313
731 10.6770782520313
732 10.6770782520313
733 10.6770782520313
734 10.6770782520313
735 10.6770782520313
736 10.6770782520313
737 10.6770782520313
738 10.6770782520313
739 10.6770782520313
740 10.6770782520313
741 10.6770782520313
742 10.6770782520313
743 10.6770782520313
744 10.6770782520313
745 10.6770782520313
746 10.6770782520313
747 10.6770782520313
748 10.6770782520313
749 10.6770782520313
750 10.6770782520313
751 10.6770782520313
752 10.6770782520313
753 10.6770782520313
754 10.6770782520313
755 10.6770782520313
756 10.6770782520313
757 10.6770782520313
758 10.6770782520313
759 10.6770782520313
760 10.6770782520313
761 10.6770782520313
762 10.6770782520313
763 10.6770782520313
764 10.6770782520313
765 10.6770782520313
766 10.6770782520313
767 10.6770782520313
768 10.6770782520313
769 10.6770782520313
770 10.6770782520313
771 10.6770782520313
772 10.6770782520313
773 10.6770782520313
774 10.6770782520313
775 10.6770782520313
776 10.6770782520313
777 10.6770782520313
778 10.6770782520313
779 10.6770782520313
780 10.6770782520313
781 10.6770782520313
782 10.6770782520313
783 10.6770782520313
784 10.6770782520313
785 10.6770782520313
786 10.6770782520313
787 10.6770782520313
788 10.6770782520313
789 10.6770782520313
790 10.6770782520313
791 10.6770782520313
792 10.6770782520313
793 10.6770782520313
794 10.6770782520313
795 10.6770782520313
796 10.6770782520313
797 10.6770782520313
798 10.6770782520313
799 10.6770782520313
800 10.6770782520313
801 10.6770782520313
802 10.6770782520313
803 10.6770782520313
804 10.6770782520313
805 10.6770782520313
806 10.6770782520313
807 10.6770782520313
808 10.6770782520313
809 10.6770782520313
810 10.6770782520313
811 10.6770782520313
812 10.6770782520313
813 10.6770782520313
814 10.6770782520313
815 10.6770782520313
816 10.6770782520313
817 10.6770782520313
818 10.6770782520313
819 10.6770782520313
820 10.6770782520313
821 10.6770782520313
822 10.6770782520313
823 10.6770782520313
824 10.6770782520313
825 10.6770782520313
826 10.6770782520313
827 10.6770782520313
828 10.6770782520313
829 10.6770782520313
830 10.6770782520313
831 10.6770782520313
832 10.6770782520313
833 10.6770782520313
834 10.6770782520313
835 10.6770782520313
836 10.6770782520313
837 10.6770782520313
838 10.6770782520313
839 10.6770782520313
840 10.6770782520313
841 10.6770782520313
842 10.6770782520313
843 10.6770782520313
844 10.6770782520313
845 10.6770782520313
846 10.6770782520313
847 10.6770782520313
848 10.6770782520313
849 10.6770782520313
850 10.6770782520313
851 10.6770782520313
852 10.6770782520313
853 10.6770782520313
854 10.6770782520313
855 10.6770782520313
856 10.6770782520313
857 10.6770782520313
858 10.6770782520313
859 10.6770782520313
860 10.6770782520313
861 10.6770782520313
862 10.6770782520313
863 10.6770782520313
864 10.6770782520313
865 10.6770782520313
866 10.6770782520313
867 10.6770782520313
868 10.6770782520313
869 10.6770782520313
870 10.6770782520313
871 10.6770782520313
872 10.6770782520313
873 10.6770782520313
874 10.6770782520313
875 10.6770782520313
876 10.6770782520313
877 10.6770782520313
878 10.6770782520313
879 10.6770782520313
880 10.6770782520313
881 10.6770782520313
882 10.6770782520313
883 10.6770782520313
884 10.6770782520313
885 10.6770782520313
886 10.6770782520313
887 10.6770782520313
888 10.6770782520313
889 10.6770782520313
890 10.6770782520313
891 10.6770782520313
892 10.6770782520313
893 10.6770782520313
894 10.6770782520313
895 10.6770782520313
896 10.6770782520313
897 10.6770782520313
898 10.6770782520313
899 10.6770782520313
900 10.6770782520313
901 10.6770782520313
902 10.6770782520313
903 10.6770782520313
904 10.6770782520313
905 10.6770782520313
906 10.6770782520313
907 10.6770782520313
908 10.6770782520313
909 10.6770782520313
910 10.6770782520313
911 10.6770782520313
912 10.6770782520313
913 10.6770782520313
914 10.6770782520313
915 10.6770782520313
916 10.6770782520313
917 10.6770782520313
918 10.6770782520313
919 10.6770782520313
920 10.6770782520313
921 10.6770782520313
922 10.6770782520313
923 10.6770782520313
924 10.6770782520313
925 10.6770782520313
926 10.6770782520313
927 10.6770782520313
928 10.6770782520313
929 10.6770782520313
930 10.6770782520313
931 10.6770782520313
932 10.6770782520313
933 10.6770782520313
934 10.6770782520313
935 10.6770782520313
936 10.6770782520313
937 10.6770782520313
938 10.6770782520313
939 10.6770782520313
940 10.6770782520313
941 10.6770782520313
942 10.6770782520313
943 10.6770782520313
944 10.6770782520313
945 10.6770782520313
946 10.6770782520313
947 10.6770782520313
948 10.6770782520313
949 10.6770782520313
950 10.6770782520313
951 10.6770782520313
952 10.6770782520313
953 10.6770782520313
954 10.6770782520313
955 10.6770782520313
956 10.6770782520313
957 10.6770782520313
958 10.6770782520313
959 10.6770782520313
960 10.6770782520313
961 10.6770782520313
962 10.6770782520313
963 10.6770782520313
964 10.6770782520313
965 10.6770782520313
966 10.6770782520313
967 10.6770782520313
968 10.6770782520313
969 10.6770782520313
970 10.6770782520313
971 10.6770782520313
972 10.6770782520313
973 10.6770782520313
974 10.6770782520313
975 10.6770782520313
976 10.6770782520313
977 10.6770782520313
978 10.6770782520313
979 10.6770782520313
980 10.6770782520313
981 10.6770782520313
982 10.6770782520313
983 10.6770782520313
984 10.6770782520313
985 10.6770782520313
986 10.6770782520313
987 10.6770782520313
988 10.6770782520313
989 10.6770782520313
990 10.6770782520313
991 10.6770782520313
992 10.6770782520313
993 10.6770782520313
994 10.6770782520313
995 10.6770782520313
996 10.6770782520313
997 10.6770782520313
998 10.6770782520313
999 10.6770782520313
1000 10.6770782520313
};
\addlegendentry{$v_{d_{e}}$ (valor del desvío esperado)}
\end{axis}

\end{tikzpicture}

    \caption{valor del desvío para 10 corridas del experimento}
  \end{mytikzresize}
\end{figure}

\begin{figure}[!htbp]
  \begin{mytikzresize}{0.6\textwidth}
    \centering
    % This file was created by tikzplotlib v0.9.1.
\begin{tikzpicture}

\definecolor{color0}{rgb}{0.12156862745098,0.466666666666667,0.705882352941177}
\definecolor{color1}{rgb}{1,0.498039215686275,0.0549019607843137}
\definecolor{color2}{rgb}{0.172549019607843,0.627450980392157,0.172549019607843}
\definecolor{color3}{rgb}{0.83921568627451,0.152941176470588,0.156862745098039}
\definecolor{color4}{rgb}{0.580392156862745,0.403921568627451,0.741176470588235}
\definecolor{color5}{rgb}{0.549019607843137,0.337254901960784,0.294117647058824}
\definecolor{color6}{rgb}{0.890196078431372,0.466666666666667,0.76078431372549}
\definecolor{color7}{rgb}{0.737254901960784,0.741176470588235,0.133333333333333}
\definecolor{color8}{rgb}{0.0901960784313725,0.745098039215686,0.811764705882353}

\begin{axis}[
legend cell align={left},
legend style={fill opacity=0.5, draw opacity=1, text opacity=1, draw=white!80!black},
scaled ticks=false,
tick align=outside,
tick pos=left,
width=\figW,
x grid style={white!69.0196078431373!black},
xlabel={\(\displaystyle n\) (número de tiradas)},
xmajorgrids,
xmin=-48.95, xmax=1049.95,
xtick style={color=black},
xticklabel style={/pgf/number format/.cd,fixed,precision=2},
y grid style={white!69.0196078431373!black},
ylabel={\(\displaystyle v_{d}\) (valor de la varianza)},
ymajorgrids,
ymin=-12.8, ymax=268.8,
ytick style={color=black},
yticklabel style={/pgf/number format/.cd,fixed,precision=2}
]
\addplot [semithick, color0, forget plot]
table {%
1 0
2 196
3 174.222222222222
4 154.75
5 144.96
6 157.25
7 134.816326530612
8 126
9 127.432098765432
10 127.89
11 120.198347107438
12 110.305555555556
13 102.390532544379
14 95.454081632653
15 96.3288888888889
16 93.125
17 120.878892733564
18 117.472222222222
19 141.218836565097
20 140.76
21 159.746031746032
22 155.743801652893
23 150.854442344045
24 146.776041666667
25 154.5856
26 150.390532544379
27 148.666666666667
28 154.515306122449
29 155.384066587396
30 164.182222222222
31 163.281997918835
32 158.796875
33 162.855831037649
34 158.06660899654
35 154.599183673469
36 152.045524691358
37 151.601168736304
38 152.98891966759
39 149.096646942801
40 145.3975
41 142.722189173111
42 143.311224489796
43 149.26014061655
44 148.681818181818
45 147.550617283951
46 145.629489603025
47 142.770484382073
48 140.072482638889
49 143.979175343607
50 144.9524
51 142.314494425221
52 139.850591715976
53 143.706657173371
54 141.252743484225
55 139.076363636364
56 137.142857142857
57 137.016312711604
58 134.703032104637
59 133.978741740879
60 136.203055555556
61 133.980112872884
62 134.407127991675
63 136.559838750315
64 139.0302734375
65 137.129940828402
66 135.070018365473
67 133.117398084206
68 133.261894463668
69 135.890359168242
70 138.825306122449
71 138.985518746281
72 141.409722222222
73 142.280728091574
74 140.970964207451
75 140.594488888889
76 141.492209141274
77 140.220610558273
78 140.079717291256
79 138.312770389361
80 139.109375
81 139.111415942692
82 138.945419393218
83 138.947307301495
84 137.354166666667
85 139.000138408304
86 138.432666306111
87 138.108072400581
88 138.605371900826
89 137.177124100492
90 138.004444444444
91 138.177031759449
92 136.794305293006
93 135.519482021043
94 135.07888184699
95 135.597340720222
96 137.33984375
97 135.953023700712
98 134.70179092045
99 136.019793898582
100 136.2811
101 134.932849720616
102 135.169165705498
103 135.347346592516
104 134.151534763314
105 134.300589569161
106 133.284442862229
107 132.989256703642
108 133.070644718793
109 133.191313862469
110 132.121983471074
111 131.161431701972
112 131.731425382653
113 131.460098676482
114 133.448753462604
115 133.177769376181
116 132.103076694411
117 133.690700562495
118 135.057454754381
119 134.232751924299
120 134.521597222222
121 133.544839833345
122 132.973730180059
123 131.923590455417
124 133.435158688866
125 133.878656
126 134.428886369363
127 134.650629301259
128 133.6396484375
129 133.061955411333
130 132.140591715976
131 132.98269331624
132 133.147842056933
133 133.915088473062
134 134.74654711517
135 133.767352537723
136 132.832828719723
137 131.881187063775
138 132.038437303088
139 131.623207908493
140 132.634897959184
141 131.695789950204
142 131.403689744098
143 131.11614259866
144 131.869164737654
145 130.999952437574
146 130.181647588666
147 129.890971354528
148 130.58838568298
149 129.780910769785
150 129.5524
151 131.004780492084
152 130.180055401662
153 131.583237216455
154 131.195985832349
155 130.616607700312
156 131.102399737015
157 130.5497180413
158 130.7277679859
159 131.61884419129
160 131.759375
161 131.011303576251
162 130.993446121018
163 130.360043659904
164 129.765429803688
165 130.576822773186
166 131.689976774568
167 131.862167879809
168 131.247980442177
169 131.518014075137
170 131.36
171 130.637392702028
172 129.896261492699
173 129.409068127903
174 129.670927467301
175 130.535510204082
176 130.569085743802
177 130.223307478694
178 129.648939527837
179 129.106707031616
180 130.044320987654
181 129.696285217179
182 130.721651974399
183 130.023709277673
184 130.003987476371
185 129.378407596786
186 130.37093883686
187 130.171351768709
188 129.539384336804
189 129.793622798914
190 130.035346260388
191 130.407828732765
192 129.850667317708
193 129.7111868775
194 130.335875225848
195 130.099408284024
196 130.452519783424
197 129.790357906671
198 130.748290990715
199 130.511855761218
200 130.590775
201 131.064577609465
202 131.249509851975
203 130.64549006285
204 131.11272106882
205 130.478048780488
206 130.327740597606
207 129.702490139793
208 129.471061390533
209 129.236464366658
210 128.625306122449
211 128.136340154085
212 127.647739409042
213 127.068218387004
214 127.042034238798
215 126.451141157382
216 126.423782578875
217 126.879653422243
218 126.410739836714
219 125.833531410938
220 126.285847107438
221 126.004340615467
222 125.980683386089
223 126.174385167608
224 126.608099489796
225 126.495802469136
226 126.289392278174
227 125.886549321741
228 125.497441520468
229 125.584218455026
230 125.077958412098
231 125.383407357433
232 125.275862068966
233 125.460406343827
234 125.354536489152
235 124.888474422816
236 125.582950301637
237 125.653830404672
238 126.055663441847
239 125.533796677229
240 125.205399305556
241 126.149584201374
242 125.942848849122
243 125.947264136565
244 125.462022977694
245 125.465156184923
246 125.617175622976
247 125.49079643987
248 125.059166883455
249 124.980145481525
250 125.179136
251 124.98827002746
252 124.645549886621
253 124.259432267337
254 123.809923119846
255 123.728350634371
256 123.915771484375
257 124.821329618919
258 124.394702842377
259 123.95298221553
260 123.871301775148
261 123.935790725327
262 124.455800944001
263 123.996270005349
264 123.531206955923
265 123.101231755073
266 122.883599977387
267 122.514413163321
268 122.290529627979
269 121.849172897003
270 121.400768175583
271 121.040399776692
272 120.971764165225
273 121.136041004173
274 120.818530555704
275 121.405699173554
276 122.105899495904
277 121.671102190827
278 122.488289943585
279 122.686347811565
280 123.491517857143
281 123.617254087461
282 123.22418389417
283 123.333017018567
284 123.456791807181
285 123.98325638658
286 123.8636241381
287 123.62517451954
288 124.26384066358
289 124.462949437866
290 125.090951248514
291 125.197954676964
292 125.69585053481
293 125.372712553437
294 125.5695427831
295 125.447905774203
296 126.170745069394
297 126.440975410672
298 126.035043466511
299 125.619467343766
300 125.578888888889
301 125.257050142934
302 124.88628788211
303 124.482959187008
304 124.425802891274
305 124.628325718893
306 124.661081208082
307 124.350221222506
308 124.545285882948
309 124.814989369613
310 124.906347554631
311 124.86791906618
312 124.598115959895
313 124.85753656769
314 124.468832406994
315 124.479778281683
316 124.210743470598
317 124.058454159162
318 124.441052569123
319 124.917365198848
320 125.285
321 125.73507632884
322 125.573637205355
323 125.224300050801
324 125.583095183661
325 125.203446153846
326 125.863750987994
327 125.479860468161
328 125.11745240928
329 124.959562457849
330 124.71015610652
331 124.339974991101
332 124.780474306866
333 124.680265851437
334 125.124242532898
335 124.965613722433
336 125.055236678005
337 125.006278121671
338 125.430167010959
339 125.232063765543
340 125.774429065744
341 126.11448129961
342 125.746528846483
343 125.38058122041
344 125.102994862088
345 125.621373660996
346 126.254368672525
347 125.891137705653
348 126.410787422381
349 126.922422640208
350 126.565681632653
351 126.653663525458
352 126.294477982955
353 126.612636326429
354 126.513230553162
355 126.277294187661
356 126.039854500694
357 126.442875189291
358 126.12199993758
359 126.517485121934
360 126.288510802469
361 126.194888007305
362 126.156039192943
363 125.815313161669
364 126.221916133317
365 125.882049164947
366 125.961897936636
367 125.61919681637
368 126.101717568526
369 126.189834093463
370 126.666975894814
371 126.614947581026
372 127.177094172737
373 127.02026177145
374 127.258743458492
375 126.920632888889
376 126.885779764599
377 126.606519429532
378 126.347330701828
379 126.555398528275
380 126.714542936288
381 126.847955029243
382 126.814423946712
383 127.295939027466
384 127.046440972222
385 127.412879068983
386 127.860915729281
387 127.683005161282
388 127.468354766713
389 127.254221158993
390 126.928967784352
391 126.841203288832
392 127.060853550604
393 127.434738975325
394 127.186999149682
395 126.917186348342
396 127.204519946944
397 126.960249731932
398 127.322851695664
399 127.146261644085
400 126.84359375
401 127.201435314457
402 127.727067399322
403 127.481654341816
404 127.338765807274
405 127.530949550373
406 127.821980635298
407 128.187806748003
408 128.469717176086
409 128.428416855471
410 128.219565734682
411 127.956003102042
412 127.822226411537
413 128.091224079405
414 128.002129571285
415 128.066970532733
416 128.020057091346
417 128.209558971528
418 128.159571667315
419 127.853988072522
420 127.625504535147
421 127.810969245265
422 127.729369286404
423 127.428410151513
424 127.128159487362
425 127.004711418685
426 127.413123498424
427 127.489878187233
428 127.917476417154
429 127.619671703588
430 127.395678745268
431 127.592734750567
432 128.009066358025
433 128.339678594478
434 128.175072734609
435 127.927578279826
436 127.841737858766
437 127.618315014479
438 127.57857947082
439 127.587818660136
440 127.298326446281
441 127.311007244924
442 127.496611453492
443 127.209269856152
444 127.466033601169
445 127.717328620124
446 127.457620302037
447 127.645421377415
448 127.374157963967
449 127.693344775076
450 128.083535802469
451 127.846441266267
452 127.68466598794
453 127.431516161572
454 127.197388848998
455 127.203207342108
456 127.116877308403
457 127.289074881852
458 127.479987033047
459 127.496670321481
460 127.308015122873
461 127.311870356341
462 127.037218193062
463 126.777966963507
464 126.53328905321
465 126.767737310672
466 126.501667004366
467 126.277290463985
468 126.013309043758
469 125.830369929215
470 125.602539610684
471 125.375507683431
472 125.149270145073
473 124.885696534647
474 124.81995406719
475 124.568571745152
476 124.330039368689
477 124.1928721174
478 124.170585248858
479 124.559943514891
480 124.365815972222
481 124.10759808265
482 124.033182107746
483 123.780341121956
484 123.536285089816
485 123.371852481667
486 123.207725787058
487 122.998030939963
488 123.012345471647
489 123.258116183857
490 123.118738025823
491 122.868322265131
492 122.940346354683
493 122.691630082823
494 122.924371813995
495 122.715086215692
496 122.787521949792
497 123.081685282723
498 122.835292979145
499 122.765306163429
500 122.577264
501 122.703431460432
502 122.883751845209
503 123.061669742974
504 122.842576845553
505 122.74384864229
506 122.973019419144
507 122.993763834911
508 123.420965341931
509 123.178835962498
510 123.110976547482
511 123.474917758434
512 123.585678100586
513 123.369842192659
514 123.437254916804
515 123.222109529645
516 123.513190313082
517 123.68125886213
518 123.454450589586
519 123.355964671946
520 123.330469674556
521 123.09816129472
522 122.97030651341
523 122.758559421487
524 123.108064798089
525 122.873999092971
526 122.721251572236
527 122.624731302817
528 122.393021120294
529 122.613183915152
530 122.486094695621
531 122.500842315072
532 122.294826587145
533 122.237545276304
534 122.405448245872
535 122.627373569744
536 122.737270967922
537 122.712108444389
538 122.586752532442
539 122.645323401751
540 122.915624142661
541 122.712680358479
542 122.489784316662
543 122.269039406612
544 122.214042901168
545 121.994492046124
546 122.109101826135
547 122.019992714123
548 122.238198625393
549 122.574218400072
550 122.845292561983
551 122.752224136284
552 122.565686699223
553 122.35482932157
554 122.418238215016
555 122.299858777697
556 122.24275399824
557 122.454241593043
558 122.334862090672
559 122.151631619202
560 121.937292729592
561 121.7422923796
562 121.537113258444
563 121.448324599567
564 121.711998893416
565 121.822307150129
566 121.838133201813
567 121.94764984183
568 121.928573199762
569 122.253514166314
570 122.515878731918
571 122.322388902009
572 122.119150080689
573 122.09669569243
574 122.475555123894
575 122.799649149338
576 122.68546248071
577 122.526034079338
578 122.675399001449
579 122.995200467723
580 122.939464922711
581 122.955643572569
582 122.933704136701
583 122.757700308042
584 122.958408589792
585 123.013911900066
586 122.804345420448
587 122.689673185922
588 122.533446712018
589 122.360462468401
590 122.506868715886
591 122.33281512593
592 122.351987650657
593 122.149717473958
594 122.037365801676
595 121.951715274345
596 121.796368969866
597 121.939805111543
598 121.808827641749
599 122.064665371613
600 122.310330555556
601 122.460823751872
602 122.479553205815
603 122.276764326516
604 122.474277443972
605 122.666768663343
606 122.483895914344
607 122.316326004413
608 122.265016339162
609 122.113162982196
610 122.25930932545
611 122.274444780765
612 122.571860715964
613 122.811541132983
614 122.682288406243
615 122.483082821072
616 122.315810001687
617 122.164853725745
618 122.10983336999
619 121.912799058359
620 121.781737773153
621 121.873063289432
622 121.728375947312
623 121.554235833962
624 121.841870069034
625 121.64762624
626 121.572977166246
627 121.63383520422
628 121.693689399164
629 121.546361474165
630 121.422816830436
631 121.347585524449
632 121.158708540298
633 121.217512834143
634 121.237312044104
635 121.426910533821
636 121.661709880938
637 121.717908465161
638 121.594031603463
639 121.54392255113
640 121.527419433594
641 121.385403559668
642 121.671635562543
643 121.655082005278
644 121.466610084487
645 121.297556637221
646 121.3521575976
647 121.30377989962
648 121.285884392623
649 121.568329609854
650 121.444991715976
651 121.72435647863
652 121.585644924536
653 121.447211480058
654 121.309055541528
655 121.23585805023
656 121.05474559414
657 121.069512497423
658 121.055175488031
659 120.889433339243
660 120.817196969697
661 120.64424003424
662 120.921479814898
663 121.010626318052
664 120.910888282044
665 120.924068064899
666 120.936623560497
667 120.755918892128
668 120.744278658252
669 120.595668523397
670 120.415896636222
671 120.404112464214
672 120.233504730017
673 120.083540687392
674 119.991152955472
675 119.813917146776
676 120.147532912013
677 120.283145949744
678 120.559245046597
679 120.487388323381
680 120.620413062284
681 120.487859048087
682 120.539772189782
683 120.592579889344
684 120.445904295339
685 120.43082955938
686 120.360098683372
687 120.450830457085
688 120.319556179016
689 120.249279050221
690 120.378147448015
691 120.307174526316
692 120.398651809282
693 120.227985898982
694 120.058170485595
695 119.903433569691
696 119.812621796142
697 119.684448003228
698 119.814451441285
699 119.723492174596
700 119.561328571429
701 119.778120109646
702 119.907955292571
703 119.959790291152
704 119.806818181818
705 119.665562094462
706 119.499273728222
707 119.358798357104
708 119.193431006416
709 119.025656430221
710 118.858379289823
711 118.947497730065
712 118.859550561798
713 118.819247436409
714 118.680988081507
715 118.523867181769
716 118.375573483974
717 118.636461779964
718 118.49984093854
719 118.587057824478
720 118.797760416667
721 118.968796228077
722 119.225220800945
723 119.353665092237
724 119.442177818137
725 119.319621403092
726 119.369269327384
727 119.233438467899
728 119.169648517691
729 119.221113914809
730 119.08538937887
731 118.964033677607
732 118.801845382066
733 118.656536798632
734 118.497954547142
735 118.339784349114
736 118.206514354915
737 118.33090990097
738 118.348831530321
739 118.436353848323
740 118.397596785975
741 118.314281499451
742 118.302584259051
743 118.553276973602
744 118.678481616372
745 118.764945723166
746 118.88806251752
747 118.876226010692
748 118.926470588235
749 119.176022859139
750 119.020024888889
751 119.269969379487
752 119.127615366116
753 118.969734166477
754 119.217916470249
755 119.464367352309
756 119.664287114023
757 119.532207542461
758 119.429041847383
759 119.280417163559
760 119.177768351801
761 119.192303508248
762 119.432900365801
763 119.519108712419
764 119.400783969738
765 119.420592079969
766 119.339841433236
767 119.28015312202
768 119.141423543294
769 119.220858325118
770 119.459438353854
771 119.510356788983
772 119.364229576633
773 119.609418120292
774 119.852941863804
775 120.094811238293
776 120.082447656499
777 119.953989289897
778 120.108188883235
779 119.969896562691
780 119.831918145957
781 120.026142655022
782 120.014337949124
783 120.298511309125
784 120.313409451791
785 120.160486835166
786 120.441621506128
787 120.296249893036
788 120.234191811178
789 120.084443737641
790 120.131799391123
791 119.987555959027
792 119.92620650954
793 119.970018239673
794 119.959489940295
795 119.878175705075
796 119.953970543673
797 119.872379012262
798 119.736686641416
799 119.785207103372
800 119.895
801 119.833298264809
802 119.698187511272
803 119.54959375567
804 119.403286737952
805 119.674293430037
806 119.791347770136
807 119.649915777222
808 119.502009606901
809 119.551033567055
810 119.403629019966
811 119.445494366152
812 119.669768739838
813 119.522727994808
814 119.467610731124
815 119.702149121156
816 119.782293769223
817 119.638088417936
818 119.505958536833
819 119.524215033373
820 119.378875669244
821 119.369845454505
822 119.360505798569
823 119.251362336472
824 119.131939862381
825 119.025004958678
826 118.991334885003
827 118.979470091194
828 118.838285315877
829 118.719977416947
830 118.6649862099
831 118.590580701777
832 118.701639931583
833 118.566736178265
834 118.427215062482
835 118.651060991789
836 118.759064295689
837 118.905866517074
838 118.771634930309
839 118.955445284343
840 118.999994331066
841 118.965734976622
842 119.009530808334
843 118.868682007573
844 119.091226612161
845 119.03573964497
846 118.961582527148
847 118.85681947118
848 118.731043075828
849 118.605529126624
850 118.480276816609
851 118.623677680644
852 118.55038131764
853 118.693064544281
854 118.708521409101
855 118.619376902295
856 118.726630055027
857 118.86818826086
858 118.941317423835
859 118.826729513295
860 119.003757436452
861 118.865886707648
862 118.742550912194
863 118.918512988591
864 118.781099965706
865 118.693475892947
866 118.605939548454
867 118.574101789437
868 118.444493406103
869 118.390520696911
870 118.268131853613
871 118.180048270526
872 118.044744497517
873 117.974894014005
874 117.903345569176
875 117.852212244898
876 118.029940993724
877 117.958917164741
878 117.950908307865
879 117.966320710395
880 117.855655991736
881 117.786237649148
882 117.73402543179
883 117.647969895689
884 117.638454372351
885 117.505753774458
886 117.611958532273
887 117.685767995434
888 117.553546790033
889 117.50292602626
890 117.377987627825
891 117.394251783328
892 117.28555269561
893 117.497477581638
894 117.634425876712
895 117.516811585157
896 117.726481534997
897 117.7416235712
898 117.623767987262
899 117.50694196122
900 117.551111111111
901 117.652666109059
902 117.669947542047
903 117.879827178753
904 117.829586058031
905 117.821255761424
906 117.863587854334
907 117.73390352386
908 117.940745745114
909 118.148680654644
910 118.051926095882
911 118.04360897001
912 118.085449369037
913 118.290840950179
914 118.282553423766
915 118.199174654364
916 118.070383859957
917 118.171416203565
918 118.18639079936
919 118.10341940961
920 118.094120982987
921 118.134306170063
922 118.019053411192
923 118.058767808115
924 117.953575645134
925 118.11775485756
926 118.021856005299
927 118.191070009275
928 118.097112812128
929 118.014520746987
930 117.98310671754
931 118.151393521397
932 118.02482547109
933 117.905365834606
934 117.785487805437
935 117.882153907747
936 117.875666593615
937 117.810931821055
938 117.685539709312
939 117.567185084624
940 117.539197600724
941 117.532396516695
942 117.629459838353
943 117.643550906439
944 117.519129515585
945 117.488878810784
946 117.527714332965
947 117.464340790514
948 117.41514447471
949 117.291597499892
950 117.362309141274
951 117.457479591464
952 117.652159760963
953 117.57447900208
954 117.451613860211
955 117.334649817713
956 117.57354125453
957 117.811178480295
958 117.876438823053
959 117.97017444092
960 117.849930555556
961 117.73428000013
962 117.642299263921
963 117.520503704566
964 117.55808590589
965 117.448182770007
966 117.348581587645
967 117.286985516887
968 117.165988277782
969 117.121288317619
970 117.185763630566
971 117.084967666871
972 117.283140908398
973 117.253524503844
974 117.295950356075
975 117.207620249836
976 117.143875134372
977 117.043876089674
978 116.956147724374
979 117.053802998828
980 117.118163265306
981 117.057374519541
982 117.252500197029
983 117.320085398882
984 117.412948641682
985 117.35172769203
986 117.392370468506
987 117.316730464632
988 117.271626932092
989 117.287225200154
990 117.199429650036
991 117.083409616926
992 117.037977204735
993 116.993110890027
994 116.917748948419
995 116.890549228555
996 116.864199770971
997 117.020310681292
998 117.085501664652
999 117.124874624374
1000 117.139879
};
\addplot [semithick, color1, forget plot]
table {%
1 0
2 0
3 72
4 90.75
5 72.64
6 81.8888888888889
7 115.959183673469
8 101.5
9 101.111111111111
10 107
11 115.867768595041
12 118.409722222222
13 130.840236686391
14 130.923469387755
15 136.062222222222
16 134.734375
17 126.955017301038
18 134.237654320988
19 148.354570637119
20 148.3275
21 146.902494331066
22 152.20867768595
23 155.119092627599
24 154.873263888889
25 158.56
26 156.616863905325
27 156.117969821674
28 150.561224489796
29 149.191438763377
30 145.515555555556
31 141.56087408949
32 138.9365234375
33 140.056932966024
34 136.98615916955
35 136.991020408163
36 133.243055555556
37 131.579254930606
38 128.296398891967
39 125.495069033531
40 124.6475
41 122.0999405116
42 122.234126984127
43 127.736073553272
44 129.361570247934
45 127.02024691358
46 124.274102079395
47 128.507922136713
48 125.930555555556
49 126.1599333611
50 124.7664
51 122.445982314494
52 124.051405325444
53 123.140619437522
54 122.911179698217
55 121.447933884298
56 120.532844387755
57 120.248076331179
58 119.911117717004
59 117.878770468256
60 117.878888888889
61 116.919107766729
62 118.43418314256
63 117.010330057949
64 116.08984375
65 118.459171597633
66 117.147842056933
67 117.735798618846
68 116.089100346021
69 115.467338794371
70 113.849183673469
71 112.413013291014
72 114.615547839506
73 113.196847438544
74 111.705624543462
75 113.721955555556
76 112.676592797784
77 111.437679203913
78 112.535831689678
79 111.306200929338
80 111.7225
81 110.343240359701
82 109.962671029149
83 109.043983161562
84 108.28514739229
85 108.600138408305
86 108.843293672255
87 110.998282467961
88 109.745738636364
89 108.551698017927
90 107.534444444444
91 106.900615867649
92 107.044777882798
93 106.400508729333
94 105.913535536442
95 105.086315789474
96 106.696940104167
97 108.198745881603
98 107.1091211995
99 107.064177124783
100 106.1584
101 106.52837957063
102 105.645520953479
103 104.659817136394
104 103.740292159763
105 102.987210884354
106 102.160911356355
107 101.206218883745
108 102.645061728395
109 102.44878377241
110 101.597024793388
111 101.126044963883
112 101.955277423469
113 101.125068525335
114 101.981224992305
115 101.798109640832
116 100.92055588585
117 100.606180144642
118 99.8305084745763
119 98.9999293835181
120 98.1830555555556
121 97.8983675978417
122 99.1900026874496
123 101.020159957697
124 102.270031217482
125 101.485696
126 102.440476190476
127 103.128774257549
128 102.3271484375
129 102.512829757827
130 102.517455621302
131 102.992482955539
132 102.719467401286
133 102.202385663407
134 103.53135442192
135 102.777283950617
136 104.468858131488
137 103.764079066546
138 103.569628229364
139 103.437089177579
140 103.435255102041
141 102.710527639455
142 103.013291013688
143 103.288180351117
144 104.361111111111
145 105.394720570749
146 106.613295177332
147 105.888379841733
148 105.325967859752
149 105.538399171209
150 104.853155555556
151 104.246568132977
152 105.157202216066
153 106.0246059208
154 105.934095125654
155 106.241831425598
156 106.118178829717
157 107.656618929774
158 109.147732735139
159 110.368339859974
160 109.7130859375
161 109.607191080591
162 111.008420972413
163 111.83221047085
164 112.823022010708
165 112.152580348944
166 112.134453476557
167 111.70798522715
168 111.563350340136
169 112.109660025909
170 112.325813148789
171 113.316712834718
172 113.748614115738
173 114.162985732901
174 113.507894041485
175 113.495314285714
176 114.034994834711
177 113.786906699863
178 113.636062365863
179 113.71030866702
180 113.411728395062
181 113.258081255151
182 112.636064484966
183 112.439009824121
184 111.978231332703
185 111.551029948868
186 112.73733957683
187 112.227287025651
188 111.901199637845
189 111.575711766188
190 110.992908587258
191 111.303418217702
192 111.597222222222
193 112.704555827002
194 112.863322350941
195 112.332044707429
196 111.80351936693
197 112.218660619959
198 112.764233241506
199 112.502007525062
200 112.028775
201 111.699314373407
202 112.291760611705
203 112.545075104953
204 112.28738946559
205 111.791695419393
206 111.382599679517
207 111.40185301874
208 112.367765347633
209 112.368169226895
210 112.582970521542
211 112.911479975742
212 112.518578675685
213 112.712204368622
214 112.685758581535
215 112.163764196863
216 112.075788751715
217 112.679861538788
218 112.760815587913
219 112.276891641125
220 111.898574380165
221 111.788333572204
222 111.382375618862
223 111.010959399948
224 111.850426498724
225 112.017777777778
226 112.76544756833
227 113.041510605678
228 112.664512157587
229 112.818901241395
230 112.549054820416
231 112.077247427897
232 113.124479785969
233 113.760246090368
234 114.007542552414
235 113.939990946129
236 114.962008043666
237 114.509996617351
238 114.029041028176
239 113.874161866914
240 113.649375
241 113.292780771681
242 113.845041322314
243 114.259953597859
244 113.79943227627
245 113.957251145356
246 114.04608698526
247 113.958301234244
248 113.522096123829
249 113.681134175255
250 113.2544
251 113.409406199902
252 113.193058704963
253 113.929572403881
254 114.57827515655
255 114.857977700884
256 114.752548217773
257 114.645278505352
258 115.548299381047
259 116.294688510905
260 115.848210059172
261 115.483096255193
262 115.118466289843
263 114.681866153913
264 115.397842056933
265 115.240811676753
266 115.309245859008
267 114.922302178457
268 115.166671307641
269 114.759995024944
270 114.485994513032
271 114.105390721804
272 114.825475778547
273 114.614203866951
274 114.203593691726
275 114.443636363636
276 114.237502625499
277 114.031357113999
278 113.774714041716
279 114.115568916124
280 113.818558673469
281 114.372664986512
282 114.042251395805
283 113.639850666134
284 113.633740825233
285 113.242277623884
286 112.954227590591
287 113.023491847661
288 113.089072145062
289 112.705163970738
290 112.338692033294
291 112.198273520624
292 111.854416869957
293 111.578259502149
294 111.239020778379
295 110.901809824763
296 111.670003195763
297 111.366209797186
298 111.128507724877
299 110.780147873066
300 110.719322222222
301 110.394476882154
302 110.067277750976
303 109.705214085765
304 110.142302198753
305 110.020037624295
306 110.660472467854
307 110.804401107704
308 111.444425704166
309 111.10361223699
310 110.752185223725
311 110.977409249284
312 111.608604536489
313 112.121079116863
314 112.052537628301
315 112.045492567397
316 111.755838407307
317 111.466349550697
318 111.608480677188
319 112.109059462859
320 112.249755859375
321 111.963393212411
322 111.61644998264
323 111.753376338314
324 111.688233500991
325 111.990059171598
326 111.869481350446
327 112.081998335344
328 111.800379238548
329 111.730194658216
330 111.523048668503
331 111.65106196548
332 111.437255044273
333 111.197593990387
334 111.251541826527
335 111.196578302517
336 110.959387400794
337 111.331402055138
338 111.022732047197
339 110.695608287432
340 110.767569204152
341 110.570445730601
342 110.696180021203
343 110.589737269335
344 111.145754461871
345 110.951447174963
346 111.020690634502
347 111.156823825462
348 111.21227209671
349 110.894081329382
350 110.637420408163
351 110.341166061964
352 110.028635072314
353 110.49290179682
354 110.214753104153
355 109.96024598294
356 109.734590013887
357 109.46270272815
358 109.315915545707
359 109.681659825731
360 109.494375
361 109.27488278942
362 109.171728579714
363 108.877414262839
364 109.008173831663
365 108.715736535935
366 108.451999163905
367 108.157132356763
368 108.358223062382
369 108.361307569715
370 108.720204528853
371 109.158288591335
372 108.9762978379
373 109.496452932171
374 109.254918642226
375 109.197155555556
376 109.02415544364
377 109.467089756489
378 109.364043559811
379 109.105659247708
380 109.173095567867
381 109.591308960396
382 110.003652586278
383 109.960351491932
384 109.814208984375
385 110.219666048237
386 109.962019114607
387 110.4472087014
388 110.885289350622
389 111.456651753557
390 111.415049309665
391 111.325239892465
392 111.281835693461
393 111.346049505015
394 111.333633950888
395 111.201320301234
396 111.188985817774
397 111.28891116624
398 111.415627130628
399 111.235582691063
400 110.9759
401 111.442789534891
402 111.210322764288
403 111.246876712497
404 111.015243603568
405 111.387373875934
406 111.178389429494
407 111.266762853986
408 111.399942329873
409 111.763487784028
410 111.791082688876
411 112.312536629549
412 112.10135262513
413 111.947458213392
414 111.736306564914
415 111.487937291334
416 111.339473927515
417 111.157359004882
418 111.082284059431
419 111.013425533006
420 110.759160997732
421 111.268318278502
422 111.12080254262
423 110.895707235831
424 110.643728862585
425 111.062732179931
426 111.563181908352
427 111.978259081112
428 112.311440955542
429 112.501355676181
430 112.391784748513
431 112.282018292322
432 112.765555341221
433 113.163246910485
434 113.362037843233
435 113.161231338354
436 112.936595194007
437 113.119071681791
438 113.253810387607
439 113.585307257642
440 113.513631198347
441 113.25871421887
442 113.386627014189
443 113.241534988713
444 113.096598287477
445 112.923590455751
446 112.992665245631
447 113.386844436437
448 113.243298588967
449 113.710904211785
450 113.779758024691
451 113.90047246572
452 113.968713289999
453 113.825602190937
454 113.594636030197
455 113.399608742905
456 113.159852069868
457 113.339245100527
458 113.718006902996
459 113.684053141954
460 113.44652173913
461 113.371883249185
462 113.26887146043
463 113.031996230798
464 112.788416877229
465 112.765008671523
466 112.630477628986
467 112.746603450885
468 113.203936554898
469 113.383026991148
470 113.352924400181
471 113.249516545634
472 113.06176386096
473 112.82502491854
474 112.589275222988
475 112.560735734072
476 112.86491067015
477 112.798702582967
478 112.732060012955
479 112.664990128181
480 112.5975
481 112.833329731459
482 113.066235085484
483 112.883539301039
484 112.652277849874
485 112.438218726751
486 112.209046723907
487 112.053118240579
488 112.223377452298
489 112.196126647179
490 112.370079133694
491 112.143105429296
492 112.20537956904
493 112.02712210295
494 112.312806307266
495 112.487036016733
496 112.309016486733
497 112.09178612925
498 112.451492717859
499 112.244721908747
500 112.669964
501 112.835845275517
502 112.898402406311
503 113.120260544091
504 113.055079207609
505 112.903534947554
506 112.752304363449
507 112.655089107524
508 113.00387500775
509 113.165743531946
510 113.580392156863
511 113.797572772776
512 113.809921264648
513 114.08928863202
514 113.86732577329
515 113.974723348101
516 113.784879965146
517 113.721095892461
518 113.50155781816
519 113.406536209771
520 113.380236686391
521 113.232113055876
522 113.062774328034
523 112.914813420149
524 112.730300536099
525 112.609393197279
526 112.442528444824
527 112.321414040305
528 112.3388393882
529 112.134126164501
530 112.193310786757
531 112.011618628108
532 112.406280032789
533 112.224260707032
534 112.430150514105
535 112.23559786881
536 112.142010330809
537 112.114298000132
538 111.937815259601
539 111.736549165121
540 111.652191358025
541 111.447213860825
542 111.512533870726
543 111.309422789292
544 111.409912954152
545 111.666842858345
546 111.470836855452
547 111.632618002801
548 111.615749373968
549 111.944684987774
550 111.77133553719
551 111.82627856957
552 111.630273970804
553 111.436340984078
554 111.635626686129
555 111.47732813895
556 111.574297396615
557 111.458048212887
558 111.284634704076
559 111.111747594254
560 111.271603954082
561 111.465552028622
562 111.281357252314
563 111.291148345737
564 111.340161460691
565 111.143187407001
566 111.331933848593
567 111.255383543449
568 111.062190041658
569 111.034380299048
570 110.88784549092
571 110.733251339555
572 110.540646853147
573 110.387038610662
574 110.343797423788
575 110.50404536862
576 110.320719401042
577 110.369868650673
578 110.18430394751
579 109.995006577358
580 109.829714625446
581 109.757477907697
582 109.625984577414
583 109.694506147595
584 109.956370801276
585 109.912589670538
586 109.901434495451
587 109.714251717363
588 109.544599472442
589 109.445337699361
590 109.339442688882
591 109.15447447757
592 109.525748721695
593 109.445543709779
594 109.290027094741
595 109.112226537674
596 108.930416985721
597 109.297767452562
598 109.495755081039
599 109.684120166889
600 109.736863888889
601 109.921988034363
602 109.778459950773
603 110.028470142378
604 109.924367352309
605 109.804865787856
606 109.680456164428
607 109.668827978906
608 109.527524995672
609 109.431014260639
610 109.533267938726
611 109.678700099914
612 109.778311760434
613 109.680229076906
614 109.604603231865
615 109.844764359839
616 110.136363636364
617 109.983750515513
618 110.328829819545
619 110.671106923721
620 110.597336108221
621 110.65274698489
622 110.57840593046
623 110.530163940339
624 110.354741946088
625 110.64157696
626 110.471243454562
627 110.8110671052
628 110.735676396608
629 110.786682876648
630 110.918843537415
631 110.873541105231
632 111.105902499599
633 110.944188635076
634 111.075841634408
635 110.914437348875
636 110.745876349828
637 111.075607057217
638 111.087314884877
639 110.913492080985
640 110.753125
641 110.62125530263
642 110.545591075397
643 110.380709591751
644 110.338432159253
645 110.169278288564
646 110.022124720835
647 110.042765444603
648 109.995025053346
649 110.122245673681
650 110.389784615385
651 110.225407679548
652 110.15041025255
653 110.100814007209
654 110.110924071113
655 109.984112813939
656 110.298871114664
657 110.165653111672
658 110.20839607912
659 110.14426603973
660 110.20391184573
661 110.538550447335
662 110.376511714935
663 110.212176017872
664 110.080962403832
665 110.412717508056
666 110.347040734428
667 110.251658278807
668 110.561841138083
669 110.551741550314
670 110.512354644687
671 110.446916207098
672 110.282596371882
673 110.363536006747
674 110.525442682422
675 110.46106776406
676 110.312635674521
677 110.405865655457
678 110.244065488466
679 110.205114078791
680 110.044980536332
681 110.006028967507
682 109.919206491172
683 109.854706113113
684 109.908183287165
685 109.772143427993
686 110.03706788838
687 110.084069080808
688 110.398424824229
689 110.606785880549
690 110.81347616047
691 110.772738601117
692 110.614362324167
693 110.569296344188
694 110.587148801169
695 110.637718544589
696 110.491129525036
697 110.446916380718
698 110.527477196411
699 110.463056768201
700 110.329083673469
701 110.346621191247
702 110.26044634378
703 110.103607987714
704 110.264002776343
705 110.112839394397
706 109.993365246491
707 110.19611122559
708 110.214082798685
709 110.378884421731
710 110.62712953779
711 110.790997802267
712 110.669612422674
713 110.605548332019
714 110.486392596254
715 110.607528974522
716 110.687689210699
717 110.982187909097
718 111.180445527269
719 111.13595803165
720 111.177775848765
721 111.023597600035
722 110.906131782291
723 111.101201578638
724 111.175916104515
725 111.193999524376
726 111.305048987243
727 111.454463236644
728 111.31523970535
729 111.163474402615
730 111.012122349409
731 110.992048446649
732 110.841311699364
733 110.880032905941
734 111.086415742934
735 110.981194872507
736 111.00044305293
737 110.886180912386
738 111.119108995968
739 111.109204736679
740 111.357434258583
741 111.220595868369
742 111.133784628127
743 111.321317491744
744 111.370965935368
745 111.326037565875
746 111.200195502016
747 111.208012773988
748 111.187857459464
749 111.481152440013
750 111.488469333333
751 111.646926157932
752 111.549044745926
753 111.411575477638
754 111.263851852894
755 111.129750449542
756 111.147364295512
757 111.164497276847
758 111.063187042697
759 110.921405149623
760 110.779984418283
761 110.68398486672
762 110.790572192255
763 110.690067660765
764 110.654957375072
765 110.520900508351
766 110.377547055335
767 110.239695115836
768 110.118591308594
769 110.23536553814
770 110.225299375949
771 110.242419348598
772 110.104276289296
773 110.027058100946
774 109.890571480079
775 109.753463475546
776 109.905301772239
777 110.013052213824
778 110.196425149186
779 110.310371250459
780 110.388553583169
781 110.433801505342
782 110.664459612378
783 110.705509314308
784 110.584103954082
785 110.489993103168
786 110.633231681655
787 110.497364252974
788 110.568829073153
789 110.680086294278
790 110.785132190354
791 110.664354519316
792 110.554936996225
793 110.416535607117
794 110.527712567176
795 110.603843202405
796 110.466331342643
797 110.541764993884
798 110.484400223617
799 110.42697614822
800 110.2901109375
801 110.231548890977
802 110.336666127698
803 110.219773607378
804 110.082688237915
805 110.024236719262
806 109.968346889643
807 109.980314749044
808 110.204353433487
809 110.218826826142
810 110.083903368389
811 110.009128490652
812 109.933453553835
813 109.842988097778
814 109.75180200303
815 109.856112010238
816 109.782367358708
817 109.653102897576
818 109.873651221597
819 110.050879523407
820 110.065133848899
821 110.170428801809
822 110.036401631532
823 110.14089460218
824 110.037619886417
825 109.963583471074
826 109.907703040998
827 109.949208178042
828 109.846546885575
829 110.062580666753
830 109.988737117143
831 109.857533222988
832 109.846061390533
833 110.021516409685
834 109.890816842929
835 109.879029007853
836 110.053585941256
837 109.923256667081
838 110.059661599102
839 110.195104280168
840 110.206427154195
841 110.421263967221
842 110.407676553393
843 110.286968672298
844 110.456968340783
845 110.625854837016
846 110.840249484432
847 110.937950318438
848 110.900541507209
849 110.914998730579
850 110.802458131488
851 110.838562774699
852 110.767742015473
853 110.870753385403
854 110.769175501429
855 110.644730344379
856 110.534981221067
857 110.707515429935
858 110.583451513521
859 110.46562521599
860 110.531671173607
861 110.414260489046
862 110.341804792179
863 110.411929067939
864 110.302049843536
865 110.217055030238
866 110.0907999936
867 109.982833326016
868 109.874009853681
869 109.914810219278
870 110.12216276919
871 110.190395720769
872 110.082232135342
873 110.326089152886
874 110.291629793317
875 110.257039673469
876 110.498180813578
877 110.597975112107
878 110.584326565346
879 110.552108935456
880 110.684085743802
881 110.651264363966
882 110.536258040631
883 110.439741999695
884 110.47675492926
885 110.352958600658
886 110.480620283415
887 110.519847629991
888 110.396411614317
889 110.276940635514
890 110.170589572024
891 110.265122858464
892 110.168778529631
893 110.046691387161
894 109.95075922506
895 110.081540526201
896 109.999596420599
897 110.038697056583
898 110.049827381809
899 110.175745884996
900 110.23897654321
901 110.14524741901
902 110.269803245805
903 110.256281203666
904 110.225879082152
905 110.287959463997
906 110.204514665536
907 110.435344843305
908 110.31454617788
909 110.505769828908
910 110.626306001691
911 110.757180502723
912 110.748632992844
913 110.763142583277
914 110.830347523809
915 110.869450864463
916 110.766760502279
917 110.664254140558
918 110.899901984517
919 110.797259167781
920 110.993152173913
921 111.226002987359
922 111.260590718094
923 111.140050403261
924 111.232974457
925 111.151964937911
926 111.058666365006
927 111.093336777881
928 111.01186157662
929 111.101725178757
930 111.294924268702
931 111.521360559601
932 111.406260936838
933 111.525827437222
934 111.411030588429
935 111.376997912437
936 111.295364891519
937 111.354277884527
938 111.235574715518
939 111.201314247931
940 111.431661385242
941 111.581834054034
942 111.645657250012
943 111.596706884124
944 111.478498231471
945 111.570330057949
946 111.559535196599
947 111.478921375678
948 111.465259974363
949 111.365319381169
950 111.334688088643
951 111.218534698657
952 111.128641162347
953 111.012035775906
954 111.047989487054
955 111.138144239467
956 111.026211901052
957 110.977527954936
958 110.946579948658
959 110.857073267796
960 110.778953993056
961 110.867845993757
962 110.789981889774
963 110.97494740497
964 110.963218393278
965 110.914122795243
966 111.031051614092
967 110.925248826582
968 111.041352195888
969 111.073631386181
970 110.976235519184
971 110.871654923789
972 110.75848024522
973 110.645536367852
974 110.792178783905
975 110.804495989481
976 110.813112024657
977 110.874814699187
978 110.865101768561
979 111.084803342094
980 110.997395876718
981 111.008359648822
982 111.019048577034
983 111.052028947861
984 110.943113184943
985 110.867376124095
986 110.926829569346
987 111.107370487051
988 111.193236858496
989 111.307803431273
990 111.22084379145
991 111.109582610803
992 111.254731399584
993 111.167479506597
994 111.283018027683
995 111.495531931012
996 111.484778471315
997 111.408621048703
998 111.61986899651
999 111.834232631029
1000 111.976864
};
\addplot [semithick, color2, forget plot]
table {%
1 0
2 256
3 202.666666666667
4 227
5 194.56
6 198.583333333333
7 170.489795918367
8 191.6875
9 180.765432098765
10 175.41
11 165.719008264463
12 159.6875
13 152.236686390533
14 141.489795918367
15 135.76
16 135.43359375
17 127.529411764706
18 132.25
19 133.501385041551
20 135.9275
21 134.793650793651
22 131
23 125.345935727788
24 120.776041666667
25 117.3856
26 113.177514792899
27 109.259259259259
28 106.738520408163
29 105.131985731272
30 112.512222222222
31 117.658688865765
32 120.58984375
33 118.249770431589
34 115.640138408304
35 120.027755102041
36 117.415895061728
37 118.705624543462
38 120.130886426593
39 120.664036817883
40 121.324375
41 124.437834622249
42 125.156462585034
43 122.559221200649
44 124.361570247934
45 123.004444444444
46 121.646502835539
47 126.035310095066
48 125.916232638889
49 124.673052894627
50 125.0116
51 122.902729719339
52 122.477440828402
53 120.640085439658
54 122.516117969822
55 120.343801652893
56 118.204081632653
57 116.220375500154
58 118.688466111772
59 116.948003447285
60 117.345555555556
61 120.511690405805
62 118.57232049948
63 116.726127488032
64 114.997802734375
65 113.52899408284
66 111.837465564738
67 113.638672310091
68 116.089100346021
69 114.408737660155
70 114.676734693878
71 113.15135885737
72 114.873263888889
73 114.730718708951
74 114.473520818115
75 114.753066666667
76 113.407721606648
77 113.088547815821
78 114.249835634451
79 113.227046947605
80 112.75609375
81 111.626276482244
82 114.883402736466
83 114.140804180578
84 112.782738095238
85 112.927335640138
86 114.513250405625
87 113.447747390672
88 112.65444214876
89 111.390733493246
90 110.282098765432
91 110.535442579399
92 109.466918714556
93 110.222222222222
94 112.211860570394
95 111.846648199446
96 111.6630859375
97 112.146242958869
98 111.727197001249
99 110.888888888889
100 109.9956
101 109.216155278894
102 111.096597462514
103 110.843246300311
104 112.268860946746
105 111.800453514739
106 111.092915628337
107 110.641628089789
108 109.836334019204
109 109.294335493645
110 109.409008264463
111 111.00073046019
112 111.78443877551
113 111.666066254209
114 110.717143736534
115 110.966805293006
116 110.010701545779
117 109.645993133173
118 108.722565354783
119 110.455617541134
120 109.579930555556
121 109.91598934499
122 109.704111797904
123 110.202657148523
124 110.670915712799
125 109.819904
126 109.022234819854
127 108.441316882634
128 109.8115234375
129 110.640826873385
130 110.770177514793
131 111.462036011887
132 110.816058310377
133 111.075131437617
134 110.985798618846
135 110.233525377229
136 109.536332179931
137 108.763279876392
138 108.571151018694
139 107.96604730604
140 108.041785714286
141 108.828429153463
142 109.112279309661
143 109.677637048266
144 109.791232638889
145 109.043709869203
146 108.64744792644
147 107.931972789116
148 108.545060262966
149 109.30705824062
150 109.097288888889
151 108.774352002105
152 109.786314058172
153 109.98034943825
154 110.01467363805
155 110.182559833507
156 111.031393819855
157 111.157450606515
158 110.456176894728
159 109.773822238044
160 109.894375
161 109.245168010493
162 109.500533455266
163 108.915201927058
164 108.664336704343
165 108.0527456382
166 108.61333284947
167 108.021585571372
168 109.321286848073
169 108.698224852071
170 108.269342560554
171 108.468451831333
172 107.837885343429
173 107.364295499348
174 106.842152199762
175 106.912653061224
176 106.573056559917
177 106.631874620958
178 107.606773134705
179 108.204612839799
180 107.71
181 107.356674094197
182 106.768747735781
183 106.49867120547
184 106.228142722117
185 106.943754565376
186 106.430801248699
187 107.120992879407
188 107.543543458579
189 108.06248425296
190 108.337950138504
191 108.253885584277
192 108.791639539931
193 109.310907675374
194 109.670953342544
195 109.122419460881
196 109.971340066639
197 111.152103893427
198 111.371110090807
199 110.845281684806
200 111.2096
201 110.657656988688
202 111.005293598667
203 110.611322769298
204 110.100994809689
205 109.574205829863
206 109.052431897446
207 108.667973581647
208 108.585059171598
209 109.071999267416
210 110.009160997732
211 109.62777116417
212 110.249977750089
213 110.153673212987
214 110.912765307014
215 111.006165494862
216 110.493055555556
217 110.04277007369
218 110.243834694049
219 111.316986718375
220 110.811962809917
221 110.311500583526
222 110.751826150475
223 110.648957348831
224 111.243542729592
225 110.835555555556
226 110.734376223667
227 110.947466475189
228 110.54593721145
229 110.843271486051
230 110.738676748582
231 111.191244541894
232 111.087916171225
233 111.947097938809
234 112.041785375119
235 111.56631960163
236 111.094423297903
237 110.634442486069
238 110.7170574112
239 111.696713993102
240 111.232222222222
241 110.822265456862
242 111.231763540742
243 111.418618435537
244 111.600779360387
245 111.153919200333
246 110.726435983872
247 110.806143355898
248 110.591506243496
249 111.362203835422
250 112.117056
251 111.743337407343
252 112.478883219955
253 112.444093799309
254 112.006696013392
255 111.625128796617
256 112.206787109375
257 111.904676830838
258 112.475752659095
259 112.233717446073
260 112.188446745562
261 112.617812422014
262 112.285079540819
263 112.234324625193
264 112.046358471074
265 111.754304022784
266 111.397888518288
267 111.078062534192
268 111.510066273112
269 111.129614018601
270 111.916543209877
271 111.681431353059
272 111.562932525952
273 111.381878195065
274 111.061550961692
275 110.693633057851
276 110.428586431422
277 111.210441945027
278 111.507841209047
279 111.548258629771
280 111.236683673469
281 111.543673459049
282 111.499471857552
283 111.108566719525
284 110.840842590756
285 111.472896275777
286 111.429605359675
287 112.174216027875
288 112.282986111111
289 112.772620059626
290 112.955873959572
291 112.583413044248
292 112.482595233627
293 112.980652075155
294 113.354215835994
295 113.002401608733
296 112.963294375457
297 113.343423006723
298 113.715913697581
299 113.501124148499
300 113.970788888889
301 114.234059226719
302 113.905409850445
303 113.585748673877
304 113.937846260388
305 114.527815103467
306 114.153584091589
307 114.340141539963
308 114.082644628099
309 113.833118630932
310 113.675379812695
311 113.93401639768
312 113.961774737015
313 113.647255764579
314 113.334425331656
315 113.424298311917
316 113.455255568018
317 113.101035934281
318 112.821288714845
319 112.481323886361
320 113.1293359375
321 112.787608815908
322 112.483237529416
323 112.293283746609
324 112.47789971041
325 112.211559763314
326 112.377582897362
327 112.826417529389
328 112.914782867341
329 112.867989024492
330 112.60577594123
331 113.048676079992
332 113.290417694876
333 113.186429672916
334 113.267130409839
335 113.082432613054
336 112.747980442177
337 112.525706839014
338 112.243408844228
339 112.749332149912
340 112.852759515571
341 113.083169219391
342 113.677823945829
343 113.368596418159
344 113.069556179016
345 113.412678008822
346 113.93659160012
347 114.160120921194
348 113.94156262386
349 113.79052717137
350 114.270824489796
351 114.442837314632
352 114.121771694215
353 113.893731592421
354 113.621915796866
355 113.447379488197
356 113.235599987375
357 113.647874836209
358 113.615406822509
359 113.581536456111
360 113.546288580247
361 113.567867036011
362 113.389128842221
363 113.212121212121
364 112.912057722497
365 112.646305122912
366 113.21974379647
367 112.921411548085
368 113.314478969754
369 113.332804547558
370 113.04984660336
371 112.756591422614
372 113.05193519482
373 113.342495094481
374 113.061261402957
375 113.369955555556
376 113.160564735174
377 113.173426957201
378 113.380063548053
379 113.18166818666
380 113.57021468144
381 113.293708365195
382 113.83029522217
383 113.752701293212
384 114.108228895399
385 114.031141845168
386 113.773497275095
387 113.862267892555
388 114.298245031353
389 114.030650075006
390 114.494023668639
391 114.764045237799
392 114.483418367347
393 114.310212432583
394 114.056720090701
395 114.010216311489
396 113.806161361086
397 113.896224200395
398 114.441510315396
399 114.271367642163
400 114.355975
401 114.776649398947
402 114.574694933294
403 114.907412766534
404 115.024335849427
405 115.315555555556
406 115.399894440535
407 115.172491231459
408 114.890330642061
409 114.797209485835
410 115.082831647829
411 115.008696372861
412 114.93392402677
413 114.905217243462
414 114.647372167378
415 114.461953839454
416 114.186847032175
417 114.67678806595
418 114.588310707172
419 114.379116090704
420 114.892171201814
421 114.638644557411
422 114.368708025426
423 114.106410922768
424 114.529636881452
425 114.591833910035
426 114.613530163768
427 114.490377285018
428 114.671079351908
429 114.546508658397
430 114.283374797188
431 114.051970004468
432 114.440538194444
433 114.495485068457
434 114.731381001933
435 115.186967895363
436 115.057360491541
437 115.219548722568
438 115.08054982173
439 114.819329497045
440 114.963383264463
441 114.727844879448
442 114.51758256383
443 114.284159409729
444 114.126750060872
445 113.941997222573
446 113.817194192523
447 113.633209715278
448 113.585374481824
449 113.333177910824
450 113.277654320988
451 113.182177078775
452 113.326395958963
453 113.664371445697
454 114.102194880553
455 114.055906291511
456 113.834699138196
457 114.101594932224
458 113.949123777197
459 113.747798804828
460 113.529201323251
461 113.527434935842
462 113.613519236896
463 113.753266563729
464 113.695674791914
465 113.599676263152
466 113.557332056218
467 113.343075533383
468 113.195754803127
469 113.3921649747
470 113.166229062924
471 113.636956198358
472 113.918629704108
473 113.724121593535
474 113.916114760811
475 113.770787811634
476 113.535356789775
477 113.689727463312
478 114.008827751615
479 114.196782615139
480 114.6527734375
481 114.554976854353
482 114.457004011639
483 114.299379739293
484 114.223464927259
485 114.017625677543
486 114.361856254975
487 114.17336161134
488 114.175519853534
489 113.998854136609
490 114.096463973344
491 114.40356560658
492 114.739539956375
493 114.510185188995
494 114.369338949991
495 114.635012753801
496 114.430915062435
497 114.826731009801
498 114.745121046435
499 114.740077349087
500 114.935084
501 114.81934334923
502 114.955036427993
503 114.945745013023
504 114.722186791383
505 115.099837270856
506 115.229674733241
507 115.25969756739
508 115.296701593403
509 115.373284802822
510 115.448381391772
511 115.485977765097
512 115.514064788818
513 115.314318935741
514 115.611701161259
515 115.556995004242
516 115.595937744126
517 115.433661692026
518 115.2115688496
519 115.444262532438
520 115.634375
521 115.580800247568
522 115.370003376345
523 115.226422061281
524 115.016957053785
525 114.819033106576
526 114.685176885599
527 114.993090386672
528 114.852096963958
529 114.670880964548
530 114.59531505874
531 114.404275768635
532 114.243466985132
533 114.679322325046
534 115.111114617963
535 115.347869682942
536 115.155933253509
537 115.386438902933
538 115.195436768425
539 115.316710323866
540 115.141851851852
541 115.220386700879
542 115.341280755981
543 115.297158884717
544 115.369593425606
545 115.158508543052
546 114.947929664963
547 115.367285074981
548 115.289583222335
549 115.079996416734
550 115.071788429752
551 114.866993191722
552 114.735494118883
553 114.690797196943
554 115.099053161126
555 115.326794903011
556 115.495480953367
557 115.419201995816
558 115.249177811179
559 115.054425709083
560 115.275889668367
561 115.264650277547
562 115.29733982599
563 115.367382930192
564 115.288551254967
565 115.084802255462
566 115.111176316348
567 114.963522857703
568 114.795179527871
569 114.872205114266
570 114.682363804247
571 115.012099705252
572 114.822913100885
573 114.645163844802
574 114.574017530867
575 114.852216257089
576 114.748887803819
577 114.743227535
578 114.861783264089
579 115.027201326807
580 114.880960166468
581 114.952070884966
582 114.944131505296
583 115.059281118247
584 115.222555826609
585 115.077343852728
586 115.068693287051
587 115.282332421083
588 115.548830926929
589 115.870408536814
590 115.980031600115
591 115.938639662621
592 115.926654378196
593 116.131665382242
594 116.247106304345
595 116.317226184591
596 116.171670195036
597 116.200819844617
598 116.099218688829
599 116.258728375896
600 116.184166666667
601 116.248327108729
602 116.106687564155
603 116.259289511536
604 116.115266326038
605 116.314978485076
606 116.267993878596
607 116.199094040152
608 116.39633559124
609 116.606534171338
610 116.530290244558
611 116.386610986256
612 116.291736084412
613 116.322341651387
614 116.283111757154
615 116.430838786437
616 116.260349025974
617 116.372309155242
618 116.3938479907
619 116.276526055627
620 116.123215400624
621 115.967301194634
622 116.161022425326
623 116.025455454243
624 115.850550624589
625 116.14761984
626 115.970891302351
627 116.127840988581
628 116.012807314698
629 115.832529995627
630 115.652811791383
631 115.960448160418
632 115.989041720077
633 116.049579599141
634 115.953059538855
635 116.246527373055
636 116.109074008148
637 115.960475048611
638 116.250999891904
639 116.236284687782
640 116.22109375
641 116.506638175043
642 116.329735251017
643 116.668159797218
644 116.650194340496
645 116.616684093504
646 116.614479195622
647 116.462450640096
648 116.46015994132
649 116.76348821584
650 116.856094674556
651 116.964782999568
652 116.802201343671
653 116.631478228649
654 116.684297524526
655 116.681139793718
656 116.504013143218
657 116.682600909535
658 116.533069724042
659 116.486537518335
660 116.337490817264
661 116.468359268609
662 116.556484515475
663 116.38804829276
664 116.217593264625
665 116.112037989711
666 116.028418057697
667 116.275331599568
668 116.117131754455
669 116.115948440548
670 116.2460770773
671 116.140680213486
672 116.156170280612
673 116.20589540524
674 116.038161822328
675 115.955665294925
676 116.083531739085
677 115.912132987439
678 115.790447350789
679 115.719382874842
680 115.570276816609
681 115.72097696874
682 115.551381997059
683 115.415338839715
684 115.608922403475
685 115.460442218552
686 115.339849892477
687 115.657803796436
688 115.566114707274
689 115.550519147036
690 115.49434152489
691 115.393169571145
692 115.707196615323
693 115.544761155151
694 115.612927605079
695 115.447390921795
696 115.496777563086
697 115.386701357941
698 115.645725404553
699 115.590966862532
700 115.426726530612
701 115.44112038844
702 115.27672867915
703 115.324002598091
704 115.297034397598
705 115.25316432775
706 115.089961399257
707 114.999957987312
708 114.838504261228
709 114.841742576306
710 115.095346161476
711 114.970871635402
712 114.842341560409
713 114.746796126436
714 114.606605387253
715 114.539656706929
716 114.386214147499
717 114.312272777671
718 114.154972416415
719 113.998104305741
720 113.956358024691
721 114.191289259601
722 114.297336960275
723 114.185744429714
724 114.485216797412
725 114.782603567182
726 115.07791855444
727 114.940190604489
728 115.139474021857
729 115.014234882141
730 114.941610058172
731 115.086381678304
732 114.940226641584
733 114.829951850866
734 114.933973821173
735 115.000816326531
736 115.200656825969
737 115.058031662337
738 114.933646565463
739 115.036209924174
740 115.327719138057
741 115.429694343822
742 115.310423493
743 115.323407885894
744 115.179146433114
745 115.333219224359
746 115.214182521257
747 115.079789035661
748 114.939208656238
749 114.836155372272
750 114.683066666667
751 114.667087469703
752 114.549506281123
753 114.775292808403
754 114.71087709053
755 114.935334415157
756 115.158203507741
757 115.160644203201
758 115.173698665423
759 115.109899475942
760 114.989162049861
761 114.973820669601
762 114.85780099338
763 114.741994163207
764 114.626399701214
765 114.63849630484
766 114.574862464125
767 114.50649255723
768 114.784478081597
769 114.636781931849
770 114.71039973014
771 114.977560430724
772 115.011705012215
773 115.149015361598
774 115.15246813426
775 115.111831841831
776 115.012021402381
777 114.949015203841
778 114.902439185572
779 115.08865089867
780 115.025378040763
781 114.962051672156
782 114.815052230166
783 114.769630510415
784 114.687519523115
785 114.671060083573
786 114.742609210807
787 114.92432459487
788 114.906353874101
789 114.975687567166
790 115.078474603429
791 114.944372611602
792 114.819404588817
793 114.957749793671
794 114.817872075833
795 114.993941695344
796 114.881447122547
797 114.981465942705
798 114.883325167555
799 114.74480146491
800 114.7796859375
801 114.63639551684
802 114.674692632508
803 114.531887737299
804 114.40979647781
805 114.628071447861
806 114.490921069645
807 114.368654692752
808 114.232176992452
809 114.367928175149
810 114.6280521262
811 114.491728255598
812 114.450714649712
813 114.340590859783
814 114.299265314007
815 114.189929617223
816 114.149485774702
817 114.185662984708
818 114.22246997567
819 114.205074534745
820 114.096339976205
821 114.104601945579
822 114.043316106346
823 114.296993041786
824 114.177255160713
825 114.214605693297
826 114.077624011397
827 113.970271767976
828 114.222303904408
829 114.231586881458
830 114.170301930614
831 114.052574645831
832 114.264053254438
833 114.472620260589
834 114.431516657178
835 114.598767973036
836 114.490732526728
837 114.55328454442
838 114.458205410085
839 114.496018729374
840 114.37101899093
841 114.240161972399
842 114.12420517826
843 114.063288627719
844 114.04455622066
845 114.215743146248
846 114.460100095569
847 114.492076346965
848 114.734045423193
849 114.801752494794
850 115.042552249135
851 114.967143099775
852 114.86003107849
853 114.75310503306
854 114.791093298378
855 114.951234225916
856 114.834296390514
857 114.892799908503
858 114.85664199825
859 114.740143735914
860 114.661108707409
861 114.690424256159
862 114.885240981692
863 114.924340298804
864 114.926563571674
865 114.794445521067
866 114.850929921222
867 114.838615438034
868 114.906839973242
869 114.863916436363
870 114.732631787554
871 114.639530423481
872 114.695643253935
873 114.569890399132
874 114.491410909624
875 114.372422530612
876 114.318894518463
877 114.35834300878
878 114.228829499639
879 114.38266413509
880 114.482044163223
881 114.421098199987
882 114.367661879567
883 114.61278022391
884 114.558694283491
885 114.712132528967
886 114.588050130192
887 114.527659325672
888 114.41846111314
889 114.51593719514
890 114.718222446661
891 114.721377637203
892 114.910142974924
893 114.811522743147
894 114.691995856043
895 114.703378795918
896 114.632653061224
897 114.832797048007
898 114.714009355112
899 114.865727708825
900 114.743288888889
901 114.730378504092
902 114.671447043033
903 114.896622431197
904 115.134618020205
905 115.189450871463
906 115.06382273682
907 114.962653574003
908 114.887281918919
909 115.085342637674
910 115.072272672383
911 114.971721404808
912 115.035288502231
913 115.269456997561
914 115.234317617034
915 115.457632058288
916 115.357414761351
917 115.242347087428
918 115.207618389888
919 115.439348963544
920 115.447783553875
921 115.455909346518
922 115.545514325643
923 115.488452676221
924 115.364390659845
925 115.368548137327
926 115.517104851914
927 115.550384078741
928 115.453775267539
929 115.334177634666
930 115.389350213898
931 115.397557112053
932 115.362222319439
933 115.369926558521
934 115.489113389488
935 115.404677285596
936 115.282539812989
937 115.228430616571
938 115.114966971418
939 114.996780160618
940 114.959891353554
941 115.015956299458
942 115.071524199765
943 114.959481540041
944 114.841657165685
945 114.846442148876
946 115.064122219292
947 115.209316588036
948 115.396494285104
949 115.382234752127
950 115.436104155125
951 115.380259420323
952 115.28415961973
953 115.245948895023
954 115.432187765076
955 115.43546065075
956 115.324866292607
957 115.253855166081
958 115.134353711847
959 115.053339146943
960 115.0199609375
961 115.02363454648
962 115.050683779894
963 115.110045731526
964 114.991446213047
965 115.108857687455
966 115.005413885781
967 114.968209443165
968 115.02694061198
969 115.085184581681
970 114.993198001913
971 115.103713139331
972 115.156378600823
973 115.366581144129
974 115.353372489659
975 115.379688099934
976 115.296723830959
977 115.19621509666
978 115.205031134865
979 115.256590650859
980 115.142528113286
981 115.025174544688
982 115.051215151754
983 114.937657367516
984 114.847783024324
985 114.748613981293
986 114.647323173516
987 114.70605613605
988 114.922372109033
989 115.129334714212
990 115.116620752984
991 115.000473484366
992 114.885323515544
993 114.853340959729
994 114.83604241141
995 114.893526931138
996 114.781756544249
997 114.987270738997
998 115.044606246561
999 114.933924915907
1000 115.018
};
\addplot [semithick, color3, forget plot]
table {%
1 0
2 2.25
3 134.888888888889
4 136.1875
5 153.84
6 130
7 155.632653061225
8 140.5
9 124.913580246914
10 114.29
11 103.900826446281
12 97.2222222222222
13 111.076923076923
14 127.086734693878
15 118.622222222222
16 113.8125
17 113.515570934256
18 116.237654320988
19 114.515235457064
20 110.44
21 105.705215419501
22 101.884297520661
23 99.8223062381853
24 99.8315972222222
25 105.9264
26 103.386094674556
27 99.838134430727
28 97.0344387755102
29 95.1105826397146
30 96.8233333333333
31 101.59209157128
32 98.58984375
33 105.456382001837
34 107.360726643599
35 108.878367346939
36 109.434413580247
37 112.241051862673
38 113.891966759003
39 119.940828402367
40 117.984375
41 122.290303390839
42 120.501700680272
43 117.700378583018
44 115.133780991736
45 118.027654320988
46 123.557655954631
47 120.946129470349
48 119.697482638889
49 117.314452311537
50 118.6436
51 116.720492118416
52 119.001479289941
53 118.679245283019
54 120.043895747599
55 119.225123966942
56 119.059948979592
57 119.293321021853
58 123.09631391201
59 122.151106004022
60 123.869722222222
61 123.179252889008
62 121.584027055151
63 120.682287729907
64 121.3896484375
65 121.783668639053
66 124.235307621671
67 123.649365114725
68 125.609861591695
69 125.816005040958
70 124.570612244898
71 123.98413013291
72 126.678819444444
73 125.089885531995
74 125.241051862673
75 126.770488888889
76 127.094702216066
77 125.871816495193
78 125.990959894806
79 124.553757410671
80 124.9275
81 123.615912208505
82 125.883997620464
83 128.446508927275
84 130.202239229025
85 129.955155709343
86 129.993645213629
87 128.840269520412
88 128.884297520661
89 129.786643100619
90 128.792098765432
91 128.767298635431
92 130.406899810964
93 129.07827494508
94 128.843028519692
95 128.039889196676
96 126.756510416667
97 125.6860452758
98 126.316638900458
99 125.920008162432
100 127.2856
101 126.059994118224
102 126.254517493272
103 125.641059477802
104 125.826553254438
105 126.47201814059
106 125.310786756853
107 127.196261682243
108 127.128515089163
109 128.620823163033
110 127.461570247934
111 126.321402483565
112 127.170200892857
113 127.570209100164
114 127.308402585411
115 126.40347826087
116 126.310642092747
117 127.7652129447
118 127.547328353921
119 127.494668455618
120 127.628888888889
121 128.165289256198
122 127.256920182747
123 127.005089563091
124 125.981269510926
125 125.086464
126 126.17435122197
127 126.148056296113
128 125.194763183594
129 124.60368968211
130 123.674792899408
131 123.106811957345
132 122.445534894399
133 122.995420883035
134 124.495488973045
135 125.937777777778
136 125.035413062284
137 124.319676061591
138 123.693131695022
139 123.986232596656
140 125.341785714286
141 124.736783863991
142 124.658202737552
143 123.877451220109
144 123.763888888889
145 123.214744351962
146 122.516044285982
147 122.160581239298
148 121.336011687363
149 120.881942254853
150 121.707955555556
151 121.387746151485
152 121.604397506925
153 121.843479003802
154 122.054520155169
155 121.717544224766
156 121.532996383958
157 120.77528500142
158 120.978088447364
159 121.386337565761
160 122.2999609375
161 122.986458855754
162 124.168571864045
163 125.312507057097
164 124.727059785842
165 125.794159779614
166 126.625489911453
167 126.752124493528
168 126.277742346939
169 127.06676937082
170 126.319342560554
171 125.6054170514
172 124.929015684154
173 124.348558254536
174 124.31734707359
175 124.906971428571
176 125.157928719008
177 126.264100354304
178 126.789073349325
179 126.456040697856
180 125.901358024691
181 126.58703946766
182 127.083232701365
183 126.391352384365
184 127.143549149338
185 126.482980277575
186 125.819979188345
187 126.223569447225
188 125.595150520598
189 125.13065143753
190 124.981855955679
191 124.978975357035
192 125.08984375
193 124.853445730087
194 124.635349133808
195 124.183668639053
196 123.877446897126
197 123.752634698137
198 123.205591266197
199 122.5865003409
200 122.566775
201 123.536942154897
202 123.344476031762
203 123.064718872091
204 122.478373702422
205 123.479500297442
206 123.458407955509
207 123.660575509347
208 124.167691383136
209 123.613607747075
210 123.202176870748
211 122.783270816019
212 122.513171947312
213 122.058365844519
214 121.948314263254
215 121.83502433748
216 121.316529492455
217 121.548217205717
218 121.754145273967
219 121.498509205396
220 122.239896694215
221 121.689891689359
222 122.566999431864
223 122.117999557602
224 121.718570631378
225 121.646933333333
226 121.485864202365
227 122.250072774554
228 122.586545860265
229 122.051257603783
230 122.136275992439
231 121.826427540713
232 122.677170035672
233 122.184825655289
234 121.76134122288
235 121.263811679493
236 120.793073111175
237 120.564688707294
238 120.156627356825
239 120.570473206001
240 120.794375
241 120.389352800399
242 119.897752885732
243 119.463835119985
244 119.068042864821
245 118.654627238651
246 118.230501024522
247 117.793735350522
248 117.389958376691
249 117.026402799955
250 116.710096
251 116.245202457104
252 115.797477324263
253 116.0944554672
254 115.706878913758
255 115.893825451749
256 115.849365234375
257 116.6262017593
258 116.901748692987
259 116.471191544551
260 116.43100591716
261 115.996241981181
262 116.824544024241
263 116.609984241495
264 116.882403581267
265 117.320840156639
266 117.025510204082
267 116.688661644854
268 117.3031159501
269 117.062119097304
270 116.680987654321
271 116.471562206397
272 116.094236591695
273 116.814072374512
274 117.244725344984
275 117.011226446281
276 117.480623818526
277 117.901419280846
278 117.52715956731
279 117.318533934559
280 116.999948979592
281 116.589506211927
282 116.274885569136
283 116.104808400654
284 115.832622495537
285 115.488408741151
286 115.146168516798
287 114.74724714395
288 114.950954861111
289 114.71519737551
290 114.354149821641
291 114.011714552261
292 114.406865734659
293 114.694521776608
294 114.333472164376
295 114.360769893709
296 114.249817384953
297 114.349442800621
298 114.445160127922
299 114.093757340522
300 113.979155555556
301 113.873886601693
302 113.581202578834
303 113.209380343975
304 113.894206630886
305 113.570438054286
306 113.24846212995
307 113.916667550849
308 114.257041659639
309 114.194949780585
310 114.433132154006
311 114.756857352591
312 114.397672172913
313 114.049954577468
314 113.858452675565
315 113.79283446712
316 114.496314693158
317 114.778513071082
318 114.441131679918
319 114.855868161673
320 114.680849609375
321 114.461311516775
322 114.962935457737
323 114.941674893845
324 115.337553345527
325 115.393363313609
326 115.979722609056
327 116.355768781154
328 116.060789708507
329 115.976958823366
330 115.79059687787
331 115.781144750413
332 115.45267818261
333 115.107016926837
334 114.885089103231
335 114.62868344843
336 114.622440121882
337 114.554535128424
338 114.277056125486
339 114.641988844511
340 115.093935986159
341 115.538617658947
342 115.918684381519
343 115.955460734898
344 116.013757436452
345 116.192934257509
346 115.938554579171
347 115.757592870965
348 116.184329171621
349 116.045073521564
350 115.823355102041
351 115.68399607146
352 115.57631714876
353 115.817958574421
354 115.664240799259
355 116.041864709383
356 116.205427029415
357 116.441776710684
358 116.164328516588
359 116.6559694602
360 116.604320987654
361 116.412412427774
362 116.784629590061
363 116.567629715639
364 116.4159144427
365 116.903253893789
366 116.585901340739
367 116.270267059671
368 115.963728733459
369 116.168550465992
370 116.322724616508
371 116.405576826672
372 116.950651809458
373 116.649569823689
374 116.731512196517
375 116.422286222222
376 116.440803248076
377 116.297335519141
378 116.498425296044
379 116.230031815429
380 115.93641966759
381 115.939997657773
382 115.907266796415
383 116.25827430823
384 116.281243218316
385 116.077814133918
386 116.294699186555
387 116.818460429061
388 116.561928738442
389 116.274158907224
390 116.036265614727
391 115.95349323984
392 116.469745678884
393 116.210399549366
394 116.291691360251
395 116.040980612081
396 115.759584481175
397 115.586102316492
398 115.666346809424
399 115.435732187612
400 115.18424375
401 114.955454257125
402 115.29172545234
403 115.024992457315
404 115.209752720322
405 115.461508916324
406 115.784173360188
407 115.545436434871
408 115.262471164937
409 115.059654114932
410 115.350273646639
411 115.214070482652
412 115.063601658969
413 114.913237458155
414 114.913160167098
415 114.764174771375
416 114.764030140533
417 114.79444933262
418 114.526916737254
419 114.347480362951
420 114.087409297052
421 114.143048166056
422 113.938658161317
423 113.892359539259
424 113.924077741189
425 114.069458823529
426 114.419339196368
427 114.534067537254
428 114.319460214866
429 114.085806966926
430 113.872801514332
431 113.672708480252
432 114.221445258916
433 114.094480209506
434 113.848202340249
435 113.611732065002
436 113.84494150324
437 113.843670962303
438 113.627223160485
439 113.468641196341
440 113.454772727273
441 113.240316534777
442 113.105735959542
443 112.856748314641
444 112.842235816898
445 113.072894836511
446 113.416457198013
447 113.212027486249
448 113.507528499681
449 113.712074840899
450 113.558498765432
451 113.921583473041
452 113.863184274415
453 113.653416760473
454 113.415067243688
455 113.464441492573
456 113.230912396122
457 113.49839357622
458 113.32066512843
459 113.190918972285
460 113.240543478261
461 113.226768178204
462 113.275955285696
463 113.100028455607
464 113.055918363555
465 112.937199676263
466 113.435484168064
467 113.842633053478
468 113.640747132734
469 113.548401762131
470 113.399927569036
471 113.308351476959
472 113.584530307383
473 113.436103500217
474 113.383627979847
475 113.185613296399
476 113.526106030648
477 113.732429712256
478 113.772658041701
479 113.76400033124
480 113.642669270833
481 113.431010412299
482 113.238791515298
483 113.016704602446
484 112.901010006147
485 112.709091295568
486 112.538112415113
487 112.319190113379
488 112.173609244827
489 112.12523366831
490 112.047080383174
491 112.435397231636
492 112.230054861524
493 112.066480421643
494 112.271111639266
495 112.523375165799
496 112.339388495057
497 112.374002566708
498 112.288753407203
499 112.53517857358
500 112.778736
501 112.984553846399
502 112.803003126934
503 112.751483148821
504 112.898774092971
505 112.686026860112
506 112.716727335218
507 112.67915455808
508 113.142053909108
509 112.920036590873
510 112.709592464437
511 112.660237973966
512 112.737243652344
513 112.531453172676
514 112.789584248058
515 112.570821001037
516 112.661919355808
517 112.449071978271
518 112.70191261311
519 112.590441823427
520 112.374227071006
521 112.612751942411
522 112.397781154123
523 112.537420163858
524 112.322952479459
525 112.571958276644
526 112.699923376079
527 112.589761962201
528 112.389448461892
529 112.189843518284
530 111.990943396226
531 111.804398480641
532 112.099270733224
533 112.224000225282
534 112.254667620531
535 112.045048475849
536 112.034682000446
537 111.966424962461
538 112.15780600047
539 112.200921792228
540 112.323796296296
541 112.627652632046
542 112.813445486853
543 112.741396443603
544 113.091178498054
545 112.896133322111
546 112.692928256115
547 112.586666844914
548 112.757885342853
549 113.184627788229
550 112.98918677686
551 112.787592926242
552 113.144612476371
553 113.067771059714
554 113.409053291454
555 113.526044963883
556 113.418647456136
557 113.582986568853
558 113.429747819273
559 113.332874638778
560 113.688456632653
561 113.49840017031
562 113.334950165271
563 113.229489319145
564 113.580764926312
565 113.534873521811
566 113.334502865562
567 113.257467596092
568 113.070518622297
569 112.904895895429
570 112.905989535242
571 113.120417370821
572 112.946192356594
573 113.074446667824
574 113.074421202151
575 113.343364839319
576 113.201955536265
577 113.010431653596
578 112.852088097604
579 113.064404413541
580 113.05355529132
581 113.135048183884
582 112.989678912625
583 113.11553477753
584 112.926440232689
585 113.038129885309
586 113.262999569011
587 113.582986281412
588 113.481220209172
589 113.740246338504
590 113.76341281241
591 113.637495311798
592 113.639788965486
593 113.548345082739
594 113.375471890623
595 113.710997810889
596 114.043635421828
597 113.864330025336
598 113.678429212201
599 113.562141688568
600 113.818788888889
601 114.020465059621
602 114.4098740632
603 114.317060358792
604 114.224233586246
605 114.221749880473
606 114.295156248298
607 114.178233622564
608 114.022569143871
609 113.906401244604
610 113.978395592583
611 114.137886698043
612 114.452614379085
613 114.266312548401
614 114.197667879765
615 114.015704937537
616 114.044503183505
617 113.977425142308
618 113.796548527979
619 114.052855066147
620 114.046188865765
621 114.249999351729
622 114.182310459983
623 114.248917241433
624 114.18053244165
625 114.17552384
626 114.425787749186
627 114.455931767944
628 114.292737534991
629 114.49449880068
630 114.344593096498
631 114.544769578135
632 114.743029963147
633 114.580804564138
634 114.685776552658
635 114.885460970922
636 115.033631976583
637 115.277170015452
638 115.186309096805
639 115.481055346161
640 115.34625
641 115.368732065975
642 115.20682058598
643 115.048032526382
644 115.247530959454
645 115.086819301725
646 115.153984031286
647 115.143592230459
648 114.966154168572
649 114.792044653265
650 114.751233136095
651 114.781782015616
652 114.923529301065
653 114.945674223574
654 114.903842736769
655 114.736847503059
656 114.594881673855
657 114.583423105347
658 114.721300154285
659 115.020951872175
660 115.088980716253
661 114.917676193179
662 114.748133459899
663 114.885790399232
664 115.130026128611
665 115.266843801232
666 115.093949805662
667 115.188008694304
668 115.016251927283
669 115.242163638029
670 115.199111160615
671 115.25212497307
672 115.285074316185
673 115.118126682107
674 115.014035520256
675 114.95561920439
676 114.801775147929
677 114.648337766103
678 114.580766352538
679 114.470209807805
680 114.309513408304
681 114.142422497795
682 114.336916607184
683 114.17199119379
684 114.063017253172
685 114.033915498961
686 113.875419680575
687 113.729541558874
688 113.574761695511
689 113.575506455371
690 113.813224112581
691 113.651152611308
692 113.796668782786
693 113.6330320313
694 113.90346236577
695 113.958565291652
696 113.810722602061
697 113.95604033684
698 114.047530397944
699 114.330654255722
700 114.171206122449
701 114.024350784797
702 114.24209219081
703 114.186487902891
704 114.450639204545
705 114.503024998743
706 114.341542344453
707 114.301871165092
708 114.61188515433
709 114.49668477623
710 114.355040666534
711 114.663034770069
712 114.695800719606
713 114.618814286473
714 114.875520796554
715 114.753319966747
716 114.857097156768
717 114.760179657608
718 114.638961522645
719 114.741653625709
720 114.924992283951
721 114.94253819918
722 115.217288464637
723 115.05806259075
724 115.087594624706
725 115.173505826397
726 115.029925475643
727 114.886698743115
728 114.823383347422
729 115.095598570679
730 115.194212797898
731 115.044054487509
732 114.941525575562
733 114.887369739563
734 114.76050011508
735 115.012789115646
736 114.866308778355
737 115.116834723631
738 114.975536313629
739 115.197284118355
740 115.095388970051
741 115.11227669506
742 115.160388256406
743 115.107778476186
744 114.967539599954
745 114.939878383857
746 115.233754285591
747 115.433303047083
748 115.653727229832
749 115.57901322814
750 115.565598222222
751 115.721596238305
752 115.574404425079
753 115.472593909444
754 115.363486691667
755 115.271109161879
756 115.558626228269
757 115.734221680868
758 115.611301787094
759 115.468908018143
760 115.36841932133
761 115.447169071748
762 115.374763193971
763 115.230127334159
764 115.139063827746
765 115.039507881584
766 114.98939252432
767 115.204758205576
768 115.250608656141
769 115.328964879321
770 115.501366166301
771 115.410407424791
772 115.623063706408
773 115.515608447456
774 115.798362812064
775 115.666837044745
776 115.818217597513
777 115.704708569573
778 115.643407061809
779 115.498919815899
780 115.350920447074
781 115.212939843695
782 115.163453928219
783 115.137420505016
784 115.415809818825
785 115.32679459613
786 115.203084837066
787 115.434346891756
788 115.528221546548
789 115.423704581853
790 115.279585002403
791 115.168672214755
792 115.39828684573
793 115.507560638563
794 115.520027409602
795 115.650213203592
796 115.517853084518
797 115.437983403888
798 115.467680793462
799 115.40745706852
800 115.2632484375
801 115.161276244894
802 115.411877102754
803 115.539460522418
804 115.396455223881
805 115.505043786891
806 115.49147676545
807 115.520119032812
808 115.684392155181
809 115.726769761078
810 115.624827008078
811 115.565989226435
812 115.578939066709
813 115.437519762652
814 115.313211972303
815 115.516483119425
816 115.543967464437
817 115.725844171215
818 115.688669065823
819 115.551229126321
820 115.414115110054
821 115.538737257823
822 115.420160903618
823 115.281808397396
824 115.269168571496
825 115.130158310377
826 115.309493518752
827 115.350277587293
828 115.220417921072
829 115.102725244856
830 115.143128175352
831 115.233982226045
832 115.099365234375
833 115.053978374063
834 115.066618647528
835 114.96116174836
836 114.855879844784
837 114.895914606556
838 115.057360404646
839 115.314215657723
840 115.22220946712
841 115.346644403002
842 115.332603630086
843 115.295393365783
844 115.168347521394
845 115.032885403172
846 114.94174365251
847 114.810053958063
848 114.766770870417
849 114.902240701664
850 115.154143944637
851 115.21329023296
852 115.122308184002
853 115.064860385179
854 115.234296244701
855 115.155013850416
856 115.403429065857
857 115.500305671326
858 115.596417374389
859 115.491638895703
860 115.385736884803
861 115.34440815787
862 115.238738217387
863 115.47697339712
864 115.72199583119
865 115.845348658492
866 116.04120375062
867 116.162641730822
868 116.087338603495
869 116.02506485372
870 116.053403355793
871 115.950529960811
872 115.871463628903
873 116.037082961021
874 115.994245139263
875 116.058767673469
876 116.251000813161
877 116.306692375401
878 116.284412440782
879 116.178634061615
880 116.11611053719
881 116.344181168598
882 116.264572117585
883 116.275107126046
884 116.212331852337
885 116.101005458202
886 115.97878333138
887 115.956037922186
888 115.851183954225
889 115.904297114717
890 115.817731347052
891 115.707807593329
892 115.745464165779
893 115.627389337751
894 115.628176158231
895 115.499661059268
896 115.372417839206
897 115.360106585932
898 115.299171383078
899 115.176380628086
900 115.416635802469
901 115.314139795344
902 115.216239104036
903 115.089364968991
904 115.076900109641
905 114.95849821434
906 114.831650902251
907 114.742265534876
908 114.615938355101
909 114.515243603568
910 114.419061707523
911 114.502512889781
912 114.540184239381
913 114.481395061477
914 114.608731667377
915 114.575596763116
916 114.466461118209
917 114.560670909002
918 114.45475030971
919 114.602926253995
920 114.785633270321
921 114.666230469878
922 114.668391594242
923 114.573477367245
924 114.452985326362
925 114.547045726808
926 114.507150054346
927 114.383666791182
928 114.268876337693
929 114.164826468267
930 114.107522256908
931 114.302461693593
932 114.235130505259
933 114.114172608729
934 114.028287763253
935 113.989426063084
936 113.87276280225
937 113.909600452863
938 113.991707620896
939 114.184659319671
940 114.19581258488
941 114.18402879339
942 114.133230557021
943 114.224002500987
944 114.25222188308
945 114.19666414714
946 114.141085867277
947 114.029098726708
948 114.140958535847
949 114.252093879532
950 114.240985041551
951 114.362841261785
952 114.373445334016
953 114.516759908788
954 114.519608489467
955 114.453590636222
956 114.358239526619
957 114.278554652568
958 114.159295853836
959 114.064565865773
960 114.009478081597
961 114.045112130639
962 113.954313821258
963 114.015749933414
964 114.076686007472
965 113.994057290129
966 113.915075292877
967 113.814950234684
968 113.819256497165
969 113.74070701552
970 113.662239345308
971 113.804263921488
972 113.695545225152
973 113.605927356192
974 113.541647517171
975 113.425956607495
976 113.336766830153
977 113.320305616697
978 113.221971303231
979 113.363023910705
980 113.513394418992
981 113.687194306502
982 113.73924531589
983 113.67576987837
984 113.564990043955
985 113.625328145533
986 113.742517969628
987 113.858999824466
988 113.827774385746
989 113.832552414272
990 113.842286501377
991 113.98890722863
992 113.911346213254
993 114.084521154628
994 113.973373439834
995 113.860047978586
996 113.796313769133
997 113.682183964129
998 113.821518789081
999 113.80979978978
1000 113.988111
};
\addplot [semithick, color4, forget plot]
table {%
1 0
2 4
3 130.666666666667
4 98.1875
5 82.16
6 85.8888888888889
7 95.3877551020408
8 86.6875
9 103.135802469136
10 120.56
11 113.421487603306
12 123.972222222222
13 126.130177514793
14 118.678571428571
15 112.648888888889
16 106.99609375
17 109.266435986159
18 104.904320987654
19 100.443213296399
20 108.1475
21 103.895691609977
22 99.3078512396694
23 99.4177693761815
24 97.1597222222222
25 99.3536
26 96.4866863905325
27 97.9506172839506
28 94.4630102040816
29 94.1878715814507
30 109.395555555556
31 105.902185223725
32 115.984375
33 112.481175390266
34 109.246539792388
35 109.151020408163
36 118.191358024691
37 128.375456537619
38 125.936288088643
39 123.128205128205
40 129.8
41 129.25758477097
42 128.166099773243
43 125.55327203894
44 124.526859504132
45 122.131358024691
46 126.400756143667
47 132.593028519692
48 130.618055555556
49 127.969179508538
50 130.3636
51 127.846212995002
52 127.128698224852
53 129.817016731933
54 133.475994513032
55 135.026115702479
56 135.287946428571
57 135.77162203755
58 135.246135552913
59 138.249353634013
60 137.416388888889
61 137.609782316582
62 136.193548387097
63 134.59410430839
64 138.5849609375
65 142.268402366864
66 140.114095500459
67 143.492537313433
68 141.614186851211
69 139.695442134005
70 137.699795918367
71 135.815116048403
72 137.906635802469
73 138.026646650403
74 137.487216946676
75 138.953955555556
76 137.880886426593
77 138.417271040648
78 137.054076265615
79 135.558083640442
80 138.00484375
81 139.456790123457
82 141.611094586556
83 140.381768035999
84 141.817885487528
85 142.408858131488
86 141.730665224446
87 140.211917029991
88 140.795325413223
89 139.412952909986
90 139.24
91 138.10989010989
92 139.715855387524
93 138.926118626431
94 137.536894522408
95 137.320332409972
96 135.998263888889
97 134.830481453927
98 134.847251145356
99 136.167125803489
100 135.2616
101 134.058033526125
102 133.018166089965
103 132.234140823829
104 134.115014792899
105 133.444353741497
106 132.775008899964
107 133.8653157481
108 132.673525377229
109 133.172628566619
110 131.963305785124
111 130.945864783703
112 130.43997130102
113 129.907432062025
114 129.04593721145
115 128.106616257089
116 127.052913198573
117 126.454671634159
118 125.384946854352
119 124.384436127392
120 123.5975
121 123.030940509528
122 122.77351518409
123 121.936281314033
124 121.000195109261
125 120.1344
126 119.280738221214
127 118.668361336723
128 118.781188964844
129 118.555735833183
130 119.738520710059
131 119.243633820873
132 119.314221763085
133 120.914918876138
134 120.047059478726
135 120.070562414266
136 121.536980968858
137 120.684639565241
138 120.370352867045
139 121.627037937995
140 121.318112244898
141 121.615311101051
142 123.092838722476
143 122.237175412001
144 121.456597222222
145 122.868965517241
146 124.231375492588
147 124.222222222222
148 125.111166910153
149 125.589477951444
150 124.813155555556
151 125.464146309372
152 124.969355955679
153 125.261053440984
154 124.681438691179
155 123.933985431842
156 124.394806048652
157 123.661406142237
158 122.885435026438
159 122.416597444721
160 122.71234375
161 122.105242853285
162 121.850480109739
163 121.319733524032
164 120.606038072576
165 120.356437098255
166 121.142292059805
167 120.778012836602
168 120.103599773243
169 120.140541297574
170 121.082906574394
171 120.563318627954
172 121.22238372093
173 122.332319823582
174 122.129079138592
175 122.290612244898
176 122.251259039256
177 121.590028408184
178 122.574422421411
179 121.889828657033
180 121.230987654321
181 120.929641952321
182 120.409763313609
183 120.83460240676
184 120.315099243856
185 119.750299488678
186 120.83281304197
187 120.193428465212
188 119.882384563151
189 119.290389406792
190 120.392908587258
191 120.947616567528
192 121.815972222222
193 121.817981690784
194 121.603039642895
195 121.722971729126
196 121.107116826322
197 120.618877064598
198 120.254489337823
199 119.965758440443
200 119.676775
201 119.408975025371
202 119.225272032154
203 119.460069402315
204 118.948168973472
205 119.631267102915
206 119.173908945235
207 120.160890569208
208 119.813771264793
209 120.799020168952
210 121.437755102041
211 122.058983401092
212 121.939012993948
213 122.686415834601
214 122.813608175386
215 122.257133585722
216 122.988147290809
217 123.115462209858
218 122.701982156384
219 123.127833031004
220 122.990578512397
221 122.445854916975
222 121.919892865839
223 121.632488085423
224 122.012675382653
225 122.512711111111
226 122.305583835852
227 122.147645015428
228 122.631117266851
229 123.179535096585
230 123.58202268431
231 123.70375367778
232 123.340350029727
233 123.504282635525
234 123.069782306962
235 122.840633770937
236 122.48770109164
237 122.087806441275
238 121.9723183391
239 121.748603840969
240 121.970815972222
241 121.946316351302
242 122.33783553036
243 122.718792866941
244 123.180109513572
245 122.732761349438
246 122.309488399762
247 122.807585766035
248 122.664867325702
249 122.814277189078
250 123.147024
251 123.470579832066
252 123.189342403628
253 122.773141277008
254 122.44156488313
255 122.367550941945
256 123.034896850586
257 122.798679768051
258 122.963163271438
259 123.494104142753
260 123.022174556213
261 123.022063680803
262 123.400166074238
263 123.660078937096
264 124.061065197429
265 124.14524741901
266 123.769362315563
267 124.372259394858
268 124.653263533081
269 124.225563494147
270 123.894224965706
271 124.145858580357
272 124.202408628893
273 123.869822485207
274 123.445628429858
275 123.30612892562
276 122.870969859273
277 122.427556725619
278 122.816676155478
279 122.376562479927
280 121.974987244898
281 121.576018540799
282 121.729704240229
283 121.394074092572
284 121.130666038484
285 121.143194829178
286 120.779549122206
287 120.489310298777
288 120.161349826389
289 120.430502508351
290 120.783781212842
291 120.371511909401
292 120.229639707262
293 119.979825041643
294 119.837012355963
295 120.209640907785
296 120.547799488678
297 120.489836637985
298 120.256396108283
299 120.131698750573
300 120.284655555556
301 120.456462952948
302 120.790985044516
303 120.59543182041
304 121.165122922438
305 120.770438054286
306 120.632918962792
307 120.31654447262
308 120.737930089391
309 120.514929671872
310 120.653995837669
311 120.266208992876
312 120.722129766601
313 120.709530565791
314 120.431498235223
315 120.988561350466
316 121.332078192597
317 121.411756510663
318 122.055614888652
319 122.379064671141
320 122.89859375
321 122.519074931338
322 122.335027583812
323 122.482627073968
324 122.117131534827
325 121.818452071006
326 121.814595957695
327 122.433914092529
328 122.572863994646
329 123.070093587458
330 122.70795224977
331 122.409105429852
332 122.238886267963
333 121.89688787887
334 121.560875255477
335 121.200392069503
336 120.868330144558
337 120.554711232819
338 120.308156227023
339 119.956770303078
340 120.256012110727
341 119.905556367764
342 119.568243220136
343 119.579664935529
344 119.588933207139
345 119.245570258349
346 119.644299842962
347 119.345813020621
348 119.736119368477
349 120.031707457246
350 119.712857142857
351 119.515003936656
352 119.280184659091
353 119.572583039748
354 119.569185100067
355 119.627772267407
356 119.993750789042
357 120.584249386029
358 120.806123404388
359 120.565560478271
360 120.952191358025
361 121.230883740917
362 120.923079271085
363 120.662173955938
364 120.46063277382
365 121.028860949521
366 120.97515602138
367 120.711980933855
368 120.519346644612
369 120.978312438951
370 120.86953981008
371 120.676484477736
372 120.523319169846
373 120.203034593794
374 120.267351082387
375 120.161806222222
376 119.865634902671
377 119.715427534142
378 119.721375381428
379 119.852214896861
380 119.854092797784
381 119.854299708599
382 119.852861763658
383 119.670895568175
384 119.422302246094
385 119.688352167313
386 119.387366103788
387 119.835106063338
388 120.09809756616
389 120.03946577144
390 120.049579224195
391 119.74258410136
392 119.751770095793
393 120.011149311423
394 120.449406065603
395 120.401371575068
396 120.586751351903
397 120.343432164407
398 120.134573621878
399 120.65186776465
400 120.99244375
401 120.753092331515
402 121.169407935447
403 121.113928415297
404 120.837442407607
405 120.897570492303
406 121.083743842365
407 120.906338100441
408 120.69751417724
409 121.109928802434
410 121.226347412255
411 121.018132736605
412 121.421151852201
413 121.167316452579
414 121.225559522976
415 120.942697053273
416 120.768023067678
417 120.878986019817
418 120.779223232069
419 120.886415547872
420 120.784914965986
421 120.683013523959
422 120.460597246243
423 120.216890498466
424 120.3838499021
425 120.615064359862
426 120.603545372391
427 121.016810271542
428 121.205105249367
429 121.10339543906
430 121.590530016225
431 121.3626326301
432 121.275972007888
433 121.074729717477
434 121.235835333093
435 121.29827454089
436 121.097903164717
437 120.820866213888
438 120.722190946811
439 120.972348628328
440 120.715619834711
441 120.444999768615
442 120.811879363649
443 120.656186783117
444 120.665915915916
445 120.405282161343
446 120.466206841079
447 120.62583767498
448 120.460613639987
449 120.522041061304
450 120.306291358025
451 120.253558242093
452 120.019617824419
453 119.754757344951
454 119.933784859011
455 120.316174375075
456 120.323157125269
457 120.232780621406
458 120.142045346199
459 119.890251137976
460 119.62965973535
461 119.688887215852
462 120.138059631566
463 120.435090894672
464 120.177536972354
465 120.279708636837
466 120.054762474903
467 120.339714520219
468 120.090218423552
469 120.199908165538
470 120.625717519239
471 120.680893072065
472 120.845262675955
473 120.695099875296
474 120.799039505777
475 120.957854847645
476 120.756020055081
477 120.855504133539
478 120.676725722589
479 120.426645630031
480 120.381232638889
481 120.148659454273
482 120.003322945542
483 120.100261906905
484 120.01553855611
485 119.769692847274
486 119.600099070264
487 119.357015461549
488 119.114078036818
489 118.872127500303
490 118.700274885464
491 118.618854244009
492 118.575099973561
493 118.722520973137
494 118.485026799325
495 118.481599836751
496 118.707840140479
497 118.799355489071
498 118.753298333898
499 118.97854225485
500 118.747136
501 118.742809789602
502 118.966413231536
503 118.797030935658
504 118.681957042076
505 118.741590040192
506 118.509697854989
507 118.926834961428
508 118.712195424391
509 118.972074370564
510 118.740134563629
511 118.523358902578
512 118.307369232178
513 118.207106460107
514 118.313903314206
515 118.15062682628
516 117.930638182802
517 117.705090744475
518 117.523203291543
519 117.84222660296
520 117.68038091716
521 117.616719655468
522 117.630572804275
523 117.771051698357
524 117.548991900239
525 117.327767800454
526 117.137941129697
527 117.541416272697
528 117.346931674702
529 117.140433317491
530 117.007650409398
531 116.945286759516
532 117.139164876477
533 117.033950628148
534 116.816002468824
535 116.777025067691
536 116.58586266429
537 116.95634412853
538 116.764997719766
539 116.743312875833
540 116.568981481481
541 116.494770757241
542 116.30551054588
543 116.201072413337
544 116.362010705017
545 116.305583705075
546 116.202541963531
547 116.472459050363
548 116.262667163941
549 116.625266671312
550 116.585441322314
551 116.743080556388
552 117.133900441084
553 116.947911931958
554 116.845755841989
555 116.717454752049
556 116.799363386988
557 116.712337509549
558 116.543392942023
559 116.649217072398
560 116.441020408163
561 116.258578232784
562 116.13265092894
563 116.517659455657
564 116.836451888738
565 116.748418826846
566 116.831827716665
567 116.91392240481
568 116.70913199266
569 116.70749101961
570 116.765527854725
571 116.561125747989
572 116.405041200059
573 116.718401359612
574 116.718343672984
575 116.662533081285
576 116.761571059992
577 116.565015363636
578 116.377162629758
579 116.201371550616
580 116.570083234245
581 116.483622219391
582 116.364252311617
583 116.366254865559
584 116.246783519422
585 116.494803126598
586 116.461219699705
587 116.54281145431
588 116.581378129483
589 116.497715618253
590 116.3652398736
591 116.169462409922
592 116.266515248356
593 116.566502393011
594 116.410014851092
595 116.21559494386
596 116.513051213909
597 116.3757536987
598 116.343866399705
599 116.178254798621
600 116.180975
601 116.126887799314
602 115.958698027616
603 115.876262688767
604 116.227873777466
605 116.277467386107
606 116.404687993552
607 116.380883107296
608 116.191414365478
609 116.082964616683
610 116.264630475679
611 116.081216968775
612 116.169251142723
613 116.084658393854
614 116.26208500886
615 116.599680084606
616 116.826232817507
617 117.104077081292
618 117.330002827788
619 117.553858560762
620 117.370801248699
621 117.64572231687
622 117.458059780193
623 117.326414671411
624 117.145124506903
625 117.47803136
626 117.752332370444
627 117.693739205197
628 117.818217777597
629 117.823526884221
630 117.642854623331
631 117.584625314885
632 117.398573946483
633 117.616255999042
634 117.530575485874
635 117.613277946556
636 117.780160990467
637 117.723734440039
638 117.540904668783
639 117.580550596222
640 117.474509277344
641 117.416142386725
642 117.498483613319
643 117.472239623267
644 117.314346958065
645 117.232267291629
646 117.064862118874
647 116.922316529267
648 117.001476527968
649 116.835415870333
650 117.15140591716
651 117.360997260507
652 117.4849142798
653 117.306332652453
654 117.473568910211
655 117.333008565934
656 117.305240556217
657 117.312320890353
658 117.171441505529
659 116.993642365197
660 117.255782828283
661 117.375168508724
662 117.294285831637
663 117.11892785888
664 116.980364984033
665 117.189200067839
666 117.228537095654
667 117.201450249388
668 117.145747427301
669 117.007022506429
670 117.259532189797
671 117.091313318867
672 117.247287326389
673 117.547681866253
674 117.759258688551
675 117.879615912208
676 117.706444102097
677 117.587418644685
678 117.659498699106
679 117.582883084151
680 117.790395761246
681 118.096312540296
682 118.254581573946
683 118.118335051845
684 118.419485910195
685 118.456164952848
686 118.283489447424
687 118.583695115569
688 118.528858504597
689 118.601207867358
690 118.573074984247
691 118.769211759211
692 119.018092819673
693 119.055444155011
694 118.907249458097
695 118.807026551421
696 119.100506589378
697 119.000611351375
698 119.037117921856
699 118.872470584383
700 119.023010204082
701 118.903893154471
702 118.803337635246
703 118.702872671279
704 118.736342248838
705 118.810289220864
706 118.815095619097
707 118.762638316772
708 118.64624628938
709 118.478904116129
710 118.451965879786
711 118.285368955988
712 118.258387514203
713 118.494750860104
714 118.562358276644
715 118.40266810113
716 118.307455681783
717 118.19122369862
718 118.117653106354
719 117.974721497366
720 118.207708333333
721 118.109541186632
722 118.149056944007
723 118.126570364376
724 118.103757135008
725 118.342361950059
726 118.490898845707
727 118.327913889304
728 118.560462202633
729 118.448004576237
730 118.595348095327
731 118.482673698118
732 118.762091656962
733 118.828913303641
734 118.898983955631
735 118.874161691888
736 118.761074477198
737 118.601197785588
738 118.446096899993
739 118.298245260666
740 118.302520087655
741 118.531735754834
742 118.676288315255
743 118.528529170418
744 118.756214591282
745 118.857932525562
746 118.765363439685
747 118.834129915467
748 118.868633646944
749 119.050465150686
750 118.892958222222
751 118.746285910841
752 119.020823902218
753 118.864032140583
754 118.791634712128
755 118.893995877374
756 119.074625220459
757 118.929421393284
758 118.805774465508
759 118.808952213317
760 118.759001038781
761 118.614869776782
762 118.46450492901
763 118.393177925997
764 118.243428085853
765 118.219546328335
766 118.358869785737
767 118.205827407958
768 118.208887736003
769 118.386501646203
770 118.244054646652
771 118.153222943909
772 118.000179535021
773 118.063832885098
774 117.916865305904
775 117.769490114464
776 117.910296989584
777 117.791429598379
778 117.769721320901
779 117.905520556455
780 117.818139381986
781 117.699879172603
782 117.581838161707
783 117.846580186563
784 117.707633212724
785 117.562860967991
786 117.49489637356
787 117.42674237813
788 117.431472081218
789 117.435800559339
790 117.535428617209
791 117.391869019516
792 117.289139819916
793 117.161398046272
794 117.194857209931
795 117.109332700447
796 117.28381859044
797 117.197889198673
798 117.455683381386
799 117.489922478192
800 117.7463734375
801 117.618840993078
802 117.750380905591
803 117.925035165452
804 117.95568797802
805 118.170558234636
806 118.234845359555
807 118.132447036387
808 118.229805901382
809 118.14381471731
810 117.999122085048
811 118.060803897093
812 118.065477201582
813 118.319706370495
814 118.273954264741
815 118.251351575144
816 118.185035503172
817 118.216524916515
818 118.149934840179
819 118.065075339801
820 117.998477096966
821 117.859809714839
822 117.892464524837
823 117.87004542844
824 117.963421387501
825 117.96828503214
826 117.97277934443
827 117.907891023776
828 118.00142943826
829 117.93559173565
830 118.185170561765
831 118.188892798753
832 118.051669690736
833 118.217043359961
834 118.151644152304
835 118.106493599627
836 118.007715024839
837 118.213343724886
838 118.082843000439
839 118.085461294662
840 118.290113378685
841 118.535071633481
842 118.47025801028
843 118.330891903035
844 118.286724242492
845 118.151479289941
846 118.243224966327
847 118.104841171494
848 118.135611816928
849 118.091734056973
850 118.183507266436
851 118.074029171459
852 117.946003879301
853 117.902629021906
854 117.859133763691
855 117.837671762252
856 117.997592584505
857 118.123861561524
858 118.028538477839
859 118.032460518702
860 118.231329096809
861 118.234173859907
862 118.190434213855
863 118.25002651829
864 118.252634977281
865 118.412608506799
866 118.570760151262
867 118.475874996175
868 118.35759280299
869 118.278142012101
870 118.257423701942
871 118.17795505384
872 118.181003282552
873 118.063810719708
874 118.020920934811
875 118.051207836735
876 118.109589041096
877 117.993137692117
878 117.863279040686
879 118.097652079038
880 118.218376807851
881 118.085255507556
882 118.141830050236
883 118.372658842179
884 118.377142155157
885 118.314640109802
886 118.252118482132
887 118.123637306503
888 117.991588294375
889 117.863045522009
890 117.735438707234
891 117.695541271299
892 117.848101711275
893 117.720262988605
894 117.805063835963
895 117.702423769545
896 117.853235361527
897 117.830809747343
898 117.768677982748
899 117.654335988201
900 117.715017283951
901 117.837745950054
902 117.711993795507
903 117.897626835122
904 118.129904456105
905 118.028052867739
906 118.251369335653
907 118.212423524492
908 118.243338702478
909 118.113266552178
910 118.035914744596
911 117.978209974202
912 118.126210709064
913 118.126717764216
914 118.104242778275
915 118.042836752366
916 118.06747582998
917 118.005874734953
918 117.928281382754
919 117.89125001036
920 117.768147448015
921 117.775170028329
922 117.732478202154
923 117.615660459968
924 117.593339470775
925 117.822230533236
926 117.805849959649
927 117.830688828144
928 117.719761398261
929 117.839625232173
930 117.712933287085
931 117.587314150037
932 117.486590285325
933 117.371176430719
934 117.282367519682
935 117.158310503589
936 117.384122287968
937 117.284282246867
938 117.508932265265
939 117.53440600825
940 117.459760072431
941 117.360203098655
942 117.319115943401
943 117.219768591249
944 117.226941791152
945 117.403761372862
946 117.344775375566
947 117.245620862413
948 117.469471594652
949 117.361362023804
950 117.320267036011
951 117.378673840476
952 117.343053765624
953 117.34980384471
954 117.428538779672
955 117.387797483622
956 117.495640832618
957 117.383282615366
958 117.469703540344
959 117.411550309292
960 117.304677734375
961 117.311918191357
962 117.343052632034
963 117.450609196555
964 117.536090761178
965 117.708916749443
966 117.595481355743
967 117.595484493989
968 117.680960146165
969 117.711619545434
970 117.642683600808
971 117.545959499004
972 117.6226904774
973 117.759859474042
974 117.744878124881
975 117.729710453649
976 117.835056730382
977 117.762330950657
978 117.793050380351
979 117.696719985894
980 117.75306122449
981 117.63377568293
982 117.682127583675
983 117.563132768768
984 117.593717983674
985 117.641922234533
986 117.557183942333
987 117.461746370496
988 117.440577414808
989 117.336576967471
990 117.336124885216
991 117.322477473854
992 117.242715920916
993 117.265263693792
994 117.152107817934
995 117.072801191889
996 117.051962871567
997 117.01919801531
998 116.901977301296
999 116.975912849787
1000 116.859584
};
\addplot [semithick, color5, forget plot]
table {%
1 0
2 49
3 38.2222222222222
4 149
5 150.56
6 142.888888888889
7 130.979591836735
8 114.6875
9 108.666666666667
10 98.8
11 90.1487603305785
12 139.076388888889
13 128.461538461538
14 138.454081632653
15 161.528888888889
16 158.87109375
17 172.934256055363
18 172.682098765432
19 163.93351800554
20 167.11
21 167.664399092971
22 168.489669421488
23 171.391304347826
24 173.234375
25 166.3104
26 169.812130177515
27 166.241426611797
28 163.473214285714
29 166.095124851367
30 169.022222222222
31 166.289281997919
32 162.58984375
33 157.961432506887
34 158.838235294118
35 154.30693877551
36 150.08024691358
37 148.369612856099
38 145.446675900277
39 144.202498356345
40 141.044375
41 140.931588340274
42 137.725623582766
43 134.667387777177
44 137.628099173554
45 138.190617283951
46 139.733459357278
47 139.913082842915
48 137.083333333333
49 134.78550603915
50 134.5056
51 136.124567474048
52 135.621301775148
53 133.167675329299
54 130.83024691358
55 128.684297520661
56 130.09693877551
57 128.197599261311
58 126.210760998811
59 129.731686297041
60 128.862222222222
61 128.449341574845
62 132.787721123829
63 131.495590828924
64 130.1708984375
65 132.434082840237
66 130.607897153352
67 128.744932056137
68 128.190311418685
69 129.750892669607
70 131.170612244898
71 129.548502281293
72 132.678819444444
73 131.160818164759
74 133.538349159971
75 132.9216
76 132.062846260388
77 131.34693877551
78 131.896942800789
79 131.05047268066
80 133.41734375
81 133.303155006859
82 131.791344437835
83 133.658295833938
84 132.083191609977
85 133.199446366782
86 131.94064359113
87 130.528735632184
88 130.404829545455
89 131.397803307663
90 133.614320987654
91 133.439922714648
92 135.020793950851
93 134.003468609088
94 135.505319148936
95 134.289861495845
96 133.332899305556
97 131.981932192582
98 131.545710120783
99 131.753902662994
100 130.7611
101 132.190569552005
102 132.412629757785
103 133.782071825808
104 133.190828402367
105 131.925079365079
106 130.966625133499
107 130.024630972137
108 129.212191358025
109 131.128524534972
110 129.937272727273
111 128.771041311582
112 130.586734693878
113 129.437544052001
114 129.030163127116
115 128.595538752363
116 127.888228299643
117 127.30981079699
118 128.871301350187
119 128.047454275828
120 127.721597222222
121 127.38897616283
122 126.825047030368
123 126.703946063851
124 128.409144120708
125 129.864064
126 130.27771478962
127 129.271002542005
128 129.107360839844
129 128.418364280993
130 127.676982248521
131 127.366237398753
132 126.401974288338
133 126.402396969868
134 126.953274671419
135 126.18962962963
136 125.536332179931
137 124.651606372209
138 124.771686620458
139 124.041405724341
140 123.227755102041
141 122.622000905387
142 122.088722475699
143 123.051982982053
144 122.529320987654
145 122.688894173603
146 122.525051604429
147 123.441991762691
148 123.576515704894
149 122.772037295617
150 124.323955555556
151 123.651769659225
152 122.84829466759
153 122.045708915374
154 121.690040479002
155 122.719417273673
156 121.936020710059
157 122.451458477017
158 121.821502964269
159 121.064752185436
160 120.631875
161 119.911808957988
162 119.232929431489
163 119.128608528736
164 118.901844140393
165 118.564407713499
166 117.942517056177
167 118.257162322062
168 117.647073412698
169 117.043100731767
170 116.947024221453
171 118.36058958312
172 117.872498647918
173 118.510407965518
174 117.835777513542
175 117.307755102041
176 116.691115702479
177 116.48230074372
178 115.92163236965
179 115.297899566181
180 115.444783950617
181 114.889350141937
182 115.865112909069
183 115.282689838454
184 114.70356805293
185 114.999737034332
186 115.282922881258
187 116.042780748663
188 115.557803304663
189 115.370174407211
190 114.843656509695
191 114.292261725282
192 113.703016493056
193 113.86517758866
194 113.607290891699
195 113.638553583169
196 114.04131091212
197 115.255378906955
198 115.622793592491
199 115.044064543825
200 115.2679
201 115.900695527338
202 115.537128712871
203 115.322138367832
204 115.02816224529
205 115.940368828079
206 115.661254595155
207 115.104389834068
208 115.539940828402
209 115.091962180353
210 114.987052154195
211 115.557287572157
212 115.115766286935
213 115.874539884062
214 115.423268407721
215 116.323201730665
216 115.795524691358
217 115.591157170464
218 115.087955559296
219 114.589103646713
220 114.200723140496
221 114.616244548637
222 115.501440629819
223 115.785195761025
224 115.276287468112
225 115.715555555556
226 115.204244655024
227 115.189699004444
228 114.775084641428
229 114.275128239355
230 113.959338374291
231 114.042090665467
232 113.792638971463
233 113.546022214445
234 113.113229600409
235 112.810683567225
236 113.476946279805
237 113.752390108423
238 113.393828119483
239 113.165175679697
240 112.917899305556
241 112.978943200014
242 112.587254968923
243 112.475063083202
244 112.226753560871
245 111.788154935444
246 111.442345825897
247 111.65608352866
248 111.21871748179
249 110.843212206255
250 110.4
251 110.71586165299
252 110.970946712018
253 110.797200393695
254 111.223448446897
255 111.693133410227
256 111.583923339844
257 111.215945737256
258 110.798569797488
259 110.408312338814
260 110.375428994083
261 110.011391494546
262 110.346148243109
263 110.892104844656
264 110.707859848485
265 110.842776788893
266 111.518570863248
267 112.3118573693
268 112.346889619069
269 112.656320393582
270 112.958093278464
271 113.271415149576
272 113.294928633218
273 112.93598465027
274 112.697173530822
275 113.477024793388
276 113.300081390464
277 112.905811362067
278 112.931939340614
279 112.752456931437
280 112.854897959184
281 112.620711490483
282 113.143315225592
283 113.634157000337
284 113.752169708391
285 113.710360110803
286 113.435143527801
287 113.315276378249
288 113.13638117284
289 112.875444498988
290 113.147158145065
291 112.792952374204
292 112.536533589792
293 112.202937716223
294 111.91156462585
295 112.525021545533
296 112.412481738495
297 112.44854833407
298 113.145500202694
299 113.180568450017
300 113.128888888889
301 112.840299775941
302 112.481437217666
303 113.163088586086
304 112.790934470222
305 113.499768879334
306 113.178190012388
307 112.921749832889
308 113.20217363805
309 113.091672688807
310 112.753475546306
311 112.40549622109
312 112.093184993425
313 111.897314456614
314 111.555154367317
315 111.451206853112
316 111.422638599583
317 111.220670919205
318 111.411030022547
319 111.171116636039
320 111.5343359375
321 111.19034170864
322 111.632354075846
323 111.531079565605
324 111.239330894681
325 111.152037869823
326 110.884828183221
327 110.625162490999
328 110.97988548483
329 110.669062554854
330 110.638943985308
331 111.077810534771
332 111.252866889244
333 111.686389091795
334 111.499775897307
335 111.178667854756
336 110.894761550454
337 110.568464986044
338 110.346276390883
339 110.069090940733
340 109.985077854671
341 110.319140702264
342 110.023844943743
343 109.805999201013
344 109.841806381828
345 109.710010501995
346 109.464641317785
347 110.082900779842
348 110.40903686088
349 110.658204776644
350 111.073991836735
351 111.580214446311
352 111.673287383781
353 111.76351627892
354 112.356738165917
355 112.679658797858
356 112.365697197324
357 112.332948865821
358 112.063301707188
359 112.218371986561
360 111.952469135802
361 112.188089409995
362 112.499381887
363 112.287548664709
364 111.989577043835
365 111.682762244323
366 111.420205141987
367 111.504176287596
368 111.204395085066
369 111.235243571948
370 111.25738495252
371 111.726491379749
372 111.701374436351
373 111.405170740823
374 111.868748033973
375 111.844266666667
376 112.302597329108
377 112.66806914845
378 112.370118417737
379 112.764795566725
380 112.900519390582
381 112.909679597137
382 112.863545407198
383 113.389211188296
384 113.29062906901
385 112.996633496374
386 112.722462347982
387 112.655369268674
388 112.369380380487
389 112.080742263136
390 111.821176857331
391 112.388537489943
392 112.904148011245
393 112.751678547611
394 112.580181143549
395 112.445402980292
396 112.310548668503
397 112.5686857984
398 112.956869776016
399 112.687244426857
400 113.1854
401 112.920479350253
402 112.644216727309
403 112.509060458472
404 112.277987452211
405 112.333827160494
406 112.801020408163
407 112.856111416308
408 112.909307718185
409 112.849409078138
410 113.227864366449
411 113.404526376235
412 113.309000612687
413 113.481054587879
414 113.340264650284
415 113.328169545652
416 113.186893259985
417 112.928983431959
418 112.836496417207
419 112.697307488565
420 112.479795918367
421 112.651835636224
422 112.385688776083
423 112.483778481968
424 112.425952296191
425 112.192387543253
426 112.093345676563
427 111.835133193293
428 112.299693204647
429 112.210127091246
430 112.260573282856
431 112.206469603415
432 111.947139703361
433 111.715535311405
434 111.487735989297
435 111.940335579337
436 112.387614678899
437 112.43237384078
438 112.204004295156
439 112.304564629698
440 112.463217975207
441 112.456661576195
442 112.971115456276
443 113.251216566709
444 113.674879271163
445 113.435293523545
446 113.187073940759
447 113.236821164212
448 113.382528499681
449 113.296432061349
450 113.057066666667
451 112.876760684559
452 112.642273279035
453 112.848978358649
454 112.952298705583
455 112.773452481584
456 113.113727300708
457 113.44994709096
458 113.231822429015
459 112.996862555238
460 112.901304347826
461 113.381265851375
462 113.283221828676
463 113.416482793688
464 113.230510552913
465 113.059128222916
466 113.065579583341
467 112.839767250985
468 112.657293264665
469 112.487850118885
470 112.331321865097
471 112.761689678644
472 112.522941503878
473 112.285202186574
474 112.24768555609
475 112.040305817174
476 112.258138549537
477 112.413863727261
478 112.182529192416
479 111.953975095994
480 111.744166666667
481 111.633265762164
482 111.413250460564
483 111.241961687006
484 111.517122293559
485 111.77404612605
486 111.588447729851
487 111.373594356767
488 111.335192152647
489 111.196808310437
490 111.465593502707
491 111.350799109013
492 111.212419029678
493 111.351883776522
494 111.130443868937
495 111.44811345781
496 111.332579994797
497 111.284147541183
498 111.061579651941
499 111.02589949438
500 110.861036
501 111.309931036131
502 111.275047221473
503 111.226343726903
504 111.59873000126
505 111.387942358592
506 111.177947632364
507 111.127559336936
508 110.973882447765
509 110.799989192569
510 110.732041522491
511 110.551162104925
512 110.916896820068
513 110.715502205807
514 110.867446895487
515 110.767898953719
516 110.563352562947
517 110.376169614163
518 110.163250398772
519 110.196093718096
520 109.989937130178
521 109.778964857925
522 109.56880036993
523 109.489823748122
524 109.566906794476
525 109.372785487528
526 109.207914672758
527 109.006520745043
528 108.886234504132
529 108.897981353697
530 109.132303310787
531 109.232212965623
532 109.052896574142
533 109.436049970256
534 109.791636858421
535 109.596156869596
536 109.628644325017
537 109.660129903006
538 109.86057752104
539 109.87012987013
540 110.183055555556
541 110.492918911716
542 110.539113029507
543 110.359777377573
544 110.293384380407
545 110.419459641444
546 110.785640488937
547 110.683014214145
548 110.983829719218
549 110.85652005136
550 110.754462809917
551 110.757355871687
552 110.557337612896
553 110.517290204016
554 110.641051623245
555 110.881077834591
556 111.225091868951
557 111.182643618513
558 110.984147171799
559 110.842585629206
560 110.64943877551
561 110.693109134757
562 110.500778865516
563 110.543586281308
564 110.504259720336
565 110.635852455165
566 110.63172533057
567 110.812015341116
568 110.621292898234
569 110.43044715083
570 110.777743921207
571 110.683276029702
572 110.806310822045
573 110.772018310902
574 111.054768784373
575 110.959558412098
576 110.923246407215
577 111.002159619619
578 111.339199123574
579 111.157919228257
580 110.976825208086
581 110.806532745193
582 110.772041544148
583 111.169169934891
584 110.97921455714
585 110.810595368544
586 111.085825111533
587 110.968020918887
588 110.831216159933
589 111.160904067497
590 111.045090491238
591 111.376816946814
592 111.451778214025
593 111.264306168936
594 111.381956489701
595 111.711670079797
596 111.544488423945
597 111.814488410786
598 111.778201586112
599 111.641734554809
600 111.907330555556
601 111.724613165523
602 111.657321663116
603 111.589861637088
604 111.439813495022
605 111.815843180111
606 111.666263656069
607 111.603869192208
608 111.423606301939
609 111.334972673175
610 111.173351249664
611 111.292164116136
612 111.158857170319
613 110.998091912851
614 111.262581035343
615 111.334459647036
616 111.366522073706
617 111.362387670776
618 111.202859731255
619 111.073272071009
620 111.383777315297
621 111.20848146333
622 111.209626141169
623 111.049965346573
624 111.160315479126
625 111.03251456
626 111.186377833805
627 111.262323563003
628 111.623889407278
629 111.467239239614
630 111.290541698161
631 111.605179814196
632 111.605689693158
633 111.965055192431
634 111.904934868493
635 111.819207638415
636 111.676246489458
637 111.743193787598
638 111.654042314836
639 111.479870004237
640 111.735458984375
641 111.730530250851
642 112.083704544793
643 112.046104840017
644 112.015940260792
645 111.859993990746
646 111.923654976085
647 112.230912378855
648 112.090513545953
649 111.918366765511
650 111.750144378698
651 111.744276205106
652 112.090283789379
653 111.919269996646
654 111.837399115301
655 111.670727813064
656 111.77249405116
657 111.974706300721
658 112.009728291498
659 111.850087846348
660 112.087169421488
661 111.961127984235
662 111.962753625834
663 111.826079637099
664 111.82716785818
665 111.706108881226
666 111.733481228977
667 111.671351730432
668 111.635633672774
669 111.580084055581
670 111.460672755625
671 111.304736796516
672 111.27635832979
673 111.220663724336
674 111.454886456692
675 111.748424691358
676 111.647112758657
677 111.789208188877
678 111.815934424518
679 111.682518474496
680 111.626980968858
681 111.653916564782
682 111.536246248312
683 111.676112405652
684 111.521106921788
685 111.55527518781
686 111.549875052062
687 111.471952607057
688 111.41206184086
689 111.358136673962
690 111.464851921865
691 111.570483432849
692 111.426214958736
693 111.282313966297
694 111.349726764611
695 111.677884167486
696 111.704212907914
697 111.883439788065
698 111.726941897029
699 111.597675813189
700 111.733257142857
701 111.828124077892
702 111.726018863483
703 111.718722240995
704 111.64375443892
705 111.59191187566
706 111.43446500654
707 111.277461793462
708 111.466077755434
709 111.432371623356
710 111.277992461813
711 111.183606615749
712 111.132180280268
713 111.448982923821
714 111.553672449372
715 111.405459435669
716 111.353958599919
717 111.202340450778
718 111.176304109993
719 111.102141167322
720 111.196844135802
721 111.191864435472
722 111.504408345547
723 111.380168309009
724 111.413622523732
725 111.263638525565
726 111.114057555267
727 110.97073575906
728 110.847691628427
729 110.843167915159
730 111.116834302871
731 111.141984538542
732 111.064295365642
733 111.156022178008
734 111.064862386683
735 111.098068397427
736 111.356443466446
737 111.323683789023
738 111.182467813838
739 111.284319775288
740 111.234804601899
741 111.08828752042
742 110.993157925327
743 111.248831172595
744 111.281903977338
745 111.204990766182
746 111.081163883878
747 110.986138216409
748 111.254716677629
749 111.222782134078
750 111.322979555556
751 111.575380185496
752 111.826045439113
753 111.714949850884
754 111.584878877639
755 111.439350905662
756 111.687639973685
757 111.542276489445
758 111.576012419852
759 111.438613667175
760 111.685346260388
761 111.538687079211
762 111.392992952653
763 111.346536830371
764 111.339231380719
765 111.558757742749
766 111.437219218892
767 111.369255586965
768 111.659910413954
769 111.796111004953
770 111.665037949064
771 111.609330446588
772 111.507025356385
773 111.795042583707
774 111.951385133105
775 112.033105515088
776 112.066995363482
777 112.305064027072
778 112.247533389285
779 112.422950987161
780 112.301768573307
781 112.335182085412
782 112.268372132574
783 112.222574536487
784 112.188769002499
785 112.236851799262
786 112.227214161309
787 112.461088624067
788 112.539275683475
789 112.798366641447
790 112.740753084442
791 112.971648491803
792 112.84172533415
793 112.709502599193
794 112.573503099442
795 112.70809857205
796 112.608841948436
797 112.467600427576
798 112.326712771905
799 112.55584186115
800 112.70269375
801 112.567795249696
802 112.752439350502
803 112.644817302488
804 112.789015123388
805 112.796055707727
806 112.930367159455
807 112.800541728279
808 112.673163170277
809 112.591968292433
810 112.549550373419
811 112.732398691847
812 112.720049018418
813 112.6780983677
814 112.742624464983
815 112.816358914524
816 113.037629757785
817 112.917699018261
818 112.789768413627
819 112.68102349421
820 112.850303390839
821 112.85603101295
822 112.799730347322
823 112.967379220438
824 112.842492223584
825 112.8751456382
826 112.832468092092
827 112.708360078312
828 112.850607657122
829 112.864875640423
830 112.760924662505
831 112.823602838851
832 113.001825998521
833 113.178868666314
834 113.137393682177
835 113.003455125677
836 113.168849156384
837 113.173917201589
838 113.111922636576
839 112.994475232306
840 112.887528344671
841 113.129995009056
842 113.109291868134
843 113.08836851948
844 112.955070360055
845 112.822086061412
846 112.750556086492
847 112.679049189514
848 112.551975792097
849 112.728805870136
850 112.608200692042
851 112.5209078695
852 112.624791983513
853 112.520073281117
854 112.591360946421
855 112.831462672275
856 112.96857804175
857 112.896568720224
858 113.148701919681
859 113.209460604081
860 113.077847485127
861 113.316193659697
862 113.190190621282
863 113.230370759256
864 113.119255829904
865 113.219625112767
866 113.310716895391
867 113.270301946683
868 113.478583108582
869 113.438262834018
870 113.37865107676
871 113.468222782581
872 113.526555003788
873 113.416581707296
874 113.28685283999
875 113.246124408163
876 113.250015637706
877 113.132585041001
878 113.112952143254
879 113.142147517411
880 113.04458161157
881 112.916956662342
882 112.986898463089
883 112.97677920299
884 112.957612456747
885 112.873445050911
886 113.064286951781
887 113.005019262325
888 112.903391567243
889 113.021440655126
890 113.248111349577
891 113.302724211815
892 113.550016087193
893 113.431243878919
894 113.307725127497
895 113.309684466777
896 113.1865234375
897 113.432504483544
898 113.321739475499
899 113.232230596102
900 113.286235802469
901 113.219348091466
902 113.097113583512
903 112.975142535832
904 112.851984053176
905 112.777131345197
906 112.696236519841
907 112.829131257681
908 112.779991606668
909 112.70592449784
910 112.639957734573
911 112.524582460258
912 112.690669244383
913 112.839021124826
914 112.854039281969
915 112.823988772433
916 112.971887454473
917 112.952220804411
918 112.979110598488
919 112.861796838831
920 112.960283553875
921 113.000564698004
922 112.889400341613
923 112.816553961656
924 112.730238376342
925 112.721898027757
926 112.615401480625
927 112.586043296572
928 112.529604590889
929 112.642583608426
930 112.545590241646
931 112.466825341703
932 112.447856609995
933 112.357061375847
934 112.384081728102
935 112.265624982127
936 112.363155407627
937 112.355454463654
938 112.382737167043
939 112.292500688993
940 112.187940244454
941 112.103699571194
942 112.048129065412
943 111.948657800009
944 112.076790747989
945 112.013802525125
946 111.903406576707
947 111.943492984571
948 111.915286902028
949 111.969127282781
950 111.880128531856
951 111.872492401048
952 111.763071772827
953 111.993994774331
954 112.047980696966
955 112.24083331049
956 112.177627317449
957 112.148655509806
958 112.362534377029
959 112.246876906232
960 112.300380859375
961 112.410734569111
962 112.328443428236
963 112.28218109512
964 112.492988240561
965 112.684339445354
966 112.777062999113
967 113.003164404672
968 112.993698133666
969 113.021003002255
970 112.922628334573
971 112.886134565637
972 112.823302469136
973 112.886088838517
974 112.91346255202
975 112.799112163051
976 112.782004795418
977 112.705895787346
978 112.62506952547
979 112.797979218335
980 112.789321116202
981 112.760691673914
982 112.673358746645
983 112.880964183593
984 112.884409701567
985 113.03671416424
986 113.242865636147
987 113.133138090003
988 113.169666565589
989 113.374936229771
990 113.57914600551
991 113.525153220559
992 113.693141909469
993 113.755807267184
994 113.718192454526
995 113.617904598369
996 113.838603409622
997 113.989046376844
998 114.123621993486
999 114.198432666901
1000 114.144016
};
\addplot [semithick, color6, forget plot]
table {%
1 0
2 72.25
3 49.5555555555556
4 38.5
5 43.76
6 57.1388888888889
7 56.4897959183673
8 57.734375
9 75.4320987654321
10 71.25
11 69.4214876033058
12 81.2430555555555
13 79.4437869822485
14 86.2040816326531
15 83.76
16 91.359375
17 89.4186851211073
18 92.7654320987654
19 88.7257617728532
20 89.14
21 95.9319727891156
22 99.0661157024793
23 96.5406427221172
24 103.076388888889
25 99.4016
26 110.788461538462
27 118.891632373114
28 118.673469387755
29 114.592152199762
30 115.765555555556
31 120.056191467222
32 121.046875
33 117.820018365473
34 121.885813148789
35 121.311020408163
36 121.793209876543
37 119.060628195763
38 116.706371191136
39 113.717291255753
40 116.099375
41 114.343842950625
42 112.621882086168
43 110.060573282856
44 111.061983471074
45 113.662222222222
46 112.939508506616
47 111.595291987325
48 109.600260416667
49 107.363598500625
50 105.5204
51 108.286812764321
52 109.128328402367
53 107.159843360627
54 109.358024691358
55 108.272396694215
56 108.454081632653
57 107.128347183749
58 106.281807372176
59 104.99396725079
60 103.35
61 103.110991668906
62 105.18418314256
63 107.977828168304
64 106.46484375
65 104.828402366864
66 107.208448117539
67 109.343729115616
68 108.206747404844
69 110.122243226213
70 110.13387755102
71 113.313231501686
72 112.800733024691
73 115.308313004316
74 114.620891161432
75 117.342222222222
76 116.034626038781
77 116.965086861191
78 117.408284023669
79 118.821342733536
80 120.51109375
81 122.902911141594
82 125.044021415824
83 125.20307736972
84 125.605867346939
85 125.16955017301
86 125.273120605733
87 125.332804861937
88 124.121771694215
89 123.04002019947
90 123.190617283951
91 121.979471078372
92 121.857632325142
93 121.727367325702
94 120.866115889543
95 120.50459833795
96 120.439995659722
97 120.352428525879
98 119.427009579342
99 118.957045199469
100 117.8891
101 117.268895206352
102 117.168396770473
103 116.463191629748
104 117.384615384615
105 116.55201814059
106 115.455322178711
107 116.614202113722
108 115.802126200274
109 114.936116488511
110 114.722396694215
111 113.709439168899
112 113.332828443878
113 112.347560498081
114 111.616266543552
115 112.952741020794
116 112.796893579073
117 113.422894294689
118 112.807885665039
119 113.579267000918
120 113.129930555556
121 114.213100198074
122 113.293738242408
123 115.154603741159
124 116.697450572321
125 115.857536
126 114.939342403628
127 114.48496496993
128 113.745056152344
129 114.4648759089
130 113.925207100592
131 113.098071207972
132 114.013028007346
133 113.193057832551
134 114.296558253509
135 114.491632373114
136 113.660034602076
137 112.835313548937
138 114.052299936988
139 114.383002950158
140 115.472397959184
141 116.297268749057
142 117.087532235668
143 116.954374297032
144 117.937065972222
145 119.590772889417
146 118.826843685494
147 120.010643713268
148 120.720735938641
149 120.724111526508
150 120.246266666667
151 120.394105521688
152 121.254458102493
153 122.383784014695
154 121.687637038286
155 121.966451612903
156 121.796309993425
157 122.79346018094
158 122.382470757891
159 121.649301847237
160 122.17734375
161 121.664441958258
162 121.08401920439
163 120.405209078249
164 119.949546400952
165 119.739357208448
166 121.226774568152
167 121.520814658109
168 121.959183673469
169 121.259409684535
170 120.552975778547
171 120.051981806368
172 119.734282044348
173 119.244679073808
174 118.60747787026
175 117.949910204082
176 117.49173553719
177 116.832583229596
178 116.182931448049
179 116.485752629444
180 116.774691358025
181 117.197887732365
182 116.915348387876
183 116.296336110365
184 116.429938563327
185 115.804353542732
186 115.189039195283
187 115.318539277646
188 114.713077184246
189 114.109347442681
190 113.549168975069
191 113.941229681204
192 114.265516493056
193 114.069048833526
194 113.481480497396
195 113.940407626561
196 114.384084756352
197 114.547501868123
198 114.098459340884
199 113.568596752607
200 113.5956
201 114.655825350858
202 114.484584844623
203 113.966948967459
204 114.666282199154
205 114.282831647829
206 114.545692336695
207 114.579710144928
208 114.071907359467
209 113.91305144113
210 113.375079365079
211 113.894341995912
212 113.520291918832
213 113.354801736869
214 113.590269892567
215 114.436906435911
216 113.952932098765
217 113.506381532842
218 113.063210167494
219 112.992347949376
220 114.087169421488
221 113.745869249196
222 113.406074993913
223 113.559331577148
224 114.355149872449
225 114.066488888889
226 114.423545305036
227 115.328649886472
228 115.697656971376
229 115.937644209683
230 115.587599243856
231 115.614549952212
232 115.154503567182
233 115.004439204995
234 115.73343560523
235 115.594676324129
236 115.123437948865
237 115.338069041642
238 114.964144481322
239 114.545788764202
240 114.21109375
241 114.252061775796
242 114.384058465952
243 113.976765059527
244 113.7655200215
245 113.362532278217
246 113.497455218455
247 113.610745955515
248 114.331019120708
249 114.200222577055
250 113.743424
251 113.545753242012
252 114.108024691358
253 114.328047618304
254 114.270398040796
255 114.584021530181
256 114.614990234375
257 115.167072930703
258 114.72443062316
259 114.316542687199
260 114.738032544379
261 114.331292846552
262 114.133616922091
263 113.948589686131
264 113.817478764922
265 113.401409754361
266 112.979761433659
267 112.619730954285
268 112.212965025618
269 111.982753140504
270 112.644224965706
271 113.417763919337
272 113.031993403979
273 112.622308094836
274 113.377963663488
275 113.691451239669
276 113.899220226843
277 113.661562121232
278 114.161171781999
279 113.972816382112
280 114.697755102041
281 114.867972796697
282 114.647603239274
283 114.403001660653
284 115.230807379488
285 115.448519544475
286 115.660374590445
287 115.961927424152
288 116.536446277006
289 117.345026999198
290 117.380606420927
291 117.300834898029
292 116.917374272847
293 116.536406947082
294 116.604296820769
295 116.443826486642
296 116.050755569759
297 115.950707977644
298 116.065627674429
299 116.556011677722
300 116.370622222222
301 116.000132448869
302 115.705243191088
303 116.184753128778
304 116.075787742382
305 116.01558720774
306 116.563341022684
307 116.401235026366
308 116.276227019734
309 116.923325059436
310 116.604120707596
311 116.422576276093
312 116.049638395792
313 115.70284477743
314 115.438770335511
315 115.260730662635
316 114.89749238904
317 114.544019743454
318 114.218276571338
319 114.217666886135
320 113.882802734375
321 113.528401315981
322 113.753028432545
323 114.358174620671
324 114.137517146776
325 114.065325443787
326 114.052090782491
327 113.83370273733
328 114.415377751338
329 114.192773533134
330 113.910459136823
331 113.833663438632
332 113.968092248512
333 113.63321880439
334 113.312632220589
335 113.009668077523
336 112.843245110544
337 112.635120499432
338 113.207870872869
339 112.891917056065
340 112.596505190311
341 112.354004523525
342 112.026332888752
343 112.394699487458
344 112.158320375879
345 111.83915984037
346 111.829362825353
347 111.948841033478
348 111.659557074911
349 112.013119760921
350 112.082448979592
351 111.764352562073
352 111.466813016529
353 111.182787760114
354 111.251500207476
355 111.255179527871
356 111.539286390607
357 111.30971604328
358 111.031592334821
359 110.780533980959
360 110.479598765432
361 110.842473584457
362 111.378201214859
363 111.156493560701
364 110.932465885763
365 111.121636329518
366 111.464003105497
367 111.510665310456
368 112.021000708885
369 111.717628395796
370 111.420080350621
371 111.261949564447
372 111.036846745288
373 110.759539707753
374 110.567445451686
375 111.156451555556
376 110.962171797193
377 111.198643485847
378 111.292517006803
379 111.750892850927
380 112.294037396122
381 112.454516020143
382 112.413338450152
383 112.219225708813
384 112.086561414931
385 111.968871647833
386 112.272738865473
387 112.410071510126
388 112.158359017962
389 111.878668525849
390 111.891788297173
391 111.77704227471
392 111.933673469388
393 112.087265051894
394 111.805895539694
395 112.243512257651
396 112.675288235894
397 112.521429613791
398 112.894327163455
399 112.622759907287
400 112.892775
401 112.611314606252
402 112.587559713868
403 112.672943001927
404 112.511291785119
405 112.318463648834
406 112.045014438594
407 111.932037018032
408 112.078425365244
409 111.813332058034
410 111.662938726948
411 111.94270694585
412 111.709038316524
413 111.524180830045
414 111.406689771057
415 111.196434896211
416 111.086145525148
417 110.934573204746
418 110.705690116984
419 110.443412830868
420 110.57570861678
421 110.499455543582
422 110.531619909706
423 110.305719028218
424 110.637810386258
425 110.893597231834
426 110.723511648923
427 110.706930877699
428 110.536968294174
429 110.791856162486
430 111.189918875068
431 111.278050828753
432 111.027949245542
433 110.789550320285
434 110.567292786001
435 110.314514466904
436 110.06874421345
437 110.494247757489
438 110.995293050604
439 110.920709211762
440 110.721647727273
441 111.18942210293
442 111.306811899838
443 111.072657695071
444 110.884445256067
445 110.638707233935
446 110.492992016731
447 110.764990565991
448 110.540796396684
449 110.345593523842
450 110.487920987654
451 110.429830728463
452 110.269573772418
453 110.511420064422
454 110.650357080479
455 110.553218210361
456 110.313634964604
457 110.072291464168
458 110.153048187487
459 109.945899250526
460 109.849981096408
461 109.878590821613
462 109.676716141002
463 109.611641608628
464 109.99029707937
465 110.228437969707
466 110.023485420619
467 109.824998051254
468 109.73224852071
469 110.103009169807
470 110.400312358533
471 110.166010791513
472 109.95439528871
473 110.346311832619
474 110.597803058627
475 110.981788365651
476 110.798990184309
477 110.704279454487
478 111.082583463175
479 110.924272470918
480 110.769752604167
481 110.674115343554
482 110.575700315077
483 110.422540282654
484 110.203004405437
485 109.984338399405
486 110.217886839743
487 110.041455670851
488 110.280754501478
489 110.184375274443
490 110.03133694294
491 109.809582671384
492 109.713906239672
493 109.525182164913
494 109.544411480273
495 109.983699622487
496 110.347875747919
497 110.225093012805
498 110.003758003903
499 109.911823647295
500 110.083216
501 109.990087688894
502 109.802098379391
503 110.160713650503
504 109.992000503905
505 110.21932751691
506 110.160493055664
507 110.228547864415
508 110.649218798438
509 110.433872032299
510 110.342026143791
511 110.25005265758
512 110.051696777344
513 110.278110263747
514 110.08074308468
515 110.100164011688
516 110.21631812992
517 110.003135183266
518 110.416928787585
519 110.826719532523
520 111.047895710059
521 111.20950777517
522 111.317596629527
523 111.660511316899
524 111.516734893071
525 111.398037188209
526 111.45984472813
527 111.24834640963
528 111.054490214646
529 110.861303382992
530 111.198081167675
531 111.357386305198
532 111.629610916389
533 111.690259038541
534 111.846768084838
535 111.954916586601
536 111.762638533081
537 111.571014914918
538 111.371249706334
539 111.166559388134
540 111.274060356653
541 111.114571837598
542 110.97598412331
543 111.082533771524
544 111.290167468642
545 111.500580759195
546 111.516835728374
547 111.459514921008
548 111.343981432149
549 111.494102541133
550 111.292879338843
551 111.671700686098
552 111.484322490023
553 111.536952803874
554 111.547589568481
555 111.45677785894
556 111.282412918586
557 111.530332088097
558 111.721101347619
559 111.602625439627
560 111.716808035714
561 111.829468004995
562 111.633100517977
563 111.46026267553
564 111.321799079523
565 111.143187407001
566 110.946909063667
567 111.273449480387
568 111.19549382563
569 111.496536025031
570 111.368408741151
571 111.263908526842
572 111.287679104113
573 111.13992367412
574 111.294774126188
575 111.102530056711
576 110.926646291474
577 111.021112609595
578 110.846161444427
579 110.822214466608
580 110.808831747919
581 110.635085214228
582 110.932703321879
583 110.830371091739
584 110.642870965472
585 110.479000657462
586 110.377779589745
587 110.218986618065
588 110.357698181313
589 110.178265368773
590 110.24170066073
591 110.080347914716
592 110.143158783784
593 109.959550574579
594 109.853915133376
595 110.106077254431
596 109.921354893924
597 109.777261516965
598 109.656125211127
599 109.557203017829
600 109.390530555556
601 109.222599051498
602 109.055178750786
603 108.982896462959
604 108.80440550853
605 108.858919472714
606 109.219150627934
607 109.358250395577
608 109.603780730609
609 109.430825520854
610 109.25841440473
611 109.185805245352
612 109.04898489897
613 109.144607458306
614 108.96868136532
615 109.068395796153
616 109.163599995783
617 109.353671894906
618 109.178226558163
619 109.106036365914
620 109.291207075963
621 109.116981856182
622 108.943311173375
623 109.285077899358
624 109.576365774984
625 109.40713984
626 109.258523104247
627 109.597719832421
628 109.479073694673
629 109.331459580782
630 109.347079868985
631 109.175675166578
632 109.509651398013
633 109.4346238604
634 109.303476002349
635 109.592346704693
636 109.446370297852
637 109.29852206551
638 109.164847043563
639 109.00905904913
640 109.233200683594
641 109.141829386124
642 109.281557826496
643 109.46520127029
644 109.296574206242
645 109.28404302626
646 109.420094125315
647 109.436653296991
648 109.344307270233
649 109.250414885055
650 109.181824852071
651 109.113140365407
652 109.165463039633
653 108.999946061176
654 109.092877049257
655 108.981565176855
656 108.815434915973
657 109.089570461185
658 109.31150857808
659 109.159631667054
660 109.486306244261
661 109.33433275123
662 109.169175162695
663 109.078374498657
664 108.987625199594
665 108.829631974674
666 108.788231925619
667 108.638671668663
668 108.482000071713
669 108.751490858229
670 108.971621742036
671 109.023345275086
672 109.009032649518
673 108.870357164147
674 109.191128300857
675 109.509522085048
676 109.384300269598
677 109.246536876349
678 109.461873808964
679 109.629985185699
680 109.504636678201
681 109.345499082502
682 109.23899218273
683 109.45586283921
684 109.319557727164
685 109.337149555117
686 109.354174281124
687 109.218168989912
688 109.130273120606
689 109.217687862976
690 109.303810123924
691 109.263283774642
692 109.249396488356
693 109.184535522198
694 109.309746364474
695 109.326088711764
696 109.376403752147
697 109.425584952111
698 109.291968046239
699 109.158667297038
700 109.015663265306
701 109.180355758332
702 109.16788824766
703 109.048313567742
704 108.98435684078
705 108.920517076606
706 109.128652424785
707 109.065045643784
708 109.228167193335
709 109.109677907062
710 109.097165244991
711 108.978776351526
712 108.876396603964
713 108.7460722428
714 108.683088921843
715 108.767830211746
716 108.617318435754
717 108.7012054333
718 108.662013795672
719 108.711906700892
720 108.729164737654
721 108.812383017115
722 108.799449052724
723 108.788006787456
724 108.672497405452
725 108.633866349584
726 108.794748764884
727 108.712327989571
728 108.914767238256
729 108.832645580601
730 109.126719834866
731 109.326024915741
732 109.31440734271
733 109.302446169566
734 109.187268448054
735 109.477599148503
736 109.678454336011
737 109.922613403931
738 110.039146671955
739 110.281344244224
740 110.241890065741
741 110.357317044298
742 110.20858792075
743 110.065729672547
744 110.003380087293
745 110.158360434215
746 110.012048171122
747 109.898145011998
748 110.012503932054
749 110.208213532596
750 110.082659555556
751 109.948067467965
752 110.232074397352
753 110.090855700703
754 110.375315734298
755 110.567915442305
756 110.612432112763
757 110.689222038604
758 110.544560745191
759 110.530043518186
760 110.449556786704
761 110.305787564257
762 110.455142910286
763 110.393325649425
764 110.248835010005
765 110.105897731642
766 110.044538104425
767 110.086651288737
768 110.132973564996
769 110.037469498327
770 110.150718502277
771 110.028214742927
772 109.967655171951
773 109.82655904567
774 109.748828195421
775 109.823573777315
776 110.056593089064
777 110.071535407443
778 109.931399144864
779 110.006219091986
780 110.11476660092
781 110.035428494609
782 109.914490682295
783 109.903465891575
784 109.889211201062
785 109.996095581971
786 110.219580573523
787 110.230043802644
788 110.300310494988
789 110.577024710813
790 110.58679858997
791 110.511187010633
792 110.623118814407
793 110.612185119162
794 110.50547398943
795 110.52082433448
796 110.596328059897
797 110.518021627527
798 110.380941388559
799 110.321569045161
800 110.3102984375
801 110.216595672388
802 110.479693845188
803 110.347535471744
804 110.417458726269
805 110.686846958065
806 110.825380674716
807 110.869191353845
808 111.085726889521
809 110.958768856544
810 111.103209876543
811 111.32585701232
812 111.590288225873
813 111.853082368462
814 111.735532662437
815 111.951832586849
816 112.02274215446
817 111.963412131136
818 111.837285764671
819 111.77806700151
820 111.76242712671
821 111.746531739167
822 111.622043736421
823 111.531084598474
824 111.635331734848
825 111.770872359963
826 111.714121557845
827 111.968134119185
828 111.83394449112
829 111.904970745343
830 111.89187980839
831 111.816899593229
832 111.991261499168
833 112.092435533637
834 111.959226863114
835 111.997323676001
836 112.098086124402
837 112.350791999075
838 112.451051201577
839 112.461193798736
840 112.446415816327
841 112.331746505279
842 112.241057373858
843 112.278697359737
844 112.289077064756
845 112.299093169007
846 112.33586087219
847 112.321911211039
848 112.24750244749
849 112.379580494478
850 112.247370242215
851 112.344550753175
852 112.306685181512
853 112.406383098615
854 112.293847934229
855 112.424130501693
856 112.30366735086
857 112.190817878437
858 112.155483017371
859 112.120008510857
860 112.00852352623
861 111.993794860796
862 112.00307518801
863 111.990243954837
864 111.861689814815
865 112.108237495406
866 112.071776210871
867 112.138147558365
868 112.027517042197
869 112.037390702115
870 112.04814374422
871 112.012627843736
872 112.141865162865
873 112.069179889494
874 111.942037712927
875 112.038726530612
876 112.003423354392
877 112.039751459118
878 111.913309914332
879 111.804276500988
880 111.814740444215
881 111.688943402206
882 111.784498228619
883 111.74971559173
884 111.633479044246
885 111.643852022088
886 111.535749226748
887 111.464882322511
888 111.593365747504
889 111.523054556313
890 111.617654336574
891 111.744019569684
892 111.779066289288
893 111.726347390241
894 111.672183184942
895 111.683637839019
896 111.694763183594
897 111.682123863889
898 111.842953407969
899 111.789557300721
900 111.799995061728
901 111.685873754775
902 111.651865035079
903 111.598937465738
904 111.476621847835
905 111.464087176826
906 111.451287467899
907 111.329483169614
908 111.317175765103
909 111.194714630979
910 111.205396691221
911 111.330362769468
912 111.525073820791
913 111.442150559822
914 111.675053507558
915 111.606514377855
916 111.616782431685
917 111.564874793225
918 111.496713040094
919 111.507014887024
920 111.412873345936
921 111.535514081493
922 111.692916229455
923 111.571905640024
924 111.607141685875
925 111.573979254931
926 111.69592734957
927 111.614550422481
928 111.498661145586
929 111.466050859693
930 111.453467452885
931 111.402718857909
932 111.284241513014
933 111.174539827614
934 111.235900251732
935 111.121320026309
936 111.012363941851
937 111.201215532667
938 111.082665790754
939 110.981054097611
940 111.171688546854
941 111.326176394525
942 111.293729517988
943 111.414629648164
944 111.534767981543
945 111.654148540074
946 111.604204193466
947 111.827323320796
948 111.945186624295
949 112.166162373793
950 112.131855955679
951 112.041099025764
952 111.98924753725
953 111.924385246127
954 111.935609306945
955 111.845267399468
956 111.745522662418
957 111.629925893897
958 111.551663608509
959 111.4856412169
960 111.378514539931
961 111.345669454187
962 111.353963070699
963 111.265123160252
964 111.15058603502
965 111.186411447287
966 111.07248520076
967 110.96732182712
968 110.902376033058
969 110.796787087004
970 110.699633329791
971 110.815045166682
972 110.701903292181
973 110.597890209342
974 110.548672043986
975 110.473267587114
976 110.658743239385
977 110.583085479854
978 110.668553786577
979 110.591362431282
980 110.495471678467
981 110.64110661176
982 110.537047714254
983 110.528138062215
984 110.479439321832
985 110.451105671365
986 110.482990055503
987 110.627115007766
988 110.746008785589
989 110.638464975192
990 110.552883379247
991 110.456754585416
992 110.44427090108
993 110.333859676345
994 110.5098123955
995 110.398761647433
996 110.617226939888
997 110.509888743462
998 110.501336340015
999 110.649714779845
1000 110.7059
};
\addplot [semithick, white!49.8039215686275!black, forget plot]
table {%
1 0
2 64
3 42.8888888888889
4 155.6875
5 128.16
6 106.888888888889
7 113.387755102041
8 100.5
9 124.987654320988
10 112.6
11 109.057851239669
12 113.243055555556
13 105.621301775148
14 101.69387755102
15 99.04
16 92.87109375
17 92.3391003460207
18 87.2746913580247
19 91.180055401662
20 86.6275
21 82.7528344671202
22 88.4462809917355
23 84.6351606805293
24 87.9930555555555
25 93.6
26 90.2899408284024
27 90.2962962962963
28 92.5969387755102
29 93.5909631391201
30 97.8888888888889
31 99.7398543184184
32 97.80859375
33 95.9247015610652
34 93.2318339100346
35 95.9624489795918
36 93.8333333333333
37 92.1417092768444
38 93.7292243767313
39 91.7633136094674
40 96.5775
41 99.8286734086853
42 104.18820861678
43 103.866955110871
44 103.150826446281
45 103.295802469136
46 104.843572778828
47 110.078768673608
48 110.020833333333
49 108.826322365681
50 110.0144
51 111.026528258362
52 109.927514792899
53 108.267710929156
54 111.126543209877
55 109.736859504132
56 112.359375
57 110.840874115112
58 110.023781212842
59 108.558460212583
60 106.76
61 105.033055630207
62 104.869146722164
63 104.290753338372
64 104.453125
65 105.260591715976
66 105.332644628099
67 103.78970817554
68 102.268814878893
69 104.172652804033
70 104.044081632653
71 107.758778020234
72 106.610339506173
73 105.269281291049
74 103.903761869978
75 107.759288888889
76 107.32271468144
77 106.634845673807
78 107.514792899408
79 106.154141964429
80 105.15484375
81 103.871056241427
82 103.207168352171
83 105.006241834809
84 105.807256235828
85 106.825743944637
86 105.847079502434
87 108.848460827058
88 110.87590392562
89 110.774144678702
90 110.426666666667
91 113.081753411424
92 112.159262759924
93 113.580992022199
94 115.61385242191
95 116.94404432133
96 115.748263888889
97 115.670953342544
98 114.625156184923
99 114.170798898072
100 113.3075
101 113.318498186452
102 114.896674356017
103 114.069940616458
104 115.239552514793
105 114.14693877551
106 113.626201495194
107 114.043846624159
108 113.372685185185
109 115.415705748674
110 114.777190082645
111 116.117847577307
112 115.211415816327
113 114.281462918005
114 115.45837180671
115 117.343364839319
116 118.758917954816
117 120.278033457521
118 119.327204826199
119 118.641621354424
120 119.799722222222
121 120.419916672358
122 119.569470572427
123 120.909908123471
124 120.553850156087
125 121.2064
126 120.905076845553
127 120.464008928018
128 119.554626464844
129 119.003906015264
130 118.569467455621
131 119.175339432434
132 118.2802456382
133 117.757363333145
134 117.060425484518
135 117.237640603567
136 116.502541089965
137 117.539666471309
138 118.093520268851
139 118.877490813105
140 118.600969387755
141 120.280971782104
142 120.163261257687
143 120.711526236002
144 120.437065972222
145 119.606468489893
146 119.77017264027
147 119.938359017076
148 121.072132943755
149 120.315841628755
150 120.981155555556
151 122.261216613306
152 121.459790512465
153 120.850783886539
154 120.750548153146
155 120.184640998959
156 119.42451512163
157 119.535721530285
158 119.494151578273
159 118.743641469879
160 118.0733984375
161 118.173295783342
162 118.741960067063
163 120.090556663781
164 119.602171326591
165 120.895720844812
166 120.673573813326
167 120.529814622252
168 119.87287414966
169 119.437274605231
170 120.272975778547
171 120.856263465682
172 120.153866955111
173 119.808145945404
174 119.308230942
175 119.786253061224
176 119.323217975207
177 120.434932490664
178 119.761299078399
179 120.071283667801
180 119.775802469136
181 120.324776410976
182 119.992905446202
183 119.458269879662
184 119.499970463138
185 118.975602629657
186 119.202480055498
187 118.594126226086
188 119.04920212766
189 119.488032249937
190 118.935761772853
191 118.813793481538
192 119.399305555556
193 120.127842358184
194 121.185460729089
195 120.58419460881
196 119.989691795085
197 119.884614393568
198 119.32621671258
199 118.745688240196
200 118.9856
201 118.900621271751
202 118.312028232526
203 119.134315319469
204 119.934832756632
205 119.722688875669
206 119.144028654916
207 119.756260356134
208 120.034185465976
209 119.484901902429
210 120.331428571429
211 120.999034163653
212 120.431893022428
213 119.912186735436
214 120.559350161586
215 120.376722552731
216 120.597650891632
217 121.377689056892
218 121.045534887636
219 121.25881445341
220 121.71520661157
221 121.179951270449
222 120.803912020128
223 120.265599549559
224 119.744240274235
225 120.189313580247
226 121.117393687838
227 120.589376855751
228 120.34831871345
229 119.859766213459
230 119.499508506616
231 119.51035400386
232 119.033293697979
233 118.627788318075
234 119.358682153554
235 118.955980081485
236 118.719387388681
237 119.443037974684
238 120.296801073371
239 120.385917613487
240 119.903055555556
241 120.320862244107
242 120.015162898709
243 119.77019085844
244 119.465919779629
245 119.745006247397
246 119.777248992002
247 119.718139946565
248 120.293899583767
249 120.469282753504
250 120.028736
251 119.563276773385
252 119.734867094986
253 120.179099814089
254 120.452166904334
255 120.566920415225
256 120.164978027344
257 120.332904358885
258 120.885418544559
259 120.989460502974
260 120.581346153846
261 120.961025234509
262 120.730260474331
263 120.98873772933
264 121.196840564738
265 121.558647205411
266 122.025496071005
267 122.448035461291
268 122.272777901537
269 121.85926120424
270 121.800617283951
271 121.570403453112
272 121.87314824827
273 121.508942827624
274 121.654656614631
275 121.676350413223
276 121.576822096198
277 122.139712494624
278 122.69137208219
279 122.569314371604
280 122.531581632653
281 122.09665531085
282 121.664365474574
283 121.435902558404
284 121.397887323944
285 120.99378270237
286 120.767176879065
287 120.980830166689
288 120.63175154321
289 120.841488966847
290 120.621224732461
291 120.207460941652
292 120.421127322199
293 120.113408426423
294 120.518360405387
295 120.357437517955
296 120.56738495252
297 120.865648629958
298 121.063972343588
299 121.276272077494
300 121.115555555556
301 120.713899405084
302 120.49810315337
303 120.142099358451
304 119.766609677978
305 120.257608169847
306 119.907054978854
307 119.524769493576
308 119.269554309327
309 119.378473204093
310 119.016024973985
311 119.521344899246
312 119.728457840237
313 120.315895844604
314 119.938912329101
315 119.91427563618
316 120.285400977407
317 119.923713043219
318 120.312497527788
319 120.000943386956
320 119.646474609375
321 119.33830222921
322 119.454930365341
323 119.363283459057
324 119.086753162628
325 118.994120710059
326 119.356101095261
327 118.996829672025
328 118.730480368828
329 118.931384595486
330 118.606437098255
331 118.2832394739
332 117.934605893453
333 117.63866569272
334 117.760631431747
335 117.415228335932
336 117.083297902494
337 117.132421699583
338 116.873717656945
339 117.06969135319
340 117.340302768166
341 117.013579174586
342 117.661819021237
343 117.338778910148
344 117.597451325041
345 117.719588321781
346 118.274090347155
347 118.377264157995
348 118.037546241247
349 117.86463165327
350 117.651395918367
351 117.828929959984
352 117.875637590393
353 118.226612844979
354 117.909867854065
355 117.947026383654
356 117.990436813534
357 117.910238605246
358 118.244904965513
359 118.59181725778
360 118.420709876543
361 118.177423439047
362 117.885168340405
363 117.57777625997
364 117.260981463591
365 117.136993807469
366 117.311624712592
367 117.009822628425
368 116.775512464556
369 117.379969301048
370 117.146975894814
371 117.18206057788
372 116.918805642271
373 116.951721064623
374 117.267436872659
375 117.073749333333
376 117.306919137619
377 117.493023943038
378 117.337567537303
379 117.366768541015
380 117.357472299169
381 117.329585770283
382 117.126648118199
383 117.383484787544
384 117.151848687066
385 117.538026648676
386 117.410809686166
387 117.1152107579
388 117.265782761186
389 116.991111610418
390 116.699204470743
391 116.722352679535
392 116.556480372761
393 116.296913544277
394 116.245336133371
395 116.749726005448
396 116.605773645546
397 116.411156723284
398 116.14537511679
399 116.514946514155
400 116.2359
401 115.946144613529
402 115.753397193139
403 115.474240959553
404 115.425620037251
405 115.678463648834
406 116.212702322308
407 116.235401360708
408 116.044093617839
409 116.226768132663
410 115.978322427127
411 115.70387340828
412 115.628823404656
413 115.460652287344
414 115.412565054027
415 115.731769487589
416 115.507113304364
417 115.748057668973
418 115.515716215288
419 115.367285444945
420 115.538837868481
421 115.271037739575
422 115.175580063341
423 115.207160382051
424 115.139996440014
425 115.309541868512
426 115.045361370099
427 114.869790324084
428 114.776896453839
429 114.515732907341
430 114.411081665765
431 114.392536646551
432 114.372937028464
433 114.109990452773
434 113.898240565737
435 114.041664684899
436 113.917115562663
437 113.660175211684
438 113.442343779321
439 113.303957534467
440 113.539746900826
441 113.772234819854
442 113.521221924203
443 113.553638489878
444 113.360908205503
445 113.10634515844
446 113.520988759074
447 113.283976197268
448 113.062420280612
449 112.814132866404
450 112.896498765432
451 112.834607499471
452 112.796616806328
453 112.570598755415
454 112.437307147432
455 112.337901219659
456 112.347294359803
457 112.8256970347
458 112.724876527908
459 112.489384424794
460 112.860699432892
461 112.64735249693
462 112.586144375105
463 112.345954872206
464 112.738030432521
465 112.576371834894
466 112.554205271786
467 112.387548202798
468 112.355810322156
469 112.541323234573
470 112.304463558171
471 112.075387326959
472 112.307450265728
473 112.200707105471
474 112.000182485001
475 112.293026038781
476 112.402178518466
477 112.367557542116
478 112.13910820889
479 112.000714780706
480 111.848224826389
481 111.652689952066
482 111.528795991805
483 111.404969801405
484 111.513438289734
485 111.523035391646
486 111.653135531508
487 111.620439433484
488 111.838076961838
489 112.289543787455
490 112.21742607247
491 112.017662113563
492 112.325236301144
493 112.143814621743
494 112.384844858955
495 112.262038567493
496 112.070788078824
497 112.001951345902
498 112.056789406622
499 111.867655149979
500 111.8516
501 112.286731925371
502 112.069903652323
503 111.876549846053
504 111.657214506173
505 111.639823546711
506 111.452877720321
507 111.476660092045
508 111.304339233678
509 111.713062710118
510 111.768938869666
511 111.550269798293
512 111.561019897461
513 111.462178296076
514 111.888976366031
515 111.737558676595
516 111.58647316868
517 111.87688980841
518 111.70600095407
519 111.542205441768
520 111.455562130177
521 111.241676828482
522 111.571706962611
523 111.358437313777
524 111.190213274285
525 110.981101133787
526 110.772773930518
527 110.891703783184
528 110.696955348944
529 110.559360493995
530 110.359473122108
531 110.68550615156
532 110.739622788173
533 111.060322645368
534 111.088179803336
535 111.499309983405
536 111.567439156828
537 111.374724745032
538 111.26282458783
539 111.058835677972
540 110.859533607682
541 111.008627140129
542 111.033744774717
543 111.370125047058
544 111.2135292766
545 111.037151754903
546 110.977968307639
547 111.178280065105
548 111.023949064948
549 111.278297019585
550 111.186895867769
551 110.98769766898
552 111.359952609746
553 111.428407927824
554 111.676826884229
555 111.524856748641
556 111.373256430827
557 111.588117931081
558 111.480534037333
559 111.490401016382
560 111.441961096939
561 111.269321081212
562 111.209394511214
563 111.477141297729
564 111.570805668729
565 111.491938288041
566 111.43219418397
567 111.59038100837
568 111.419708391192
569 111.34116833096
570 111.723496460449
571 112.025389444886
572 112.034634945474
573 112.12551434689
574 112.506285738567
575 112.567712665406
576 112.402099609375
577 112.493390482655
578 112.749955101112
579 112.732320927333
580 112.919176575505
581 112.938171174988
582 113.123702483438
583 113.212431117218
584 113.345456464628
585 113.577967711301
586 113.487204277278
587 113.33247622392
588 113.587703618863
589 113.460343997625
590 113.430361964953
591 113.43465576427
592 113.610410769722
593 113.507685220205
594 113.760557312746
595 113.656962078949
596 113.740124318724
597 113.650435314484
598 113.957027885594
599 113.77206863972
600 113.996663888889
601 113.807924119811
602 113.664575998057
603 113.521551996788
604 113.824423819131
605 113.865910798443
606 114.037741397902
607 114.138515778303
608 114.090547091413
609 113.932269380206
610 113.753442622951
611 113.772233547001
612 113.743589538212
613 113.934209580886
614 113.75632632707
615 114.049906801507
616 114.065947039973
617 114.355523800267
618 114.399422398173
619 114.270471159643
620 114.186043184183
621 114.19188312446
622 114.197095253358
623 114.013840759129
624 114.182928583169
625 114.0128
626 114.10856495422
627 113.927647159075
628 114.06591291736
629 114.159366698598
630 114.240385487528
631 114.472396844493
632 114.487021310687
633 114.321451300136
634 114.164975768492
635 114.207278814558
636 114.287577627467
637 114.45361769874
638 114.406029814959
639 114.232263341832
640 114.157802734375
641 114.101620663891
642 113.93949010588
643 113.85768498654
644 113.811376393658
645 113.634964244937
646 113.618593583759
647 113.479375742041
648 113.376521776406
649 113.691354009131
650 113.61015147929
651 113.494621296316
652 113.773175505288
653 113.611227717989
654 113.439600108483
655 113.769584523046
656 113.992675490779
657 113.844915660641
658 113.728540941048
659 113.770512640433
660 113.600025252525
661 113.467711554263
662 113.790929254023
663 113.743507845185
664 113.786305614022
665 113.629496297134
666 113.63762185609
667 113.808473574307
668 113.662922657679
669 113.517732778325
670 113.349986633994
671 113.181065251721
672 113.445153061225
673 113.290259621265
674 113.243006454226
675 113.081051303155
676 112.96658030181
677 113.279164093915
678 113.113567146126
679 113.041130832182
680 113.351208910035
681 113.190721082627
682 113.354761310962
683 113.399012624092
684 113.41071398037
685 113.421863711439
686 113.349820227966
687 113.469948238126
688 113.480172982017
689 113.640230788189
690 113.594120982987
691 113.672481208676
692 113.521707624712
693 113.363355259459
694 113.252782184056
695 113.415156565395
696 113.304300435989
697 113.143222953877
698 113.393806290589
699 113.403181737246
700 113.241181632653
701 113.488625379273
702 113.340453811252
703 113.201107223867
704 113.209420196281
705 113.053971128213
706 113.207884261971
707 113.061512576798
708 112.954578824731
709 113.075568800094
710 113.272438008332
711 113.113140700386
712 113.041590392627
713 112.889346124567
714 112.818131174038
715 112.681273411903
716 112.920623185918
717 113.042302013854
718 113.152111249913
719 113.080162720205
720 112.934228395062
721 112.950471394138
722 112.794064272067
723 112.780534464321
724 112.79600744788
725 112.673133888228
726 112.755207977597
727 112.873810140976
728 112.991335587489
729 113.183661027282
730 113.049435166072
731 113.166540971366
732 113.205627220878
733 113.285025377404
734 113.297811996525
735 113.176713406451
736 113.466222752245
737 113.334494420705
738 113.345994814962
739 113.19405040275
740 113.343579254931
741 113.190622148645
742 113.300041412079
743 113.258058614362
744 113.105834128223
745 112.95521102653
746 112.939178747781
747 113.015417314058
748 112.865731076096
749 112.84773467427
750 112.781696
751 112.6522896236
752 112.568553007583
753 112.644712870519
754 112.794385734087
755 112.983332309986
756 112.835131995185
757 113.06864857979
758 112.966807875189
759 112.949168606498
760 112.84829466759
761 112.806770951148
762 112.880493038764
763 112.890662333446
764 112.849337668924
765 112.734305608954
766 112.670193402368
767 112.742376621014
768 112.643105400933
769 112.830663503342
770 112.813393489627
771 112.959684813127
772 113.144970334774
773 113.154635172519
774 113.013475752659
775 112.86901144641
776 112.787090485174
777 112.724540315274
778 112.70785449475
779 112.690882784782
780 112.55169460881
781 112.412849346106
782 112.351386045356
783 112.310832521869
784 112.418263223657
785 112.28019635685
786 112.549385233961
787 112.468286271996
788 112.452915238733
789 112.633977006398
790 112.523409709982
791 112.562206619667
792 112.440669319457
793 112.298879381219
794 112.177927656416
795 112.057228748863
796 112.097888942198
797 112.137387851872
798 112.241292454193
799 112.380469328839
800 112.44899375
801 112.368359775
802 112.511675922413
803 112.579588684401
804 112.451374965966
805 112.557632807376
806 112.663017443615
807 112.601736048731
808 112.507689197138
809 112.76653103757
810 112.637904282884
811 112.499023142031
812 112.566029265452
813 112.427581323784
814 112.607060712712
815 112.487831683541
816 112.559182465879
817 112.440411752104
818 112.511289387318
819 112.392974546821
820 112.257006246282
821 112.266075802511
822 112.205989782206
823 112.214489561203
824 112.227348242059
825 112.102764370983
826 112.355716748061
827 112.488229041319
828 112.45251201895
829 112.317839011351
830 112.421802874147
831 112.365071297105
832 112.305421366494
833 112.201594203107
834 112.209921271616
835 112.093814765678
836 112.106824191296
837 112.0017414416
838 112.176593890443
839 112.273922783949
840 112.184302721088
841 112.081540434424
842 112.121316456125
843 112.190347133395
844 112.13235327149
845 112.304227443017
846 112.228784881154
847 112.21518547997
848 112.087429634656
849 112.030102621944
850 112.16124567474
851 112.261430183057
852 112.130743183672
853 112.073418553097
854 111.943234482721
855 112.188548955234
856 112.172433181937
857 112.084719292967
858 112.092920870893
859 112.058551446642
860 111.947598702001
861 111.959914800741
862 112.202804948294
863 112.115641225669
864 112.077910665295
865 112.171833338902
866 112.114214700596
867 112.060735224275
868 111.942718840918
869 112.042425390082
870 112.076661381953
871 112.08429283416
872 112.322358176921
873 112.519292665677
874 112.652865124706
875 112.525583673469
876 112.439996820333
877 112.598841026668
878 112.524094416281
879 112.422806969861
880 112.311827221074
881 112.195668166785
882 112.209425342321
883 112.083176753808
884 111.961584529391
885 112.030024577867
886 112.0624882165
887 112.184783589592
888 112.152786654898
889 112.345737997598
890 112.313810124984
891 112.434772969752
892 112.351394558507
893 112.255432008818
894 112.421653929503
895 112.542425017946
896 112.418316276706
897 112.301987673516
898 112.389978224314
899 112.351476922201
900 112.540424691358
901 112.420906108763
902 112.296316143972
903 112.395065543795
904 112.30045300141
905 112.526850828729
906 112.767852530834
907 112.968575905398
908 113.194249839896
909 113.223237615291
910 113.10239826108
911 112.98353216752
912 112.864915166205
913 112.83371382573
914 112.872416195433
915 112.816033921586
916 112.944027573845
917 112.824551159547
918 112.806978322677
919 113.043169173097
920 113.072703213611
921 113.294919722107
922 113.478386841771
923 113.605577460094
924 113.548631022657
925 113.475887216947
926 113.378175015977
927 113.613095799164
928 113.594307372176
929 113.483030354294
930 113.364505723205
931 113.242780881927
932 113.270601549117
933 113.157535592064
934 113.037045655673
935 113.270547055964
936 113.298294935715
937 113.18570017848
938 113.239266279022
939 113.220365625861
940 113.43536781349
941 113.530449552277
942 113.689503969059
943 113.593148825582
944 113.71810251185
945 113.606002071611
946 113.567468902109
947 113.496022006916
948 113.560686277128
949 113.446183159912
950 113.55672465374
951 113.46602005084
952 113.427688722548
953 113.311946919707
954 113.469096247072
955 113.350923494422
956 113.243168011765
957 113.434127241499
958 113.576184073466
959 113.520933889033
960 113.491141493056
961 113.632203274208
962 113.54869230337
963 113.586011609187
964 113.622931767704
965 113.505838009074
966 113.436038561612
967 113.346830087831
968 113.328647932177
969 113.365641384466
970 113.248813901584
971 113.370103760867
972 113.288142051517
973 113.278475677834
974 113.17028890791
975 113.069090335306
976 112.954699677506
977 113.094579630373
978 113.06556199581
979 112.968065848602
980 113.190265514369
981 113.142964022856
982 113.132927314886
983 113.193781570524
984 113.08878540386
985 112.977478419954
986 112.886796489597
987 112.891617368239
988 112.78078131915
989 112.816128065955
990 112.876628915417
991 112.786582776777
992 112.690149217937
993 112.638879000942
994 112.526296612674
995 112.579122749426
996 112.631509975646
997 112.68346061253
998 112.734976967964
999 112.796273751229
1000 112.822975
};
\addplot [semithick, color7, forget plot]
table {%
1 0
2 49
3 104.666666666667
4 90.5
5 91.76
6 83.6666666666667
7 77.7142857142857
8 68.4375
9 60.8395061728395
10 60.84
11 62.6115702479339
12 78.0763888888889
13 72.1538461538462
14 70.25
15 82.5066666666667
16 87.87109375
17 95.8961937716263
18 103.126543209877
19 112.775623268698
20 113.39
21 109.569160997732
22 118.057851239669
23 116.431001890359
24 113.659722222222
25 111.04
26 112.273668639053
27 108.913580246914
28 107.479591836735
29 110.432818073722
30 113.672222222222
31 118.851196670135
32 121.3193359375
33 123.531680440771
34 121.602941176471
35 118.968163265306
36 117.543209876543
37 120.371073776479
38 118.036703601108
39 117.792241946088
40 115.294375
41 113.590719809637
42 112.914399092971
43 113.494862087615
44 113.906508264463
45 118.151111111111
46 121.922967863894
47 119.680398370303
48 118.485677083333
49 117.287796751354
50 115.08
51 112.826605151865
52 110.764792899408
53 109.024563901744
54 107.964677640604
55 106.616859504132
56 105.110969387755
57 106.980609418283
58 106.069262782402
59 104.931341568515
60 103.193333333333
61 107.823165815641
62 110.417273673257
63 108.947845804989
64 108.040771484375
65 110.218698224852
66 108.629247015611
67 107.487636444642
68 108.217128027682
69 110.106280193237
70 109.511224489796
71 112.869271969847
72 113.805555555556
73 112.540814411709
74 112.751643535427
75 114.514488888889
76 114.452042936288
77 115.333783100017
78 115.228139381986
79 113.790097740747
80 113.019375
81 112.089010821521
82 111.170136823319
83 111.834809115982
84 111.72264739229
85 113.40678200692
86 112.390481341266
87 114.409301096578
88 113.40444214876
89 112.314101754829
90 113.530987654321
91 113.553677092139
92 114.838728733459
93 114.438663429298
94 114.224535989135
95 113.583379501385
96 112.685763888889
97 113.756403443512
98 113.142857142857
99 113.372104887256
100 114.9024
101 114.43034996569
102 113.954632833526
103 113.330002827788
104 115.329049556213
105 115.552290249433
106 115.746618013528
107 116.641977465281
108 117.239626200274
109 116.689335914485
110 116.139586776859
111 115.892865838812
112 115.218112244898
113 116.939776020049
114 117.493459526008
115 116.487712665406
116 118.360582639715
117 119.304112791292
118 118.697787991956
119 118.249558646988
120 117.674930555556
121 116.998292466362
122 116.039303950551
123 116.450128891533
124 116.453368886576
125 116.911616
126 116.603489543966
127 115.798251596503
128 115.194274902344
129 114.362478216453
130 113.617514792899
131 114.199289085718
132 113.5020087236
133 113.349200067839
134 112.933002895968
135 114.083731138546
136 114.418685121107
137 115.535191006447
138 115.615259399286
139 115.468039956524
140 116.523010204082
141 117.24651677481
142 118.506298353501
143 118.351606435523
144 117.888888888889
145 117.080618311534
146 117.458997935823
147 116.717478828266
148 118.078661431702
149 117.544344849331
150 117.447155555556
151 116.768738213236
152 116.513677285319
153 116.052629330599
154 116.045707539214
155 116.874672216441
156 116.249506903353
157 116.222240253154
158 115.616728088447
159 115.581108342233
160 116.36
161 116.770572122989
162 117.530140222527
163 117.624148443675
164 117.090868530636
165 116.50189164371
166 115.814486863115
167 115.195668543153
168 114.761585884354
169 114.118763348622
170 113.460069204152
171 114.011969494887
172 113.779171173607
173 113.135754619266
174 114.028702602722
175 114.107624489796
176 113.468459452479
177 113.970889591114
178 113.338467365232
179 113.618363971162
180 113.064783950617
181 113.267604773969
182 114.186692428451
183 113.852070829228
184 114.661832466919
185 114.461358655953
186 113.84622499711
187 113.647916726243
188 113.653463105478
189 113.334061196495
190 112.746149584488
191 112.211671829171
192 111.862847222222
193 112.722220730758
194 112.17642682538
195 112.384589086128
196 112.626275510204
197 112.989667345203
198 112.876160595858
199 112.971339107598
200 113.459375
201 113.046013712532
202 113.521713557494
203 113.397995583489
204 113.055940023068
205 112.771112433076
206 112.420044302008
207 112.495694181895
208 112.461446005917
209 112.888990636661
210 113.13126984127
211 113.248175018531
212 113.156706123176
213 113.923207476471
214 113.40014411739
215 112.962595997837
216 112.845593278464
217 113.922317314022
218 113.429930140561
219 113.859802756406
220 113.373450413223
221 114.421080649454
222 113.995292589887
223 113.485008747411
224 112.979731345663
225 113.28987654321
226 113.107702247631
227 113.411903976402
228 113.227531548169
229 112.784233710265
230 112.302533081285
231 113.150203332022
232 113.695600475624
233 113.520160621857
234 113.036032580904
235 113.117066545948
236 113.528781241023
237 113.050935569442
238 112.701239319257
239 112.609723219131
240 112.191041666667
241 112.853910917512
242 113.370688477563
243 112.905654625819
244 113.175272104273
245 113.17524364848
246 113.261484566065
247 113.16156632628
248 112.706750780437
249 112.7039563878
250 112.336064
251 112.114347391311
252 112.388306248425
253 112.898186192567
254 112.462458924918
255 112.185866974241
256 111.866943359375
257 111.860255265031
258 111.54241031188
259 111.1125355913
260 111.034482248521
261 111.112813963389
262 111.997858516403
263 111.595396781795
264 111.383551423324
265 111.555628337487
266 111.144609644412
267 110.751869152324
268 110.495586433504
269 110.242561600862
270 109.857283950617
271 110.226059013358
272 110.586451124567
273 110.76037515598
274 110.680910011189
275 110.611226446281
276 110.683824301617
277 110.293344107182
278 109.942666011076
279 109.947765316478
280 110.108112244898
281 110.04161548106
282 109.724372516473
283 109.44426825157
284 109.374975203333
285 109.826654355186
286 110.402366863905
287 110.333596377278
288 109.958333333333
289 109.599406137379
290 109.943115338882
291 109.870573091957
292 109.533965096641
293 109.199384966627
294 108.829237817576
295 108.84079287561
296 108.912481738495
297 109.472117357639
298 109.622415656952
299 109.688929654031
300 110.107233333333
301 110.164987141422
302 110.422130608307
303 110.973128996068
304 111.051322281856
305 111.274990593926
306 111.493367935409
307 112.033125019894
308 111.677896778546
309 111.334799593636
310 111.017898022893
311 110.80108766452
312 110.467414529915
313 110.297624758852
314 111.039808917197
315 110.865306122449
316 110.743660871655
317 111.357004249221
318 111.356285352636
319 111.574611098554
320 111.226943359375
321 110.898846090391
322 111.291231048185
323 111.519309108685
324 111.742455418381
325 111.461775147929
326 111.120704580526
327 110.800269337598
328 111.391842653183
329 111.684500327972
330 111.68621671258
331 111.354879929902
332 111.845287777616
333 111.727421114808
334 111.788196062964
335 111.473504121185
336 111.470096371882
337 111.602479549877
338 111.397893981303
339 111.45785365599
340 111.165147058824
341 110.846174353506
342 110.643522793338
343 110.586864316739
344 110.58717550027
345 110.273186305398
346 110.162760199138
347 110.105889094669
348 109.996300700225
349 110.553246689272
350 110.67693877551
351 110.444022369948
352 110.130649535124
353 110.343506488295
354 110.47844648728
355 110.322872445943
356 110.31725792198
357 110.013668212383
358 109.760034018913
359 109.455373561658
360 109.827283950617
361 109.681678317386
362 109.962432465431
363 109.743111050399
364 110.280785835044
365 110.059898667667
366 109.909948042641
367 109.617266443436
368 109.672967863894
369 109.876278816989
370 109.632169466764
371 109.751077077324
372 109.463059313215
373 109.40519949112
374 109.230175298121
375 109.047139555556
376 109.49133516297
377 109.387880024485
378 109.503184401333
379 109.834462305331
380 110.425484764543
381 110.43743154153
382 110.481209396672
383 110.837772430107
384 110.745598687066
385 111.235594535335
386 110.9665293028
387 111.096301637856
388 110.987937081518
389 110.854686395147
390 110.707435897436
391 110.975543069446
392 110.764935183257
393 110.564859597667
394 110.331888221804
395 110.900778721359
396 110.734440363228
397 110.860889923799
398 110.586778111664
399 110.777017732301
400 110.68619375
401 110.48866611526
402 110.214208806713
403 110.333971639503
404 110.515758258994
405 110.512385307118
406 110.514456793419
407 110.761042928119
408 110.666372308727
409 110.616710803977
410 110.362195121951
411 110.230202283908
412 109.992005608446
413 109.79945945629
414 109.70779364746
415 109.449092756568
416 109.501479289941
417 109.309743572049
418 109.076303198187
419 109.311134021793
420 109.307120181406
421 109.092568875147
422 109.255480559736
423 109.318947739047
424 109.188145247419
425 108.975247058824
426 109.224123300051
427 109.324945565434
428 109.287683422133
429 109.677778321135
430 109.449329367226
431 109.834863076749
432 109.62910986797
433 109.503064179765
434 109.280362717407
435 109.198012947549
436 108.962566282299
437 108.728128649152
438 109.030634265341
439 109.329258357937
440 109.696508264463
441 110.21714203444
442 110.517188427755
443 110.96264439564
444 110.914429632335
445 110.964150991036
446 110.960993585232
447 111.317157885781
448 111.681042729592
449 111.782014970164
450 111.60027654321
451 111.816126764372
452 111.574124833581
453 111.853086365608
454 111.61946185643
455 111.978215191402
456 112.025084641428
457 111.824859108734
458 112.172026658531
459 112.515110522544
460 112.732703213611
461 112.829903868324
462 112.779052116714
463 112.654814828637
464 112.47885237069
465 112.748368597526
466 112.664632798541
467 112.753132895286
468 112.574293228139
469 112.660598924355
470 112.44905387053
471 112.449889785928
472 112.850886957771
473 113.176387504526
474 113.257010094536
475 113.018752354571
476 113.337670362263
477 113.439816462956
478 113.517467306245
479 113.567749443212
480 113.489166666667
481 113.490752546886
482 113.569657030699
483 113.553575179284
484 113.329635100061
485 113.575784886811
486 113.918935968433
487 113.966791612732
488 113.771025094061
489 113.595451675093
490 113.503190337359
491 113.379187907799
492 113.255254808646
493 113.260256162338
494 113.359209296989
495 113.6008815427
496 113.725009755463
497 113.666951406629
498 113.56052724956
499 113.896161059594
500 113.6964
501 113.96298022717
502 114.142013618831
503 114.539300973483
504 114.483241213152
505 114.554549553965
506 114.417878735803
507 114.195371310528
508 114.088082801166
509 114.209818550955
510 114.06245290273
511 114.419950903987
512 114.2412109375
513 114.411849419954
514 114.615092582779
515 114.636501083985
516 114.782795505078
517 114.691371511734
518 114.504092067799
519 114.727818800792
520 114.50850591716
521 114.729757111122
522 114.582144272691
523 114.744198969762
524 115.068906240895
525 115.229699773243
526 115.160581329786
527 115.091157207206
528 115.290371757346
529 115.565910642115
530 115.709537201851
531 115.801142711226
532 116.138005964159
533 115.920158823467
534 116.319463732133
535 116.411284828369
536 116.221708621074
537 116.020307314587
538 116.110432415251
539 115.944816381604
540 116.275432098765
541 116.174360481207
542 115.961319290315
543 115.796716285285
544 115.626997053417
545 115.429610302163
546 115.47425834239
547 115.663572954022
548 115.513489663807
549 115.493153639172
550 115.759804958678
551 115.847622372785
552 115.669646607856
553 115.756586627601
554 115.797876291884
555 115.621511240971
556 115.454427824647
557 115.329106620811
558 115.369422926221
559 115.431907859998
560 115.295379464286
561 115.589827180264
562 115.913745393295
563 115.710571065309
564 115.625421256476
565 115.605306601927
566 115.922926369414
567 115.744283630233
568 115.926601864709
569 115.905541433341
570 115.707959372115
571 115.857312423898
572 115.74434874566
573 115.5564814561
574 115.867101093858
575 116.159848771267
576 116.410578221451
577 116.225759846694
578 116.591937955724
579 116.435268955766
580 116.234530321046
581 116.375232920865
582 116.283452014029
583 116.574664081509
584 116.481700952336
585 116.299142377091
586 116.12646623723
587 116.009228920768
588 116.354236082188
589 116.436203054874
590 116.436946279805
591 116.245412719272
592 116.300102150292
593 116.396457831531
594 116.312371753449
595 116.122663653697
596 115.972216904644
597 115.985454912755
598 116.280838581224
599 116.094007541785
600 116.032655555556
601 116.044379722094
602 115.988487985784
603 115.875652142824
604 116.051686329547
605 116.093310566218
606 116.143961920945
607 116.035663008992
608 116.26025525883
609 116.171143843982
610 116.057876914808
611 115.906455838273
612 115.719231919347
613 115.856566135046
614 115.9808061624
615 115.900043624826
616 116.229075307809
617 116.267320568758
618 116.080259423341
619 116.34870981128
620 116.467127991675
621 116.63004711636
622 116.478613744688
623 116.476264334796
624 116.374496630506
625 116.534784
626 116.721779338362
627 116.547474238736
628 116.498681488093
629 116.4346870016
630 116.27883345931
631 116.276787530672
632 116.29400086124
633 116.145609188173
634 116.402932161729
635 116.327970735941
636 116.382112554883
637 116.398448378264
638 116.335789251285
639 116.407140460569
640 116.691345214844
641 116.510157442179
642 116.340362088877
643 116.475328275071
644 116.322352050461
645 116.63057268193
646 116.568039567139
647 116.448647783492
648 116.414942463039
649 116.749703823115
650 116.586340828402
651 116.763834913084
652 116.585670800557
653 116.499553245827
654 116.700747692394
655 116.976782238797
656 117.304580606782
657 117.126989753249
658 117.027836032557
659 116.908448677239
660 117.035858585859
661 117.303956550498
662 117.416473471399
663 117.309801191622
664 117.168182519234
665 117.136805924586
666 117.307036766496
667 117.522640478861
668 117.63356520492
669 117.462741391676
670 117.313343729116
671 117.173309405407
672 117.010832979025
673 116.851930435013
674 117.019529977371
675 116.871624691358
676 116.796663282098
677 116.831948229329
678 116.715291374074
679 116.568587175544
680 116.67955017301
681 116.844887776247
682 116.788735906984
683 116.79005078362
684 116.716500376184
685 116.552834993873
686 116.451342552848
687 116.303329244082
688 116.16828141901
689 116.112748330072
690 115.944515858013
691 115.957338616615
692 116.214858498446
693 116.470410141406
694 116.665699823103
695 116.499692562497
696 116.427260866693
697 116.30969372737
698 116.312035615471
699 116.218620101064
700 116.052620408163
701 115.924660307977
702 115.771243739905
703 116.023564929008
704 116.21692398954
705 116.504288516674
706 116.664229710535
707 116.866969824887
708 117.068099843595
709 116.904147958646
710 116.788591549296
711 116.988390195462
712 116.836060787779
713 117.07308667523
714 117.075128090452
715 117.369976037948
716 117.272547283168
717 117.506548477015
718 117.515128296646
719 117.796061985334
720 118.026820987654
721 117.979362920585
722 117.817780710707
723 117.89355096044
724 117.731761850981
725 117.772035196195
726 117.623145428743
727 117.482151405126
728 117.711900736626
729 117.573879320564
730 117.417594295365
731 117.263190988863
732 117.295171548867
733 117.225589952521
734 117.291013000319
735 117.215075200148
736 117.075774973417
737 116.921459067068
738 116.76754907793
739 116.963086934947
740 116.868298027757
741 117.092673030027
742 117.315414738341
743 117.42882968722
744 117.667560917447
745 117.935992072429
746 117.778780843677
747 117.788523124179
748 117.850639137522
749 118.001044561418
750 118.098312888889
751 118.107588461723
752 117.961123175079
753 118.09577625752
754 118.206192965545
755 118.48683127933
756 118.365224587777
757 118.540313306541
758 118.383965928948
759 118.600679418346
760 118.495013850416
761 118.383691836421
762 118.251362280502
763 118.106742200289
764 118.201452468408
765 118.097254901961
766 117.986940056855
767 117.943500558399
768 118.220458984375
769 118.089329529678
770 118.27494687131
771 118.335582505244
772 118.406413594996
773 118.259180725956
774 118.186059865526
775 118.039532154006
776 118.311370762036
777 118.26615915419
778 118.334930379789
779 118.593911749536
780 118.441886916502
781 118.323761683124
782 118.504073429661
783 118.363503504385
784 118.224748477197
785 118.118534626151
786 118.375355295275
787 118.229410900449
788 118.258076412688
789 118.109127242294
790 118.024106713668
791 118.153976227503
792 118.282813934803
793 118.134521959962
794 118.356630966506
795 118.496094300067
796 118.563563735764
797 118.414817800126
798 118.27685127606
799 118.250892464141
800 118.121775
801 118.32895522295
802 118.494028333157
803 118.530854873304
804 118.70785840697
805 118.834344353999
806 118.79085672592
807 118.701768448013
808 118.651539983825
809 118.504909997387
810 118.358641975309
811 118.222586780717
812 118.172891540683
813 118.131089355174
814 118.157688848107
815 118.034166133464
816 117.984751958381
817 117.868624052232
818 118.128789282704
819 117.990315477495
820 118.126762343843
821 118.100854399065
822 118.082714700955
823 118.182295457599
824 118.056978449901
825 118.122895867769
826 118.054904466814
827 118.01289607547
828 117.98806710775
829 118.121692390297
830 118.291914646538
831 118.169734462271
832 118.037859132304
833 118.071561878053
834 117.929997009587
835 117.90646634874
836 117.766855154415
837 117.654911935869
838 117.71895238692
839 117.919414252452
840 118.079590419501
841 117.980245475278
842 117.961787904604
843 118.018560498924
844 117.880240504481
845 117.839249326004
846 117.717748380643
847 117.580253384053
848 117.441614842916
849 117.442511872209
850 117.638600692042
851 117.638431871815
852 117.512115927175
853 117.495300360496
854 117.377873240132
855 117.534207448446
856 117.487984758494
857 117.395337184747
858 117.356362169299
859 117.486887452042
860 117.38095592212
861 117.536664953509
862 117.401849688578
863 117.324993387211
864 117.490450049297
865 117.450748103846
866 117.332547509454
867 117.198655294942
868 117.080988925227
869 117.31368277758
870 117.178862465319
871 117.061403219705
872 116.954149482367
873 116.954840978365
874 116.832464955045
875 116.757467428571
876 116.875565563687
877 117.064944892209
878 117.049700084578
879 117.02619716013
880 116.902190082645
881 116.822063978994
882 116.777340717088
883 116.906743586225
884 116.817810599701
885 116.691198570015
886 116.589766062502
887 116.777066712084
888 116.812848744014
889 116.739395764506
890 116.739786643101
891 116.854810733599
892 116.735289016469
893 116.656312817497
894 116.583843820849
895 116.732269279985
896 116.602039570711
897 116.473693433705
898 116.450513638325
899 116.351313596494
900 116.376072839506
901 116.332135584952
902 116.246311473395
903 116.117639123434
904 116.213089317879
905 116.359554348158
906 116.471432783163
907 116.689895690629
908 116.57666920569
909 116.498099314882
910 116.394494626253
911 116.572931640481
912 116.812206640505
913 116.692314613427
914 116.714498273873
915 116.780141539013
916 116.658363684903
917 116.707230086254
918 116.58193667203
919 116.49819965639
920 116.436086956522
921 116.403470240179
922 116.301254934806
923 116.268754790599
924 116.444969172242
925 116.322192549306
926 116.344121584744
927 116.391940688607
928 116.430197263303
929 116.525300651997
930 116.522922881258
931 116.398268957648
932 116.276422249443
933 116.486597762867
934 116.385150099271
935 116.36036031914
936 116.283228093725
937 116.161092248132
938 116.051390928392
939 115.928235802482
940 115.942073336351
941 115.932556429782
942 115.871741472496
943 115.828198850941
944 115.803934528512
945 115.930964978584
946 115.814244018433
947 115.6977572705
948 115.868035526714
949 115.994345997839
950 116.163905817175
951 116.04906009613
952 115.96968346162
953 116.072681508331
954 116.01189848327
955 116.21507743757
956 116.234094072233
957 116.132967334135
958 116.128717186553
959 116.195213340278
960 116.321141493056
961 116.222195272224
962 116.260905035853
963 116.462180642225
964 116.341742566416
965 116.22714596365
966 116.166591652414
967 116.210317948345
968 116.202290442934
969 116.11460752896
970 116.051508130513
971 115.96205935041
972 115.97633744856
973 116.100282129311
974 116.164507165776
975 116.048122024984
976 116.119620901639
977 116.183682213951
978 116.097361795911
979 116.252121935518
980 116.406008954602
981 116.570205982994
982 116.661001903924
983 116.571259736994
984 116.464054960672
985 116.65031307171
986 116.537411180462
987 116.514428194697
988 116.441656149093
989 116.324381134849
990 116.206988062443
991 116.370048906353
992 116.367153965596
993 116.379607909952
994 116.36998651871
995 116.520770687609
996 116.404254568475
997 116.465124561246
998 116.553146372906
999 116.454851247644
1000 116.432736
};
\addplot [semithick, color8, forget plot]
table {%
1 0
2 90.25
3 100.666666666667
4 129.6875
5 104.24
6 86.9166666666667
7 107.836734693878
8 116.234375
9 116.098765432099
10 130.16
11 118.446280991736
12 130.888888888889
13 121.325443786982
14 113.086734693878
15 118.888888888889
16 125.83984375
17 119.044982698962
18 122.888888888889
19 130.271468144044
20 125.0475
21 125.678004535147
22 129.057851239669
23 131.810964083176
24 127.972222222222
25 128.2304
26 128.309171597633
27 124.183813443073
28 132.132653061225
29 131.098692033294
30 127.226666666667
31 123.442247658689
32 121.02734375
33 126.552800734619
34 130.2776816609
35 126.633469387755
36 123.166666666667
37 120.191380569759
38 122.466066481994
39 120.17094017094
40 121.0775
41 122.365258774539
42 119.828231292517
43 120.922660897783
44 121.794938016529
45 119.138765432099
46 116.597353497164
47 119.10547759167
48 121.305555555556
49 124.494793835902
50 122.8304
51 125.744713571703
52 123.859097633136
53 123.452474190103
54 121.45438957476
55 119.311074380165
56 117.352040816327
57 117.041551246537
58 118.782699167658
59 116.95777075553
60 115.978888888889
61 114.082773447998
62 113.12174817898
63 112.990677752583
64 111.296875
65 113.523313609467
66 112.560376492195
67 112.992203163288
68 111.339965397924
69 113.477420709935
70 114.225510204082
71 116.267407260464
72 114.708140432099
73 113.351473071871
74 112.310628195763
75 111.140266666667
76 112.252077562327
77 113.638050261427
78 112.592537804076
79 111.406184906265
80 112.675
81 111.799115988416
82 112.605145746579
83 112.059805487008
84 113.678429705215
85 112.560553633218
86 113.737155219037
87 112.498876998282
88 112.958161157025
89 115.012245928544
90 116.511728395062
91 115.976089844222
92 115.214555765595
93 114.887963926465
94 115.488909008601
95 115.778393351801
96 115.062391493056
97 116.546285471357
98 118.272386505623
99 119.905315784104
100 118.7611
101 117.863738849132
102 119.423010380623
103 118.320859647469
104 117.246671597633
105 118.129160997732
106 117.289159843361
107 117.188226045943
108 117.484482167353
109 116.406868108745
110 117.075123966942
111 116.18407596786
112 116.821348852041
113 117.597775863419
114 118.726146506617
115 119.067221172023
116 118.809973246136
117 120.12813207685
118 119.87596954898
119 119.465856931008
120 121.646597222222
121 121.909978826583
122 121.779696318194
123 120.794104038601
124 119.904981789802
125 119.028736
126 118.493134290753
127 117.56029512059
128 118.8427734375
129 118.50621957815
130 119.233136094675
131 119.907231513315
132 119.214129935721
133 118.321329639889
134 118.561817776788
135 117.81475994513
136 117.862835207612
137 119.379082529703
138 119.236557445915
139 119.788830805859
140 120.970612244898
141 120.235199436648
142 119.573348541956
143 120.769035160644
144 120.179783950617
145 119.413745541023
146 119.576515293676
147 120.101624323199
148 119.717448867787
149 119.862078284762
150 120.725733333333
151 119.9266698829
152 119.825441481994
153 120.107650903499
154 119.44560634171
155 119.348012486993
156 118.67587113741
157 117.940362692198
158 117.197764781285
159 116.480439856018
160 116.53609375
161 116.829520466031
162 117.027777777778
163 116.716474086341
164 117.943002676978
165 117.237245179063
166 116.992778342285
167 116.374986553838
168 116.309204931973
169 115.816673085676
170 115.751730103806
171 115.524434868849
172 116.672390481341
173 117.403655317585
174 117.922182586868
175 117.646171428571
176 117.280442923554
177 116.618276995755
178 116.491257416993
179 116.167160825193
180 115.584691358025
181 115.535301120234
182 114.929356357928
183 115.410433276598
184 114.929790879017
185 114.668224981738
186 114.235663082437
187 113.67811490177
188 113.091557265731
189 114.168304358781
190 113.772188365651
191 113.350456401963
192 113.358696831597
193 113.002228247738
194 113.079418641726
195 112.887153188692
196 113.557892544773
197 114.341157978819
198 114.998367513519
199 114.566500845938
200 114.4311
201 114.324694933294
202 114.81190569552
203 115.739037588876
204 116.865700692042
205 116.737798929209
206 117.186445470827
207 117.505986137366
208 116.942215236686
209 116.383919782056
210 115.972698412698
211 115.620314009119
212 116.343182627269
213 115.885736956953
214 115.369639269805
215 114.982022714981
216 114.475222908093
217 115.327316358385
218 114.798607019611
219 115.011988907654
220 114.491301652893
221 114.364734546795
222 114.575054784514
223 114.085664300509
224 115.069176498724
225 114.657777777778
226 115.535750646096
227 115.968716644996
228 116.389658356417
229 115.890848763372
230 116.431398865784
231 116.059406682783
232 116.230900713436
233 116.505922009984
234 116.032526115859
235 115.912286102309
236 116.841712151681
237 117.328562018195
238 117.159098933691
239 117.514224190753
240 118.111388888889
241 118.799159794081
242 118.919421487603
243 118.938305475114
244 118.485403789304
245 118.928946272387
246 118.939007865688
247 118.715796030094
248 119.021055541103
249 118.609796616184
250 118.8016
251 118.888335105792
252 118.485197782817
253 118.035088815635
254 117.588381176762
255 117.311403306421
256 117.229843139648
257 116.792441974897
258 117.076032089418
259 117.404361890848
260 117.041479289941
261 116.850163679335
262 117.660334479343
263 117.217597478639
264 117.021335514233
265 117.324172303311
266 117.259045169314
267 116.8237455989
268 116.391442971709
269 117.026768563176
270 117.528340192044
271 117.097370678504
272 117.399640462803
273 117.000469615854
274 116.576389258884
275 116.873388429752
276 116.814889203949
277 117.556282500749
278 117.220071424874
279 117.614316362842
280 118.228367346939
281 118.405592634338
282 118.897150545747
283 119.494187716166
284 119.856216524499
285 119.71161588181
286 120.168663504328
287 119.785574670082
288 120.392312885802
289 120.151123669496
290 120.498977407848
291 120.416575146727
292 120.119300056296
293 120.667916923901
294 121.04785043269
295 120.729491525424
296 120.33253058802
297 120.87378838894
298 120.755112382325
299 121.124573550631
300 121.041155555556
301 120.862396662289
302 121.287531248629
303 120.889150301169
304 121.356291118421
305 121.675420585864
306 121.588320731343
307 121.194007363473
308 120.835132400067
309 120.550371278055
310 120.406087408949
311 120.978898067638
312 120.736676117686
313 120.476232277557
314 120.831068197493
315 120.965643738977
316 120.669604230091
317 120.433161838609
318 120.064050077133
319 119.703560303063
320 119.541396484375
321 119.378926834949
322 119.023118321052
323 119.191538306703
324 118.848679698217
325 118.554925443787
326 118.863609846061
327 119.403922228769
328 119.708794988102
329 119.427204109349
330 119.109054178145
331 119.127444984986
332 119.079356582958
333 119.163127091055
334 118.948259170282
335 118.622267765649
336 118.704852253401
337 118.715846753956
338 118.439769265782
339 118.749279940133
340 118.410865051903
341 118.139145690182
342 117.868574946137
343 117.818697991483
344 117.581361546782
345 117.890325561857
346 117.78559256908
347 118.007507744438
348 117.848890210067
349 118.078127437377
350 118.385706122449
351 118.968579800489
352 119.190139139979
353 119.574364612508
354 119.34001244853
355 119.139821463995
356 119.283068741321
357 119.226859371199
358 118.961986205175
359 118.764131252861
360 118.764166666667
361 119.318636290391
362 119.541863801471
363 119.685950413223
364 119.383317232218
365 119.056258209795
366 119.605751142166
367 119.279896650803
368 119.49689862949
369 119.505599988249
370 119.86726807889
371 119.711713806206
372 119.490540813967
373 119.562319861424
374 119.918677971918
375 119.774264888889
376 120.04470348574
377 119.850839730104
378 119.696096133927
379 119.709038505719
380 119.463566481994
381 119.159126762698
382 119.706264905019
383 119.897470157953
384 119.74619547526
385 119.447124304267
386 119.633849230852
387 119.423111591851
388 119.117121904559
389 119.111927624057
390 119.022642998028
391 118.730241167967
392 118.744474958351
393 118.610298545151
394 118.813606379963
395 118.514955936549
396 118.867054382206
397 118.567645248685
398 118.333406984672
399 118.524896200401
400 119.04219375
401 119.045366633292
402 118.999603970199
403 119.504522532618
404 119.327486275855
405 119.053827160494
406 119.396588123953
407 119.65832573695
408 119.777465397924
409 119.522779036472
410 119.522790005949
411 119.384209186543
412 119.135138797248
413 118.867132949129
414 118.921421736797
415 118.691978516476
416 118.595021264793
417 118.32993461346
418 118.460520592477
419 118.259727388201
420 118.041105442177
421 117.991299981381
422 117.910019990566
423 117.631317449938
424 117.355637014952
425 117.484755709343
426 117.228553637947
427 116.955843557525
428 117.083648135208
429 117.096560005651
430 116.826289886425
431 116.666630778258
432 117.070210691015
433 117.315554512531
434 117.380364841046
435 117.168925881887
436 116.958189546334
437 117.276699359582
438 117.456287400179
439 117.633023905023
440 117.421234504132
441 117.176176593086
442 116.946213222497
443 116.69181499014
444 116.429003327652
445 116.167372806464
446 115.989020491062
447 116.164236846188
448 116.632653061224
449 117.090034275624
450 117.464933333333
451 117.204507352471
452 117.50236412405
453 117.378769936991
454 117.1378835219
455 117.320874290545
456 117.146102646968
457 116.911117601712
458 116.657538948533
459 116.82205799289
460 117.282873345936
461 117.249834134039
462 117.03219579843
463 116.812505539514
464 116.695284631391
465 116.934119551393
466 117.382789331172
467 117.269435872511
468 117.155978340273
469 116.940257591118
470 117.379832503395
471 117.132603982131
472 117.018600976731
473 116.80582311636
474 116.969538357457
475 116.743109141274
476 116.710238507168
477 116.878191351432
478 116.66732462667
479 116.49860312673
480 116.7225
481 116.497888581049
482 116.258501058866
483 116.093523483748
484 115.955890137286
485 116.245322563503
486 116.211218648919
487 116.102863359039
488 116.524350812954
489 116.288439743895
490 116.454743856726
491 116.249293805816
492 116.046400951814
493 115.819509646203
494 115.827521349309
495 115.624715845322
496 115.522791200572
497 115.290357841212
498 115.400574184287
499 115.21995493994
500 115.437296
501 115.370106095195
502 115.474516277519
503 115.253362528606
504 115.220505794911
505 115.441490049995
506 115.33814385477
507 115.118051422102
508 115.398486421973
509 115.455699182881
510 115.387481737793
511 115.396823694762
512 115.405212402344
513 115.210294525571
514 115.07918742146
515 115.139070600434
516 114.983996604771
517 114.808331057395
518 114.739691566912
519 114.838324776044
520 114.844227071006
521 114.656820450853
522 114.674623097136
523 114.573781939027
524 114.363833692675
525 114.211424943311
526 114.091157165782
527 114.05912238189
528 113.858456726354
529 114.004874196419
530 113.936222855109
531 114.252297303528
532 114.216631805077
533 114.087711949424
534 114.131650745557
535 113.938426063412
536 114.027313293607
537 113.989846342707
538 113.781194289742
539 113.622430048086
540 113.927146776406
541 113.844800311602
542 113.716891790689
543 113.510739395419
544 113.497931985294
545 113.632805319418
546 113.532021092461
547 113.939032582576
548 113.901859449092
549 113.820723886119
550 113.672317355372
551 113.787958537686
552 114.18785116047
553 113.984290848209
554 114.224295898552
555 114.069000892785
556 114.400548625848
557 114.639209151359
558 114.623077812464
559 114.834341927989
560 114.632028061224
561 114.430425678617
562 114.817558668203
563 114.631828349145
564 114.43713847392
565 114.282277390555
566 114.640053565408
567 114.933742678599
568 115.254162740528
569 115.435027690179
570 115.237919359803
571 115.526850917523
572 115.325012836814
573 115.363272814768
574 115.51767351795
575 115.335337618147
576 115.440028814622
577 115.344106401064
578 115.470854635361
579 115.29530099242
580 115.19896254459
581 115.003581574886
582 114.806830930197
583 114.686835996458
584 114.840436761118
585 114.7115201987
586 114.992384885089
587 115.32916774289
588 115.224304687862
589 115.222013080788
590 115.342548118357
591 115.461608275286
592 115.415072589481
593 115.618056641708
594 115.928884807673
595 115.736939481675
596 115.935419575695
597 116.265992160692
598 116.536327893424
599 116.342496258372
600 116.149308333333
601 116.101112676875
602 116.297957528063
603 116.117125813717
604 115.925669378536
605 115.831650843522
606 115.865168447538
607 116.168908044261
608 116.268868485976
609 116.626416559489
610 116.464942219833
611 116.348686519108
612 116.36559389551
613 116.250041914048
614 116.54498456217
615 116.356443915659
616 116.277354950245
617 116.22637901279
618 116.302887485468
619 116.177465869439
620 116.213798126951
621 116.034669550177
622 115.850541764457
623 115.664668190215
624 115.562859549638
625 115.53091584
626 115.351356041197
627 115.615032215888
628 115.581755345044
629 115.420019664292
630 115.331229528849
631 115.199976893769
632 115.06897682663
633 114.88990713496
634 114.708843754043
635 114.946419492839
636 115.204370376963
637 115.493150043498
638 115.835211918122
639 115.86951442615
640 115.963278808594
641 115.843244150983
642 115.768444114479
643 115.803550135567
644 116.058851896146
645 115.895501472267
646 116.03075127721
647 116.029879911803
648 116.232424554184
649 116.055878309881
650 116.190790532544
651 116.040183954262
652 115.922153167225
653 115.786725889932
654 115.669107538647
655 115.622362333197
656 115.516173408685
657 115.362000143635
658 115.213846878724
659 115.05068377387
660 115.102614784206
661 114.929422023661
662 114.755880285868
663 114.609506129868
664 114.44174589926
665 114.644115552038
666 114.498823147472
667 114.395869531501
668 114.696338610205
669 114.894573030269
670 114.739175762976
671 114.570618846351
672 114.540548380811
673 114.81405253362
674 114.651322103743
675 114.482269410151
676 114.633617520395
677 114.808969102981
678 114.794800341104
679 114.990566999464
680 114.85428200692
681 115.146064459927
682 114.993225462457
683 114.925590957129
684 114.912099021921
685 114.963929884384
686 114.80077391223
687 114.6359697353
688 114.724392830584
689 114.597631029594
690 114.431623608486
691 114.270553173844
692 114.485599251562
693 114.33522610146
694 114.181730601533
695 114.493750841054
696 114.331250412868
697 114.174023124314
698 114.035404060722
699 113.992402799012
700 114.143826530612
701 114.451736972452
702 114.300232952655
703 114.225026254075
704 114.077009216813
705 113.919975856345
706 113.868998226452
707 114.124745673193
708 114.081745347761
709 114.065691760779
710 113.997928982345
711 113.861382613185
712 113.980684257038
713 113.891224681285
714 114.008978101044
715 114.305071152624
716 114.313114447115
717 114.602417091204
718 114.478629510944
719 114.409822017522
720 114.560617283951
721 114.676218305213
722 114.523246445316
723 114.565581783295
724 114.639058255243
725 114.645890130797
726 114.493879440536
727 114.357618976442
728 114.33975365294
729 114.62972935848
730 114.702625258022
731 114.938043756936
732 114.781032577861
733 114.625603725369
734 114.603384463468
735 114.536413531399
736 114.430955739012
737 114.544011164113
738 114.40141450195
739 114.634910578425
740 114.56792366691
741 114.478206312001
742 114.328935418952
743 114.566833741208
744 114.425330963117
745 114.661224269177
746 114.556082843979
747 114.414642057745
748 114.326751908834
749 114.186338348773
750 114.034090666667
751 113.946961086948
752 114.224123967293
753 114.119271475409
754 114.393960768035
755 114.377081706943
756 114.334593096498
757 114.518611846456
758 114.372365828698
759 114.408411317159
760 114.633059210526
761 114.488350448352
762 114.472387555886
763 114.388505743178
764 114.536818823497
765 114.767933700713
766 114.61956418
767 114.517881517417
768 114.3742251926
769 114.246749447461
770 114.227034913139
771 114.110577836993
772 114.296417353486
773 114.303784418832
774 114.157535938679
775 114.07245119667
776 114.03110718461
777 114.038494092548
778 113.994442278335
779 114.179378123759
780 114.116048652202
781 113.989950177142
782 113.907635677422
783 113.94704847094
784 113.863696116202
785 113.738666883038
786 113.775770642736
787 113.812318666255
788 113.884348669123
789 113.865276834035
790 113.732036532607
791 113.869761108296
792 113.731479887256
793 113.60004388971
794 113.743042909986
795 113.601386021123
796 113.558942640337
797 113.448691690451
798 113.59058203152
799 113.574577734057
800 113.6133984375
801 113.715293461201
802 113.85235166448
803 114.111099565918
804 113.988286181035
805 114.245012152309
806 114.500113909943
807 114.461316639258
808 114.321194000588
809 114.392194731398
810 114.348690748362
811 114.269926002059
812 114.130196073673
813 114.300352966629
814 114.333089846603
815 114.595846287026
816 114.600488694252
817 114.506099725988
818 114.510685612831
819 114.550362945967
820 114.490922070196
821 114.529798632427
822 114.697935721432
823 114.638428297031
824 114.499316617966
825 114.392042607897
826 114.273837860338
827 114.448356481448
828 114.339436159537
829 114.377247573995
830 114.335562490928
831 114.199359651066
832 114.334815030973
833 114.255406844611
834 114.427393337129
835 114.386317185987
836 114.34511000206
837 114.208501946275
838 114.342702821242
839 114.472615534982
840 114.380068027211
841 114.44301486962
842 114.42400601441
843 114.288280149553
844 114.498276094427
845 114.53117747978
846 114.695096937892
847 114.619077820323
848 114.512326450694
849 114.497324504267
850 114.595659515571
851 114.797576915801
852 114.682408417642
853 114.666962613162
854 114.651274618958
855 114.818393351801
856 115.060702626867
857 114.944539375777
858 114.814933900598
859 114.711120085759
860 114.807349918875
861 114.768900381886
862 114.710543386394
863 114.606840510279
864 114.516069744513
865 114.716944769287
866 114.603220722282
867 114.475523787098
868 114.508551625645
869 114.747432666676
870 114.643519619501
871 114.628162727825
872 114.792057697164
873 114.689618161755
874 114.887390099964
875 114.811188244898
876 114.736968578637
877 114.610937827075
878 114.535350325081
879 114.567104256699
880 114.724895402893
881 114.688352030056
882 114.558327034518
883 114.752466688641
884 114.676475706886
885 114.706558141019
886 114.870042649899
887 114.927891159921
888 114.957327682412
889 114.839021106614
890 114.875015780836
891 114.751089898864
892 114.633276106497
893 114.578414418979
894 114.477105635882
895 114.359943821978
896 114.251045071349
897 114.150468115569
898 114.064746702645
899 114.301618038087
900 114.265988888889
901 114.143325765797
902 114.021505548154
903 114.142036695695
904 114.234821638343
905 114.359068404505
906 114.450410313388
907 114.35140624981
908 114.505778299598
909 114.695337300507
910 114.569304431832
911 114.461737924453
912 114.353155538243
913 114.47649324771
914 114.567199268371
915 114.575085550479
916 114.58124845064
917 114.702751492765
918 114.931342883317
919 114.807380402363
920 114.752155009452
921 114.680639110819
922 114.834026990274
923 114.71391864815
924 114.801431523022
925 114.807451862673
926 114.693403663776
927 114.700809585153
928 114.82003265272
929 114.697505680495
930 114.728958261071
931 114.787296613484
932 114.71673359244
933 114.834609053072
934 115.020088129158
935 114.964827132603
936 114.949685879173
937 114.836384883749
938 114.740981355786
939 114.96224315855
940 115.18230421005
941 115.401170663176
942 115.317984502414
943 115.205763515056
944 115.293493922364
945 115.439921614736
946 115.421774781097
947 115.304509655902
948 115.251841540708
949 115.155863695466
950 115.118192797784
951 115.145998290581
952 115.149525986865
953 115.327379490844
954 115.23412839682
955 115.385685699405
956 115.473179741251
957 115.417657277564
958 115.300909602033
959 115.219575048305
960 115.124495442708
961 115.020433752995
962 115.053242551683
963 115.170319473694
964 115.318964032988
965 115.226056001503
966 115.208883616459
967 115.089748676329
968 115.26652892562
969 115.19941083815
970 115.231022425337
971 115.122053453339
972 115.030012150926
973 114.996116100806
974 114.981713461709
975 115.065841946088
976 114.998512454649
977 114.917352956275
978 115.131817155331
979 115.116444309039
980 115.32929092045
981 115.313914425044
982 115.425160008462
983 115.344597734218
984 115.292088042832
985 115.179196578113
986 115.321029915778
987 115.304625788749
988 115.47846526742
989 115.371196406171
990 115.356837057443
991 115.496805253335
992 115.406248983806
993 115.299470918788
994 115.326745179325
995 115.30979116689
996 115.361384292189
997 115.343973746717
998 115.311388307677
999 115.523433343253
1000 115.4731
};
\addplot [semithick, color0, dashed]
table {%
1 114
2 114
3 114
4 114
5 114
6 114
7 114
8 114
9 114
10 114
11 114
12 114
13 114
14 114
15 114
16 114
17 114
18 114
19 114
20 114
21 114
22 114
23 114
24 114
25 114
26 114
27 114
28 114
29 114
30 114
31 114
32 114
33 114
34 114
35 114
36 114
37 114
38 114
39 114
40 114
41 114
42 114
43 114
44 114
45 114
46 114
47 114
48 114
49 114
50 114
51 114
52 114
53 114
54 114
55 114
56 114
57 114
58 114
59 114
60 114
61 114
62 114
63 114
64 114
65 114
66 114
67 114
68 114
69 114
70 114
71 114
72 114
73 114
74 114
75 114
76 114
77 114
78 114
79 114
80 114
81 114
82 114
83 114
84 114
85 114
86 114
87 114
88 114
89 114
90 114
91 114
92 114
93 114
94 114
95 114
96 114
97 114
98 114
99 114
100 114
101 114
102 114
103 114
104 114
105 114
106 114
107 114
108 114
109 114
110 114
111 114
112 114
113 114
114 114
115 114
116 114
117 114
118 114
119 114
120 114
121 114
122 114
123 114
124 114
125 114
126 114
127 114
128 114
129 114
130 114
131 114
132 114
133 114
134 114
135 114
136 114
137 114
138 114
139 114
140 114
141 114
142 114
143 114
144 114
145 114
146 114
147 114
148 114
149 114
150 114
151 114
152 114
153 114
154 114
155 114
156 114
157 114
158 114
159 114
160 114
161 114
162 114
163 114
164 114
165 114
166 114
167 114
168 114
169 114
170 114
171 114
172 114
173 114
174 114
175 114
176 114
177 114
178 114
179 114
180 114
181 114
182 114
183 114
184 114
185 114
186 114
187 114
188 114
189 114
190 114
191 114
192 114
193 114
194 114
195 114
196 114
197 114
198 114
199 114
200 114
201 114
202 114
203 114
204 114
205 114
206 114
207 114
208 114
209 114
210 114
211 114
212 114
213 114
214 114
215 114
216 114
217 114
218 114
219 114
220 114
221 114
222 114
223 114
224 114
225 114
226 114
227 114
228 114
229 114
230 114
231 114
232 114
233 114
234 114
235 114
236 114
237 114
238 114
239 114
240 114
241 114
242 114
243 114
244 114
245 114
246 114
247 114
248 114
249 114
250 114
251 114
252 114
253 114
254 114
255 114
256 114
257 114
258 114
259 114
260 114
261 114
262 114
263 114
264 114
265 114
266 114
267 114
268 114
269 114
270 114
271 114
272 114
273 114
274 114
275 114
276 114
277 114
278 114
279 114
280 114
281 114
282 114
283 114
284 114
285 114
286 114
287 114
288 114
289 114
290 114
291 114
292 114
293 114
294 114
295 114
296 114
297 114
298 114
299 114
300 114
301 114
302 114
303 114
304 114
305 114
306 114
307 114
308 114
309 114
310 114
311 114
312 114
313 114
314 114
315 114
316 114
317 114
318 114
319 114
320 114
321 114
322 114
323 114
324 114
325 114
326 114
327 114
328 114
329 114
330 114
331 114
332 114
333 114
334 114
335 114
336 114
337 114
338 114
339 114
340 114
341 114
342 114
343 114
344 114
345 114
346 114
347 114
348 114
349 114
350 114
351 114
352 114
353 114
354 114
355 114
356 114
357 114
358 114
359 114
360 114
361 114
362 114
363 114
364 114
365 114
366 114
367 114
368 114
369 114
370 114
371 114
372 114
373 114
374 114
375 114
376 114
377 114
378 114
379 114
380 114
381 114
382 114
383 114
384 114
385 114
386 114
387 114
388 114
389 114
390 114
391 114
392 114
393 114
394 114
395 114
396 114
397 114
398 114
399 114
400 114
401 114
402 114
403 114
404 114
405 114
406 114
407 114
408 114
409 114
410 114
411 114
412 114
413 114
414 114
415 114
416 114
417 114
418 114
419 114
420 114
421 114
422 114
423 114
424 114
425 114
426 114
427 114
428 114
429 114
430 114
431 114
432 114
433 114
434 114
435 114
436 114
437 114
438 114
439 114
440 114
441 114
442 114
443 114
444 114
445 114
446 114
447 114
448 114
449 114
450 114
451 114
452 114
453 114
454 114
455 114
456 114
457 114
458 114
459 114
460 114
461 114
462 114
463 114
464 114
465 114
466 114
467 114
468 114
469 114
470 114
471 114
472 114
473 114
474 114
475 114
476 114
477 114
478 114
479 114
480 114
481 114
482 114
483 114
484 114
485 114
486 114
487 114
488 114
489 114
490 114
491 114
492 114
493 114
494 114
495 114
496 114
497 114
498 114
499 114
500 114
501 114
502 114
503 114
504 114
505 114
506 114
507 114
508 114
509 114
510 114
511 114
512 114
513 114
514 114
515 114
516 114
517 114
518 114
519 114
520 114
521 114
522 114
523 114
524 114
525 114
526 114
527 114
528 114
529 114
530 114
531 114
532 114
533 114
534 114
535 114
536 114
537 114
538 114
539 114
540 114
541 114
542 114
543 114
544 114
545 114
546 114
547 114
548 114
549 114
550 114
551 114
552 114
553 114
554 114
555 114
556 114
557 114
558 114
559 114
560 114
561 114
562 114
563 114
564 114
565 114
566 114
567 114
568 114
569 114
570 114
571 114
572 114
573 114
574 114
575 114
576 114
577 114
578 114
579 114
580 114
581 114
582 114
583 114
584 114
585 114
586 114
587 114
588 114
589 114
590 114
591 114
592 114
593 114
594 114
595 114
596 114
597 114
598 114
599 114
600 114
601 114
602 114
603 114
604 114
605 114
606 114
607 114
608 114
609 114
610 114
611 114
612 114
613 114
614 114
615 114
616 114
617 114
618 114
619 114
620 114
621 114
622 114
623 114
624 114
625 114
626 114
627 114
628 114
629 114
630 114
631 114
632 114
633 114
634 114
635 114
636 114
637 114
638 114
639 114
640 114
641 114
642 114
643 114
644 114
645 114
646 114
647 114
648 114
649 114
650 114
651 114
652 114
653 114
654 114
655 114
656 114
657 114
658 114
659 114
660 114
661 114
662 114
663 114
664 114
665 114
666 114
667 114
668 114
669 114
670 114
671 114
672 114
673 114
674 114
675 114
676 114
677 114
678 114
679 114
680 114
681 114
682 114
683 114
684 114
685 114
686 114
687 114
688 114
689 114
690 114
691 114
692 114
693 114
694 114
695 114
696 114
697 114
698 114
699 114
700 114
701 114
702 114
703 114
704 114
705 114
706 114
707 114
708 114
709 114
710 114
711 114
712 114
713 114
714 114
715 114
716 114
717 114
718 114
719 114
720 114
721 114
722 114
723 114
724 114
725 114
726 114
727 114
728 114
729 114
730 114
731 114
732 114
733 114
734 114
735 114
736 114
737 114
738 114
739 114
740 114
741 114
742 114
743 114
744 114
745 114
746 114
747 114
748 114
749 114
750 114
751 114
752 114
753 114
754 114
755 114
756 114
757 114
758 114
759 114
760 114
761 114
762 114
763 114
764 114
765 114
766 114
767 114
768 114
769 114
770 114
771 114
772 114
773 114
774 114
775 114
776 114
777 114
778 114
779 114
780 114
781 114
782 114
783 114
784 114
785 114
786 114
787 114
788 114
789 114
790 114
791 114
792 114
793 114
794 114
795 114
796 114
797 114
798 114
799 114
800 114
801 114
802 114
803 114
804 114
805 114
806 114
807 114
808 114
809 114
810 114
811 114
812 114
813 114
814 114
815 114
816 114
817 114
818 114
819 114
820 114
821 114
822 114
823 114
824 114
825 114
826 114
827 114
828 114
829 114
830 114
831 114
832 114
833 114
834 114
835 114
836 114
837 114
838 114
839 114
840 114
841 114
842 114
843 114
844 114
845 114
846 114
847 114
848 114
849 114
850 114
851 114
852 114
853 114
854 114
855 114
856 114
857 114
858 114
859 114
860 114
861 114
862 114
863 114
864 114
865 114
866 114
867 114
868 114
869 114
870 114
871 114
872 114
873 114
874 114
875 114
876 114
877 114
878 114
879 114
880 114
881 114
882 114
883 114
884 114
885 114
886 114
887 114
888 114
889 114
890 114
891 114
892 114
893 114
894 114
895 114
896 114
897 114
898 114
899 114
900 114
901 114
902 114
903 114
904 114
905 114
906 114
907 114
908 114
909 114
910 114
911 114
912 114
913 114
914 114
915 114
916 114
917 114
918 114
919 114
920 114
921 114
922 114
923 114
924 114
925 114
926 114
927 114
928 114
929 114
930 114
931 114
932 114
933 114
934 114
935 114
936 114
937 114
938 114
939 114
940 114
941 114
942 114
943 114
944 114
945 114
946 114
947 114
948 114
949 114
950 114
951 114
952 114
953 114
954 114
955 114
956 114
957 114
958 114
959 114
960 114
961 114
962 114
963 114
964 114
965 114
966 114
967 114
968 114
969 114
970 114
971 114
972 114
973 114
974 114
975 114
976 114
977 114
978 114
979 114
980 114
981 114
982 114
983 114
984 114
985 114
986 114
987 114
988 114
989 114
990 114
991 114
992 114
993 114
994 114
995 114
996 114
997 114
998 114
999 114
1000 114
};
\addlegendentry{$v_{v_{e}}$ (valor de la varianza esperada)}
\end{axis}

\end{tikzpicture}

    \caption{valor de la varianza para 10 corridas del experimento}
  \end{mytikzresize}
\end{figure}

%\bibliographystyle{unsrt}
%\bibliography{references}

\end{document}
