\documentclass{article}

\usepackage{arxiv}

\usepackage[utf8]{inputenc} % allow utf-8 input
\usepackage[spanish]{babel} % idioma español
\usepackage[T1]{fontenc}    % use 8-bit T1 fonts
\usepackage{hyperref}       % hyperlinks
\usepackage{url}            % simple URL typesetting
\usepackage{booktabs}       % professional-quality tables
\usepackage{amsfonts}       % blackboard math symbols
\usepackage{nicefrac}       % compact symbols for 1/2, etc.
\usepackage{microtype}      % microtypography
\usepackage{graphicx}
\graphicspath{ {./images/} }
\usepackage{kpfonts}        % use the same fonts for text and maths

\usepackage{pgfplots}       % TikZ graphics
\pgfplotsset{compat=1.15}

\usepackage{mytikz}

\title{TP 1.1 - Simulación de una Ruleta}

\author{
 Marcelo G. Catellano \\
  UTN -- FRRo \\
  \texttt{marce.geek22@gmail.com} \\
}

\begin{document}
\maketitle
\begin{abstract}
Simulación de un modelo simple de una ruleta empleando el lenguaje de programación Python 3.x.
\end{abstract}

% keywords can be removed
%\keywords{First keyword \and Second keyword \and More}

\section[Introducción]{Introducción\footnote{Wikipedia - \url{https://es.wikipedia.org/wiki/Ruleta}}}
La ruleta es un juego de azar típico de los casinos, cuyo nombre viene del término francés roulette, que significa ``ruedita'' o ``rueda pequeña''. Su uso como elemento de juego de azar, aún en configuraciones distintas de la actual, no está documentado hasta bien entrada la Edad Media. Es de suponer que su referencia más antigua es la llamada Rueda de la Fortuna, de la que hay noticias a lo largo de toda la historia, prácticamente en todos los campos del saber humano.

La ``magia'' del movimiento de las ruedas tuvo que impactar a todas las generaciones. La aparente quietud del centro, el aumento de velocidad conforme nos alejamos de él, la posibilidad de que se detenga en un punto al azar; todo esto tuvo que influir en el desarrollo de distintos juegos que tienen la rueda como base.

Las ruedas, y por extensión las ruletas, siempre han tenido conexión con el mundo mágico y esotérico. Así, una de ellas forma parte del tarot, más precisamente de los que se conocen como arcanos mayores.

Según los indicios, la creación de una ruleta y sus normas de juego, muy similares a las que conocemos hoy en día, se debe a Blaise Pascal, matemático francés, quien ideó una ruleta con treinta y seis números (sin el cero), en la que se halla un extremado equilibrio en la posición en que está colocado cada número. La elección de 36 números da un alcance aún más vinculado a la magia (la suma de los primeros 36 números da el número mágico por excelencia: seiscientos sesenta y seis).

Esta ruleta podía usarse como entretenimiento en círculos de amistades. Sin embargo, a nivel de empresa que pone los medios y el personal para el entretenimiento de sus clientes, no era rentable, ya que estadísticamente todo lo que se apostaba se repartía en premios (probabilidad de 1/36 de acertar el número y ganar 36 veces lo apostado).

En 1842, los hermanos Blanc modificaron la ruleta añadiéndole un nuevo número, el 0, y la introdujeron inicialmente en el Casino de Montecarlo. Ésta es la ruleta que se conoce hoy en día, con una probabilidad de acertar de 1/37 y ganar 36 veces lo apostado, consiguiendo un margen para la casa del $2.7\%$ (1/37).

Más adelante, en algunas ruletas (sobre todo las que se usan en países anglosajones) se añadió un nuevo número (el doble cero), con lo cual el beneficio para el casino resultó ser doble (2/38 o $5.26\%$).

\section{Gráficas}
\begin{figure}[!htbp]
  \begin{mytikzresize}{0.6\textwidth}
    \centering
    % This file was created by tikzplotlib v0.9.1.
\begin{tikzpicture}

\definecolor{color0}{rgb}{0.12156862745098,0.466666666666667,0.705882352941177}
\definecolor{color1}{rgb}{1,0.498039215686275,0.0549019607843137}

\begin{axis}[
legend cell align={left},
legend style={fill opacity=0.5, draw opacity=1, text opacity=1, draw=white!80!black},
scaled ticks=false,
tick align=outside,
tick pos=left,
width=\figW,
x grid style={white!69.0196078431373!black},
xlabel={\(\displaystyle n\) (número de tiradas)},
xmajorgrids,
xmin=-48.95, xmax=1049.95,
xtick style={color=black},
xticklabel style={/pgf/number format/.cd,fixed,precision=2},
y grid style={white!69.0196078431373!black},
ylabel={\(\displaystyle f_{r}\) (frecuencia relativa)},
ymajorgrids,
ymin=-0.00327868852459016, ymax=0.0688524590163934,
ytick style={color=black},
yticklabel style={/pgf/number format/.cd,fixed,precision=2}
]
\addplot [semithick, color0]
table {%
1 0
2 0
3 0
4 0
5 0
6 0
7 0
8 0
9 0
10 0
11 0
12 0
13 0
14 0
15 0
16 0
17 0
18 0
19 0
20 0
21 0
22 0
23 0
24 0
25 0
26 0
27 0.037037037037037
28 0.0357142857142857
29 0.0344827586206897
30 0.0333333333333333
31 0.032258064516129
32 0.03125
33 0.0303030303030303
34 0.0294117647058824
35 0.0285714285714286
36 0.0277777777777778
37 0.027027027027027
38 0.0263157894736842
39 0.0256410256410256
40 0.025
41 0.0487804878048781
42 0.0476190476190476
43 0.0465116279069767
44 0.0454545454545455
45 0.0444444444444444
46 0.0434782608695652
47 0.0425531914893617
48 0.0416666666666667
49 0.0408163265306122
50 0.04
51 0.0392156862745098
52 0.0384615384615385
53 0.0377358490566038
54 0.037037037037037
55 0.0545454545454545
56 0.0535714285714286
57 0.0526315789473684
58 0.0517241379310345
59 0.0508474576271186
60 0.05
61 0.0655737704918033
62 0.0645161290322581
63 0.0634920634920635
64 0.0625
65 0.0615384615384615
66 0.0606060606060606
67 0.0597014925373134
68 0.0588235294117647
69 0.0579710144927536
70 0.0571428571428571
71 0.0563380281690141
72 0.0555555555555556
73 0.0547945205479452
74 0.0540540540540541
75 0.0533333333333333
76 0.0526315789473684
77 0.051948051948052
78 0.0512820512820513
79 0.0506329113924051
80 0.05
81 0.0493827160493827
82 0.0487804878048781
83 0.0481927710843374
84 0.0476190476190476
85 0.0470588235294118
86 0.0465116279069767
87 0.0459770114942529
88 0.0454545454545455
89 0.0449438202247191
90 0.0444444444444444
91 0.0549450549450549
92 0.0543478260869565
93 0.0537634408602151
94 0.0531914893617021
95 0.0526315789473684
96 0.0520833333333333
97 0.0515463917525773
98 0.0510204081632653
99 0.0505050505050505
100 0.05
101 0.0495049504950495
102 0.0490196078431373
103 0.0485436893203883
104 0.0480769230769231
105 0.0476190476190476
106 0.0471698113207547
107 0.0467289719626168
108 0.0462962962962963
109 0.0458715596330275
110 0.0454545454545455
111 0.045045045045045
112 0.0446428571428571
113 0.0442477876106195
114 0.043859649122807
115 0.0434782608695652
116 0.0431034482758621
117 0.0427350427350427
118 0.0423728813559322
119 0.0420168067226891
120 0.0416666666666667
121 0.0413223140495868
122 0.040983606557377
123 0.040650406504065
124 0.0403225806451613
125 0.04
126 0.0396825396825397
127 0.0393700787401575
128 0.0390625
129 0.0387596899224806
130 0.0384615384615385
131 0.0381679389312977
132 0.0378787878787879
133 0.037593984962406
134 0.0373134328358209
135 0.037037037037037
136 0.0367647058823529
137 0.0364963503649635
138 0.036231884057971
139 0.0359712230215827
140 0.0357142857142857
141 0.0354609929078014
142 0.0352112676056338
143 0.034965034965035
144 0.0347222222222222
145 0.0344827586206897
146 0.0342465753424658
147 0.0340136054421769
148 0.0337837837837838
149 0.0335570469798658
150 0.0333333333333333
151 0.033112582781457
152 0.0328947368421053
153 0.0326797385620915
154 0.0324675324675325
155 0.032258064516129
156 0.032051282051282
157 0.0318471337579618
158 0.0316455696202532
159 0.0314465408805031
160 0.03125
161 0.031055900621118
162 0.0308641975308642
163 0.0306748466257669
164 0.0304878048780488
165 0.0303030303030303
166 0.0301204819277108
167 0.029940119760479
168 0.0297619047619048
169 0.029585798816568
170 0.0294117647058824
171 0.0292397660818713
172 0.0290697674418605
173 0.0289017341040462
174 0.028735632183908
175 0.0285714285714286
176 0.0284090909090909
177 0.0282485875706215
178 0.0280898876404494
179 0.0279329608938547
180 0.0277777777777778
181 0.0276243093922652
182 0.0274725274725275
183 0.0273224043715847
184 0.0271739130434783
185 0.027027027027027
186 0.0268817204301075
187 0.0267379679144385
188 0.0265957446808511
189 0.0264550264550265
190 0.0263157894736842
191 0.0261780104712042
192 0.0260416666666667
193 0.0259067357512953
194 0.0257731958762887
195 0.0307692307692308
196 0.0306122448979592
197 0.0304568527918782
198 0.0303030303030303
199 0.0301507537688442
200 0.03
201 0.0298507462686567
202 0.0297029702970297
203 0.0295566502463054
204 0.0294117647058824
205 0.0292682926829268
206 0.029126213592233
207 0.0289855072463768
208 0.0288461538461538
209 0.0287081339712919
210 0.0285714285714286
211 0.028436018957346
212 0.0283018867924528
213 0.028169014084507
214 0.0280373831775701
215 0.027906976744186
216 0.0277777777777778
217 0.0276497695852535
218 0.0275229357798165
219 0.0273972602739726
220 0.0272727272727273
221 0.0271493212669683
222 0.027027027027027
223 0.0269058295964126
224 0.0267857142857143
225 0.0266666666666667
226 0.0265486725663717
227 0.026431718061674
228 0.0307017543859649
229 0.0305676855895196
230 0.0304347826086957
231 0.0303030303030303
232 0.0301724137931034
233 0.0300429184549356
234 0.0341880341880342
235 0.0340425531914894
236 0.0338983050847458
237 0.0337552742616034
238 0.0336134453781513
239 0.0334728033472803
240 0.0333333333333333
241 0.033195020746888
242 0.0330578512396694
243 0.0329218106995885
244 0.0327868852459016
245 0.0326530612244898
246 0.032520325203252
247 0.0323886639676113
248 0.032258064516129
249 0.0321285140562249
250 0.032
251 0.0318725099601594
252 0.0317460317460317
253 0.0316205533596838
254 0.031496062992126
255 0.0313725490196078
256 0.03125
257 0.0311284046692607
258 0.0310077519379845
259 0.0308880308880309
260 0.0307692307692308
261 0.0306513409961686
262 0.0305343511450382
263 0.0304182509505703
264 0.0303030303030303
265 0.030188679245283
266 0.0300751879699248
267 0.0299625468164794
268 0.0298507462686567
269 0.033457249070632
270 0.0333333333333333
271 0.033210332103321
272 0.0330882352941176
273 0.032967032967033
274 0.0328467153284672
275 0.0327272727272727
276 0.0326086956521739
277 0.0324909747292419
278 0.0323741007194245
279 0.032258064516129
280 0.0321428571428571
281 0.0320284697508897
282 0.0319148936170213
283 0.0318021201413428
284 0.0316901408450704
285 0.0315789473684211
286 0.0314685314685315
287 0.0313588850174216
288 0.03125
289 0.0311418685121107
290 0.0310344827586207
291 0.0309278350515464
292 0.0308219178082192
293 0.0307167235494881
294 0.0306122448979592
295 0.0305084745762712
296 0.0304054054054054
297 0.0303030303030303
298 0.0302013422818792
299 0.0301003344481605
300 0.03
301 0.0299003322259136
302 0.0298013245033113
303 0.0297029702970297
304 0.0296052631578947
305 0.0327868852459016
306 0.0359477124183007
307 0.0358306188925081
308 0.0357142857142857
309 0.0355987055016181
310 0.0354838709677419
311 0.0353697749196141
312 0.0352564102564103
313 0.0351437699680511
314 0.035031847133758
315 0.0349206349206349
316 0.0348101265822785
317 0.0347003154574132
318 0.0345911949685535
319 0.0344827586206897
320 0.034375
321 0.0342679127725857
322 0.0341614906832298
323 0.0340557275541796
324 0.0339506172839506
325 0.0338461538461538
326 0.0337423312883436
327 0.0336391437308868
328 0.0335365853658537
329 0.033434650455927
330 0.0333333333333333
331 0.0332326283987915
332 0.0331325301204819
333 0.033033033033033
334 0.0329341317365269
335 0.0328358208955224
336 0.0357142857142857
337 0.0356083086053412
338 0.0355029585798817
339 0.0353982300884956
340 0.0352941176470588
341 0.0351906158357771
342 0.0350877192982456
343 0.0349854227405248
344 0.0348837209302326
345 0.0347826086956522
346 0.0346820809248555
347 0.0345821325648415
348 0.0344827586206897
349 0.0343839541547278
350 0.0342857142857143
351 0.0341880341880342
352 0.0340909090909091
353 0.0339943342776204
354 0.0338983050847458
355 0.0338028169014084
356 0.0337078651685393
357 0.0336134453781513
358 0.0335195530726257
359 0.0334261838440111
360 0.0333333333333333
361 0.0332409972299169
362 0.0331491712707182
363 0.0330578512396694
364 0.032967032967033
365 0.0328767123287671
366 0.0327868852459016
367 0.0326975476839237
368 0.0326086956521739
369 0.032520325203252
370 0.0324324324324324
371 0.032345013477089
372 0.032258064516129
373 0.032171581769437
374 0.0320855614973262
375 0.032
376 0.0319148936170213
377 0.0318302387267905
378 0.0317460317460317
379 0.0316622691292876
380 0.0315789473684211
381 0.031496062992126
382 0.031413612565445
383 0.031331592689295
384 0.03125
385 0.0311688311688312
386 0.0310880829015544
387 0.0310077519379845
388 0.0309278350515464
389 0.0308483290488432
390 0.0307692307692308
391 0.030690537084399
392 0.0306122448979592
393 0.0305343511450382
394 0.0304568527918782
395 0.030379746835443
396 0.0303030303030303
397 0.0302267002518892
398 0.0301507537688442
399 0.0300751879699248
400 0.03
401 0.029925187032419
402 0.0298507462686567
403 0.0297766749379653
404 0.0297029702970297
405 0.0296296296296296
406 0.0320197044334975
407 0.0319410319410319
408 0.0318627450980392
409 0.0317848410757946
410 0.0317073170731707
411 0.0316301703163017
412 0.0315533980582524
413 0.0314769975786925
414 0.0314009661835749
415 0.0313253012048193
416 0.03125
417 0.0311750599520384
418 0.0311004784688995
419 0.0310262529832936
420 0.030952380952381
421 0.0308788598574822
422 0.0308056872037915
423 0.0307328605200946
424 0.0306603773584906
425 0.0305882352941176
426 0.0305164319248826
427 0.0304449648711944
428 0.0303738317757009
429 0.0303030303030303
430 0.0302325581395349
431 0.0301624129930394
432 0.0300925925925926
433 0.0300230946882217
434 0.0299539170506912
435 0.0298850574712644
436 0.0298165137614679
437 0.0297482837528604
438 0.0296803652968037
439 0.0296127562642369
440 0.0295454545454545
441 0.0294784580498866
442 0.0294117647058824
443 0.0293453724604966
444 0.0292792792792793
445 0.0292134831460674
446 0.0291479820627803
447 0.029082774049217
448 0.0290178571428571
449 0.0289532293986637
450 0.0288888888888889
451 0.0288248337028825
452 0.0287610619469027
453 0.0286975717439294
454 0.0286343612334802
455 0.0285714285714286
456 0.0285087719298246
457 0.0284463894967177
458 0.0283842794759825
459 0.028322440087146
460 0.0282608695652174
461 0.0281995661605206
462 0.0281385281385281
463 0.0280777537796976
464 0.0280172413793103
465 0.0279569892473118
466 0.0278969957081545
467 0.0278372591006424
468 0.0277777777777778
469 0.0277185501066098
470 0.0276595744680851
471 0.0276008492569002
472 0.0296610169491525
473 0.0295983086680761
474 0.029535864978903
475 0.0294736842105263
476 0.0294117647058824
477 0.0293501048218029
478 0.0292887029288703
479 0.0292275574112735
480 0.0291666666666667
481 0.0311850311850312
482 0.0311203319502075
483 0.031055900621118
484 0.0309917355371901
485 0.0309278350515464
486 0.0308641975308642
487 0.0308008213552361
488 0.0307377049180328
489 0.0306748466257669
490 0.0306122448979592
491 0.0305498981670061
492 0.0304878048780488
493 0.0304259634888438
494 0.0303643724696356
495 0.0323232323232323
496 0.032258064516129
497 0.0321931589537223
498 0.0321285140562249
499 0.032064128256513
500 0.032
501 0.031936127744511
502 0.0318725099601594
503 0.0318091451292246
504 0.0317460317460317
505 0.0316831683168317
506 0.0316205533596838
507 0.0315581854043393
508 0.031496062992126
509 0.031434184675835
510 0.0313725490196078
511 0.0313111545988258
512 0.03125
513 0.0311890838206628
514 0.0311284046692607
515 0.0310679611650485
516 0.0310077519379845
517 0.0309477756286267
518 0.0308880308880309
519 0.0308285163776493
520 0.0307692307692308
521 0.0307101727447217
522 0.0306513409961686
523 0.0305927342256214
524 0.0305343511450382
525 0.0304761904761905
526 0.0304182509505703
527 0.0303605313092979
528 0.0303030303030303
529 0.0302457466918715
530 0.030188679245283
531 0.0301318267419962
532 0.0300751879699248
533 0.0300187617260788
534 0.0299625468164794
535 0.0299065420560748
536 0.0298507462686567
537 0.0297951582867784
538 0.0297397769516729
539 0.0296846011131725
540 0.0296296296296296
541 0.0295748613678373
542 0.029520295202952
543 0.0294659300184162
544 0.0294117647058824
545 0.0293577981651376
546 0.0293040293040293
547 0.0292504570383912
548 0.0291970802919708
549 0.029143897996357
550 0.0290909090909091
551 0.029038112522686
552 0.0289855072463768
553 0.0289330922242315
554 0.0288808664259928
555 0.0288288288288288
556 0.0287769784172662
557 0.0287253141831239
558 0.028673835125448
559 0.0286225402504472
560 0.0285714285714286
561 0.0285204991087344
562 0.0284697508896797
563 0.0284191829484902
564 0.0283687943262411
565 0.0283185840707965
566 0.0282685512367491
567 0.0282186948853616
568 0.028169014084507
569 0.0281195079086116
570 0.0280701754385965
571 0.0280210157618214
572 0.0297202797202797
573 0.0296684118673647
574 0.029616724738676
575 0.0295652173913043
576 0.0295138888888889
577 0.0294627383015598
578 0.0294117647058824
579 0.0310880829015544
580 0.0310344827586207
581 0.0309810671256454
582 0.0309278350515464
583 0.0308747855917667
584 0.0308219178082192
585 0.0307692307692308
586 0.0307167235494881
587 0.030664395229983
588 0.0306122448979592
589 0.0305602716468591
590 0.0305084745762712
591 0.0304568527918782
592 0.0304054054054054
593 0.03035413153457
594 0.0303030303030303
595 0.0302521008403361
596 0.0302013422818792
597 0.0301507537688442
598 0.0301003344481605
599 0.0317195325542571
600 0.0316666666666667
601 0.0316139767054908
602 0.0315614617940199
603 0.0315091210613599
604 0.0314569536423841
605 0.031404958677686
606 0.0313531353135314
607 0.0313014827018122
608 0.03125
609 0.0328407224958949
610 0.0327868852459016
611 0.0327332242225859
612 0.0326797385620915
613 0.032626427406199
614 0.0325732899022801
615 0.032520325203252
616 0.0324675324675325
617 0.0324149108589951
618 0.0323624595469256
619 0.0323101777059774
620 0.032258064516129
621 0.0322061191626409
622 0.0321543408360129
623 0.0321027287319422
624 0.032051282051282
625 0.032
626 0.0319488817891374
627 0.0318979266347687
628 0.0318471337579618
629 0.0317965023847377
630 0.0333333333333333
631 0.0332805071315372
632 0.0332278481012658
633 0.033175355450237
634 0.0331230283911672
635 0.0330708661417323
636 0.0330188679245283
637 0.032967032967033
638 0.0329153605015674
639 0.0328638497652582
640 0.0328125
641 0.0327613104524181
642 0.0327102803738318
643 0.0326594090202177
644 0.0326086956521739
645 0.0325581395348837
646 0.0325077399380805
647 0.0324574961360124
648 0.0324074074074074
649 0.0323574730354391
650 0.0323076923076923
651 0.032258064516129
652 0.0322085889570552
653 0.0321592649310873
654 0.0321100917431193
655 0.0320610687022901
656 0.0320121951219512
657 0.0319634703196347
658 0.0319148936170213
659 0.031866464339909
660 0.0318181818181818
661 0.0317700453857791
662 0.0317220543806647
663 0.0316742081447964
664 0.0316265060240964
665 0.0315789473684211
666 0.0315315315315315
667 0.0314842578710645
668 0.031437125748503
669 0.031390134529148
670 0.0313432835820895
671 0.0327868852459016
672 0.0327380952380952
673 0.0341753343239227
674 0.0341246290801187
675 0.0340740740740741
676 0.0340236686390533
677 0.03397341211226
678 0.0339233038348083
679 0.0338733431516937
680 0.0338235294117647
681 0.0337738619676946
682 0.0337243401759531
683 0.0336749633967789
684 0.033625730994152
685 0.0335766423357664
686 0.0335276967930029
687 0.0334788937409025
688 0.0334302325581395
689 0.0333817126269956
690 0.0333333333333333
691 0.0332850940665702
692 0.0332369942196532
693 0.0331890331890332
694 0.0331412103746398
695 0.0330935251798561
696 0.0330459770114943
697 0.0329985652797704
698 0.0329512893982808
699 0.0329041487839771
700 0.0328571428571429
701 0.0328102710413695
702 0.0327635327635328
703 0.0327169274537696
704 0.0326704545454545
705 0.0340425531914894
706 0.0339943342776204
707 0.0339462517680339
708 0.0338983050847458
709 0.0338504936530324
710 0.0338028169014084
711 0.0337552742616034
712 0.0337078651685393
713 0.0336605890603086
714 0.0336134453781513
715 0.0335664335664336
716 0.0335195530726257
717 0.0334728033472803
718 0.0334261838440111
719 0.0333796940194715
720 0.0333333333333333
721 0.0332871012482663
722 0.0332409972299169
723 0.033195020746888
724 0.0331491712707182
725 0.0331034482758621
726 0.0330578512396694
727 0.0330123796423659
728 0.032967032967033
729 0.0329218106995885
730 0.0328767123287671
731 0.0328317373461012
732 0.0327868852459016
733 0.0327421555252387
734 0.0326975476839237
735 0.0326530612244898
736 0.0326086956521739
737 0.0325644504748982
738 0.032520325203252
739 0.0324763193504736
740 0.0324324324324324
741 0.0323886639676113
742 0.032345013477089
743 0.0323014804845222
744 0.032258064516129
745 0.0322147651006711
746 0.032171581769437
747 0.0321285140562249
748 0.0320855614973262
749 0.0320427236315087
750 0.032
751 0.0319573901464714
752 0.0319148936170213
753 0.0318725099601594
754 0.0318302387267905
755 0.0317880794701987
756 0.0317460317460317
757 0.0317040951122853
758 0.0316622691292876
759 0.0316205533596838
760 0.0315789473684211
761 0.0315374507227332
762 0.031496062992126
763 0.0314547837483617
764 0.031413612565445
765 0.0313725490196078
766 0.031331592689295
767 0.0312907431551499
768 0.03125
769 0.0312093628088427
770 0.0311688311688312
771 0.0311284046692607
772 0.0310880829015544
773 0.0310478654592497
774 0.0310077519379845
775 0.0309677419354839
776 0.0309278350515464
777 0.0308880308880309
778 0.0308483290488432
779 0.030808729139923
780 0.0307692307692308
781 0.030729833546735
782 0.030690537084399
783 0.0306513409961686
784 0.0306122448979592
785 0.0305732484076433
786 0.0305343511450382
787 0.0304955527318933
788 0.0304568527918782
789 0.0304182509505703
790 0.030379746835443
791 0.0303413400758533
792 0.0303030303030303
793 0.0302648171500631
794 0.0302267002518892
795 0.030188679245283
796 0.0301507537688442
797 0.0301129234629862
798 0.0300751879699248
799 0.0300375469336671
800 0.03
801 0.0299625468164794
802 0.0311720698254364
803 0.0311332503113325
804 0.0310945273631841
805 0.031055900621118
806 0.0310173697270471
807 0.0309789343246592
808 0.0309405940594059
809 0.030902348578492
810 0.0308641975308642
811 0.030826140567201
812 0.0307881773399015
813 0.030750307503075
814 0.0307125307125307
815 0.0306748466257669
816 0.0306372549019608
817 0.0305997552019584
818 0.0305623471882641
819 0.0305250305250305
820 0.0304878048780488
821 0.0304506699147381
822 0.0304136253041363
823 0.0303766707168894
824 0.0303398058252427
825 0.0303030303030303
826 0.0302663438256659
827 0.030229746070133
828 0.0301932367149758
829 0.0301568154402895
830 0.0301204819277108
831 0.0300842358604091
832 0.0300480769230769
833 0.0300120048019208
834 0.0299760191846523
835 0.029940119760479
836 0.0299043062200957
837 0.031063321385902
838 0.0310262529832936
839 0.0309892729439809
840 0.030952380952381
841 0.0309155766944114
842 0.0308788598574822
843 0.0308422301304864
844 0.0308056872037915
845 0.0307692307692308
846 0.0307328605200946
847 0.0306965761511216
848 0.0306603773584906
849 0.0306242638398115
850 0.0305882352941176
851 0.0305522914218566
852 0.0305164319248826
853 0.0304806565064478
854 0.0304449648711944
855 0.0304093567251462
856 0.0303738317757009
857 0.0303383897316219
858 0.0303030303030303
859 0.030267753201397
860 0.0302325581395349
861 0.0301974448315912
862 0.0301624129930394
863 0.0301274623406721
864 0.0300925925925926
865 0.0300578034682081
866 0.0300230946882217
867 0.0299884659746251
868 0.0299539170506912
869 0.0299194476409666
870 0.0298850574712644
871 0.0298507462686567
872 0.0298165137614679
873 0.0297823596792669
874 0.0297482837528604
875 0.0297142857142857
876 0.0296803652968037
877 0.0296465222348917
878 0.030751708428246
879 0.0307167235494881
880 0.0306818181818182
881 0.0306469920544835
882 0.0306122448979592
883 0.0305775764439411
884 0.0305429864253394
885 0.0305084745762712
886 0.0304740406320542
887 0.0304396843291995
888 0.0304054054054054
889 0.0303712035995501
890 0.0303370786516854
891 0.0303030303030303
892 0.0302690582959641
893 0.0302351623740202
894 0.0302013422818792
895 0.0301675977653631
896 0.0301339285714286
897 0.0301003344481605
898 0.0300668151447661
899 0.0300333704115684
900 0.03
901 0.0299667036625971
902 0.0299334811529933
903 0.0299003322259136
904 0.0298672566371681
905 0.0298342541436464
906 0.0298013245033113
907 0.0297684674751929
908 0.0297356828193833
909 0.0297029702970297
910 0.0296703296703297
911 0.0296377607025247
912 0.0296052631578947
913 0.0295728368017525
914 0.0306345733041575
915 0.0306010928961749
916 0.0305676855895196
917 0.0305343511450382
918 0.0305010893246187
919 0.0304678998911861
920 0.0304347826086957
921 0.0304017372421281
922 0.0303687635574837
923 0.0303358613217768
924 0.0303030303030303
925 0.0302702702702703
926 0.0302375809935205
927 0.0302049622437972
928 0.0301724137931034
929 0.0301399354144241
930 0.0301075268817204
931 0.0300751879699248
932 0.0300429184549356
933 0.030010718113612
934 0.0299785867237687
935 0.0299465240641711
936 0.0299145299145299
937 0.0298826040554963
938 0.0309168443496802
939 0.0308839190628328
940 0.0308510638297872
941 0.0308182784272051
942 0.0307855626326964
943 0.0307529162248144
944 0.0307203389830508
945 0.0306878306878307
946 0.0306553911205074
947 0.030623020063358
948 0.0305907172995781
949 0.0305584826132771
950 0.0305263157894737
951 0.0304942166140904
952 0.0304621848739496
953 0.0304302203567681
954 0.030398322851153
955 0.0303664921465969
956 0.0303347280334728
957 0.0303030303030303
958 0.0302713987473904
959 0.0302398331595412
960 0.0302083333333333
961 0.0301768990634755
962 0.0301455301455301
963 0.0301142263759086
964 0.0300829875518672
965 0.0300518134715026
966 0.0300207039337474
967 0.0299896587383661
968 0.0299586776859504
969 0.0299277605779154
970 0.0298969072164948
971 0.0298661174047374
972 0.0298353909465021
973 0.0298047276464543
974 0.0297741273100616
975 0.0297435897435897
976 0.0297131147540984
977 0.0296827021494371
978 0.0296523517382413
979 0.0296220633299285
980 0.0295918367346939
981 0.0295616717635066
982 0.0295315682281059
983 0.0295015259409969
984 0.0294715447154472
985 0.0294416243654822
986 0.0294117647058824
987 0.0293819655521783
988 0.0293522267206478
989 0.0293225480283114
990 0.0292929292929293
991 0.029263370332997
992 0.0292338709677419
993 0.0292044310171198
994 0.0291750503018109
995 0.0291457286432161
996 0.0291164658634538
997 0.0290872617853561
998 0.0290581162324649
999 0.029029029029029
1000 0.029
};
\addlegendentry{$f_{r}$ (frecuencia relativa de $18$)}
\addplot [semithick, color1]
table {%
1 0.027027027027027
2 0.027027027027027
3 0.027027027027027
4 0.027027027027027
5 0.027027027027027
6 0.027027027027027
7 0.027027027027027
8 0.027027027027027
9 0.027027027027027
10 0.027027027027027
11 0.027027027027027
12 0.027027027027027
13 0.027027027027027
14 0.027027027027027
15 0.027027027027027
16 0.027027027027027
17 0.027027027027027
18 0.027027027027027
19 0.027027027027027
20 0.027027027027027
21 0.027027027027027
22 0.027027027027027
23 0.027027027027027
24 0.027027027027027
25 0.027027027027027
26 0.027027027027027
27 0.027027027027027
28 0.027027027027027
29 0.027027027027027
30 0.027027027027027
31 0.027027027027027
32 0.027027027027027
33 0.027027027027027
34 0.027027027027027
35 0.027027027027027
36 0.027027027027027
37 0.027027027027027
38 0.027027027027027
39 0.027027027027027
40 0.027027027027027
41 0.027027027027027
42 0.027027027027027
43 0.027027027027027
44 0.027027027027027
45 0.027027027027027
46 0.027027027027027
47 0.027027027027027
48 0.027027027027027
49 0.027027027027027
50 0.027027027027027
51 0.027027027027027
52 0.027027027027027
53 0.027027027027027
54 0.027027027027027
55 0.027027027027027
56 0.027027027027027
57 0.027027027027027
58 0.027027027027027
59 0.027027027027027
60 0.027027027027027
61 0.027027027027027
62 0.027027027027027
63 0.027027027027027
64 0.027027027027027
65 0.027027027027027
66 0.027027027027027
67 0.027027027027027
68 0.027027027027027
69 0.027027027027027
70 0.027027027027027
71 0.027027027027027
72 0.027027027027027
73 0.027027027027027
74 0.027027027027027
75 0.027027027027027
76 0.027027027027027
77 0.027027027027027
78 0.027027027027027
79 0.027027027027027
80 0.027027027027027
81 0.027027027027027
82 0.027027027027027
83 0.027027027027027
84 0.027027027027027
85 0.027027027027027
86 0.027027027027027
87 0.027027027027027
88 0.027027027027027
89 0.027027027027027
90 0.027027027027027
91 0.027027027027027
92 0.027027027027027
93 0.027027027027027
94 0.027027027027027
95 0.027027027027027
96 0.027027027027027
97 0.027027027027027
98 0.027027027027027
99 0.027027027027027
100 0.027027027027027
101 0.027027027027027
102 0.027027027027027
103 0.027027027027027
104 0.027027027027027
105 0.027027027027027
106 0.027027027027027
107 0.027027027027027
108 0.027027027027027
109 0.027027027027027
110 0.027027027027027
111 0.027027027027027
112 0.027027027027027
113 0.027027027027027
114 0.027027027027027
115 0.027027027027027
116 0.027027027027027
117 0.027027027027027
118 0.027027027027027
119 0.027027027027027
120 0.027027027027027
121 0.027027027027027
122 0.027027027027027
123 0.027027027027027
124 0.027027027027027
125 0.027027027027027
126 0.027027027027027
127 0.027027027027027
128 0.027027027027027
129 0.027027027027027
130 0.027027027027027
131 0.027027027027027
132 0.027027027027027
133 0.027027027027027
134 0.027027027027027
135 0.027027027027027
136 0.027027027027027
137 0.027027027027027
138 0.027027027027027
139 0.027027027027027
140 0.027027027027027
141 0.027027027027027
142 0.027027027027027
143 0.027027027027027
144 0.027027027027027
145 0.027027027027027
146 0.027027027027027
147 0.027027027027027
148 0.027027027027027
149 0.027027027027027
150 0.027027027027027
151 0.027027027027027
152 0.027027027027027
153 0.027027027027027
154 0.027027027027027
155 0.027027027027027
156 0.027027027027027
157 0.027027027027027
158 0.027027027027027
159 0.027027027027027
160 0.027027027027027
161 0.027027027027027
162 0.027027027027027
163 0.027027027027027
164 0.027027027027027
165 0.027027027027027
166 0.027027027027027
167 0.027027027027027
168 0.027027027027027
169 0.027027027027027
170 0.027027027027027
171 0.027027027027027
172 0.027027027027027
173 0.027027027027027
174 0.027027027027027
175 0.027027027027027
176 0.027027027027027
177 0.027027027027027
178 0.027027027027027
179 0.027027027027027
180 0.027027027027027
181 0.027027027027027
182 0.027027027027027
183 0.027027027027027
184 0.027027027027027
185 0.027027027027027
186 0.027027027027027
187 0.027027027027027
188 0.027027027027027
189 0.027027027027027
190 0.027027027027027
191 0.027027027027027
192 0.027027027027027
193 0.027027027027027
194 0.027027027027027
195 0.027027027027027
196 0.027027027027027
197 0.027027027027027
198 0.027027027027027
199 0.027027027027027
200 0.027027027027027
201 0.027027027027027
202 0.027027027027027
203 0.027027027027027
204 0.027027027027027
205 0.027027027027027
206 0.027027027027027
207 0.027027027027027
208 0.027027027027027
209 0.027027027027027
210 0.027027027027027
211 0.027027027027027
212 0.027027027027027
213 0.027027027027027
214 0.027027027027027
215 0.027027027027027
216 0.027027027027027
217 0.027027027027027
218 0.027027027027027
219 0.027027027027027
220 0.027027027027027
221 0.027027027027027
222 0.027027027027027
223 0.027027027027027
224 0.027027027027027
225 0.027027027027027
226 0.027027027027027
227 0.027027027027027
228 0.027027027027027
229 0.027027027027027
230 0.027027027027027
231 0.027027027027027
232 0.027027027027027
233 0.027027027027027
234 0.027027027027027
235 0.027027027027027
236 0.027027027027027
237 0.027027027027027
238 0.027027027027027
239 0.027027027027027
240 0.027027027027027
241 0.027027027027027
242 0.027027027027027
243 0.027027027027027
244 0.027027027027027
245 0.027027027027027
246 0.027027027027027
247 0.027027027027027
248 0.027027027027027
249 0.027027027027027
250 0.027027027027027
251 0.027027027027027
252 0.027027027027027
253 0.027027027027027
254 0.027027027027027
255 0.027027027027027
256 0.027027027027027
257 0.027027027027027
258 0.027027027027027
259 0.027027027027027
260 0.027027027027027
261 0.027027027027027
262 0.027027027027027
263 0.027027027027027
264 0.027027027027027
265 0.027027027027027
266 0.027027027027027
267 0.027027027027027
268 0.027027027027027
269 0.027027027027027
270 0.027027027027027
271 0.027027027027027
272 0.027027027027027
273 0.027027027027027
274 0.027027027027027
275 0.027027027027027
276 0.027027027027027
277 0.027027027027027
278 0.027027027027027
279 0.027027027027027
280 0.027027027027027
281 0.027027027027027
282 0.027027027027027
283 0.027027027027027
284 0.027027027027027
285 0.027027027027027
286 0.027027027027027
287 0.027027027027027
288 0.027027027027027
289 0.027027027027027
290 0.027027027027027
291 0.027027027027027
292 0.027027027027027
293 0.027027027027027
294 0.027027027027027
295 0.027027027027027
296 0.027027027027027
297 0.027027027027027
298 0.027027027027027
299 0.027027027027027
300 0.027027027027027
301 0.027027027027027
302 0.027027027027027
303 0.027027027027027
304 0.027027027027027
305 0.027027027027027
306 0.027027027027027
307 0.027027027027027
308 0.027027027027027
309 0.027027027027027
310 0.027027027027027
311 0.027027027027027
312 0.027027027027027
313 0.027027027027027
314 0.027027027027027
315 0.027027027027027
316 0.027027027027027
317 0.027027027027027
318 0.027027027027027
319 0.027027027027027
320 0.027027027027027
321 0.027027027027027
322 0.027027027027027
323 0.027027027027027
324 0.027027027027027
325 0.027027027027027
326 0.027027027027027
327 0.027027027027027
328 0.027027027027027
329 0.027027027027027
330 0.027027027027027
331 0.027027027027027
332 0.027027027027027
333 0.027027027027027
334 0.027027027027027
335 0.027027027027027
336 0.027027027027027
337 0.027027027027027
338 0.027027027027027
339 0.027027027027027
340 0.027027027027027
341 0.027027027027027
342 0.027027027027027
343 0.027027027027027
344 0.027027027027027
345 0.027027027027027
346 0.027027027027027
347 0.027027027027027
348 0.027027027027027
349 0.027027027027027
350 0.027027027027027
351 0.027027027027027
352 0.027027027027027
353 0.027027027027027
354 0.027027027027027
355 0.027027027027027
356 0.027027027027027
357 0.027027027027027
358 0.027027027027027
359 0.027027027027027
360 0.027027027027027
361 0.027027027027027
362 0.027027027027027
363 0.027027027027027
364 0.027027027027027
365 0.027027027027027
366 0.027027027027027
367 0.027027027027027
368 0.027027027027027
369 0.027027027027027
370 0.027027027027027
371 0.027027027027027
372 0.027027027027027
373 0.027027027027027
374 0.027027027027027
375 0.027027027027027
376 0.027027027027027
377 0.027027027027027
378 0.027027027027027
379 0.027027027027027
380 0.027027027027027
381 0.027027027027027
382 0.027027027027027
383 0.027027027027027
384 0.027027027027027
385 0.027027027027027
386 0.027027027027027
387 0.027027027027027
388 0.027027027027027
389 0.027027027027027
390 0.027027027027027
391 0.027027027027027
392 0.027027027027027
393 0.027027027027027
394 0.027027027027027
395 0.027027027027027
396 0.027027027027027
397 0.027027027027027
398 0.027027027027027
399 0.027027027027027
400 0.027027027027027
401 0.027027027027027
402 0.027027027027027
403 0.027027027027027
404 0.027027027027027
405 0.027027027027027
406 0.027027027027027
407 0.027027027027027
408 0.027027027027027
409 0.027027027027027
410 0.027027027027027
411 0.027027027027027
412 0.027027027027027
413 0.027027027027027
414 0.027027027027027
415 0.027027027027027
416 0.027027027027027
417 0.027027027027027
418 0.027027027027027
419 0.027027027027027
420 0.027027027027027
421 0.027027027027027
422 0.027027027027027
423 0.027027027027027
424 0.027027027027027
425 0.027027027027027
426 0.027027027027027
427 0.027027027027027
428 0.027027027027027
429 0.027027027027027
430 0.027027027027027
431 0.027027027027027
432 0.027027027027027
433 0.027027027027027
434 0.027027027027027
435 0.027027027027027
436 0.027027027027027
437 0.027027027027027
438 0.027027027027027
439 0.027027027027027
440 0.027027027027027
441 0.027027027027027
442 0.027027027027027
443 0.027027027027027
444 0.027027027027027
445 0.027027027027027
446 0.027027027027027
447 0.027027027027027
448 0.027027027027027
449 0.027027027027027
450 0.027027027027027
451 0.027027027027027
452 0.027027027027027
453 0.027027027027027
454 0.027027027027027
455 0.027027027027027
456 0.027027027027027
457 0.027027027027027
458 0.027027027027027
459 0.027027027027027
460 0.027027027027027
461 0.027027027027027
462 0.027027027027027
463 0.027027027027027
464 0.027027027027027
465 0.027027027027027
466 0.027027027027027
467 0.027027027027027
468 0.027027027027027
469 0.027027027027027
470 0.027027027027027
471 0.027027027027027
472 0.027027027027027
473 0.027027027027027
474 0.027027027027027
475 0.027027027027027
476 0.027027027027027
477 0.027027027027027
478 0.027027027027027
479 0.027027027027027
480 0.027027027027027
481 0.027027027027027
482 0.027027027027027
483 0.027027027027027
484 0.027027027027027
485 0.027027027027027
486 0.027027027027027
487 0.027027027027027
488 0.027027027027027
489 0.027027027027027
490 0.027027027027027
491 0.027027027027027
492 0.027027027027027
493 0.027027027027027
494 0.027027027027027
495 0.027027027027027
496 0.027027027027027
497 0.027027027027027
498 0.027027027027027
499 0.027027027027027
500 0.027027027027027
501 0.027027027027027
502 0.027027027027027
503 0.027027027027027
504 0.027027027027027
505 0.027027027027027
506 0.027027027027027
507 0.027027027027027
508 0.027027027027027
509 0.027027027027027
510 0.027027027027027
511 0.027027027027027
512 0.027027027027027
513 0.027027027027027
514 0.027027027027027
515 0.027027027027027
516 0.027027027027027
517 0.027027027027027
518 0.027027027027027
519 0.027027027027027
520 0.027027027027027
521 0.027027027027027
522 0.027027027027027
523 0.027027027027027
524 0.027027027027027
525 0.027027027027027
526 0.027027027027027
527 0.027027027027027
528 0.027027027027027
529 0.027027027027027
530 0.027027027027027
531 0.027027027027027
532 0.027027027027027
533 0.027027027027027
534 0.027027027027027
535 0.027027027027027
536 0.027027027027027
537 0.027027027027027
538 0.027027027027027
539 0.027027027027027
540 0.027027027027027
541 0.027027027027027
542 0.027027027027027
543 0.027027027027027
544 0.027027027027027
545 0.027027027027027
546 0.027027027027027
547 0.027027027027027
548 0.027027027027027
549 0.027027027027027
550 0.027027027027027
551 0.027027027027027
552 0.027027027027027
553 0.027027027027027
554 0.027027027027027
555 0.027027027027027
556 0.027027027027027
557 0.027027027027027
558 0.027027027027027
559 0.027027027027027
560 0.027027027027027
561 0.027027027027027
562 0.027027027027027
563 0.027027027027027
564 0.027027027027027
565 0.027027027027027
566 0.027027027027027
567 0.027027027027027
568 0.027027027027027
569 0.027027027027027
570 0.027027027027027
571 0.027027027027027
572 0.027027027027027
573 0.027027027027027
574 0.027027027027027
575 0.027027027027027
576 0.027027027027027
577 0.027027027027027
578 0.027027027027027
579 0.027027027027027
580 0.027027027027027
581 0.027027027027027
582 0.027027027027027
583 0.027027027027027
584 0.027027027027027
585 0.027027027027027
586 0.027027027027027
587 0.027027027027027
588 0.027027027027027
589 0.027027027027027
590 0.027027027027027
591 0.027027027027027
592 0.027027027027027
593 0.027027027027027
594 0.027027027027027
595 0.027027027027027
596 0.027027027027027
597 0.027027027027027
598 0.027027027027027
599 0.027027027027027
600 0.027027027027027
601 0.027027027027027
602 0.027027027027027
603 0.027027027027027
604 0.027027027027027
605 0.027027027027027
606 0.027027027027027
607 0.027027027027027
608 0.027027027027027
609 0.027027027027027
610 0.027027027027027
611 0.027027027027027
612 0.027027027027027
613 0.027027027027027
614 0.027027027027027
615 0.027027027027027
616 0.027027027027027
617 0.027027027027027
618 0.027027027027027
619 0.027027027027027
620 0.027027027027027
621 0.027027027027027
622 0.027027027027027
623 0.027027027027027
624 0.027027027027027
625 0.027027027027027
626 0.027027027027027
627 0.027027027027027
628 0.027027027027027
629 0.027027027027027
630 0.027027027027027
631 0.027027027027027
632 0.027027027027027
633 0.027027027027027
634 0.027027027027027
635 0.027027027027027
636 0.027027027027027
637 0.027027027027027
638 0.027027027027027
639 0.027027027027027
640 0.027027027027027
641 0.027027027027027
642 0.027027027027027
643 0.027027027027027
644 0.027027027027027
645 0.027027027027027
646 0.027027027027027
647 0.027027027027027
648 0.027027027027027
649 0.027027027027027
650 0.027027027027027
651 0.027027027027027
652 0.027027027027027
653 0.027027027027027
654 0.027027027027027
655 0.027027027027027
656 0.027027027027027
657 0.027027027027027
658 0.027027027027027
659 0.027027027027027
660 0.027027027027027
661 0.027027027027027
662 0.027027027027027
663 0.027027027027027
664 0.027027027027027
665 0.027027027027027
666 0.027027027027027
667 0.027027027027027
668 0.027027027027027
669 0.027027027027027
670 0.027027027027027
671 0.027027027027027
672 0.027027027027027
673 0.027027027027027
674 0.027027027027027
675 0.027027027027027
676 0.027027027027027
677 0.027027027027027
678 0.027027027027027
679 0.027027027027027
680 0.027027027027027
681 0.027027027027027
682 0.027027027027027
683 0.027027027027027
684 0.027027027027027
685 0.027027027027027
686 0.027027027027027
687 0.027027027027027
688 0.027027027027027
689 0.027027027027027
690 0.027027027027027
691 0.027027027027027
692 0.027027027027027
693 0.027027027027027
694 0.027027027027027
695 0.027027027027027
696 0.027027027027027
697 0.027027027027027
698 0.027027027027027
699 0.027027027027027
700 0.027027027027027
701 0.027027027027027
702 0.027027027027027
703 0.027027027027027
704 0.027027027027027
705 0.027027027027027
706 0.027027027027027
707 0.027027027027027
708 0.027027027027027
709 0.027027027027027
710 0.027027027027027
711 0.027027027027027
712 0.027027027027027
713 0.027027027027027
714 0.027027027027027
715 0.027027027027027
716 0.027027027027027
717 0.027027027027027
718 0.027027027027027
719 0.027027027027027
720 0.027027027027027
721 0.027027027027027
722 0.027027027027027
723 0.027027027027027
724 0.027027027027027
725 0.027027027027027
726 0.027027027027027
727 0.027027027027027
728 0.027027027027027
729 0.027027027027027
730 0.027027027027027
731 0.027027027027027
732 0.027027027027027
733 0.027027027027027
734 0.027027027027027
735 0.027027027027027
736 0.027027027027027
737 0.027027027027027
738 0.027027027027027
739 0.027027027027027
740 0.027027027027027
741 0.027027027027027
742 0.027027027027027
743 0.027027027027027
744 0.027027027027027
745 0.027027027027027
746 0.027027027027027
747 0.027027027027027
748 0.027027027027027
749 0.027027027027027
750 0.027027027027027
751 0.027027027027027
752 0.027027027027027
753 0.027027027027027
754 0.027027027027027
755 0.027027027027027
756 0.027027027027027
757 0.027027027027027
758 0.027027027027027
759 0.027027027027027
760 0.027027027027027
761 0.027027027027027
762 0.027027027027027
763 0.027027027027027
764 0.027027027027027
765 0.027027027027027
766 0.027027027027027
767 0.027027027027027
768 0.027027027027027
769 0.027027027027027
770 0.027027027027027
771 0.027027027027027
772 0.027027027027027
773 0.027027027027027
774 0.027027027027027
775 0.027027027027027
776 0.027027027027027
777 0.027027027027027
778 0.027027027027027
779 0.027027027027027
780 0.027027027027027
781 0.027027027027027
782 0.027027027027027
783 0.027027027027027
784 0.027027027027027
785 0.027027027027027
786 0.027027027027027
787 0.027027027027027
788 0.027027027027027
789 0.027027027027027
790 0.027027027027027
791 0.027027027027027
792 0.027027027027027
793 0.027027027027027
794 0.027027027027027
795 0.027027027027027
796 0.027027027027027
797 0.027027027027027
798 0.027027027027027
799 0.027027027027027
800 0.027027027027027
801 0.027027027027027
802 0.027027027027027
803 0.027027027027027
804 0.027027027027027
805 0.027027027027027
806 0.027027027027027
807 0.027027027027027
808 0.027027027027027
809 0.027027027027027
810 0.027027027027027
811 0.027027027027027
812 0.027027027027027
813 0.027027027027027
814 0.027027027027027
815 0.027027027027027
816 0.027027027027027
817 0.027027027027027
818 0.027027027027027
819 0.027027027027027
820 0.027027027027027
821 0.027027027027027
822 0.027027027027027
823 0.027027027027027
824 0.027027027027027
825 0.027027027027027
826 0.027027027027027
827 0.027027027027027
828 0.027027027027027
829 0.027027027027027
830 0.027027027027027
831 0.027027027027027
832 0.027027027027027
833 0.027027027027027
834 0.027027027027027
835 0.027027027027027
836 0.027027027027027
837 0.027027027027027
838 0.027027027027027
839 0.027027027027027
840 0.027027027027027
841 0.027027027027027
842 0.027027027027027
843 0.027027027027027
844 0.027027027027027
845 0.027027027027027
846 0.027027027027027
847 0.027027027027027
848 0.027027027027027
849 0.027027027027027
850 0.027027027027027
851 0.027027027027027
852 0.027027027027027
853 0.027027027027027
854 0.027027027027027
855 0.027027027027027
856 0.027027027027027
857 0.027027027027027
858 0.027027027027027
859 0.027027027027027
860 0.027027027027027
861 0.027027027027027
862 0.027027027027027
863 0.027027027027027
864 0.027027027027027
865 0.027027027027027
866 0.027027027027027
867 0.027027027027027
868 0.027027027027027
869 0.027027027027027
870 0.027027027027027
871 0.027027027027027
872 0.027027027027027
873 0.027027027027027
874 0.027027027027027
875 0.027027027027027
876 0.027027027027027
877 0.027027027027027
878 0.027027027027027
879 0.027027027027027
880 0.027027027027027
881 0.027027027027027
882 0.027027027027027
883 0.027027027027027
884 0.027027027027027
885 0.027027027027027
886 0.027027027027027
887 0.027027027027027
888 0.027027027027027
889 0.027027027027027
890 0.027027027027027
891 0.027027027027027
892 0.027027027027027
893 0.027027027027027
894 0.027027027027027
895 0.027027027027027
896 0.027027027027027
897 0.027027027027027
898 0.027027027027027
899 0.027027027027027
900 0.027027027027027
901 0.027027027027027
902 0.027027027027027
903 0.027027027027027
904 0.027027027027027
905 0.027027027027027
906 0.027027027027027
907 0.027027027027027
908 0.027027027027027
909 0.027027027027027
910 0.027027027027027
911 0.027027027027027
912 0.027027027027027
913 0.027027027027027
914 0.027027027027027
915 0.027027027027027
916 0.027027027027027
917 0.027027027027027
918 0.027027027027027
919 0.027027027027027
920 0.027027027027027
921 0.027027027027027
922 0.027027027027027
923 0.027027027027027
924 0.027027027027027
925 0.027027027027027
926 0.027027027027027
927 0.027027027027027
928 0.027027027027027
929 0.027027027027027
930 0.027027027027027
931 0.027027027027027
932 0.027027027027027
933 0.027027027027027
934 0.027027027027027
935 0.027027027027027
936 0.027027027027027
937 0.027027027027027
938 0.027027027027027
939 0.027027027027027
940 0.027027027027027
941 0.027027027027027
942 0.027027027027027
943 0.027027027027027
944 0.027027027027027
945 0.027027027027027
946 0.027027027027027
947 0.027027027027027
948 0.027027027027027
949 0.027027027027027
950 0.027027027027027
951 0.027027027027027
952 0.027027027027027
953 0.027027027027027
954 0.027027027027027
955 0.027027027027027
956 0.027027027027027
957 0.027027027027027
958 0.027027027027027
959 0.027027027027027
960 0.027027027027027
961 0.027027027027027
962 0.027027027027027
963 0.027027027027027
964 0.027027027027027
965 0.027027027027027
966 0.027027027027027
967 0.027027027027027
968 0.027027027027027
969 0.027027027027027
970 0.027027027027027
971 0.027027027027027
972 0.027027027027027
973 0.027027027027027
974 0.027027027027027
975 0.027027027027027
976 0.027027027027027
977 0.027027027027027
978 0.027027027027027
979 0.027027027027027
980 0.027027027027027
981 0.027027027027027
982 0.027027027027027
983 0.027027027027027
984 0.027027027027027
985 0.027027027027027
986 0.027027027027027
987 0.027027027027027
988 0.027027027027027
989 0.027027027027027
990 0.027027027027027
991 0.027027027027027
992 0.027027027027027
993 0.027027027027027
994 0.027027027027027
995 0.027027027027027
996 0.027027027027027
997 0.027027027027027
998 0.027027027027027
999 0.027027027027027
1000 0.027027027027027
};
\addlegendentry{$f_{r_{e}}$ (frecuencia relativa esperada de $18$)}
\end{axis}

\end{tikzpicture}

    \caption{frecuencia relativa con respecto al número de tiradas}
  \end{mytikzresize}
\end{figure}

\begin{figure}[!htbp]
  \begin{mytikzresize}{0.6\textwidth}
    \centering
    % This file was created by tikzplotlib v0.9.1.
\begin{tikzpicture}

\definecolor{color0}{rgb}{0.12156862745098,0.466666666666667,0.705882352941177}
\definecolor{color1}{rgb}{1,0.498039215686275,0.0549019607843137}

\begin{axis}[
legend cell align={left},
legend style={fill opacity=0.5, draw opacity=1, text opacity=1, draw=white!80!black},
scaled ticks=false,
tick align=outside,
tick pos=left,
width=\figW,
x grid style={white!69.0196078431373!black},
xlabel={\(\displaystyle n\) (número de tiradas)},
xmajorgrids,
xmin=-48.95, xmax=1049.95,
xtick style={color=black},
xticklabel style={/pgf/number format/.cd,fixed,precision=2},
y grid style={white!69.0196078431373!black},
ylabel={\(\displaystyle v_{p}\) (valor promedio)},
ymajorgrids,
ymin=17.25, ymax=33.75,
ytick style={color=black},
yticklabel style={/pgf/number format/.cd,fixed,precision=2}
]
\addplot [semithick, color0]
table {%
1 33
2 19
3 23.6666666666667
4 26.5
5 24.2
6 21.5
7 21.4285714285714
8 22.5
9 23.8888888888889
10 25.1
11 25.7272727272727
12 25.8333333333333
13 25.6153846153846
14 25.7857142857143
15 25.0666666666667
16 25.5
17 24.0588235294118
18 24.5
19 23.2105263157895
20 23.8
21 22.6666666666667
22 22.2727272727273
23 22.5652173913043
24 22.875
25 22.12
26 22.3846153846154
27 22
28 21.3571428571429
29 21.8275862068966
30 21.1333333333333
31 21.5161290322581
32 21.375
33 20.8484848484848
34 20.8529411764706
35 21.0285714285714
36 20.8055555555556
37 20.4864864864865
38 20.1052631578947
39 20.0769230769231
40 20.05
41 19.9024390243902
42 20.2142857142857
43 19.7441860465116
44 20
45 20.2222222222222
46 20.3913043478261
47 20.3191489361702
48 20.3958333333333
49 20.0204081632653
50 19.74
51 19.8039215686275
52 19.7307692307692
53 19.377358490566
54 19.3148148148148
55 19.4
56 19.5
57 19.7017543859649
58 19.6724137931034
59 19.5084745762712
60 19.7833333333333
61 19.7704918032787
62 19.5645161290323
63 19.3015873015873
64 19.03125
65 19.0923076923077
66 19.0757575757576
67 19.044776119403
68 18.8676470588235
69 18.6086956521739
70 18.3428571428571
71 18.169014084507
72 18.4166666666667
73 18.2191780821918
74 18.3108108108108
75 18.4533333333333
76 18.6447368421053
77 18.5584415584416
78 18.7051282051282
79 18.6962025316456
80 18.875
81 18.7283950617284
82 18.8658536585366
83 18.7228915662651
84 18.75
85 18.5529411764706
86 18.4418604651163
87 18.5632183908046
88 18.4090909090909
89 18.3707865168539
90 18.5333333333333
91 18.6703296703297
92 18.7065217391304
93 18.752688172043
94 18.6489361702128
95 18.5052631578947
96 18.6875
97 18.6701030927835
98 18.6326530612245
99 18.7979797979798
100 18.67
101 18.6732673267327
102 18.5490196078431
103 18.6699029126214
104 18.7019230769231
105 18.8190476190476
106 18.8679245283019
107 18.9626168224299
108 18.8518518518519
109 18.9633027522936
110 18.9272727272727
111 18.972972972973
112 19.0982142857143
113 19.0088495575221
114 18.8421052631579
115 18.9304347826087
116 18.9051724137931
117 18.7521367521368
118 18.8983050847458
119 18.9495798319328
120 19.0583333333333
121 19.0247933884298
122 18.9590163934426
123 18.9430894308943
124 18.7983870967742
125 18.688
126 18.8015873015873
127 18.7007874015748
128 18.71875
129 18.6589147286822
130 18.6307692307692
131 18.5114503816794
132 18.6060606060606
133 18.7218045112782
134 18.6044776119403
135 18.5925925925926
136 18.5735294117647
137 18.5620437956204
138 18.6521739130435
139 18.589928057554
140 18.4714285714286
141 18.468085106383
142 18.5352112676056
143 18.4685314685315
144 18.5763888888889
145 18.5931034482759
146 18.6164383561644
147 18.6802721088435
148 18.7837837837838
149 18.8053691275168
150 18.74
151 18.6158940397351
152 18.6315789473684
153 18.5098039215686
154 18.4545454545455
155 18.4129032258065
156 18.3205128205128
157 18.3630573248408
158 18.4430379746835
159 18.5471698113208
160 18.625
161 18.6459627329193
162 18.7160493827161
163 18.7484662576687
164 18.7134146341463
165 18.8121212121212
166 18.7048192771084
167 18.6287425149701
168 18.6607142857143
169 18.7396449704142
170 18.8
171 18.7836257309942
172 18.7732558139535
173 18.7341040462428
174 18.8103448275862
175 18.7142857142857
176 18.6477272727273
177 18.6949152542373
178 18.7247191011236
179 18.6927374301676
180 18.7888888888889
181 18.8342541436464
182 18.7362637362637
183 18.7267759562842
184 18.7880434782609
185 18.7675675675676
186 18.6720430107527
187 18.620320855615
188 18.6382978723404
189 18.7089947089947
190 18.7789473684211
191 18.8534031413613
192 18.828125
193 18.880829015544
194 18.7989690721649
195 18.8461538461538
196 18.9183673469388
197 18.9187817258883
198 18.8282828282828
199 18.8743718592965
200 18.935
201 19.0099502487562
202 18.9455445544554
203 18.9310344827586
204 19.0049019607843
205 19
206 19.0485436893204
207 19.0531400966184
208 19.0096153846154
209 19.0526315789474
210 19.0571428571429
211 19.0331753554502
212 19.0566037735849
213 19.0469483568075
214 18.9953271028037
215 18.9953488372093
216 19.0462962962963
217 18.9769585253456
218 18.954128440367
219 18.9543378995434
220 19.0227272727273
221 18.9864253393665
222 19.036036036036
223 18.9775784753363
224 18.9107142857143
225 18.9555555555556
226 18.9159292035398
227 18.8898678414097
228 18.9166666666667
229 18.9694323144105
230 18.9826086956522
231 19.04329004329
232 19
233 19.0557939914163
234 19.0128205128205
235 19.0297872340426
236 18.9576271186441
237 18.9071729957806
238 18.844537815126
239 18.8493723849372
240 18.8208333333333
241 18.7427385892116
242 18.7066115702479
243 18.7530864197531
244 18.7418032786885
245 18.7877551020408
246 18.7357723577236
247 18.6963562753036
248 18.7137096774194
249 18.7550200803213
250 18.808
251 18.7729083665339
252 18.797619047619
253 18.8181818181818
254 18.8307086614173
255 18.8705882352941
256 18.921875
257 18.8482490272374
258 18.8333333333333
259 18.8455598455598
260 18.8846153846154
261 18.8390804597701
262 18.9007633587786
263 18.893536121673
264 18.8977272727273
265 18.9094339622641
266 18.9398496240602
267 18.9213483146067
268 18.8917910447761
269 18.8847583643123
270 18.8814814814815
271 18.8634686346863
272 18.9007352941176
273 18.8534798534799
274 18.8321167883212
275 18.7709090909091
276 18.7065217391304
277 18.7111913357401
278 18.6438848920863
279 18.6917562724014
280 18.625
281 18.5800711743772
282 18.5673758865248
283 18.6113074204947
284 18.5669014084507
285 18.5087719298246
286 18.541958041958
287 18.5679442508711
288 18.5069444444444
289 18.4602076124567
290 18.5206896551724
291 18.4776632302405
292 18.4212328767123
293 18.4402730375427
294 18.4863945578231
295 18.4542372881356
296 18.3918918918919
297 18.3434343434343
298 18.3355704697987
299 18.3311036789298
300 18.3666666666667
301 18.3488372093023
302 18.3609271523179
303 18.3663366336634
304 18.3322368421053
305 18.3770491803279
306 18.4150326797386
307 18.3973941368078
308 18.4415584415584
309 18.3948220064725
310 18.3548387096774
311 18.3890675241158
312 18.3685897435897
313 18.3226837060703
314 18.328025477707
315 18.2920634920635
316 18.2721518987342
317 18.2996845425867
318 18.3490566037736
319 18.4012539184953
320 18.45
321 18.398753894081
322 18.4254658385093
323 18.4365325077399
324 18.4845679012346
325 18.48
326 18.4233128834356
327 18.4250764525994
328 18.4329268292683
329 18.4589665653495
330 18.4787878787879
331 18.4743202416918
332 18.4246987951807
333 18.4534534534535
334 18.502994011976
335 18.5283582089552
336 18.5654761904762
337 18.5964391691395
338 18.6449704142012
339 18.622418879056
340 18.5705882352941
341 18.524926686217
342 18.5233918128655
343 18.5247813411079
344 18.5406976744186
345 18.5913043478261
346 18.5375722543353
347 18.5389048991354
348 18.4885057471264
349 18.4383954154728
350 18.4342857142857
351 18.4700854700855
352 18.46875
353 18.4249291784703
354 18.4519774011299
355 18.4704225352113
356 18.4522471910112
357 18.406162464986
358 18.3966480446927
359 18.3509749303621
360 18.3694444444444
361 18.3961218836565
362 18.4254143646409
363 18.4297520661157
364 18.4752747252747
365 18.4712328767123
366 18.4371584699454
367 18.4359673024523
368 18.3885869565217
369 18.4227642276423
370 18.3756756756757
371 18.3477088948787
372 18.2983870967742
373 18.2761394101877
374 18.3155080213904
375 18.3173333333333
376 18.3457446808511
377 18.3580901856764
378 18.3439153439153
379 18.3060686015831
380 18.3421052631579
381 18.3070866141732
382 18.3350785340314
383 18.3812010443864
384 18.3958333333333
385 18.3532467532468
386 18.3082901554404
387 18.328165374677
388 18.3453608247423
389 18.3624678663239
390 18.3641025641026
391 18.3887468030691
392 18.4260204081633
393 18.4681933842239
394 18.4543147208122
395 18.4658227848101
396 18.5050505050505
397 18.4911838790932
398 18.4497487437186
399 18.468671679198
400 18.4625
401 18.4214463840399
402 18.3756218905473
403 18.3622828784119
404 18.3415841584158
405 18.3061728395062
406 18.3448275862069
407 18.3857493857494
408 18.4240196078431
409 18.4498777506112
410 18.4658536585366
411 18.4549878345499
412 18.4757281553398
413 18.4382566585956
414 18.4613526570048
415 18.4313253012048
416 18.40625
417 18.3717026378897
418 18.3468899521531
419 18.346062052506
420 18.3595238095238
421 18.3254156769596
422 18.3483412322275
423 18.3498817966903
424 18.3490566037736
425 18.3694117647059
426 18.3286384976526
427 18.3583138173302
428 18.3995327102804
429 18.3986013986014
430 18.4116279069767
431 18.445475638051
432 18.4861111111111
433 18.4480369515012
434 18.4654377880184
435 18.4758620689655
436 18.4977064220184
437 18.5102974828375
438 18.486301369863
439 18.4601366742597
440 18.4590909090909
441 18.4852607709751
442 18.4524886877828
443 18.451467268623
444 18.4864864864865
445 18.4516853932584
446 18.4439461883408
447 18.4765100671141
448 18.4709821428571
449 18.43429844098
450 18.3955555555556
451 18.4057649667406
452 18.3893805309735
453 18.3973509933775
454 18.4074889867841
455 18.3824175824176
456 18.3618421052632
457 18.3304157549234
458 18.3624454148472
459 18.3877995642702
460 18.3739130434783
461 18.3492407809111
462 18.3506493506494
463 18.3563714902808
464 18.364224137931
465 18.3311827956989
466 18.3347639484979
467 18.3447537473233
468 18.3482905982906
469 18.3347547974414
470 18.3255319148936
471 18.3163481953291
472 18.3072033898305
473 18.3086680761099
474 18.3291139240506
475 18.3242105263158
476 18.3172268907563
477 18.3333333333333
478 18.3556485355649
479 18.39248434238
480 18.4041666666667
481 18.4033264033264
482 18.3838174273859
483 18.3809523809524
484 18.3760330578512
485 18.3896907216495
486 18.40329218107
487 18.4127310061602
488 18.3893442622951
489 18.4212678936605
490 18.4061224489796
491 18.4052953156823
492 18.4308943089431
493 18.4320486815416
494 18.4008097165992
495 18.3919191919192
496 18.4173387096774
497 18.3843058350101
498 18.3855421686747
499 18.3667334669339
500 18.356
501 18.3832335329341
502 18.4123505976096
503 18.441351888668
504 18.4484126984127
505 18.4653465346535
506 18.4347826086957
507 18.4575936883629
508 18.4212598425197
509 18.4204322200393
510 18.4019607843137
511 18.4363992172211
512 18.41015625
513 18.4171539961014
514 18.4416342412451
515 18.4485436893204
516 18.4806201550388
517 18.5087040618955
518 18.5135135135135
519 18.4971098265896
520 18.5173076923077
521 18.5143953934741
522 18.5
523 18.4933078393881
524 18.5267175572519
525 18.527619047619
526 18.5152091254753
527 18.5313092979127
528 18.530303030303
529 18.5595463137996
530 18.5735849056604
531 18.5951035781544
532 18.5883458646617
533 18.5703564727955
534 18.5430711610487
535 18.5140186915888
536 18.5391791044776
537 18.5586592178771
538 18.5724907063197
539 18.595547309833
540 18.6259259259259
541 18.6192236598891
542 18.6217712177122
543 18.6187845303867
544 18.6011029411765
545 18.5981651376147
546 18.5732600732601
547 18.5575868372943
548 18.529197080292
549 18.4972677595628
550 18.4672727272727
551 18.4519056261343
552 18.4438405797101
553 18.4394213381555
554 18.4620938628159
555 18.4756756756757
556 18.4928057553957
557 18.5206463195691
558 18.5340501792115
559 18.5420393559928
560 18.5446428571429
561 18.5383244206774
562 18.5338078291815
563 18.5488454706927
564 18.5780141843972
565 18.553982300885
566 18.5742049469965
567 18.5502645502646
568 18.5316901408451
569 18.5008787346221
570 18.4719298245614
571 18.4658493870403
572 18.4615384615385
573 18.4432809773124
574 18.411149825784
575 18.4417391304348
576 18.4548611111111
577 18.4644714038128
578 18.439446366782
579 18.4697754749568
580 18.4862068965517
581 18.5060240963855
582 18.4879725085911
583 18.4957118353345
584 18.5222602739726
585 18.5435897435897
586 18.5443686006826
587 18.557069846678
588 18.547619047619
589 18.5398981324278
590 18.564406779661
591 18.5719120135364
592 18.5523648648649
593 18.5497470489039
594 18.5622895622896
595 18.5764705882353
596 18.5855704697987
597 18.6097152428811
598 18.5986622073579
599 18.5709515859766
600 18.5983333333333
601 18.5740432612313
602 18.5548172757475
603 18.5555555555556
604 18.5298013245033
605 18.5553719008264
606 18.5610561056106
607 18.5535420098847
608 18.5378289473684
609 18.5467980295567
610 18.5229508196721
611 18.5417348608838
612 18.5702614379085
613 18.5970636215334
614 18.586319218241
615 18.5869918699187
616 18.5941558441558
617 18.6029173419773
618 18.6181229773463
619 18.6187399030695
620 18.6290322580645
621 18.6505636070853
622 18.6414790996785
623 18.6356340288925
624 18.6634615384615
625 18.6624
626 18.6485623003195
627 18.6283891547049
628 18.6082802547771
629 18.6168521462639
630 18.6063492063492
631 18.5927099841521
632 18.5949367088608
633 18.5750394944708
634 18.5567823343849
635 18.5322834645669
636 18.5581761006289
637 18.5384615384615
638 18.5282131661442
639 18.5430359937402
640 18.5265625
641 18.5179407176287
642 18.5451713395639
643 18.5287713841369
644 18.527950310559
645 18.522480620155
646 18.5030959752322
647 18.5177743431221
648 18.5339506172839
649 18.5608628659476
650 18.5707692307692
651 18.5975422427035
652 18.5889570552147
653 18.5803981623277
654 18.5718654434251
655 18.5587786259542
656 18.5564024390244
657 18.5738203957382
658 18.5577507598784
659 18.5629742033384
660 18.55
661 18.5461422087746
662 18.5725075528701
663 18.5927601809955
664 18.6039156626506
665 18.6210526315789
666 18.6381381381381
667 18.6371814092954
668 18.6212574850299
669 18.6143497757848
670 18.6149253731343
671 18.5991058122206
672 18.6026785714286
673 18.6092124814265
674 18.5979228486647
675 18.597037037037
676 18.5695266272189
677 18.5480059084195
678 18.5221238938053
679 18.5346097201767
680 18.5132352941176
681 18.5051395007342
682 18.5234604105572
683 18.5051244509517
684 18.4985380116959
685 18.5138686131387
686 18.5262390670554
687 18.5065502183406
688 18.5145348837209
689 18.5268505079826
690 18.5478260869565
691 18.5600578871201
692 18.5404624277457
693 18.5425685425685
694 18.5403458213256
695 18.5352517985612
696 18.5244252873563
697 18.5164992826399
698 18.4957020057307
699 18.5064377682403
700 18.51
701 18.5335235378031
702 18.5128205128205
703 18.4950213371266
704 18.5
705 18.4936170212766
706 18.4957507082153
707 18.4893917963225
708 18.4915254237288
709 18.490832157969
710 18.4915492957746
711 18.5105485232068
712 18.5
713 18.5133239831697
714 18.5196078431373
715 18.5160839160839
716 18.5111731843575
717 18.5355648535565
718 18.5292479108635
719 18.547983310153
720 18.5708333333333
721 18.5492371705964
722 18.5734072022161
723 18.5532503457815
724 18.5345303867403
725 18.5268965517241
726 18.5440771349862
727 18.5378266850069
728 18.5260989010989
729 18.5089163237311
730 18.5150684931507
731 18.5075239398085
732 18.5081967213115
733 18.5034106412005
734 18.5013623978202
735 18.5034013605442
736 18.4972826086957
737 18.4776119402985
738 18.4620596205962
739 18.4803788903924
740 18.4675675675676
741 18.4777327935223
742 18.4636118598383
743 18.4401076716016
744 18.4596774193548
745 18.4778523489933
746 18.4973190348525
747 18.4832663989291
748 18.5
749 18.5233644859813
750 18.5253333333333
751 18.5019973368842
752 18.5066489361702
753 18.5073041168659
754 18.4840848806366
755 18.4609271523179
756 18.4391534391534
757 18.4332892998679
758 18.4248021108179
759 18.4281949934124
760 18.4197368421053
761 18.4047306176084
762 18.3818897637795
763 18.3997378768021
764 18.3926701570681
765 18.4078431372549
766 18.4177545691906
767 18.4289439374185
768 18.43359375
769 18.4161248374512
770 18.3935064935065
771 18.4098573281453
772 18.4132124352332
773 18.4359637774903
774 18.4586563307494
775 18.4812903225806
776 18.4948453608247
777 18.5006435006435
778 18.5205655526992
779 18.5160462130937
780 18.5115384615385
781 18.5326504481434
782 18.5191815856777
783 18.4955300127714
784 18.4808673469388
785 18.4815286624204
786 18.4580152671756
787 18.4548919949174
788 18.4441624365482
789 18.4423320659062
790 18.4582278481013
791 18.4551201011378
792 18.4444444444444
793 18.4287515762926
794 18.4420654911839
795 18.4327044025157
796 18.4158291457286
797 18.4065244667503
798 18.4022556390977
799 18.4180225281602
800 18.4
801 18.3895131086142
802 18.3852867830424
803 18.3860523038605
804 18.384328358209
805 18.3614906832298
806 18.3796526054591
807 18.3767038413879
808 18.3762376237624
809 18.3918417799753
810 18.3913580246914
811 18.3760789149199
812 18.3546798029557
813 18.3542435424354
814 18.3648648648649
815 18.3865030674847
816 18.4031862745098
817 18.4014687882497
818 18.3973105134474
819 18.4114774114774
820 18.4121951219512
821 18.4250913520097
822 18.4379562043796
823 18.4313487241798
824 18.4368932038835
825 18.4436363636364
826 18.455205811138
827 18.4425634824667
828 18.4408212560386
829 18.4463208685163
830 18.4566265060241
831 18.4657039711191
832 18.4831730769231
833 18.4861944777911
834 18.484412470024
835 18.4634730538922
836 18.4461722488038
837 18.4647550776583
838 18.4677804295943
839 18.4874851013111
840 18.502380952381
841 18.4910820451843
842 18.4762470308789
843 18.4768683274021
844 18.4976303317536
845 18.5076923076923
846 18.4988179669031
847 18.5053128689492
848 18.5094339622642
849 18.5135453474676
850 18.5176470588235
851 18.4994124559342
852 18.4906103286385
853 18.5087924970692
854 18.5222482435597
855 18.5146198830409
856 18.4976635514019
857 18.5157526254376
858 18.5314685314685
859 18.5261932479627
860 18.5453488372093
861 18.5447154471545
862 18.5406032482599
863 18.5596755504056
864 18.5601851851852
865 18.5526011560694
866 18.5450346420323
867 18.5340253748558
868 18.536866359447
869 18.5466052934407
870 18.5505747126437
871 18.5579793340987
872 18.5584862385321
873 18.5498281786942
874 18.558352402746
875 18.5485714285714
876 18.5296803652968
877 18.5381984036488
878 18.5261958997722
879 18.5392491467577
880 18.5340909090909
881 18.5255391600454
882 18.5351473922902
883 18.5424688561721
884 18.5542986425339
885 18.554802259887
886 18.538374717833
887 18.5231116121759
888 18.5225225225225
889 18.5129358830146
890 18.5101123595506
891 18.5230078563412
892 18.5179372197309
893 18.498320268757
894 18.4809843400447
895 18.4849162011173
896 18.4654017857143
897 18.4526198439242
898 18.4487750556793
899 18.4527252502781
900 18.4666666666667
901 18.4506104328524
902 18.4634146341463
903 18.4828349944629
904 18.4922566371681
905 18.5038674033149
906 18.4900662251656
907 18.4895259095921
908 18.4702643171806
909 18.4895489548955
910 18.4835164835165
911 18.4950603732162
912 18.5087719298246
913 18.5279299014239
914 18.5164113785558
915 18.5234972677596
916 18.5240174672489
917 18.5081788440567
918 18.520697167756
919 18.5277475516866
920 18.5391304347826
921 18.5526601520087
922 18.556399132321
923 18.5698808234019
924 18.5649350649351
925 18.5827027027027
926 18.5885529157667
927 18.5706580366775
928 18.5646551724138
929 18.5715823466093
930 18.5817204301075
931 18.5639097744361
932 18.5643776824034
933 18.561629153269
934 18.5642398286938
935 18.579679144385
936 18.5683760683761
937 18.5602988260406
938 18.5607675906183
939 18.5580404685836
940 18.5478723404255
941 18.5366631243358
942 18.552016985138
943 18.5641569459173
944 18.5646186440678
945 18.5746031746032
946 18.5877378435518
947 18.5797254487856
948 18.5886075949367
949 18.5890410958904
950 18.5747368421053
951 18.589905362776
952 18.6081932773109
953 18.601259181532
954 18.6006289308176
955 18.6031413612565
956 18.5836820083682
957 18.564263322884
958 18.5782881002088
959 18.5933263816476
960 18.5916666666667
961 18.5889698231009
962 18.5945945945946
963 18.5939771547248
964 18.606846473029
965 18.6103626943005
966 18.6055900621118
967 18.5977249224405
968 18.5981404958678
969 18.5892672858617
970 18.6030927835052
971 18.6076210092688
972 18.5895061728395
973 18.5991778006167
974 18.5862422997947
975 18.5805128205128
976 18.5881147540984
977 18.5926305015353
978 18.5869120654397
979 18.5720122574055
980 18.5857142857143
981 18.5779816513761
982 18.5600814663951
983 18.5462868769074
984 18.5609756097561
985 18.5532994923858
986 18.5405679513185
987 18.5339412360689
988 18.5253036437247
989 18.5136501516684
990 18.5080808080808
991 18.5095862764884
992 18.5181451612903
993 18.5095669687815
994 18.5160965794769
995 18.5256281407035
996 18.5160642570281
997 18.4994984954865
998 18.5130260521042
999 18.5005005005005
1000 18.489
};
\addlegendentry{$v_{p}$ (valor promedio de las tiradas)}
\addplot [semithick, color1, dashed]
table {%
1 18
2 18
3 18
4 18
5 18
6 18
7 18
8 18
9 18
10 18
11 18
12 18
13 18
14 18
15 18
16 18
17 18
18 18
19 18
20 18
21 18
22 18
23 18
24 18
25 18
26 18
27 18
28 18
29 18
30 18
31 18
32 18
33 18
34 18
35 18
36 18
37 18
38 18
39 18
40 18
41 18
42 18
43 18
44 18
45 18
46 18
47 18
48 18
49 18
50 18
51 18
52 18
53 18
54 18
55 18
56 18
57 18
58 18
59 18
60 18
61 18
62 18
63 18
64 18
65 18
66 18
67 18
68 18
69 18
70 18
71 18
72 18
73 18
74 18
75 18
76 18
77 18
78 18
79 18
80 18
81 18
82 18
83 18
84 18
85 18
86 18
87 18
88 18
89 18
90 18
91 18
92 18
93 18
94 18
95 18
96 18
97 18
98 18
99 18
100 18
101 18
102 18
103 18
104 18
105 18
106 18
107 18
108 18
109 18
110 18
111 18
112 18
113 18
114 18
115 18
116 18
117 18
118 18
119 18
120 18
121 18
122 18
123 18
124 18
125 18
126 18
127 18
128 18
129 18
130 18
131 18
132 18
133 18
134 18
135 18
136 18
137 18
138 18
139 18
140 18
141 18
142 18
143 18
144 18
145 18
146 18
147 18
148 18
149 18
150 18
151 18
152 18
153 18
154 18
155 18
156 18
157 18
158 18
159 18
160 18
161 18
162 18
163 18
164 18
165 18
166 18
167 18
168 18
169 18
170 18
171 18
172 18
173 18
174 18
175 18
176 18
177 18
178 18
179 18
180 18
181 18
182 18
183 18
184 18
185 18
186 18
187 18
188 18
189 18
190 18
191 18
192 18
193 18
194 18
195 18
196 18
197 18
198 18
199 18
200 18
201 18
202 18
203 18
204 18
205 18
206 18
207 18
208 18
209 18
210 18
211 18
212 18
213 18
214 18
215 18
216 18
217 18
218 18
219 18
220 18
221 18
222 18
223 18
224 18
225 18
226 18
227 18
228 18
229 18
230 18
231 18
232 18
233 18
234 18
235 18
236 18
237 18
238 18
239 18
240 18
241 18
242 18
243 18
244 18
245 18
246 18
247 18
248 18
249 18
250 18
251 18
252 18
253 18
254 18
255 18
256 18
257 18
258 18
259 18
260 18
261 18
262 18
263 18
264 18
265 18
266 18
267 18
268 18
269 18
270 18
271 18
272 18
273 18
274 18
275 18
276 18
277 18
278 18
279 18
280 18
281 18
282 18
283 18
284 18
285 18
286 18
287 18
288 18
289 18
290 18
291 18
292 18
293 18
294 18
295 18
296 18
297 18
298 18
299 18
300 18
301 18
302 18
303 18
304 18
305 18
306 18
307 18
308 18
309 18
310 18
311 18
312 18
313 18
314 18
315 18
316 18
317 18
318 18
319 18
320 18
321 18
322 18
323 18
324 18
325 18
326 18
327 18
328 18
329 18
330 18
331 18
332 18
333 18
334 18
335 18
336 18
337 18
338 18
339 18
340 18
341 18
342 18
343 18
344 18
345 18
346 18
347 18
348 18
349 18
350 18
351 18
352 18
353 18
354 18
355 18
356 18
357 18
358 18
359 18
360 18
361 18
362 18
363 18
364 18
365 18
366 18
367 18
368 18
369 18
370 18
371 18
372 18
373 18
374 18
375 18
376 18
377 18
378 18
379 18
380 18
381 18
382 18
383 18
384 18
385 18
386 18
387 18
388 18
389 18
390 18
391 18
392 18
393 18
394 18
395 18
396 18
397 18
398 18
399 18
400 18
401 18
402 18
403 18
404 18
405 18
406 18
407 18
408 18
409 18
410 18
411 18
412 18
413 18
414 18
415 18
416 18
417 18
418 18
419 18
420 18
421 18
422 18
423 18
424 18
425 18
426 18
427 18
428 18
429 18
430 18
431 18
432 18
433 18
434 18
435 18
436 18
437 18
438 18
439 18
440 18
441 18
442 18
443 18
444 18
445 18
446 18
447 18
448 18
449 18
450 18
451 18
452 18
453 18
454 18
455 18
456 18
457 18
458 18
459 18
460 18
461 18
462 18
463 18
464 18
465 18
466 18
467 18
468 18
469 18
470 18
471 18
472 18
473 18
474 18
475 18
476 18
477 18
478 18
479 18
480 18
481 18
482 18
483 18
484 18
485 18
486 18
487 18
488 18
489 18
490 18
491 18
492 18
493 18
494 18
495 18
496 18
497 18
498 18
499 18
500 18
501 18
502 18
503 18
504 18
505 18
506 18
507 18
508 18
509 18
510 18
511 18
512 18
513 18
514 18
515 18
516 18
517 18
518 18
519 18
520 18
521 18
522 18
523 18
524 18
525 18
526 18
527 18
528 18
529 18
530 18
531 18
532 18
533 18
534 18
535 18
536 18
537 18
538 18
539 18
540 18
541 18
542 18
543 18
544 18
545 18
546 18
547 18
548 18
549 18
550 18
551 18
552 18
553 18
554 18
555 18
556 18
557 18
558 18
559 18
560 18
561 18
562 18
563 18
564 18
565 18
566 18
567 18
568 18
569 18
570 18
571 18
572 18
573 18
574 18
575 18
576 18
577 18
578 18
579 18
580 18
581 18
582 18
583 18
584 18
585 18
586 18
587 18
588 18
589 18
590 18
591 18
592 18
593 18
594 18
595 18
596 18
597 18
598 18
599 18
600 18
601 18
602 18
603 18
604 18
605 18
606 18
607 18
608 18
609 18
610 18
611 18
612 18
613 18
614 18
615 18
616 18
617 18
618 18
619 18
620 18
621 18
622 18
623 18
624 18
625 18
626 18
627 18
628 18
629 18
630 18
631 18
632 18
633 18
634 18
635 18
636 18
637 18
638 18
639 18
640 18
641 18
642 18
643 18
644 18
645 18
646 18
647 18
648 18
649 18
650 18
651 18
652 18
653 18
654 18
655 18
656 18
657 18
658 18
659 18
660 18
661 18
662 18
663 18
664 18
665 18
666 18
667 18
668 18
669 18
670 18
671 18
672 18
673 18
674 18
675 18
676 18
677 18
678 18
679 18
680 18
681 18
682 18
683 18
684 18
685 18
686 18
687 18
688 18
689 18
690 18
691 18
692 18
693 18
694 18
695 18
696 18
697 18
698 18
699 18
700 18
701 18
702 18
703 18
704 18
705 18
706 18
707 18
708 18
709 18
710 18
711 18
712 18
713 18
714 18
715 18
716 18
717 18
718 18
719 18
720 18
721 18
722 18
723 18
724 18
725 18
726 18
727 18
728 18
729 18
730 18
731 18
732 18
733 18
734 18
735 18
736 18
737 18
738 18
739 18
740 18
741 18
742 18
743 18
744 18
745 18
746 18
747 18
748 18
749 18
750 18
751 18
752 18
753 18
754 18
755 18
756 18
757 18
758 18
759 18
760 18
761 18
762 18
763 18
764 18
765 18
766 18
767 18
768 18
769 18
770 18
771 18
772 18
773 18
774 18
775 18
776 18
777 18
778 18
779 18
780 18
781 18
782 18
783 18
784 18
785 18
786 18
787 18
788 18
789 18
790 18
791 18
792 18
793 18
794 18
795 18
796 18
797 18
798 18
799 18
800 18
801 18
802 18
803 18
804 18
805 18
806 18
807 18
808 18
809 18
810 18
811 18
812 18
813 18
814 18
815 18
816 18
817 18
818 18
819 18
820 18
821 18
822 18
823 18
824 18
825 18
826 18
827 18
828 18
829 18
830 18
831 18
832 18
833 18
834 18
835 18
836 18
837 18
838 18
839 18
840 18
841 18
842 18
843 18
844 18
845 18
846 18
847 18
848 18
849 18
850 18
851 18
852 18
853 18
854 18
855 18
856 18
857 18
858 18
859 18
860 18
861 18
862 18
863 18
864 18
865 18
866 18
867 18
868 18
869 18
870 18
871 18
872 18
873 18
874 18
875 18
876 18
877 18
878 18
879 18
880 18
881 18
882 18
883 18
884 18
885 18
886 18
887 18
888 18
889 18
890 18
891 18
892 18
893 18
894 18
895 18
896 18
897 18
898 18
899 18
900 18
901 18
902 18
903 18
904 18
905 18
906 18
907 18
908 18
909 18
910 18
911 18
912 18
913 18
914 18
915 18
916 18
917 18
918 18
919 18
920 18
921 18
922 18
923 18
924 18
925 18
926 18
927 18
928 18
929 18
930 18
931 18
932 18
933 18
934 18
935 18
936 18
937 18
938 18
939 18
940 18
941 18
942 18
943 18
944 18
945 18
946 18
947 18
948 18
949 18
950 18
951 18
952 18
953 18
954 18
955 18
956 18
957 18
958 18
959 18
960 18
961 18
962 18
963 18
964 18
965 18
966 18
967 18
968 18
969 18
970 18
971 18
972 18
973 18
974 18
975 18
976 18
977 18
978 18
979 18
980 18
981 18
982 18
983 18
984 18
985 18
986 18
987 18
988 18
989 18
990 18
991 18
992 18
993 18
994 18
995 18
996 18
997 18
998 18
999 18
1000 18
};
\addlegendentry{$v_{p_{e}}$ (valor promedio esperado)}
\end{axis}

\end{tikzpicture}

    \caption{valor promedio con respecto al número de tiradas}
  \end{mytikzresize}
\end{figure}

\begin{figure}[!htbp]
  \begin{mytikzresize}{0.6\textwidth}
    \centering
    % This file was created by tikzplotlib v0.9.1.
\begin{tikzpicture}

\definecolor{color0}{rgb}{0.12156862745098,0.466666666666667,0.705882352941177}
\definecolor{color1}{rgb}{1,0.498039215686275,0.0549019607843137}

\begin{axis}[
legend cell align={left},
legend style={fill opacity=0.5, draw opacity=1, text opacity=1, at={(0.97,0.03)}, anchor=south east, draw=white!80!black},
scaled ticks=false,
tick align=outside,
tick pos=left,
width=\figW,
x grid style={white!69.0196078431373!black},
xlabel={\(\displaystyle n\) (número de tiradas)},
xmajorgrids,
xmin=-48.95, xmax=1049.95,
xtick style={color=black},
xticklabel style={/pgf/number format/.cd,fixed,precision=2},
y grid style={white!69.0196078431373!black},
ylabel={\(\displaystyle v_{d}\) (valor del desvío)},
ymajorgrids,
ymin=-0.581377674149945, ymax=12.2089311571489,
ytick style={color=black},
yticklabel style={/pgf/number format/.cd,fixed,precision=2}
]
\addplot [semithick, color0]
table {%
1 0
2 11
3 9.0921211313239
4 9.4339811320566
5 11.6275534829989
6 11.3785861257989
7 10.5540397527222
8 10.2583807201722
9 9.94553066693344
10 9.94183081730925
11 9.48770409138164
12 9.0905934288631
13 9.5867265238864
14 9.28681312179715
15 9.21496367630147
16 8.99978298349466
17 9.36865125285267
18 9.31479640672653
19 9.90732681674223
20 10.3469802357983
21 10.1084818788823
22 10.133917356666
23 10.4727569885584
24 10.2610881759955
25 10.6572792024982
26 10.7155254284193
27 10.5174751699549
28 10.3560960062079
29 10.490128987916
30 10.5953868368372
31 10.4614648906685
32 10.7084954592137
33 10.5528535520015
34 10.4267505977501
35 10.3993720375564
36 10.2968291895139
37 10.1644620744157
38 10.2641025206307
39 10.1336239966618
40 10.4264087777144
41 10.3004247846196
42 10.4398721115354
43 10.32259541992
44 10.3365625230726
45 10.2567051239665
46 10.3676281118396
47 10.3866404425644
48 10.3745607336729
49 10.6387815782348
50 10.5578217450381
51 10.5309999897595
52 10.4617505634607
53 10.3665810244261
54 10.3983748862189
55 10.3040391662928
56 10.4615427901921
57 10.4386112339569
58 10.4002681068524
59 10.3182899335654
60 10.293727863769
61 10.2093380381181
62 10.214975947638
63 10.1895366228579
64 10.3695392255394
65 10.3031448062913
66 10.2303250215188
67 10.1739164789369
68 10.2426444195765
69 10.1688537732402
70 10.1961277163521
71 10.1561308894638
72 10.0864224683421
73 10.0181295595733
74 9.97257012297032
75 9.9411602273913
76 9.89067665857816
77 9.96688732630236
78 9.91546285766773
79 9.96569986756296
80 10.0466956632517
81 10.0238551010439
82 10.0632443857299
83 10.0383934187678
84 9.9794786602841
85 9.93355085183297
86 9.98340971026143
87 9.92941131903373
88 9.89714015584617
89 9.99449412875843
90 9.98700390074923
91 9.93229100160252
92 10.0122030929785
93 10.1218726837006
94 10.1241277589297
95 10.161444702834
96 10.1606400768275
97 10.2061334998151
98 10.2485409004041
99 10.28322517144
100 10.2365765761801
101 10.2565655818864
102 10.2181886965533
103 10.2478108171209
104 10.3009270278522
105 10.264616810999
106 10.2264499059492
107 10.2584813624505
108 10.327503167442
109 10.2959734902506
110 10.259847114354
111 10.2208388416856
112 10.203275159457
113 10.2198252096675
114 10.2853860353999
115 10.3475191770343
116 10.3047499523982
117 10.3915315859539
118 10.4188806418616
119 10.4309851688324
120 10.457320668104
121 10.4402312746918
122 10.4371527479743
123 10.4601513537226
124 10.4649954148226
125 10.4367473860394
126 10.4585860984954
127 10.4301496430321
128 10.4036425740504
129 10.3806467550757
130 10.3435341264411
131 10.3042012372307
132 10.279295299082
133 10.3556082811669
134 10.3170636415171
135 10.2827248687798
136 10.3558965534427
137 10.4140351694133
138 10.4824294632597
139 10.5479059515496
140 10.5531212135055
141 10.5247318915622
142 10.5855167117506
143 10.6122423645667
144 10.5969855805819
145 10.5724295386436
146 10.5755398696886
147 10.539841093019
148 10.5198753592841
149 10.4960821353438
150 10.5630151419417
151 10.6064104748208
152 10.6072464516159
153 10.6116822614816
154 10.5815090424021
155 10.5573504385825
156 10.541430347937
157 10.5163884360246
158 10.5007992157361
159 10.469380444975
160 10.4703730085179
161 10.4832172279028
162 10.4907130455578
163 10.4646464859307
164 10.5108015034579
165 10.54925044883
166 10.5383418183254
167 10.5747858753912
168 10.6110636225505
169 10.5829820854561
170 10.5881111103829
171 10.562407262567
172 10.6166802440821
173 10.6010678194271
174 10.6522829595335
175 10.6276558745452
176 10.6031861635162
177 10.5839824126638
178 10.5881827376156
179 10.6318969871101
180 10.6251666653595
181 10.6645777870808
182 10.6549951491224
183 10.6262587891863
184 10.6266775613283
185 10.626776824747
186 10.6119172445654
187 10.597049753004
188 10.6199242929187
189 10.594381657583
190 10.5747621991167
191 10.5936738586952
192 10.6161676933372
193 10.6051405402924
194 10.60294618726
195 10.5757292255524
196 10.5967361166791
197 10.5786095655153
198 10.5709580245009
199 10.5848853075895
200 10.6346179997215
201 10.6609892442516
202 10.6429161672303
203 10.6398444484932
204 10.6654816599548
205 10.7134945233497
206 10.7253576723804
207 10.7560535642697
208 10.7562421673245
209 10.7327943994553
210 10.7079145078104
211 10.7404637516331
212 10.770018698999
213 10.7457804706635
214 10.721704733744
215 10.7268258016685
216 10.7342563040911
217 10.7726579976664
218 10.7655742339561
219 10.7542380132348
220 10.7299341844816
221 10.7063757569621
222 10.6921462256101
223 10.66842626769
224 10.658451991356
225 10.6452692791066
226 10.6316044230709
227 10.6090856571369
228 10.5857968574022
229 10.5633973038018
230 10.5812331641497
231 10.6238529090196
232 10.6216832271483
233 10.6186606615371
234 10.5959502516622
235 10.6176054075442
236 10.6043128364146
237 10.582037897924
238 10.5701158822141
239 10.5855623707479
240 10.5841043026271
241 10.613387639032
242 10.6286839678194
243 10.6074187747242
244 10.614342671943
245 10.6165109164915
246 10.5981378931786
247 10.6036508276068
248 10.6006760811135
249 10.6287521114015
250 10.6076199026926
251 10.5929459828305
252 10.5763155144609
253 10.5560225208593
254 10.5447859335797
255 10.5421101360554
256 10.5567546319584
257 10.5423087909899
258 10.5450813681656
259 10.5545369333294
260 10.5413499691262
261 10.533440197735
262 10.5321410448511
263 10.548201502184
264 10.567841718216
265 10.5480061773902
266 10.5349373929657
267 10.5178195615039
268 10.5130390085746
269 10.493482859158
270 10.5183026842749
271 10.5275750386801
272 10.5248074676848
273 10.531640695003
274 10.547354329919
275 10.5564231357732
276 10.571872096565
277 10.5567468233121
278 10.5656226444206
279 10.571980940311
280 10.5643889447289
281 10.587671271034
282 10.5697037989941
283 10.5518272385826
284 10.5580003905839
285 10.5940138849213
286 10.5779887343873
287 10.591223809364
288 10.572936621948
289 10.5905429785019
290 10.5777125305106
291 10.5597678642341
292 10.5419141938066
293 10.5348294858645
294 10.5475685919727
295 10.5518329129334
296 10.5734062937996
297 10.5945274260846
298 10.5884581148633
299 10.5912651450387
300 10.5785627043039
301 10.5612815189045
302 10.5686302496243
303 10.5792695669268
304 10.5629128408559
305 10.5456069202192
306 10.5283857818972
307 10.51966790704
308 10.5115390097189
309 10.4974859852617
310 10.4888965622039
311 10.4764122412681
312 10.4600300571673
313 10.4522357369075
314 10.4598113488964
315 10.460170531521
316 10.4480823068352
317 10.4324776098203
318 10.4616228184527
319 10.4955421820742
320 10.4811568272543
321 10.4911064527496
322 10.5240547768264
323 10.508216090837
324 10.5032691726681
325 10.5056013236558
326 10.5304474410691
327 10.5267845898686
328 10.514697158526
329 10.5154452206015
330 10.5104888797903
331 10.529980807051
332 10.52654048143
333 10.5153425186315
334 10.5094277162946
335 10.5044670085156
336 10.4888384148349
337 10.5205697050571
338 10.5069514870292
339 10.5125525743062
340 10.5434608869217
341 10.5351627596225
342 10.5208611598327
343 10.5319221796
344 10.5168066933818
345 10.5175557619653
346 10.5455152558371
347 10.5758773021004
348 10.5608804305492
349 10.5467723471435
350 10.5766303804228
351 10.5815333319636
352 10.5669461032196
353 10.5539368057996
354 10.5588126704084
355 10.5817215417692
356 10.572996393921
357 10.5887273313564
358 10.5956035408115
359 10.6135495690669
360 10.6139027014362
361 10.606086427053
362 10.5973421840407
363 10.5829457250629
364 10.5742678984968
365 10.6002779344451
366 10.6032788152519
367 10.5983776761712
368 10.5929810552578
369 10.5989766198664
370 10.5855021656671
371 10.576450532239
372 10.5691605840529
373 10.5552222948478
374 10.5425379816653
375 10.545567915796
376 10.5388218137427
377 10.5272817097302
378 10.5157771783593
379 10.5364093332583
380 10.5489687172363
381 10.540410532826
382 10.5335568289001
383 10.5456754218046
384 10.5699294707641
385 10.5586520961102
386 10.5708193281167
387 10.5596217723826
388 10.5832045258477
389 10.5698875888911
390 10.5563517122496
391 10.5436345196475
392 10.5316998330957
393 10.5234525865593
394 10.5352969779628
395 10.5255021029929
396 10.51572974812
397 10.5032787756697
398 10.5006661477127
399 10.5157077470527
400 10.512941726748
401 10.5241304230296
402 10.519154614164
403 10.5162268119025
404 10.5381339980086
405 10.5521516423374
406 10.5392281225112
407 10.5383186950561
408 10.5351154814924
409 10.5392604023136
410 10.5269083837078
411 10.5363951467059
412 10.5490192302749
413 10.5415976340496
414 10.5319275341478
415 10.5490513848975
416 10.5421445259312
417 10.5296365904223
418 10.5395928120299
419 10.5326660071403
420 10.5219778625784
421 10.5457458490938
422 10.5549333063758
423 10.5539064045192
424 10.5484175435726
425 10.5416428468258
426 10.5648306255258
427 10.5525970572451
428 10.5585316817419
429 10.5715630462322
430 10.591472073903
431 10.5802260130478
432 10.5734616540399
433 10.566479869339
434 10.5597690715269
435 10.5751739029836
436 10.5648672554164
437 10.5544187586061
438 10.542843307444
439 10.5359608244154
440 10.5583727693802
441 10.5492495247506
442 10.5423377735325
443 10.5539476445126
444 10.5531309555438
445 10.5477793653072
446 10.565901735051
447 10.5612343211018
448 10.5604393594341
449 10.5595990180953
450 10.5580217002155
451 10.5468115955338
452 10.5523382873895
453 10.5775086533584
454 10.5985641477616
455 10.5874544401741
456 10.576955812873
457 10.5661425852921
458 10.5655118844208
459 10.5763428735434
460 10.5671138036782
461 10.5617167617257
462 10.5787166289862
463 10.5767640606551
464 10.5665217684438
465 10.5768867352164
466 10.5968674613129
467 10.612758435901
468 10.6325016059746
469 10.6515890401416
470 10.6586749524915
471 10.6741905186205
472 10.6629129080594
473 10.6559477521645
474 10.6657179667439
475 10.6615140833034
476 10.6658552818534
477 10.663612223038
478 10.654413056005
479 10.6474653038472
480 10.6363907487325
481 10.625355152682
482 10.6251189888382
483 10.631837482888
484 10.6208701299273
485 10.6330990400771
486 10.651919174796
487 10.6497088261981
488 10.6390064105594
489 10.6542032176297
490 10.646175935869
491 10.6381637394776
492 10.6312991887326
493 10.620704376153
494 10.6099660137515
495 10.5992755593687
496 10.6013179486165
497 10.5947217335189
498 10.5846914443782
499 10.5781220961784
500 10.5687121258931
501 10.5728465666336
502 10.5634352981861
503 10.5786467058361
504 10.5935702197729
505 10.5977711643346
506 10.6041944896326
507 10.6020228336526
508 10.5968960591492
509 10.5883109042386
510 10.5784740772185
511 10.5700468083231
512 10.5739941779233
513 10.5718443596048
514 10.5785500665161
515 10.5805538381108
516 10.5742250388464
517 10.5705300653724
518 10.5655305400529
519 10.5619923205905
520 10.5518521047367
521 10.5498715112746
522 10.5709110830172
523 10.5727227088854
524 10.5626492015554
525 10.553162425937
526 10.5495836596869
527 10.5413539119502
528 10.5559270684887
529 10.5470278697204
530 10.5469721856802
531 10.5375903606356
532 10.5374784730216
533 10.5488471839035
534 10.5508110838355
535 10.5473970695104
536 10.538621435879
537 10.5306154465996
538 10.5257543060135
539 10.5165204776137
540 10.5067970683976
541 10.5154395271232
542 10.5083208453074
543 10.5121415310905
544 10.5072571055194
545 10.5109873535683
546 10.5168900305849
547 10.5089069787772
548 10.5085679772454
549 10.519274713301
550 10.512527906037
551 10.5032349120062
552 10.5067391181903
553 10.5158597016937
554 10.5079513915937
555 10.5030369997754
556 10.4951552751876
557 10.4890924537357
558 10.4862039802197
559 10.4885241934076
560 10.5001018884755
561 10.4923268727674
562 10.483135310373
563 10.4753982108115
564 10.4672592122642
565 10.4814795884412
566 10.4850416887404
567 10.4776292220442
568 10.471766459352
569 10.4698412560995
570 10.4868337038715
571 10.4834412224526
572 10.4743118328627
573 10.4667224265738
574 10.4591485755507
575 10.4588698800703
576 10.4569393657014
577 10.448121027325
578 10.4405910569983
579 10.4316153666163
580 10.4244662991387
581 10.4187412959273
582 10.4102099591045
583 10.4015263705046
584 10.411765469838
585 10.4172300614681
586 10.4298382674192
587 10.4497833450829
588 10.4458119268466
589 10.4373462290018
590 10.4472999306173
591 10.4462173283926
592 10.4380143439315
593 10.4298305054233
594 10.425913421304
595 10.442507363228
596 10.4589347083509
597 10.4549096309917
598 10.4530979388114
599 10.4444078205639
600 10.4529081546185
601 10.4562846598278
602 10.451832126646
603 10.4571617925475
604 10.4633974961694
605 10.4666690883914
606 10.4582641965217
607 10.4564170334027
608 10.4685750595215
609 10.460022475064
610 10.4596464171209
611 10.4558144704821
612 10.4704274617817
613 10.4727083933279
614 10.4860209872325
615 10.4992027307487
616 10.496551891891
617 10.5084984075271
618 10.4999983791339
619 10.5020928752574
620 10.505192215069
621 10.4969611743284
622 10.5029401769162
623 10.4979452230815
624 10.5123314262404
625 10.5056175449138
626 10.5115133666982
627 10.5134958906439
628 10.5148148385258
629 10.5328281462432
630 10.5244990077951
631 10.5295220051882
632 10.5217307424994
633 10.5164088624226
634 10.5137735692039
635 10.5262910089739
636 10.5188480647733
637 10.518558998133
638 10.5335447500043
639 10.5257101462393
640 10.5214884284497
641 10.5270969503042
642 10.5198751429087
643 10.534637968074
644 10.5467487998206
645 10.5538255708837
646 10.5479554704311
647 10.5480171445093
648 10.565077535929
649 10.5793559923461
650 10.5727017030547
651 10.5667603195902
652 10.5809035115167
653 10.5729616738229
654 10.5869819449797
655 10.5884643998198
656 10.5934754493614
657 10.6025784719055
658 10.6040721696181
659 10.6038506935483
660 10.6009095716621
661 10.5930620211606
662 10.5865403841503
663 10.5789792942993
664 10.586073806171
665 10.5782788316582
666 10.5725057094369
667 10.5650292121123
668 10.5648882072279
669 10.57612985093
670 10.568653343928
671 10.5607883974814
672 10.562127235838
673 10.5542894730085
674 10.5495011485128
675 10.5468407429459
676 10.5559932778891
677 10.5531986985433
678 10.5458676269516
679 10.5508967678897
680 10.5494550237241
681 10.5431231294172
682 10.5520573447941
683 10.5457152454936
684 10.5524848322182
685 10.5540075057413
686 10.5524860882693
687 10.5538286171109
688 10.5536379437306
689 10.5463769855702
690 10.5550254015153
691 10.5596573620547
692 10.553329892027
693 10.5624050749343
694 10.5669975521895
695 10.5601634415584
696 10.5588882997424
697 10.5534589262396
698 10.5463609293823
699 10.5509533703305
700 10.5638003365749
701 10.5566331842595
702 10.5518208171598
703 10.5506091973434
704 10.5438655026753
705 10.5363926770665
706 10.5468442659489
707 10.5451963828257
708 10.5435365007482
709 10.5547459903787
710 10.5574142776963
711 10.5503381467239
712 10.5505497383892
713 10.5507393871314
714 10.549558099464
715 10.5557835560639
716 10.5590400562396
717 10.5714850005198
718 10.5662519797435
719 10.5610250105813
720 10.5678290304427
721 10.5634010844913
722 10.5596266380566
723 10.5697912061214
724 10.563708833018
725 10.5609321103413
726 10.5538434994056
727 10.554025696325
728 10.5476543085022
729 10.5543986473945
730 10.5644827661175
731 10.5586684254128
732 10.5552551777104
733 10.548168907644
734 10.5428209676328
735 10.5444208012037
736 10.5374306809964
737 10.5340434465229
738 10.5289477515766
739 10.5274563308991
740 10.5276050217759
741 10.5239777326589
742 10.5213551505661
743 10.5319788038536
744 10.5441237116332
745 10.5483718308376
746 10.5424900515009
747 10.5421920485918
748 10.5352516070288
749 10.5437955016205
750 10.5435040348706
751 10.5393008942194
752 10.5329879873204
753 10.5288020441509
754 10.542942429796
755 10.5377245379119
756 10.5319114158961
757 10.5261073961196
758 10.5257863396962
759 10.5195200925816
760 10.5129187773347
761 10.5107726925793
762 10.5045436424111
763 10.5062277097693
764 10.4995320356076
765 10.5053640672345
766 10.5170118730759
767 10.5179896488777
768 10.5196987837105
769 10.5137252676178
770 10.5111010396924
771 10.5043175151959
772 10.5118229996728
773 10.5078433325126
774 10.510244755016
775 10.5048642615203
776 10.5022389421171
777 10.4961110444001
778 10.4934867506815
779 10.4983852495536
780 10.491847356245
781 10.4942292514533
782 10.4922275415236
783 10.486917989345
784 10.4872867840048
785 10.4829639301082
786 10.496420692818
787 10.4973397938957
788 10.496489473141
789 10.4918638461673
790 10.4872427670781
791 10.4807007014319
792 10.4872328956825
793 10.4806537238519
794 10.4756969129494
795 10.4795997158905
796 10.4919529877923
797 10.4859935964922
798 10.4883487775135
799 10.4848861916126
800 10.47894309735
801 10.4880141780481
802 10.4814754230493
803 10.4755475915106
804 10.4726613369917
805 10.4681575107413
806 10.479155046629
807 10.4772638421001
808 10.474384193962
809 10.4689452639156
810 10.4727245334043
811 10.4736059421276
812 10.4784065723362
813 10.4816039256514
814 10.4833281899232
815 10.4788584881883
816 10.4724719127933
817 10.4825055409096
818 10.4896113486695
819 10.4834908859754
820 10.4784288066389
821 10.4913065065152
822 10.4935382436655
823 10.4877466913916
824 10.4814609927644
825 10.4786435309125
826 10.4738508612946
827 10.4755744426401
828 10.4727569885584
829 10.4667178855281
830 10.4704286045655
831 10.4784823680898
832 10.4756669142344
833 10.4709330132546
834 10.464837655136
835 10.4607735372142
836 10.4604597703749
837 10.4542111602298
838 10.4492919647659
839 10.4539206346934
840 10.4665512337107
841 10.4625011434972
842 10.4563656508745
843 10.4530844382322
844 10.4478758307778
845 10.4471630059758
846 10.4464389014224
847 10.4415767304777
848 10.4375846692281
849 10.4320092157744
850 10.4324603885079
851 10.4264097901878
852 10.421587213157
853 10.4320968939609
854 10.4269681048896
855 10.4218465052922
856 10.4167320797434
857 10.4153557670466
858 10.4130038224075
859 10.4165923341808
860 10.4183888967486
861 10.4149541300484
862 10.4103892903974
863 10.4226574271124
864 10.4348405632363
865 10.4380083561784
866 10.4411466657593
867 10.4446431581671
868 10.4432084149625
869 10.4385148408676
870 10.4361369996893
871 10.4423327768123
872 10.4519634569803
873 10.4487687928813
874 10.4461746924116
875 10.4576222216435
876 10.455910127179
877 10.4603600583936
878 10.4544033857897
879 10.4505377054889
880 10.4526589206956
881 10.4586555470106
882 10.4535133250696
883 10.4508945631291
884 10.445935035319
885 10.4480654765117
886 10.4436158096793
887 10.4377639273816
888 10.4346601349604
889 10.4329922166468
890 10.4448950495874
891 10.4483219654978
892 10.4527565670298
893 10.4626493056723
894 10.45771072446
895 10.4597236481022
896 10.4583186969282
897 10.4590597445004
898 10.459784473624
899 10.4547800553341
900 10.4491641818265
901 10.4434077710499
902 10.4376804101431
903 10.4453099969046
904 10.4472875435578
905 10.4458852095364
906 10.4518584084629
907 10.4495543870409
908 10.4481520964889
909 10.4577466788728
910 10.4520585106596
911 10.4505115942226
912 10.4461295587852
913 10.4522951338965
914 10.4465758161114
915 10.4496291475187
916 10.4573881365616
917 10.4529630064955
918 10.4548301288541
919 10.4524360995918
920 10.4501191643064
921 10.4470167731941
922 10.4431978383899
923 10.4508657380128
924 10.4452628790881
925 10.4404508605198
926 10.4390120703298
927 10.4408070175444
928 10.4403175754962
929 10.4349090789053
930 10.4441218342168
931 10.441061848011
932 10.4440840954745
933 10.4551835567088
934 10.4528861315085
935 10.4524308760454
936 10.4530101801308
937 10.4515853286648
938 10.4460126468577
939 10.4534589784944
940 10.4511775324746
941 10.4507268036259
942 10.4566129405659
943 10.4515222789286
944 10.4461889414906
945 10.4438963526764
946 10.4388262361548
947 10.4383832755575
948 10.4379198876217
949 10.4328682703087
950 10.4387431450704
951 10.4340575403987
952 10.428778749065
953 10.4346210743653
954 10.4454223427436
955 10.4431645521783
956 10.4461571892615
957 10.4551117704614
958 10.4501120769762
959 10.4543920916859
960 10.4549222524874
961 10.4591503695723
962 10.4538980657166
963 10.4569462360806
964 10.4575803722446
965 10.4522146765847
966 10.4476128543256
967 10.4548011151899
968 10.4501672613853
969 10.4545263387054
970 10.4574016582526
971 10.4524761727108
972 10.4580761538265
973 10.4559001146023
974 10.451789402569
975 10.4482280331897
976 10.4538180767806
977 10.4492233601339
978 10.4455983856104
979 10.4542470989302
980 10.4529460743205
981 10.4632454378249
982 10.4581318646488
983 10.4540762380769
984 10.4491723637227
985 10.4450327952103
986 10.4399091131333
987 10.4471931470221
988 10.442644872571
989 10.4454207183405
990 10.4559768697518
991 10.4546834468892
992 10.4525578723764
993 10.4473487126675
994 10.4501340289894
995 10.4452901956108
996 10.4537779912689
997 10.4525069776869
998 10.4484218136285
999 10.4469834779286
1000 10.4474321725484
};
\addlegendentry{$v_{d}$ (valor del desvío de las tiradas)}
\addplot [semithick, color1]
table {%
1 10.6770782520313
2 10.6770782520313
3 10.6770782520313
4 10.6770782520313
5 10.6770782520313
6 10.6770782520313
7 10.6770782520313
8 10.6770782520313
9 10.6770782520313
10 10.6770782520313
11 10.6770782520313
12 10.6770782520313
13 10.6770782520313
14 10.6770782520313
15 10.6770782520313
16 10.6770782520313
17 10.6770782520313
18 10.6770782520313
19 10.6770782520313
20 10.6770782520313
21 10.6770782520313
22 10.6770782520313
23 10.6770782520313
24 10.6770782520313
25 10.6770782520313
26 10.6770782520313
27 10.6770782520313
28 10.6770782520313
29 10.6770782520313
30 10.6770782520313
31 10.6770782520313
32 10.6770782520313
33 10.6770782520313
34 10.6770782520313
35 10.6770782520313
36 10.6770782520313
37 10.6770782520313
38 10.6770782520313
39 10.6770782520313
40 10.6770782520313
41 10.6770782520313
42 10.6770782520313
43 10.6770782520313
44 10.6770782520313
45 10.6770782520313
46 10.6770782520313
47 10.6770782520313
48 10.6770782520313
49 10.6770782520313
50 10.6770782520313
51 10.6770782520313
52 10.6770782520313
53 10.6770782520313
54 10.6770782520313
55 10.6770782520313
56 10.6770782520313
57 10.6770782520313
58 10.6770782520313
59 10.6770782520313
60 10.6770782520313
61 10.6770782520313
62 10.6770782520313
63 10.6770782520313
64 10.6770782520313
65 10.6770782520313
66 10.6770782520313
67 10.6770782520313
68 10.6770782520313
69 10.6770782520313
70 10.6770782520313
71 10.6770782520313
72 10.6770782520313
73 10.6770782520313
74 10.6770782520313
75 10.6770782520313
76 10.6770782520313
77 10.6770782520313
78 10.6770782520313
79 10.6770782520313
80 10.6770782520313
81 10.6770782520313
82 10.6770782520313
83 10.6770782520313
84 10.6770782520313
85 10.6770782520313
86 10.6770782520313
87 10.6770782520313
88 10.6770782520313
89 10.6770782520313
90 10.6770782520313
91 10.6770782520313
92 10.6770782520313
93 10.6770782520313
94 10.6770782520313
95 10.6770782520313
96 10.6770782520313
97 10.6770782520313
98 10.6770782520313
99 10.6770782520313
100 10.6770782520313
101 10.6770782520313
102 10.6770782520313
103 10.6770782520313
104 10.6770782520313
105 10.6770782520313
106 10.6770782520313
107 10.6770782520313
108 10.6770782520313
109 10.6770782520313
110 10.6770782520313
111 10.6770782520313
112 10.6770782520313
113 10.6770782520313
114 10.6770782520313
115 10.6770782520313
116 10.6770782520313
117 10.6770782520313
118 10.6770782520313
119 10.6770782520313
120 10.6770782520313
121 10.6770782520313
122 10.6770782520313
123 10.6770782520313
124 10.6770782520313
125 10.6770782520313
126 10.6770782520313
127 10.6770782520313
128 10.6770782520313
129 10.6770782520313
130 10.6770782520313
131 10.6770782520313
132 10.6770782520313
133 10.6770782520313
134 10.6770782520313
135 10.6770782520313
136 10.6770782520313
137 10.6770782520313
138 10.6770782520313
139 10.6770782520313
140 10.6770782520313
141 10.6770782520313
142 10.6770782520313
143 10.6770782520313
144 10.6770782520313
145 10.6770782520313
146 10.6770782520313
147 10.6770782520313
148 10.6770782520313
149 10.6770782520313
150 10.6770782520313
151 10.6770782520313
152 10.6770782520313
153 10.6770782520313
154 10.6770782520313
155 10.6770782520313
156 10.6770782520313
157 10.6770782520313
158 10.6770782520313
159 10.6770782520313
160 10.6770782520313
161 10.6770782520313
162 10.6770782520313
163 10.6770782520313
164 10.6770782520313
165 10.6770782520313
166 10.6770782520313
167 10.6770782520313
168 10.6770782520313
169 10.6770782520313
170 10.6770782520313
171 10.6770782520313
172 10.6770782520313
173 10.6770782520313
174 10.6770782520313
175 10.6770782520313
176 10.6770782520313
177 10.6770782520313
178 10.6770782520313
179 10.6770782520313
180 10.6770782520313
181 10.6770782520313
182 10.6770782520313
183 10.6770782520313
184 10.6770782520313
185 10.6770782520313
186 10.6770782520313
187 10.6770782520313
188 10.6770782520313
189 10.6770782520313
190 10.6770782520313
191 10.6770782520313
192 10.6770782520313
193 10.6770782520313
194 10.6770782520313
195 10.6770782520313
196 10.6770782520313
197 10.6770782520313
198 10.6770782520313
199 10.6770782520313
200 10.6770782520313
201 10.6770782520313
202 10.6770782520313
203 10.6770782520313
204 10.6770782520313
205 10.6770782520313
206 10.6770782520313
207 10.6770782520313
208 10.6770782520313
209 10.6770782520313
210 10.6770782520313
211 10.6770782520313
212 10.6770782520313
213 10.6770782520313
214 10.6770782520313
215 10.6770782520313
216 10.6770782520313
217 10.6770782520313
218 10.6770782520313
219 10.6770782520313
220 10.6770782520313
221 10.6770782520313
222 10.6770782520313
223 10.6770782520313
224 10.6770782520313
225 10.6770782520313
226 10.6770782520313
227 10.6770782520313
228 10.6770782520313
229 10.6770782520313
230 10.6770782520313
231 10.6770782520313
232 10.6770782520313
233 10.6770782520313
234 10.6770782520313
235 10.6770782520313
236 10.6770782520313
237 10.6770782520313
238 10.6770782520313
239 10.6770782520313
240 10.6770782520313
241 10.6770782520313
242 10.6770782520313
243 10.6770782520313
244 10.6770782520313
245 10.6770782520313
246 10.6770782520313
247 10.6770782520313
248 10.6770782520313
249 10.6770782520313
250 10.6770782520313
251 10.6770782520313
252 10.6770782520313
253 10.6770782520313
254 10.6770782520313
255 10.6770782520313
256 10.6770782520313
257 10.6770782520313
258 10.6770782520313
259 10.6770782520313
260 10.6770782520313
261 10.6770782520313
262 10.6770782520313
263 10.6770782520313
264 10.6770782520313
265 10.6770782520313
266 10.6770782520313
267 10.6770782520313
268 10.6770782520313
269 10.6770782520313
270 10.6770782520313
271 10.6770782520313
272 10.6770782520313
273 10.6770782520313
274 10.6770782520313
275 10.6770782520313
276 10.6770782520313
277 10.6770782520313
278 10.6770782520313
279 10.6770782520313
280 10.6770782520313
281 10.6770782520313
282 10.6770782520313
283 10.6770782520313
284 10.6770782520313
285 10.6770782520313
286 10.6770782520313
287 10.6770782520313
288 10.6770782520313
289 10.6770782520313
290 10.6770782520313
291 10.6770782520313
292 10.6770782520313
293 10.6770782520313
294 10.6770782520313
295 10.6770782520313
296 10.6770782520313
297 10.6770782520313
298 10.6770782520313
299 10.6770782520313
300 10.6770782520313
301 10.6770782520313
302 10.6770782520313
303 10.6770782520313
304 10.6770782520313
305 10.6770782520313
306 10.6770782520313
307 10.6770782520313
308 10.6770782520313
309 10.6770782520313
310 10.6770782520313
311 10.6770782520313
312 10.6770782520313
313 10.6770782520313
314 10.6770782520313
315 10.6770782520313
316 10.6770782520313
317 10.6770782520313
318 10.6770782520313
319 10.6770782520313
320 10.6770782520313
321 10.6770782520313
322 10.6770782520313
323 10.6770782520313
324 10.6770782520313
325 10.6770782520313
326 10.6770782520313
327 10.6770782520313
328 10.6770782520313
329 10.6770782520313
330 10.6770782520313
331 10.6770782520313
332 10.6770782520313
333 10.6770782520313
334 10.6770782520313
335 10.6770782520313
336 10.6770782520313
337 10.6770782520313
338 10.6770782520313
339 10.6770782520313
340 10.6770782520313
341 10.6770782520313
342 10.6770782520313
343 10.6770782520313
344 10.6770782520313
345 10.6770782520313
346 10.6770782520313
347 10.6770782520313
348 10.6770782520313
349 10.6770782520313
350 10.6770782520313
351 10.6770782520313
352 10.6770782520313
353 10.6770782520313
354 10.6770782520313
355 10.6770782520313
356 10.6770782520313
357 10.6770782520313
358 10.6770782520313
359 10.6770782520313
360 10.6770782520313
361 10.6770782520313
362 10.6770782520313
363 10.6770782520313
364 10.6770782520313
365 10.6770782520313
366 10.6770782520313
367 10.6770782520313
368 10.6770782520313
369 10.6770782520313
370 10.6770782520313
371 10.6770782520313
372 10.6770782520313
373 10.6770782520313
374 10.6770782520313
375 10.6770782520313
376 10.6770782520313
377 10.6770782520313
378 10.6770782520313
379 10.6770782520313
380 10.6770782520313
381 10.6770782520313
382 10.6770782520313
383 10.6770782520313
384 10.6770782520313
385 10.6770782520313
386 10.6770782520313
387 10.6770782520313
388 10.6770782520313
389 10.6770782520313
390 10.6770782520313
391 10.6770782520313
392 10.6770782520313
393 10.6770782520313
394 10.6770782520313
395 10.6770782520313
396 10.6770782520313
397 10.6770782520313
398 10.6770782520313
399 10.6770782520313
400 10.6770782520313
401 10.6770782520313
402 10.6770782520313
403 10.6770782520313
404 10.6770782520313
405 10.6770782520313
406 10.6770782520313
407 10.6770782520313
408 10.6770782520313
409 10.6770782520313
410 10.6770782520313
411 10.6770782520313
412 10.6770782520313
413 10.6770782520313
414 10.6770782520313
415 10.6770782520313
416 10.6770782520313
417 10.6770782520313
418 10.6770782520313
419 10.6770782520313
420 10.6770782520313
421 10.6770782520313
422 10.6770782520313
423 10.6770782520313
424 10.6770782520313
425 10.6770782520313
426 10.6770782520313
427 10.6770782520313
428 10.6770782520313
429 10.6770782520313
430 10.6770782520313
431 10.6770782520313
432 10.6770782520313
433 10.6770782520313
434 10.6770782520313
435 10.6770782520313
436 10.6770782520313
437 10.6770782520313
438 10.6770782520313
439 10.6770782520313
440 10.6770782520313
441 10.6770782520313
442 10.6770782520313
443 10.6770782520313
444 10.6770782520313
445 10.6770782520313
446 10.6770782520313
447 10.6770782520313
448 10.6770782520313
449 10.6770782520313
450 10.6770782520313
451 10.6770782520313
452 10.6770782520313
453 10.6770782520313
454 10.6770782520313
455 10.6770782520313
456 10.6770782520313
457 10.6770782520313
458 10.6770782520313
459 10.6770782520313
460 10.6770782520313
461 10.6770782520313
462 10.6770782520313
463 10.6770782520313
464 10.6770782520313
465 10.6770782520313
466 10.6770782520313
467 10.6770782520313
468 10.6770782520313
469 10.6770782520313
470 10.6770782520313
471 10.6770782520313
472 10.6770782520313
473 10.6770782520313
474 10.6770782520313
475 10.6770782520313
476 10.6770782520313
477 10.6770782520313
478 10.6770782520313
479 10.6770782520313
480 10.6770782520313
481 10.6770782520313
482 10.6770782520313
483 10.6770782520313
484 10.6770782520313
485 10.6770782520313
486 10.6770782520313
487 10.6770782520313
488 10.6770782520313
489 10.6770782520313
490 10.6770782520313
491 10.6770782520313
492 10.6770782520313
493 10.6770782520313
494 10.6770782520313
495 10.6770782520313
496 10.6770782520313
497 10.6770782520313
498 10.6770782520313
499 10.6770782520313
500 10.6770782520313
501 10.6770782520313
502 10.6770782520313
503 10.6770782520313
504 10.6770782520313
505 10.6770782520313
506 10.6770782520313
507 10.6770782520313
508 10.6770782520313
509 10.6770782520313
510 10.6770782520313
511 10.6770782520313
512 10.6770782520313
513 10.6770782520313
514 10.6770782520313
515 10.6770782520313
516 10.6770782520313
517 10.6770782520313
518 10.6770782520313
519 10.6770782520313
520 10.6770782520313
521 10.6770782520313
522 10.6770782520313
523 10.6770782520313
524 10.6770782520313
525 10.6770782520313
526 10.6770782520313
527 10.6770782520313
528 10.6770782520313
529 10.6770782520313
530 10.6770782520313
531 10.6770782520313
532 10.6770782520313
533 10.6770782520313
534 10.6770782520313
535 10.6770782520313
536 10.6770782520313
537 10.6770782520313
538 10.6770782520313
539 10.6770782520313
540 10.6770782520313
541 10.6770782520313
542 10.6770782520313
543 10.6770782520313
544 10.6770782520313
545 10.6770782520313
546 10.6770782520313
547 10.6770782520313
548 10.6770782520313
549 10.6770782520313
550 10.6770782520313
551 10.6770782520313
552 10.6770782520313
553 10.6770782520313
554 10.6770782520313
555 10.6770782520313
556 10.6770782520313
557 10.6770782520313
558 10.6770782520313
559 10.6770782520313
560 10.6770782520313
561 10.6770782520313
562 10.6770782520313
563 10.6770782520313
564 10.6770782520313
565 10.6770782520313
566 10.6770782520313
567 10.6770782520313
568 10.6770782520313
569 10.6770782520313
570 10.6770782520313
571 10.6770782520313
572 10.6770782520313
573 10.6770782520313
574 10.6770782520313
575 10.6770782520313
576 10.6770782520313
577 10.6770782520313
578 10.6770782520313
579 10.6770782520313
580 10.6770782520313
581 10.6770782520313
582 10.6770782520313
583 10.6770782520313
584 10.6770782520313
585 10.6770782520313
586 10.6770782520313
587 10.6770782520313
588 10.6770782520313
589 10.6770782520313
590 10.6770782520313
591 10.6770782520313
592 10.6770782520313
593 10.6770782520313
594 10.6770782520313
595 10.6770782520313
596 10.6770782520313
597 10.6770782520313
598 10.6770782520313
599 10.6770782520313
600 10.6770782520313
601 10.6770782520313
602 10.6770782520313
603 10.6770782520313
604 10.6770782520313
605 10.6770782520313
606 10.6770782520313
607 10.6770782520313
608 10.6770782520313
609 10.6770782520313
610 10.6770782520313
611 10.6770782520313
612 10.6770782520313
613 10.6770782520313
614 10.6770782520313
615 10.6770782520313
616 10.6770782520313
617 10.6770782520313
618 10.6770782520313
619 10.6770782520313
620 10.6770782520313
621 10.6770782520313
622 10.6770782520313
623 10.6770782520313
624 10.6770782520313
625 10.6770782520313
626 10.6770782520313
627 10.6770782520313
628 10.6770782520313
629 10.6770782520313
630 10.6770782520313
631 10.6770782520313
632 10.6770782520313
633 10.6770782520313
634 10.6770782520313
635 10.6770782520313
636 10.6770782520313
637 10.6770782520313
638 10.6770782520313
639 10.6770782520313
640 10.6770782520313
641 10.6770782520313
642 10.6770782520313
643 10.6770782520313
644 10.6770782520313
645 10.6770782520313
646 10.6770782520313
647 10.6770782520313
648 10.6770782520313
649 10.6770782520313
650 10.6770782520313
651 10.6770782520313
652 10.6770782520313
653 10.6770782520313
654 10.6770782520313
655 10.6770782520313
656 10.6770782520313
657 10.6770782520313
658 10.6770782520313
659 10.6770782520313
660 10.6770782520313
661 10.6770782520313
662 10.6770782520313
663 10.6770782520313
664 10.6770782520313
665 10.6770782520313
666 10.6770782520313
667 10.6770782520313
668 10.6770782520313
669 10.6770782520313
670 10.6770782520313
671 10.6770782520313
672 10.6770782520313
673 10.6770782520313
674 10.6770782520313
675 10.6770782520313
676 10.6770782520313
677 10.6770782520313
678 10.6770782520313
679 10.6770782520313
680 10.6770782520313
681 10.6770782520313
682 10.6770782520313
683 10.6770782520313
684 10.6770782520313
685 10.6770782520313
686 10.6770782520313
687 10.6770782520313
688 10.6770782520313
689 10.6770782520313
690 10.6770782520313
691 10.6770782520313
692 10.6770782520313
693 10.6770782520313
694 10.6770782520313
695 10.6770782520313
696 10.6770782520313
697 10.6770782520313
698 10.6770782520313
699 10.6770782520313
700 10.6770782520313
701 10.6770782520313
702 10.6770782520313
703 10.6770782520313
704 10.6770782520313
705 10.6770782520313
706 10.6770782520313
707 10.6770782520313
708 10.6770782520313
709 10.6770782520313
710 10.6770782520313
711 10.6770782520313
712 10.6770782520313
713 10.6770782520313
714 10.6770782520313
715 10.6770782520313
716 10.6770782520313
717 10.6770782520313
718 10.6770782520313
719 10.6770782520313
720 10.6770782520313
721 10.6770782520313
722 10.6770782520313
723 10.6770782520313
724 10.6770782520313
725 10.6770782520313
726 10.6770782520313
727 10.6770782520313
728 10.6770782520313
729 10.6770782520313
730 10.6770782520313
731 10.6770782520313
732 10.6770782520313
733 10.6770782520313
734 10.6770782520313
735 10.6770782520313
736 10.6770782520313
737 10.6770782520313
738 10.6770782520313
739 10.6770782520313
740 10.6770782520313
741 10.6770782520313
742 10.6770782520313
743 10.6770782520313
744 10.6770782520313
745 10.6770782520313
746 10.6770782520313
747 10.6770782520313
748 10.6770782520313
749 10.6770782520313
750 10.6770782520313
751 10.6770782520313
752 10.6770782520313
753 10.6770782520313
754 10.6770782520313
755 10.6770782520313
756 10.6770782520313
757 10.6770782520313
758 10.6770782520313
759 10.6770782520313
760 10.6770782520313
761 10.6770782520313
762 10.6770782520313
763 10.6770782520313
764 10.6770782520313
765 10.6770782520313
766 10.6770782520313
767 10.6770782520313
768 10.6770782520313
769 10.6770782520313
770 10.6770782520313
771 10.6770782520313
772 10.6770782520313
773 10.6770782520313
774 10.6770782520313
775 10.6770782520313
776 10.6770782520313
777 10.6770782520313
778 10.6770782520313
779 10.6770782520313
780 10.6770782520313
781 10.6770782520313
782 10.6770782520313
783 10.6770782520313
784 10.6770782520313
785 10.6770782520313
786 10.6770782520313
787 10.6770782520313
788 10.6770782520313
789 10.6770782520313
790 10.6770782520313
791 10.6770782520313
792 10.6770782520313
793 10.6770782520313
794 10.6770782520313
795 10.6770782520313
796 10.6770782520313
797 10.6770782520313
798 10.6770782520313
799 10.6770782520313
800 10.6770782520313
801 10.6770782520313
802 10.6770782520313
803 10.6770782520313
804 10.6770782520313
805 10.6770782520313
806 10.6770782520313
807 10.6770782520313
808 10.6770782520313
809 10.6770782520313
810 10.6770782520313
811 10.6770782520313
812 10.6770782520313
813 10.6770782520313
814 10.6770782520313
815 10.6770782520313
816 10.6770782520313
817 10.6770782520313
818 10.6770782520313
819 10.6770782520313
820 10.6770782520313
821 10.6770782520313
822 10.6770782520313
823 10.6770782520313
824 10.6770782520313
825 10.6770782520313
826 10.6770782520313
827 10.6770782520313
828 10.6770782520313
829 10.6770782520313
830 10.6770782520313
831 10.6770782520313
832 10.6770782520313
833 10.6770782520313
834 10.6770782520313
835 10.6770782520313
836 10.6770782520313
837 10.6770782520313
838 10.6770782520313
839 10.6770782520313
840 10.6770782520313
841 10.6770782520313
842 10.6770782520313
843 10.6770782520313
844 10.6770782520313
845 10.6770782520313
846 10.6770782520313
847 10.6770782520313
848 10.6770782520313
849 10.6770782520313
850 10.6770782520313
851 10.6770782520313
852 10.6770782520313
853 10.6770782520313
854 10.6770782520313
855 10.6770782520313
856 10.6770782520313
857 10.6770782520313
858 10.6770782520313
859 10.6770782520313
860 10.6770782520313
861 10.6770782520313
862 10.6770782520313
863 10.6770782520313
864 10.6770782520313
865 10.6770782520313
866 10.6770782520313
867 10.6770782520313
868 10.6770782520313
869 10.6770782520313
870 10.6770782520313
871 10.6770782520313
872 10.6770782520313
873 10.6770782520313
874 10.6770782520313
875 10.6770782520313
876 10.6770782520313
877 10.6770782520313
878 10.6770782520313
879 10.6770782520313
880 10.6770782520313
881 10.6770782520313
882 10.6770782520313
883 10.6770782520313
884 10.6770782520313
885 10.6770782520313
886 10.6770782520313
887 10.6770782520313
888 10.6770782520313
889 10.6770782520313
890 10.6770782520313
891 10.6770782520313
892 10.6770782520313
893 10.6770782520313
894 10.6770782520313
895 10.6770782520313
896 10.6770782520313
897 10.6770782520313
898 10.6770782520313
899 10.6770782520313
900 10.6770782520313
901 10.6770782520313
902 10.6770782520313
903 10.6770782520313
904 10.6770782520313
905 10.6770782520313
906 10.6770782520313
907 10.6770782520313
908 10.6770782520313
909 10.6770782520313
910 10.6770782520313
911 10.6770782520313
912 10.6770782520313
913 10.6770782520313
914 10.6770782520313
915 10.6770782520313
916 10.6770782520313
917 10.6770782520313
918 10.6770782520313
919 10.6770782520313
920 10.6770782520313
921 10.6770782520313
922 10.6770782520313
923 10.6770782520313
924 10.6770782520313
925 10.6770782520313
926 10.6770782520313
927 10.6770782520313
928 10.6770782520313
929 10.6770782520313
930 10.6770782520313
931 10.6770782520313
932 10.6770782520313
933 10.6770782520313
934 10.6770782520313
935 10.6770782520313
936 10.6770782520313
937 10.6770782520313
938 10.6770782520313
939 10.6770782520313
940 10.6770782520313
941 10.6770782520313
942 10.6770782520313
943 10.6770782520313
944 10.6770782520313
945 10.6770782520313
946 10.6770782520313
947 10.6770782520313
948 10.6770782520313
949 10.6770782520313
950 10.6770782520313
951 10.6770782520313
952 10.6770782520313
953 10.6770782520313
954 10.6770782520313
955 10.6770782520313
956 10.6770782520313
957 10.6770782520313
958 10.6770782520313
959 10.6770782520313
960 10.6770782520313
961 10.6770782520313
962 10.6770782520313
963 10.6770782520313
964 10.6770782520313
965 10.6770782520313
966 10.6770782520313
967 10.6770782520313
968 10.6770782520313
969 10.6770782520313
970 10.6770782520313
971 10.6770782520313
972 10.6770782520313
973 10.6770782520313
974 10.6770782520313
975 10.6770782520313
976 10.6770782520313
977 10.6770782520313
978 10.6770782520313
979 10.6770782520313
980 10.6770782520313
981 10.6770782520313
982 10.6770782520313
983 10.6770782520313
984 10.6770782520313
985 10.6770782520313
986 10.6770782520313
987 10.6770782520313
988 10.6770782520313
989 10.6770782520313
990 10.6770782520313
991 10.6770782520313
992 10.6770782520313
993 10.6770782520313
994 10.6770782520313
995 10.6770782520313
996 10.6770782520313
997 10.6770782520313
998 10.6770782520313
999 10.6770782520313
1000 10.6770782520313
};
\addlegendentry{$v_{d_{e}}$ (valor del desvío esperado)}
\end{axis}

\end{tikzpicture}

    \caption{valor del desvío con respecto al número de tiradas}
  \end{mytikzresize}
\end{figure}

\begin{figure}[!htbp]
  \begin{mytikzresize}{0.6\textwidth}
    \centering
    % This file was created by tikzplotlib v0.9.1.
\begin{tikzpicture}

\definecolor{color0}{rgb}{0.12156862745098,0.466666666666667,0.705882352941177}
\definecolor{color1}{rgb}{1,0.498039215686275,0.0549019607843137}

\begin{axis}[
legend cell align={left},
legend style={fill opacity=0.5, draw opacity=1, text opacity=1, draw=white!80!black},
scaled ticks=false,
tick align=outside,
tick pos=left,
width=\figW,
x grid style={white!69.0196078431373!black},
xlabel={\(\displaystyle n\) (número de tiradas)},
xmajorgrids,
xmin=-48.95, xmax=1049.95,
xtick style={color=black},
xticklabel style={/pgf/number format/.cd,fixed,precision=2},
y grid style={white!69.0196078431373!black},
ylabel={\(\displaystyle v_{d}\) (valor de la varianza)},
ymajorgrids,
ymin=-6.76, ymax=141.96,
ytick style={color=black},
yticklabel style={/pgf/number format/.cd,fixed,precision=2}
]
\addplot [semithick, color0]
table {%
1 0
2 121
3 82.6666666666667
4 89
5 135.2
6 129.472222222222
7 111.387755102041
8 105.234375
9 98.9135802469136
10 98.84
11 90.0165289256198
12 82.6388888888889
13 91.905325443787
14 86.2448979591837
15 84.9155555555556
16 80.99609375
17 87.7716262975779
18 86.7654320987654
19 98.1551246537396
20 107.06
21 102.181405895692
22 102.696280991736
23 109.678638941399
24 105.289930555556
25 113.5776
26 114.822485207101
27 110.617283950617
28 107.248724489796
29 110.042806183115
30 112.262222222222
31 109.442247658689
32 114.671875
33 111.362718089991
34 108.717128027682
35 108.14693877551
36 106.024691358025
37 103.316289262235
38 105.351800554017
39 102.69033530572
40 108.71
41 106.098750743605
42 108.990929705215
43 106.555976203353
44 106.844524793388
45 105.2
46 107.487712665406
47 107.882299683115
48 107.631510416667
49 113.183673469388
50 111.4676
51 110.901960784314
52 109.448224852071
53 107.466002135991
54 108.126200274348
55 106.173223140496
56 109.44387755102
57 108.96460449369
58 108.165576694411
59 106.467107153117
60 105.960833333333
61 104.230583176565
62 104.345733610822
63 103.826656588561
64 107.52734375
65 106.154792899408
66 104.659550045914
67 103.508576520383
68 104.911764705882
69 103.405587061542
70 103.961020408163
71 103.14699464392
72 101.735918209877
73 100.362919872396
74 99.4521548575603
75 98.8266666666667
76 97.8254847645429
77 99.3388429752066
78 98.3164036817883
79 99.3151738503445
80 100.93609375
81 100.477671086725
82 101.268887566924
83 100.769342429961
84 99.5899943310658
85 98.6754325259516
86 99.6684694429421
87 98.5932091425551
88 97.9533832644628
89 99.8899128897866
90 99.7402469135803
91 98.6504045405144
92 100.244210775047
93 102.452306625043
94 102.497962879131
95 103.254958448753
96 103.238606770833
97 104.165161016048
98 105.032590587255
99 105.744719926538
100 104.7875
101 105.197137535536
102 104.41138023837
103 105.017626543501
104 106.109097633136
105 105.362358276644
106 104.580277678889
107 105.236439863744
108 106.657321673525
109 106.007070111943
110 105.264462809917
111 104.465546627709
112 104.106823979592
113 104.444827316156
114 105.789165897199
115 107.071153119093
116 106.187871581451
117 107.983928701877
118 108.553073829359
119 108.805451592402
120 109.355555555556
121 108.998429069053
122 108.934157484547
123 109.414766342785
124 109.516129032258
125 108.925696
126 109.382023179642
127 108.788021576043
128 108.235778808594
129 107.757827053663
130 106.988698224852
131 106.176563137346
132 105.66391184573
133 107.238622872972
134 106.441802183114
135 105.734430727023
136 107.244593425606
137 108.452128509777
138 109.881327452216
139 111.258319962735
140 111.368367346939
141 110.769981389266
142 112.053164054751
143 112.619688004303
144 112.296103395062
145 111.776266349584
146 111.842043535372
147 111.088250266093
148 110.667777574872
149 110.167740191883
150 111.577288888889
151 112.495943160388
152 112.513677285319
153 112.607800418642
154 111.968333614438
155 111.457648283039
156 111.121753780408
157 110.594425737352
158 110.266784169204
159 109.607926901626
160 109.6287109375
161 109.897843447398
162 110.055060204237
163 109.508826075502
164 110.476948245092
165 111.28668503214
166 111.056648279866
167 111.826096310373
168 112.594671201814
169 111.999509821085
170 112.108096885813
171 111.564447180329
172 112.713899405084
173 112.382638912092
174 113.471132249967
175 112.947069387755
176 112.427556818182
177 112.020683711577
178 112.109613685141
179 113.037233544521
180 112.894166666667
181 113.733219376698
182 113.528921627823
183 112.917375854758
184 112.926275992439
185 112.92838568298
186 112.612787605504
187 112.297463467643
188 112.782791987325
189 112.240922706531
190 111.825595567867
191 112.225925824402
192 112.703016493056
193 112.469005879352
194 112.422467849931
195 111.846048652202
196 112.290816326531
197 111.906980339612
198 111.74515355576
199 112.039796974824
200 113.0951
201 113.656691666048
202 113.271664542692
203 113.206289888131
204 113.752499038831
205 114.778964901844
206 115.03329720049
207 115.692688277439
208 115.69674556213
209 115.192875620979
210 114.659433106576
211 115.357561600144
212 115.993302776789
213 115.471797923692
214 114.954952397589
215 115.06479177934
216 115.22425840192
217 116.050160334685
218 115.897588586819
219 115.653635245303
220 115.131487603306
221 114.626481849266
222 114.321990909829
223 113.815319029138
224 113.602598852041
225 113.321758024691
226 113.031012608662
227 112.552698480467
228 112.059095106187
229 111.585362597967
230 111.962495274102
231 112.866250632484
232 112.820154577883
233 112.755954244875
234 112.2741617357
235 112.733544590312
236 112.451450732548
237 111.979526073101
238 111.727349763435
239 112.054130704995
240 112.023263888889
241 112.643997176357
242 112.968922887781
243 112.517333062372
244 112.664270357431
245 112.710304039983
246 112.320526802829
247 112.437410873806
248 112.374333376691
249 112.970371445622
250 112.5216
251 112.210504595165
252 111.858449861426
253 111.429611460888
254 111.192510385021
255 111.136086120723
256 111.445068359375
257 111.140274644582
258 111.198741061234
259 111.398249877014
260 111.120059171598
261 110.95336239926
262 110.925994988637
263 111.264554930677
264 111.679278581267
265 111.260434318263
266 110.984905873707
267 110.624528328354
268 110.523989195812
269 110.113182515443
270 110.634691358025
271 110.829836195041
272 110.771572231834
273 110.915455728643
274 111.246683360861
275 111.438069421488
276 111.764479626129
277 111.444903491509
278 111.632381864293
279 111.7667810023
280 111.60631377551
281 112.098782943478
282 111.718638398471
283 111.341058072894
284 111.47137224757
285 112.233130193906
286 111.893845664825
287 112.174021780039
288 111.786988811728
289 112.159600579495
290 111.888002378121
291 111.508697346512
292 111.131954869582
293 110.982632296241
294 111.251203202369
295 111.341177822465
296 111.796920653762
297 112.244011382059
298 112.115445250214
299 112.174897372513
300 111.905988888889
301 111.540667321553
302 111.695945353274
303 111.920944569705
304 111.575127683518
305 111.209825315775
306 110.846907172455
307 110.663412874407
308 110.492452352842
309 110.197212010767
310 110.016951092612
311 109.755213448992
312 109.412228796844
313 109.249231899887
314 109.407653454501
315 109.415167548501
316 109.162423890402
317 108.836589079402
318 109.44555199557
319 110.156405695699
320 109.8546484375
321 110.063314602925
322 110.755728945643
323 110.422605411726
324 110.318663313519
325 110.367659171598
326 110.89032330912
327 110.813193801494
328 110.558856335515
329 110.574588187471
330 110.470376492195
331 110.880495796862
332 110.808054507185
333 110.57242828414
334 110.44807092402
335 110.343827132992
336 110.015731292517
337 110.682386918966
338 110.396029550786
339 110.513761627553
340 111.164567474048
341 110.989654371737
342 110.688519544475
343 110.921384797151
344 110.60322302596
345 110.618979206049
346 111.207892011093
347 111.849180709083
348 111.532195468358
349 111.234406942472
350 111.865110204082
351 111.968847655457
352 111.660349948347
353 111.385582100811
354 111.488525008778
355 111.972830787542
356 111.788252745865
357 112.121146497815
358 112.266814394058
359 112.64743445504
360 112.654930555556
361 112.489069298118
362 112.303661365648
363 111.998740219627
364 111.81514158918
365 112.365892287484
366 112.429521633969
367 112.325609366763
368 112.211247637051
369 112.338305388474
370 112.052856099343
371 111.861305860899
372 111.707155451497
373 111.412717693651
374 111.145107094855
375 111.209002666667
376 111.06676522182
377 110.823660196019
378 110.581569664903
379 111.015921637972
380 111.28074099723
381 111.100254200508
382 110.955819467668
383 111.211270102053
384 111.723409016927
385 111.485134086693
386 111.742221267685
387 111.505611975776
388 112.004218035923
389 111.722523641795
390 111.436561472715
391 111.168228883903
392 110.916701374427
393 110.743054341563
394 110.992482413873
395 110.786194520109
396 110.580572135496
397 110.318865039433
398 110.263989545719
399 110.580109421423
400 110.52194375
401 110.757321160938
402 110.652613796688
403 110.591026359377
404 111.052268159984
405 111.347904282884
406 111.075329418331
407 111.056160918569
408 110.988658208381
409 111.076009827775
410 110.815800118977
411 111.015622687528
412 111.281806720709
413 111.125280678201
414 110.921497584541
415 111.282485121208
416 111.136811205621
417 110.87324672636
418 111.083016643392
419 110.93705321797
420 110.71201814059
421 111.212755513679
422 111.406617102042
423 111.384940395352
424 111.269112673549
425 111.126233910035
426 111.615646146047
427 111.357304652579
428 111.482591274347
429 111.757945240463
430 112.179280692266
431 111.941182487174
432 111.798091349451
433 111.650496829147
434 111.508722843976
435 111.834303078346
436 111.616420124569
437 111.395755332017
438 111.151545005317
439 111.006470493615
440 111.47923553719
441 111.286665535451
442 111.14088573125
443 111.385810883113
444 111.368572964857
445 111.2556495392
446 111.638279474753
447 111.539670385218
448 111.522879464286
449 111.505131422959
450 111.471822222222
451 111.235234831687
452 111.351843331506
453 111.883689311872
454 112.329561994217
455 112.094191522763
456 111.871994267467
457 111.643369132723
458 111.630041379836
459 111.859028578752
460 111.663894139887
461 111.549860954917
462 111.909245516388
463 111.867937994766
464 111.651382282996
465 111.870533009596
466 112.293599992632
467 112.630641618789
468 113.050090401052
469 113.456349080064
470 113.60735174287
471 113.938343227807
472 113.697711684861
473 113.54922249686
474 113.757539746123
475 113.667882548476
476 113.76046889344
477 113.712625643325
478 113.51651756797
479 113.368517396629
480 113.132808159722
481 112.898172120625
482 112.893153526971
483 113.035968262541
484 112.802882316782
485 113.062795196089
486 113.463382106386
487 113.416298082802
488 113.188457403924
489 113.51204620255
490 113.341062057476
491 113.170527747935
492 113.024522440346
493 112.799361445634
494 112.571378812962
495 112.34464238343
496 112.387942247659
497 112.248128610698
498 112.035692972694
499 111.896667081658
500 111.697676
501 111.785084521576
502 111.586165298964
503 111.907766126897
504 112.22373000126
505 112.312753651603
506 112.448940773954
507 112.402888165291
508 112.294206088412
509 112.112327804818
510 111.904113802384
511 111.725889530141
512 111.809352874756
513 111.763893163708
514 111.905721509788
515 111.948119521161
516 111.814235172165
517 111.736105862942
518 111.630435592791
519 111.555681780213
520 111.341582840237
521 111.299788904403
522 111.744161125057
523 111.78246547898
524 111.569558155119
525 111.369237188209
526 111.293715392734
527 111.120142296987
528 111.427596275253
529 111.239796884659
530 111.238622285511
531 111.040810608559
532 111.038452569393
533 111.278176909349
534 111.319614526785
535 111.247584941916
536 111.062541768768
537 110.893861684162
538 110.791503710562
539 110.597202956069
540 110.392784636488
541 110.574468448584
542 110.424806987922
543 110.505119569678
544 110.402451881488
545 110.480855146873
546 110.604975915415
547 110.437125888593
548 110.430000932389
549 110.655140493894
550 110.513242975207
551 110.317943616787
552 110.391566897711
553 110.583305265705
554 110.417042448097
555 110.313786218651
556 110.148284250298
557 110.021060503015
558 109.960473914775
559 110.009139755697
560 110.252139668367
561 110.088923204997
562 109.896125935588
563 109.733967675072
564 109.56351541673
565 109.86141436291
566 109.936099214624
567 109.780714114635
568 109.65789277921
569 109.617575927922
570 109.973681132656
571 109.902539864618
572 109.711208372048
573 109.552278354943
574 109.393788925445
575 109.387959168242
576 109.347580897955
577 109.163233001631
578 109.005941619473
579 108.818599157024
580 108.669497621879
581 108.550170191462
582 108.372471392638
583 108.191750836302
584 108.404860198912
585 108.518682153554
586 108.781526284523
587 109.197971959172
588 109.114986811051
589 108.938196304058
590 109.146075840276
591 109.12345647201
592 108.95214344412
593 108.781364371859
594 108.699670668526
595 109.045960031071
596 109.389315233548
597 109.305135392204
598 109.267256518383
599 109.085654722255
600 109.263288888889
601 109.333888887351
602 109.240794803589
603 109.352232755515
604 109.482687162844
605 109.551161805888
606 109.375290004248
607 109.336657176434
608 109.591063776835
609 109.412070178844
610 109.404203171191
611 109.324056241144
612 109.629851232432
613 109.677621091681
614 109.956636144681
615 110.23325798136
616 110.17760161916
617 110.428538781
618 110.249965961814
619 110.293954760531
620 110.359063475546
621 110.186193895359
622 110.31175235988
623 110.20685390682
624 110.509112015122
625 110.368
626 110.491913258276
627 110.533595842586
628 110.561331088482
629 110.940468758294
630 110.765079365079
631 110.870833657741
632 110.706817817657
633 110.59485536164
634 110.539434664491
635 110.802802405605
636 110.646164609786
637 110.640083397204
638 110.955565000344
639 110.790574082646
640 110.70171875
641 110.819770201104
642 110.667773022389
643 110.978597118387
644 111.233910246518
645 111.383234180638
646 111.259364606198
647 111.260665680862
648 111.620863340192
649 111.922773212789
650 111.782021301775
651 111.656423651667
652 111.955519120027
653 111.787518556128
654 112.084186703327
655 112.11557834625
656 112.221722096222
657 112.414670252914
658 112.446346578468
659 112.441649531064
660 112.379283746556
661 112.212962984155
662 112.074837305245
663 111.914802909213
664 112.0649586297
665 111.899983040308
666 111.777876976075
667 111.619842252787
668 111.616862831224
669 111.854522623732
670 111.696433504121
671 111.530251576378
672 111.558531746032
673 111.393026280057
674 111.291974482473
675 111.235849657064
676 111.42899408284
677 111.370002770935
678 111.215324005186
679 111.321422606666
680 111.291001297578
681 111.157445322052
682 111.345914207824
683 111.212110039036
684 111.354936134195
685 111.387074431243
686 111.354962643116
687 111.383298479349
688 111.37927384735
689 111.226067521765
690 111.408561226633
691 111.506363603997
692 111.37277180995
693 111.564400966998
694 111.661437267978
695 111.517051912427
696 111.490122126437
697 111.375495307827
698 111.225728852801
699 111.322617022888
700 111.59387755102
701 111.442504187008
702 111.340922557447
703 111.315354435067
704 111.173099738507
705 111.01557064534
706 111.235923970179
707 111.201166752359
708 111.166161942609
709 111.402662921415
710 111.458996230907
711 111.309635010217
712 111.314099782224
713 111.318101615165
714 111.293176093967
715 111.424566482469
716 111.493326909272
717 111.756295116215
718 111.645680899434
719 111.535249274123
720 111.679010416667
721 111.585442471833
722 111.505714735154
723 111.720486141002
724 111.591944308782
725 111.533287039239
726 111.383612609946
727 111.387458398688
728 111.253011411665
729 111.395330808124
730 111.608296115594
731 111.48547891781
732 111.413411866583
733 111.263867304188
734 111.151073955557
735 111.184810032857
736 111.037445356805
737 110.966071333231
738 110.858740755429
739 110.827336798988
740 110.830467494522
741 110.7541073175
742 110.698914204343
743 110.922577524821
744 111.178544846225
745 111.268148281609
746 111.144096485995
747 111.137813189393
748 110.991526423404
749 111.171623579994
750 111.165477333333
751 111.076863338895
752 110.943835941037
753 110.855672484916
754 111.153635077993
755 111.043638436911
756 110.921158072282
757 110.798936914644
758 110.792178068936
759 110.660302978227
760 110.521461218837
761 110.476342595071
762 110.345437135319
763 110.380820689525
764 110.24017296675
765 110.362674185142
766 110.607538738419
767 110.628106253899
768 110.6640625
769 110.538419002944
770 110.483245066622
771 110.340686460052
772 110.49842277645
773 110.41477150063
774 110.465244810341
775 110.352173152966
776 110.297022797322
777 110.168347056378
778 110.113264186729
779 110.216092848044
780 110.078860946746
781 110.128847582059
782 110.086838783106
783 109.975448915247
784 109.983184089963
785 109.89253275995
786 110.174847360617
787 110.194142748507
788 110.176291259759
789 110.079206966512
790 109.982260855632
791 109.845087192994
792 109.982053808285
793 109.844102479292
794 109.740225811978
795 109.822010205293
796 110.081077498043
797 109.956061705675
798 110.005460078768
799 109.932838451068
800 109.8082484375
801 109.998441398938
802 109.861327043986
803 109.737097342004
804 109.67663547932
805 109.582321669689
806 109.812690491291
807 109.773057616979
808 109.712724242721
809 109.59881493886
810 109.677959152568
811 109.696421430971
812 109.797004295178
813 109.864020854231
814 109.900169937639
815 109.806475215477
816 109.672667964245
817 109.882922415201
818 110.031946246137
819 109.903581156328
820 109.7974702558
821 110.067512213649
822 110.114344871271
823 109.992830662795
824 109.861024542841
825 109.801970247934
826 109.701551864641
827 109.737659903294
828 109.678638941399
829 109.552183295234
830 109.629875163304
831 109.798592738368
832 109.739597297984
833 109.640438168064
834 109.512827148353
835 109.42778299688
836 109.421218607633
837 109.290530982673
838 109.18770256492
839 109.284456636469
840 109.548694727891
841 109.46393017768
842 109.335582624788
843 109.266974272813
844 109.158109375351
845 109.143214873429
846 109.128085721152
847 109.026524618453
848 108.943173727305
849 108.826816278002
850 108.836229757785
851 108.710021112923
852 108.609480041438
853 108.828645604988
854 108.721663860384
855 108.614884579871
856 108.508307221155
857 108.47963575415
858 108.430648605474
859 108.505395856513
860 108.542827203894
861 108.471269531013
862 108.376205177621
863 108.631787842942
864 108.885897580161
865 108.95201844365
866 109.017543695897
867 109.090570701447
868 109.060601998344
869 108.962592083013
870 108.912955476285
871 109.042313821689
872 109.243540106052
873 109.176769287089
874 109.122565704381
875 109.361862530612
876 109.326056587644
877 109.419132551237
878 109.294550152812
879 109.213738333845
880 109.258078512397
881 109.383475851015
882 109.275940837408
883 109.221197169641
884 109.117558762106
885 109.162072201475
886 109.069111180184
887 108.946915803749
888 108.882132132132
889 108.847326592612
890 109.095832596894
891 109.167431894705
892 109.260119849585
893 109.467030493486
894 109.363713596485
895 109.405818794669
896 109.376429966518
897 109.39193073903
898 109.407091234666
899 109.302426005412
900 109.185032098765
901 109.064765872424
902 108.945172344285
903 109.104500931434
904 109.145817017777
905 109.116517810812
906 109.241344190557
907 109.193186887725
908 109.163882231365
909 109.364465599475
910 109.245527110252
911 109.213192580981
912 109.121622758926
913 109.250473566076
914 109.130946281763
915 109.194749320672
916 109.356966638699
917 109.264435615164
918 109.303473023196
919 109.25342041605
920 109.204990548204
921 109.140159459399
922 109.060381091751
923 109.220594673969
924 109.103516613257
925 109.003014170928
926 108.972973004492
927 109.010451177605
928 109.000231077215
929 108.887327485021
930 109.079680887964
931 109.015772513992
932 109.078892593343
933 109.310863204475
934 109.262828478282
935 109.253311218508
936 109.265421825919
937 109.23563588236
938 109.119180218311
939 109.274804615065
940 109.227111815301
941 109.217690724025
942 109.340754188811
943 109.234317946942
944 109.122863401321
945 109.074971025447
946 108.969093188634
947 108.959845407439
948 108.950171580409
949 108.844740345614
950 108.967358448753
951 108.86955675635
952 108.759426196949
953 108.881316965588
954 109.106847918287
955 109.059685863874
956 109.122200022759
957 109.309362132841
958 109.204842421363
959 109.294314006705
960 109.305399305556
961 109.393826453324
962 109.283984768392
963 109.347724584281
964 109.360987241955
965 109.248791645413
966 109.15261435387
967 109.302866358176
968 109.20599579093
969 109.297120966686
970 109.357249442024
971 109.254258141086
972 109.371356839235
973 109.32584720654
974 109.239901715654
975 109.165469033531
976 109.282312382424
977 109.186268829967
978 109.110525633466
979 109.29128240549
980 109.264081632653
981 109.479505092164
982 109.372522098382
983 109.287709991524
984 109.185203086787
985 109.09871009302
986 108.991702290485
987 109.143844651185
988 109.048831934633
989 109.106813983137
990 109.327452300786
991 109.30040597466
992 109.255966075377
993 109.147095124076
994 109.205301223842
995 109.104087270523
996 109.281474290737
997 109.254902118592
998 109.169518395508
999 109.139463788112
1000 109.148839
};
\addlegendentry{$v_{v}$ (valor de la varianza de las tiradas)}
\addplot [semithick, color1]
table {%
1 114
2 114
3 114
4 114
5 114
6 114
7 114
8 114
9 114
10 114
11 114
12 114
13 114
14 114
15 114
16 114
17 114
18 114
19 114
20 114
21 114
22 114
23 114
24 114
25 114
26 114
27 114
28 114
29 114
30 114
31 114
32 114
33 114
34 114
35 114
36 114
37 114
38 114
39 114
40 114
41 114
42 114
43 114
44 114
45 114
46 114
47 114
48 114
49 114
50 114
51 114
52 114
53 114
54 114
55 114
56 114
57 114
58 114
59 114
60 114
61 114
62 114
63 114
64 114
65 114
66 114
67 114
68 114
69 114
70 114
71 114
72 114
73 114
74 114
75 114
76 114
77 114
78 114
79 114
80 114
81 114
82 114
83 114
84 114
85 114
86 114
87 114
88 114
89 114
90 114
91 114
92 114
93 114
94 114
95 114
96 114
97 114
98 114
99 114
100 114
101 114
102 114
103 114
104 114
105 114
106 114
107 114
108 114
109 114
110 114
111 114
112 114
113 114
114 114
115 114
116 114
117 114
118 114
119 114
120 114
121 114
122 114
123 114
124 114
125 114
126 114
127 114
128 114
129 114
130 114
131 114
132 114
133 114
134 114
135 114
136 114
137 114
138 114
139 114
140 114
141 114
142 114
143 114
144 114
145 114
146 114
147 114
148 114
149 114
150 114
151 114
152 114
153 114
154 114
155 114
156 114
157 114
158 114
159 114
160 114
161 114
162 114
163 114
164 114
165 114
166 114
167 114
168 114
169 114
170 114
171 114
172 114
173 114
174 114
175 114
176 114
177 114
178 114
179 114
180 114
181 114
182 114
183 114
184 114
185 114
186 114
187 114
188 114
189 114
190 114
191 114
192 114
193 114
194 114
195 114
196 114
197 114
198 114
199 114
200 114
201 114
202 114
203 114
204 114
205 114
206 114
207 114
208 114
209 114
210 114
211 114
212 114
213 114
214 114
215 114
216 114
217 114
218 114
219 114
220 114
221 114
222 114
223 114
224 114
225 114
226 114
227 114
228 114
229 114
230 114
231 114
232 114
233 114
234 114
235 114
236 114
237 114
238 114
239 114
240 114
241 114
242 114
243 114
244 114
245 114
246 114
247 114
248 114
249 114
250 114
251 114
252 114
253 114
254 114
255 114
256 114
257 114
258 114
259 114
260 114
261 114
262 114
263 114
264 114
265 114
266 114
267 114
268 114
269 114
270 114
271 114
272 114
273 114
274 114
275 114
276 114
277 114
278 114
279 114
280 114
281 114
282 114
283 114
284 114
285 114
286 114
287 114
288 114
289 114
290 114
291 114
292 114
293 114
294 114
295 114
296 114
297 114
298 114
299 114
300 114
301 114
302 114
303 114
304 114
305 114
306 114
307 114
308 114
309 114
310 114
311 114
312 114
313 114
314 114
315 114
316 114
317 114
318 114
319 114
320 114
321 114
322 114
323 114
324 114
325 114
326 114
327 114
328 114
329 114
330 114
331 114
332 114
333 114
334 114
335 114
336 114
337 114
338 114
339 114
340 114
341 114
342 114
343 114
344 114
345 114
346 114
347 114
348 114
349 114
350 114
351 114
352 114
353 114
354 114
355 114
356 114
357 114
358 114
359 114
360 114
361 114
362 114
363 114
364 114
365 114
366 114
367 114
368 114
369 114
370 114
371 114
372 114
373 114
374 114
375 114
376 114
377 114
378 114
379 114
380 114
381 114
382 114
383 114
384 114
385 114
386 114
387 114
388 114
389 114
390 114
391 114
392 114
393 114
394 114
395 114
396 114
397 114
398 114
399 114
400 114
401 114
402 114
403 114
404 114
405 114
406 114
407 114
408 114
409 114
410 114
411 114
412 114
413 114
414 114
415 114
416 114
417 114
418 114
419 114
420 114
421 114
422 114
423 114
424 114
425 114
426 114
427 114
428 114
429 114
430 114
431 114
432 114
433 114
434 114
435 114
436 114
437 114
438 114
439 114
440 114
441 114
442 114
443 114
444 114
445 114
446 114
447 114
448 114
449 114
450 114
451 114
452 114
453 114
454 114
455 114
456 114
457 114
458 114
459 114
460 114
461 114
462 114
463 114
464 114
465 114
466 114
467 114
468 114
469 114
470 114
471 114
472 114
473 114
474 114
475 114
476 114
477 114
478 114
479 114
480 114
481 114
482 114
483 114
484 114
485 114
486 114
487 114
488 114
489 114
490 114
491 114
492 114
493 114
494 114
495 114
496 114
497 114
498 114
499 114
500 114
501 114
502 114
503 114
504 114
505 114
506 114
507 114
508 114
509 114
510 114
511 114
512 114
513 114
514 114
515 114
516 114
517 114
518 114
519 114
520 114
521 114
522 114
523 114
524 114
525 114
526 114
527 114
528 114
529 114
530 114
531 114
532 114
533 114
534 114
535 114
536 114
537 114
538 114
539 114
540 114
541 114
542 114
543 114
544 114
545 114
546 114
547 114
548 114
549 114
550 114
551 114
552 114
553 114
554 114
555 114
556 114
557 114
558 114
559 114
560 114
561 114
562 114
563 114
564 114
565 114
566 114
567 114
568 114
569 114
570 114
571 114
572 114
573 114
574 114
575 114
576 114
577 114
578 114
579 114
580 114
581 114
582 114
583 114
584 114
585 114
586 114
587 114
588 114
589 114
590 114
591 114
592 114
593 114
594 114
595 114
596 114
597 114
598 114
599 114
600 114
601 114
602 114
603 114
604 114
605 114
606 114
607 114
608 114
609 114
610 114
611 114
612 114
613 114
614 114
615 114
616 114
617 114
618 114
619 114
620 114
621 114
622 114
623 114
624 114
625 114
626 114
627 114
628 114
629 114
630 114
631 114
632 114
633 114
634 114
635 114
636 114
637 114
638 114
639 114
640 114
641 114
642 114
643 114
644 114
645 114
646 114
647 114
648 114
649 114
650 114
651 114
652 114
653 114
654 114
655 114
656 114
657 114
658 114
659 114
660 114
661 114
662 114
663 114
664 114
665 114
666 114
667 114
668 114
669 114
670 114
671 114
672 114
673 114
674 114
675 114
676 114
677 114
678 114
679 114
680 114
681 114
682 114
683 114
684 114
685 114
686 114
687 114
688 114
689 114
690 114
691 114
692 114
693 114
694 114
695 114
696 114
697 114
698 114
699 114
700 114
701 114
702 114
703 114
704 114
705 114
706 114
707 114
708 114
709 114
710 114
711 114
712 114
713 114
714 114
715 114
716 114
717 114
718 114
719 114
720 114
721 114
722 114
723 114
724 114
725 114
726 114
727 114
728 114
729 114
730 114
731 114
732 114
733 114
734 114
735 114
736 114
737 114
738 114
739 114
740 114
741 114
742 114
743 114
744 114
745 114
746 114
747 114
748 114
749 114
750 114
751 114
752 114
753 114
754 114
755 114
756 114
757 114
758 114
759 114
760 114
761 114
762 114
763 114
764 114
765 114
766 114
767 114
768 114
769 114
770 114
771 114
772 114
773 114
774 114
775 114
776 114
777 114
778 114
779 114
780 114
781 114
782 114
783 114
784 114
785 114
786 114
787 114
788 114
789 114
790 114
791 114
792 114
793 114
794 114
795 114
796 114
797 114
798 114
799 114
800 114
801 114
802 114
803 114
804 114
805 114
806 114
807 114
808 114
809 114
810 114
811 114
812 114
813 114
814 114
815 114
816 114
817 114
818 114
819 114
820 114
821 114
822 114
823 114
824 114
825 114
826 114
827 114
828 114
829 114
830 114
831 114
832 114
833 114
834 114
835 114
836 114
837 114
838 114
839 114
840 114
841 114
842 114
843 114
844 114
845 114
846 114
847 114
848 114
849 114
850 114
851 114
852 114
853 114
854 114
855 114
856 114
857 114
858 114
859 114
860 114
861 114
862 114
863 114
864 114
865 114
866 114
867 114
868 114
869 114
870 114
871 114
872 114
873 114
874 114
875 114
876 114
877 114
878 114
879 114
880 114
881 114
882 114
883 114
884 114
885 114
886 114
887 114
888 114
889 114
890 114
891 114
892 114
893 114
894 114
895 114
896 114
897 114
898 114
899 114
900 114
901 114
902 114
903 114
904 114
905 114
906 114
907 114
908 114
909 114
910 114
911 114
912 114
913 114
914 114
915 114
916 114
917 114
918 114
919 114
920 114
921 114
922 114
923 114
924 114
925 114
926 114
927 114
928 114
929 114
930 114
931 114
932 114
933 114
934 114
935 114
936 114
937 114
938 114
939 114
940 114
941 114
942 114
943 114
944 114
945 114
946 114
947 114
948 114
949 114
950 114
951 114
952 114
953 114
954 114
955 114
956 114
957 114
958 114
959 114
960 114
961 114
962 114
963 114
964 114
965 114
966 114
967 114
968 114
969 114
970 114
971 114
972 114
973 114
974 114
975 114
976 114
977 114
978 114
979 114
980 114
981 114
982 114
983 114
984 114
985 114
986 114
987 114
988 114
989 114
990 114
991 114
992 114
993 114
994 114
995 114
996 114
997 114
998 114
999 114
1000 114
};
\addlegendentry{$v_{v_{e}}$ (valor de la varianza esperada)}
\end{axis}

\end{tikzpicture}

    \caption{valor de la varianza con respecto al número de tiradas}
  \end{mytikzresize}
\end{figure}

%\bibliographystyle{unsrt}
%\bibliography{references}

\end{document}
