\documentclass{article}

\usepackage{arxiv}

\usepackage[utf8]{inputenc} % allow utf-8 input
\usepackage[spanish]{babel} % idioma español
\usepackage[T1]{fontenc}    % use 8-bit T1 fonts
\usepackage{hyperref}       % hyperlinks
\usepackage{url}            % simple URL typesetting
\usepackage{booktabs}       % professional-quality tables
\usepackage{amsfonts}       % blackboard math symbols
\usepackage{nicefrac}       % compact symbols for 1/2, etc.
\usepackage{microtype}      % microtypography
\usepackage{graphicx}
\graphicspath{ {./images/} }
\usepackage{kpfonts}        % use the same fonts for text and maths

\usepackage{pgfplots}       % TikZ graphics
\pgfplotsset{compat=1.15}

\usepackage{mytikz}

\title{TP 1.1 - Simulación de una Ruleta}

\author{
 Marcelo G. Catellano \\
  UTN -- FRRo \\
  \texttt{marce.geek22@gmail.com} \\
}

\begin{document}
\maketitle
\begin{abstract}
Simulación de un modelo simple de una ruleta empleando el lenguaje de programación Python 3.x.
\end{abstract}

% keywords can be removed
%\keywords{First keyword \and Second keyword \and More}

\section[Introducción]{Introducción\footnote{Wikipedia - \url{https://es.wikipedia.org/wiki/Ruleta}}}
La ruleta es un juego de azar típico de los casinos, cuyo nombre viene del término francés roulette, que significa ``ruedita'' o ``rueda pequeña''. Su uso como elemento de juego de azar, aún en configuraciones distintas de la actual, no está documentado hasta bien entrada la Edad Media. Es de suponer que su referencia más antigua es la llamada Rueda de la Fortuna, de la que hay noticias a lo largo de toda la historia, prácticamente en todos los campos del saber humano.

La ``magia'' del movimiento de las ruedas tuvo que impactar a todas las generaciones. La aparente quietud del centro, el aumento de velocidad conforme nos alejamos de él, la posibilidad de que se detenga en un punto al azar; todo esto tuvo que influir en el desarrollo de distintos juegos que tienen la rueda como base.

Las ruedas, y por extensión las ruletas, siempre han tenido conexión con el mundo mágico y esotérico. Así, una de ellas forma parte del tarot, más precisamente de los que se conocen como arcanos mayores.

Según los indicios, la creación de una ruleta y sus normas de juego, muy similares a las que conocemos hoy en día, se debe a Blaise Pascal, matemático francés, quien ideó una ruleta con treinta y seis números (sin el cero), en la que se halla un extremado equilibrio en la posición en que está colocado cada número. La elección de 36 números da un alcance aún más vinculado a la magia (la suma de los primeros 36 números da el número mágico por excelencia: seiscientos sesenta y seis).

Esta ruleta podía usarse como entretenimiento en círculos de amistades. Sin embargo, a nivel de empresa que pone los medios y el personal para el entretenimiento de sus clientes, no era rentable, ya que estadísticamente todo lo que se apostaba se repartía en premios (probabilidad de 1/36 de acertar el número y ganar 36 veces lo apostado).

En 1842, los hermanos Blanc modificaron la ruleta añadiéndole un nuevo número, el 0, y la introdujeron inicialmente en el Casino de Montecarlo. Ésta es la ruleta que se conoce hoy en día, con una probabilidad de acertar de 1/37 y ganar 36 veces lo apostado, consiguiendo un margen para la casa del $2.7\%$ (1/37).

Más adelante, en algunas ruletas (sobre todo las que se usan en países anglosajones) se añadió un nuevo número (el doble cero), con lo cual el beneficio para el casino resultó ser doble (2/38 o $5.26\%$).

\section{Gráficas}
\begin{figure}[!htbp]
  \begin{mytikzresize}{0.6\textwidth}
    \centering
    % This file was created by tikzplotlib v0.9.1.
\begin{tikzpicture}

\definecolor{color0}{rgb}{0.12156862745098,0.466666666666667,0.705882352941177}
\definecolor{color1}{rgb}{1,0.498039215686275,0.0549019607843137}

\begin{axis}[
legend cell align={left},
legend style={fill opacity=0.5, draw opacity=1, text opacity=1, draw=white!80!black},
scaled ticks=false,
tick align=outside,
tick pos=left,
width=\figW,
x grid style={white!69.0196078431373!black},
xlabel={\(\displaystyle n\) (número de tiradas)},
xmajorgrids,
xmin=-48.95, xmax=1049.95,
xtick style={color=black},
xticklabel style={/pgf/number format/.cd,fixed,precision=2},
y grid style={white!69.0196078431373!black},
ylabel={\(\displaystyle f_{r}\) (frecuencia relativa)},
ymajorgrids,
ymin=-0.00327868852459016, ymax=0.0688524590163934,
ytick style={color=black},
yticklabel style={/pgf/number format/.cd,fixed,precision=2}
]
\addplot [semithick, color0]
table {%
1 0
2 0
3 0
4 0
5 0
6 0
7 0
8 0
9 0
10 0
11 0
12 0
13 0
14 0
15 0
16 0
17 0
18 0
19 0
20 0
21 0
22 0
23 0
24 0
25 0
26 0
27 0.037037037037037
28 0.0357142857142857
29 0.0344827586206897
30 0.0333333333333333
31 0.032258064516129
32 0.03125
33 0.0303030303030303
34 0.0294117647058824
35 0.0285714285714286
36 0.0277777777777778
37 0.027027027027027
38 0.0263157894736842
39 0.0256410256410256
40 0.025
41 0.0487804878048781
42 0.0476190476190476
43 0.0465116279069767
44 0.0454545454545455
45 0.0444444444444444
46 0.0434782608695652
47 0.0425531914893617
48 0.0416666666666667
49 0.0408163265306122
50 0.04
51 0.0392156862745098
52 0.0384615384615385
53 0.0377358490566038
54 0.037037037037037
55 0.0545454545454545
56 0.0535714285714286
57 0.0526315789473684
58 0.0517241379310345
59 0.0508474576271186
60 0.05
61 0.0655737704918033
62 0.0645161290322581
63 0.0634920634920635
64 0.0625
65 0.0615384615384615
66 0.0606060606060606
67 0.0597014925373134
68 0.0588235294117647
69 0.0579710144927536
70 0.0571428571428571
71 0.0563380281690141
72 0.0555555555555556
73 0.0547945205479452
74 0.0540540540540541
75 0.0533333333333333
76 0.0526315789473684
77 0.051948051948052
78 0.0512820512820513
79 0.0506329113924051
80 0.05
81 0.0493827160493827
82 0.0487804878048781
83 0.0481927710843374
84 0.0476190476190476
85 0.0470588235294118
86 0.0465116279069767
87 0.0459770114942529
88 0.0454545454545455
89 0.0449438202247191
90 0.0444444444444444
91 0.0549450549450549
92 0.0543478260869565
93 0.0537634408602151
94 0.0531914893617021
95 0.0526315789473684
96 0.0520833333333333
97 0.0515463917525773
98 0.0510204081632653
99 0.0505050505050505
100 0.05
101 0.0495049504950495
102 0.0490196078431373
103 0.0485436893203883
104 0.0480769230769231
105 0.0476190476190476
106 0.0471698113207547
107 0.0467289719626168
108 0.0462962962962963
109 0.0458715596330275
110 0.0454545454545455
111 0.045045045045045
112 0.0446428571428571
113 0.0442477876106195
114 0.043859649122807
115 0.0434782608695652
116 0.0431034482758621
117 0.0427350427350427
118 0.0423728813559322
119 0.0420168067226891
120 0.0416666666666667
121 0.0413223140495868
122 0.040983606557377
123 0.040650406504065
124 0.0403225806451613
125 0.04
126 0.0396825396825397
127 0.0393700787401575
128 0.0390625
129 0.0387596899224806
130 0.0384615384615385
131 0.0381679389312977
132 0.0378787878787879
133 0.037593984962406
134 0.0373134328358209
135 0.037037037037037
136 0.0367647058823529
137 0.0364963503649635
138 0.036231884057971
139 0.0359712230215827
140 0.0357142857142857
141 0.0354609929078014
142 0.0352112676056338
143 0.034965034965035
144 0.0347222222222222
145 0.0344827586206897
146 0.0342465753424658
147 0.0340136054421769
148 0.0337837837837838
149 0.0335570469798658
150 0.0333333333333333
151 0.033112582781457
152 0.0328947368421053
153 0.0326797385620915
154 0.0324675324675325
155 0.032258064516129
156 0.032051282051282
157 0.0318471337579618
158 0.0316455696202532
159 0.0314465408805031
160 0.03125
161 0.031055900621118
162 0.0308641975308642
163 0.0306748466257669
164 0.0304878048780488
165 0.0303030303030303
166 0.0301204819277108
167 0.029940119760479
168 0.0297619047619048
169 0.029585798816568
170 0.0294117647058824
171 0.0292397660818713
172 0.0290697674418605
173 0.0289017341040462
174 0.028735632183908
175 0.0285714285714286
176 0.0284090909090909
177 0.0282485875706215
178 0.0280898876404494
179 0.0279329608938547
180 0.0277777777777778
181 0.0276243093922652
182 0.0274725274725275
183 0.0273224043715847
184 0.0271739130434783
185 0.027027027027027
186 0.0268817204301075
187 0.0267379679144385
188 0.0265957446808511
189 0.0264550264550265
190 0.0263157894736842
191 0.0261780104712042
192 0.0260416666666667
193 0.0259067357512953
194 0.0257731958762887
195 0.0307692307692308
196 0.0306122448979592
197 0.0304568527918782
198 0.0303030303030303
199 0.0301507537688442
200 0.03
201 0.0298507462686567
202 0.0297029702970297
203 0.0295566502463054
204 0.0294117647058824
205 0.0292682926829268
206 0.029126213592233
207 0.0289855072463768
208 0.0288461538461538
209 0.0287081339712919
210 0.0285714285714286
211 0.028436018957346
212 0.0283018867924528
213 0.028169014084507
214 0.0280373831775701
215 0.027906976744186
216 0.0277777777777778
217 0.0276497695852535
218 0.0275229357798165
219 0.0273972602739726
220 0.0272727272727273
221 0.0271493212669683
222 0.027027027027027
223 0.0269058295964126
224 0.0267857142857143
225 0.0266666666666667
226 0.0265486725663717
227 0.026431718061674
228 0.0307017543859649
229 0.0305676855895196
230 0.0304347826086957
231 0.0303030303030303
232 0.0301724137931034
233 0.0300429184549356
234 0.0341880341880342
235 0.0340425531914894
236 0.0338983050847458
237 0.0337552742616034
238 0.0336134453781513
239 0.0334728033472803
240 0.0333333333333333
241 0.033195020746888
242 0.0330578512396694
243 0.0329218106995885
244 0.0327868852459016
245 0.0326530612244898
246 0.032520325203252
247 0.0323886639676113
248 0.032258064516129
249 0.0321285140562249
250 0.032
251 0.0318725099601594
252 0.0317460317460317
253 0.0316205533596838
254 0.031496062992126
255 0.0313725490196078
256 0.03125
257 0.0311284046692607
258 0.0310077519379845
259 0.0308880308880309
260 0.0307692307692308
261 0.0306513409961686
262 0.0305343511450382
263 0.0304182509505703
264 0.0303030303030303
265 0.030188679245283
266 0.0300751879699248
267 0.0299625468164794
268 0.0298507462686567
269 0.033457249070632
270 0.0333333333333333
271 0.033210332103321
272 0.0330882352941176
273 0.032967032967033
274 0.0328467153284672
275 0.0327272727272727
276 0.0326086956521739
277 0.0324909747292419
278 0.0323741007194245
279 0.032258064516129
280 0.0321428571428571
281 0.0320284697508897
282 0.0319148936170213
283 0.0318021201413428
284 0.0316901408450704
285 0.0315789473684211
286 0.0314685314685315
287 0.0313588850174216
288 0.03125
289 0.0311418685121107
290 0.0310344827586207
291 0.0309278350515464
292 0.0308219178082192
293 0.0307167235494881
294 0.0306122448979592
295 0.0305084745762712
296 0.0304054054054054
297 0.0303030303030303
298 0.0302013422818792
299 0.0301003344481605
300 0.03
301 0.0299003322259136
302 0.0298013245033113
303 0.0297029702970297
304 0.0296052631578947
305 0.0327868852459016
306 0.0359477124183007
307 0.0358306188925081
308 0.0357142857142857
309 0.0355987055016181
310 0.0354838709677419
311 0.0353697749196141
312 0.0352564102564103
313 0.0351437699680511
314 0.035031847133758
315 0.0349206349206349
316 0.0348101265822785
317 0.0347003154574132
318 0.0345911949685535
319 0.0344827586206897
320 0.034375
321 0.0342679127725857
322 0.0341614906832298
323 0.0340557275541796
324 0.0339506172839506
325 0.0338461538461538
326 0.0337423312883436
327 0.0336391437308868
328 0.0335365853658537
329 0.033434650455927
330 0.0333333333333333
331 0.0332326283987915
332 0.0331325301204819
333 0.033033033033033
334 0.0329341317365269
335 0.0328358208955224
336 0.0357142857142857
337 0.0356083086053412
338 0.0355029585798817
339 0.0353982300884956
340 0.0352941176470588
341 0.0351906158357771
342 0.0350877192982456
343 0.0349854227405248
344 0.0348837209302326
345 0.0347826086956522
346 0.0346820809248555
347 0.0345821325648415
348 0.0344827586206897
349 0.0343839541547278
350 0.0342857142857143
351 0.0341880341880342
352 0.0340909090909091
353 0.0339943342776204
354 0.0338983050847458
355 0.0338028169014084
356 0.0337078651685393
357 0.0336134453781513
358 0.0335195530726257
359 0.0334261838440111
360 0.0333333333333333
361 0.0332409972299169
362 0.0331491712707182
363 0.0330578512396694
364 0.032967032967033
365 0.0328767123287671
366 0.0327868852459016
367 0.0326975476839237
368 0.0326086956521739
369 0.032520325203252
370 0.0324324324324324
371 0.032345013477089
372 0.032258064516129
373 0.032171581769437
374 0.0320855614973262
375 0.032
376 0.0319148936170213
377 0.0318302387267905
378 0.0317460317460317
379 0.0316622691292876
380 0.0315789473684211
381 0.031496062992126
382 0.031413612565445
383 0.031331592689295
384 0.03125
385 0.0311688311688312
386 0.0310880829015544
387 0.0310077519379845
388 0.0309278350515464
389 0.0308483290488432
390 0.0307692307692308
391 0.030690537084399
392 0.0306122448979592
393 0.0305343511450382
394 0.0304568527918782
395 0.030379746835443
396 0.0303030303030303
397 0.0302267002518892
398 0.0301507537688442
399 0.0300751879699248
400 0.03
401 0.029925187032419
402 0.0298507462686567
403 0.0297766749379653
404 0.0297029702970297
405 0.0296296296296296
406 0.0320197044334975
407 0.0319410319410319
408 0.0318627450980392
409 0.0317848410757946
410 0.0317073170731707
411 0.0316301703163017
412 0.0315533980582524
413 0.0314769975786925
414 0.0314009661835749
415 0.0313253012048193
416 0.03125
417 0.0311750599520384
418 0.0311004784688995
419 0.0310262529832936
420 0.030952380952381
421 0.0308788598574822
422 0.0308056872037915
423 0.0307328605200946
424 0.0306603773584906
425 0.0305882352941176
426 0.0305164319248826
427 0.0304449648711944
428 0.0303738317757009
429 0.0303030303030303
430 0.0302325581395349
431 0.0301624129930394
432 0.0300925925925926
433 0.0300230946882217
434 0.0299539170506912
435 0.0298850574712644
436 0.0298165137614679
437 0.0297482837528604
438 0.0296803652968037
439 0.0296127562642369
440 0.0295454545454545
441 0.0294784580498866
442 0.0294117647058824
443 0.0293453724604966
444 0.0292792792792793
445 0.0292134831460674
446 0.0291479820627803
447 0.029082774049217
448 0.0290178571428571
449 0.0289532293986637
450 0.0288888888888889
451 0.0288248337028825
452 0.0287610619469027
453 0.0286975717439294
454 0.0286343612334802
455 0.0285714285714286
456 0.0285087719298246
457 0.0284463894967177
458 0.0283842794759825
459 0.028322440087146
460 0.0282608695652174
461 0.0281995661605206
462 0.0281385281385281
463 0.0280777537796976
464 0.0280172413793103
465 0.0279569892473118
466 0.0278969957081545
467 0.0278372591006424
468 0.0277777777777778
469 0.0277185501066098
470 0.0276595744680851
471 0.0276008492569002
472 0.0296610169491525
473 0.0295983086680761
474 0.029535864978903
475 0.0294736842105263
476 0.0294117647058824
477 0.0293501048218029
478 0.0292887029288703
479 0.0292275574112735
480 0.0291666666666667
481 0.0311850311850312
482 0.0311203319502075
483 0.031055900621118
484 0.0309917355371901
485 0.0309278350515464
486 0.0308641975308642
487 0.0308008213552361
488 0.0307377049180328
489 0.0306748466257669
490 0.0306122448979592
491 0.0305498981670061
492 0.0304878048780488
493 0.0304259634888438
494 0.0303643724696356
495 0.0323232323232323
496 0.032258064516129
497 0.0321931589537223
498 0.0321285140562249
499 0.032064128256513
500 0.032
501 0.031936127744511
502 0.0318725099601594
503 0.0318091451292246
504 0.0317460317460317
505 0.0316831683168317
506 0.0316205533596838
507 0.0315581854043393
508 0.031496062992126
509 0.031434184675835
510 0.0313725490196078
511 0.0313111545988258
512 0.03125
513 0.0311890838206628
514 0.0311284046692607
515 0.0310679611650485
516 0.0310077519379845
517 0.0309477756286267
518 0.0308880308880309
519 0.0308285163776493
520 0.0307692307692308
521 0.0307101727447217
522 0.0306513409961686
523 0.0305927342256214
524 0.0305343511450382
525 0.0304761904761905
526 0.0304182509505703
527 0.0303605313092979
528 0.0303030303030303
529 0.0302457466918715
530 0.030188679245283
531 0.0301318267419962
532 0.0300751879699248
533 0.0300187617260788
534 0.0299625468164794
535 0.0299065420560748
536 0.0298507462686567
537 0.0297951582867784
538 0.0297397769516729
539 0.0296846011131725
540 0.0296296296296296
541 0.0295748613678373
542 0.029520295202952
543 0.0294659300184162
544 0.0294117647058824
545 0.0293577981651376
546 0.0293040293040293
547 0.0292504570383912
548 0.0291970802919708
549 0.029143897996357
550 0.0290909090909091
551 0.029038112522686
552 0.0289855072463768
553 0.0289330922242315
554 0.0288808664259928
555 0.0288288288288288
556 0.0287769784172662
557 0.0287253141831239
558 0.028673835125448
559 0.0286225402504472
560 0.0285714285714286
561 0.0285204991087344
562 0.0284697508896797
563 0.0284191829484902
564 0.0283687943262411
565 0.0283185840707965
566 0.0282685512367491
567 0.0282186948853616
568 0.028169014084507
569 0.0281195079086116
570 0.0280701754385965
571 0.0280210157618214
572 0.0297202797202797
573 0.0296684118673647
574 0.029616724738676
575 0.0295652173913043
576 0.0295138888888889
577 0.0294627383015598
578 0.0294117647058824
579 0.0310880829015544
580 0.0310344827586207
581 0.0309810671256454
582 0.0309278350515464
583 0.0308747855917667
584 0.0308219178082192
585 0.0307692307692308
586 0.0307167235494881
587 0.030664395229983
588 0.0306122448979592
589 0.0305602716468591
590 0.0305084745762712
591 0.0304568527918782
592 0.0304054054054054
593 0.03035413153457
594 0.0303030303030303
595 0.0302521008403361
596 0.0302013422818792
597 0.0301507537688442
598 0.0301003344481605
599 0.0317195325542571
600 0.0316666666666667
601 0.0316139767054908
602 0.0315614617940199
603 0.0315091210613599
604 0.0314569536423841
605 0.031404958677686
606 0.0313531353135314
607 0.0313014827018122
608 0.03125
609 0.0328407224958949
610 0.0327868852459016
611 0.0327332242225859
612 0.0326797385620915
613 0.032626427406199
614 0.0325732899022801
615 0.032520325203252
616 0.0324675324675325
617 0.0324149108589951
618 0.0323624595469256
619 0.0323101777059774
620 0.032258064516129
621 0.0322061191626409
622 0.0321543408360129
623 0.0321027287319422
624 0.032051282051282
625 0.032
626 0.0319488817891374
627 0.0318979266347687
628 0.0318471337579618
629 0.0317965023847377
630 0.0333333333333333
631 0.0332805071315372
632 0.0332278481012658
633 0.033175355450237
634 0.0331230283911672
635 0.0330708661417323
636 0.0330188679245283
637 0.032967032967033
638 0.0329153605015674
639 0.0328638497652582
640 0.0328125
641 0.0327613104524181
642 0.0327102803738318
643 0.0326594090202177
644 0.0326086956521739
645 0.0325581395348837
646 0.0325077399380805
647 0.0324574961360124
648 0.0324074074074074
649 0.0323574730354391
650 0.0323076923076923
651 0.032258064516129
652 0.0322085889570552
653 0.0321592649310873
654 0.0321100917431193
655 0.0320610687022901
656 0.0320121951219512
657 0.0319634703196347
658 0.0319148936170213
659 0.031866464339909
660 0.0318181818181818
661 0.0317700453857791
662 0.0317220543806647
663 0.0316742081447964
664 0.0316265060240964
665 0.0315789473684211
666 0.0315315315315315
667 0.0314842578710645
668 0.031437125748503
669 0.031390134529148
670 0.0313432835820895
671 0.0327868852459016
672 0.0327380952380952
673 0.0341753343239227
674 0.0341246290801187
675 0.0340740740740741
676 0.0340236686390533
677 0.03397341211226
678 0.0339233038348083
679 0.0338733431516937
680 0.0338235294117647
681 0.0337738619676946
682 0.0337243401759531
683 0.0336749633967789
684 0.033625730994152
685 0.0335766423357664
686 0.0335276967930029
687 0.0334788937409025
688 0.0334302325581395
689 0.0333817126269956
690 0.0333333333333333
691 0.0332850940665702
692 0.0332369942196532
693 0.0331890331890332
694 0.0331412103746398
695 0.0330935251798561
696 0.0330459770114943
697 0.0329985652797704
698 0.0329512893982808
699 0.0329041487839771
700 0.0328571428571429
701 0.0328102710413695
702 0.0327635327635328
703 0.0327169274537696
704 0.0326704545454545
705 0.0340425531914894
706 0.0339943342776204
707 0.0339462517680339
708 0.0338983050847458
709 0.0338504936530324
710 0.0338028169014084
711 0.0337552742616034
712 0.0337078651685393
713 0.0336605890603086
714 0.0336134453781513
715 0.0335664335664336
716 0.0335195530726257
717 0.0334728033472803
718 0.0334261838440111
719 0.0333796940194715
720 0.0333333333333333
721 0.0332871012482663
722 0.0332409972299169
723 0.033195020746888
724 0.0331491712707182
725 0.0331034482758621
726 0.0330578512396694
727 0.0330123796423659
728 0.032967032967033
729 0.0329218106995885
730 0.0328767123287671
731 0.0328317373461012
732 0.0327868852459016
733 0.0327421555252387
734 0.0326975476839237
735 0.0326530612244898
736 0.0326086956521739
737 0.0325644504748982
738 0.032520325203252
739 0.0324763193504736
740 0.0324324324324324
741 0.0323886639676113
742 0.032345013477089
743 0.0323014804845222
744 0.032258064516129
745 0.0322147651006711
746 0.032171581769437
747 0.0321285140562249
748 0.0320855614973262
749 0.0320427236315087
750 0.032
751 0.0319573901464714
752 0.0319148936170213
753 0.0318725099601594
754 0.0318302387267905
755 0.0317880794701987
756 0.0317460317460317
757 0.0317040951122853
758 0.0316622691292876
759 0.0316205533596838
760 0.0315789473684211
761 0.0315374507227332
762 0.031496062992126
763 0.0314547837483617
764 0.031413612565445
765 0.0313725490196078
766 0.031331592689295
767 0.0312907431551499
768 0.03125
769 0.0312093628088427
770 0.0311688311688312
771 0.0311284046692607
772 0.0310880829015544
773 0.0310478654592497
774 0.0310077519379845
775 0.0309677419354839
776 0.0309278350515464
777 0.0308880308880309
778 0.0308483290488432
779 0.030808729139923
780 0.0307692307692308
781 0.030729833546735
782 0.030690537084399
783 0.0306513409961686
784 0.0306122448979592
785 0.0305732484076433
786 0.0305343511450382
787 0.0304955527318933
788 0.0304568527918782
789 0.0304182509505703
790 0.030379746835443
791 0.0303413400758533
792 0.0303030303030303
793 0.0302648171500631
794 0.0302267002518892
795 0.030188679245283
796 0.0301507537688442
797 0.0301129234629862
798 0.0300751879699248
799 0.0300375469336671
800 0.03
801 0.0299625468164794
802 0.0311720698254364
803 0.0311332503113325
804 0.0310945273631841
805 0.031055900621118
806 0.0310173697270471
807 0.0309789343246592
808 0.0309405940594059
809 0.030902348578492
810 0.0308641975308642
811 0.030826140567201
812 0.0307881773399015
813 0.030750307503075
814 0.0307125307125307
815 0.0306748466257669
816 0.0306372549019608
817 0.0305997552019584
818 0.0305623471882641
819 0.0305250305250305
820 0.0304878048780488
821 0.0304506699147381
822 0.0304136253041363
823 0.0303766707168894
824 0.0303398058252427
825 0.0303030303030303
826 0.0302663438256659
827 0.030229746070133
828 0.0301932367149758
829 0.0301568154402895
830 0.0301204819277108
831 0.0300842358604091
832 0.0300480769230769
833 0.0300120048019208
834 0.0299760191846523
835 0.029940119760479
836 0.0299043062200957
837 0.031063321385902
838 0.0310262529832936
839 0.0309892729439809
840 0.030952380952381
841 0.0309155766944114
842 0.0308788598574822
843 0.0308422301304864
844 0.0308056872037915
845 0.0307692307692308
846 0.0307328605200946
847 0.0306965761511216
848 0.0306603773584906
849 0.0306242638398115
850 0.0305882352941176
851 0.0305522914218566
852 0.0305164319248826
853 0.0304806565064478
854 0.0304449648711944
855 0.0304093567251462
856 0.0303738317757009
857 0.0303383897316219
858 0.0303030303030303
859 0.030267753201397
860 0.0302325581395349
861 0.0301974448315912
862 0.0301624129930394
863 0.0301274623406721
864 0.0300925925925926
865 0.0300578034682081
866 0.0300230946882217
867 0.0299884659746251
868 0.0299539170506912
869 0.0299194476409666
870 0.0298850574712644
871 0.0298507462686567
872 0.0298165137614679
873 0.0297823596792669
874 0.0297482837528604
875 0.0297142857142857
876 0.0296803652968037
877 0.0296465222348917
878 0.030751708428246
879 0.0307167235494881
880 0.0306818181818182
881 0.0306469920544835
882 0.0306122448979592
883 0.0305775764439411
884 0.0305429864253394
885 0.0305084745762712
886 0.0304740406320542
887 0.0304396843291995
888 0.0304054054054054
889 0.0303712035995501
890 0.0303370786516854
891 0.0303030303030303
892 0.0302690582959641
893 0.0302351623740202
894 0.0302013422818792
895 0.0301675977653631
896 0.0301339285714286
897 0.0301003344481605
898 0.0300668151447661
899 0.0300333704115684
900 0.03
901 0.0299667036625971
902 0.0299334811529933
903 0.0299003322259136
904 0.0298672566371681
905 0.0298342541436464
906 0.0298013245033113
907 0.0297684674751929
908 0.0297356828193833
909 0.0297029702970297
910 0.0296703296703297
911 0.0296377607025247
912 0.0296052631578947
913 0.0295728368017525
914 0.0306345733041575
915 0.0306010928961749
916 0.0305676855895196
917 0.0305343511450382
918 0.0305010893246187
919 0.0304678998911861
920 0.0304347826086957
921 0.0304017372421281
922 0.0303687635574837
923 0.0303358613217768
924 0.0303030303030303
925 0.0302702702702703
926 0.0302375809935205
927 0.0302049622437972
928 0.0301724137931034
929 0.0301399354144241
930 0.0301075268817204
931 0.0300751879699248
932 0.0300429184549356
933 0.030010718113612
934 0.0299785867237687
935 0.0299465240641711
936 0.0299145299145299
937 0.0298826040554963
938 0.0309168443496802
939 0.0308839190628328
940 0.0308510638297872
941 0.0308182784272051
942 0.0307855626326964
943 0.0307529162248144
944 0.0307203389830508
945 0.0306878306878307
946 0.0306553911205074
947 0.030623020063358
948 0.0305907172995781
949 0.0305584826132771
950 0.0305263157894737
951 0.0304942166140904
952 0.0304621848739496
953 0.0304302203567681
954 0.030398322851153
955 0.0303664921465969
956 0.0303347280334728
957 0.0303030303030303
958 0.0302713987473904
959 0.0302398331595412
960 0.0302083333333333
961 0.0301768990634755
962 0.0301455301455301
963 0.0301142263759086
964 0.0300829875518672
965 0.0300518134715026
966 0.0300207039337474
967 0.0299896587383661
968 0.0299586776859504
969 0.0299277605779154
970 0.0298969072164948
971 0.0298661174047374
972 0.0298353909465021
973 0.0298047276464543
974 0.0297741273100616
975 0.0297435897435897
976 0.0297131147540984
977 0.0296827021494371
978 0.0296523517382413
979 0.0296220633299285
980 0.0295918367346939
981 0.0295616717635066
982 0.0295315682281059
983 0.0295015259409969
984 0.0294715447154472
985 0.0294416243654822
986 0.0294117647058824
987 0.0293819655521783
988 0.0293522267206478
989 0.0293225480283114
990 0.0292929292929293
991 0.029263370332997
992 0.0292338709677419
993 0.0292044310171198
994 0.0291750503018109
995 0.0291457286432161
996 0.0291164658634538
997 0.0290872617853561
998 0.0290581162324649
999 0.029029029029029
1000 0.029
};
\addlegendentry{$f_{r}$ (frecuencia relativa de $18$)}
\addplot [semithick, color1]
table {%
1 0.027027027027027
2 0.027027027027027
3 0.027027027027027
4 0.027027027027027
5 0.027027027027027
6 0.027027027027027
7 0.027027027027027
8 0.027027027027027
9 0.027027027027027
10 0.027027027027027
11 0.027027027027027
12 0.027027027027027
13 0.027027027027027
14 0.027027027027027
15 0.027027027027027
16 0.027027027027027
17 0.027027027027027
18 0.027027027027027
19 0.027027027027027
20 0.027027027027027
21 0.027027027027027
22 0.027027027027027
23 0.027027027027027
24 0.027027027027027
25 0.027027027027027
26 0.027027027027027
27 0.027027027027027
28 0.027027027027027
29 0.027027027027027
30 0.027027027027027
31 0.027027027027027
32 0.027027027027027
33 0.027027027027027
34 0.027027027027027
35 0.027027027027027
36 0.027027027027027
37 0.027027027027027
38 0.027027027027027
39 0.027027027027027
40 0.027027027027027
41 0.027027027027027
42 0.027027027027027
43 0.027027027027027
44 0.027027027027027
45 0.027027027027027
46 0.027027027027027
47 0.027027027027027
48 0.027027027027027
49 0.027027027027027
50 0.027027027027027
51 0.027027027027027
52 0.027027027027027
53 0.027027027027027
54 0.027027027027027
55 0.027027027027027
56 0.027027027027027
57 0.027027027027027
58 0.027027027027027
59 0.027027027027027
60 0.027027027027027
61 0.027027027027027
62 0.027027027027027
63 0.027027027027027
64 0.027027027027027
65 0.027027027027027
66 0.027027027027027
67 0.027027027027027
68 0.027027027027027
69 0.027027027027027
70 0.027027027027027
71 0.027027027027027
72 0.027027027027027
73 0.027027027027027
74 0.027027027027027
75 0.027027027027027
76 0.027027027027027
77 0.027027027027027
78 0.027027027027027
79 0.027027027027027
80 0.027027027027027
81 0.027027027027027
82 0.027027027027027
83 0.027027027027027
84 0.027027027027027
85 0.027027027027027
86 0.027027027027027
87 0.027027027027027
88 0.027027027027027
89 0.027027027027027
90 0.027027027027027
91 0.027027027027027
92 0.027027027027027
93 0.027027027027027
94 0.027027027027027
95 0.027027027027027
96 0.027027027027027
97 0.027027027027027
98 0.027027027027027
99 0.027027027027027
100 0.027027027027027
101 0.027027027027027
102 0.027027027027027
103 0.027027027027027
104 0.027027027027027
105 0.027027027027027
106 0.027027027027027
107 0.027027027027027
108 0.027027027027027
109 0.027027027027027
110 0.027027027027027
111 0.027027027027027
112 0.027027027027027
113 0.027027027027027
114 0.027027027027027
115 0.027027027027027
116 0.027027027027027
117 0.027027027027027
118 0.027027027027027
119 0.027027027027027
120 0.027027027027027
121 0.027027027027027
122 0.027027027027027
123 0.027027027027027
124 0.027027027027027
125 0.027027027027027
126 0.027027027027027
127 0.027027027027027
128 0.027027027027027
129 0.027027027027027
130 0.027027027027027
131 0.027027027027027
132 0.027027027027027
133 0.027027027027027
134 0.027027027027027
135 0.027027027027027
136 0.027027027027027
137 0.027027027027027
138 0.027027027027027
139 0.027027027027027
140 0.027027027027027
141 0.027027027027027
142 0.027027027027027
143 0.027027027027027
144 0.027027027027027
145 0.027027027027027
146 0.027027027027027
147 0.027027027027027
148 0.027027027027027
149 0.027027027027027
150 0.027027027027027
151 0.027027027027027
152 0.027027027027027
153 0.027027027027027
154 0.027027027027027
155 0.027027027027027
156 0.027027027027027
157 0.027027027027027
158 0.027027027027027
159 0.027027027027027
160 0.027027027027027
161 0.027027027027027
162 0.027027027027027
163 0.027027027027027
164 0.027027027027027
165 0.027027027027027
166 0.027027027027027
167 0.027027027027027
168 0.027027027027027
169 0.027027027027027
170 0.027027027027027
171 0.027027027027027
172 0.027027027027027
173 0.027027027027027
174 0.027027027027027
175 0.027027027027027
176 0.027027027027027
177 0.027027027027027
178 0.027027027027027
179 0.027027027027027
180 0.027027027027027
181 0.027027027027027
182 0.027027027027027
183 0.027027027027027
184 0.027027027027027
185 0.027027027027027
186 0.027027027027027
187 0.027027027027027
188 0.027027027027027
189 0.027027027027027
190 0.027027027027027
191 0.027027027027027
192 0.027027027027027
193 0.027027027027027
194 0.027027027027027
195 0.027027027027027
196 0.027027027027027
197 0.027027027027027
198 0.027027027027027
199 0.027027027027027
200 0.027027027027027
201 0.027027027027027
202 0.027027027027027
203 0.027027027027027
204 0.027027027027027
205 0.027027027027027
206 0.027027027027027
207 0.027027027027027
208 0.027027027027027
209 0.027027027027027
210 0.027027027027027
211 0.027027027027027
212 0.027027027027027
213 0.027027027027027
214 0.027027027027027
215 0.027027027027027
216 0.027027027027027
217 0.027027027027027
218 0.027027027027027
219 0.027027027027027
220 0.027027027027027
221 0.027027027027027
222 0.027027027027027
223 0.027027027027027
224 0.027027027027027
225 0.027027027027027
226 0.027027027027027
227 0.027027027027027
228 0.027027027027027
229 0.027027027027027
230 0.027027027027027
231 0.027027027027027
232 0.027027027027027
233 0.027027027027027
234 0.027027027027027
235 0.027027027027027
236 0.027027027027027
237 0.027027027027027
238 0.027027027027027
239 0.027027027027027
240 0.027027027027027
241 0.027027027027027
242 0.027027027027027
243 0.027027027027027
244 0.027027027027027
245 0.027027027027027
246 0.027027027027027
247 0.027027027027027
248 0.027027027027027
249 0.027027027027027
250 0.027027027027027
251 0.027027027027027
252 0.027027027027027
253 0.027027027027027
254 0.027027027027027
255 0.027027027027027
256 0.027027027027027
257 0.027027027027027
258 0.027027027027027
259 0.027027027027027
260 0.027027027027027
261 0.027027027027027
262 0.027027027027027
263 0.027027027027027
264 0.027027027027027
265 0.027027027027027
266 0.027027027027027
267 0.027027027027027
268 0.027027027027027
269 0.027027027027027
270 0.027027027027027
271 0.027027027027027
272 0.027027027027027
273 0.027027027027027
274 0.027027027027027
275 0.027027027027027
276 0.027027027027027
277 0.027027027027027
278 0.027027027027027
279 0.027027027027027
280 0.027027027027027
281 0.027027027027027
282 0.027027027027027
283 0.027027027027027
284 0.027027027027027
285 0.027027027027027
286 0.027027027027027
287 0.027027027027027
288 0.027027027027027
289 0.027027027027027
290 0.027027027027027
291 0.027027027027027
292 0.027027027027027
293 0.027027027027027
294 0.027027027027027
295 0.027027027027027
296 0.027027027027027
297 0.027027027027027
298 0.027027027027027
299 0.027027027027027
300 0.027027027027027
301 0.027027027027027
302 0.027027027027027
303 0.027027027027027
304 0.027027027027027
305 0.027027027027027
306 0.027027027027027
307 0.027027027027027
308 0.027027027027027
309 0.027027027027027
310 0.027027027027027
311 0.027027027027027
312 0.027027027027027
313 0.027027027027027
314 0.027027027027027
315 0.027027027027027
316 0.027027027027027
317 0.027027027027027
318 0.027027027027027
319 0.027027027027027
320 0.027027027027027
321 0.027027027027027
322 0.027027027027027
323 0.027027027027027
324 0.027027027027027
325 0.027027027027027
326 0.027027027027027
327 0.027027027027027
328 0.027027027027027
329 0.027027027027027
330 0.027027027027027
331 0.027027027027027
332 0.027027027027027
333 0.027027027027027
334 0.027027027027027
335 0.027027027027027
336 0.027027027027027
337 0.027027027027027
338 0.027027027027027
339 0.027027027027027
340 0.027027027027027
341 0.027027027027027
342 0.027027027027027
343 0.027027027027027
344 0.027027027027027
345 0.027027027027027
346 0.027027027027027
347 0.027027027027027
348 0.027027027027027
349 0.027027027027027
350 0.027027027027027
351 0.027027027027027
352 0.027027027027027
353 0.027027027027027
354 0.027027027027027
355 0.027027027027027
356 0.027027027027027
357 0.027027027027027
358 0.027027027027027
359 0.027027027027027
360 0.027027027027027
361 0.027027027027027
362 0.027027027027027
363 0.027027027027027
364 0.027027027027027
365 0.027027027027027
366 0.027027027027027
367 0.027027027027027
368 0.027027027027027
369 0.027027027027027
370 0.027027027027027
371 0.027027027027027
372 0.027027027027027
373 0.027027027027027
374 0.027027027027027
375 0.027027027027027
376 0.027027027027027
377 0.027027027027027
378 0.027027027027027
379 0.027027027027027
380 0.027027027027027
381 0.027027027027027
382 0.027027027027027
383 0.027027027027027
384 0.027027027027027
385 0.027027027027027
386 0.027027027027027
387 0.027027027027027
388 0.027027027027027
389 0.027027027027027
390 0.027027027027027
391 0.027027027027027
392 0.027027027027027
393 0.027027027027027
394 0.027027027027027
395 0.027027027027027
396 0.027027027027027
397 0.027027027027027
398 0.027027027027027
399 0.027027027027027
400 0.027027027027027
401 0.027027027027027
402 0.027027027027027
403 0.027027027027027
404 0.027027027027027
405 0.027027027027027
406 0.027027027027027
407 0.027027027027027
408 0.027027027027027
409 0.027027027027027
410 0.027027027027027
411 0.027027027027027
412 0.027027027027027
413 0.027027027027027
414 0.027027027027027
415 0.027027027027027
416 0.027027027027027
417 0.027027027027027
418 0.027027027027027
419 0.027027027027027
420 0.027027027027027
421 0.027027027027027
422 0.027027027027027
423 0.027027027027027
424 0.027027027027027
425 0.027027027027027
426 0.027027027027027
427 0.027027027027027
428 0.027027027027027
429 0.027027027027027
430 0.027027027027027
431 0.027027027027027
432 0.027027027027027
433 0.027027027027027
434 0.027027027027027
435 0.027027027027027
436 0.027027027027027
437 0.027027027027027
438 0.027027027027027
439 0.027027027027027
440 0.027027027027027
441 0.027027027027027
442 0.027027027027027
443 0.027027027027027
444 0.027027027027027
445 0.027027027027027
446 0.027027027027027
447 0.027027027027027
448 0.027027027027027
449 0.027027027027027
450 0.027027027027027
451 0.027027027027027
452 0.027027027027027
453 0.027027027027027
454 0.027027027027027
455 0.027027027027027
456 0.027027027027027
457 0.027027027027027
458 0.027027027027027
459 0.027027027027027
460 0.027027027027027
461 0.027027027027027
462 0.027027027027027
463 0.027027027027027
464 0.027027027027027
465 0.027027027027027
466 0.027027027027027
467 0.027027027027027
468 0.027027027027027
469 0.027027027027027
470 0.027027027027027
471 0.027027027027027
472 0.027027027027027
473 0.027027027027027
474 0.027027027027027
475 0.027027027027027
476 0.027027027027027
477 0.027027027027027
478 0.027027027027027
479 0.027027027027027
480 0.027027027027027
481 0.027027027027027
482 0.027027027027027
483 0.027027027027027
484 0.027027027027027
485 0.027027027027027
486 0.027027027027027
487 0.027027027027027
488 0.027027027027027
489 0.027027027027027
490 0.027027027027027
491 0.027027027027027
492 0.027027027027027
493 0.027027027027027
494 0.027027027027027
495 0.027027027027027
496 0.027027027027027
497 0.027027027027027
498 0.027027027027027
499 0.027027027027027
500 0.027027027027027
501 0.027027027027027
502 0.027027027027027
503 0.027027027027027
504 0.027027027027027
505 0.027027027027027
506 0.027027027027027
507 0.027027027027027
508 0.027027027027027
509 0.027027027027027
510 0.027027027027027
511 0.027027027027027
512 0.027027027027027
513 0.027027027027027
514 0.027027027027027
515 0.027027027027027
516 0.027027027027027
517 0.027027027027027
518 0.027027027027027
519 0.027027027027027
520 0.027027027027027
521 0.027027027027027
522 0.027027027027027
523 0.027027027027027
524 0.027027027027027
525 0.027027027027027
526 0.027027027027027
527 0.027027027027027
528 0.027027027027027
529 0.027027027027027
530 0.027027027027027
531 0.027027027027027
532 0.027027027027027
533 0.027027027027027
534 0.027027027027027
535 0.027027027027027
536 0.027027027027027
537 0.027027027027027
538 0.027027027027027
539 0.027027027027027
540 0.027027027027027
541 0.027027027027027
542 0.027027027027027
543 0.027027027027027
544 0.027027027027027
545 0.027027027027027
546 0.027027027027027
547 0.027027027027027
548 0.027027027027027
549 0.027027027027027
550 0.027027027027027
551 0.027027027027027
552 0.027027027027027
553 0.027027027027027
554 0.027027027027027
555 0.027027027027027
556 0.027027027027027
557 0.027027027027027
558 0.027027027027027
559 0.027027027027027
560 0.027027027027027
561 0.027027027027027
562 0.027027027027027
563 0.027027027027027
564 0.027027027027027
565 0.027027027027027
566 0.027027027027027
567 0.027027027027027
568 0.027027027027027
569 0.027027027027027
570 0.027027027027027
571 0.027027027027027
572 0.027027027027027
573 0.027027027027027
574 0.027027027027027
575 0.027027027027027
576 0.027027027027027
577 0.027027027027027
578 0.027027027027027
579 0.027027027027027
580 0.027027027027027
581 0.027027027027027
582 0.027027027027027
583 0.027027027027027
584 0.027027027027027
585 0.027027027027027
586 0.027027027027027
587 0.027027027027027
588 0.027027027027027
589 0.027027027027027
590 0.027027027027027
591 0.027027027027027
592 0.027027027027027
593 0.027027027027027
594 0.027027027027027
595 0.027027027027027
596 0.027027027027027
597 0.027027027027027
598 0.027027027027027
599 0.027027027027027
600 0.027027027027027
601 0.027027027027027
602 0.027027027027027
603 0.027027027027027
604 0.027027027027027
605 0.027027027027027
606 0.027027027027027
607 0.027027027027027
608 0.027027027027027
609 0.027027027027027
610 0.027027027027027
611 0.027027027027027
612 0.027027027027027
613 0.027027027027027
614 0.027027027027027
615 0.027027027027027
616 0.027027027027027
617 0.027027027027027
618 0.027027027027027
619 0.027027027027027
620 0.027027027027027
621 0.027027027027027
622 0.027027027027027
623 0.027027027027027
624 0.027027027027027
625 0.027027027027027
626 0.027027027027027
627 0.027027027027027
628 0.027027027027027
629 0.027027027027027
630 0.027027027027027
631 0.027027027027027
632 0.027027027027027
633 0.027027027027027
634 0.027027027027027
635 0.027027027027027
636 0.027027027027027
637 0.027027027027027
638 0.027027027027027
639 0.027027027027027
640 0.027027027027027
641 0.027027027027027
642 0.027027027027027
643 0.027027027027027
644 0.027027027027027
645 0.027027027027027
646 0.027027027027027
647 0.027027027027027
648 0.027027027027027
649 0.027027027027027
650 0.027027027027027
651 0.027027027027027
652 0.027027027027027
653 0.027027027027027
654 0.027027027027027
655 0.027027027027027
656 0.027027027027027
657 0.027027027027027
658 0.027027027027027
659 0.027027027027027
660 0.027027027027027
661 0.027027027027027
662 0.027027027027027
663 0.027027027027027
664 0.027027027027027
665 0.027027027027027
666 0.027027027027027
667 0.027027027027027
668 0.027027027027027
669 0.027027027027027
670 0.027027027027027
671 0.027027027027027
672 0.027027027027027
673 0.027027027027027
674 0.027027027027027
675 0.027027027027027
676 0.027027027027027
677 0.027027027027027
678 0.027027027027027
679 0.027027027027027
680 0.027027027027027
681 0.027027027027027
682 0.027027027027027
683 0.027027027027027
684 0.027027027027027
685 0.027027027027027
686 0.027027027027027
687 0.027027027027027
688 0.027027027027027
689 0.027027027027027
690 0.027027027027027
691 0.027027027027027
692 0.027027027027027
693 0.027027027027027
694 0.027027027027027
695 0.027027027027027
696 0.027027027027027
697 0.027027027027027
698 0.027027027027027
699 0.027027027027027
700 0.027027027027027
701 0.027027027027027
702 0.027027027027027
703 0.027027027027027
704 0.027027027027027
705 0.027027027027027
706 0.027027027027027
707 0.027027027027027
708 0.027027027027027
709 0.027027027027027
710 0.027027027027027
711 0.027027027027027
712 0.027027027027027
713 0.027027027027027
714 0.027027027027027
715 0.027027027027027
716 0.027027027027027
717 0.027027027027027
718 0.027027027027027
719 0.027027027027027
720 0.027027027027027
721 0.027027027027027
722 0.027027027027027
723 0.027027027027027
724 0.027027027027027
725 0.027027027027027
726 0.027027027027027
727 0.027027027027027
728 0.027027027027027
729 0.027027027027027
730 0.027027027027027
731 0.027027027027027
732 0.027027027027027
733 0.027027027027027
734 0.027027027027027
735 0.027027027027027
736 0.027027027027027
737 0.027027027027027
738 0.027027027027027
739 0.027027027027027
740 0.027027027027027
741 0.027027027027027
742 0.027027027027027
743 0.027027027027027
744 0.027027027027027
745 0.027027027027027
746 0.027027027027027
747 0.027027027027027
748 0.027027027027027
749 0.027027027027027
750 0.027027027027027
751 0.027027027027027
752 0.027027027027027
753 0.027027027027027
754 0.027027027027027
755 0.027027027027027
756 0.027027027027027
757 0.027027027027027
758 0.027027027027027
759 0.027027027027027
760 0.027027027027027
761 0.027027027027027
762 0.027027027027027
763 0.027027027027027
764 0.027027027027027
765 0.027027027027027
766 0.027027027027027
767 0.027027027027027
768 0.027027027027027
769 0.027027027027027
770 0.027027027027027
771 0.027027027027027
772 0.027027027027027
773 0.027027027027027
774 0.027027027027027
775 0.027027027027027
776 0.027027027027027
777 0.027027027027027
778 0.027027027027027
779 0.027027027027027
780 0.027027027027027
781 0.027027027027027
782 0.027027027027027
783 0.027027027027027
784 0.027027027027027
785 0.027027027027027
786 0.027027027027027
787 0.027027027027027
788 0.027027027027027
789 0.027027027027027
790 0.027027027027027
791 0.027027027027027
792 0.027027027027027
793 0.027027027027027
794 0.027027027027027
795 0.027027027027027
796 0.027027027027027
797 0.027027027027027
798 0.027027027027027
799 0.027027027027027
800 0.027027027027027
801 0.027027027027027
802 0.027027027027027
803 0.027027027027027
804 0.027027027027027
805 0.027027027027027
806 0.027027027027027
807 0.027027027027027
808 0.027027027027027
809 0.027027027027027
810 0.027027027027027
811 0.027027027027027
812 0.027027027027027
813 0.027027027027027
814 0.027027027027027
815 0.027027027027027
816 0.027027027027027
817 0.027027027027027
818 0.027027027027027
819 0.027027027027027
820 0.027027027027027
821 0.027027027027027
822 0.027027027027027
823 0.027027027027027
824 0.027027027027027
825 0.027027027027027
826 0.027027027027027
827 0.027027027027027
828 0.027027027027027
829 0.027027027027027
830 0.027027027027027
831 0.027027027027027
832 0.027027027027027
833 0.027027027027027
834 0.027027027027027
835 0.027027027027027
836 0.027027027027027
837 0.027027027027027
838 0.027027027027027
839 0.027027027027027
840 0.027027027027027
841 0.027027027027027
842 0.027027027027027
843 0.027027027027027
844 0.027027027027027
845 0.027027027027027
846 0.027027027027027
847 0.027027027027027
848 0.027027027027027
849 0.027027027027027
850 0.027027027027027
851 0.027027027027027
852 0.027027027027027
853 0.027027027027027
854 0.027027027027027
855 0.027027027027027
856 0.027027027027027
857 0.027027027027027
858 0.027027027027027
859 0.027027027027027
860 0.027027027027027
861 0.027027027027027
862 0.027027027027027
863 0.027027027027027
864 0.027027027027027
865 0.027027027027027
866 0.027027027027027
867 0.027027027027027
868 0.027027027027027
869 0.027027027027027
870 0.027027027027027
871 0.027027027027027
872 0.027027027027027
873 0.027027027027027
874 0.027027027027027
875 0.027027027027027
876 0.027027027027027
877 0.027027027027027
878 0.027027027027027
879 0.027027027027027
880 0.027027027027027
881 0.027027027027027
882 0.027027027027027
883 0.027027027027027
884 0.027027027027027
885 0.027027027027027
886 0.027027027027027
887 0.027027027027027
888 0.027027027027027
889 0.027027027027027
890 0.027027027027027
891 0.027027027027027
892 0.027027027027027
893 0.027027027027027
894 0.027027027027027
895 0.027027027027027
896 0.027027027027027
897 0.027027027027027
898 0.027027027027027
899 0.027027027027027
900 0.027027027027027
901 0.027027027027027
902 0.027027027027027
903 0.027027027027027
904 0.027027027027027
905 0.027027027027027
906 0.027027027027027
907 0.027027027027027
908 0.027027027027027
909 0.027027027027027
910 0.027027027027027
911 0.027027027027027
912 0.027027027027027
913 0.027027027027027
914 0.027027027027027
915 0.027027027027027
916 0.027027027027027
917 0.027027027027027
918 0.027027027027027
919 0.027027027027027
920 0.027027027027027
921 0.027027027027027
922 0.027027027027027
923 0.027027027027027
924 0.027027027027027
925 0.027027027027027
926 0.027027027027027
927 0.027027027027027
928 0.027027027027027
929 0.027027027027027
930 0.027027027027027
931 0.027027027027027
932 0.027027027027027
933 0.027027027027027
934 0.027027027027027
935 0.027027027027027
936 0.027027027027027
937 0.027027027027027
938 0.027027027027027
939 0.027027027027027
940 0.027027027027027
941 0.027027027027027
942 0.027027027027027
943 0.027027027027027
944 0.027027027027027
945 0.027027027027027
946 0.027027027027027
947 0.027027027027027
948 0.027027027027027
949 0.027027027027027
950 0.027027027027027
951 0.027027027027027
952 0.027027027027027
953 0.027027027027027
954 0.027027027027027
955 0.027027027027027
956 0.027027027027027
957 0.027027027027027
958 0.027027027027027
959 0.027027027027027
960 0.027027027027027
961 0.027027027027027
962 0.027027027027027
963 0.027027027027027
964 0.027027027027027
965 0.027027027027027
966 0.027027027027027
967 0.027027027027027
968 0.027027027027027
969 0.027027027027027
970 0.027027027027027
971 0.027027027027027
972 0.027027027027027
973 0.027027027027027
974 0.027027027027027
975 0.027027027027027
976 0.027027027027027
977 0.027027027027027
978 0.027027027027027
979 0.027027027027027
980 0.027027027027027
981 0.027027027027027
982 0.027027027027027
983 0.027027027027027
984 0.027027027027027
985 0.027027027027027
986 0.027027027027027
987 0.027027027027027
988 0.027027027027027
989 0.027027027027027
990 0.027027027027027
991 0.027027027027027
992 0.027027027027027
993 0.027027027027027
994 0.027027027027027
995 0.027027027027027
996 0.027027027027027
997 0.027027027027027
998 0.027027027027027
999 0.027027027027027
1000 0.027027027027027
};
\addlegendentry{$f_{r_{e}}$ (frecuencia relativa esperada de $18$)}
\end{axis}

\end{tikzpicture}

    \caption{frecuencia relativa con respecto al número de tiradas}
  \end{mytikzresize}
\end{figure}

\begin{figure}[!htbp]
  \begin{mytikzresize}{0.6\textwidth}
    \centering
    % This file was created by tikzplotlib v0.9.1.
\begin{tikzpicture}

\definecolor{color0}{rgb}{0.12156862745098,0.466666666666667,0.705882352941177}
\definecolor{color1}{rgb}{1,0.498039215686275,0.0549019607843137}

\begin{axis}[
legend cell align={left},
legend style={fill opacity=0.5, draw opacity=1, text opacity=1, draw=white!80!black},
scaled ticks=false,
tick align=outside,
tick pos=left,
width=\figW,
x grid style={white!69.0196078431373!black},
xlabel={\(\displaystyle n\) (número de tiradas)},
xmajorgrids,
xmin=-48.95, xmax=1049.95,
xtick style={color=black},
xticklabel style={/pgf/number format/.cd,fixed,precision=2},
y grid style={white!69.0196078431373!black},
ylabel={\(\displaystyle v_{p}\) (valor promedio)},
ymajorgrids,
ymin=17.25, ymax=33.75,
ytick style={color=black},
yticklabel style={/pgf/number format/.cd,fixed,precision=2}
]
\addplot [semithick, color0]
table {%
1 33
2 19
3 23.6666666666667
4 26.5
5 24.2
6 21.5
7 21.4285714285714
8 22.5
9 23.8888888888889
10 25.1
11 25.7272727272727
12 25.8333333333333
13 25.6153846153846
14 25.7857142857143
15 25.0666666666667
16 25.5
17 24.0588235294118
18 24.5
19 23.2105263157895
20 23.8
21 22.6666666666667
22 22.2727272727273
23 22.5652173913043
24 22.875
25 22.12
26 22.3846153846154
27 22
28 21.3571428571429
29 21.8275862068966
30 21.1333333333333
31 21.5161290322581
32 21.375
33 20.8484848484848
34 20.8529411764706
35 21.0285714285714
36 20.8055555555556
37 20.4864864864865
38 20.1052631578947
39 20.0769230769231
40 20.05
41 19.9024390243902
42 20.2142857142857
43 19.7441860465116
44 20
45 20.2222222222222
46 20.3913043478261
47 20.3191489361702
48 20.3958333333333
49 20.0204081632653
50 19.74
51 19.8039215686275
52 19.7307692307692
53 19.377358490566
54 19.3148148148148
55 19.4
56 19.5
57 19.7017543859649
58 19.6724137931034
59 19.5084745762712
60 19.7833333333333
61 19.7704918032787
62 19.5645161290323
63 19.3015873015873
64 19.03125
65 19.0923076923077
66 19.0757575757576
67 19.044776119403
68 18.8676470588235
69 18.6086956521739
70 18.3428571428571
71 18.169014084507
72 18.4166666666667
73 18.2191780821918
74 18.3108108108108
75 18.4533333333333
76 18.6447368421053
77 18.5584415584416
78 18.7051282051282
79 18.6962025316456
80 18.875
81 18.7283950617284
82 18.8658536585366
83 18.7228915662651
84 18.75
85 18.5529411764706
86 18.4418604651163
87 18.5632183908046
88 18.4090909090909
89 18.3707865168539
90 18.5333333333333
91 18.6703296703297
92 18.7065217391304
93 18.752688172043
94 18.6489361702128
95 18.5052631578947
96 18.6875
97 18.6701030927835
98 18.6326530612245
99 18.7979797979798
100 18.67
101 18.6732673267327
102 18.5490196078431
103 18.6699029126214
104 18.7019230769231
105 18.8190476190476
106 18.8679245283019
107 18.9626168224299
108 18.8518518518519
109 18.9633027522936
110 18.9272727272727
111 18.972972972973
112 19.0982142857143
113 19.0088495575221
114 18.8421052631579
115 18.9304347826087
116 18.9051724137931
117 18.7521367521368
118 18.8983050847458
119 18.9495798319328
120 19.0583333333333
121 19.0247933884298
122 18.9590163934426
123 18.9430894308943
124 18.7983870967742
125 18.688
126 18.8015873015873
127 18.7007874015748
128 18.71875
129 18.6589147286822
130 18.6307692307692
131 18.5114503816794
132 18.6060606060606
133 18.7218045112782
134 18.6044776119403
135 18.5925925925926
136 18.5735294117647
137 18.5620437956204
138 18.6521739130435
139 18.589928057554
140 18.4714285714286
141 18.468085106383
142 18.5352112676056
143 18.4685314685315
144 18.5763888888889
145 18.5931034482759
146 18.6164383561644
147 18.6802721088435
148 18.7837837837838
149 18.8053691275168
150 18.74
151 18.6158940397351
152 18.6315789473684
153 18.5098039215686
154 18.4545454545455
155 18.4129032258065
156 18.3205128205128
157 18.3630573248408
158 18.4430379746835
159 18.5471698113208
160 18.625
161 18.6459627329193
162 18.7160493827161
163 18.7484662576687
164 18.7134146341463
165 18.8121212121212
166 18.7048192771084
167 18.6287425149701
168 18.6607142857143
169 18.7396449704142
170 18.8
171 18.7836257309942
172 18.7732558139535
173 18.7341040462428
174 18.8103448275862
175 18.7142857142857
176 18.6477272727273
177 18.6949152542373
178 18.7247191011236
179 18.6927374301676
180 18.7888888888889
181 18.8342541436464
182 18.7362637362637
183 18.7267759562842
184 18.7880434782609
185 18.7675675675676
186 18.6720430107527
187 18.620320855615
188 18.6382978723404
189 18.7089947089947
190 18.7789473684211
191 18.8534031413613
192 18.828125
193 18.880829015544
194 18.7989690721649
195 18.8461538461538
196 18.9183673469388
197 18.9187817258883
198 18.8282828282828
199 18.8743718592965
200 18.935
201 19.0099502487562
202 18.9455445544554
203 18.9310344827586
204 19.0049019607843
205 19
206 19.0485436893204
207 19.0531400966184
208 19.0096153846154
209 19.0526315789474
210 19.0571428571429
211 19.0331753554502
212 19.0566037735849
213 19.0469483568075
214 18.9953271028037
215 18.9953488372093
216 19.0462962962963
217 18.9769585253456
218 18.954128440367
219 18.9543378995434
220 19.0227272727273
221 18.9864253393665
222 19.036036036036
223 18.9775784753363
224 18.9107142857143
225 18.9555555555556
226 18.9159292035398
227 18.8898678414097
228 18.9166666666667
229 18.9694323144105
230 18.9826086956522
231 19.04329004329
232 19
233 19.0557939914163
234 19.0128205128205
235 19.0297872340426
236 18.9576271186441
237 18.9071729957806
238 18.844537815126
239 18.8493723849372
240 18.8208333333333
241 18.7427385892116
242 18.7066115702479
243 18.7530864197531
244 18.7418032786885
245 18.7877551020408
246 18.7357723577236
247 18.6963562753036
248 18.7137096774194
249 18.7550200803213
250 18.808
251 18.7729083665339
252 18.797619047619
253 18.8181818181818
254 18.8307086614173
255 18.8705882352941
256 18.921875
257 18.8482490272374
258 18.8333333333333
259 18.8455598455598
260 18.8846153846154
261 18.8390804597701
262 18.9007633587786
263 18.893536121673
264 18.8977272727273
265 18.9094339622641
266 18.9398496240602
267 18.9213483146067
268 18.8917910447761
269 18.8847583643123
270 18.8814814814815
271 18.8634686346863
272 18.9007352941176
273 18.8534798534799
274 18.8321167883212
275 18.7709090909091
276 18.7065217391304
277 18.7111913357401
278 18.6438848920863
279 18.6917562724014
280 18.625
281 18.5800711743772
282 18.5673758865248
283 18.6113074204947
284 18.5669014084507
285 18.5087719298246
286 18.541958041958
287 18.5679442508711
288 18.5069444444444
289 18.4602076124567
290 18.5206896551724
291 18.4776632302405
292 18.4212328767123
293 18.4402730375427
294 18.4863945578231
295 18.4542372881356
296 18.3918918918919
297 18.3434343434343
298 18.3355704697987
299 18.3311036789298
300 18.3666666666667
301 18.3488372093023
302 18.3609271523179
303 18.3663366336634
304 18.3322368421053
305 18.3770491803279
306 18.4150326797386
307 18.3973941368078
308 18.4415584415584
309 18.3948220064725
310 18.3548387096774
311 18.3890675241158
312 18.3685897435897
313 18.3226837060703
314 18.328025477707
315 18.2920634920635
316 18.2721518987342
317 18.2996845425867
318 18.3490566037736
319 18.4012539184953
320 18.45
321 18.398753894081
322 18.4254658385093
323 18.4365325077399
324 18.4845679012346
325 18.48
326 18.4233128834356
327 18.4250764525994
328 18.4329268292683
329 18.4589665653495
330 18.4787878787879
331 18.4743202416918
332 18.4246987951807
333 18.4534534534535
334 18.502994011976
335 18.5283582089552
336 18.5654761904762
337 18.5964391691395
338 18.6449704142012
339 18.622418879056
340 18.5705882352941
341 18.524926686217
342 18.5233918128655
343 18.5247813411079
344 18.5406976744186
345 18.5913043478261
346 18.5375722543353
347 18.5389048991354
348 18.4885057471264
349 18.4383954154728
350 18.4342857142857
351 18.4700854700855
352 18.46875
353 18.4249291784703
354 18.4519774011299
355 18.4704225352113
356 18.4522471910112
357 18.406162464986
358 18.3966480446927
359 18.3509749303621
360 18.3694444444444
361 18.3961218836565
362 18.4254143646409
363 18.4297520661157
364 18.4752747252747
365 18.4712328767123
366 18.4371584699454
367 18.4359673024523
368 18.3885869565217
369 18.4227642276423
370 18.3756756756757
371 18.3477088948787
372 18.2983870967742
373 18.2761394101877
374 18.3155080213904
375 18.3173333333333
376 18.3457446808511
377 18.3580901856764
378 18.3439153439153
379 18.3060686015831
380 18.3421052631579
381 18.3070866141732
382 18.3350785340314
383 18.3812010443864
384 18.3958333333333
385 18.3532467532468
386 18.3082901554404
387 18.328165374677
388 18.3453608247423
389 18.3624678663239
390 18.3641025641026
391 18.3887468030691
392 18.4260204081633
393 18.4681933842239
394 18.4543147208122
395 18.4658227848101
396 18.5050505050505
397 18.4911838790932
398 18.4497487437186
399 18.468671679198
400 18.4625
401 18.4214463840399
402 18.3756218905473
403 18.3622828784119
404 18.3415841584158
405 18.3061728395062
406 18.3448275862069
407 18.3857493857494
408 18.4240196078431
409 18.4498777506112
410 18.4658536585366
411 18.4549878345499
412 18.4757281553398
413 18.4382566585956
414 18.4613526570048
415 18.4313253012048
416 18.40625
417 18.3717026378897
418 18.3468899521531
419 18.346062052506
420 18.3595238095238
421 18.3254156769596
422 18.3483412322275
423 18.3498817966903
424 18.3490566037736
425 18.3694117647059
426 18.3286384976526
427 18.3583138173302
428 18.3995327102804
429 18.3986013986014
430 18.4116279069767
431 18.445475638051
432 18.4861111111111
433 18.4480369515012
434 18.4654377880184
435 18.4758620689655
436 18.4977064220184
437 18.5102974828375
438 18.486301369863
439 18.4601366742597
440 18.4590909090909
441 18.4852607709751
442 18.4524886877828
443 18.451467268623
444 18.4864864864865
445 18.4516853932584
446 18.4439461883408
447 18.4765100671141
448 18.4709821428571
449 18.43429844098
450 18.3955555555556
451 18.4057649667406
452 18.3893805309735
453 18.3973509933775
454 18.4074889867841
455 18.3824175824176
456 18.3618421052632
457 18.3304157549234
458 18.3624454148472
459 18.3877995642702
460 18.3739130434783
461 18.3492407809111
462 18.3506493506494
463 18.3563714902808
464 18.364224137931
465 18.3311827956989
466 18.3347639484979
467 18.3447537473233
468 18.3482905982906
469 18.3347547974414
470 18.3255319148936
471 18.3163481953291
472 18.3072033898305
473 18.3086680761099
474 18.3291139240506
475 18.3242105263158
476 18.3172268907563
477 18.3333333333333
478 18.3556485355649
479 18.39248434238
480 18.4041666666667
481 18.4033264033264
482 18.3838174273859
483 18.3809523809524
484 18.3760330578512
485 18.3896907216495
486 18.40329218107
487 18.4127310061602
488 18.3893442622951
489 18.4212678936605
490 18.4061224489796
491 18.4052953156823
492 18.4308943089431
493 18.4320486815416
494 18.4008097165992
495 18.3919191919192
496 18.4173387096774
497 18.3843058350101
498 18.3855421686747
499 18.3667334669339
500 18.356
501 18.3832335329341
502 18.4123505976096
503 18.441351888668
504 18.4484126984127
505 18.4653465346535
506 18.4347826086957
507 18.4575936883629
508 18.4212598425197
509 18.4204322200393
510 18.4019607843137
511 18.4363992172211
512 18.41015625
513 18.4171539961014
514 18.4416342412451
515 18.4485436893204
516 18.4806201550388
517 18.5087040618955
518 18.5135135135135
519 18.4971098265896
520 18.5173076923077
521 18.5143953934741
522 18.5
523 18.4933078393881
524 18.5267175572519
525 18.527619047619
526 18.5152091254753
527 18.5313092979127
528 18.530303030303
529 18.5595463137996
530 18.5735849056604
531 18.5951035781544
532 18.5883458646617
533 18.5703564727955
534 18.5430711610487
535 18.5140186915888
536 18.5391791044776
537 18.5586592178771
538 18.5724907063197
539 18.595547309833
540 18.6259259259259
541 18.6192236598891
542 18.6217712177122
543 18.6187845303867
544 18.6011029411765
545 18.5981651376147
546 18.5732600732601
547 18.5575868372943
548 18.529197080292
549 18.4972677595628
550 18.4672727272727
551 18.4519056261343
552 18.4438405797101
553 18.4394213381555
554 18.4620938628159
555 18.4756756756757
556 18.4928057553957
557 18.5206463195691
558 18.5340501792115
559 18.5420393559928
560 18.5446428571429
561 18.5383244206774
562 18.5338078291815
563 18.5488454706927
564 18.5780141843972
565 18.553982300885
566 18.5742049469965
567 18.5502645502646
568 18.5316901408451
569 18.5008787346221
570 18.4719298245614
571 18.4658493870403
572 18.4615384615385
573 18.4432809773124
574 18.411149825784
575 18.4417391304348
576 18.4548611111111
577 18.4644714038128
578 18.439446366782
579 18.4697754749568
580 18.4862068965517
581 18.5060240963855
582 18.4879725085911
583 18.4957118353345
584 18.5222602739726
585 18.5435897435897
586 18.5443686006826
587 18.557069846678
588 18.547619047619
589 18.5398981324278
590 18.564406779661
591 18.5719120135364
592 18.5523648648649
593 18.5497470489039
594 18.5622895622896
595 18.5764705882353
596 18.5855704697987
597 18.6097152428811
598 18.5986622073579
599 18.5709515859766
600 18.5983333333333
601 18.5740432612313
602 18.5548172757475
603 18.5555555555556
604 18.5298013245033
605 18.5553719008264
606 18.5610561056106
607 18.5535420098847
608 18.5378289473684
609 18.5467980295567
610 18.5229508196721
611 18.5417348608838
612 18.5702614379085
613 18.5970636215334
614 18.586319218241
615 18.5869918699187
616 18.5941558441558
617 18.6029173419773
618 18.6181229773463
619 18.6187399030695
620 18.6290322580645
621 18.6505636070853
622 18.6414790996785
623 18.6356340288925
624 18.6634615384615
625 18.6624
626 18.6485623003195
627 18.6283891547049
628 18.6082802547771
629 18.6168521462639
630 18.6063492063492
631 18.5927099841521
632 18.5949367088608
633 18.5750394944708
634 18.5567823343849
635 18.5322834645669
636 18.5581761006289
637 18.5384615384615
638 18.5282131661442
639 18.5430359937402
640 18.5265625
641 18.5179407176287
642 18.5451713395639
643 18.5287713841369
644 18.527950310559
645 18.522480620155
646 18.5030959752322
647 18.5177743431221
648 18.5339506172839
649 18.5608628659476
650 18.5707692307692
651 18.5975422427035
652 18.5889570552147
653 18.5803981623277
654 18.5718654434251
655 18.5587786259542
656 18.5564024390244
657 18.5738203957382
658 18.5577507598784
659 18.5629742033384
660 18.55
661 18.5461422087746
662 18.5725075528701
663 18.5927601809955
664 18.6039156626506
665 18.6210526315789
666 18.6381381381381
667 18.6371814092954
668 18.6212574850299
669 18.6143497757848
670 18.6149253731343
671 18.5991058122206
672 18.6026785714286
673 18.6092124814265
674 18.5979228486647
675 18.597037037037
676 18.5695266272189
677 18.5480059084195
678 18.5221238938053
679 18.5346097201767
680 18.5132352941176
681 18.5051395007342
682 18.5234604105572
683 18.5051244509517
684 18.4985380116959
685 18.5138686131387
686 18.5262390670554
687 18.5065502183406
688 18.5145348837209
689 18.5268505079826
690 18.5478260869565
691 18.5600578871201
692 18.5404624277457
693 18.5425685425685
694 18.5403458213256
695 18.5352517985612
696 18.5244252873563
697 18.5164992826399
698 18.4957020057307
699 18.5064377682403
700 18.51
701 18.5335235378031
702 18.5128205128205
703 18.4950213371266
704 18.5
705 18.4936170212766
706 18.4957507082153
707 18.4893917963225
708 18.4915254237288
709 18.490832157969
710 18.4915492957746
711 18.5105485232068
712 18.5
713 18.5133239831697
714 18.5196078431373
715 18.5160839160839
716 18.5111731843575
717 18.5355648535565
718 18.5292479108635
719 18.547983310153
720 18.5708333333333
721 18.5492371705964
722 18.5734072022161
723 18.5532503457815
724 18.5345303867403
725 18.5268965517241
726 18.5440771349862
727 18.5378266850069
728 18.5260989010989
729 18.5089163237311
730 18.5150684931507
731 18.5075239398085
732 18.5081967213115
733 18.5034106412005
734 18.5013623978202
735 18.5034013605442
736 18.4972826086957
737 18.4776119402985
738 18.4620596205962
739 18.4803788903924
740 18.4675675675676
741 18.4777327935223
742 18.4636118598383
743 18.4401076716016
744 18.4596774193548
745 18.4778523489933
746 18.4973190348525
747 18.4832663989291
748 18.5
749 18.5233644859813
750 18.5253333333333
751 18.5019973368842
752 18.5066489361702
753 18.5073041168659
754 18.4840848806366
755 18.4609271523179
756 18.4391534391534
757 18.4332892998679
758 18.4248021108179
759 18.4281949934124
760 18.4197368421053
761 18.4047306176084
762 18.3818897637795
763 18.3997378768021
764 18.3926701570681
765 18.4078431372549
766 18.4177545691906
767 18.4289439374185
768 18.43359375
769 18.4161248374512
770 18.3935064935065
771 18.4098573281453
772 18.4132124352332
773 18.4359637774903
774 18.4586563307494
775 18.4812903225806
776 18.4948453608247
777 18.5006435006435
778 18.5205655526992
779 18.5160462130937
780 18.5115384615385
781 18.5326504481434
782 18.5191815856777
783 18.4955300127714
784 18.4808673469388
785 18.4815286624204
786 18.4580152671756
787 18.4548919949174
788 18.4441624365482
789 18.4423320659062
790 18.4582278481013
791 18.4551201011378
792 18.4444444444444
793 18.4287515762926
794 18.4420654911839
795 18.4327044025157
796 18.4158291457286
797 18.4065244667503
798 18.4022556390977
799 18.4180225281602
800 18.4
801 18.3895131086142
802 18.3852867830424
803 18.3860523038605
804 18.384328358209
805 18.3614906832298
806 18.3796526054591
807 18.3767038413879
808 18.3762376237624
809 18.3918417799753
810 18.3913580246914
811 18.3760789149199
812 18.3546798029557
813 18.3542435424354
814 18.3648648648649
815 18.3865030674847
816 18.4031862745098
817 18.4014687882497
818 18.3973105134474
819 18.4114774114774
820 18.4121951219512
821 18.4250913520097
822 18.4379562043796
823 18.4313487241798
824 18.4368932038835
825 18.4436363636364
826 18.455205811138
827 18.4425634824667
828 18.4408212560386
829 18.4463208685163
830 18.4566265060241
831 18.4657039711191
832 18.4831730769231
833 18.4861944777911
834 18.484412470024
835 18.4634730538922
836 18.4461722488038
837 18.4647550776583
838 18.4677804295943
839 18.4874851013111
840 18.502380952381
841 18.4910820451843
842 18.4762470308789
843 18.4768683274021
844 18.4976303317536
845 18.5076923076923
846 18.4988179669031
847 18.5053128689492
848 18.5094339622642
849 18.5135453474676
850 18.5176470588235
851 18.4994124559342
852 18.4906103286385
853 18.5087924970692
854 18.5222482435597
855 18.5146198830409
856 18.4976635514019
857 18.5157526254376
858 18.5314685314685
859 18.5261932479627
860 18.5453488372093
861 18.5447154471545
862 18.5406032482599
863 18.5596755504056
864 18.5601851851852
865 18.5526011560694
866 18.5450346420323
867 18.5340253748558
868 18.536866359447
869 18.5466052934407
870 18.5505747126437
871 18.5579793340987
872 18.5584862385321
873 18.5498281786942
874 18.558352402746
875 18.5485714285714
876 18.5296803652968
877 18.5381984036488
878 18.5261958997722
879 18.5392491467577
880 18.5340909090909
881 18.5255391600454
882 18.5351473922902
883 18.5424688561721
884 18.5542986425339
885 18.554802259887
886 18.538374717833
887 18.5231116121759
888 18.5225225225225
889 18.5129358830146
890 18.5101123595506
891 18.5230078563412
892 18.5179372197309
893 18.498320268757
894 18.4809843400447
895 18.4849162011173
896 18.4654017857143
897 18.4526198439242
898 18.4487750556793
899 18.4527252502781
900 18.4666666666667
901 18.4506104328524
902 18.4634146341463
903 18.4828349944629
904 18.4922566371681
905 18.5038674033149
906 18.4900662251656
907 18.4895259095921
908 18.4702643171806
909 18.4895489548955
910 18.4835164835165
911 18.4950603732162
912 18.5087719298246
913 18.5279299014239
914 18.5164113785558
915 18.5234972677596
916 18.5240174672489
917 18.5081788440567
918 18.520697167756
919 18.5277475516866
920 18.5391304347826
921 18.5526601520087
922 18.556399132321
923 18.5698808234019
924 18.5649350649351
925 18.5827027027027
926 18.5885529157667
927 18.5706580366775
928 18.5646551724138
929 18.5715823466093
930 18.5817204301075
931 18.5639097744361
932 18.5643776824034
933 18.561629153269
934 18.5642398286938
935 18.579679144385
936 18.5683760683761
937 18.5602988260406
938 18.5607675906183
939 18.5580404685836
940 18.5478723404255
941 18.5366631243358
942 18.552016985138
943 18.5641569459173
944 18.5646186440678
945 18.5746031746032
946 18.5877378435518
947 18.5797254487856
948 18.5886075949367
949 18.5890410958904
950 18.5747368421053
951 18.589905362776
952 18.6081932773109
953 18.601259181532
954 18.6006289308176
955 18.6031413612565
956 18.5836820083682
957 18.564263322884
958 18.5782881002088
959 18.5933263816476
960 18.5916666666667
961 18.5889698231009
962 18.5945945945946
963 18.5939771547248
964 18.606846473029
965 18.6103626943005
966 18.6055900621118
967 18.5977249224405
968 18.5981404958678
969 18.5892672858617
970 18.6030927835052
971 18.6076210092688
972 18.5895061728395
973 18.5991778006167
974 18.5862422997947
975 18.5805128205128
976 18.5881147540984
977 18.5926305015353
978 18.5869120654397
979 18.5720122574055
980 18.5857142857143
981 18.5779816513761
982 18.5600814663951
983 18.5462868769074
984 18.5609756097561
985 18.5532994923858
986 18.5405679513185
987 18.5339412360689
988 18.5253036437247
989 18.5136501516684
990 18.5080808080808
991 18.5095862764884
992 18.5181451612903
993 18.5095669687815
994 18.5160965794769
995 18.5256281407035
996 18.5160642570281
997 18.4994984954865
998 18.5130260521042
999 18.5005005005005
1000 18.489
};
\addlegendentry{$v_{p}$ (valor promedio de las tiradas)}
\addplot [semithick, color1, dashed]
table {%
1 18
2 18
3 18
4 18
5 18
6 18
7 18
8 18
9 18
10 18
11 18
12 18
13 18
14 18
15 18
16 18
17 18
18 18
19 18
20 18
21 18
22 18
23 18
24 18
25 18
26 18
27 18
28 18
29 18
30 18
31 18
32 18
33 18
34 18
35 18
36 18
37 18
38 18
39 18
40 18
41 18
42 18
43 18
44 18
45 18
46 18
47 18
48 18
49 18
50 18
51 18
52 18
53 18
54 18
55 18
56 18
57 18
58 18
59 18
60 18
61 18
62 18
63 18
64 18
65 18
66 18
67 18
68 18
69 18
70 18
71 18
72 18
73 18
74 18
75 18
76 18
77 18
78 18
79 18
80 18
81 18
82 18
83 18
84 18
85 18
86 18
87 18
88 18
89 18
90 18
91 18
92 18
93 18
94 18
95 18
96 18
97 18
98 18
99 18
100 18
101 18
102 18
103 18
104 18
105 18
106 18
107 18
108 18
109 18
110 18
111 18
112 18
113 18
114 18
115 18
116 18
117 18
118 18
119 18
120 18
121 18
122 18
123 18
124 18
125 18
126 18
127 18
128 18
129 18
130 18
131 18
132 18
133 18
134 18
135 18
136 18
137 18
138 18
139 18
140 18
141 18
142 18
143 18
144 18
145 18
146 18
147 18
148 18
149 18
150 18
151 18
152 18
153 18
154 18
155 18
156 18
157 18
158 18
159 18
160 18
161 18
162 18
163 18
164 18
165 18
166 18
167 18
168 18
169 18
170 18
171 18
172 18
173 18
174 18
175 18
176 18
177 18
178 18
179 18
180 18
181 18
182 18
183 18
184 18
185 18
186 18
187 18
188 18
189 18
190 18
191 18
192 18
193 18
194 18
195 18
196 18
197 18
198 18
199 18
200 18
201 18
202 18
203 18
204 18
205 18
206 18
207 18
208 18
209 18
210 18
211 18
212 18
213 18
214 18
215 18
216 18
217 18
218 18
219 18
220 18
221 18
222 18
223 18
224 18
225 18
226 18
227 18
228 18
229 18
230 18
231 18
232 18
233 18
234 18
235 18
236 18
237 18
238 18
239 18
240 18
241 18
242 18
243 18
244 18
245 18
246 18
247 18
248 18
249 18
250 18
251 18
252 18
253 18
254 18
255 18
256 18
257 18
258 18
259 18
260 18
261 18
262 18
263 18
264 18
265 18
266 18
267 18
268 18
269 18
270 18
271 18
272 18
273 18
274 18
275 18
276 18
277 18
278 18
279 18
280 18
281 18
282 18
283 18
284 18
285 18
286 18
287 18
288 18
289 18
290 18
291 18
292 18
293 18
294 18
295 18
296 18
297 18
298 18
299 18
300 18
301 18
302 18
303 18
304 18
305 18
306 18
307 18
308 18
309 18
310 18
311 18
312 18
313 18
314 18
315 18
316 18
317 18
318 18
319 18
320 18
321 18
322 18
323 18
324 18
325 18
326 18
327 18
328 18
329 18
330 18
331 18
332 18
333 18
334 18
335 18
336 18
337 18
338 18
339 18
340 18
341 18
342 18
343 18
344 18
345 18
346 18
347 18
348 18
349 18
350 18
351 18
352 18
353 18
354 18
355 18
356 18
357 18
358 18
359 18
360 18
361 18
362 18
363 18
364 18
365 18
366 18
367 18
368 18
369 18
370 18
371 18
372 18
373 18
374 18
375 18
376 18
377 18
378 18
379 18
380 18
381 18
382 18
383 18
384 18
385 18
386 18
387 18
388 18
389 18
390 18
391 18
392 18
393 18
394 18
395 18
396 18
397 18
398 18
399 18
400 18
401 18
402 18
403 18
404 18
405 18
406 18
407 18
408 18
409 18
410 18
411 18
412 18
413 18
414 18
415 18
416 18
417 18
418 18
419 18
420 18
421 18
422 18
423 18
424 18
425 18
426 18
427 18
428 18
429 18
430 18
431 18
432 18
433 18
434 18
435 18
436 18
437 18
438 18
439 18
440 18
441 18
442 18
443 18
444 18
445 18
446 18
447 18
448 18
449 18
450 18
451 18
452 18
453 18
454 18
455 18
456 18
457 18
458 18
459 18
460 18
461 18
462 18
463 18
464 18
465 18
466 18
467 18
468 18
469 18
470 18
471 18
472 18
473 18
474 18
475 18
476 18
477 18
478 18
479 18
480 18
481 18
482 18
483 18
484 18
485 18
486 18
487 18
488 18
489 18
490 18
491 18
492 18
493 18
494 18
495 18
496 18
497 18
498 18
499 18
500 18
501 18
502 18
503 18
504 18
505 18
506 18
507 18
508 18
509 18
510 18
511 18
512 18
513 18
514 18
515 18
516 18
517 18
518 18
519 18
520 18
521 18
522 18
523 18
524 18
525 18
526 18
527 18
528 18
529 18
530 18
531 18
532 18
533 18
534 18
535 18
536 18
537 18
538 18
539 18
540 18
541 18
542 18
543 18
544 18
545 18
546 18
547 18
548 18
549 18
550 18
551 18
552 18
553 18
554 18
555 18
556 18
557 18
558 18
559 18
560 18
561 18
562 18
563 18
564 18
565 18
566 18
567 18
568 18
569 18
570 18
571 18
572 18
573 18
574 18
575 18
576 18
577 18
578 18
579 18
580 18
581 18
582 18
583 18
584 18
585 18
586 18
587 18
588 18
589 18
590 18
591 18
592 18
593 18
594 18
595 18
596 18
597 18
598 18
599 18
600 18
601 18
602 18
603 18
604 18
605 18
606 18
607 18
608 18
609 18
610 18
611 18
612 18
613 18
614 18
615 18
616 18
617 18
618 18
619 18
620 18
621 18
622 18
623 18
624 18
625 18
626 18
627 18
628 18
629 18
630 18
631 18
632 18
633 18
634 18
635 18
636 18
637 18
638 18
639 18
640 18
641 18
642 18
643 18
644 18
645 18
646 18
647 18
648 18
649 18
650 18
651 18
652 18
653 18
654 18
655 18
656 18
657 18
658 18
659 18
660 18
661 18
662 18
663 18
664 18
665 18
666 18
667 18
668 18
669 18
670 18
671 18
672 18
673 18
674 18
675 18
676 18
677 18
678 18
679 18
680 18
681 18
682 18
683 18
684 18
685 18
686 18
687 18
688 18
689 18
690 18
691 18
692 18
693 18
694 18
695 18
696 18
697 18
698 18
699 18
700 18
701 18
702 18
703 18
704 18
705 18
706 18
707 18
708 18
709 18
710 18
711 18
712 18
713 18
714 18
715 18
716 18
717 18
718 18
719 18
720 18
721 18
722 18
723 18
724 18
725 18
726 18
727 18
728 18
729 18
730 18
731 18
732 18
733 18
734 18
735 18
736 18
737 18
738 18
739 18
740 18
741 18
742 18
743 18
744 18
745 18
746 18
747 18
748 18
749 18
750 18
751 18
752 18
753 18
754 18
755 18
756 18
757 18
758 18
759 18
760 18
761 18
762 18
763 18
764 18
765 18
766 18
767 18
768 18
769 18
770 18
771 18
772 18
773 18
774 18
775 18
776 18
777 18
778 18
779 18
780 18
781 18
782 18
783 18
784 18
785 18
786 18
787 18
788 18
789 18
790 18
791 18
792 18
793 18
794 18
795 18
796 18
797 18
798 18
799 18
800 18
801 18
802 18
803 18
804 18
805 18
806 18
807 18
808 18
809 18
810 18
811 18
812 18
813 18
814 18
815 18
816 18
817 18
818 18
819 18
820 18
821 18
822 18
823 18
824 18
825 18
826 18
827 18
828 18
829 18
830 18
831 18
832 18
833 18
834 18
835 18
836 18
837 18
838 18
839 18
840 18
841 18
842 18
843 18
844 18
845 18
846 18
847 18
848 18
849 18
850 18
851 18
852 18
853 18
854 18
855 18
856 18
857 18
858 18
859 18
860 18
861 18
862 18
863 18
864 18
865 18
866 18
867 18
868 18
869 18
870 18
871 18
872 18
873 18
874 18
875 18
876 18
877 18
878 18
879 18
880 18
881 18
882 18
883 18
884 18
885 18
886 18
887 18
888 18
889 18
890 18
891 18
892 18
893 18
894 18
895 18
896 18
897 18
898 18
899 18
900 18
901 18
902 18
903 18
904 18
905 18
906 18
907 18
908 18
909 18
910 18
911 18
912 18
913 18
914 18
915 18
916 18
917 18
918 18
919 18
920 18
921 18
922 18
923 18
924 18
925 18
926 18
927 18
928 18
929 18
930 18
931 18
932 18
933 18
934 18
935 18
936 18
937 18
938 18
939 18
940 18
941 18
942 18
943 18
944 18
945 18
946 18
947 18
948 18
949 18
950 18
951 18
952 18
953 18
954 18
955 18
956 18
957 18
958 18
959 18
960 18
961 18
962 18
963 18
964 18
965 18
966 18
967 18
968 18
969 18
970 18
971 18
972 18
973 18
974 18
975 18
976 18
977 18
978 18
979 18
980 18
981 18
982 18
983 18
984 18
985 18
986 18
987 18
988 18
989 18
990 18
991 18
992 18
993 18
994 18
995 18
996 18
997 18
998 18
999 18
1000 18
};
\addlegendentry{$v_{p_{e}}$ (valor promedio esperado)}
\end{axis}

\end{tikzpicture}

    \caption{valor promedio con respecto al número de tiradas}
  \end{mytikzresize}
\end{figure}

%\bibliographystyle{unsrt}
%\bibliography{references}

\end{document}
